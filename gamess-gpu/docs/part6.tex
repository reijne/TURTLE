\documentclass[11pt,fleqn]{article}

\usepackage{hyperref}

% package HTML requires Latex2HTML to be installed for html.sty
\usepackage{html}
\newcommand{\doi}[1]{doi:\href{http://dx.doi.org/#1}{#1}}
\begin{htmlonly}
\renewcommand{\href}[2]{\htmladdnormallink{#2}{#1}}
\end{htmlonly}
\hypersetup{colorlinks,
            %citecolor=black,
            %filecolor=black,
            %linkcolor=black,
            %urlcolor=black,
            bookmarksopen=true,
            pdftex}
 
\addtolength{\textwidth}{1.0in}
\addtolength{\oddsidemargin}{-0.5in}
\addtolength{\topmargin}{-0.5in}
\addtolength{\textheight}{1.0in}
\newcommand{\water}{\mbox{H$_{2}$O}}
\newcommand{\dinfh}{\mbox{D$_{\infty h}$}}
\newcommand{\dtwoh}{\mbox{D$_{2h}$}}
\newcommand{\cinfv}{\mbox{C$_{\infty v}$}}
\newcommand{\ctwov}{\mbox{C$_{2v}$}}
\newcommand{\nitrog}{\mbox{N$_{2}$}}
\newcommand{\phosphine}{\mbox{PH$_{3}$}}
\newcommand{\cah}{\mbox{CaH$_{2}$}}
\newcommand{\formaldehyde}{\mbox{H$_{2}$CO}}
\newcommand{\formion}{\mbox{H$_{2}$CO$^{+}$}}
\newcommand{\bstate}{\mbox{$^{2}$B$_{2}$}}
\newcommand{\apstate}{\mbox{$^{2}$A$^{'}$}}
\newcommand{\appstate}{\mbox{$^{2}$A$^{''}$}}
\newcommand{\bstateo}{\mbox{$^{2}$B$_{1}$}}
\newcommand{\cucl}{\mbox{CuCl}}
\newcommand{\silane}{\mbox{SiH$_{4}$}}
\newcommand{\ethyne}{\mbox{C$_{2}$H$_{2}$}}
\newcommand{\ethene}{\mbox{C$_{2}$H$_{4}$}}
\newcommand{\astate}{\mbox{$^{1}$A$_{1}$}}

\pagestyle{headings}
\pagenumbering{roman}
\begin{document}
\sf
\parindent 0cm
\parskip 1ex
\begin{flushleft}
 
Computing for Science (CFS) Ltd.,\\CCLRC Daresbury Laboratory.\\[0.30in]
{\large Generalised Atomic and Molecular Electronic Structure System }\\[.2in]
\rule{150mm}{3mm}\\
\vspace{.2in}
{\huge G~A~M~E~S~S~-~U~K}\\[.3in]
{\huge USER'S GUIDE~~and}\\[.2in]
{\huge REFERENCE MANUAL}\\[0.3in]
{\huge Version 8.0~~~June 2008}\\ [.2in]
{\large PART 6. TABLE-CI CALCULATIONS}\\
\vspace{.1in}
{\large M.F. Guest, J. Kendrick, J.H. van Lenthe and P. Sherwood}\\[0.2in]
 
Copyright (c) 1993-2008 Computing for Science Ltd.\\[.1in]
This document may be freely reproduced provided that it is reproduced\\
unaltered and in its entirety.\\
\vspace{.2in}
\rule{150mm}{3mm}\\
\end{flushleft}

\tableofcontents
\newpage

\pagenumbering{arabic}

\section[Introduction]{Introduction}

In this chapter we describe the data requirements of Table-CI, a
conventional configuration-driven CI module featuring configuration
selection and energy extrapolation.  The methods used in the package
are described in \cite{ref:18}.  Note that Version 6.2 of GAMESS-UK
also contains a new, more efficient semi-direct version of the Table-CI
module that is capable of performing significantly larger calculations.
At this stage both "old" (Conventional) and "new" (Semi-direct) modules
are available, and are described in detail below.  To maintain
compatibility with previous documentation, we first describe below the
data input and file requirements of the older version. This is then
followed by a description of the new code.

\subsection[Sub-Module Structure of Conventional Table-CI]{Sub-Module Structure of Conventional Table-CI}

An outline of the sub-module structure and philosophy
behind Table-CI have already been given in Part 2, material
that should be taken in conjunction with the present chapter.
As pointed out previously,
the module comprises a set of 9 sub-modules,
which must be user--driven (either implicitly or explicitly, see below)
through data input. These sub modules are as follows:
\begin{itemize}
\item ADAPT: generation of a symmetry adapted list of integrals,
derived by a pseudo-transformation from the list of `raw' integrals;
\item TRAN: integral transformation, using the list of adapted integrals
generated above together with a molecular orbital coefficient array
nominated by the user.
Note that in contrast to the Direct-CI module, 
transformation is an integral part of the Table-CI module;
\item TABLE: generation of the data base of pattern matrix elements
required by both the SELECT and CI sub-modules (see below -- this
data base will have usually been made available on a given machine, but 
may be generated by the user using this sub-module);
\item SELECT:  configuration generation and subsequent selection
based on a user-specified set of reference configurations and
appropriate thresholds;
\item CI: generation of the CI-Hamiltonian based on the set of selected
configurations from SELECT and integrals from TRAN;
\item DIAG: calculation of one or more CI eigenfunctions of 
the Hamiltonian generated under CI.
\end{itemize}
     
The remaining modules are optional, and may be used to
analyse one or more of the CI eigenvectors:
\begin{itemize}
\item NATORB: to generate the spin-free natural orbitals for
one or more of the calculated CI eigenvectors.
\item PROP: to compute various 1-electron properties of the
CI wavefunctions. Note that the natural orbitals generated
above may be routed to the Dumpfile and examined by the
other analysis modules of GAMESS--UK in a subsequent job.
\item TM: to compute the transition moments between nominated
CI eigenvectors.
\end{itemize}

In addition to the Mainfile, Dumpfile and Scratchfile, 
the following data sets will be used by the program.
\begin{itemize}
\item The Tablefile: A dataset normally assigned using the 
local file name (LFN) TABLE will be used as a source of pattern
symbolic matrix elements in the SELECT and CI phases of the Table-CI
procedure.
The space requirements of the Tablefile are about 2 MBytes.
\item The Sortfile: A dataset normally assigned using the LFN
SORT will be used as a scratchfile in the generation of symmetry
adapted and transformed integrals. The maximum
space requirements of the Sortfile are about twice that of the
Mainfile, although this will be much reduced in
high symmetry.
\item In contrast to the other post-Hartree Fock modules of GAMESS--UK, the
Table-CI routines make extensive use of unformatted sequential
FORTRAN data sets (or {\em interfaces}). The data set reference
numbers and associated LFNs of these files have been given in
Table 6 of Part 2.
\end{itemize}

\section[Directives Controlling Conventional Table-CI Calculations]{Directives Controlling Conventional Table-CI Calculations}

Data input characterising conventional Table-CI calculation commences
with the MRDCI data line, and is typically followed by a sequence of
directives, terminated by presenting a valid {\em Class 2} directive,
such as VECTORS or ENTER. An overview of the data structure has been
given in Part 2: we provide additional detail on the directives
associated with each sub-module below.

\subsection[MRDCI]{MRDCI}

The Table-CI data initiator consists of a single line containing the
character string MRDCI in the first data field. It acts to transfer
control to those routines responsible for inputing all data relevant to
the MRDCI calculation. Termination of this data is achieved by
presenting a valid {\em Class 2} directive that is not recognised by
the Table-CI input routines, for example, VECTORS or ENTER.

\section[Data for Conventional Table-CI Symmetry Adaptation]{Data for Conventional Table-CI Symmetry Adaptation}

The Symmetry Adaptation module generates the list of symmetry
adapted 1- and 2-electron integrals using as input the full list
of integrals in the basis function representation. It is assumed
that this latter list is not skeletonised, but has been
generated with the SUPER OFF NOSYM specification in effect.
The list of generated two-electron integrals is routed 
to the FORTRAN {\em interface} FTN021, the one-electron integrals to
the Dumpfile.
\subsection[ADAPT]{ADAPT}
The ADAPT directive is used to control the symmetry adapted
integral generator, and comprises a single data line
read to the variables TEXT, TEXTF and TEXTB using format (3A).
\begin{itemize}
\item TEXT should be set to the character string ADAPT.
\item TEXTF is an optional parameter that may be used to
control the quantity of printed output produced by the 
module. Valid settings include the strings,
\begin{itemize}
\item NOPRINT, to suppress output from the module;
\item IPRINT, to produce an intermediate level of output,
charactering for example the symmetry adapted functions;
\item FPRINT, to produce output suitable for debugging purposes.
\end{itemize}
\item TEXTB is an optional parameter that should be set
to the string BYPASS if the user wishes to bypass generation
of the symmetry adapted integrals. Such usage is typically
associated with restarting Table-CI calculations.
\end{itemize}
{\bf Example}
{
\footnotesize
\begin{verbatim}
          ADAPT BYPASS
\end{verbatim}
}
is a valid data line to bypass the module in a Table-CI restart
job.
\section[Data for Conventional Table-CI Transformation]{Data for Conventional Table-CI Transformation}

\subsection[TRAN]{TRAN}
The TRAN directive is used to control the integral transformation
module, and comprises one or more data lines. The first data line
is read to the variables TEXT, ISECV, TEXTF, TEXTC, TEXTD 
and TEXTB using format (A,I,4A).
\begin{itemize}
\item TEXT should be set to the character string TRAN.
\item ISECV is an optional integer parameter used to specify the
section number on the Dumpfile wherein lies the set of eigenvectors to
be used as the molecular orbital coefficient array in the integral
transformation. If ISECV is omitted, the MOs will be either be taken
from the section nominated on the ENTER directive, or from the default
eigenvector section deemed to be in effect through the associated
SCFTYPE. Note that examples of restoring orbitals from both MCSCF and
CASSCF calculations are given below.
\item TEXTF is an optional parameter that may be used to
control the quantity of printed output produced by the 
module. Valid settings include the strings,
\begin{itemize}
\item NOPRINT, to suppress output from the module;
\item IPRINT, to produce an intermediate level of output;
\item FPRINT, to produce output suitable for debugging purposes.
\end{itemize}
\item TEXTB is an optional parameter that should be set
to the string BYPASS if the user wishes to bypass generation
of the transformed integrals. Such usage is typically
associated with restarting Table-CI calculations.
\item TEXTC is an optional parameter that should be set to one of
the strings  CORE or FREEZE if orbitals are to be frozen in
the transformation.
\item TEXTD is an optional parameter that should be set to one of
the strings  DISCARD or DELETE if orbitals are to be discarded in
the transformation.
\end{itemize}
Additional data lines for the TRAN directive are triggered
by the presence of the CORE and/or DISCARD keywords on the
first line above.
\begin{enumerate}
\item If the CORE keyword has been presented, two additional
data lines are now required to define the number ({\em Line 1})
and the sequence numbers ({\em Line 2})  of the orbitals to be
frozen.

{\em Line 1}  is read in I-format to the variables
(NOCORE(I),I=1,NIRREP), where NOCORE(I) specifies the number
of orbitals of irreducible representation (IRrep) I that are
to be frozen. NIRREP is the number of irreducible representations
characterising the associated Abelian point-group in use containing more than 
zero orbitals.
Note again that each IRrep has an associated sequence number
(see Table~\ref{table:1})
and that the input orbital set will be reordered such that
\begin{itemize}
\item IRreps having zero orbitals are discarded, and
\item orbitals of common IRrep are grouped together, these groups
being arranged in order of increasing IRrep number, and
\item orbitals of common IRrep are ordered according to their
relative disposition in the input orbital set e.g., by eigenvalue
ordering if SCF MOs.
\end{itemize}

{\em Line 2} line is also read in I-format and specifies the
sequence numbers of the frozen orbitals, in the spirit of the
re-ordered sequence above. Thus the first NOCORE(1) integers
specify the frozen orbitals of IRrep$_{1}$, the next NOCORE(2)
integers the frozen orbitals of IRrep$_{2}$, and so on until
all IRreps have been specified. Note that the integer specification
within each IRrep refers to the relative ordering within that
IRrep, and not within the total orbital manifold (see below).
\item If the DISCARD keyword has been presented, two additional
data lines are now required to define the number ({\em Line 1})
and the sequence numbers ({\em Line 2}) of the orbitals to be
discarded.

{\em Line 1} is read in I-format to the variables
(NODISC(I),I=1,NIRREP), where NODISC(I) specifies the number
of orbitals of irreducible representation (IRrep) I that are
to be discarded. 

{\em Line 2} is also read in I-format and specifies the
sequence numbers of the discarded orbitals, in the spirit of the
re-ordered sequence above. Thus the first NODISC(1) integers
specify the discarded orbitals of IRrep$_{1}$, the next NODISC(2)
integers the discarded orbitals of IRrep$_{2}$, and so on until
all IRreps (with more than zero orbitals) have been specified. 
Note that the integer specification
within each IRrep again refers to the relative ordering within that
IRrep, and not within the total orbital manifold (see below).
\end{enumerate}

{\bf Example 1}\\

In this example we wish to perform a valence-CI calculation 
on the \formaldehyde\
molecule using a DZ basis of 24 gtos, looking to freeze both the
oxygen and carbon 1s orbitals, and to discard the inner--shell
complement orbitals. An examination of the SCF output
reveals the following orbital analysis.

{
\footnotesize
\begin{verbatim}
          =============================
          IRREP  NO. OF SYMMETRY ADAPTED
                 BASIS FUNCTIONS
          =============================
            1          14
            2           4
            3           6
          =============================
\end{verbatim}
}
with the following orbital assignments from the closed
shell SCF:
{
\footnotesize
\begin{verbatim}
          ===============================================
          M.O.  IRREP  ORBITAL ENERGY   ORBITAL OCCUPANCY
          ===============================================

            1      1    -20.58952765           2.0000000
            2      1    -11.35779935           2.0000000
            3      1     -1.43525479           2.0000000
            4      1     -0.87463564           2.0000000
            5      3     -0.70990765           2.0000000
            6      1     -0.64751394           2.0000000
            7      2     -0.53989416           2.0000000
            8      3     -0.44423257           2.0000000
            9      2      0.10853108           0.0000000
           10      1      0.25726604           0.0000000
           11      1      0.28106873           0.0000000
           12      3      0.38903939           0.0000000
           13      3      0.40966861           0.0000000
           14      2      0.46216570           0.0000000
           15      1      0.65466944           0.0000000
           16      1      0.82879998           0.0000000
           17      2      0.98111608           0.0000000
           18      1      0.98701051           0.0000000
           19      3      1.07064863           0.0000000
           20      1      1.16621340           0.0000000
           21      3      1.29856111           0.0000000
           22      1      1.82320845           0.0000000
           23      1     23.76352004           0.0000000
           24      1     43.36689896           0.0000000
          ===============================================
\end{verbatim}
}
Thus the orbitals of interest are of common IRrep (a$_{1}$),
with sequence numbers 1,2 (core) and 13,14 (complement MOs)
within the re-ordered a$_{1}$ set. The following TRAN data would
freeze and discard these MOs:

{
\footnotesize
\begin{verbatim}
          TRAN 1 CORE DISCARD
          2 0 0 0
          1 2
          2 0 0 0
          13 14
\end{verbatim}
}
or, assuming the default eigenvector specification is in effect,
simply

{
\footnotesize
\begin{verbatim}
          TRAN CORE DISCARD
          2 0 0 0
          1 2
          2 0 0 0
          13 14
\end{verbatim}
}
The following sequence would be used to simply freeze the orbitals
while retaining the complete virtual manifold:

{
\footnotesize
\begin{verbatim}
          TRAN CORE 
          2 0 0 0
          1 2
\end{verbatim}
}
{\bf Example 2}\\

In this example we wish to perform a valence-CI calculation on the
\nitrog\ molecule using a TZVP basis. While the molecular symmetry is
\dinfh, the symmetry adaptation  and subsequent CI will be conducted in
the \dtwoh\ point group.  The resolution of the \dinfh\ into the \dtwoh\
orbital species is given below in Table~\ref{table:2}.  An examination
of the SCF output reveals the following orbital analysis.

{
\footnotesize
\begin{verbatim}
          =============================
          IRREP  NO. OF SYMMETRY ADAPTED
                 BASIS FUNCTIONS
          =============================
            1          11
            2           4
            3           4
            4           1
            5          11
            6           4
            7           4
            8           1
          =============================
\end{verbatim}
}
and the following orbital assignments from the converged closed shell SCF:
{
\footnotesize
\begin{verbatim}
          ===============================================
          M.O.  IRREP  ORBITAL ENERGY   ORBITAL OCCUPANCY
          ===============================================
            1      1    -15.66716423           2.0000000
            2      5    -15.66241865           2.0000000
            3      1     -1.51005217           2.0000000
            4      5     -0.76128176           2.0000000
            5      1     -0.63704931           2.0000000
            6      3     -0.63448705           2.0000000
            7      2     -0.63448705           2.0000000
            8      6      0.17408343           0.0000000
            9      7      0.17408343           0.0000000
           10      5      0.30302673           0.0000000
           11      3      0.38747796           0.0000000
           12      2      0.38747796           0.0000000
           13      1      0.42599317           0.0000000
           14      1      0.49515007           0.0000000
           15      7      0.57046706           0.0000000
           16      6      0.57046706           0.0000000
           17      5      0.92638361           0.0000000
           18      5      1.12927523           0.0000000
           19      1      1.83974927           0.0000000
           20      3      2.01160646           0.0000000
           21      2      2.01160646           0.0000000
           22      5      2.01683641           0.0000000
           23      4      2.08974396           0.0000000
           24      1      2.08974396           0.0000000
           25      7      2.16718057           0.0000000
           26      6      2.16718057           0.0000000
           27      1      2.16945852           0.0000000
           28      3      2.20023738           0.0000000
           29      2      2.20023738           0.0000000
           30      8      2.67650248           0.0000000
           31      5      2.67650248           0.0000000
           32      5      2.98166352           0.0000000
           33      1      3.57616884           0.0000000
           34      7      3.58056531           0.0000000
           35      6      3.58056531           0.0000000
           36      5      4.37707106           0.0000000
           37      1      6.02043977           0.0000000
           38      5      6.12816913           0.0000000
           39      1     35.89673292           0.0000000
           40      5     35.91926868           0.0000000
          ===============================================
\end{verbatim}
}
Thus the inner-shell N1s orbitals, the 1$\sigma_{g}$ and
1$\sigma_{u}$ transform as  a$_{g}$ and b$_{1u}$ respectively
in \dtwoh\ symmetry, with IRrep numbers 1 and 5. Both correspond
of course to the first orbital in the appropriate IRrep.
The following sequence would be used to simply freeze the orbitals
while retaining the complete virtual manifold:

{
\footnotesize
\begin{verbatim}
          TRAN CORE 
          1 0 0 0 1 0 0 0
          1 1
\end{verbatim}
}
Turning to the inner-shell complement orbitals, we again find
the corresponding IRreps, 1 and 5, with the orbitals the highest
lying member in each case, with relative sequence number of 11.
Thus the following TRAN data would
act to both freeze and discard the inner-shells and their
complement MOs:

{
\footnotesize
\begin{verbatim}
          TRAN CORE DISCARD
          1 0 0 0 1 0 0 0
          1 1
          1 0 0 0 1 0 0 0
          11 11
\end{verbatim}
}
Further examples of TRAN data will be discussed below in the
section discussing Reference Function specification within
the SELECT data.

\section[Data for Conventional Table-CI Data-base Generation]{Data for Conventional Table-CI Data-base Generation}

The `data-base' of pattern symbolic matrix elements required
by both the Selection and CI modules may be generated by the
user in the course of any Table-CI calculation. It is not
envisaged that this step will be necessary, since the
data base will in general be already installed  on those
machines  on which GAMESS--UK is  available, the data set being allocated
to the program with LFN TABLE.

\subsection[TABLE]{TABLE}
The TABLE directive is used to request and control the data-base
generator, and comprises a single data line
read to the variables TEXT, TEXTF  using format (2A).
\begin{itemize}
\item TEXT should be set to the character string TABLE.
\item TEXTF is an optional parameter that may be used to
control the quantity of printed output produced by the 
module. Valid settings include the strings,
\begin{itemize}
\item NOPRINT, to suppress output from the module;
\item IPRINT, to produce an intermediate level of output;
\item FPRINT, to produce output suitable for debugging purposes.
\end{itemize}
\end{itemize}

\section[Data for Conventional Table-CI Selection]{Data for Conventional Table-CI Selection}

Data for the configuration selection module is initiated with
the SELECT directive, followed by those directives characterising
the symmetry of the state(s) of interest and reference configurations
(CNTRL, SPIN, SYMMETRY, CONF etc.) and terminated by data (ROOTS,
THRESH) controlling the process of selection.

\subsection[SELECT]{SELECT}
The SELECT directive is used to control the configuration
selection module, and comprises a single data line
read to the variables TEXT, TEXTF and TEXTB using format (3A).
\begin{itemize}
\item TEXT should be set to the character string SELECT.
\item TEXTF is an optional parameter that may be used to
control the quantity of printed output produced by the 
module. Valid settings include the strings,
\begin{itemize}
\item NOPRINT, to suppress the major part of the
output from the module, in particular all details
of the perturbative energy lowerings associated with 
the initial set of configurations;
\item IPRINT, to produce an intermediate level of output;
\item FPRINT, to produce output suitable for debugging purposes.
This includes the energy lowerings associated with the complete
configuration list.
\end{itemize}
\item TEXTB is an optional parameter that should be set
to the string BYPASS if the user wishes to bypass 
SELECT processing. Such usage is typically
associated with restarting Table-CI calculations.
\end{itemize}

\subsection[CNTRL]{CNTRL}
This directive consists of one line read to variables TEXT, NELEC
using format (A,I).
\begin{itemize}
\item  TEXT should be set to the character string CNTRL.
\item  NELEC is used to specify the total number of `active'
electrons in the CI calculation. Note that any inner shell
electrons frozen out under control of the TRAN directive
should not be included.
\end{itemize}
The CNTRL directive may be omitted, when the program will set NELEC
to the value characterising the SCF process implicit within
the RUNTYPE CI processing, subtracting out those electrons
nominated through the CORE parameter of the TRAN data.

\subsection[SPIN]{SPIN}
This directive consists of one line read to variables TEXT, NSPIN
using format (A,I).
\begin{itemize}
\item  TEXT should be set to the character string SPIN.
\item  NSPIN is used to specify the spin degeneracy 
of the CI wavefunction of the electronic eigenstate(s) of interest,
using the values 1,2,3 etc. for singlet, doublet, triplet states etc.
respectively. It is also possible to use one of the character strings
SINGLET, DOUBLET, TRIPLET, QUARTET and QUINTET to specify NSPIN.
\end{itemize}
The SPIN directive may be omitted, when the program will set NSPIN
to the value specified on the MULTIPLICITY directive 
(see section 4.6.2)\\

{\bf Example}
{
\footnotesize
\begin{verbatim}
          SPIN 3

          SPIN TRIPLET
\end{verbatim}
}
are equivalent; the wavefunction will be three-fold spin degenerate.

\subsection[SYMMETRY]{SYMMETRY}
This directive consists of one line read to variables TEXT, NSYM
using format (A,I).
\begin{itemize}
\item  TEXT should be set to the character string SYMMETRY.
\item  NSYM is an integer parameter used to specify the 
spatial symmetry of the CI wavefunction,
and is set to the appropriate sequence number of the 
required irreducible representation (see Table~\ref{table:1}).
\end{itemize}
The SYMMETRY directive may be omitted, when the program will set NSYM
to 1 i.e., the totally symmetric representation.
(see section 4.6.2)\\

{\bf Example}\\

In a system of C$_{2v}$ symmetry, the data line
{
\footnotesize
\begin{verbatim}
          SYMMETRY 3
\end{verbatim}
}
would be required when performing calculations on states of
B$_{2}$ symmetry. Failure to present the directive in such
cases will lead to the default A$_{1}$ symmetry.

\subsection[SINGLES]{SINGLES}
This directive consists of one line read to variables TEXT, NREF
using format (A,I).
\begin{itemize}
\item  TEXT should be set to the character string SINGLES.
\item  NREF is an integer parameter used to nominate
a particular configuration within the set of
reference functions. The selection module will then retain in the
final CI all single excitations with respect to the nominated function
{\em regardless} of their computed energy lowerings.
\end{itemize}
The SINGLES directive may be omitted, when the program will 
use the energy lowerings as the sole criteria for
including configurations in the final CI.\\

{\bf Example}\\

Presenting the data line
{
\footnotesize
\begin{verbatim}
          SINGLES 1
\end{verbatim}
}
in a Table-CI calculation of a closed--shell system, where the
SCF configuration is the first in the CONF list, will lead
to the inclusion of all single excitations with respect to the
SCF function in the final CI. Such inclusion leads, of course, to
a marked improvement in the quality of one-electron properties
computed from the CI wavefunction.

 
\begin{table}
 \caption{\label{table:1}\ Resolution of the \cinfv\ Species into the \ctwov\ Species}
 
 \begin{centering}
 \begin{tabular}{llr}
 \\ \hline\hline
\multicolumn{2}{c}{Orbital} &  IRrep  \\ 
         \cline{1-2}
    $\cinfv$ & $\ctwov$ & Sequence No. \\ \cline{1-3}

 $\sigma$         &   a$_{1}$         &    1 \\
 $\delta_{x2-y2}$ &                   &      \\
 $\pi_{x}$        &   b$_{1}$         &    2 \\
 $\pi_{y}$        &   b$_{2}$         &    3 \\
 $\delta_{xy}$    &   a$_{2}$         &    4 \\  \hline\hline
\end{tabular}

\end{centering}
\end{table}
 
\begin{table}
 \caption{\label{table:2}\ Resolution of the \dinfh\ Species into the \dtwoh\ Species}
 
 \begin{centering}
 \begin{tabular}{llr}
 \\ \hline\hline
\multicolumn{2}{c}{Orbital} &  IRrep  \\ 
         \cline{1-2}
    $\dinfh$ & $\dtwoh$ & Sequence No. \\ \cline{1-3}

 $\sigma_{g}$            &   a$_{g}$         &    1 \\
 $\delta_{g,x2-y2}$ &                   &      \\
 $\pi_{u,x}$        &   b$_{3u}$         &    2 \\
 $\pi_{u,y}$        &   b$_{2u}$         &    3 \\
 $\delta_{g,xy}$    &   b$_{1g}$         &    4 \\
 $\sigma_{u}$      &   b$_{1u}$         &    5 \\
 $\delta_{u,x2-y2}$ &                   &      \\
 $\pi_{g,x}$        &   b$_{2g}$         &    6 \\
 $\pi_{g,y}$        &   b$_{3g}$         &    7 \\
 $\delta_{u,xy}$    &   a$_{u}$         &     8 \\  \hline\hline
\end{tabular}

\end{centering}
\end{table}

\subsection[CONF]{CONF}
The CONF directive is used to specify the reference CSFs for
the CI expansion.  The first
line of the CONF directive is set to the character string CONF. 
Each subsequent line defines a reference CSF by specifying 
the sequence numbers of the component active orbitals in I-format. 
A given reference CSF is defined by 
\begin{enumerate}
\item the number of open--shell orbitals (NOPEN). NOPEN
includes any unpaired orbitals together with those
non-identical spin-coupled pairs open to substitution.
\item NOPEN integers specifying the sequence numbers of these orbitals
\item the (NELEC-NOPEN)/2 sequence numbers of 
the doubly-occupied orbitals i.e., the identically spin-coupled orbitals
\end{enumerate}
where the sequence--numbers refers to the symmetry ordered orbitals 
performed at the outset of processing. Within the set of open-- and 
doubly--occupied orbitals, the MOs are presented in groups of
common IRrep, with the groups presented in order of increasing
IRrep sequence number. Note that all reference function
nominated by CONF {\em must} be of the same symmetry
as that nominated on the SYMMETRY directive. A few examples 
will help clarify this order of presentation.\\

{\bf Example 1}\\

Consider performing a valence-CI calculation on the \phosphine\ molecule
using a 6-31G(*)  basis. While the molecular symmetry is C$_{3v}$, the
symmetry adaptation  and subsequent CI will be conducted in the C$_{s}$
point group. An examination of the SCF output reveals the following
orbital analysis.

{
\footnotesize
\begin{verbatim}
           =============================
           IRREP  NO. OF SYMMETRY ADAPTED
                  BASIS FUNCTIONS
           =============================
             1          18
             2           7
           =============================
\end{verbatim}
}
and the following orbital assignments characterising the closed--shell
SCF configuration:
\begin{equation}
  1a_{1}^{2}  2a_{1}^{2}  1e^{4}  3a_{1}^{2}  4a_{1}^{2}  2e^{4}  5a_{1}^{2}
\end{equation}
or, in the C$_{s}$ symmetry representation:
\begin{equation}
  1a'^{2}  2a'^{2}  1a''^{2}  3a'^{2}  4a'^{2}  5a'^{2}  6a'^{2} 2a''^{2} 7a'^{2}
\end{equation}
{
\footnotesize
\begin{verbatim}
          ===============================================
           M.O. IRREP  ORBITAL ENERGY   ORBITAL OCCUPANCY
          ===============================================
             1     1    -79.93661395           2.0000000
             2     1     -7.48916431           2.0000000
             3     1     -5.38319410           2.0000000
             4     2     -5.38319405           2.0000000
             5     1     -5.38149104           2.0000000
             6     1     -0.85610769           2.0000000
             7     1     -0.52191424           2.0000000
             8     2     -0.52191424           2.0000000
             9     1     -0.38579686           2.0000000
            10     1      0.16819544           0.0000000
            11     2      0.16819544           0.0000000
            12     1      0.26587776           0.0000000
            13     1      0.46072690           0.0000000
            14     2      0.46072690           0.0000000
            15     1      0.47871033           0.0000000
            16     1      0.56106989           0.0000000
            17     1      0.89229884           0.0000000
            18     2      0.89229885           0.0000000
            19     2      0.91131383           0.0000000
            20     1      0.91131383           0.0000000
            21     1      0.93118300           0.0000000
            22     1      1.17900613           0.0000000
            23     2      1.45058658           0.0000000
            24     1      1.45058658           0.0000000
            25     1      3.78674557           0.0000000
          ===============================================
\end{verbatim}
}
Based on the above output, the CONF data lines may be
deduced from the following table, where we
assume that we wish to freeze the five inner shell orbitals:
\begin{equation}
  1a'^{2}  2a'^{2}  1a''^{2}  3a'^{2}  4a'^{2}  
\end{equation}

\begin{centering}
\begin{tabular}{llrrrr}

\\ \hline
IRrep & IRrep &  No. of Basis & Frozen & Active & Sequence  \\
      & No.  &   Functions    & MOs    & MOs    & Nos.       \\ \hline
 a$^{'}$  & 1  &   18          & 4      & 14     & 1-14       \\
 a$^{''}$ & 2  &   7           & 1      &  6     & 15-20   \\ \hline
\end{tabular}
 
\end{centering}
\vspace{.15in}
To perform an 8-electron valence-CI calculation,
involving the SCF configuration and  two  degenerate (1e)' to (2e)'
doubly-excited  configurations 
\begin{equation}
    5a'^{2}  8a'^{2} 2a''^{2} 7a'^{2}
\end{equation}
and
\begin{equation}
    5a'^{2}  6a'^{2} 3a''^{2} 7a'^{2}
\end{equation}
would require the following CONF data:
{
\footnotesize
\begin{verbatim}
          CONF
          0 1 2 3 15
          0 1 3 4 15
          0 1 2 3 16
\end{verbatim}
}
The complete data file for performing the
SCF and subsequent CI would then be as follows:
{
\footnotesize
\begin{verbatim}
          TITLE
          PH3 * 6-31G*  VALENCE-CI 3M/1R
          SUPER OFF NOSYM
          ZMAT 
          P
          H 1 RPH
          H 1 RPH 2 THETA
          H 1 RPH 2 THETA 3 THETA  1
          VARIABLES
          RPH 2.685   
          THETA 93.83  
          END
          BASIS 6-31G*
          RUNTYPE CI
          MRDCI
          TRAN CORE
          4 1
          1 TO 4 1
          SELECT
          SINGLES 1
          CONF
          0 1 2 3 15
          0 1 3 4 15
          0 1 2 3 16
          NATORB
          ENTER
\end{verbatim}
}

{\bf Example 2}\\

In this example we wish to perform a valence-CI calculation on the \cucl\
molecule using a 3-21G  basis. While the molecular symmetry is \cinfv,
the symmetry adaptation  and subsequent CI will be conducted in the
\ctwov\ point group.  The resolution of the \cinfv\ into the \ctwov\
orbital species is given in Table~\ref{table:2}.  An examination of the
SCF output reveals the following orbital analysis.

{
\footnotesize
\begin{verbatim}
           =============================
           IRREP  NO. OF SYMMETRY ADAPTED
                  BASIS FUNCTIONS
           =============================
             1          22
             2           9
             3           9
             4           2
           =============================
\end{verbatim}
}
and the following orbital assignments from the converged closed shell SCF:
{
\footnotesize
\begin{verbatim}
           ===============================================
            M.O. IRREP  ORBITAL ENERGY   ORBITAL OCCUPANCY
           ===============================================
              1     1   -326.84723972           2.0000000
              2     1   -104.02836336           2.0000000
              3     1    -40.71695637           2.0000000
              4     1    -35.46377378           2.0000000
              5     3    -35.45608069           2.0000000
              6     2    -35.45608068           2.0000000
              7     1    -10.42193940           2.0000000
              8     1     -7.88512031           2.0000000
              9     2     -7.88222844           2.0000000
             10     3     -7.88222844           2.0000000
             11     1     -5.07729175           2.0000000
             12     1     -3.38247056           2.0000000
             13     3     -3.35978308           2.0000000
             14     2     -3.35978307           2.0000000
             15     1     -1.01099628           2.0000000
             16     3     -0.53702948           2.0000000
             17     2     -0.53702947           2.0000000
             18     4     -0.49640067           2.0000000
             19     1     -0.49640067           2.0000000
             20     1     -0.44715317           2.0000000
             21     3     -0.39988537           2.0000000
             22     2     -0.39988537           2.0000000
             23     1     -0.35127248           2.0000000
             24     1      0.00023285           0.0000000
             25     3      0.06300102           0.0000000
             26     2      0.06300102           0.0000000
             27     1      0.12855448           0.0000000
             28     1      0.19287013           0.0000000
             29     3      0.25729975           0.0000000
             30     2      0.25729975           0.0000000
             31     1      0.39720201           0.0000000
             32     1      0.86197727           0.0000000
             33     2      0.88942618           0.0000000
             34     3      0.88942618           0.0000000
             35     1      1.01877167           0.0000000
             36     1      2.16694989           0.0000000
             37     3      3.96181512           0.0000000
             38     2      3.96181512           0.0000000
             39     4      3.98212497           0.0000000
             40     1      3.98212497           0.0000000
             41     1      4.08851360           0.0000000
             42     1     24.51368240           0.0000000
           ===============================================
\end{verbatim}
}
Based on the above output, the CONF data lines may be
deduced from the following table, where we
assume that we wish to freeze the first 14 inner shell orbitals:
\begin{equation}
 1\sigma^{2}  2\sigma^{2}   3\sigma^{2}  4\sigma^{2}  1\pi^{4}  5\sigma^{2} 6\sigma^{2}   2\pi^{4}  7\sigma^{2}  8\sigma^{2}  3\pi^{4}  
\end{equation}

\begin{centering}
\begin{tabular}{llrrrr}

\\ \hline
IRrep & IRrep &  No. of Basis & Frozen & Active & Sequence  \\
      & No.  &   Functions    & MOs    & MOs    & Nos.       \\ \hline
 a$_{1}$  & 1  &   22          & 8      & 14     & 1-14       \\
 b$_{1}$  & 2  &   9           & 3      &  6     & 15-20   \\ 
 b$_{2}$  & 3  &   9           & 3      &  6     & 21-26       \\
 a$_{2}$  & 4  &   2           & 0      &  2     & 27-28   \\ \hline
\end{tabular}
 
\end{centering}
\vspace{.15in}
To perform an 18-electron valence-CI calculation,
based on the SCF configuration 
\begin{equation}
 9\sigma^{2}  4\pi^{4}   1\delta^{4} 10\sigma^{2}  5\pi^{4}  11\sigma^{2} 
\end{equation}
would require the following CONF data:
{
\footnotesize
\begin{verbatim}
          CONF
          0 1 2 3 4  15 16  21 22  27
\end{verbatim}
}
The complete data file for performing the
SCF and subsequent CI would then be as follows:
{
\footnotesize
\begin{verbatim}
          TITLE\CUCL .. 3-21G
          ZMAT ANGSTROM\CU\CL 1 CUCL\
          VARIABLES\CUCL 2.093 \END 
          BASIS 3-21G
          RUNTYPE CI
          MRDCI
          TRAN CORE
          8 3 3 0
          1 TO 8  1 TO 3  1 TO 3
          SELECT 
          SINGLES 1
          CONF
          0 1 2 3 4  15 16  21 22  27
          NATORB
          ENTER
\end{verbatim}
}
The inclusion of a second reference configuration corresponding to
the doubly excited configuration
\begin{equation}
 9\sigma^{2}  4\pi^{4}   1\delta^{4} 10\sigma^{2}  5\pi^{4}  12\sigma^{2} 
\end{equation}
would require the following CONF data;
{
\footnotesize
\begin{verbatim}
          CONF
          0 1 2 3 4  15 16  21 22  27
          0 1 2 3 5  15 16  21 22  27
\end{verbatim}
}

{\bf Example 3}\\

Consider performing a valence-CI calculation on the \silane\ molecule
using a 6-31G(*)  basis. While the molecular symmetry is T$_{d}$, the
symmetry adaptation  and subsequent CI will be conducted in the C$_{2v}$
point group. An examination of the SCF output reveals the following
orbital analysis.

{
\footnotesize
\begin{verbatim}
           =============================
           IRREP  NO. OF SYMMETRY ADAPTED
                  BASIS FUNCTIONS
           =============================
             1           9
             2           6
             3           6
             4           6
           =============================
\end{verbatim}
}
and the following orbital assignments from
the converged closed shell SCF:
{
\footnotesize
\begin{verbatim}
          ===============================================
          M.O.  IRREP  ORBITAL ENERGY   ORBITAL OCCUPANCY
          ===============================================
            1     1    -68.77130710           2.0000000
            2     1     -6.12943325           2.0000000
            3     2     -4.23503117           2.0000000
            4     3     -4.23503117           2.0000000
            5     4     -4.23503117           2.0000000
            6     1     -0.73046864           2.0000000
            7     4     -0.48480821           2.0000000
            8     3     -0.48480821           2.0000000
            9     2     -0.48480821           2.0000000
           10     2      0.16291387           0.0000000
           11     3      0.16291387           0.0000000
           12     4      0.16291387           0.0000000
           13     1      0.25681257           0.0000000
           14     1      0.33606346           0.0000000
           15     3      0.37087856           0.0000000
           16     2      0.37087856           0.0000000
           17     4      0.37087856           0.0000000
           18     1      0.79946861           0.0000000
           19     1      0.79946861           0.0000000
           20     4      0.86232544           0.0000000
           21     3      0.86232544           0.0000000
           22     2      0.86232544           0.0000000
           23     1      1.23833149           0.0000000
           24     4      1.44033091           0.0000000
           25     3      1.44033091           0.0000000
           26     2      1.44033091           0.0000000
           27     1      3.13181655           0.0000000
          ===============================================
\end{verbatim}
}
Based on the above output, the CONF data lines may be
deduced from the following table, where we
assume that we wish to freeze the first 5 silicon inner shell orbitals:

\begin{centering}
\begin{tabular}{llrrrr}

\\ \hline
IRrep & IRrep &  No. of Basis & Frozen & Active & Sequence  \\
      & No.  &   Functions    & MOs    & MOs    & Nos.       \\ \hline
 a$_{1}$  & 1  &   9          & 2      &  7     & 1-7       \\
 b$_{1}$  & 2  &   6          & 1      &  5     & 8-12   \\ 
 b$_{2}$  & 3  &   6          & 1      &  5     & 13-17       \\
 a$_{2}$  & 4  &   6          & 1      &  5     & 18-22   \\ \hline
\end{tabular}
 
\end{centering}
\vspace{.15in}
To perform a 8-electron valence-CI calculation,
based on the SCF configuration would require the following CONF data:
{
\footnotesize
\begin{verbatim}
          CONF
          0 1 8 13 18
\end{verbatim}
}
The complete data file for performing the
SCF and subsequent CI would then be as follows:
{
\footnotesize
\begin{verbatim}
          TITLE
          SIH4 * 6-31G* MRDCI VALENCE-CI 1M/1R
          ZMAT 
          SI
          H 1 SIH
          H 1 SIH 2 109.471
          H 1 SIH 2 109.471 3 120.0
          H 1 SIH 2 109.471 4 120.0
          VARIABLES
          SIH 2.80   
          END
          BASIS 6-31G*
          RUNTYPE CI
          MRDCI
          TRAN CORE
          2 1 1 1
          1 2 1 1 1
          SELECT
          CONF
          0 1 8 13 18
          SINGLES 1
          NATORB
          ENTER
\end{verbatim}
}

{\bf Example 4}\\

In this example we wish to perform a valence-CI calculation on the
\nitrog\ molecule using a 4-31G(*)  basis. While the molecular symmetry is
\dinfh, the symmetry adaptation  and subsequent CI will be conducted in
the \dtwoh\ point group.  The resolution of the \dinfh\ into the \dtwoh\
orbital species is given in Table~\ref{table:2}.  An examination of the
SCF output reveals the following orbital analysis.

{
\footnotesize
\begin{verbatim}
           =============================
           IRREP  NO. OF SYMMETRY ADAPTED
                  BASIS FUNCTIONS
           =============================
             1           8
             2           3
             3           3
             4           1
             5           8
             6           3
             7           3
             8           1
           =============================
\end{verbatim}
}
and the following orbital assignments from the converged closed shell SCF:
{
\footnotesize
\begin{verbatim}
          ===============================================
          M.O.  IRREP  ORBITAL ENERGY   ORBITAL OCCUPANCY
          ===============================================
            1     1    -15.65951533           2.0000000
            2     5    -15.65474750           2.0000000
            3     1     -1.50615941           2.0000000
            4     5     -0.75782277           2.0000000
            5     1     -0.63244925           2.0000000
            6     3     -0.63135826           2.0000000
            7     2     -0.63135826           2.0000000
            8     6      0.20154861           0.0000000
            9     7      0.20154861           0.0000000
           10     5      0.63883097           0.0000000
           11     1      0.82491489           0.0000000
           12     3      0.89634343           0.0000000
           13     2      0.89634343           0.0000000
           14     1      0.91812387           0.0000000
           15     7      1.10036132           0.0000000
           16     6      1.10036132           0.0000000
           17     5      1.17625689           0.0000000
           18     5      1.66995008           0.0000000
           19     4      1.70518236           0.0000000
           20     1      1.70518236           0.0000000
           21     3      1.91001614           0.0000000
           22     2      1.91001614           0.0000000
           23     8      2.29436539           0.0000000
           24     5      2.29436539           0.0000000
           25     1      2.84356916           0.0000000
           26     7      3.00847817           0.0000000
           27     6      3.00847817           0.0000000
           28     5      3.37447679           0.0000000
           29     1      3.71753400           0.0000000
           30     5      4.09917273           0.0000000
          ===============================================
\end{verbatim}
}
Based on the above output, the CONF data lines may be
deduced from the following table, where we
assume that we wish to freeze the two N1s inner shell orbitals:

\begin{centering}
\begin{tabular}{llrrrr}

\\ \hline
IRrep & IRrep &  No. of Basis & Frozen & Active & Sequence  \\
      & No.   &   Functions   & MOs    & MOs    & Nos.       \\ \hline
 $\sigma_{g}$   & 1  &   8        & 1      &  7     & 1-7       \\
 $\pi_{u,x}$    & 2  &   3        & 0      &  3     & 8-10   \\ 
 $\pi_{u,y}$    & 3  &   3        & 0      &  3     & 11-13       \\
 $\delta_{g,xy}$& 4  &   1        & 0      &  1     & 14   \\
 $\sigma_{u}$   & 5  &   8        & 1      &  7     & 15-21   \\
 $\pi_{g,x}$    & 6  &   3        & 0      &  3     & 22-24       \\
 $\pi_{g,y}$    & 7  &   3        & 0      &  3     & 25-27   \\ 
 $\delta_{u,xy}$& 8  &   1        & 0      &  1     & 28   \\ \hline
\end{tabular}
 
\end{centering}
\vspace{.15in}
To perform a 10-electron valence-CI calculation,
based on the SCF configuration  
\begin{equation}
 2\sigma_g^{2}  2\sigma_u^{2}  3\sigma_g^2  1\pi_u^{4}
\end{equation}
and associated $\pi$ to $\pi^{*}$ excitations
\begin{equation}
 2\sigma_g^{2}  2\sigma_u^{2}  3\sigma_g^2  1\pi_{u,y}^2 2\pi_{u,x}^2
\end{equation}
\begin{equation}
 2\sigma_g^{2}  2\sigma_u^{2}  3\sigma_g^2  1\pi_{u,x}^2 2\pi_{u,y}^2
\end{equation}
\begin{equation}
 2\sigma_g^{2}  2\sigma_u^{2}  3\sigma_g^2  (1\pi_{u,x} 2\pi_{u,x}) (1\pi_{u,y} 2\pi_{u,y})
\end{equation}
would require the following CONF data:
{
\footnotesize
\begin{verbatim}
          CONF
          0 1 2 8 11 15
          0 1 2 11 15 22
          0 1 2 8 15 25
          4 8 11 22 25  1 2 15
\end{verbatim}
}
The complete data file for performing the
SCF and subsequent CI would then be as follows:
{
\footnotesize
\begin{verbatim}
          TITLE\N2 .. 4-31G*
          SUPER OFF NOSYM
          ZMAT ANGS\N\N 1 NN
          VARIABLES\NN 1.05 \END
          BASIS 4-31G*
          RUNTYPE CI
          MRDCI
          TRAN CORE
          1 0 0 0 1 0 0 0
          1  1 
          SELECT 
          SINGLES 1
          CONF
          0 1 2 8 11 15
          0 1 2 11 15 22
          0 1 2 8 15 25
          4 8 11 22 25  1 2 15
          NATORB IPRIN
          ENTER
\end{verbatim}
}
Now consider the corresponding calculation performed in
a smaller 3-21G basis.
An examination of the SCF output
reveals the following orbital analysis.

{
\footnotesize
\begin{verbatim}
           =============================
           IRREP  NO. OF SYMMETRY ADAPTED
                  BASIS FUNCTIONS
           =============================
             1           5
             2           2
             3           2
             5           5
             6           2
             7           2
           =============================
\end{verbatim}
}
and the following orbital assignments from
the converged closed shell SCF:
{
\footnotesize
\begin{verbatim}
          ===============================================
          M.O.  IRREP  ORBITAL ENERGY   ORBITAL OCCUPANCY
          ===============================================
            1     1    -15.59983859           2.0000000
            2     5    -15.59796932           2.0000000
            3     1     -1.54485796           2.0000000
            4     5     -0.74550130           2.0000000
            5     2     -0.63373069           2.0000000
            6     3     -0.63373069           2.0000000
            7     1     -0.62012170           2.0000000
            8     6      0.20546760           0.0000000
            9     7      0.20546760           0.0000000
           10     5      0.79186362           0.0000000
           11     1      1.16445455           0.0000000
           12     2      1.26826720           0.0000000
           13     3      1.26826720           0.0000000
           14     7      1.43237859           0.0000000
           15     6      1.43237859           0.0000000
           16     5      1.55279124           0.0000000
           17     1      1.83635478           0.0000000
           18     5      2.63677794           0.0000000
          ===============================================
\end{verbatim}
}
Note that there are now {\em no} MOs of IRREP 4 or 8.
Based on the above output, the CONF data lines may be
deduced from the following table, where we again
assume that we wish to freeze the two N1s inner shell orbitals:

\begin{centering}
\begin{tabular}{llrrrr}

\\ \hline
IRrep          & IRrep &  No. of Basis & Frozen & Active & Sequence  \\
               & No.  &   Functions    & MOs    & MOs    & Nos.       \\ \hline
 $\sigma_{g}$  & 1  &   5        & 1      &  4     & 1-4       \\
 $\pi_{u,x}$   & 2  &   2        & 0      &  2     & 5-6   \\ 
 $\pi_{u,y}$   & 3  &   2        & 0      &  2     & 7-8       \\
 $\sigma_{u}$  & 5  &   5        & 1      &  4     & 9-12   \\
 $\pi_{g,x}$   & 6  &   2        & 0      &  2     & 13-14       \\
 $\pi_{g,y}$   & 7  &   2        & 0      &  2    & 15-16   \\ \hline
\end{tabular}
 
\end{centering}
\vspace{.15in}
To perform an 10-electron valence-CI calculation, based on the SCF
configuration would require the following CONF data:

{
\footnotesize
\begin{verbatim}
          CONF
          0 1 2 5 7  9
\end{verbatim}
}
The complete data file for performing the
SCF and subsequent CI would then be as follows:
{
\footnotesize
\begin{verbatim}
          TITLE\N2 .. 3-21G
          SUPER OFF NOSYM
          ZMAT ANGS\N\N 1 NN
          VARIABLES\NN 1.05 \END
          BASIS 3-21G
          RUNTYPE CI
          MRDCI
          TRAN CORE
          1 0 0 1 0 0 
          1     1 
          SELECT 
          SINGLES 1
          CONF
          0 1 2 5 7  9
          NATORB IPRIN
          ENTER 
\end{verbatim}
}

{\bf Example 5}\\

In this example we wish to perform a valence-CI calculation 
on the \cah\
molecule using a 3-21G  basis. While the molecular symmetry
is D$_{\infty h}$, the symmetry adaptation  and subsequent CI 
will be conducted in the
D$_{2h}$ point group. An examination of the SCF output
reveals the following orbital analysis.

{
\footnotesize
\begin{verbatim}
           =============================
           IRREP  NO. OF SYMMETRY ADAPTED
                  BASIS FUNCTIONS
           =============================
             1           7
             2           4
             3           4
             5           6
           =============================
\end{verbatim}
}
and the following orbital assignments from
the converged closed shell SCF:
{
\footnotesize
\begin{verbatim}
          ===============================================
          M.O.  IRREP  ORBITAL ENERGY   ORBITAL OCCUPANCY
          ===============================================
            1     1   -148.37173884           2.0000000
            2     1    -16.76521275           2.0000000
            3     3    -13.55586861           2.0000000
            4     2    -13.55586861           2.0000000
            5     5    -13.55460610           2.0000000
            6     1     -2.26357685           2.0000000
            7     3     -1.36160958           2.0000000
            8     2     -1.36160958           2.0000000
            9     5     -1.35089927           2.0000000
           10     1     -0.34923025           2.0000000
           11     5     -0.31649941           2.0000000
           12     2      0.02334207           0.0000000
           13     3      0.02334207           0.0000000
           14     1      0.04980631           0.0000000
           15     5      0.09478404           0.0000000
           16     1      0.12395484           0.0000000
           17     3      0.13549605           0.0000000
           18     2      0.13549605           0.0000000
           19     5      0.28345574           0.0000000
           20     1      1.32404002           0.0000000
           21     5      1.45900204           0.0000000
          ===============================================
\end{verbatim}
}
Based on the above output, the CONF data lines may be
deduced from the following table, where we
assume that we wish to freeze the nine Ca inner shell orbitals:

\begin{centering}
\begin{tabular}{llrrrr}

\\ \hline
IRrep          & IRrep&  No. of Basis & Frozen & Active & Sequence  \\
               & No.  &  Functions    & MOs    & MOs    & Nos.       \\ \hline
 $\sigma_{g}$  & 1    &   7           & 3      &  4     & 1-4       \\
 $\pi_{u,x}$   & 2    &   4           & 2      &  2     & 5-6   \\ 
 $\pi_{u,y}$   & 3    &   4           & 2      &  2     & 7-8       \\
 $\sigma_{u}$  & 5    &   6           & 2      &  4     & 9-12   \\ \hline
\end{tabular}
 
\end{centering}
\vspace{.15in}
To perform an 4-electron valence-CI calculation, based on the SCF
configuration would require the following CONF data:

{
\footnotesize
\begin{verbatim}
          CONF
          0 1 9
\end{verbatim}
}
The complete data file for performing the
SCF and subsequent CI would then be as follows:
{
\footnotesize
\begin{verbatim}
          TITLE\CAH2 .. 3-21G
          SUPER OFF NOSYM
          ZMAT ANGS\CA\X 1 1.0\ H 1 CAH 2 90.0\H 1 CAH 2 90.0 3 THETA
          VARIABLES\CAH 2.148 \THETA 180.0 \END
          BASIS 3-21G
          RUNTYPE CI
          MRDCI
          TRAN CORE
          3 2 2  2 
          1 2 3  1 2  1 2  1 2
          SELECT 
          SINGLES 1
          CONF
          0 1 9
          NATORB IPRIN
          ENTER
\end{verbatim}
}

\subsection[ROOTS]{ROOTS}

The ROOTS directive is used to specify those eigenvectors
of the `root' secular problem to be used in the process of
selection, with the energy contributions of the configurations
computed with respect to the nominated vectors. The directive
consists of a single data line with the character string ROOTS
in the first data field. Subsequent data comprises integer
variables used to specify the {\em number} of root eigenstates
(NROOT) and the {\em sequence numbers} of these vectors
within the matrix of zero-order eigenvectors,
(IROOT(I),I=1,NROOT).  Two formats may be used in this
specification:
\begin{enumerate}
\item If the lowest NROOT vectors are to be used, then the data
line is read to the variables TEXT, NROOT using format (A,I);
\begin{itemize}
\item TEXT is set to the character string ROOTS;
\item NROOT is an integer specifying the number of roots to
be used, where the sequence numbers of the roots will be
1--NROOT.
\end{itemize}
\item If the NROOT vectors to be used are not the lowest
in the root eigenvector matrix, then the sequence numbers
within this matrix must be specified. The data line is then
read to the variables TEXT, NROOT, (IROOT(I), I=1,NROOT), using
format (A, (NROOT+1)~I):
\begin{itemize}
\item TEXT and NROOT are defined as above;
\item NROOT integers are read to the array IROOT defining the
vectors of the zero-order matrix to be used in selection.
\end{itemize}
\end{enumerate}
We now provide some further notes on the directive:
\begin{itemize}
\item the ROOTS directive may be omitted, when the energy
contributions are calculated with reference to the lowest
eigenstate of the root problem only. Omission of the directive
is thus equivalent to presenting the data line

{
\footnotesize
\begin{verbatim}
            ROOTS 1
\end{verbatim}
}
\item The number of root eigenstates to be specified will
depend on the number of states required in the final CI. Thus if
NVEC roots of the final CI matrix are to be subsequently
generated in DIAG processing, the user should ideally perform
selection with respect to at least the corresponding NVEC 
roots of the root secular problem to ensure a consistent
treatment of each of the required states. The choice of
the reference set will clearly prove crucial and should be such
as to ensure a one to one correspondence between each of the
final CI vectors and a certain vector of the root problem. Indeed
the whole process of extrapolation to zero threshold is
meaningless if this condition is not obeyed.
\end{itemize}

\subsection[THRESH]{THRESH}
This directive defines the threshold factors
to be used in the process of configuration selection, and
consists of a single line read to variables
TEXT, TMIN, TINC using format (A,2F).
\begin{itemize}
\item TEXT should be set to the character string THRESH.
\item  TMIN should be set to the minimum threshold factor 
(in units of micro-hartree, $\mu$H) to be used in selection. Any CSF
with a computed energy lowering greater than TMIN will be
retained in the final list of selected configurations.
\item TINC should be set to the threshold increment to be
used in the process of extrapolation. This process involves
solution of the final secular problem at a range of
increasing thresholds defined by TMIN, TMIN + TINC, 
TMIN + 2 $\times$ TINC, ...., TMIN + (NEXTRP-1) $\times$ TINC , 
TMIN + NEXTRP $\times$ TINC
where NEXTRP is the number of extrapolation passes requested
under control of the EXTRAP directive (see the DIAG directives).
\end{itemize}
The THRESH directive may be omitted, when TMIN will be set to  30.0
and TINC to 10.0. With the default EXTRAP setting, this would lead
to the solution of the T=50, 40 and 30 $\mu$H secular problem.\\

{\bf Example}
{
\footnotesize
\begin{verbatim}
          THRESH 5.0 5.0

          THRESH 5 5
\end{verbatim}
}
are equivalent, causing T$_{min}$ and T$_{inc}$ to be set to 
5 microhartree.


\section[Data for Conventional Table-CI H-Matrix Construction]{Data for Conventional Table-CI H-Matrix Construction}

\subsection[CI]{CI}

The CI directive  comprises a single data line
read to the variables TEXT, TEXTF and TEXTB using format (3A).
\begin{itemize}
\item TEXT should be set to the character string CI.
\item TEXTF is an optional parameter that may be used to
control the quantity of printed output produced by the 
module. Valid settings include the strings,
\begin{itemize}
\item NOPRINT, to suppress the major part of the
output from the module;
\item IPRINT, to produce an intermediate level of output;
\item FPRINT, to produce output suitable for debugging purposes.
\end{itemize}
\item TEXTB is an optional parameter that should be set
to the string BYPASS if the user wishes to bypass 
CI processing. Such usage is typically
associated with restarting Table-CI calculations.
\end{itemize}

\section[Data for Conventional Table-CI Diagonalisation]{Data for Conventional Table-CI Diagonalisation}
Data input controlling the diagonalisation of the 
final CI Hamiltonian is introduced by
the DIAG directive. The process of extrapolation to 
zero selection threshold involves the diagonalisation module
solving not just one, but several secular problems
corresponding to a range of selection thresholds. The number of
so called `extrapolation passes' is specified
by the EXTRAP directive. In default,
the module will generate NROOT 
eigenvectors of the CI matrix on each pass, where NROOT is the number
of roots specified by the ROOTS directive at selection time.
Thus the solutions of the zero-order Hamiltonian will
be used through a maximum overlap criterion in deriving
the final CI eigenvectors. Additional data may be 
specified to override this default and provide various
convergence and printing controls.

\subsection[DIAG]{DIAG}

The DIAG directive  comprises a single data line
read to the variables TEXT, TEXTF and TEXTB using format (3A)
\begin{itemize}
\item TEXT should be set to the character string DIAG
\item TEXTF is an optional parameter that may be used to
control the quantity of printed output produced by the 
module. Valid settings include the strings,
\begin{itemize}
\item NOPRINT, to suppress the major part of the
output from the module;
\item IPRINT, to produce an intermediate level of output;
\item FPRINT, to produce output suitable for debugging purposes.
\end{itemize}
\item TEXTB is an optional parameter that should be set
to the string BYPASS if the user wishes to bypass 
DIAG processing. Such usage is typically
associated with restarting Table-CI calculations.
\end{itemize}

\subsection[EXTRAP]{EXTRAP}

This directive consists of one line read to variables TEXT, MAXE
using format (A,I).
\begin{itemize}
\item  TEXT should be set to the character string EXTRAP.
\item  MAXE specifies the maximum number of extrapolation cycles to be
carried out by the Davidson  diagonalizer.
\end{itemize}
The directive may be omitted, when MAXE will take the default value 2.

\subsection[ACCURACY]{ACCURACY}

This directive may be used to define the diagonalisation 
thresholds for the extrapolation passes and for the final
secular problem  (at the threshold T$_{min}$), and
consists of a single line read to variables
TEXT, THRESH0, THRESHF using format (A,2F).
\begin{itemize}
\item TEXT should be set to the character string ACCURACY  or DTHRESH;
\item THRESH0: On the  NEXTRP extrapolation passes, 
the diagonalization is converged to a threshold THRESH0;
\item THRESHF: On the final diagonalisation, solving the
secular problem corresponding to the threshold T$_{min}$,
the diagonalization is converged to a threshold THRESHF.
\end{itemize}
The THRESH directive may be omitted, when THRESH0 will be set to
0.005 and THRESHF to 0.001.\\

{\bf Example}\\

Assuming the default selection thresholds (T$_{min}$=10, T$_{inc}$=10)
and the default number of extrapolation passes (NEXTRP=2), then
presenting the data line

{
\footnotesize
\begin{verbatim}
           ACCURACY 0.005 0.0005
\end{verbatim}
}
will result in a diagonalisation threshold of 0.005 for the
two extrapolation passes (corresponding to solving the secular problem
at selection thresholds of 20 and 30 microhartree), and a threshold
of 0.0005 for the final secular problem, that corresponding to the
10 microhartree selection.

\subsection[MAXD]{MAXD}

This directive consists of one line read to variables TEXT, MAXD
using format (A,I).
\begin{itemize}
\item  TEXT should be set to the character string MAXD.
\item  MAXD specifies the maximum number of iterative cycles to be
carried out by the Davidson  diagonalizer.
\end{itemize}
The directive may be omitted, when MAXD will take the default value 50.


\subsection[TRIAL]{TRIAL}

This directive may be used to define a trial CI wavefunction
by diagonalising a sub-Hamiltonian obtained by
sampling the diagonal elements of the CI-Hamiltonian, and
selecting the lowest energy terms. 
The directive may be used to define the number of such
elements to be included in the sub-Hamiltonian, and
is read to the variables TEXT, NTRIAL using format (A,I).
\begin{itemize}
\item TEXT should be set to the character string TRIAL;
\item NTRIAL should be set to the number of lowest-energy
configurations in the CI list to be used in constructing the
sub-Hamiltonian.
\end{itemize}
 The TRIAL directive may be omitted, when NTRIAL will be
set to the maximum allowed value of 80.\\


\subsection[PRINT]{PRINT}

The PRINT directive may be used to control the printing of CI
coefficients and weights throughout the extrapolation passes
and in the final analysis.  This directive consists of a single data 
line read to variables
TEXT, PTHR, PTHRCC, IFLAG using format (A,2F,I).
\begin{itemize}
\item TEXT should be set to the character string PRINT.
\item  CI coefficients less than PTHR in absolute magnitude will not be
printed during the extrapolation passes.
\item  CI weights (coefficients$^{2}$) less than PTHRCC in absolute 
magnitude will not be printed in the final analysis of the CI
wavefunctions.
\item  IFLAG may be used to control the printing of the CI
wavefunctions in the event that the diagonalisation does not
converge. Setting IFLAG=1 will cause a detailed print of the
CI vectors corresponding to each  root.
\end{itemize}
 This directive may be omitted, when the defaults PTHR=0.05 and
PTHRCC=0.002 will be taken.

\section[Data for Conventional Table-CI Natural Orbitals]{Data for Conventional Table-CI Natural Orbitals}

\subsection[Natural Orbital Data - NATORB]{Natural Orbital Data - NATORB}
The NATORB directive is used to request
Natural Orbital (NO) generation, and comprises a single data line
read to the variables TEXT, TEXTF and TEXTB using format (3A).
\begin{itemize}
\item TEXT should be set to the character string NATORB.
\item TEXTF is an optional parameter that may be used to
control the quantity of printed output produced by the 
module. Valid settings include the strings,
\begin{itemize}
\item NOPRINT, to suppress the major part of the
output from the module;
\item IPRINT, to produce an intermediate level of output. This
option should be set to generate a print of the natural 
orbital coefficient array(s);
\item FPRINT, to produce output suitable for debugging purposes.
\end{itemize}
\item TEXTB is an optional parameter that should be set
to the string BYPASS if the user wishes to bypass 
NATORB processing. Such usage is typically
associated with restarting Table-CI calculations.
\end{itemize}
\subsection[Natural Orbital Data - CIVEC]{Natural Orbital Data - CIVEC}
The CIVEC directive is used to specify those eigenvectors
of the CI-matrix to be analysed. The directive
consists of a single data line with the character string CIVEC
in the first data field. If natural orbitals associated with
NVEC eigenvectors of the secular problem are to be generated, 
subsequent data fields should contain
NVEC integers, the integers specifying
the numbering of the CI-eigenvectors on the FORTRAN {\em interface},
FTN036, as generated by the DIAG sub-module. 
If the CIVEC directive is omitted under NATORB processing
the natural orbitals of the first CI-vector will be generated.\\

{\bf Example}
{
\footnotesize
\begin{verbatim}
          CIVEC 1 3
\end{verbatim}
}
The above data line
may be used to generate natural orbitals from the first and
third CI-eigenvector generated by the DIAG sub-module.
\subsection[Natural Orbital Data - PUTQ]{Natural Orbital Data - PUTQ}
The PUTQ directive may be used to route spin-free natural orbitals to
the Dumpfile, and consists of a single dataline 
with the first two fields read to variables
TEXT, TYPE  using format (2A).
\begin{itemize}
\item  TEXT should be set to the character string PUTQ.
\item  TYPE should be set to one of the character strings AOS,
A.O.  or SABF,
defining the basis representation required for the output NOs.
The character string AOS and A.O. will yield the 
NOs in the basis function
representation, suitable for subsequent input to the other
analysis modules of GAMESS--UK. The string SABF will result in
the NO expansion in the symmetry adapted basis representation, 
and should be used when performing iterative natural orbital
calculations (see section 6.13 below)
\end{itemize}
The remaining data consists of a sequence of NVEC integers,
(between 0 and 350 inclusive) specifying the
section number of the Dumpfile where the spin-free NOs 
derived from the NVEC CI-vectors nominated by the
CIVEC directive are to be placed.\\

{\bf Example}
{
\footnotesize
\begin{verbatim}
          PUTQ AOS 100 120
\end{verbatim}
}
The spin-free NOs in the basis-set representation
are output to sections 100
and 120 respectively of the Dumpfile. A section setting of 0 on the
PUTQ directive will act to suppress natural orbital output to the
Dumpfile.

\section[Data for Conventional Table-CI One-electron Properties]{Data for Conventional Table-CI One-electron Properties}

\subsection[PROP]{PROP}

The PROP directive is used to request the computation of one-electron
properties and comprises a single data line
read to the variables TEXT, TEXTF and TEXTB using format (3A).
\begin{itemize}
\item TEXT should be set to the character string PROP.
\item TEXTF is an optional parameter that may be used to
control the quantity of printed output produced by the 
module. Valid settings include the strings,
\begin{itemize}
\item NOPRINT, to suppress the major part of the
output from the module;
\item IPRINT, to produce an intermediate level of output;
\item FPRINT, to produce output suitable for debugging purposes.
\end{itemize}
\item TEXTB is an optional parameter that should be set
to the string BYPASS if the user wishes to bypass 
PROP processing. Such usage is typically
associated with restarting Table-CI calculations.
\end{itemize}

\subsection[CIVEC]{CIVEC}
The CIVEC directive is used to specify those eigenvectors
of the CI-matrix to be analysed. The directive
consists of a single data line with the character string CIVEC
in the first data field. If the properties associated with
NVEC eigenvectors of the secular problem are to be generated, 
subsequent data fields should contain
NVEC integers, the integers specifying
the numbering of the CI-eigenvectors on the FORTRAN {\em interface},
FTN036. If the CIVEC directive is omitted under PROP processing
an analysis of the first CI-vector will be performed.\\

{\bf Example}\\

The data line
{
\footnotesize
\begin{verbatim}
          CIVEC 1 3
\end{verbatim}
}
may be used to  analyse the  first and
third CI-eigenvector generated by the DIAG sub-module.


\subsection[AOPR]{AOPR}

The AOPR directive may be used to request printing of the
property integrals in the basis function (AO) representation.
If specified, the directive consists of a single data line
with the character string AOPR in the first data field. Subsequent
data fields are used to specify those integrals to be printed. Valid
character strings include S, T, X, Y, Z, XX, YY, ZZ, XY, XZ and   
YZ, requesting in obvious notation printing of the components
of the overlap,
kinetic energy, dipole and quadrupole moments respectively.\\

{\bf Example}
{
\footnotesize
\begin{verbatim}
          AOPR X Y Z
\end{verbatim}
}
would result in printing of integrals of the x-, y- and z-components
of the dipole moment.

\subsection[MOPR]{MOPR}

The MOPR directive may be used to request printing of the
property integrals in the molecular orbital (MO) basis.
If specified, the directive consists of a single data line
with the character string MOPR in the first data field. Subsequent
data fields are used to specify those integrals to be printed. Valid
character strings include S, T, X, Y, Z, XX, YY, ZZ, XY, XZ and   
YZ, requesting in obvious notation printing of the components
of the overlap,
kinetic energy, dipole and quadrupole moments respectively.\\

{\bf Example}
{
\footnotesize
\begin{verbatim}
          MOPR XX YY ZZ
\end{verbatim}
}
would result in printing of integrals of the diagonal components
of the quadrupole moment.

\subsection[Configuration Data Lines]{Configuration Data Lines}

In addition to evaluating the properties of a given CI-vector, 
the module will also look to evaluating the corresponding properties
of a nominated single configuration, typically the leading
term in the CI-vector: the idea here of course is to
provide a guide to the effect of the CI treatment on the property,
with the nominated CSF being typically the corresponding SCF
configuration. Thus the final data for the properties module
comprises a sequence of NVEC data lines, each line
a sequence of integers defining the single configuration for the
CI-vector under consideration. The format of these lines is
identical to that of the CONF data used in nominating the
reference functions, and in most instances will be a repeat of
that data.\\

{\bf Example}\\

Consider the valence-CI calculation on \phosphine\ described in
example 1 of the CONF directive. Considering just the CI data,

{
\footnotesize
\begin{verbatim}
          MRDCI
          TRAN 1 CORE
          4 1
          1 TO 4 1
          SELECT
          SINGLES 1
          CONF
          0 1 2 3 15
          0 1 3 4 15
          0 1 2 3 16
          NATORB
\end{verbatim}
}
then the first data line of the CONF directive specifies the
SCF configuration, and it is this configuration that should be
nominated in the PROP data. Thus the CI data including the 
property analysis would appear as follows,

{
\footnotesize
\begin{verbatim}
          MRDCI
          TRAN 1 CORE
          4 1
          1 TO 4 1
          SELECT
          SINGLES 1
          CONF
          0 1 2 3 15
          0 1 3 4 15
          0 1 2 3 16
          NATORB
          PROP
          CIVEC 1
          0 1 2 3 15
\end{verbatim}
}
where one such data line is required given the specification
of CIVEC.

\section[Data for Conventional Table-CI Transition Moments]{Data for Conventional Table-CI Transition Moments}

This Table-CI module will calculate both electrical and magnetic dipole 
moments as well as oscillator strengths and lifetimes of excited states.
The module will look to calculate the moment between a specific
state (typically the ground state) and a set of additional states 
(typically the excited states).
It is assumed that the  CI eigen-vectors have been
generated and are available on the appropriate FORTRAN {\em interfaces}.
While in most cases the ground and excited state CI-vectors will
reside on the same {\em interface}, FTN036, the module will
allow the use of differing data sets for these vectors, a situation most
likely to occur when the ground and excited states are of
different symmetry.

\subsection[TM]{TM}

The TM directive is used to request
Transition Moment analysis, and comprises two data lines. The
first line is
read to the variables TEXT, TEXTF and TEXTB using format (3A).
\begin{itemize}
\item TEXT should be set to the character string TM.
\item TEXTF is an optional parameter that may be used to
control the quantity of printed output produced by the 
module. Valid settings include the strings,
\begin{itemize}
\item NOPRINT, to suppress the major part of the
output from the module;
\item IPRINT, to produce an intermediate level of output;
\item FPRINT, to produce output suitable for debugging purposes.
\end{itemize}
\item TEXTB is an optional parameter that should be set
to the string BYPASS if the user wishes to bypass 
TM processing. Such usage is typically
associated with restarting Table-CI calculations.
\end{itemize}

The second data line of the TM directive is used
to specify the location of the CI-vectors, and the number
of excited state vectors involved in the subsequent analysis. The
line is read to the variables IFTNX, ISECX, IFTNE, ISECE, NSTATE
using format (5I);
\begin{itemize}    
\item IFTNX defines the FORTRAN data set reference number of the
{\em interface} holding the CI-vector of the first state.
Normally this vector will reside on FTN036, with IFTNX=36.

\item ISECX  defines the position of this first vector on
the {\em interface} defined by IFTNX. Typically, for the
first state of a given symmetry, the vector will be located
first on the data set i.e., ISECX=1.

\item IFTNE defines the FORTRAN data set reference number of the
{\em interface} holding the CI-vector(s) of the  set of
additional states. Assuming these states are of the same
symmetry as the first, then we would expect all the states
involved to lie on the same {\em interface} i.e., IFTNE will
also be set to 36. If, however, the set of states is of
different symmetry to the first, then their vectors will almost
certainly reside on a different {\em interface}, which we will
assume reside on FTN037 i.e., IFTNE should be set to 37.

\item ISECE  defines the position of the first 
of the excited state vectors  on
the {\em interface} defined by IFTNE. Typically, if all the 
states involved are of the same symmetry, residing on the same
data set (IFTNX=IFTNE), then
the first excited state vector will be located
second on the  data set i.e., ISECE=2.
When the first and excited states are of different symmetry, then
different data sets will be involved, and the first of the excited
state vectors will be the first on IFTNE.

\item NSTATE - defines the number of excited 
state vectors involved, and is usually equal to the 
number of transition moment calculations to be performed.
Thus if we wished to calculate the transition moment between the two
lowest states of \water, then NSTATE would equal 1, and the TM data
would appear as follows:

{
\footnotesize
\begin{verbatim}
          TM
          36 1 36 2 1
\end{verbatim}
}
\end{itemize}

\section[Calculating the \astate\ states of \water]{Calculating the \astate\ states of \water}

To clarify our discussion of the Table-CI module, we work through
a typical example of using the Table-CI method in calculating
the energetics and properties of the three low lying \astate\ 
states of the \water\ molecule. The basis set employed is the
TZVP triple-zeta plus polarisation set; this is augmented
with a diffuse s- and p-orbital on the oxygen to provide
a reasonable description of the known Rydberg character of
the states of interest.
The computation is split into four separate jobs, in which we,
\begin{enumerate}
\item perform the initial SCF;
\item carry out an initial CI, where the reference set employed
acts to provide at least a qualitative description of the states
of interest;
\item based on the output from the initial CI, we augment the
reference set to provide a quantitative description of the
first three states;
\item finally, having generated the CI vectors for the three
states, we carry out in the final job an analysis of each
vector in terms of natural orbitals and one-electron properties,
and generate the transition moments between the ground 
and two excited states.
\end{enumerate}
We now consider various aspects of each job in turn.\\

{\bf Job 1: The SCF}
{
\footnotesize
\begin{verbatim}
          TITLE  
          ****  H2O  TZVP + DIFFUSE S,P MRDCI *
          SUPER OFF NOSYM
          ZMAT ANGSTROM
          O
          H 1 0.951
          H 1 0.951 2 104.5
          END
          BASIS 
          TZVP O
          TZVP H
          S O
          1.0 0.02
          P O
          1.0 0.02
          END
          ENTER
\end{verbatim}
}
The only points to note here are (i) the use of the SUPER
directive in suppressing skeletonisation, and (ii) use
of the default section for eigenvector output (section 1 for
the closed-shell SCF).\\

{\bf Job 2: The Initial 3M/3R CI}\\

An examination of the SCF output
reveals the following orbital analysis.
{
\footnotesize
\begin{verbatim}
          =============================
          IRREP  NO. OF SYMMETRY ADAPTED
                 BASIS FUNCTIONS
          =============================
            1          18
            2           6
            3          10
            4           2
          =============================
\end{verbatim}
}
and the following orbital assignments characterising the closed--shell
SCF configuration:
\begin{equation}
  1a_{1}^{2}  2a_{1}^{2}  1b_{2}^{2}  3a_{1}^{2}  1b_{1}^{2}
\end{equation}
{
\footnotesize
\begin{verbatim}
          ===============================================
          M.O.  IRREP  ORBITAL ENERGY   ORBITAL OCCUPANCY
          ===============================================
            1     1    -20.56084959           2.0000000
            2     1     -1.35696939           2.0000000
            3     3     -0.72200122           2.0000000
            4     1     -0.58247942           2.0000000
            5     2     -0.50858566           2.0000000
            6     1      0.02724259           0.0000000
            7     3      0.04894440           0.0000000
            8     2      0.05589681           0.0000000
            9     1      0.06133571           0.0000000
           10     1      0.20403420           0.0000000
           11     3      0.22824210           0.0000000
           12     3      0.53700802           0.0000000
           13     1      0.56235022           0.0000000
           14     2      0.58645643           0.0000000
           15     1      0.66887228           0.0000000
           16     3      0.74805617           0.0000000
           17     1      1.07690608           0.0000000
           18     1      1.88545053           0.0000000
           19     4      1.92243836           0.0000000
           20     2      2.12944874           0.0000000
           21     3      2.20541910           0.0000000
           22     1      2.34202871           0.0000000
           23     3      2.39946430           0.0000000
           24     3      2.69788310           0.0000000
           25     1      2.72651832           0.0000000
           26     2      2.73832720           0.0000000
           27     1      3.07664215           0.0000000
           28     3      3.26840142           0.0000000
           29     2      3.54616570           0.0000000
           30     1      3.58631019           0.0000000
           31     4      3.59701772           0.0000000
           32     1      3.84174131           0.0000000
           33     1      4.84610143           0.0000000
           34     3      5.14220270           0.0000000
           35     1      7.73115986           0.0000000
           36     1     47.56758932           0.0000000
          ===============================================
\end{verbatim}
}
Based on the above output, the CONF data lines may be deduced from the
following table, where we assume that we wish to freeze the O1s inner
shell orbitals and discard the inner shell complement orbital:

\begin{centering}
\begin{tabular}{llrrrr}

\\ \hline
IRrep     & IRrep &  No. of Basis & Frozen & Active & Sequence  \\
          & No.   &  Functions    & MOs    & MOs    & Nos.       \\ \hline
 a$_{1}$  & 1     &   18          & 1      &  16    & 1-16       \\
 b$_{1}$  & 2     &   6           & 0      &  6     & 17-22   \\ 
 b$_{2}$  & 3     &   10          & 0      &  10    & 23-32       \\
 a$_{2}$  & 4     &   2           & 0      &  2     & 33-34   \\ \hline
\end{tabular}
 
\end{centering}
\vspace{.15in}
Note that the virtual SCF MOs dominated by the diffuse oxygen
basis functions are the 4a$_{1}$, the 2b$_{2}$, the 2b$_{1}$ and
the 5a$_{1}$, with SCF sequence numbers 6,7,8 and 9 respectively.
The symmetry re-ordered sequence numbers, allowing for the effective
removal of the two a$_{1}$ orbitals, are 3, 24, 18 and 5 respectively.
To perform a three-root 8-electron valence-CI calculation, based on the
SCF configurations of the ground and excited Rydberg states, involving
the single excitations (1b$_{1}$ to 2b$_{1}$) and (3a$_{1}$ to 4a$_{1}$)
would require the following CONF data:

{
\footnotesize
\begin{verbatim}
          CONF
          0        1 2  17  23
          2  2  3  1    17  23
          2 17 18  1 2  23
\end{verbatim}
}
The following data will perform this three root-CI, where
\begin{itemize}
\item the SCF computation is BYPASS'ed;
\item both CORE and DISCard are specified on the TRAN data line
flagging the freezing and discarding of the two a$_{1}$ MOs;
\item the default sub-module specifications are in effect, with no
specific need to reference ADAPT, CI or DIAG activity;
we also assume that the TABLE data set is available to the job;
\item the ROOT directive is specifying selection with respect
to the first 3 roots of the zero order problem, which we
assume will correspond to the states of interest.
\end{itemize}
{
\footnotesize
\begin{verbatim}
          RESTART NEW
          TITLE  
          ****            H2O  TZVP + DIFFUSE S,P TABLE-CI 3M/3R*
          SUPER OFF NOSYM
          BYPASS SCF
          ZMAT ANGSTROM
          O
          H 1 0.951
          H 1 0.951 2 104.5
          END
          BASIS 
          TZVP O
          TZVP H
          S O
          1.0 0.02
          P O
          1.0 0.02
          END
          RUNTYPE CI
          MRDCI
          TRAN CORE DISC
          1 0 0 0
          1
          1 0 0 0
          18
          SELECT
          CONF
          0        1 2  17  23
          2  2  3  1    17  23
          2 17 18  1 2  23
          THRE 10 10
          ROOT 3
          ENTER
\end{verbatim}
}

{\bf Job 3: The Final 12M/3R CI}\\

An examination of the output from the initial CI reveals that
the dominant configurations have, as expected, been included.
We show below the final CI vectors for each of the states:
not surprisingly the ground state is more accurate, by virtue
of its SCF MOs having been employed. Augmenting the reference
set to improve the description of the two excited states
follows straightforwardly from the statistics below: \\

{\bf Description of the X$^{1}A_{1}$ state}\\
{
\footnotesize
\begin{verbatim}
      EXTRAPOLATED ENERGY =   -76.2724087+/-0.0000790

       =================================================
           CSF NO.               C*C       CONFIGURATION
       =================================================
         (     1-     1)M     0.94952507     1   2  17  23
         (   118-   118)M     0.00002448     2   3   1  17  23
         (   129-   129)M     0.00016368    17  18   1   2  23

      SUM OF MAIN REFERENCE C*C = 0.94971323
\end{verbatim}
}
{\bf Description of the 1$^{1}A_{1}$ state}\\
{
\footnotesize
\begin{verbatim}
      EXTRAPOLATED ENERGY =   -75.9009069+/-0.0002121

       =================================================
           CSF NO.               C*C       CONFIGURATION
       =================================================
         (     1-     1)M     0.00020304     1   2  17  23
         (    71-    71)      0.00284448     1   2  18  23
         (   118-   118)M     0.06411957     2   3   1  17  23
         (   120-   120)      0.00757395     2   5   1  17  23
         (   129-   129)M     0.84004197    17  18   1   2  23
         (   130-   130)      0.00261567    17  19   1   2  23
         (   363-   363)      0.01150383    18  19   1   2  23
         (   886-   887)      0.00221466     1   5  17  18   2  23
         (  1226-  1227)      0.00227863     2   5  17  18   1  23
         (  1236-  1237)      0.00441966     2   6  17  18   1  23
         (  1246-  1247)      0.00550738     2   7  17  18   1  23
         (  1640-  1641)      0.00666055    17  18  23  25   1   2
         (  1644-  1645)      0.00664455    17  18  23  27   1   2
 
      SUM OF MAIN REFERENCE C*C = 0.90436458
\end{verbatim}
}
{\bf Description of the 2$^{1}A_{1}$ state}\\
{
\footnotesize
\begin{verbatim}
      EXTRAPOLATED ENERGY =   -75.8825335+/-0.0002079

       =================================================
           CSF NO.               C*C       CONFIGURATION
       =================================================
         (     1-     1)M     0.00017134     1   2  17  23
         (   118-   118)M     0.78352745     2   3   1  17  23
         (   119-   119)      0.00620129     2   4   1  17  23
         (   120-   120)      0.05602432     2   5   1  17  23
         (   129-   129)M     0.07076103    17  18   1   2  23
         (   211-   211)      0.00236797     3   5   1  17  23
         (   212-   212)      0.00343646     3   6   1  17  23
         (   213-   213)      0.00493434     3   7   1  17  23
         (  1212-  1213)      0.00870751     2   3  17  19   1  23
         (  1372-  1373)      0.00638562     2   3  23  25   1  17
         (  1376-  1377)      0.00600107     2   3  23  27   1  17
 
      SUM OF MAIN REFERENCE C*C = 0.85445982
\end{verbatim}
}
Taking as the criterion for inclusion a weight of 0.005, the final 12
reference set-CI is shown below. We have assumed that the FORTRAN {\em
interface} FTN031 has been saved from the second job, enabling us to
bypass the transformation. Note also the specific appearance now of
ADAPT in the data to enable bypassing.

{
\footnotesize
\begin{verbatim}
          RESTART CI
          TITLE  
          ****  H2O  TZVP + DIFFUSE S,P TABLE-CI 12M/3R *
          SUPER OFF NOSYM
          BYPASS SCF
          ZMAT ANGSTROM
          O
          H 1 0.951
          H 1 0.951 2 104.5
          END
          BASIS 
          TZVP O
          TZVP H
          S O
          1.0 0.02
          P O
          1.0 0.02
          END
          RUNTYPE CI
          MRDCI
          ADAPT BYPASS
          TRAN CORE DISC BYPASS
          1 0 0 0
          1
          1 0 0 0
          18
          SELECT
          CONF
          0        1 2  17  23
          2  2  3  1    17  23
          2  2  4  1    17  23
          2  2  5  1    17  23
          2 17 18  1 2  23
          2 18 19  1 2  23
          4 17 18 23 25  1  2
          4 17 18 23 27  1  2
          4  2  7 17 18  1  23
          4  2  3 17 19  1  23
          4  2  3 23 25  1  17
          4  2  3 23 27  1  17
          THRE 10 10
          ROOT 3
          ENTER
\end{verbatim}
}
{\bf Job 4: The Analysis}\\

Assuming that the diagonalisation {\em interface}, FTN036, had been
saved above, then the final analysis job is straightforward: again
bypassing of the various sub-modules involves explicit
mention of the ADAPT, CI and DIAG modules, in addition to
flagging the previous TRAN and SELECT data lines with the BYPASS
keyword. We have routed the natural orbitals from the 3 A$_{1}$ states
to the Dumpfile using the PUTQ directive.

{
\footnotesize
\begin{verbatim}
          RESTART CI
          TITLE  
          ****  H2O  TZVP + DIFFUSE S,P TABLE-CI / ANALYSIS *
          SUPER OFF NOSYM
          BYPASS SCF
          ZMAT ANGSTROM
          O
          H 1 0.951
          H 1 0.951 2 104.5
          END
          BASIS 
          TZVP O
          TZVP H
          S O
          1.0 0.02
          P O
          1.0 0.02
          END
          RUNTYPE CI
          MRDCI
          ADAPT BYPASS
          TRAN CORE DISC BYPASS
          1 0 0 0
          1
          1 0 0 0
          18
          SELECT BYPASS
          CONF
          0        1 2  17  23
          2  2  3  1    17  23
          2  2  4  1    17  23
          2  2  5  1    17  23
          2 17 18  1 2  23
          2 18 19  1 2  23
          4 17 18 23 25  1  2
          4 17 18 23 27  1  2
          4  2  7 17 18  1  23
          4  2  3 17 19  1  23
          4  2  3 23 25  1  17
          4  2  3 23 27  1  17
          THRE 10 10
          ROOT 3
          CI BYPASS
          DIAG BYPASS
          NATORB IPRIN
          CIVE 1 2 3
          PUTQ AOS 50 51 52
          PROP
          CIVE 1 2 3
          0        1 2  17  23
          2 17 18  1 2  23
          2  2  3  1    17  23
          MOMENT
          36 1 36 2 2
          ENTER
\end{verbatim}
}

{\bf Description of the Output for MRDCI Moments}

The MRDCI module calculates the oscillator strength using both the
dipole length formalism: 

$$f({\bf r}) = 2/3 <\Psi^{'}|{\bf r}|\Psi^{''}>^{2}/ \Delta E$$

and the dipole velocity formalism: 

$$f(\bigtriangledown) = 2/3 |<\Psi^{'}|{\bf \bigtriangledown}|\Psi^{''}>^{2}/ \Delta E$$

The most significant contributions due to individual molecular orbitals 
are  printed out as a table containing the largest coefficients of the 
transition density matrix and the following corresponding integrals.

$$<\psi_{i}|x|\psi_{j}> \hspace{.75in}   <\psi_{i}|y|\psi_{j}> \hspace{.75in}   <\psi_{i}|z|\psi_{j}>$$ 

$$<\psi_{i}|\bigtriangledown(x)|\psi_{j}> \hspace{.75in}   <\psi_{i}|\bigtriangledown(y)|\psi_{j}> \hspace{.75in}  <\psi_{i}|\bigtriangledown(z)|\psi_{j}>$$ 


The f({\bf r}) and f($\bigtriangledown$) values are printed out in x,y,z 
components and the expectation values for 
$<\psi|\sum_{i} x_{i},y_{i}, z_{i}|\psi>$ are also printed.

\section[Iterative Natural Orbital Calculations]{Iterative Natural Orbital Calculations}

We work through an example of using the natural orbitals generated by
the module in a subsequent CI calculation.  We consider a DZ
calculation on the ground state of \ethene, with the computation split
into four separate jobs, in which we,
\begin{enumerate}
\item perform the initial SCF;
\item carry out an initial CI, where the reference set employed
comprises just the SCF configuration, using the SCF MOs;
of interest;
\item based on the output from the initial CI, we augment the
reference set to include the leading secondary configuration,
generating the resulting natural orbitals;
\item carry out the 2-reference CI based on the natural orbitals
generated in the previous step.
\end{enumerate}
We now consider various aspects of each job in turn.\\

{\bf Job 1: The SCF}
{
\footnotesize
\begin{verbatim}
          TITLE 
          ETHYLENE DZ GROUND STATE SCF
          SUPER OFF NOSYM
          ZMATRIX ANGSTROM
          C
          C 1 1.4
          H 1 1.1 2 120.0
          H 1 1.1 2 120.0 3 180.0
          H 2 1.1 1 120.0 3 0.0
          H 2 1.1 1 120.0 3 180.0
          END
          BASIS DZ
          ENTER
\end{verbatim}
}
The only points to note here is the use of the SUPER directive in
suppressing skeletonisation, and use of the default eigenvector section
(section 1) for storage of the closed-shell eigenvectors.\\

{\bf Job 2: The Initial 1M/1R CI}\\

An examination of the SCF output
reveals the following orbital analysis.
{
\footnotesize
\begin{verbatim}
           ==============================
           IRREP  NO. OF SYMMETRY ADAPTED
                  BASIS FUNCTIONS
           ==============================
             1           8
             2           2
             3           4
             5           8
             6           2
             7           4
           =============================
\end{verbatim}
}
and the following orbital assignments characterising the closed--shell
SCF configuration:
\begin{equation}
  1a_{g}^{2}  1b_{1u}^{2}  2a_{g}^{2}  2b_{1u}^{2}  1b_{2u}^{2} 3a_{g}^{2}  1b_{3g}^{2}  1b_{3u}^{2}  
\end{equation}
{
\footnotesize
\begin{verbatim}
          ===============================================
          M.O.  IRREP  ORBITAL ENERGY   ORBITAL OCCUPANCY
          ===============================================
            1     1    -11.25533463           2.0000000
            2     5    -11.25413119           2.0000000
            3     1     -1.02052567           2.0000000
            4     5     -0.79195744           2.0000000
            5     3     -0.64152856           2.0000000
            6     1     -0.57038928           2.0000000
            7     7     -0.51438205           2.0000000
            8     2     -0.36388693           2.0000000
            9     6      0.13190687           0.0000000
           10     5      0.25441028           0.0000000
           11     1      0.25558269           0.0000000
           12     3      0.33918097           0.0000000
           13     5      0.36641681           0.0000000
           14     1      0.41844853           0.0000000
           15     3      0.45487336           0.0000000
           16     7      0.49648833           0.0000000
           17     2      0.49708917           0.0000000
           18     6      0.60222488           0.0000000
           19     7      0.64242537           0.0000000
           20     1      0.76465797           0.0000000
           21     5      0.82560838           0.0000000
           22     5      1.10140194           0.0000000
           23     1      1.20630804           0.0000000
           24     3      1.30189632           0.0000000
           25     5      1.35219192           0.0000000
           26     7      1.50761510           0.0000000
           27     1     23.76609415           0.0000000
           28     5     24.01493168           0.0000000
           ===============================================
\end{verbatim}
}
Based on the above output, the CONF data lines may be
deduced from the following table, where we
assume that we wish to freeze the C1s inner shell orbitals:

\begin{centering}
\begin{tabular}{llrrrr}

\\ \hline
IRrep & IRrep &  No. of Basis & Frozen & Active & Sequence  \\
      & No.  &   Functions    & MOs    & MOs    & Nos.       \\ \hline
 a$_{g}$   & 1  &   8         & 1      &  7     & 1-7       \\
 b$_{3u}$  & 2  &   2         & 0      &  2     & 8-9   \\ 
 b$_{2u}$  & 3  &   4         & 0      &  4     & 10-13       \\
 b$_{1u}$  & 5  &   8         & 1      &  7     & 14-20   \\ 
 b$_{2g}$  & 6  &   2         & 0      &  2     & 21-22   \\ 
 b$_{3g}$  & 7  &   4         & 0      &  4     & 23-26   \\  \hline
\end{tabular}
 
\end{centering}
\vspace{.15in}
Note that within the DZ basis employed, there are no basis functions of
b$_{1g}$ (IRrep 4) or  a$_{u}$ (IRrep 8) symmetry.
The symmetry re-ordered sequence numbers of the ground
state orbitals, allowing for the
effective removal of the 1a$_{g}$ and 1b$_{1u}$ orbitals, 
are 1, 14, 10, 2, 23 and 8 respectively.
To perform a single reference 12-electron valence-CI calculation,
based on the SCF configuration 
would require the following CONF data:

{
\footnotesize
\begin{verbatim}
          CONF
          0 1 2 8 10 14 23
\end{verbatim}
}
The following data will perform this CI, where
\begin{itemize}
\item the SCF computation is BYPASS'ed;
\item CORE is specified on the TRAN data line
flagging the freezing and discarding of the 1a$_{g}$ and 1b$_{1u}$ MOs.
The default section for retrieval of the closed-shell eigenvectors
is assumed;
\item the default sub-module specifications are in effect, with no
specific need to reference ADAPT, CI or DIAG activity;
we also assume that the TABLE data set is available to the job;
\end{itemize}
{
\footnotesize
\begin{verbatim}
          RESTART
          TITLE 
          ETHYLENE CI GROUND STATE SCF-MOS
          BYPASS SCF
          ZMATRIX ANGSTROM
          C
          C 1 1.4
          H 1 1.1 2 120.0
          H 1 1.1 2 120.0 3 180.0
          H 2 1.1 1 120.0 3 0.0
          H 2 1.1 1 120.0 3 180.0
          END
          BASIS DZ
          RUNTYPE CI
          MRDCI
          TRAN CORE 
          1 0 0 1 0 0 
          1 1
          SELECT
          SYMMETRY 1
          SPIN 1
          SINGLES 1
          CONF
          0 1 2 8 10 14 23
          ENTER
\end{verbatim}
}

{\bf Job 3: The 2M/1R CI}\\

We show below the final CI vector for the ground state.\\

{\bf Description of the X$^{1}A_{g}$ state}\\
{
\footnotesize
\begin{verbatim}
      EXTRAPOLATED ENERGY =   -78.1944869+/-0.0001635 

       =================================================
           CSF NO.               C*C       CONFIGURATION
       =================================================

         (     1-     1)M     0.90889665     1   2   8  10  14  23
         (    43-    43)      0.01645740     1   2  10  14  21  23
         (   142-   142)      0.00543252    21  22   1   2  10  14  23
         (   374-   375)      0.00328957     2   8  17  21   1  10  14  23
\end{verbatim}
}
Taking as the criterion for inclusion a weight of 0.01, the 2 reference
set-CI job is shown below. We have assumed that the FORTRAN {\em
interface} FTN031 has been saved from the second job, enabling us to
bypass the transformation. Note also the specific appearance now of
ADAPT in the data to enable bypassing.  The natural orbitals are routed
to section 200 of the Dumpfile. Note that subsequent usage of the NOs
by the Conventional Table-CI module requires the SABF specification on
the PUTQ directive.

{
\footnotesize
\begin{verbatim}
          RESTART NEW
          TITLE 
          ETHYLENE CI GROUND STATE 2M SCF-MOS
          BYPASS SCF
          ZMATRIX ANGSTROM
          C
          C 1 1.4
          H 1 1.1 2 120.0
          H 1 1.1 2 120.0 3 180.0
          H 2 1.1 1 120.0 3 0.0
          H 2 1.1 1 120.0 3 180.0
          END
          BASIS DZ
          RUNTYPE CI
          MRDCI
          ADAPT BYPASS
          TRAN CORE BYPASS
          1 0 0 1 0 0 
          1 1
          SELECT
          SYMMETRY 1
          SPIN 1
          SINGLES 1
          CONF
          0 1 2 8 10 14 23
          0 1 2   10 14 21 23
          NATORB IPRINT
          PUTQ SABF 200
          ENTER
\end{verbatim}
}

{\bf Job 4: The Natural Orbital CI}\\

We show below the data for using the NOs from the 2-reference
CI, where the orbitals routed to section 200 are now restored
by specification on the TRAN directive. The following points
should be noted:
\begin{itemize}
\item We assume that the adaptation {\em interface} FTN022 has
been saved from the initial CI job, allowing the BYPASS
specification on the ADAPT directive.
\item Restoring the NOs {\em must} be controlled via TRAN
specification; an attempt to restore such orbitals 
through VECTORS and ENTER specification will lead to an error
condition.
\item The resulting NOs from the natural orbital CI are now
routed to section 210, and could be used in a subsequent CI in
obvious fashion.
\end{itemize}

{
\footnotesize
\begin{verbatim}
          RESTART NEW
          TITLE 
          ETHYLENE CI GROUND STATE 2M NOS
          BYPASS SCF
          ZMATRIX ANGSTROM
          C
          C 1 1.4
          H 1 1.1 2 120.0
          H 1 1.1 2 120.0 3 180.0
          H 2 1.1 1 120.0 3 0.0
          H 2 1.1 1 120.0 3 180.0
          END
          BASIS DZ
          RUNTYPE CI
          MRDCI
          ADAPT BYPASS
          TRAN 200 CORE 
          1 0 0 1 0 0 
          1 1
          SELECT
          SYMMETRY 1
          SPIN 1
          SINGLES 1
          CONF
          0 1 2 8 10 14 23
          0 1 2   10 14 21 23
          NATORB IPRINT
          PUTQ SABF 210
          ENTER
\end{verbatim}
}
We show below the final CI vector from the natural orbital CI\\

{\bf Description of the X$^{1}A_{g}$ state}\\
{
\footnotesize
\begin{verbatim}
      EXTRAPOLATED ENERGY =   -78.2021125+/-0.0002347

       =================================================
           CSF NO.               C*C       CONFIGURATION
       =================================================
         (     1-     1)M     0.89795112     1   2   8  10  14  23
         (    29-    29)M     0.03202257     1   2  10  14  21  23
         (   194-   195)      0.00279273     1   8  15  21   2  10  14  23
         (   308-   309)      0.00499259     2   8  15  21   1  10  14  23
         (   430-   431)      0.00259858     3   8  14  21   1   2  10  23
         (   514-   515)      0.00297122    10  11  23  24   1   2   8  14


      SUM OF MAIN REFERENCE C*C = 0.92993884
\end{verbatim}
}

\section[Table-CI Calculations Using MCSCF Orbitals]{Table-CI Calculations Using MCSCF Orbitals}

To conclude our discussion of the Conventional Table-CI module, we work
through an example of using the natural orbitals generated from the
MCSCF module in a subsequent CI calculation.  We consider a calculation
on the ground state of \formaldehyde, with the computation split into
three separate jobs, in which we,
\begin{enumerate}
\item perform an initial SCF;
\item carry out the MCSCF calculation;
\item perform the MRDCI calculation using the MCSCF natural orbitals.
\end{enumerate}
We now consider various aspects of each job in turn.\\

{\bf Job 1: The SCF}

{
\footnotesize
\begin{verbatim}
          TITLE 
          H2CO - DZP + F
          SUPER OFF NOSYM
          ZMATRIX ANGSTROM
          C
          O 1 1.203
          H 1 1.099 2 121.8
          H 1 1.099 2 121.8 3 180.0
          END
          BASIS
          DZP O
          DZP C
          DZP H
          F C
          1 1.0
          F O
          1.0 1.0
          END
          ENTER
\end{verbatim}
}
The only points to note here is the use of the SUPER directive in
suppressing skeletonisation, and use of the default eigenvector section
(section 1) for storage of the closed-shell eigenvectors.  An
examination of the SCF output reveals the following orbital analysis.

{
\footnotesize
\begin{verbatim}
           ==============================
           IRREP  NO. OF SYMMETRY ADAPTED
                  BASIS FUNCTIONS
           ==============================
             1          28
             2          13
             3          16
             4           5
           =============================
\end{verbatim}
}
and the following orbital assignments characterising the closed--shell
SCF configuration:

{
\footnotesize
\begin{verbatim}
           ===============================================
           M.O.  IRREP  ORBITAL ENERGY   ORBITAL OCCUPANCY
           ===============================================
              1     1    -20.57768533           2.0000000
              2     1    -11.34457777           2.0000000
              3     1     -1.40746540           2.0000000
              4     1     -0.87003449           2.0000000
              5     3     -0.69591811           2.0000000
              6     1     -0.65109519           2.0000000
              7     2     -0.53687971           2.0000000
              8     3     -0.44174805           2.0000000
              9     2      0.11697212           0.0000000
             10     1      0.26220763           0.0000000
             11     1      0.27217357           0.0000000
             12     3      0.38931080           0.0000000
             13     3      0.41757152           0.0000000
             14     2      0.46526352           0.0000000
             15     1      0.60968525           0.0000000
             16     1      0.75001014           0.0000000
             17     2      0.86980119           0.0000000
             18     1      0.89167074           0.0000000
             19     3      0.93051881           0.0000000
             20     1      1.07098621           0.0000000
             21     3      1.18616042           0.0000000
             22     1      1.35370640           0.0000000
             23     4      1.52221224           0.0000000
             24     2      1.68895841           0.0000000
             25     1      1.88480823           0.0000000
             26     3      1.97392259           0.0000000
             27     4      2.13452238           0.0000000
             28     1      2.13982211           0.0000000
             29     3      2.19187925           0.0000000
             30     1      2.34994146           0.0000000
             31     2      2.36355601           0.0000000
             32     4      2.67698155           0.0000000
             33     1      2.82812279           0.0000000
             34     2      2.84696649           0.0000000
             35     3      2.97321688           0.0000000
             36     1      3.14466153           0.0000000
             37     1      3.33275225           0.0000000
             38     3      3.51306001           0.0000000
             39     1      3.51697350           0.0000000
             40     2      3.52974692           0.0000000
             41     3      3.68500148           0.0000000
             42     1      3.79236483           0.0000000
             43     3      3.83958499           0.0000000
             44     4      3.85444369           0.0000000
             45     2      3.87171104           0.0000000
             46     2      4.14123645           0.0000000
             47     3      4.16304613           0.0000000
             48     1      4.16535425           0.0000000
             49     2      4.16620625           0.0000000
             50     2      4.32796704           0.0000000
             51     3      4.47390388           0.0000000
             52     1      4.49325977           0.0000000
             53     1      4.68584478           0.0000000
             54     4      4.89513471           0.0000000
             55     3      5.13208791           0.0000000
             56     1      5.14756255           0.0000000
             57     2      5.85901984           0.0000000
             58     1      5.92954017           0.0000000
             59     3      6.06228738           0.0000000
             60     1      8.35480844           0.0000000
             61     1     27.71561806           0.0000000
             62     1     45.72664766           0.0000000
           ===============================================
\end{verbatim}
}

{\bf Job 2: The MCSCF}\\

The following data performs a 10 electron in 9 orbital CASSCF
calculation using the MCSCF module, with the natural orbitals routed to
section 10 of the Dumpfile under control of the CANONICAL directive. In
the absence of the VECTORS directive, the SCF MOs will be used as the
starting orbitals.

{
\footnotesize
\begin{verbatim}
          RESTART
          TITLE
          H2CO  - MCSCF (10E IN 9 M.O.)
          SUPER OFF NOSYM
          NOPRINT
          BYPASS
          ZMATRIX ANGSTROM
          C
          O 1 1.203
          H 1 1.099 2 121.8
          H 1 1.099 2 121.8 3 180.0
          END
          BASIS
          DZP O
          DZP C
          DZP H
          F C
          1 1.0
          F O
          1.0 1.0
          END
          SCFTYPE MCSCF
          MCSCF
          ORBITAL
          COR1 COR1 COR1 DOC1 DOC3 DOC1 DOC2 DOC3 UOC2 UOC1 UOC3 UOC1
          END
          PRINT ORBITALS VIRTUALS NATORB
          CANONICAL 10 FOCK DENSITY FOCK
          ENTER
\end{verbatim}
}

{\bf Job 3: The Table-CI Job}\\

Performing a Table-CI calculation using the natural orbitals generated in
the previous step is fairly straightforward. The following points should be
noted:
\begin{itemize}
\item The MCSCF data presented in the preceding step must remain as
part of the input data set, with that computation now BYPASS'ed.
\item Specification of the input orbital set must be driven through
section specification on the TRAN directive; without such
specification, the default MCSCF sections will be used which are not
appropriate as input to the subsequent CI. In the data below this is
achieved through the data line "TRAN 10", where the section specified is
just that nominated on the CANONICAL directive.
\item With no frozen or discarded orbital, the orbital indices
specified on the CONF directive follow in obvious fashion from the list
of IRREPs given above. We are performing a simple 16 electron, 3
reference calculation, deriving just the first root, and using a 2
micro-hartree threshold.  
\end{itemize}

{
\footnotesize
\begin{verbatim}
          RESTART
          TITLE
          H2CO  - MCSCF (10E IN 9 M.O.) MRDCI FROM MCSCF NOS
          SUPER OFF NOSYM
          BYPASS SCF
          ZMAT ANGSTROM
          C
          O 1 1.203
          H 1 1.099 2 121.8
          H 1 1.099 2 121.8 3 180.0
          END
          BASIS
          DZP O
          DZP C
          DZP H
          F C
          1 1.0
          F O
          1.0 1.0
          END
          RUNTYPE CI
          SCFTYPE MCSCF
          MCSCF
          ORBITAL
          COR1 COR1 COR1 DOC1 DOC3 DOC1 DOC2 DOC3 UOC2 UOC1 UOC3 UOC1
          END
          PRINT ORBITALS VIRTUALS NATORB
          CANONICAL 10 FOCK DENSITY FOCK
          MRDCI
          ADAPT
          TRAN 10
          SELECT
          SYMMETRY 1
          SPIN 1
          CNTRL 16
          SINGLES 1
          CONF
          0 1 2 3 4 5  29  42 43
          0 1 2 3 4 5  30  42 43
          0 1 2 3 4 5  29  42 44
          ROOTS 1
          THRESH 2 2
          CI
          DIAG
          ENTER
\end{verbatim}
}

Finally, we consider the data for performing exactly the same
calculation as above, but now freezing the oxygen and carbon 1s core
orbitals in the Table-CI calculation. The following points should be
noted:
\begin{itemize}
\item The TRAN directive now appends the CORE descriptor after the
section specification, with the following 2 data lines requesting
that the first two orbitals of symmetry 1 be removed from the
Table-CI calculation.
\item The orbital indices specified on the CONF data lines reflect
the removal of these two orbitals, with the CNTRL directive
now pointing to a 12-electron CI calculation, as distinct from
the 16 electron calculation above.
\end{itemize}

{
\footnotesize
\begin{verbatim}
          RESTART
          TITLE
          H2CO  - MRDCI FROM MCSCF NOS - FREEZE 1S MOS
          SUPER OFF NOSYM
          BYPASS SCF
          ZMAT ANGSTROM
          C
          O 1 1.203
          H 1 1.099 2 121.8
          H 1 1.099 2 121.8 3 180.0
          END
          BASIS 
          DZP O
          DZP C
          DZP H
          F C
          1 1.0
          F O
          1.0 1.0
          END
          RUNTYPE CI
          SCFTYPE MCSCF
          MCSCF
          ORBITAL
          COR1 COR1 COR1 DOC1 DOC3 DOC1 DOC2 DOC3 UOC2 UOC1 UOC3 UOC1
          END
          PRINT ORBITALS VIRTUALS NATORB
          CANONICAL 10 FOCK DENSITY FOCK
          MRDCI 
          ADAPT
          TRAN 10 CORE
          2 0 0 0 
          1 2 
          SELECT
          SYMMETRY 1
          SPIN 1
          CNTRL 12
          SINGLES 1
          CONF
          0 1 2 3  27  40 41
          0 1 2 3  28  40 41
          0 1 2 3  27  40 42
          ROOTS 1
          THRESH 2 2
          CI
          DIAG
          ENTER
\end{verbatim}
}

\section[The Semi-direct Table-CI Module]{The Semi-direct Table-CI Module}

We now consider the data requirements and file structure of the new,
semi-direct version of the Table-CI module that is capable of
performing significantly larger calculations. The main differences as
far as the user is concerned include the following.
\begin{itemize}
\item The original adapt and transformation modules of the Conventional
Table-CI module have now been replaced by the standard 4-index transformation
module of GAMESS-UK.
\item Semi-direct Table-CI calculations require at least two reference
configurations i.e. CISD calculations based on a single reference
configuration are not possible with this module. However we do not
consider this to be a major disadvantage given that the process of
configuration choice and specification has been simplified through the
use of automated configuration generation (see below).
\item The original CI and Diagonalisation modules have now been
condensed into a single CI module.
\item The formal limits that apply to conventional calculations are
significantly extended in the semi-direct module.  There is now a
limit of 800,000 selected configurations derived from an
initial list of configurations generated by single plus double
excitations from a user--specified list of reference functions, the
number of which may not exceed 256.  The selection and extrapolation
procedure may now be applied on up to thirty roots of a given secular
problem.
\item The memory requirements of the semi-direct module may
be significantly greater than those associated with the conventional
algorithm. While the default memory allocations will prove
adequate for "small-medium" cases, the user should use the MEMORY
pre-directive to request at least 8 MWords in calculations with,
say, more than 20 active electrons.
\item The overall filespace requirements are significantly reduced
compared to the Conventional module; note that the FORTRAN unit numbers
for some the key data sets have been modified compared to the original
settings.
\item An automatic procedure for obtaining a number of excited states
within a single given run of the program is now possible under control
of the ITERATE directive. This is designed to remove much of the labour
involved in generating, for example, vertical excitation spectra.
\end{itemize} 

\subsection[Sub-Module Structure of Semi-direct Table-CI]{Sub-Module Structure of Semi-direct Table-CI}

An outline of the sub-module structure and philosophy
behind the semi-direct Table-CI module has already been given in Part 2, material
that should be taken in conjunction with the present chapter.
As pointed out previously,
this module comprises a reduced set of 6 sub-modules,
which must be user--driven (either implicitly or explicitly, see below)
through data input. These sub modules are as follows:
\begin{itemize}
\item TABLE: generates an input a `data-base' of pattern symbolic
matrix elements for use in both the selection process and in solving
the secular problem. This data base is written to a file with LFN
{\em table-ci} (note the name change compared to the LFN TABLE employed
in the Conventional module).
\item SELECT: configuration generation and subsequent selection
based on a user-specified set of reference configurations and
appropriate thresholds;
\item CI: provides pre-processing prior to the semi-direct evaluation
of the CI eigenfunctions, followed by the calculation, in semi-direct
fashion, of one or more CI eigenfunctions of the secular problem. In
contrast to the conventional module, just two secular problems are
solved as part of the extrapolation process, one at the lowest
threshold (T$_{min}$) and one at the threshold (T$_{min}$ +
T$_{inc}$).
\item NATORB: generates the spin-free natural orbitals for one or more
of the calculated CI eigenvectors. Note that this module is now
executed in default.
\end{itemize}
     
AS with the original code, the remaining modules are optional, and may 
be used to further analyse one or more of the CI eigenvectors:
\begin{itemize}
\item PROP: to compute various 1-electron properties of the
CI wavefunctions. Note that the natural orbitals generated
above may be routed to the Dumpfile and examined by the
other analysis modules of GAMESS--UK in a subsequent job.
\item TM: to compute the transition moments between nominated
CI eigenvectors.
\end{itemize}

In addition to the Mainfile, Dumpfile, Scratchfile amd Transformed
Integral File (ED6), the following data sets will be used by the
program.
\begin{itemize}
\item The Tablefile: A dataset normally assigned using the local file
name "table-ci" will be used as a source of pattern symbolic matrix
elements in the SELECT and CI phases of the Table-CI procedure.  The
space requirements of the Tablefile are now about 6 MBytes.
\item In contrast to the other post-Hartree Fock modules of GAMESS--UK, the
Table-CI routines make extensive use of unformatted sequential
FORTRAN data sets (or {\em interfaces}). The data set reference
numbers and associated LFNs of these files have been given in
Table 8 of Part 2.
\end{itemize}

\section[Directives Controlling Semi-direct Table-CI Calculations]{Directives Controlling Semi-direct Table-CI Calculations}

Data input characterising the Semi-direct Table-CI calculation
commences with the MRDCI data line, and is typically followed by a
sequence of directives, terminated by presenting a valid {\em Class 2}
directive, such as VECTORS or ENTER. A fairly thorough overview of the
data structure has been given in Part 2: we provide additional detail
on the directives associated with each sub-module below.

\subsection[MRDCI]{MRDCI}

The Table-CI data initiator for the semi-direct module consists of a
single line containing the character string MRDCI in the first data
field, and the string DIRECT in the second. It acts to transfer control
to those routines responsible for inputing all data relevant to the
MRDCI calculation. Termination of this data is achieved by presenting a
valid {\em Class 2} directive that is not recognised by the Table-CI
input routines, for example, VECTORS or ENTER.

\section[Data for Semi-direct Table-CI Integral Transformation]{Data for Semi-direct Table-CI Integral Transformation}

In contrast to the Conventional Table-CI code, integral transformation
is now performed under control of the conventional transformation code
of GAMESS-UK. Control of the transformation is now carried out under
the ACTIVE and CORE directives described previously in Part 5 of the
manual. The following points should be noted:
\begin{enumerate}
\item The MOs to be used in the transformation process will be taken
from the section as nominated on the ENTER directive, or the default
section in effect if explicit section specification is omitted.
\item The transformation may be bypassed under control of the BYPASS
directive. Such usage is typically associated with restarting Table-CI
calculations.
\item The freezing and discarding of orbitals in the transformation is
controlled by a combination of the CORE and ACTIVE directives.
\item Users of the old module will be familiar with the ordering of the
MOs required within the Table-CI module, namely in terms of irreducible
representation (IRrep) and numbering within each IRrep. While this
ordering is still the preferred means of driving the CI module and
associated specification of the reference functions etc., it is no
longer necessary to adhere to this numbering scheme when specifying
CORE and ACTIVE orbitals, for the code will automatically generate the
appropriate numbering of these MOS. This is best illustrated by
considering the same examples used at the beginning of the chapter.
\item Note again that the input orbital set will be reordered by the
transformation module such that
\begin{itemize}
\item IRreps having zero orbitals are discarded, and
\item orbitals of common IRrep are grouped together, these groups
being arranged in order of increasing IRrep number, and
\item orbitals of common IRrep are ordered according to their
relative disposition in the input orbital set e.g., by eigenvalue
ordering if SCF MOs.
\end{itemize}
\item Generation of the Transformed Integral Interface (to FTN031) is
carried out on completion of the integral transformation, using the
integrals written to the Transformed Integral File (ED6). This
conversion is triggered by the appearance of the "MRDCI DIRECT" data
line in the job input; any attempt to use the Transformed Integral File
produced, say, during a run of the Direct-CI module (see Part 5) as
input to a subsequent semi-direct Table-CI calculation will lead to an
error condition, as the generation of FTN031 would have not have
attempted by the previous run of the transformation module.
\end{enumerate}

{\bf Example 1}\\

In this example we wish to perform a valence-CI calculation on the
\formaldehyde\ molecule using a DZ basis of 24 gtos, looking to freeze
both the oxygen and carbon 1s orbitals, and to discard the inner--shell
complement orbitals. An examination of the SCF output reveals the
following orbital analysis.

{
\footnotesize
\begin{verbatim}
          =============================
          IRREP  NO. OF SYMMETRY ADAPTED
                 BASIS FUNCTIONS
          =============================
            1          14
            2           4
            3           6
          =============================
\end{verbatim}
}
with the following orbital assignments from the closed
shell SCF:
{
\footnotesize
\begin{verbatim}
          ===============================================
          M.O.  IRREP  ORBITAL ENERGY   ORBITAL OCCUPANCY
          ===============================================
            1      1    -20.58952765           2.0000000
            2      1    -11.35779935           2.0000000
            3      1     -1.43525479           2.0000000
            4      1     -0.87463564           2.0000000
            5      3     -0.70990765           2.0000000
            6      1     -0.64751394           2.0000000
            7      2     -0.53989416           2.0000000
            8      3     -0.44423257           2.0000000
            9      2      0.10853108           0.0000000
           10      1      0.25726604           0.0000000
           11      1      0.28106873           0.0000000
           12      3      0.38903939           0.0000000
           13      3      0.40966861           0.0000000
           14      2      0.46216570           0.0000000
           15      1      0.65466944           0.0000000
           16      1      0.82879998           0.0000000
           17      2      0.98111608           0.0000000
           18      1      0.98701051           0.0000000
           19      3      1.07064863           0.0000000
           20      1      1.16621340           0.0000000
           21      3      1.29856111           0.0000000
           22      1      1.82320845           0.0000000
           23      1     23.76352004           0.0000000
           24      1     43.36689896           0.0000000
          ===============================================
\end{verbatim}
}
Thus the orbitals of interest are of common IRrep (a$_{1}$),
with sequence numbers 1,2 (core) and 23,24 (complement MOs)
within the SCF orbital set.  The following ACTIVE and CORE
data would freeze and discard these MOs:

{
\footnotesize
\begin{verbatim}
          ACTIVE
          3 TO 22 END
          CORE
          1 2
          END
\end{verbatim}
}
The following sequence would be used to simply freeze the orbitals
while retaining the complete virtual manifold:

{
\footnotesize
\begin{verbatim}
          ACTIVE
          3 TO 24 END
          CORE 
          1 2
          END
\end{verbatim}
}
Note that if no orbitals are to be discarded, as in the example
above, the user may omit the ACTIVE directive, with the CORE specification
acting to define this set. Thus the above data sequence is equivalent
to merely presenting the sequence:

{
\footnotesize
\begin{verbatim}
          CORE 
          1 2
          END
\end{verbatim}
}

{\bf Example 2}\\

In this example we wish to perform a valence-CI calculation on the
\nitrog\ molecule using a TZVP basis. While the molecular symmetry is
\dinfh, the symmetry adaptation  and subsequent CI will be conducted in
the \dtwoh\ point group.  The resolution of the \dinfh\ into the
\dtwoh\ orbital species has been given in Table~\ref{table:2}. An
examination of the SCF output reveals the following orbital analysis.

{
\footnotesize
\begin{verbatim}
          =============================
          IRREP  NO. OF SYMMETRY ADAPTED
                 BASIS FUNCTIONS
          =============================
            1          11
            2           4
            3           4
            4           1
            5          11
            6           4
            7           4
            8           1
          =============================
\end{verbatim}
}
and the following orbital assignments from the converged closed shell SCF:
{
\footnotesize
\begin{verbatim}
          ===============================================
          M.O.  IRREP  ORBITAL ENERGY   ORBITAL OCCUPANCY
          ===============================================
            1      1    -15.66716423           2.0000000
            2      5    -15.66241865           2.0000000
            3      1     -1.51005217           2.0000000
            4      5     -0.76128176           2.0000000
            5      1     -0.63704931           2.0000000
            6      3     -0.63448705           2.0000000
            7      2     -0.63448705           2.0000000
            8      6      0.17408343           0.0000000
            9      7      0.17408343           0.0000000
           10      5      0.30302673           0.0000000
           11      3      0.38747796           0.0000000
           12      2      0.38747796           0.0000000
           13      1      0.42599317           0.0000000
           14      1      0.49515007           0.0000000
           15      7      0.57046706           0.0000000
           16      6      0.57046706           0.0000000
           17      5      0.92638361           0.0000000
           18      5      1.12927523           0.0000000
           19      1      1.83974927           0.0000000
           20      3      2.01160646           0.0000000
           21      2      2.01160646           0.0000000
           22      5      2.01683641           0.0000000
           23      4      2.08974396           0.0000000
           24      1      2.08974396           0.0000000
           25      7      2.16718057           0.0000000
           26      6      2.16718057           0.0000000
           27      1      2.16945852           0.0000000
           28      3      2.20023738           0.0000000
           29      2      2.20023738           0.0000000
           30      8      2.67650248           0.0000000
           31      5      2.67650248           0.0000000
           32      5      2.98166352           0.0000000
           33      1      3.57616884           0.0000000
           34      7      3.58056531           0.0000000
           35      6      3.58056531           0.0000000
           36      5      4.37707106           0.0000000
           37      1      6.02043977           0.0000000
           38      5      6.12816913           0.0000000
           39      1     35.89673292           0.0000000
           40      5     35.91926868           0.0000000
          ===============================================
\end{verbatim}
}
Thus the inner-shell N1s orbitals, the 1$\sigma_{g}$ and
1$\sigma_{u}$ transform as  a$_{g}$ and b$_{1u}$ respectively
in \dtwoh\ symmetry, with IRrep numbers 1 and 5. Both correspond
of course to the first orbital in the appropriate IRrep.
The following sequence would be used to simply freeze the orbitals
while retaining the complete virtual manifold:

{
\footnotesize
\begin{verbatim}
          CORE
          1 2 END 
\end{verbatim}
}
Turning to the inner-shell complement orbitals, we again find
the corresponding IRreps, 1 and 5, with the orbitals the highest
lying member in each case, with relative sequence number of 11.
Thus the following ACTIVE and CORE  data would
act to both freeze and discard the inner-shells and their
complement MOs:

{
\footnotesize
\begin{verbatim}
          ACTIVE
          3 TO 38
          END
          CORE
          1 2 END
\end{verbatim}
}
Further examples of transformation data will be discussed below in
the section discussing Reference Function specification within the
SELECT data.

\section[Data for Semi-direct Table-CI Data-base Generation]{Data for Semi-direct Table-CI Data-base Generation}

The `data-base' of pattern symbolic matrix elements required by both
the Selection and CI modules may be generated by the user in the course
of any Table-CI calculation. In contrast to the Conventional module,
we envisage that this step will be executed in each run of the module
rather than the user allocating a previously generated version.

\subsection[TABLE]{TABLE}
The TABLE directive is used to request and control the data-base
generator, and comprises a single data line
read to the variables TEXT, TEXTF, TEXTB  using format (3A).
\begin{itemize}
\item TEXT should be set to the character string TABLE.
\item TEXTF is an optional parameter that may be used to
control the quantity of printed output produced by the 
module. Valid settings include the strings,
\begin{itemize}
\item NOPRINT, to suppress output from the module;
\item IPRINT, to produce an intermediate level of output;
\item FPRINT or DEBUG, to produce output suitable for debugging purposes.
\end{itemize}
\item TEXTB is a further optional parameter that should be set to the
string BYPASS if the user wishes to bypass generation of the Table-CI
data base. Such usage assumes that the data set "table-ci" is resident
in the directory in which the calculation is preceding, having been
generated there is some previous run of the direct-CI module.  Such
usage is typically associated with restarting Table-CI calculations.
\end{itemize}

\section[Data for Semi-direct Table-CI Selection]{Data for Semi-direct Table-CI Selection}

Data for the configuration selection module is initiated with
the SELECT directive, followed by those directives characterising
the symmetry of the state(s) of interest and reference configurations
(CNTRL, SPIN, SYMMETRY, CONF etc.) and terminated by data (ROOTS,
THRESH) controlling the process of selection.

\subsection[SELECT]{SELECT}
The SELECT directive is used to control the configuration
selection module, and comprises a single data line
read to the variables TEXT, TEXTF and TEXTB using format (3A).
\begin{itemize}
\item TEXT should be set to the character string SELECT.
\item TEXTF is an optional parameter that may be used to
control the quantity of printed output produced by the 
module. Valid settings include the strings,
\begin{itemize}
\item NOPRINT, to suppress the major part of the
output from the module, in particular all details
of the perturbative energy lowerings associated with 
the initial set of configurations;
\item IPRINT, to produce an intermediate level of output;
\item FPRINT or DEBUG, to produce output suitable for debugging purposes.
This includes the energy lowerings associated with the complete
configuration list.
\end{itemize}
\item TEXTB is an optional parameter that should be set
to the string BYPASS if the user wishes to bypass 
SELECT processing. Such usage is typically
associated with restarting Table-CI calculations.
\end{itemize}

\subsection[CNTRL]{CNTRL}
This directive consists of one line read to variables TEXT, NELEC
using format (A,I).
\begin{itemize}
\item  TEXT should be set to the character string CNTRL or NELEC.
\item  NELEC is used to specify the total number of `active'
electrons in the CI calculation. Note that any inner shell
electrons frozen out under control of the CORE directive
should not be included.
\end{itemize}

\subsection[SPIN]{SPIN}
This directive consists of one line read to variables TEXT, NSPIN
using format (A,I).
\begin{itemize}
\item  TEXT should be set to the character string SPIN.
\item  NSPIN is used to specify the spin degeneracy 
of the CI wavefunction of the electronic eigenstate(s) of interest,
using the values 1,2,3 etc. for singlet, doublet, triplet states etc.
respectively. It is also possible to use one of the character strings
SINGLET, DOUBLET, TRIPLET, QUARTET and QUINTET to specify NSPIN.
\end{itemize}

{\bf Example}
{
\footnotesize
\begin{verbatim}
          SPIN 3

          SPIN TRIPLET
\end{verbatim}
}
are equivalent; the wavefunction will be three-fold spin degenerate.

\subsection[SYMMETRY]{SYMMETRY}
This directive consists of one line read to variables TEXT, NSYM
using format (A,I).
\begin{itemize}
\item  TEXT should be set to the character string SYMMETRY.
\item  NSYM is an integer parameter used to specify the 
spatial symmetry of the CI wavefunction,
and is set to the appropriate sequence number of the 
required irreducible representation (see Table~\ref{table:1}).
\end{itemize}

{\bf Example}\\

In a system of C$_{2v}$ symmetry, the data line
{
\footnotesize
\begin{verbatim}
          SYMMETRY 3
\end{verbatim}
}
would be required when performing calculations on states of
B$_{2}$ symmetry. Failure to present the directive in such
cases will lead to the default A$_{1}$ symmetry.

\subsection[SINGLES]{SINGLES}
This directive consists of one line read to variables TEXT, TEXTS
using format 2A.
\begin{itemize}
\item  TEXT should be set to the character string SINGLES.
\item  TEXTS may be set to one of the following character strings:
\begin{itemize}
\item  ALL; The selection module will retain in the final CI all
single excitations with respect to ALL nominated reference functions
(specified under the CONF directive) {\em regardless} of their computed
energy lowerings.
\item  OFF; The selection module will use the computed energy lowerings
as the sole criteria for including configurations in the final CI, with
no automatic inclusion of single excitations.
\end{itemize}
\end{itemize}
Note that this usage differs from that described in the Conventional
module.  An alternative form of the SINGLES directive is also
possible, comprising a single data line read to variables TEXT, NREF
using format (A,I).
\begin{itemize}
\item  TEXT should be set to the character string SINGLES.
\item  NREF is an integer parameter used to nominate
a particular configuration within the set of
reference functions. The selection module will then retain in the
final CI all single excitations with respect to the nominated function
{\em regardless} of their computed energy lowerings.
\end{itemize}

The SINGLES directive may be omitted, when the program will include all
single excitations with respect to ALL nominated reference functions
(i.e. SINGLES ALL).\\

{\bf Example}\\

Presenting the data line
{
\footnotesize
\begin{verbatim}
          SINGLES 1
\end{verbatim}
}
in a Table-CI calculation of a closed--shell system, where the
SCF configuration is the first in the CONF list, will lead
to the inclusion of all single excitations with respect to the
SCF function in the final CI. Such inclusion leads, of course, to
a marked improvement in the quality of one-electron properties
computed from the CI wavefunction.

\subsection[CONF]{CONF}
The CONF directive is used to specify the reference CSFs for the CI
expansion.  The first line of the CONF directive is set to the
character string CONF.  In contrast to CONF specification in the
conventional module, the last line of the directive, the directive
terminator, now consists of the text END in the first data field. Lines
between the initiator and terminator define the reference configuration
set; each line defines a reference CSF by specifying the sequence
numbers of the component active orbitals in I-format.  A given
reference CSF is defined by
\begin{enumerate}
\item the number of open--shell orbitals (NOPEN). NOPEN
includes any unpaired orbitals together with those
non-identical spin-coupled pairs open to substitution.
\item NOPEN integers specifying the sequence numbers of these orbitals
\item the (NELEC-NOPEN)/2 sequence numbers of 
the doubly-occupied orbitals i.e., the identically spin-coupled orbitals
\end{enumerate}
where the sequence--numbers refers to the symmetry ordered orbitals
performed at the outset of processing. Within the set of open-- and
doubly--occupied orbitals, the MOs are presented in groups of common
IRrep, with the groups presented in order of increasing IRrep sequence
number. A few examples below will help clarify this order of
presentation.  Note again that:
\begin{itemize}
\item all reference function nominated by CONF {\em must} be of the
same symmetry as that nominated on the SYMMETRY directive.
\item at least two reference functions must be specified; it is
not possible to conduct simple CISD calculations with the semi-direct
module.
\item note again the requirement for a directive terminator.
\end{itemize}


{\bf Example 1}\\

Consider performing a valence-CI calculation on the \phosphine\
molecule using a 6-31G(*)  basis. While the molecular symmetry is
C$_{3v}$, the symmetry adaptation  and subsequent CI will be conducted
in the C$_{s}$ point group. An examination of the SCF output reveals
the following orbital analysis.

{
\footnotesize
\begin{verbatim}
           =============================
           IRREP  NO. OF SYMMETRY ADAPTED
                  BASIS FUNCTIONS
           =============================
             1          18
             2           7
           =============================
\end{verbatim}
}
and the following orbital assignments characterising the closed--shell
SCF configuration:
\begin{equation}
  1a_{1}^{2}  2a_{1}^{2}  1e^{4}  3a_{1}^{2}  4a_{1}^{2}  2e^{4}  5a_{1}^{2}
\end{equation}
or, in the C$_{s}$ symmetry representation:
\begin{equation}
  1a'^{2}  2a'^{2}  1a''^{2}  3a'^{2}  4a'^{2}  5a'^{2}  6a'^{2} 2a''^{2} 7a'^{2}
\end{equation}
{
\footnotesize
\begin{verbatim}
          ===============================================
          M.O.  IRREP  ORBITAL ENERGY   ORBITAL OCCUPANCY
          ===============================================
            1     1    -79.93661395           2.0000000
            2     1     -7.48916431           2.0000000
            3     1     -5.38319410           2.0000000
            4     2     -5.38319405           2.0000000
            5     1     -5.38149104           2.0000000
            6     1     -0.85610769           2.0000000
            7     1     -0.52191424           2.0000000
            8     2     -0.52191424           2.0000000
            9     1     -0.38579686           2.0000000
           10     1      0.16819544           0.0000000
           11     2      0.16819544           0.0000000
           12     1      0.26587776           0.0000000
           13     1      0.46072690           0.0000000
           14     2      0.46072690           0.0000000
           15     1      0.47871033           0.0000000
           16     1      0.56106989           0.0000000
           17     1      0.89229884           0.0000000
           18     2      0.89229885           0.0000000
           19     2      0.91131383           0.0000000
           20     1      0.91131383           0.0000000
           21     1      0.93118300           0.0000000
           22     1      1.17900613           0.0000000
           23     2      1.45058658           0.0000000
           24     1      1.45058658           0.0000000
           25     1      3.78674557           0.0000000
          ===============================================
\end{verbatim}
}
Based on the above output, the CONF data lines may be
deduced from the following table, where we
assume that we wish to freeze the five inner shell orbitals:
\begin{equation}
  1a'^{2}  2a'^{2}  1a''^{2}  3a'^{2}  4a'^{2}  
\end{equation}

\begin{centering}
\begin{tabular}{llrrrr}

\\ \hline
IRrep & IRrep &  No. of Basis & Frozen & Active & Sequence  \\
      & No.  &   Functions    & MOs    & MOs    & Nos.       \\ \hline
 a$^{'}$  & 1  &   18          & 4      & 14     & 1-14       \\
 a$^{''}$ & 2  &   7           & 1      &  6     & 15-20   \\ \hline
\end{tabular}
 
\end{centering}
\vspace{.15in}
To perform an 8-electron valence-CI calculation,
involving the SCF configuration and  two  degenerate (1e)' to (2e)'
doubly-excited  configurations 
\begin{equation}
    5a'^{2}  8a'^{2} 2a''^{2} 7a'^{2}
\end{equation}
and
\begin{equation}
    5a'^{2}  6a'^{2} 3a''^{2} 7a'^{2}
\end{equation}
would require the following CONF data:
{
\footnotesize
\begin{verbatim}
          CONF
          0 1 2 3 15
          0 1 3 4 15
          0 1 2 3 16
          END
\end{verbatim}
}
The complete data file for performing the SCF and subsequent
semi-direct CI would then be as follows:
{
\footnotesize
\begin{verbatim}
          TITLE
          PH3 * 6-31G*  VALENCE-CI 3M/1R
          SUPER OFF NOSYM
          ZMAT 
          P
          H 1 RPH
          H 1 RPH 2 THETA
          H 1 RPH 2 THETA 3 THETA  1
          VARIABLES
          RPH 2.685   
          THETA 93.83  
          END
          BASIS 6-31G*
          RUNTYPE CI
          CORE
          1 TO 5 END
          MRDCI DIRECT
          TABLE 
          SELECT
          CNTRL 8
          SPIN 1
          SYMMETRY 1
          SINGLES ALL
          CONF
          0 1 2 3 15
          0 1 3 4 15
          0 1 2 3 16
          END
          CI
          NATORB
          ENTER
\end{verbatim}
}

{\bf Example 2}\\

In this example we wish to perform a valence-CI calculation on the
\cucl\ molecule using a 3-21G  basis. While the molecular symmetry is
\cinfv, the symmetry adaptation  and subsequent CI will be conducted in
the \ctwov\ point group.  The resolution of the \cinfv\ into the
\ctwov\ orbital species is given in Table~\ref{table:2}.  An
examination of the SCF output reveals the following orbital analysis.

{
\footnotesize
\begin{verbatim}
           =============================
           IRREP  NO. OF SYMMETRY ADAPTED
                  BASIS FUNCTIONS
           =============================
             1          22
             2           9
             3           9
             4           2
           =============================
\end{verbatim}
}
and the following orbital assignments from the converged closed shell SCF:
{
\footnotesize
\begin{verbatim}
          ===============================================
          M.O.  IRREP  ORBITAL ENERGY   ORBITAL OCCUPANCY
          ===============================================
            1     1   -326.84723972           2.0000000
            2     1   -104.02836336           2.0000000
            3     1    -40.71695637           2.0000000
            4     1    -35.46377378           2.0000000
            5     3    -35.45608069           2.0000000
            6     2    -35.45608068           2.0000000
            7     1    -10.42193940           2.0000000
            8     1     -7.88512031           2.0000000
            9     2     -7.88222844           2.0000000
           10     3     -7.88222844           2.0000000
           11     1     -5.07729175           2.0000000
           12     1     -3.38247056           2.0000000
           13     3     -3.35978308           2.0000000
           14     2     -3.35978307           2.0000000
           15     1     -1.01099628           2.0000000
           16     3     -0.53702948           2.0000000
           17     2     -0.53702947           2.0000000
           18     4     -0.49640067           2.0000000
           19     1     -0.49640067           2.0000000
           20     1     -0.44715317           2.0000000
           21     3     -0.39988537           2.0000000
           22     2     -0.39988537           2.0000000
           23     1     -0.35127248           2.0000000
           24     1      0.00023285           0.0000000
           25     3      0.06300102           0.0000000
           26     2      0.06300102           0.0000000
           27     1      0.12855448           0.0000000
           28     1      0.19287013           0.0000000
           29     3      0.25729975           0.0000000
           30     2      0.25729975           0.0000000
           31     1      0.39720201           0.0000000
           32     1      0.86197727           0.0000000
           33     2      0.88942618           0.0000000
           34     3      0.88942618           0.0000000
           35     1      1.01877167           0.0000000
           36     1      2.16694989           0.0000000
           37     3      3.96181512           0.0000000
           38     2      3.96181512           0.0000000
           39     4      3.98212497           0.0000000
           40     1      3.98212497           0.0000000
           41     1      4.08851360           0.0000000
           42     1     24.51368240           0.0000000
          ===============================================
\end{verbatim}
}
Based on the above output, the CONF data lines may be
deduced from the following table, where we
assume that we wish to freeze the first 14 inner shell orbitals:
\begin{equation}
 1\sigma^{2}  2\sigma^{2}   3\sigma^{2}  4\sigma^{2}  1\pi^{4}  5\sigma^{2} 6\sigma^{2}   2\pi^{4}  7\sigma^{2}  8\sigma^{2}  3\pi^{4}  
\end{equation}

\begin{centering}
\begin{tabular}{llrrrr}

\\ \hline
IRrep & IRrep &  No. of Basis & Frozen & Active & Sequence  \\
      & No.  &   Functions    & MOs    & MOs    & Nos.       \\ \hline
 a$_{1}$  & 1  &   22          & 8      & 14     & 1-14       \\
 b$_{1}$  & 2  &   9           & 3      &  6     & 15-20   \\ 
 b$_{2}$  & 3  &   9           & 3      &  6     & 21-26       \\
 a$_{2}$  & 4  &   2           & 0      &  2     & 27-28   \\ \hline
\end{tabular}
 
\end{centering}
\vspace{.15in}
To perform an 18-electron valence-CI calculation,
based on the SCF configuration 
\begin{equation}
 9\sigma^{2}  4\pi^{4}   1\delta^{4} 10\sigma^{2}  5\pi^{4}  11\sigma^{2} 
\end{equation}
and the doubly excited configuration
\begin{equation}
 9\sigma^{2}  4\pi^{4}   1\delta^{4} 10\sigma^{2}  5\pi^{4}  12\sigma^{2} 
\end{equation}
would require the following CONF data:
{
\footnotesize
\begin{verbatim}
          CONF
          0 1 2 3 4  15 16  21 22  27
          0 1 2 3 5  15 16  21 22  27
          END
\end{verbatim}
}
The complete data file for performing the
SCF and subsequent CI would then be as follows:
{
\footnotesize
\begin{verbatim}
          TITLE\CUCL .. 3-21G
          ZMAT ANGSTROM\CU\CL 1 CUCL\
          VARIABLES\CUCL 2.093 \END 
          BASIS 3-21G
          RUNTYPE CI
          CORE
          1 TO 14 END
          MRDCI DIRECT
          TABLE
          SELECT 
          CNTRL 18
          SPIN 1
          SYMM 1
          SINGLES 1
          CONF
          0 1 2 3 4  15 16  21 22  27
          0 1 2 3 5  15 16  21 22  27
          END
          CI
          NATORB
          ENTER
\end{verbatim}
}

{\bf Example 3}\\

Consider performing a valence-CI calculation on the \silane\ molecule
using a 6-31G(*)  basis. While the molecular symmetry is T$_{d}$, the
symmetry adaptation  and subsequent CI will be conducted in the
C$_{2v}$ point group. An examination of the SCF output reveals the
following orbital analysis.

{
\footnotesize
\begin{verbatim}
           =============================
           IRREP  NO. OF SYMMETRY ADAPTED
                  BASIS FUNCTIONS
           =============================
             1           9
             2           6
             3           6
             4           6
           =============================
\end{verbatim}
}
and the following orbital assignments from
the converged closed shell SCF:
{
\footnotesize
\begin{verbatim}
          ===============================================
          M.O.  IRREP  ORBITAL ENERGY   ORBITAL OCCUPANCY
          ===============================================
            1     1    -68.77130710           2.0000000
            2     1     -6.12943325           2.0000000
            3     2     -4.23503117           2.0000000
            4     3     -4.23503117           2.0000000
            5     4     -4.23503117           2.0000000
            6     1     -0.73046864           2.0000000
            7     4     -0.48480821           2.0000000
            8     3     -0.48480821           2.0000000
            9     2     -0.48480821           2.0000000
           10     2      0.16291387           0.0000000
           11     3      0.16291387           0.0000000
           12     4      0.16291387           0.0000000
           13     1      0.25681257           0.0000000
           14     1      0.33606346           0.0000000
           15     3      0.37087856           0.0000000
           16     2      0.37087856           0.0000000
           17     4      0.37087856           0.0000000
           18     1      0.79946861           0.0000000
           19     1      0.79946861           0.0000000
           20     4      0.86232544           0.0000000
           21     3      0.86232544           0.0000000
           22     2      0.86232544           0.0000000
           23     1      1.23833149           0.0000000
           24     4      1.44033091           0.0000000
           25     3      1.44033091           0.0000000
           26     2      1.44033091           0.0000000
           27     1      3.13181655           0.0000000
          ===============================================
\end{verbatim}
}
Based on the above output, the CONF data lines may be
deduced from the following table, where we
assume that we wish to freeze the first 5 silicon inner shell orbitals:

\begin{centering}
\begin{tabular}{llrrrr}

\\ \hline
IRrep & IRrep &  No. of Basis & Frozen & Active & Sequence  \\
      & No.  &   Functions    & MOs    & MOs    & Nos.       \\ \hline
 a$_{1}$  & 1  &   9          & 2      &  7     & 1-7       \\
 b$_{1}$  & 2  &   6          & 1      &  5     & 8-12   \\ 
 b$_{2}$  & 3  &   6          & 1      &  5     & 13-17       \\
 a$_{2}$  & 4  &   6          & 1      &  5     & 18-22   \\ \hline
\end{tabular}
 
\end{centering}
\vspace{.15in}
To perform a 4-reference, 8-electron valence-CI calculation,
based on the SCF configuration and configurations arising
from the 2t$_2$ to 3t$_2$ would require the following CONF data:
{
\footnotesize
\begin{verbatim}
          CONF
          0 1 8 13 18
          0 1 8 13 19
          0 1 8 14 18
          0 1 9 13 18
          END
\end{verbatim}
}
The complete data file for performing the SCF and subsequent CI would
then be as follows (note that we are retaining all single excitations
with respect to each reference function in the final CI):
{
\footnotesize
\begin{verbatim}
          TITLE
          SIH4 * 6-31G* DIRECT-TABLE-CI VALENCE-CI 4M/1R
          ZMAT 
          SI
          H 1 SIH
          H 1 SIH 2 109.471
          H 1 SIH 2 109.471 3 120.0
          H 1 SIH 2 109.471 4 120.0
          VARIABLES
          SIH 2.80   
          END
          BASIS 6-31G*
          RUNTYPE CI
          CORE
          1 TO 5 END
          MRDCI DIRECT
          TABLE
          SELECT
          CNTRL 8
          SYMM 1
          SPIN 1
          CONF
          0 1 8 13 18
          0 1 8 13 19
          0 1 8 14 18
          0 1 9 13 18
          END
          SINGLES ALL
          CI
          NATORB
          ENTER
\end{verbatim}
}

{\bf Example 4}\\

In this example we wish to perform a valence-CI calculation on the
\nitrog\ molecule using a 4-31G(*)  basis. While the molecular symmetry
is \dinfh, the symmetry adaptation  and subsequent CI will be conducted
in the \dtwoh\ point group.  The resolution of the \dinfh\ into the
\dtwoh\ orbital species is given in Table~\ref{table:2}.  An
examination of the SCF output reveals the following orbital analysis.

{
\footnotesize
\begin{verbatim}
           =============================
           IRREP  NO. OF SYMMETRY ADAPTED
                  BASIS FUNCTIONS
           =============================
             1           8
             2           3
             3           3
             4           1
             5           8
             6           3
             7           3
             8           1
           =============================
\end{verbatim}
}
and the following orbital assignments from
the converged closed shell SCF:
{
\footnotesize
\begin{verbatim}
          ===============================================
          M.O.  IRREP  ORBITAL ENERGY   ORBITAL OCCUPANCY
          ===============================================
            1      1    -15.65953319           2.0000000
            2      5    -15.65476539           2.0000000
            3      1     -1.50616872           2.0000000
            4      5     -0.75782843           2.0000000
            5      1     -0.63245667           2.0000000
            6      2     -0.63136574           2.0000000
            7      3     -0.63136574           2.0000000
            8      7      0.20154120           0.0000000
            9      6      0.20154120           0.0000000
           10      5      0.63882720           0.0000000
           11      1      0.82490877           0.0000000
           12      2      0.89633728           0.0000000
           13      3      0.89633728           0.0000000
           14      1      0.91811776           0.0000000
           15      6      1.10035435           0.0000000
           16      7      1.10035436           0.0000000
           17      5      1.17624961           0.0000000
           18      5      1.66993831           0.0000000
           19      4      1.70516525           0.0000000
           20      1      1.70516525           0.0000000
           21      2      1.91000364           0.0000000
           22      3      1.91000364           0.0000000
           23      8      2.29434948           0.0000000
           24      5      2.29434948           0.0000000
           25      1      2.84353563           0.0000000
           26      6      3.00847612           0.0000000
           27      7      3.00847612           0.0000000
           28      5      3.37444027           0.0000000
           29      1      3.71749475           0.0000000
           30      5      4.09916047           0.0000000
          ===============================================
\end{verbatim}
}
Based on the above output, the CONF data lines may be
deduced from the following table, where we
assume that we wish to freeze the two N1s inner shell orbitals:

\begin{centering}
\begin{tabular}{llrrrr}

\\ \hline
IRrep & IRrep &  No. of Basis & Frozen & Active & Sequence  \\
      & No.  &   Functions    & MOs    & MOs    & Nos.       \\ \hline
 $\sigma_{g}$    & 1  &   8        & 1      &  7     & 1-7       \\
 $\pi_{u,x}$     & 2  &   3        & 0      &  3     & 8-10   \\ 
 $\pi_{u,y}$     & 3  &   3        & 0      &  3     & 11-13       \\
 $\delta_{g,xy}$ & 4  &   1        & 0      &  1     & 14   \\
 $\sigma_{u}$    & 5  &   8        & 1      &  7     & 15-21   \\
 $\pi_{g,x}$     & 6  &   3        & 0      &  3     & 22-24       \\
 $\pi_{g,y}$     & 7  &   3        & 0      &  3     & 25-27   \\ 
 $\delta_{u,xy}$ & 8  &   1        & 0      &  1     & 28   \\ \hline
\end{tabular}
 
\end{centering}
\vspace{.15in}
To perform a 10-electron valence-CI calculation,
based on the SCF configuration  
\begin{equation}
 2\sigma_g^{2}  2\sigma_u^{2}  3\sigma_g^2  1\pi_u^{4}
\end{equation}
and associated $\pi$ to $\pi^{*}$ excitations
\begin{equation}
 2\sigma_g^{2}  2\sigma_u^{2}  3\sigma_g^2  1\pi_{u,y}^2 1\pi_{g,x}^2
\end{equation}
\begin{equation}
 2\sigma_g^{2}  2\sigma_u^{2}  3\sigma_g^2  1\pi_{u,x}^2 1\pi_{g,y}^2
\end{equation}
\begin{equation}
 2\sigma_g^{2}  2\sigma_u^{2}  3\sigma_g^2  (1\pi_{u,x} 1\pi_{g,x}) (1\pi_{u,y} 1\pi_{g,y})
\end{equation}
would require the following CONF data:
{
\footnotesize
\begin{verbatim}
          CONF
          0 1 2 8 11 15
          0 1 2 11 15 22
          0 1 2 8 15 25
          4 8 11 22 25  1 2 15
          END
\end{verbatim}
}
The complete data file for performing the
SCF and subsequent CI would then be as follows:
{
\footnotesize
\begin{verbatim}
          TITLE\N2 .. 4-31G*
          SUPER OFF NOSYM
          ZMAT ANGS\N\N 1 NN
          VARIABLES\NN 1.05 \END
          BASIS 4-31G*
          RUNTYPE CI
          CORE
          1 2 END
          MRDCI DIRECT
          TABLE
          SELECT 
          CNTRL 10
          SYMM 1
          SPIN 1
          SINGLES 1
          CONF
          0 1 2 8 11 15
          0 1 2 11 15 22
          0 1 2 8 15 25
          4 8 11 22 25  1 2 15
          END
          CI
          NATORB IPRIN
          ENTER
\end{verbatim}
}
Now consider the corresponding calculation performed in a larger
CC-PVTZ basis.  An examination of the SCF output reveals the following
orbital analysis.

{
\footnotesize
\begin{verbatim}
           =============================
           IRREP  NO. OF SYMMETRY ADAPTED
                  BASIS FUNCTIONS
           =============================
             1          16
             2           8
             3           8
             4           3
             5          16
             6           8
             7           8
             8           3
           =============================
\end{verbatim}
}
and the following orbital assignments from
the converged closed shell SCF:
{
\footnotesize
\begin{verbatim}
          ===============================================
          M.O.  IRREP  ORBITAL ENERGY   ORBITAL OCCUPANCY
          ===============================================
             1     1    -15.66553669           2.0000000
             2     5    -15.66076870           2.0000000
             3     1     -1.50580210           2.0000000
             4     5     -0.76246663           2.0000000
             5     1     -0.63737662           2.0000000
             6     2     -0.63425262           2.0000000
             7     3     -0.63425262           2.0000000
             8     7      0.18426391           0.0000000
             9     6      0.18426391           0.0000000
            10     1      0.40582614           0.0000000
            11     5      0.42020784           0.0000000
            12     2      0.51942385           0.0000000
            13     3      0.51942385           0.0000000
            14     1      0.53416038           0.0000000
            15     6      0.71767715           0.0000000
            16     7      0.71767715           0.0000000
            17     5      0.74113094           0.0000000
            18     4      1.04543157           0.0000000
            19     1      1.04543157           0.0000000
            20     5      1.15104310           0.0000000
            21     2      1.37114036           0.0000000
            22     3      1.37114036           0.0000000
            23     8      1.53065438           0.0000000
            24     5      1.53065438           0.0000000
            25     1      1.58044223           0.0000000
            26     5      1.89387038           0.0000000
            27     6      1.93835376           0.0000000
            28     7      1.93835376           0.0000000
            29     1      2.02899399           0.0000000
            30     5      2.38397418           0.0000000
            31     2      2.58126398           0.0000000
            32     3      2.58126398           0.0000000
            33     1      3.03473543           0.0000000
            34     6      3.21032541           0.0000000
            35     7      3.21032541           0.0000000
            36     4      3.84680067           0.0000000
            37     1      3.84680067           0.0000000
            38     3      3.93355912           0.0000000
            39     2      3.93355912           0.0000000
            40     2      4.03730269           0.0000000
            41     3      4.03730269           0.0000000
            42     7      4.48707133           0.0000000
            43     6      4.48707133           0.0000000
            44     1      4.52955492           0.0000000
            45     5      4.56413770           0.0000000
            46     6      4.76466065           0.0000000
            47     7      4.76466065           0.0000000
            48     4      4.78192405           0.0000000
            49     1      4.78192405           0.0000000
            50     8      4.91469244           0.0000000
            51     5      4.91469244           0.0000000
            52     5      5.11932527           0.0000000
            53     8      5.42107737           0.0000000
            54     5      5.42107737           0.0000000
            55     2      5.42155085           0.0000000
            56     3      5.42155085           0.0000000
            57     2      5.53491607           0.0000000
            58     3      5.53491607           0.0000000
            59     1      5.70188467           0.0000000
            60     6      6.24854532           0.0000000
            61     7      6.24854532           0.0000000
            62     5      6.31686796           0.0000000
            63     6      7.07828789           0.0000000
            64     7      7.07828789           0.0000000
            65     1      7.42407537           0.0000000
            66     1      8.86035408           0.0000000
            67     5     11.17202050           0.0000000
            68     1     11.65153449           0.0000000
            69     5     12.12852685           0.0000000
            70     5     25.31135299           0.0000000
          ===============================================
\end{verbatim}
}
Based on the above output, the CONF data lines may be
deduced from the following table, where we again
assume that we wish to freeze the two N1s inner shell orbitals:

\begin{centering}
\begin{tabular}{llrrrr}

\\ \hline
IRrep          & IRrep &  No. of Basis & Frozen & Active & Sequence  \\
               & No.   &  Functions    & MOs    & MOs    & Nos.       \\ \hline
 $\sigma_{g}$  & 1  &       16        & 1      &  15     & 1-15       \\
 $\pi_{u,x}$   & 2  &        8        & 0      &  8      & 16-23   \\ 
 $\pi_{u,y}$   & 3  &        8        & 0      &  8      & 24-31       \\
 $\delta_{g,xy}$ & 4  &      3        & 0      &  3      & 32-34   \\
 $\sigma_{u}$  & 5  &       16        & 1      &  15     & 35-49   \\
 $\pi_{g,x}$   & 6  &        8        & 0      &  8      & 50-57       \\
 $\pi_{g,y}$   & 7  &        8        & 0      &  8      & 58-65   \\ 
 $\delta_{u,xy}$ & 8  &      3        & 0      &  3      & 66-68   \\ \hline
\end{tabular}
 
\end{centering}
\vspace{.15in}
To perform a 10-electron valence-CI calculation,
based on the SCF configuration
\begin{equation}
 2\sigma_g^{2}  2\sigma_u^{2}  3\sigma_g^2  1\pi_u^{4}
\end{equation}
and associated $\pi$ to $\pi^{*}$ excitations
\begin{equation}
 2\sigma_g^{2}  2\sigma_u^{2}  3\sigma_g^2  1\pi_{u,y}^2 1\pi_{g,x}^2
\end{equation}
\begin{equation}
 2\sigma_g^{2}  2\sigma_u^{2}  3\sigma_g^2  1\pi_{u,x}^2 1\pi_{g,y}^2
\end{equation}
\begin{equation}
 2\sigma_g^{2}  2\sigma_u^{2}  3\sigma_g^2  (1\pi_{u,x} 1\pi_{g,x}) (1\pi_{u,y} 1\pi_{g,y})
\end{equation}
based on the SCF configuration would require the following CONF data:

{
\footnotesize
\begin{verbatim}
          CONF
          0 1 2 16 24 35
          0 1 2 24 35 50
          0 1 2 16 35 38
          4 16 24 50 58 1 2 35
          END
\end{verbatim}
}
The complete data file for performing the
SCF and subsequent CI would then be as follows:
{
\footnotesize
\begin{verbatim}
          TITLE\N2 .. CC-PVTZ
          SUPER OFF NOSYM
          ZMAT ANGS\N\N 1 NN
          VARIABLES\NN 1.05 \END
          BASIS CC-PVTZ
          RUNTYPE CI
          CORE
          1 2 END
          MRDCI DIRECT
          TABLE
          SELECT 
          CNTRL 10
          SYMM 1
          SPIN 1
          SINGLES 1
          CONF
          0 1 2 16 24 35
          0 1 2 24 35 50
          0 1 2 16 35 58
          4 16 24 50 58 1 2 35
          END
          CI
          NATORB IPRIN
          ENTER
\end{verbatim}
}

{\bf Example 5}\\

In this example we wish to perform a valence-CI calculation on the
\cah\ molecule using a 6-31G**  basis. While the molecular symmetry is
D$_{\infty h}$, the symmetry adaptation  and subsequent CI will be
conducted in the D$_{2h}$ point group. An examination of the SCF output
reveals the following orbital analysis.

{
\footnotesize
\begin{verbatim}
           =============================
           IRREP  NO. OF SYMMETRY ADAPTED
                  BASIS FUNCTIONS
           =============================
             1          11
             2           5
             3           5
             4           1
             5           7
             6           2
             7           2
           =============================
\end{verbatim}
}
and the following orbital assignments from the converged closed shell SCF:
{
\footnotesize
\begin{verbatim}
          ===============================================
          M.O.  IRREP  ORBITAL ENERGY   ORBITAL OCCUPANCY
          ===============================================
             1     1   -149.34279776           2.0000000
             2     1    -16.81835576           2.0000000
             3     2    -13.62701326           2.0000000
             4     3    -13.62701326           2.0000000
             5     5    -13.62532936           2.0000000
             6     1     -2.24495780           2.0000000
             7     3     -1.34937153           2.0000000
             8     2     -1.34937153           2.0000000
             9     5     -1.33498314           2.0000000
            10     1     -0.35141213           2.0000000
            11     5     -0.31217469           2.0000000
            12     3      0.02646555           0.0000000
            13     2      0.02646555           0.0000000
            14     1      0.04279408           0.0000000
            15     5      0.11107022           0.0000000
            16     1      0.11631297           0.0000000
            17     2      0.20535510           0.0000000
            18     3      0.20535510           0.0000000
            19     4      0.30735852           0.0000000
            20     1      0.30735852           0.0000000
            21     7      0.32858094           0.0000000
            22     6      0.32858094           0.0000000
            23     5      0.36615374           0.0000000
            24     1      0.47836712           0.0000000
            25     1      0.49656774           0.0000000
            26     1      1.18797119           0.0000000
            27     5      1.30595420           0.0000000
            28     3      2.42553536           0.0000000
            29     2      2.42553536           0.0000000
            30     7      2.45164750           0.0000000
            31     6      2.45164750           0.0000000
            32     5      2.50399980           0.0000000
            33     1      2.74252653           0.0000000
          ===============================================
\end{verbatim}
}
Based on the above output, the CONF data lines may be
deduced from the following table, where we
assume that we wish to freeze the nine Ca inner shell orbitals:

\begin{centering}
\begin{tabular}{llrrrr}

\\ \hline
IRrep & IRrep &  No. of Basis & Frozen & Active & Sequence  \\
      & No.  &   Functions    & MOs    & MOs    & Nos.       \\ \hline
 $\sigma_{g}$    & 1  &  11        & 3     &  8     & 1-8       \\
 $\pi_{u,x}$     & 2  &   5        & 2     &  3     & 9-11   \\ 
 $\pi_{u,y}$     & 3  &   5        & 2     &  3     & 12-14       \\
 $\delta_{g,xy}$ & 4  &   1        & 0     &  1     & 15   \\
 $\sigma_{u}$    & 5  &   7        & 2     &  5     & 16-20   \\
 $\pi_{g,x}$     & 6  &   2        & 0     &  2     & 21-22       \\
 $\pi_{g,y}$     & 7  &   2        & 0     &  2     & 23-24   \\  \hline
\end{tabular}
 
\end{centering}
\vspace{.15in}
To perform an 4-electron valence-CI calculation, based on the SCF configuration 
\begin{equation}
 4\sigma_g^{2}  3\sigma_u^{2}  
\end{equation}
and associated $\sigma$ to $\sigma^{*}$ excitations,
\begin{equation}
 5\sigma_g^{2}  3\sigma_u^{2} 
\end{equation}
\begin{equation}
 4\sigma_g^{2}  4\sigma_u^{2} 
\end{equation}
would require the following CONF data:

{
\footnotesize
\begin{verbatim}
          CONF
          0 1 16
          0 2 16
          0 1 17
          END
\end{verbatim}
}
The complete data file for performing the SCF and subsequent CI (in which
all singles are retained) would then be as follows:
{
\footnotesize
\begin{verbatim}
          TITLE\CAH2 .. 6-31G** 3M/1R
          SUPER OFF NOSYM
          ZMAT ANGS\CA\X 1 1.0\ H 1 CAH 2 90.0\H 1 CAH 2 90.0 3 THETA
          VARIABLES\CAH 2.148 \THETA 180.0 \END
          BASIS 6-31G**
          RUNTYPE CI
          CORE
          1 TO 9 END
          MRDCI DIRECT
          TABLE
          SELECT 
          CNTRL 4
          SPIN 1
          SYMMETRY 1
          SINGLES ALL
          CONF
          0 1 16
          0 2 16
          0 1 17
          END
          THRESH 2.0 2.0
          CI
          NATORB IPRIN
          ENTER
\end{verbatim}
}

\subsection[ROOTS]{ROOTS}

The ROOTS directive is used to specify those eigenvectors
of the `root' secular problem to be used in the process of
selection, with the energy contributions of the configurations
computed with respect to the nominated vectors. The directive
consists of a single data line with the character string ROOTS
in the first data field. Subsequent data comprises integer
variables used to specify the {\em number} of root eigenstates
(NROOT) and the {\em sequence numbers} of these vectors
within the matrix of zero-order eigenvectors,
(IROOT(I),I=1,NROOT).  Two formats may be used in this
specification:
\begin{enumerate}
\item If the lowest NROOT vectors are to be used, then the data
line is read to the variables TEXT, NROOT using format (A,I);
\begin{itemize}
\item TEXT is set to the character string ROOTS;
\item NROOT is an integer specifying the number of roots to
be used, where the sequence numbers of the roots will be
1--NROOT.
\end{itemize}
\item If the NROOT vectors to be used are not the lowest
in the root eigenvector matrix, then the sequence numbers
within this matrix must be specified. The data line is then
read to the variables TEXT, NROOT, (IROOT(I), I=1,NROOT), using
format (A, (NROOT+1)~I):
\begin{itemize}
\item TEXT and NROOT are defined as above;
\item NROOT integers are read to the array IROOT defining the
vectors of the zero-order matrix to be used in selection.
\end{itemize}
\end{enumerate}
We now provide some further notes on the directive:
\begin{itemize}
\item the ROOTS directive may be omitted, when the energy
contributions are calculated with reference to the lowest
eigenstate of the root problem only. Omission of the directive
is thus equivalent to presenting the data line

{
\footnotesize
\begin{verbatim}
            ROOTS 1
\end{verbatim}
}
\item The number of root eigenstates to be specified will
depend on the number of states required in the final CI. Thus if
NVEC roots of the final CI matrix are to be subsequently
generated in DIAG processing, the user should ideally perform
selection with respect to at least the corresponding NVEC 
roots of the root secular problem to ensure a consistent
treatment of each of the required states. The choice of
the reference set will clearly prove crucial and should be such
as to ensure a one to one correspondence between each of the
final CI vectors and a certain vector of the root problem. Indeed
the whole process of extrapolation to zero threshold is
meaningless if this condition is not obeyed.
\end{itemize}

\subsection[THRESH]{THRESH}
This directive defines the threshold factors to be used in the process
of configuration selection, and consists of a single line read to
variables TEXT, TMIN, TINC using format (A,2F).
\begin{itemize}
\item TEXT should be set to the character string THRESH.
\item  TMIN should be set to the minimum threshold factor 
(in units of micro-hartree, $\mu$H) to be used in selection. Any CSF
with a computed energy lowering greater than TMIN will be
retained in the final list of selected configurations.
\item TINC should be set to the threshold increment to be
used in the process of extrapolation. This process now involves
solution of the final secular problem at just two 
thresholds, defined by TMIN, and TMIN + TINC. Note the restricted
number of calculations here compared to the conventional module.
\end{itemize}
The THRESH directive may be omitted, when TMIN will be set to  10.0
and TINC to 10.0, leading to the solution 
of the T=10 and 20 $\mu$H secular problem.\\

{\bf Example}
{
\footnotesize
\begin{verbatim}
          THRESH 5.0 5.0

          THRESH 5 5
\end{verbatim}
}
are equivalent, causing T$_{min}$ and T$_{inc}$ to be set to 
5 microhartree.


\section[Data for Semi-direct Table-CI Eigen Solution]{Data for Semi-direct Table-CI Eigen Solution}

Data input controlling the semi-direct construction and diagonalisation
of the CI Hamiltonian is introduced by the CI directive. The process of
extrapolation to zero selection threshold involves the CI module
solving two secular problems, the first corresponding corresponding to
the selection threshold TMIN, the second to the threshold (TMIN+TINC).
In default, the module will generate NROOT eigenvectors of the CI
matrix on both passes, where NROOT is the number of roots specified by
the ROOTS directive at selection time.  Thus the solutions of the
zero-order Hamiltonian will be used through a maximum overlap criterion
in deriving the final CI eigenvectors. Additional data may be specified
to provide various convergence and printing controls.

\subsection[CI]{CI}

The CI directive  comprises a single data line
read to the variables TEXT, TEXTF and TEXTB using format (3A).
\begin{itemize}
\item TEXT should be set to the character string CI.
\item TEXTF is an optional parameter that may be used to
control the quantity of printed output produced by the 
module. Valid settings include the strings,
\begin{itemize}
\item NOPRINT, to suppress the major part of the
output from the module;
\item IPRINT, to produce an intermediate level of output;
\item FPRINT or DEBUG, to produce output suitable for debugging purposes.
\end{itemize}
\item TEXTB is an optional parameter that should be set
to the string BYPASS if the user wishes to bypass 
CI processing. Such usage is typically
associated with restarting Table-CI calculations.
\end{itemize}

\subsection[ACCURACY]{ACCURACY}

This directive may be used to define the diagonalisation 
thresholds for the two extrapolation passes,
consists of a single line read to variables
TEXT, THRESHE using format (A,F).
\begin{itemize}
\item TEXT should be set to the character string ACCURACY or DTHRESH;
\item THRESHE: On both extrapolation passes, 
the diagonalization is converged to an energy threshold THRESHE.
\end{itemize}
The THRESH directive may be omitted, when THRESHE will be set to
0.000001\\

{\bf Example}\\

Presenting the data line

{
\footnotesize
\begin{verbatim}
           ACCURACY 0.0000001
\end{verbatim}
}
will result in a diagonalisation threshold of 0.0000001 for the
two extrapolation passes.

\subsection[PRINT]{PRINT}

The PRINT directive may be used to control the printing of CI
coefficients and weights throughout the extrapolation passes
and in the final analysis.  This directive consists of a single data 
line read to variables
TEXT, PTHR, PTHRCC, IFLAG using format (A,2F,I).
\begin{itemize}
\item TEXT should be set to the character string PRINT.
\item  CI coefficients less than PTHR in absolute magnitude will not be
printed during the extrapolation passes.
\item  CI weights (coefficients$^{2}$) less than PTHRCC in absolute 
magnitude will not be printed in the final analysis of the CI
wavefunctions.
\item  IFLAG may be used to control the printing of the CI
wavefunctions in the event that the diagonalisation does not
converge. Setting IFLAG=1 will cause a detailed print of the
CI vectors corresponding to each  root.
\end{itemize}
 This directive may be omitted, when the defaults PTHR=0.05 and
PTHRCC=0.002 will be taken.

\section[Data for Semi-direct Table-CI Natural Orbitals]{Data for Semi-direct Table-CI Natural Orbitals}

\subsection[Natural Orbital Data - NATORB]{Natural Orbital Data - NATORB}

The NATORB directive is used to request Natural Orbital (NO)
generation, and comprises a single data line read to the variables
TEXT, TEXTF and TEXTB using format (3A).
\begin{itemize}
\item TEXT should be set to the character string NATORB.
\item TEXTF is an optional parameter that may be used to
control the quantity of printed output produced by the 
module. Valid settings include the strings,
\begin{itemize}
\item NOPRINT, to suppress the major part of the
output from the module;
\item IPRINT, to produce an intermediate level of output. This
option should be set to generate a print of the natural 
orbital coefficient array(s);
\item FPRINT or DEBUG, to produce output suitable for debugging purposes.
\end{itemize}
\item TEXTB is an optional parameter that should be set
to the string BYPASS if the user wishes to bypass 
NATORB processing. Such usage is typically
associated with restarting Table-CI calculations.
\end{itemize}
\subsection[Natural Orbital Data - CIVEC]{Natural Orbital Data - CIVEC}
The CIVEC directive is used to specify those eigenvectors of the
CI-matrix to be analysed. The directive consists of a single data line
with the character string CIVEC in the first data field. If natural
orbitals associated with NVEC eigenvectors of the secular problem are
to be generated, subsequent data fields should contain NVEC integers,
the integers specifying the numbering of the CI-eigenvectors on the
FORTRAN {\em interface}, FTN036, as generated by the CI sub-module.
If the CIVEC directive is omitted under NATORB processing the natural
orbitals of the first CI-vector will be generated.\\

{\bf Example}
{
\footnotesize
\begin{verbatim}
          CIVEC 1 3
\end{verbatim}
}
The above data line
may be used to generate natural orbitals from the first and
third CI-eigenvector generated by the CI sub-module.
\subsection[Natural Orbital Data - PUTQ]{Natural Orbital Data - PUTQ}
The PUTQ directive may be used to route spin-free natural orbitals to
the Dumpfile, and consists of a single dataline 
with the first two fields read to variables
TEXT, TYPE  using format (2A).
\begin{itemize}
\item  TEXT should be set to the character string PUTQ.
\item  TYPE should be set to one of the character strings AOS,
A.O.  or SABF,
defining the basis representation required for the output NOs.
The character string AOS and A.O. will yield the 
NOs in the basis function
representation, suitable for subsequent input to the other
analysis modules of GAMESS--UK. The string SABF will result in
the NO expansion in the symmetry adapted basis representation, 
and should be used when performing iterative natural orbital
calculations (see section 6.13 below)
\end{itemize}
The remaining data consists of a sequence of NVEC integers,
(between 0 and 350 inclusive) specifying the
section number of the Dumpfile where the spin-free NOs 
derived from the NVEC CI-vectors nominated by the
CIVEC directive are to be placed.\\

{\bf Example}
{
\footnotesize
\begin{verbatim}
          PUTQ AOS 100 120
\end{verbatim}
}
The spin-free NOs in the basis-set representation are output to
sections 100 and 120 respectively of the Dumpfile. A section setting of
0 on the PUTQ directive will act to suppress natural orbital output to
the Dumpfile.

\section[Data for Semi-direct Table-CI One-electron Properties]{Data for Semi-direct Table-CI One-electron Properties}

\subsection[PROP]{PROP}

The PROP directive is used to request the computation of one-electron
properties and comprises a single data line
read to the variables TEXT, TEXTF and TEXTB using format (3A).
\begin{itemize}
\item TEXT should be set to the character string PROP.
\item TEXTF is an optional parameter that may be used to
control the quantity of printed output produced by the 
module. Valid settings include the strings,
\begin{itemize}
\item NOPRINT, to suppress the major part of the
output from the module;
\item IPRINT, to produce an intermediate level of output;
\item FPRINT or DEBUG, to produce output suitable for debugging purposes.
\end{itemize}
\item TEXTB is an optional parameter that should be set
to the string BYPASS if the user wishes to bypass 
PROP processing. Such usage is typically
associated with restarting Table-CI calculations.
\end{itemize}

\subsection[CIVEC]{CIVEC}
The CIVEC directive is used to specify those eigenvectors
of the CI-matrix to be analysed. The directive
consists of a single data line with the character string CIVEC
in the first data field. If the properties associated with
NVEC eigenvectors of the secular problem are to be generated, 
subsequent data fields should contain
NVEC integers, the integers specifying
the numbering of the CI-eigenvectors on the FORTRAN {\em interface},
FTN036. If the CIVEC directive is omitted under PROP processing
an analysis of the first CI-vector will be performed.\\

{\bf Example}\\

The data line
{
\footnotesize
\begin{verbatim}
          CIVEC 1 3
\end{verbatim}
}
may be used to  analyse the  first and
third CI-eigenvector generated by the CI sub-module.


\subsection[AOPR]{AOPR}

The AOPR directive may be used to request printing of the
property integrals in the basis function (AO) representation.
If specified, the directive consists of a single data line
with the character string AOPR in the first data field. Subsequent
data fields are used to specify those integrals to be printed. Valid
character strings include S, T, X, Y, Z, XX, YY, ZZ, XY, XZ and   
YZ, requesting in obvious notation printing of the components
of the overlap, kinetic energy, dipole and quadrupole moments respectively.\\

{\bf Example}
{
\footnotesize
\begin{verbatim}
          AOPR X Y Z
\end{verbatim}
}
would result in printing of integrals of the x-, y- and z-components
of the dipole moment.

\subsection[MOPR]{MOPR}

The MOPR directive may be used to request printing of the
property integrals in the molecular orbital (MO) basis.
If specified, the directive consists of a single data line
with the character string MOPR in the first data field. Subsequent
data fields are used to specify those integrals to be printed. Valid
character strings include S, T, X, Y, Z, XX, YY, ZZ, XY, XZ and   
YZ, requesting in obvious notation printing of the components
of the overlap,
kinetic energy, dipole and quadrupole moments respectively.\\

{\bf Example}
{
\footnotesize
\begin{verbatim}
          MOPR XX YY ZZ
\end{verbatim}
}
would result in printing of integrals of the diagonal components
of the quadrupole moment.

\subsection[Configuration Data Lines]{Configuration Data Lines}

In addition to evaluating the properties of a given CI-vector, 
the module will also look to evaluating the corresponding properties
of a nominated single configuration, typically the leading
term in the CI-vector: the idea here of course is to
provide a guide to the effect of the CI treatment on the property,
with the nominated CSF being typically the corresponding SCF
configuration. Thus the final data for the properties module
comprises a sequence of NVEC data lines, each line
a sequence of integers defining the single configuration for the
CI-vector under consideration. The format of these lines is
identical to that of the CONF data used in nominating the
reference functions, and in most instances will be a repeat of
that data.\\

{\bf Example}\\

Consider the valence-CI calculation on \phosphine\ described in
example 1 of the CONF directive. Considering just the CI data,

{
\footnotesize
\begin{verbatim}
          MRDCI DIRECT
          TABLE 
          SELECT
          CNTRL 8
          SPIN 1
          SYMMETRY 1
          SINGLES 1
          CONF
          0 1 2 3 15
          0 1 3 4 15
          0 1 2 3 16
          END
          CI
          NATORB
\end{verbatim}
}
then the first data line of the CONF directive specifies the
SCF configuration, and it is this configuration that should be
nominated in the PROP data. Thus the CI data including the 
property analysis would appear as follows,

{
\footnotesize
\begin{verbatim}
          MRDCI DIRECT
          TABLE 
          SELECT
          CNTRL 8
          SPIN 1
          SYMMETRY 1
          SINGLES 1
          CONF
          0 1 2 3 15
          0 1 3 4 15
          0 1 2 3 16
          END
          CI
          NATORB
          PROP
          CIVEC 1
          0 1 2 3 15
\end{verbatim}
}
where one such data line is required given the specification
of CIVEC.

\section[Data for Semi-direct Table-CI Transition Moments]{Data for Semi-direct Table-CI Transition Moments}

This Table-CI module will calculate both electrical and magnetic dipole
moments as well as oscillator strengths and lifetimes of excited
states.  The module will look to calculate the moment between a
specific state (typically the ground state) and a set of additional
states (typically the excited states).  It is assumed that the  CI
eigen-vectors have been generated and are available on the appropriate
FORTRAN {\em interfaces}.  While in most cases the ground and excited
state CI-vectors will reside on the same {\em interface}, FTN036, the
module will allow the use of differing data sets for these vectors, a
situation most likely to occur when the ground and excited states are
of different symmetry.

\subsection[TM]{TM}

The TM directive is used to request Transition Moment analysis, and
comprises two data lines. The first line is read to the variables TEXT,
TEXTF and TEXTB using format (3A).
\begin{itemize}
\item TEXT should be set to the character string TM.
\item TEXTF is an optional parameter that may be used to
control the quantity of printed output produced by the 
module. Valid settings include the strings,
\begin{itemize}
\item NOPRINT, to suppress the major part of the
output from the module;
\item IPRINT, to produce an intermediate level of output;
\item FPRINT or DEBUG, to produce output suitable for debugging purposes.
\end{itemize}
\item TEXTB is an optional parameter that should be set
to the string BYPASS if the user wishes to bypass 
TM processing. Such usage is typically
associated with restarting Table-CI calculations.
\end{itemize}

The second data line of the TM directive is used
to specify the location of the CI-vectors, and the number
of excited state vectors involved in the subsequent analysis. The
line is read to the variables IFTNX, ISECX, IFTNE, ISECE, NSTATE
using format (5I);
\begin{itemize}    
\item IFTNX defines the FORTRAN data set reference number of the
{\em interface} holding the CI-vector of the first state.
Normally this vector will reside on FTN036, with IFTNX=36.

\item ISECX  defines the position of this first vector on the {\em
interface} defined by IFTNX. Typically, for the first state of a given
symmetry, the vector will be located first on the data set i.e.,
ISECX=1.

\item IFTNE defines the FORTRAN data set reference number of the
{\em interface} holding the CI-vector(s) of the  set of
additional states. Assuming these states are of the same
symmetry as the first, then we would expect all the states
involved to lie on the same {\em interface} i.e., IFTNE will
also be set to 36. If, however, the set of states is of
different symmetry to the first, then their vectors will almost
certainly reside on a different {\em interface}, which we will
assume reside on FTN037 i.e., IFTNE should be set to 37.

\item ISECE  defines the position of the first of the excited state
vectors  on the {\em interface} defined by IFTNE. Typically, if all the
states involved are of the same symmetry, residing on the same data set
(IFTNX=IFTNE), then the first excited state vector will be located
second on the  data set i.e., ISECE=2.  When the first and excited
states are of different symmetry, then different data sets will be
involved, and the first of the excited state vectors will be the first
on IFTNE.

\item NSTATE - defines the number of excited 
state vectors involved, and is usually equal to the 
number of transition moment calculations to be performed.
Thus if we wished to calculate the transition moment between the two
lowest states of \water, then NSTATE would equal 1, and the TM data
would appear as follows:

{
\footnotesize
\begin{verbatim}
          TM
          36 1 36 2 1
\end{verbatim}
}
\end{itemize}

\section[Semi-direct Table-CI - Using Default Options]{Semi-direct Table-CI - Using Default Options}

In order to simplify the process of configuration specification and
data preparation, the semi-direct module now provides a set of default
options that require little or no data input. While these defaults are
not expected to cover most in-depth requirements, they do provide a
starting point for users, and a route to subsequent, more extensive
calculations. To illustrate this default working of the module, we
consider below a number of example calculations.

\subsection[Calculations on the Formaldehyde Ground State]{Calculations on the Formaldehyde Ground State}
A Semi-direct Table-CI calculation is to performed on the formaldehyde
molecule in a TZVP basis.  Given the following data sequence:

{
\footnotesize
\begin{verbatim}
          TITLE
          H2CO - TZVP X1A1 DEFAULT TABLE-CI OPTIONS
          ZMAT ANGSTROM
          C
          O 1 1.203
          H 1 1.099 2 121.8
          H 1 1.099 2 121.8 3 180.0
          END
          BASIS TZVP
          RUNTYPE CI
          MRDCI DIRECT
          ENTER
\end{verbatim}
}
then the calculation undertaken will be based on the following;
\begin{enumerate}
\item Integral transformation will use the set of orbitals from section 1, 
the integer specified on the ENTER directive i.e. the closed shell
SCF orbitals.
\item The table-ci data base will be generated rather than restored
from a pre-existing {\em table-ci} data set.
\item The symmetry, spin and number of active electrons
will be taken from the corresponding SCF wavefunction. In the
present case this involves:
\begin{itemize}
\item A CI wavefunction of A$_{1}$ symmetry (i.e. SYMMETRY 1).
\item A singlet CI wavefunction (i.e. SPIN 1).
\item The number of active electrons in the CI will be set to be those
involved in the SCF calculation (i.e. CNTRL 16).
\end{itemize}
\item Singly excited configurations with respect to each of 
the default reference configurations (SINGLES ALL) will be included,
regardless of their computed energy lowerings.
\item The set of reference configurations to be employed will include
the SCF configuration, plus those generated from this configuration by
including (i) for each symmetry IRREP, the doubly excited configuration
arising from excitation of the highest occupied DOMO of that symmetry to
the lowest virtual orbital (VMO) of the same symmetry, and (ii) the lowest
singly excited configuration, again arising from the highest occupied
DOMO to the lowest VMO of the same symmetry.  In the present example,
this will correspond to the SCF configuration, the double and single
excitation arising from the DOMO 5a$_{1}$ to VMO 6a$_{1}$, the double
and single excitation arising from the DOMO 1b$_{1}$ to VMO 2b$_{1}$,
and the double and single excitation arising from the DOMO 2b$_{2}$ to
VMO 3b$_{2}$.  No reference configurations will be included involving
orbitals of a$_{2}$ symmetry given the absence of such orbitals involved
in the occupied manifold. This results in a total reference set of 7
functions, as shown thus in the job output:

{
\footnotesize
\begin{verbatim}
   numbers of open shells and corresponding main configurations

  0             1   2   3   4   5  28  37  38      ..   SCF configuration
  0             1   2   3   4   6  28  37  38      ..   5a1 -> 6a1 double
  2             5   6   1   2   3   4  28  37  38  ..   5a1 -> 6a1 single
  0             1   2   3   4   5  29  37  38      ..   1b1 -> 2b1 double
  2            28  29   1   2   3   4   5  37  38  ..   1b1 -> 2b1 single
  0             1   2   3   4   5  28  37  39      ..   2b2 -> 3b2 double
  2            38  39   1   2   3   4   5  28  37  ..   2b2 -> 3b2 single
\end{verbatim}
}
\item The default selection process subsequently undertaken is
equivalent to the following ROOTS and THRESH directives.

{
\footnotesize
\begin{verbatim}
          THRESH 10 10
          ROOTS 1
\end{verbatim}
}
Thus this default selection process involves construction of an explicit
zero-order Hamiltonian H$_{0}$ (over the reference functions described
above) followed by perturbative selection of configurations with
respect to the lowest root of H$_{0}$.  The minimum threshold to be
used in selection (T$_{min}$) is 10 micro-Hartree, with an increment of
10 uH to be used in defining the higher-threshold case to be solved in
the process of extrapolation  \cite{ref:18}.

\item In default the module will, having solved the secular problem
for the lowest root of the CI secular problem, generate the
spinfree natural orbitals from the associated CI eigenfunction. 
\end{enumerate}

The sequence of data lines defining the Semi-direct Table-CI
calculation is terminated by the ENTER directive.  Note at this stage
that the full data specification corresponding to the defaults
generated from the above data file is as follows:

{
\footnotesize
\begin{verbatim}
          TITLE
          H2CO - TZVP - EXPLICIT DATA FOR DEFAULT MRDCI SETTINGS
          SUPER OFF NOSYM
          ZMAT ANGSTROM
          C
          O 1 1.203
          H 1 1.099 2 121.8
          H 1 1.099 2 121.8 3 180.0
          END
          BASIS TZVP
          RUNTYPE CI
          ACTIVE
          1 TO 52 END
          MRDCI DIRECT
          TABLE
          SELECT
          CNTRL 16
          SPIN 1
          SYMM 1
          SINGLES ALL
          CONF
            0     1   2   3   4   5  28  37  38
            0     1   2   3   4   6  28  37  38
            2     5   6   1   2   3   4  28  37  38
            0     1   2   3   4   5  29  37  38
            2    28  29   1   2   3   4   5  37  38
            0     1   2   3   4   5  28  37  39
            2    38  39   1   2   3   4   5  28  37
          END
          THRESH 10 10
          ROOTS 1
          CI
          NATORB
          CIVEC 1
          ENTER
\end{verbatim}
}

\subsection[Calculations on the Formaldehyde Cations]{Calculations on the Formaldehyde Cations}
Let us now consider a Semi-direct Table-CI calculation  on the
\bstate\  state of \formion, again using default options available
within the module. A valid data sequence for performing such a
calculation is shown below, where we are still performing all the
computation in a single job.

{
\footnotesize
\begin{verbatim}
          TITLE
          H2CO+ 2B2 TZVP - DEFAULT MRDCI SETTINGS  -113.06446075
          MULT 2
          CHARGE 1
          ZMAT ANGSTROM
          C
          O 1 1.203
          H 1 1.099 2 121.8
          H 1 1.099 2 121.8 3 180.0
          END
          BASIS TZVP
          RUNTYPE CI
          MRDCI DIRECT
          ENTER
\end{verbatim}
}
Considering the changes to the closed-shell run, the following points
should be noted:
\begin{itemize}
\item The set of vectors used in the Table-CI transformation will be
restored from the default eigenvector section of the Dumpfile (given that
no section is specified on the ENTER directive). This will be section 5,
the section used to store the energy-ordered canonicalised orbitals from
the open-shell SCF calculation.
\item The symmetry and spin of the CI wavefunction will be deduced
from the preceding SCF calculation i.e. a CI wavefunction of B$_{2}$
symmetry (corresponding to SYMMETRY 3). and a doublet CI wavefunction
(corresponding to SPIN 2).
\item The number of active electrons in the CI will be set to be those
involved in the SCF calculation (i.e. CNTRL 15).
\item Singly excited configurations with respect to each of 
the default reference configurations (SINGLES ALL) will be included,
regardless of their computed energy lowerings.
\item The set of reference configurations to be employed will follow
the same algorithm used in the closed shell case above i.e.  the SCF
configuration, plus those generated from this configuration by including
(i) for each symmetry IRREP, the doubly excited configuration arising
from excitation of the highest occupied DOMO of that symmetry to the
lowest virtual orbital (VMO) of the same symmetry, and (ii) the lowest
singly excited configuration, again arising from the highest occupied
DOMO to the lowest VMO of the same symmetry.  In the present example,
this will correspond to the SCF configuration, the double and single
excitation arising from the DOMO 5a$_{1}$ to VMO 6a$_{1}$, the double
and single excitation arising from the DOMO 1b$_{1}$ to VMO 2b$_{1}$,
and the double and single excitation arising from the DOMO 1b$_{2}$ to
VMO 3b$_{2}$.  Note that the DOMO involved in the latter configurations
is now the 1b$_{2}$ given that the 2b$_{2}$ is now singly occupied, and
again the absence of excitations involving a$_{2}$ MOs given the absence
of such orbitals involved in the occupied manifold. This again results
in a total reference set of 7 functions, as shown thus in the job output:

{
\footnotesize
\begin{verbatim}
   numbers of open shells and corresponding main configurations

  1    38   1   2   3   4   5  28  37          ..   SCF configuration
  1    38   1   2   3   4   6  28  37          ..   5a1 -> 6a1 double
  3     5   6  38   1   2   3   4  28  37      ..   5a1 -> 6a1 single
  1    38   1   2   3   4   5  29  37          ..   1b1 -> 2b1 double
  3    28  29  38   1   2   3   4   5  37      ..   1b1 -> 2b1 single
  1    38   1   2   3   4   5  28  39          ..   1b2 -> 3b2 double
  3    37  38  39   1   2   3   4   5  28      ..   1b2 -> 3b2 single
\end{verbatim}
}
\end{itemize}

The sequence of data lines defining the Semi-direct Table-CI
calculation is again terminated by the ENTER directive.  Note at this
stage that the full data specification corresponding to the defaults
generated from the above data file is as follows

{
\footnotesize
\begin{verbatim}
          TITLE
          H2CO+ 2B2 TZVP - EXPLICIT DATA FOR DEFAULTS
          MULT 2
          CHARGE 1
          SUPER OFF NOSYM
          ZMAT ANGSTROM
          C
          O 1 1.203
          H 1 1.099 2 121.8
          H 1 1.099 2 121.8 3 180.0
          END
          BASIS TZVP
          RUNTYPE CI
          OPEN 1 1
          ACTIVE
          1 TO 52 END
          MRDCI DIRECT
          TABLE
          SELECT
          CNTRL 15
          SPIN 2
          SYMM 3
          SINGLES ALL
          CONF
          1    38   1   2   3   4   5  28  37
          1    38   1   2   3   4   6  28  37
          3     5   6  38   1   2   3   4  28  37
          1    38   1   2   3   4   5  29  37
          3    28  29  38   1   2   3   4   5  37
          1    38   1   2   3   4   5  28  39
          3    37  38  39   1   2   3   4   5  28
          END
          THRESH 10 10
          ROOTS 1
          CI
          NATORB
          CIVEC 1
          VECTORS ATOMS
          ENTER 4 5
\end{verbatim}
}

Let us now consider a Semi-direct Table-CI calculation  on the
\bstateo\  state of \formion, again using default options available
within the module. A valid data sequence for performing such a
calculation is shown below, where we are still performing all the
computation in a single job.

{
\footnotesize
\begin{verbatim}
          TITLE
          H2CO+ 2B1 TZVP - DEFAULT MRDCI SETTINGS 
          MULT 2
          CHARGE 1
          ZMAT ANGSTROM
          C
          O 1 1.203
          H 1 1.099 2 121.8
          H 1 1.099 2 121.8 3 180.0
          END
          BASIS TZVP
          RUNTYPE CI
          MRDCI DIRECT
          SWAP
          7 8
          END
          ENTER
\end{verbatim}
}
Considering the changes to the closed-shell run, the following points
should be noted:
\begin{itemize}
\item The OPEN directive is now present, specified prior to the Table-CI data.
\item In the absence of section specification on the ENTER directive,
the set of vectors used in the Table-CI transformation will be restored
from the default open-shell SCF section containing the energy ordered
eigenvectors (section 5).  The SWAP directive is to generate initial
starting MOs for the B$_1$ state.
\item The symmetry and spin of the CI wavefunction will be deduced
from the preceding SCF calculation i.e. a CI wavefunction of B$_{1}$
symmetry (corresponding to SYMMETRY 2). and a doublet CI wavefunction
(corresponding to SPIN 2).
\item The number of active electrons in the CI will be set to be those
involved in the SCF calculation (i.e. CNTRL 15).
\item Singly excited configurations with respect to each of 
the default reference configurations (SINGLES ALL) will be included,
regardless of their computed energy lowerings.
\item The set of reference configurations to be employed will be
generated using the same algorithm used in the open shell case
above i.e.  the SCF configuration, plus those generated from this
configuration by including (i) for each symmetry IRREP, the doubly
excited configuration arising from excitation of the highest occupied
DOMO of that symmetry to the lowest virtual orbital (VMO) of the same
symmetry, and (ii) the lowest singly excited configuration, again
arising from the highest occupied DOMO to the lowest VMO of the same
symmetry.  In the present example, this will correspond to the SCF
configuration, the double and single excitation arising from the DOMO
5a$_{1}$ to VMO 6a$_{1}$, and the double and single excitation arising
from the DOMO 2b$_{2}$ to VMO 3b$_{2}$. Note that there no excitations
involving a$_{2}$ or b$_{1}$ MOS, given the absence of such orbitals in
the doubly occupied manifold. This now results in a total reference set
of just 5 functions, as shown thus in the job output:

{
\footnotesize
\begin{verbatim}
   numbers of open shells and corresponding main configurations

  1            28   1   2   3   4   5  37  38       ..   SCF configuration
  1            28   1   2   3   4   6  37  38       ..   5a1 -> 6a1 double
  3             5   6  28   1   2   3   4  37  38   ..   5a1 -> 6a1 single
  1            28   1   2   3   4   5  37  39       ..   2b2 -> 3b2 double
  3            28  38  39   1   2   3   4   5  37   ..   2b2 -> 3b2 single
\end{verbatim}
}
\end{itemize}

The sequence of data lines defining the Semi-direct Table-CI
calculation is again terminated by the ENTER directive.  Note 
that the full data specification corresponding to the defaults
generated from the above data file is as follows

{
\footnotesize
\begin{verbatim}
          TITLE
          H2CO+ 2B1 TZVP - EXPLICIT DATA FOR DEFAULTS
          MULT 2
          CHARGE 1
          SUPER OFF NOSYM
          ZMAT ANGSTROM
          C
          O 1 1.203
          H 1 1.099 2 121.8
          H 1 1.099 2 121.8 3 180.0
          END
          BASIS TZVP
          RUNTYPE CI
          OPEN 1 1
          ACTIVE
          1 TO 52 END
          MRDCI DIRECT
          TABLE
          SELECT
          CNTRL 15
          SPIN 2
          SYMM 2
          SINGLES ALL
          CONF
          1    28   1   2   3   4   5  37  38
          1    28   1   2   3   4   6  37  38
          3     5   6  28   1   2   3   4  37  38
          1    28   1   2   3   4   5  37  39
          3    28  38  39   1   2   3   4   5  37
          END
          THRESH 10 10
          ROOTS 1
          CI
          NATORB
          CIVEC 1
          VECTORS ATOMS
          SWAP
          7 8
          END
          ENTER 4 5
\end{verbatim}
}

\section[Memory Specification for the Semi-direct Table-CI Module]{Memory Specification for the Semi-direct Table-CI Module}
The memory requirements of the semi-direct module are typically 
greater than those associated with the conventional algorithm. While
the default memory allocations will prove adequate for "small-medium"
cases, the user should use the MEMORY pre-directive to increase this
allocation in more demanding cases e.g. at least 8 MWords in
calculations with, say, more than 20 active electrons. In this section
we outline the main demands on memory, and the mechanisms for
increasing this allocation should the default allocations prove inadequate.

The overall memory required is determined by four integer quantities, namely;
\begin{enumerate}
\item NTEINT, the number of transformed two-electron integrals. The
default value is 3,500,001.
\item IOTM, the field length for the selected configurations. The
default value is 2,000,000.
\item NEDIM, the dimension of internal arrays required by the
direct-CI algorithm. The default value is 2,000,000.
\item MDI, the maximum dimension of the hamiltonian. The default
value is 1,000,001.
\end{enumerate}

In practice the code will attempt to use these default values and
computes the overall memory requirement based on the settings above. If
this memory is not available, each of the above values will be halved
until the calculation can proceed within the memory allocated to the
job. The latter is 4 MWords in default, a value which will not be
sufficient to accommodate the above settings. Thus an examination of the
semi-direct Table-CI output will typically reveal the following:

{
\footnotesize
\begin{verbatim}
     +++++++++++++++++++++++++++++++++++++++++++++++
     + Insufficient memory for default allocations +
     + reduce parameters as follows                +
     +++++++++++++++++++++++++++++++++++++++++++++++
     +   nteint from  3500001  to 1750001          +
     +      mdi from  1000001  to  500001          +
     +    iotm  from  2000000  to 1000000          +
     +   nedim  from  2000000  to 1000000          +
     +++++++++++++++++++++++++++++++++++++++++++++++
 
 
     +++++++++++++++++++++++++++++++++++++++++++++++
     + Insufficient memory for default allocations +
     + reduce parameters as follows                +
     +++++++++++++++++++++++++++++++++++++++++++++++
     +   nteint from  3500001  to  875001          +
     +      mdi from  1000001  to  250001          +
     +    iotm  from  2000000  to  500000          +
     +   nedim  from  2000000  to  500000          +
     +++++++++++++++++++++++++++++++++++++++++++++++
\end{verbatim}
}

This is usually not a problem, as the default values are such to
accommodate quite demanding applications. Should the decreased settings
prove inadequate (the code should inform the user accordingly), then
the first action is to increase the memory for the job using the
MEMORY pre-directive. Typically a pre-directive setting of 14 MWords
will enable the default settings to apply.

If these defaults prove inadequate, the user must resort to specifying
the parameter to be increased through data input, while increasing the
MEMORY setting to compensate for this increase. This data-driven
allocation is achieved under control of the CORE directive, which should
be presented immediately after the "MRDCI DIRECT" data line.  The
directive consists of a single line with the character string CORE in the
first data field. Subsequent data fields are read in pairs to variables
(TEXTM, IMEMM) in format (A,I), each such pair indicating the parameter
which is to be modified, followed by an integer defining the
revised value. Valid data fields are thus:
\begin{itemize}
\item  TEXTM may be set to one of the character strings, 
NTEINT, IOTM, NEDIM or MDI.
\item  IMEMM is an integer parameter used to specify the
required valued to be used in determining the memory.
\end{itemize}

{\bf Example}\\

Presenting the data line

{
\footnotesize
\begin{verbatim}
          MRDCI DIRECT
          CORE  NTEINT 5000000 
            ..
            ..
\end{verbatim}
}
will act to increase the memory allocation for holding the transformed
two-electron integrals.


\section[Calculating the \astate\ states of \water]{Calculating the \astate\ states of \water}

To clarify our discussion of the Semi-direct Table-CI module, we work
through a typical example of using the Table-CI method in calculating
the energetics and properties of the three low lying \astate\ states of
the \water\ molecule. The basis set employed is the TZVP triple-zeta
plus polarisation set; this is augmented with a diffuse s- and
p-orbital on the oxygen to provide a reasonable description of the
known Rydberg character of the states of interest.

The computation is split into four separate jobs, in which we,
\begin{enumerate}
\item perform the initial SCF;
\item carry out an initial CI, where the reference set employed
acts to provide at least a qualitative description of the states
of interest;
\item based on the output from the initial CI, we augment the
reference set to provide a quantitative description of the
first three states;
\item finally, having generated the CI vectors for the three
states, we carry out in the final job an analysis of each
vector in terms of natural orbitals and one-electron properties,
and generate the transition moments between the ground 
and two excited states.
\end{enumerate}
We now consider various aspects of each job in turn.\\

{\bf Job 1: The SCF}

{
\footnotesize
\begin{verbatim}
          TITLE  
          ****  H2O  TZVP + DIFFUSE S,P MRDCI *
          SUPER OFF NOSYM
          ZMAT ANGSTROM
          O
          H 1 0.951
          H 1 0.951 2 104.5
          END
          BASIS 
          TZVP O
          TZVP H
          S O
          1.0 0.02
          P O
          1.0 0.02
          END
          ENTER
\end{verbatim}
}
The only points to note here is the use of the SUPER directive in
suppressing skeletonisation and, in the absence of section specification
on the ENTER directive, the use of the default section for output of
the SCF eigenvectors (section 1).\\

{\bf Job 2: The Initial 3M/3R CI}\\

An examination of the SCF output
reveals the following orbital analysis.
{
\footnotesize
\begin{verbatim}
          =============================
          IRREP  NO. OF SYMMETRY ADAPTED
                 BASIS FUNCTIONS
          =============================
            1          18
            2           6
            3          10
            4           2
          =============================
\end{verbatim}
}
and the following orbital assignments characterising the closed--shell
SCF configuration:
\begin{equation}
  1a_{1}^{2}  2a_{1}^{2}  1b_{2}^{2}  3a_{1}^{2}  1b_{1}^{2}
\end{equation}
{
\footnotesize
\begin{verbatim}
          ===============================================
          M.O.  IRREP  ORBITAL ENERGY   ORBITAL OCCUPANCY
          ===============================================
            1     1    -20.56084959           2.0000000
            2     1     -1.35696939           2.0000000
            3     3     -0.72200122           2.0000000
            4     1     -0.58247942           2.0000000
            5     2     -0.50858566           2.0000000
            6     1      0.02724259           0.0000000
            7     3      0.04894440           0.0000000
            8     2      0.05589681           0.0000000
            9     1      0.06133571           0.0000000
           10     1      0.20403420           0.0000000
           11     3      0.22824210           0.0000000
           12     3      0.53700802           0.0000000
           13     1      0.56235022           0.0000000
           14     2      0.58645643           0.0000000
           15     1      0.66887228           0.0000000
           16     3      0.74805617           0.0000000
           17     1      1.07690608           0.0000000
           18     1      1.88545053           0.0000000
           19     4      1.92243836           0.0000000
           20     2      2.12944874           0.0000000
           21     3      2.20541910           0.0000000
           22     1      2.34202871           0.0000000
           23     3      2.39946430           0.0000000
           24     3      2.69788310           0.0000000
           25     1      2.72651832           0.0000000
           26     2      2.73832720           0.0000000
           27     1      3.07664215           0.0000000
           28     3      3.26840142           0.0000000
           29     2      3.54616570           0.0000000
           30     1      3.58631019           0.0000000
           31     4      3.59701772           0.0000000
           32     1      3.84174131           0.0000000
           33     1      4.84610143           0.0000000
           34     3      5.14220270           0.0000000
           35     1      7.73115986           0.0000000
           36     1     47.56758932           0.0000000
          ===============================================
\end{verbatim}
}
Based on the above output, the CONF data lines may be
deduced from the following table, where we
assume that we wish to freeze the O1s inner shell orbitals and discard 
the inner shell complement orbital:

\begin{centering}
\begin{tabular}{llrrrr}

\\ \hline
IRrep & IRrep &  No. of Basis & Frozen & Active & Sequence  \\
      & No.  &   Functions    & MOs    & MOs    & Nos.       \\ \hline
 a$_{1}$  & 1  &   18         & 1      &  16     & 1-16       \\
 b$_{1}$  & 2  &   6          & 0      &  6     & 17-22   \\ 
 b$_{2}$  & 3  &   10         & 0      &  10    & 23-32       \\
 a$_{2}$  & 4  &   2          & 0      &  2     & 33-34   \\ \hline
\end{tabular}
 
\end{centering}
\vspace{.15in}
Note that the virtual SCF MOs dominated by the diffuse oxygen basis
functions are the 4a$_{1}$, the 2b$_{2}$, the 2b$_{1}$ and the
5a$_{1}$, with SCF sequence numbers 6,7,8 and 9 respectively.  The
symmetry re-ordered sequence numbers, allowing for the effective
removal of the two a$_{1}$ orbitals, are 3, 24, 18 and 5 respectively.
To perform a three-root 8-electron valence-CI calculation, based on the
SCF configurations of the ground and excited Rydberg states, involving
the single excitations (1b$_{1}$ to 2b$_{1}$) and (3a$_{1}$ to
4a$_{1}$) would require the following CONF data:

{
\footnotesize
\begin{verbatim}
          CONF
          0        1 2  17  23
          2  2  3  1    17  23
          2 17 18  1 2  23
          END
\end{verbatim}
}
The following data will perform this three root-CI, where
\begin{itemize}
\item the SCF computation is BYPASS'ed;
\item the freezing and discarding of the two a$_{1}$ MOs is accomplished
using the CORE and ACTIVE directives;
\item the default sub-module specifications are in effect, with no
specific need to reference CI or NATORB activity;
we assume that the table-ci data set is not available to the job,
and is to be generated in this run;
\item the ROOTS directive is specifying selection with respect
to the first 3 roots of the zero order problem, which we
assume will correspond to the states of interest.
\end{itemize}
{
\footnotesize
\begin{verbatim}
          RESTART NEW
          TITLE  
          H2O  TZVP + DIFFUSE S,P TABLE-CI 3M/3R*
          SUPER OFF NOSYM
          BYPASS SCF
          ZMAT ANGSTROM
          O
          H 1 0.951
          H 1 0.951 2 104.5
          END
          BASIS 
          TZVP O
          TZVP H
          S O
          1.0 0.02
          P O
          1.0 0.02
          END
          RUNTYPE CI
          ACTIVE 
          2 TO 35
          END
          CORE
          1 END
          MRDCI DIRECT
          SELECT
          CNTRL 8
          CONF 
          0        1 2  17  23
          2  2  3  1    17  23
          2 17 18  1 2  23
          END
          THRE 10 10
          ROOTS 3
          ENTER
\end{verbatim}
}

{\bf Job 3: The Final 12M/3R CI}\\

An examination of the output from the initial CI reveals that
the dominant configurations have, as expected, been included.
We show below the final CI vectors for each of the states:
not surprisingly the ground state is more accurate, by virtue
of its SCF MOs having been employed. Augmenting the reference
set to improve the description of the two excited states
follows straightforwardly from the statistics below: \\

{\bf Description of the X$^{1}A_{1}$ state}\\
{
\footnotesize
\begin{verbatim}
      EXTRAPOLATED ENERGY =   -76.27252960

          *************************
          * CONFIGURATION WEIGHTS *
          *************************
          ***** ROOT  1 ***********
          =============================
                  C*C     CONFIGURATION
          =============================
          M   0.94942   1   2  17  23
              0.00141   1   2  19  23
              0.00112   2   7  17  19   1  23
              0.00125   2   7  23  27   1  17
              0.00147  17  19  23  27   1   2
 
 SUM OF MAIN REFERENCE C*C =  0.949613377982042

\end{verbatim}
}
{\bf Description of the 1$^{1}A_{1}$ state}\\
{
\footnotesize
\begin{verbatim}
      EXTRAPOLATED ENERGY =   -75.90090942

          *************************
          * CONFIGURATION WEIGHTS *
          *************************
          ***** ROOT  2 ***********
          =============================
                  C*C     CONFIGURATION
          =============================
              0.00283   1   2  18  23
          M   0.06414   2   3   1  17  23
              0.00758   2   5   1  17  23
          M   0.84001  17  18   1   2  23
              0.00263  17  19   1   2  23
              0.01150  18  19   1   2  23
              0.00221   1   5  17  18   2  23
              0.00136   1   8  17  18   2  23
              0.00226   2   5  17  18   1  23
              0.00444   2   6  17  18   1  23
              0.00550   2   7  17  18   1  23
              0.00667  17  18  23  25   1   2
              0.00664  17  18  23  27   1   2

      SUM OF MAIN REFERENCE C*C = 0.904355147296805
\end{verbatim}
}
{\bf Description of the 2$^{1}A_{1}$ state}\\
{
\footnotesize
\begin{verbatim}
      EXTRAPOLATED ENERGY =   -75.88256073

          *************************
          * CONFIGURATION WEIGHTS *
          *************************
          ***** ROOT  3 ***********
          =============================
                  C*C     CONFIGURATION
          =============================
          M   0.78360   2   3   1  17  23
              0.00618   2   4   1  17  23
              0.05594   2   5   1  17  23
          M   0.07077  17  18   1   2  23
              0.00236   3   5   1  17  23
              0.00344   3   6   1  17  23
              0.00494   3   7   1  17  23
              0.00132   1   2   3   5  17  23
              0.00191   2   3  17  18   1  23
              0.00872   2   3  17  19   1  23
              0.00639   2   3  23  25   1  17
              0.00600   2   3  23  27   1  17


      SUM OF MAIN REFERENCE C*C = 0.854543160127212
\end{verbatim}
}
Taking as the criterion for inclusion a weight of 0.005, the final 12
reference set-CI is shown below. We have assumed that the FORTRAN {\em
interface} FTN031, plus the TABLE data base, table-ci, has been saved
from the second job, enabling us to bypass the transformation and data
base generation.

{
\footnotesize
\begin{verbatim}
          RESTART CI
          TITLE
          H2O  TZVP + DIFFUSE S,P TABLE-CI 12M/3R *
          SUPER OFF NOSYM
          BYPASS SCF TRAN
          ZMAT ANGSTROM
          O
          H 1 0.951
          H 1 0.951 2 104.5
          END
          BASIS
          TZVP O
          TZVP H
          S O
          1.0 0.02
          P O
          1.0 0.02
          END
          RUNTYPE CI
          ACTIVE
          2 TO 35
          END
          CORE
          1 END
          MRDCI DIRECT
          TABLE BYPASS
          SELECT
          CNTRL 8
          CONF
          0        1 2  17  23
          2  2  3  1    17  23
          2  2  4  1    17  23
          2  2  5  1    17  23
          2 17 18  1 2  23
          2 18 19  1 2  23
          4 17 18 23 25  1  2
          4 17 18 23 27  1  2
          4  2  7 17 18  1  23
          4  2  3 17 19  1  23
          4  2  3 23 25  1  17
          4  2  3 23 27  1  17
          END
          THRE 10 10
          ROOT 3
          ENTER
\end{verbatim}
}
{\bf Job 4: The Analysis}\\

Assuming that the diagonalisation {\em interface}, FTN036, had been
saved above, then the final analysis job is straightforward: again
bypassing of the various sub-modules involves explicit
mention of the TABLE, and CI modules, in addition to
flagging the previous SELECT data lines with the BYPASS
keyword. We have routed the natural orbitals from the 3 A$_{1}$ states
to the Dumpfile using the PUTQ directive.

{
\footnotesize
\begin{verbatim}
          RESTART CI
          TITLE
          ****  H2O  TZVP + DIFFUSE S,P TABLE-CI / ANALYSIS *
          SUPER OFF NOSYM
          BYPASS SCF TRAN
          ZMAT ANGSTROM
          O
          H 1 0.951
          H 1 0.951 2 104.5
          END
          BASIS
          TZVP O
          TZVP H
          S O
          1.0 0.02
          P O
          1.0 0.02
          END
          ACTIVE
          2 TO 35
          END
          CORE
          1 END
          RUNTYPE CI
          MRDCI DIRECT
          TABLE BYPASS
          SELECT BYPASS
          CNTRL 8
          CONF
          0        1 2  17  23
          2  2  3  1    17  23
          2  2  4  1    17  23
          2  2  5  1    17  23
          2 17 18  1 2  23
          2 18 19  1 2  23
          4 17 18 23 25  1  2
          4 17 18 23 27  1  2
          4  2  7 17 18  1  23
          4  2  3 17 19  1  23
          4  2  3 23 25  1  17
          4  2  3 23 27  1  17
          END
          THRE 10 10
          ROOT 3
          CI BYPASS
          NATORB IPRIN
          CIVE 1 2 3
          PUTQ AOS 50 51 52
          PROP
          CIVE 1 2 3
          0        1 2  17  23
          2 17 18  1 2  23
          2  2  3  1    17  23
          MOMENT
          36 1 36 2 2
          ENTER
\end{verbatim}
}

{\bf Description of the Output for MRDCI Moments}

The MRDCI module calculates the oscillator strength using both the
dipole length formalism: 

$$f({\bf r}) = 2/3 <\Psi^{'}|{\bf r}|\Psi^{''}>^{2}/ \Delta E$$

and the dipole velocity formalism: 

$$f(\bigtriangledown) = 2/3 |<\Psi^{'}|{\bf \bigtriangledown}|\Psi^{''}>^{2}/ \Delta E$$

The most significant contributions due to individual molecular orbitals 
are  printed out as a table containing the largest coefficients of the 
transition density matrix and the following corresponding integrals.

$$<\psi_{i}|x|\psi_{j}> \hspace{.75in}   <\psi_{i}|y|\psi_{j}> \hspace{.75in}   <\psi_{i}|z|\psi_{j}>$$ 

$$<\psi_{i}|\bigtriangledown(x)|\psi_{j}> \hspace{.75in}   <\psi_{i}|\bigtriangledown(y)|\psi_{j}> \hspace{.75in}  <\psi_{i}|\bigtriangledown(z)|\psi_{j}>$$ 


The f({\bf r}) and f($\bigtriangledown$) values are printed out in x,y,z 
components and the expectation values for 
$<\psi|\sum_{i} x_{i},y_{i}, z_{i}|\psi>$ are also printed.

\section[Iterative Natural Orbital Calculations]{Iterative Natural Orbital Calculations}

We now work through an example of using the natural orbitals generated
by the module in a subsequent CI calculation.  We consider a DZ
calculation on the ground state of \ethene, with the computation split
into four separate jobs, in which we,
\begin{enumerate}
\item perform the initial SCF;
\item carry out an initial CI, where the reference set employed
comprises just the SCF configuration, using the SCF MOs;
of interest;
\item based on the output from the initial CI, we augment the
reference set to include the leading secondary configuration,
generating the resulting natural orbitals;
\item carry out the 2-reference CI based on the natural orbitals
generated in the previous step.
\end{enumerate}
We now consider various aspects of each job in turn.\\

{\bf Job 1: The SCF}
{
\footnotesize
\begin{verbatim}
          TITLE 
          ETHYLENE DZ GROUND STATE SCF
          SUPER OFF NOSYM
          ZMATRIX ANGSTROM
          C
          C 1 1.4
          H 1 1.1 2 120.0
          H 1 1.1 2 120.0 3 180.0
          H 2 1.1 1 120.0 3 0.0
          H 2 1.1 1 120.0 3 180.0
          END
          BASIS DZ
          ENTER
\end{verbatim}
}
The only points to note here is the use of the SUPER directive in
suppressing skeletonisation.\\

{\bf Job 2: The 2M/1R CI}\\

An examination of the SCF output reveals the following orbital analysis.
{
\footnotesize
\begin{verbatim}
           ==============================
           IRREP  NO. OF SYMMETRY ADAPTED
                  BASIS FUNCTIONS
           ==============================
             1           8
             2           2
             3           4
             5           8
             6           2
             7           4
           =============================
\end{verbatim}
}
and the following orbital assignments characterising the closed--shell
SCF configuration:
\begin{equation}
  1a_{g}^{2}  1b_{1u}^{2}  2a_{g}^{2}  2b_{1u}^{2}  1b_{2u}^{2} 3a_{g}^{2}  1b_{3g}^{2}  1b_{3u}^{2}  
\end{equation}
{
\footnotesize
\begin{verbatim}
          ===============================================
          M.O.  IRREP  ORBITAL ENERGY   ORBITAL OCCUPANCY
          ===============================================
            1     1    -11.25533463           2.0000000
            2     5    -11.25413119           2.0000000
            3     1     -1.02052567           2.0000000
            4     5     -0.79195744           2.0000000
            5     3     -0.64152856           2.0000000
            6     1     -0.57038928           2.0000000
            7     7     -0.51438205           2.0000000
            8     2     -0.36388693           2.0000000
            9     6      0.13190687           0.0000000
           10     5      0.25441028           0.0000000
           11     1      0.25558269           0.0000000
           12     3      0.33918097           0.0000000
           13     5      0.36641681           0.0000000
           14     1      0.41844853           0.0000000
           15     3      0.45487336           0.0000000
           16     7      0.49648833           0.0000000
           17     2      0.49708917           0.0000000
           18     6      0.60222488           0.0000000
           19     7      0.64242537           0.0000000
           20     1      0.76465797           0.0000000
           21     5      0.82560838           0.0000000
           22     5      1.10140194           0.0000000
           23     1      1.20630804           0.0000000
           24     3      1.30189632           0.0000000
           25     5      1.35219192           0.0000000
           26     7      1.50761510           0.0000000
           27     1     23.76609415           0.0000000
           28     5     24.01493168           0.0000000
           ===============================================
\end{verbatim}
}
Based on the above output, the CONF data lines may be
deduced from the following table, where we
assume that we wish to freeze the C1s inner shell orbitals:

\begin{centering}
\begin{tabular}{llrrrr}

\\ \hline
IRrep & IRrep &  No. of Basis & Frozen & Active & Sequence  \\
      & No.  &   Functions    & MOs    & MOs    & Nos.       \\ \hline
 a$_{g}$   & 1  &   8         & 1      &  7     & 1-7       \\
 b$_{3u}$  & 2  &   2         & 0      &  2     & 8-9   \\ 
 b$_{2u}$  & 3  &   4         & 0      &  4     & 10-13       \\
 b$_{1u}$  & 5  &   8         & 1      &  7     & 14-20   \\ 
 b$_{2g}$  & 6  &   2         & 0      &  2     & 21-22   \\ 
 b$_{3g}$  & 7  &   4         & 0      &  4     & 23-26   \\  \hline
\end{tabular}
 
\end{centering}
\vspace{.15in}
Note that within the DZ basis employed, there are no basis functions of
b$_{1g}$ (IRrep 4) or  a$_{u}$ (IRrep 8) symmetry.
The symmetry re-ordered sequence numbers of the ground
state orbitals, allowing for the
effective removal of the 1a$_{g}$ and 1b$_{1u}$ orbitals, 
are 1, 14, 10, 2, 23 and 8 respectively.
To perform a two reference, 12-electron valence-CI calculation,
based on the SCF configuration 
\begin{equation}
  2a_{g}^{2}  2b_{1u}^{2}  1b_{2u}^{2} 3a_{g}^{2}  1b_{3g}^{2}  1b_{3u}^{2}  
\end{equation}
and the doubly excited configuration:
\begin{equation}
  2a_{g}^{2}  2b_{1u}^{2}  1b_{2u}^{2} 3a_{g}^{2}  1b_{3g}^{2}  1b_{2g}^{2}  
\end{equation}
would require the following CONF data:

{
\footnotesize
\begin{verbatim}
          CONF
          0 1 2 8 10 14 23
          0 1 2   10 14 21 23
          END
\end{verbatim}
}
The following data will perform this CI, where
\begin{itemize}
\item The SCF computation is BYPASS'ed;
\item The freezing and discarding of the 1a$_{g}$ and 1b$_{1u}$ MOs
is accomplished using the CORE and ACTIVE directives;
\item The default sub-module specifications are in effect, with no
specific need to reference TABLE or CI activity i.e.
the table-ci data set is to be constructed in the job;
\item The natural orbitals are routed to section 200 of
the Dumpfile. Note that subsequent usage of the NOs by
the Table-CI module requires the SABF specification on the
PUTQ directive.
\end{itemize}
{
\footnotesize
\begin{verbatim}
          RESTART NEW
          TITLE 
          ETHYLENE CI GROUND STATE 2M SCF-MOS
          BYPASS SCF
          ZMATRIX ANGSTROM
          C
          C 1 1.4
          H 1 1.1 2 120.0
          H 1 1.1 2 120.0 3 180.0
          H 2 1.1 1 120.0 3 0.0
          H 2 1.1 1 120.0 3 180.0
          END
          BASIS DZ
          CORE
          1 2 END
          ACTIVE
          3 TO 28 END
          RUNTYPE CI
          MRDCI DIRECT
          SELECT
          SYMMETRY 1
          CNTRL 12
          SPIN 1
          SINGLES 1
          CONF
          0 1 2 8 10 14 23
          0 1 2   10 14 21 23
          END
          THRESH 30 10
          NATORB IPRINT
          PUTQ SABF 200
          ENTER
\end{verbatim}
}
{\bf Job 3: The Natural Orbital CI}\\

We show below the data for using the NOs from the 2-reference
CI, where the orbitals routed to section 200 are now restored
by specification on the ENTER directive. The following points
should be noted:
\begin{itemize}
\item We assume that the Table-CI data base, table-ci,
been saved from the initial CI job, allowing the BYPASS
specification on the TABLE directive.
\item In contrast to the conventional Table-CI module, 
restoring the NOs {\em must} now be controlled through
VECTORS and ENTER specification; in the conventional
module this was input through TRAN specification
within the Table-CI transformation module.
\item The resulting NOs from the natural orbital CI are now
routed to section 210, and could be used in a subsequent CI in
obvious fashion.
\end{itemize}

{
\footnotesize
\begin{verbatim}
          RESTART NEW
          TITLE 
          ETHYLENE CI GROUND STATE 2M NOS
          BYPASS SCF
          ZMATRIX ANGSTROM
          C
          C 1 1.4
          H 1 1.1 2 120.0
          H 1 1.1 2 120.0 3 180.0
          H 2 1.1 1 120.0 3 0.0
          H 2 1.1 1 120.0 3 180.0
          END
          BASIS DZ
          3 TO 28 END
          CORE
          1 2 END
          RUNTYPE CI
          MRDCI DIRECT
          TABLE BYPASS
          SELECT
          CNTRL 12
          SYMMETRY 1
          SPIN 1
          SINGLES 1
          CONF
          0 1 2 8 10 14 23
          0 1 2   10 14 21 23
          END
          THRESH 30 10
          NATORB IPRINT
          PUTQ SABF 210
          VECTORS 200
          ENTER 200
\end{verbatim}
}
We show below the final CI vector from the natural orbital CI\\

{\bf Description of the X$^{1}A_{g}$ state}\\
{
\footnotesize
\begin{verbatim}
      EXTRAPOLATED ENERGY =   -78.20206451

          *************************
          * CONFIGURATION WEIGHTS *
          *************************
          ***** ROOT  1 ***********
          =============================
                  C*C     CONFIGURATION
          =============================
          M   0.89794   1   2   8  10  14  23
              0.00125   1   8  10  14  15  23
          M   0.03203   1   2  10  14  21  23
              0.00100   1   2   8  11  14  23
              0.00102   1   2   8  10  11  14
              0.00279   1   8  15  21   2  10  14  23
              0.00106   1   3  14  16   2   8  10  23
              0.00499   2   8  15  21   1  10  14  23
              0.00143   2   8  16  21   1  10  14  23
              0.00186   2   3  14  16   1   8  10  23
              0.00119   2  11  16  23   1   8  10  14
              0.00101   8   9  10  11   1   2  14  23
              0.00144   8  10  21  24   1   2  14  23
              0.00260   3   8  14  21   1   2  10  23
              0.00110   8   9  23  24   1   2  10  14
              0.00104   8  12  21  23   1   2  10  14
              0.00117   3  10  16  23   1   2   8  14
              0.00297  10  11  23  24   1   2   8  14
              0.00125   3  11  14  23   1   2   8  10
 
      SUM OF MAIN REFERENCE C*C = 0.929973257309359
\end{verbatim}
}

\section[Table-CI Calculations Using MCSCF Orbitals]{Table-CI Calculations Using MCSCF Orbitals}

To conclude our discussion of the semi-direct Table-CI module, we work
through an example of using the natural orbitals generated from the
MCSCF module in a subsequent CI calculation.  We consider a calculation
on the ground state of \formaldehyde, with the computation split into
three separate jobs, in which we,
\begin{enumerate}
\item perform an initial SCF;
\item carry out the MCSCF calculation;
\item perform the MRDCI calculation using the MCSCF natural orbitals.
\end{enumerate}
We now consider various aspects of each job in turn, and note that
several changes will be required to the corresponding data sets shown
above for the Conventional Table-CI study.\\

{\bf Job 1: The SCF}

{
\footnotesize
\begin{verbatim}
          TITLE 
          H2CO - DZP + F
          SUPER OFF NOSYM
          ZMATRIX ANGSTROM
          C
          O 1 1.203
          H 1 1.099 2 121.8
          H 1 1.099 2 121.8 3 180.0
          END
          BASIS
          DZP O
          DZP C
          DZP H
          F C
          1 1.0
          F O
          1.0 1.0
          END
          ENTER
\end{verbatim}
}
The only points to note here again is the use of the SUPER directive in
suppressing skeletonisation, and use of the default eigenvector section
(section 1) for storage of the closed-shell eigenvectors.  An
examination of the SCF output reveals the following orbital analysis.

{
\footnotesize
\begin{verbatim}
           ==============================
           IRREP  NO. OF SYMMETRY ADAPTED
                  BASIS FUNCTIONS
           ==============================
             1          28
             2          13
             3          16
             4           5
           =============================
\end{verbatim}
}
and the following orbital assignments characterising the closed--shell
SCF configuration:

{
\footnotesize
\begin{verbatim}
           ===============================================
           M.O.  IRREP  ORBITAL ENERGY   ORBITAL OCCUPANCY
           ===============================================
              1     1    -20.57768533           2.0000000
              2     1    -11.34457777           2.0000000
              3     1     -1.40746540           2.0000000
              4     1     -0.87003449           2.0000000
              5     3     -0.69591811           2.0000000
              6     1     -0.65109519           2.0000000
              7     2     -0.53687971           2.0000000
              8     3     -0.44174805           2.0000000
              9     2      0.11697212           0.0000000
             10     1      0.26220763           0.0000000
             11     1      0.27217357           0.0000000
             12     3      0.38931080           0.0000000
             13     3      0.41757152           0.0000000
             14     2      0.46526352           0.0000000
             15     1      0.60968525           0.0000000
             16     1      0.75001014           0.0000000
             17     2      0.86980119           0.0000000
             18     1      0.89167074           0.0000000
             19     3      0.93051881           0.0000000
             20     1      1.07098621           0.0000000
             21     3      1.18616042           0.0000000
             22     1      1.35370640           0.0000000
             23     4      1.52221224           0.0000000
             24     2      1.68895841           0.0000000
             25     1      1.88480823           0.0000000
             26     3      1.97392259           0.0000000
             27     4      2.13452238           0.0000000
             28     1      2.13982211           0.0000000
             29     3      2.19187925           0.0000000
             30     1      2.34994146           0.0000000
             31     2      2.36355601           0.0000000
             32     4      2.67698155           0.0000000
             33     1      2.82812279           0.0000000
             34     2      2.84696649           0.0000000
             35     3      2.97321688           0.0000000
             36     1      3.14466153           0.0000000
             37     1      3.33275225           0.0000000
             38     3      3.51306001           0.0000000
             39     1      3.51697350           0.0000000
             40     2      3.52974692           0.0000000
             41     3      3.68500148           0.0000000
             42     1      3.79236483           0.0000000
             43     3      3.83958499           0.0000000
             44     4      3.85444369           0.0000000
             45     2      3.87171104           0.0000000
             46     2      4.14123645           0.0000000
             47     3      4.16304613           0.0000000
             48     1      4.16535425           0.0000000
             49     2      4.16620625           0.0000000
             50     2      4.32796704           0.0000000
             51     3      4.47390388           0.0000000
             52     1      4.49325977           0.0000000
             53     1      4.68584478           0.0000000
             54     4      4.89513471           0.0000000
             55     3      5.13208791           0.0000000
             56     1      5.14756255           0.0000000
             57     2      5.85901984           0.0000000
             58     1      5.92954017           0.0000000
             59     3      6.06228738           0.0000000
             60     1      8.35480844           0.0000000
             61     1     27.71561806           0.0000000
             62     1     45.72664766           0.0000000
           ===============================================
\end{verbatim}
}

{\bf Job 2: The MCSCF}\\

The following data performs a 10 electron in 9 orbital CASSCF
calculation using the MCSCF module, with the natural orbitals routed to
section 10 of the Dumpfile under control of the CANONICAL directive. In
the absence of the VECTORS directive, the SCF MOs will be used as the
starting orbitals. This data set is just that provided in the
conventional Table-CI case shown above.

{
\footnotesize
\begin{verbatim}
          RESTART
          TITLE
          H2CO  - MCSCF (10E IN 9 M.O.)
          SUPER OFF NOSYM
          NOPRINT
          BYPASS
          ZMATRIX ANGSTROM
          C
          O 1 1.203
          H 1 1.099 2 121.8
          H 1 1.099 2 121.8 3 180.0
          END
          BASIS
          DZP O
          DZP C
          DZP H
          F C
          1 1.0
          F O
          1.0 1.0
          END
          SCFTYPE MCSCF
          MCSCF
          ORBITAL
          COR1 COR1 COR1 DOC1 DOC3 DOC1 DOC2 DOC3 UOC2 UOC1 UOC3 UOC1
          END
          PRINT ORBITALS VIRTUALS NATORB
          CANONICAL 10 FOCK DENSITY FOCK
          ENTER
\end{verbatim}
}

{\bf Job 3: The Table-CI Job}\\

Performing a semi-direct Table-CI calculation using the natural
orbitals generated in the previous step is fairly straightforward. The
following points should be noted:
\begin{itemize}
\item The MCSCF data presented in the preceding step must remain as
part of the input data set, with that computation now BYPASS'ed.
\item While specification of the input orbital set in the conventional
Table-CI calculation requires use of the TRAN directive, it is assumed
in the present case that the orbital set required is that nominated on
the CANONICAL directive. i.e. no explicit section specification is required.
\item With no frozen or discarded orbital, the orbital indices
specified on the CONF directive follow in obvious fashion from the list
of IRREPs given above. We are performing a simple 16 electron, 3
reference calculation, deriving just the first root, and using a 2
micro-hartree threshold.  
\end{itemize}

{
\footnotesize
\begin{verbatim}
          RESTART
          TITLE
          H2CO  - MCSCF (10E IN 9 M.O.) DIRECT-MRDCI FROM MCSCF NOS (SEC.10)
          SUPER OFF NOSYM
          BYPASS SCF
          ZMAT ANGSTROM
          C
          O 1 1.203
          H 1 1.099 2 121.8
          H 1 1.099 2 121.8 3 180.0
          END
          BASIS 
          DZP O
          DZP C
          DZP H
          F C
          1 1.0
          F O
          1.0 1.0
          END
          RUNTYPE CI
          SCFTYPE MCSCF
          MCSCF
          ORBITAL
          COR1 COR1 COR1 DOC1 DOC3 DOC1 DOC2 DOC3 UOC2 UOC1 UOC3 UOC1
          END
          PRINT ORBITALS VIRTUALS NATORB
          CANONICAL 10 FOCK DENSITY FOCK
          MRDCI DIRECT
          TABLE
          SELECT
          SYMMETRY 1
          SPIN 1
          CNTRL 16
          SINGLES 1
          CONF
          0 1 2 3 4 5  29  42 43
          0 1 2 3 4 5  30  42 43
          0 1 2 3 4 5  29  42 44
          END
          ROOTS 1
          THRESH 2 2
          CI
          NATORB
          ENTER
\end{verbatim}
}

Finally, we consider the data for performing exactly the same
calculation as above, but now freezing the oxygen and carbon 1s core
orbitals in the Table-CI calculation. The following points should be
noted:
\begin{itemize}
\item Unlike the conventional Table-CI data, specification of the
frozen orbitals now requires use of the ACTIVE and CORE directives.
the CORE directive below specifying MOs 1 and 2, the ACTIVE directive
specifying orbitals 3 to 62.
\item The orbital indices specified on the CONF data lines reflect
the removal of these two orbitals, with the CNTRL directive
now pointing to a 12-electron CI calculation, as distinct from
the 16 electron calculation above.
\end{itemize}

{
\footnotesize
\begin{verbatim}
          RESTART
          TITLE
          H2CO  - DIRECT-MRDCI FROM MCSCF NOS (SEC.10) - FREEZE 1S
          SUPER OFF NOSYM
          BYPASS SCF
          ZMAT ANGSTROM
          C
          O 1 1.203
          H 1 1.099 2 121.8
          H 1 1.099 2 121.8 3 180.0
          END
          BASIS
          DZP O
          DZP C
          DZP H
          F C
          1 1.0
          F O
          1.0 1.0
          END
          RUNTYPE CI
          ACTIVE
          3 TO 62
          END
          CORE
          1 TO 2
          END
          SCFTYPE MCSCF
          MCSCF
          ORBITAL
          COR1 COR1 COR1 DOC1 DOC3 DOC1 DOC2 DOC3 UOC2 UOC1 UOC3 UOC1
          END
          PRINT ORBITALS VIRTUALS NATORB
          CANONICAL 10 FOCK DENSITY FOCK
          MRDCI DIRECT
          TABLE
          SELECT
          SYMMETRY 1
          SPIN 1
          CNTRL 12
          SINGLES 1
          CONF
          0 1 2 3  27  40 41
          0 1 2 3  28  40 41
          0 1 2 3  27  40 42
          END
          ROOTS 1
          THRESH 2 2
          CI
          NATORB
          ENTER
          EOF
\end{verbatim}
}

\section[Iterative MRDCI Calculations]{Iterative MRDCI Calculations}

All of the MRDCI examples presented to date involve a single run of
the module in which a number of excited states of specified symmetry are
typically generated based on a user specified list of main configurations.
Generating the entire sequence of states required for example in
simulating the vertical electronic spectra of a given species is often a
labour intensive exercise, requiring the repeated refinement of reference
configurations. The ITERATE directive is designed to shorten this process
by allowing for an iterative sequence of MRDCI calculations in which the
initial reference set and associated eigenstates are iteratively refined
with the minimum level of user intervention. This {\em modus operandi}
is designed such that the user:
\begin{enumerate}
\item need present no explicit configuration (CONF) data;
\item may iteratively improve the quality of the CI wavefunction
of a single eigen state by specifying the desired value of c**2
(coefficients$^{2}$) of the main reference configurations.
\item may generate up to 30 eigen states of a specified spin and spatial
symmetry in a single run of the module with no prior knowledge of the
composition of these states.
\end{enumerate}
This iterative treatment is requested and controlled by the user
through a number of options specified by the ITERATE directive. Before
detailing these options, we would point to the following
aspect of ITERATE usage;
\begin{itemize}
\item When specifying higher angular functions in the basis set
to describe either Rydberg or Polarisation functions, the user is
strongly recommended to use the HARMONIC directive to conduct all MRDCI
calculations in a spherical harmonic rather than cartesian basis.
\item As described previously, the memory requirements of the semi-direct
module may be significantly greater than those associated with the
conventional algorithm. As the iterative cycles of the ITERATE algorithm
proceed, each energy calculation will become more demanding in memory
as the number of associated reference configurations and overall size
of the selected configuration space increases.  While the default memory
allocations may prove adequate at the outset of ITERATE processing, they
are unlikely to prove so throughout, and the user should use the MEMORY
pre-directive to request at least 20 MWords in calculations with, say,
more than 20 active electrons. The user should try and avoid the onset
of multi-passing of the eigenstate generation, a consequence of running
with restrictive memory in the later stages of the iterative processing.
\end{itemize}

\subsection{The ITERATE Directive and Associated Options}

The role of the ITERATE directive is twofold, (i) to trigger a sequence
of iterative MRDCI calculations rather a single calculation and (ii)
to provide a mechanism for overriding the default MRDCI settings.
The latter is achieved by specifying the ITERATE options described below
on one or more data lines, each containing the character string ITERATE in
the first data field; the user may present as many data lines as desired
in specifying these options, providing the mechanism for presenting long
option lists over several lines.  Note that the ITERATE data lines should
be the last of the MRDCI options presented, being typically followed by
e.g. the VECTORS or ENTER directive.


\subsubsection{The MAXITER Directive}

This directive may be used to specify the maximum number of iterative
MRDCI calculations to be undertaken. 
The directive consists of two data fields, read to the variables TEXT,
MXITER  using format (A,I);
\begin{itemize}
\item TEXT should be set to the character string MAXITER;
\item MXITER is an integer used to specify the maximum number of
MRDCI calculations to be performed.
\end{itemize}
The directive may be omitted when MXITER will be set to the default 
value of 8.\\

{\bf Example}

{
\footnotesize 
\begin{verbatim}
          ITERATE MAXITER 20
\end{verbatim}
}

\subsubsection{The WEIGHT Directive}

This directive may be used to establish the criterion whereby a secondary
configuration will be elevated to the status of a reference function in
all subsequent iterative calculations.
The directive consists of two data fields, read to the variables TEXT,
CWEIGHT  using format (A,F);
\begin{itemize}
\item TEXT should be set to the character string WEIGHT;
\item All configurations with a CI weight (coefficients$^{2}$) greater
than CWEIGHT in magnitude in any of the derived CI wavefunctions will,
in all subsequent MRDCI iterations, be treated as a reference function.
\end{itemize}
The directive may be omitted when CWEIGHT will be set to the default 
value of 0.005.\\

{\bf Example}

{
\footnotesize 
\begin{verbatim}
          ITERATE WEIGHT 0.003
\end{verbatim}
}

\subsubsection{The C**2 Directive}

Limited to the treatment of a single CI wavefunction, this directive
may be used to define the required level of accuracy of the final
wavefunction, as reflected by the sum of the weights (C**2) of the main
reference configurations. The directive consists of two data fields,
read to the variables TEXT, WEIGHTM  using format (A,F);
\begin{itemize}
\item TEXT should be set to the character string C**2;
\item The reference set will continue to be augmented with secondary
configurations until the final value of C**2 (coefficients$^{2}$) for
the CI wavefunction equals or exceeds the value specified by WEIGHTM.
This is accomplished by reducing the default level of CWEIGHT for the
secondary coefficients in consecutive MRDCI iterations.
\end{itemize}
The directive may be omitted when WEIGHTM will be set to the default 
value of 0.95.\\

{\bf Example}

{
\footnotesize 
\begin{verbatim}
          ITERATE C**2 0.95
\end{verbatim}
}

\subsubsection{The Ethylene Ground state Wavefunction}

{
\footnotesize 
\begin{verbatim}
         TITLE   
         ETHYLENE CI GROUND STATE ITERATE to C*2=0.95
         ZMATRIX ANGSTROM
         C       
         C 1 1.4 
         H 1 1.1 2 120.0 
         H 1 1.1 2 120.0 3 180.0 
         H 2 1.1 1 120.0 3 0.0
         H 2 1.1 1 120.0 3 180.0 
         END     
         BASIS DZ
         CORE    
         1 2 END 
         RUNTYPE CI
         MRDCI DIRECT
         SELECT  
         SYMMETRY 1
         CNTRL 12
         SPIN 1  
         CONF    
         0 1 2 8 10 14 23
         0 1 2   10 14 21 23
         END     
         THRESH 3 3
         ITERATE MAXI 20 C**2 0.95 WEIGHT 0.002
         ENTER
\end{verbatim}
}

The initial CI calculation concludes with the following analysis:

{
\footnotesize 
\begin{verbatim}
        =====================================================================
        ==           Current Energies from MRDCI Iterations                ==
        =====================================================================
        == State    c**2       Energy          Energy          Davidson    ==
        ==                   (T=  3, a.u.)   (T=0, a.u.)        (a.u.)     ==
        =====================================================================
        ==  1      0.922      -78.199076      -78.199862      -78.213624   ==
        =====================================================================
\end{verbatim}
}

Five iterations of the MRDCI module are subsequently required to increase
the initial value of c**2 from 0.922 to the requested level of 0.950,
involving the addition of 35 secondary configurations as reference
functions, and a final c**2 value of 0.957. The following analysis
appears on completion of these iterations.

{
\footnotesize 
\begin{verbatim}
        =====================================================================
        == State    c**2       Energy          Energy          Davidson    ==
        ==                   (T=  3, a.u.)   (T=0, a.u.)        (a.u.)     ==
        =====================================================================
        ==  1      0.957      -78.200915      -78.207284      -78.211994   ==
        =====================================================================
\end{verbatim}
}
Note that merely presenting the data line:

{
\footnotesize
\begin{verbatim}
          ITERATE
\end{verbatim}
}
would lead to refinement of the ground state wavefunction to the point
where all secondary coefficients with c**2 greater than 0.005 are included
in the reference state, at which point iterations would cease.

\subsection{The Algorithm for Controlling Multi-root Calculations}

The ITERATE sub-directives described above provide for control over the
iterative treatment of single CI wavefunctions. This however is not
the main purpose of ITERATE; controlling the treatment of multi root
calculations with a view to the automatic handling of e.g. excitation
spectrum requires a far greater level of control that is provided by
a number of sub-directives  - MAXROOT, SROOT, DROOT and RETAIN  - that
are described below.

Before considering these directives in detail, we first outline the
algorithm that is used in controlling what is a relatively complex
procedure. This has involved a number of prototyping exercises that
have culminated in a final design criteria dominated by the desire to
make the usage and specification as simple as possible, while providing
the necessary level of robustness to deliver the required solution in a
reasonable number of iterations. Users who have driven the MRDCI module
in the search for excited states will be aware of the typical sequence
of calculations that are performed manually. This involves starting
with a number of reference configurations, and obtaining one or more CI
wavefunctions under control of the ROOTS directive whereby selection is
carried out with respect to NROOT roots of the `root' secular problem to
be used in process of configuration selection. Typically the same number
of roots are then generated in the final secular problem. The user will
then augment the reference set and increase the value of NROOT based
on an examination of earlier calculations until the desired number of
eigenstates are obtained.

The current implementation is an attempt to automate the above process.
Initially the user must specify certain of the SELECT data fields;
the fundamental decision taken here was that he/she should  NOT have
to specify any CONF data i.e. any explicit configuration date in the
entire process. The consequences of this decision are twofold:
\begin{enumerate}
\item the following SELECT directives - CNTRL, SYMMETRY, SPIN  -
must be presented, together typically with THRESH and ROOTS; 
\item the user is responsible for ensuring that the set of input orbitals
(restored under control of the VECTORS directive) are consistent with
the specifications given under SYMMETRY and SPIN.  We shall clarify this
requirement below; note that the code will check for this consistency
and abort if it is not obeyed.
\end{enumerate}
The initial calculation will be carried out using an internally
constructed set of main configurations from which NROOT states will be
generated (as specified by the ROOTS directive. Experience suggests that
setting NROOT to 5 is generally quite reasonable.  Subsequent calculations
will iteratively generate a number of higher states as specified under
control of the MAXROOT, SROOT, DROOT and RETAIN directives.


\subsubsection{The MAXROOT Directive}

This directive may be used to specify the number of states to be
obtained from the sequence of iterative MRDCI calculations.
The directive consists of two data fields, read to the variables TEXT,
MXSTATE  using format (A,I);
\begin{itemize}
\item TEXT should be set to the character string MAXROOT;
\item MXSTATE is an integer used to specify the maximum number of
eigenstates to be generated from the MRDCI treatment.
\end{itemize}
The directive may be omitted when MXSTATE will be set to the default 
value of 8.\\

{\bf Example}

{
\footnotesize 
\begin{verbatim}
          ITERATE MAXROOT 12
\end{verbatim}
}
Thus the iterative procedure will, from an initial point of generating
NROOT roots of the secular problem, attempt to generate roots (NROOT+1)
up to and including root MAXROOT.

\subsubsection{The SROOT Directive}

This directive may be used to establish the criterion whereby the number
of roots to be used in selection is increased from that in effect (NROOTS)
during the preceding MRDCI iteration.  The directive consists of two
data fields, read to the variables TEXT, ROOTDEL using format (A,F);
\begin{itemize}
\item TEXT should be set to the character string SROOT;
\item Selection will be extended from the NROOTS of the zero-order
problem to (NROOTS+1) if the difference in the zero-order eigenvalues,
\begin{equation}
	ABS[EIGVAL(NROOTS+1) - EIGVAL(NROOTS)] \leq  ROOTDEL
\end{equation}
\end{itemize}
The directive may be omitted when ROOTDEL will be set to the default
value of 0.0 i.e. no increase in NROOTS will be undertaken during the
MRDCI iteration process.\\

{\bf Example}

{
\footnotesize 
\begin{verbatim}
          SROOT 0.20
\end{verbatim}
}

\subsubsection{The DROOT Directive}

This directive should be used to confirm that the number of eigenstates
derived from the diagonalisation process is to be increased during
the iterative MRDCI cycles until the number specified by the MAXROOT
directive has been derived, at which point the iterative process will
terminate. The DROOT directive will typically appear together with the
SROOT directive in controlling the energetics of this process; clearly
the user may have little interest in deriving the final CI vector of an
eigenstate whose zero-order description is separated by a large energy
gap from the zero-order vector of the preceding state. Rejecting such
a solution, and terminating the MRDCI process, is controlled by the
value specified by SROOT.

{\bf Example}

{
\footnotesize 
\begin{verbatim}
          SROOT 0.20 DROOT
\end{verbatim}
}
Given the above data sequence, the iterative MRDCI process will continue
until either, (a) MXSTATE eigenvectors of the CI matrix have been
obtained, or (b) having obtained N roots of the CI eigen vector, the
(N+1)th root of the zero-order problem lies more than 0.20 a.u. above
the Nth root.

The following points should be noted:
\begin{itemize}
\item The strategy behind the iterative sequence follows from an
appreciation of the quantities defined by the MAXROOT, SROOT and DROOT
directives. In practice  three criteria are used in deciding
whether to continue the sequence of MRDCI calculations:
\begin{enumerate}
\item has the number of main configurations requested using WEIGHT
changed from the preceding pass.
\item does the eigen value spectra of the zero order problem 
justify expanding the no. of roots, NROOTS, to be used in selection?\\
i.e ABS[EIGVAL(NROOTS+1)-EIGVAL(NROOTS)]  $\le$ ROOTDEL.
\item based on 2., should we increase the no. of eigenstates (roots)
to be extracted from the Davidson procedure (requested by DROOT).
\end{enumerate}
Note that these criteria are executed  {\em such that 3. is not invoked
before 2. is satisfied, and 2. is not invoked before 1. is satisfied}.
\item This approach is designed to ensure approximate convergence in the
structure of the lower eigenstates before attempting to extract
solutions for higher states in the spectrum.
\end{itemize}

\subsubsection{The RETAIN Directive}

This is the most complex of the ITERATE directives in terms of
appreciating the role it plays in ensuring that all the states of
interest are obtained by the iterative procedure outlined above. The key
to obtaining these states is that at some point the associated leading
configuration is identified as a reference species and is incorporated as
a main configuration in the zero-order space. While the process outlined
above has been found to be extremely effective when all eigenstates
are of similar character (e.g. Rydberg States), this is not always
true when the derived eigenstates are of quite different character
e.g. valence and Rydberg states. In such cases it is quite common for
such a valence configuration not be included in the generated set of
reference configurations at the outset, and while low-lying in energy,
to have an extremely low value of c**2 in, for example, each of the
derived Rydberg states i.e. it will never be incorporated in the set of
main configurations, and hence never appear as one of the derived states.

A key to this behavior can be found from examining the nature of the
configurations tagged as 'r' ('retain') in the output from the selection
module. Initial attempts to implement an ITERATE strategy revealed
that such configurations often remain isolated from the selected set of
reference configurations throughout the iterative cycles, even though
their associated energy remains lower than those of the final eigenstates
of some of the higher CI states.

The RETAIN  directive may be used to establish the criterion whereby the
appearance of configurations taged 'r' will trigger their inclusion as
reference functions in the subsequent MRDCI iterations, regardless of
their computed weights in any of the current eigenstates.  The directive
consists of three data fields, read to the variables TEXT, ERETAIN,
IRETAIN using format (A,F,I);
\begin{itemize}
\item TEXT should be set to the character string RETAIN;
\item Those configurations that are tagged 'r' in the selection
process will be retained as reference configurations in the
next iterative cycle if their associated energy lies within
ERETAIN au of the energy of the NROOTS root of the 
current zero-order eigen problem. i.e.
\begin{equation}
	E(configuration) - EIGVAL(NROOTS) \leq  ERETAIN
\end{equation}
\item IRETAIN may be used to delay the onset of this selection
criteria. Specifying IRETAIN results in the criteria only coming into
effect on iteration IRETAIN of the MRDCI iterative process.
\end{itemize}

{\bf Example}

{
\footnotesize 
\begin{verbatim}
          RETAIN 0.10 
\end{verbatim}
}

The following points should be noted:
\begin{itemize}
\item The RETAIN directive may be omitted, when ERETAIN will
assume a default value of 0.10 au, and be used in
selecting configurations as reference functions from the outset.
\item A maximum of 30 configurations may be retained on a given iteration.
\item Setting IRETAIN to a high value e.g. 100 provides a potential
mechanism for the iterative process to focus on just a sub-set of the
final CI eigen states.
\end{itemize}

\subsection{Examples of Excited State Generation}

To clarify our discussion of iterative processing using the Semi-direct
Table-CI module, we work through a number of example of using the method
in calculating the energetics and properties of the lying states of a
variety of molecules of increasing complexity.

\subsubsection[Calculating the \astate\ states of Formaldehyde]{Calculating the \astate\ states of Formaldehyde}

Data for performing an iterative MRDCI calculation on the ten lowest
\astate states of formaldehyde is given below. The calculation is
initiated with an SCF calculation on the ground state.  The following
points should be noted:

\begin{enumerate}
\item The C and O 1s inner shell MOs, together with the two highest
virtual orbitals are excluded from the calculation under control of the
CORE and ACTIVE directives.
\item The SELECT directives requesting the 12 electron singlet CI states of 
A$_{1}$ symmetry are as follows:

{
\footnotesize
\begin{verbatim}
          SYMMETRY 1
          SPIN 1  
          CNTRL 12
\end{verbatim}
}
\item The ITERATE directive

{
\footnotesize
\begin{verbatim}
          ITERATE MAXI 20 MAXROOT 10 SROOT 0.10 DROOT WEIGHT 0.005
\end{verbatim}
}
requests derivation of the lowest 10 roots, allowing a maximum of 20
MRDCI iterations to derive these roots. The initial CI will be based
on the lowest 5 roots of the CI Hamiltonian (ROOTS 5) derived from the
default set of generated main configurations.
\end{enumerate}
{

\footnotesize 
\begin{verbatim}
          CORE 20000000
          TITLE
          H2CO - TZVP+D(SPD) - MRDCI TREATMENT OF THE 1A1 STATES
          HARMONIC
          ZMAT ANGSTROM
          C
          O 1 1.203 
          H 1 1.099 2 121.8 
          H 1 1.099 2 121.8 3 180.0 
          END
          BASIS
          TZVP O
          TZVP C
          TZVP H
          S O
          1.0 0.02
          P O
          1.0 0.02
          D O
          1.0 0.02
          END
          CORE
          1 2 END
          ACTIVE
          3 TO 57 END
          RUNTYPE CI
          MRDCI DIRECT
          SYMMETRY 1
          SPIN 1
          CNTRL 12
          THRESH 5 5
          ROOTS 5
          ITERATE MAXI 20 SROOT 0.10 MAXROOT 10 DROOT WEIGHT 0.005 
          NATORB BYPASS
          ENTER
\end{verbatim}
}


\subsubsection[Calculating the Excited States of the Formyl Radical]{Calculating the Excited states of the Formyl Radical}

\subsubsection[The \apstate States]{The \apstate States}

Data for performing an iterative MRDCI calculation on the eight lowest 2A'
states of the formyl radical is given below. The calculation is initiated
with an SCF calculation on the 2A' ground state (under control of the
OPEN directive):

{
\footnotesize 
\begin{verbatim}
          CORE 20000000
          TIME 300
          TITLE   
          HCO DZP + BOND(SP)
          HARMONIC
          MULT 2  
          ZMAT    
          C
          BQ 1 RCO2
          X 2 1.0 1 90.0
          O 2 RCO2 3 90.0 1 180.0 
          X 1 1.0 2 90.0 3 0.0
          H 1 RCH 5 40.0 4 180.0 
          VARIABLES
          RCO2 1.125
          RCH 2.076
          END
          BASIS   
          DZP H   
          S BQ    
          1.0 0.02
          P BQ    
          1.0 0.02
          D BQ    
          1.0 0.02
          DZP C   
          DZP O   
          END
          OPEN 1 1
          CORE    
          1 2 END 
          ACTIVE  
          3 TO 42 END
          RUNTYPE CI
          MRDCI DIRECT
          SYMMETRY 1
          SPIN 2  
          CNTRL 11
          THRESH 2 2
          ROOTS 5 
          ITERATE MAXI 20 MAXROOT 8 SROOT 0.30 DROOT 
          NATORB BYPASS
          ENTER 
\end{verbatim}
}
The following points should be noted:
\begin{enumerate}
\item The C and O 1s inner shell MOs, together with the two highest
virtual orbitals are excluded from the calculation under control of the
CORE and ACTIVE directives.
\item The SELECT directives requesting the 11 electron doublet CI states of 
A' symmetry are as follows:

{
\footnotesize
\begin{verbatim}
          SYMMETRY 1
          SPIN 2  
          CNTRL 11
\end{verbatim}
}

\item The ITERATE directive

{
\footnotesize
\begin{verbatim}
          ITERATE MAXI 20 MAXROOT 8 SROOT 0.30 DROOT 
\end{verbatim}
}

requests derivation of the lowest 8 roots, allowing a maximum of 20
MRDCI iterations to derive these roots. The initial CI will be based
on the lowest 5 roots of the CI Hamiltonian (ROOTS 5) derived from the
default set of generated reference configurations.
\end{enumerate}

\subsubsection[The \appstate States]{The \appstate States}

Treating the 2A'' states is somewhat more complex, given the need
to perform an additional SCF calculation on the lowest state of that
symmetry. We now perform an initial SCF calculation on the ground state,
and initiate the CI calculation in the second step by restoring these
MOs, interchanging the appropriate orbitals under control of the SWAP
directive, and conducting the SCF prior to the CI.

{
\footnotesize 
\begin{verbatim}
          TITLE
          HCO DZP + BOND(SP) 2AP SCF 
          HARMONIC
          MULT 2
          ZMAT
          C
          BQ 1 RCO2
          X 2 1.0 1 90.0
          O 2 RCO2 3 90.0 1 180.0 
          X 1 1.0 2 90.0 3 0.0
          H 1 RCH 5 40.0 4 180.0 
          VARIABLES
          RCO2 1.125
          RCH 2.076
          END
          BASIS
          DZP H
          S BQ
          1.0 0.02
          P BQ
          1.0 0.02
          D BQ
          1.0 0.02
          DZP C
          DZP O
          END
          OPEN 1 1
          ENTER
\end{verbatim}
}

{
\footnotesize 
\begin{verbatim}
          CORE 20000000
          TIME 300
          RESTART NEW
          TITLE
          HCO DZP + BOND(SP) 2APP SCF + CI
          HARMONIC
          MULT 2
          ZMAT
          C
          BQ 1 RCO2
          X 2 1.0 1 90.0
          O 2 RCO2 3 90.0 1 180.0 
          X 1 1.0 2 90.0 3 0.0
          H 1 RCH 5 40.0 4 180.0 
          VARIABLES
          RCO2 1.125
          RCH 2.076
          END
          BASIS
          DZP H
          S BQ
          1.0 0.02
          P BQ
          1.0 0.02
          D BQ
          1.0 0.02
          DZP C
          DZP O
          END
          OPEN 1 1
          CORE
          1 2 END
          ACTIVE
          3 TO 42 END
          RUNTYPE CI
          MRDCI DIRECT
          SYMMETRY 2
          SPIN 2
          CNTRL 11
          THRESH 2 2
          ROOTS 5
          ITERATE MAXI 20 MAXROOT 8 SROOT 0.30 DROOT 
          NATORB BYPASS
          VECTORS 5
          SWAP
          8 10
          END
          ENTER 4 5
\end{verbatim}
}

\begin{thebibliography}{10}

\bibitem{ref:18} R.J. Buenker in `Proc. of the Workshop on Quantum Chemistry and
Molecular Physics', Wollongong, Australia (1980);
R.J. Buenker in `Studies in Physical and Theoretical Chemistry',
21 (1982) 17.

\end{thebibliography}

\end{document}
