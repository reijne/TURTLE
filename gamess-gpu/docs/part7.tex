\documentclass[11pt,fleqn]{article}

\usepackage{hyperref}

% package HTML requires Latex2HTML to be installed for html.sty
\usepackage{html}
\newcommand{\doi}[1]{doi:\href{http://dx.doi.org/#1}{#1}}
\begin{htmlonly}
\renewcommand{\href}[2]{\htmladdnormallink{#2}{#1}}
\end{htmlonly}
\hypersetup{colorlinks,
            %citecolor=black,
            %filecolor=black,
            %linkcolor=black,
            %urlcolor=black,
            bookmarksopen=true,
            pdftex}
 
\addtolength{\textwidth}{1.0in}
\addtolength{\oddsidemargin}{-0.5in}
\addtolength{\topmargin}{-0.5in}
\addtolength{\textheight}{1.0in}
\newcommand{\yvec}{\mbox{\boldmath $y$}}
\newcommand{\zvec}{\mbox{\boldmath $z$}}

\pagestyle{headings}
\pagenumbering{roman}
\begin{document}
\sf
\parindent 0cm
\parskip 1ex
\begin{flushleft}
 
Computing for Science (CFS) Ltd.,\\CCLRC Daresbury Laboratory.\\[0.30in]
{\large Generalised Atomic and Molecular Electronic Structure System }\\[.2in]
\rule{150mm}{3mm}\\
\vspace{.2in}
{\huge G~A~M~E~S~S~-~U~K}\\[.3in]
{\huge USER'S GUIDE~~and}\\[.2in]
{\huge REFERENCE MANUAL}\\[0.2in]
{\huge Version 8.0~~~June 2008}\\ [.2in]
{\large PART 7. RPA and MCLR CALCULATIONS}\\
\vspace{.1in}
{\large M.F. Guest, J. Kendrick, J.H. van Lenthe and P. Sherwood}\\[0.2in]
 
Copyright (c) 1993-2008 Computing for Science Ltd.\\[.1in]
This document may be freely reproduced provided that it is reproduced\\
unaltered and in its entirety.\\
\vspace{.2in}
\rule{150mm}{3mm}\\
\end{flushleft}

% 
\tableofcontents
\newpage

\pagenumbering{arabic}

\section[Introduction]{Introduction}


Under the common runtype RESPONSE, the user may perform calculations
of electronic transition energies and corresponding oscillator strengths,
using either the Random Phase Approximation (RPA) method or the
Multiconfigurational Linear Response (MCLR) procedure. The RPA calculations
may be performed either within the conventional approach where the
two--electron integrals are transformed or with a ``direct'' 
implementation. The next sections describe in detail how to perform
calculations with the linear response module.
%
\section{RPA and TDA calculations}

Calculations of excitation energies and oscillator strengths
based on the Random Phase Approximation (RPA) are initiated
by specifying the data line

{
\footnotesize
\begin{verbatim}
          RUNTYPE RESPONSE RPA
\end{verbatim}
}
in the input file. Data input characterising the details of 
the calculation is presented immediately after the RUNTYPE data 
line. Termination of this data is accomplished by presenting
a valid {\em Class 2} directive, such as VECTORS.

RUNTYPE RESPONSE is in fact a combination of tasks, requesting
integral generation, SCF, integral transformation (in conventional
RPA calculations) and, finally, the response calculation itself. 
While in simple cases it may be feasible to perform all steps 
in a single calculation, it will often be necessary to break up the 
calculation into multiple jobs, driving through each of the tasks under
control of the appropriate RUNTYPE directive, with use made
of the BYPASS directive in the latter stages of the computation.

The data input required when performing an RPA calculation 
in a single job is shown schematically below:

{
\footnotesize
\begin{verbatim}
          TITLE
          H2CO - TZVP + R(SP) BASIS - RPA CALCULATION
          SUPER OFF NOSYM
           .
           .
          RUNTYPE RESPONSE RPA
           .
           .
           RPA Data
           .
          ENTER
\end{verbatim}
}
When splitting the RPA calculation into multiple steps, we will be 
involved in performing the following tasks:
\begin{itemize}
\item A closed-shell SCF calculation requesting, through use
of the SUPER directive,  full integral list generation required 
in the subsequent transformation.  Note that in most cases this 
part of the calculation may also be conveniently broken up 
into two parts, the first being a normal SCF where the integrals are 
generated in supermatrix form, and the second being a restart job 
with the "SUPER OFF NOSYM" data line.  Dumpfile (ED3) and 
Mainfile (ED2) from the SCF calculation must be kept.

{
\footnotesize
\begin{verbatim}
          TITLE
          H2CO - TZVP + R(SP) BASIS - SCF PRIOR TO RPA CALCULATION
          SUPER OFF NOSYM
           .
           .
           .
          ENTER
\end{verbatim}
}

\item An integral transformation, using the MOs from the
SCF job as input vectors. The Dumpfile (ED3) and the transformed
integral file (ED6) must be kept.

{
\footnotesize
\begin{verbatim}
          RESTART
          TITLE
          H2CO - TZVP + R(SP) BASIS - INTEGRAL TRANSFORMATION
          SUPER OFF NOSYM
          BYPASS SCF
            .
          RUNTYPE TRANSFORM
            .
          ENTER
\end{verbatim}
}
Note that the SCF computation is BYPASS'ed, with the SCF vectors 
from the first run now restored from the default closed shell
SCF eigenvector section (section 1) and used in the transformation.

\item The final RPA job must be declared as a restart job, and
BYPASS's the integral transformation.

{
\footnotesize
\begin{verbatim}
          RESTART
          TITLE
          H2CO - TZVP + R(SP) BASIS - CONVENTIONAL RPA CALCULATION
          SUPER OFF NOSYM
          BYPASS TRANSFORM
            .
          RUNTYPE RESPONSE RPA
            .
            RPA Data
            .
          ENTER
\end{verbatim}
}
The RPA input data is terminated by the ENTER directive, where the
default SCF eigenvectors section is again in the absence of
section specification.

\end{itemize}
In the following sections the directives controlling the RPA 
calculation are described. 


\subsection{RPA Data - SYMM}

The computation of excitation energies and corresponding oscillator
strengths is initiated by the
SYMM directive comprising the variables TEXT1, ISYM, ILOW,
TEXT2, IHIGH using format (A,2I,A,I).
\begin{itemize}
\item TEXT1 should be set to the character string SYMM.
\item TEXT2 should be set to the character string TO.
\end{itemize}
The excited states no. ILOW to IHIGH
of the irreducible representation ISYM are then computed.
\subsubsection{SYMM Example}
Calculation of the excitation energies for the lowest five states of each of
the optically allowed symmetries B$_{1u}$, B$_{2u}$, B$_{3u}$ of a molecule
with D$_{2h}$ symmetry requires the data lines

{
\footnotesize
\begin{verbatim}
          SYMM 2 1 TO 5
          SYMM 3 1 TO 5
          SYMM 5 1 TO 5
\end{verbatim}
}
in the input.
%
\subsection{RPA Data - TDA}
By presenting the data line
{
\footnotesize
\begin{verbatim}
          TDA
\end{verbatim}
}
an additional Tamm-Dancoff (TDA) calculation is performed for the
specified irreps and roots, corresponding to a CI in the space of
single excitations. The line

{
\footnotesize
\begin{verbatim}
          TDA ONLY
\end{verbatim}
}
can be used to suppress the RPA calculation, performing a TDA
calculation only.
%
\subsection{RPA Data - POLA}
This directive initiates the computation of dynamic polarisabilities within
the time-dependent Hartree-Fock model. Each data line should begin with the
character string POLA, followed by one or several (up to 10) floating point
numbers, representing the frequencies (in atomic units) for which the
polarisabilities are requested. If more than 10 frequencies are required,
a new data line must be presented.
%
\subsection{RPA Data - MAXRED}
The maximal size of the reduced matrices in the iterative RPA algorithm
can be adjusted with the MAXRED directive, consisting of a single dataline
read to the variables TEXT, MAXR, MAXRR using format (A,2I).
\begin{itemize}
\item TEXT should be set to the character string MAXRED.
\item MAXR is an integer, specifying the maximal size of the reduced matrices
      in the iterative eigenvalue search. The default value of MAXR is 50.
\item MAXRR is an integer, specifying the maximal size of the reduced matrices
      during solution of the linear response equations for polarisability
      calculations. The default value of MAXRR is 50. Within the MAXRED
      directive, the specification of MAXRR is optional.
\end{itemize}
Note that there is no danger in specifying a value of MAXR (or MAXRR) which
is too large with respect to the memory available. The program automatically
adjusts the value of MAXR if the iterative procedure consumes too much
memory. If the user, however, absolutely insists on the desired value 
for MAXR, she/he
should increase the available memory using the MEMORY predirective.
The value of MAXR should at least be three times as large
as the largest number of eigenvalues wanted for one specific symmetry,
except for very large calculations. The following table gives recommended
values for MAXR:
\begin{center}
\begin{tabular}{|c|c|}
Number of eigenvalues desired & Recommended value of MAXR \\ \hline
10         &  50 - 80    \\
20         &  80 - 100    \\
30         &  90 - 120    \\
50         &  150 - 200    \\
100        &  200 - 300    \\
200        &  400                                         \\ \hline
\end{tabular}
\end{center}
If the iterative algorithm reaches the maximum dimension before convergence,
it restarts the procedure, using the current approximate eigenvectors
as starting vectors.
%
\subsection{RPA Data - MAXIT}
This directive sets the maximal number of iterations for the iterative
algorithms in the RPA module. The data line

{
\footnotesize
\begin{verbatim}
          MAXIT 50 10
\end{verbatim}
}
sets the maximum number of cycles to 50 for the eigenvalue problem and to
10 for the linear equation solver. The default value is 30 for both 
algorithms. As for MAXRED, specification of the second integer is
optional within the MAXIT directive.
%
\subsection{RPA Data - ANALYSE}
The result table printed after successful completion of the iterative
TDA/RPA procedure contains the most important one-electron excitations
of the corresponding states. If $(\yvec,\zvec)$ denotes an RPA eigenvector,
then all components of the vector $\yvec - \zvec$ with modulus larger than a
certain threshold which may be specified in the THRESH directive
(see below) are listed in this table. With the help of the ANALYSE
directive which consists of the single data line ANALYSE, the user may,
in addition, examine smaller components, without having them listed in
the result table. The corresponding threshold is again specified by means
of the THRESH directive (see below). The additional output generated
by the ANALYSE directive also contains the dipole integrals, useful
for an investigation which mono-excitations contribute to a large
oscillator strength, and the weights of the vectors $\yvec$ and $\zvec$
in the RPA eigenvectors $(\yvec,\zvec)$.
%
\subsection{RPA Data - THRESH}
It is possible to define various thresholds to control the convergence
and output. The first data line of the THRESH directive should be set to
the character string

{
\footnotesize
\begin{verbatim}
          THRESH
\end{verbatim}
}
Each following line should begin with one of the character strings
{
\footnotesize
\begin{verbatim}
          EIGEN, EQSYS, TABLE, ANALYSIS
\end{verbatim}
}
followed by a floating point number, using format (A,F). The last data line
should be set to the character string

{
\footnotesize
\begin{verbatim}
          END
\end{verbatim}
}
The numbers following the keywords EIGEN, EQSYS define the thresholds to which
the residual in the eigenvalue algorithm and equation system solver, 
respectively, is converged. The default value is 0.001 for both algorithms.
The data line

{
\footnotesize
\begin{verbatim}
          TABLE 0.1
\end{verbatim}
}
causes printing of all eigenvector components with modulus larger than or
equal to 0.1 in the result tables for TDA and RPA. The default value is 0.2.
Finally, the data line

{
\footnotesize
\begin{verbatim}
          ANALYSIS 0.01
\end{verbatim}
}
causes printing of all eigenvector components $c_i$ with $c^2_i\geq 0.01$,
provided a detailed analysis of the eigenvectors is requested by means of
the ANALYSE directive.
%
\subsection{RPA Data - BEGIN}
In the result table of an RPA calculation, the leading components of the
eigenvectors are listed, attached with symmetry labels of the orbitals
involved in the corresponding one-electron excitation, e.g.
$$ 0.85\;(2\,a_1\rightarrow 1\,b_2) $$
Usually the labeling starts with the first MO, i.e. the one with the lowest
energy. Now, if there is a core of orbitals from which virtually no
excitations are expected in the lowest excited states, the user may wish to
start labeling the orbitals at the first valence MO. This is accomplished
with the BEGIN directive which consists of a single data line, read to
variables TEXT, NBEGIN using format (A,I). 
\begin{itemize}
\item TEXT should be set to the character string BEGIN.
\item NBEGIN is an integer specifying the number of the first MO to be labeled.
\end{itemize}
%
\subsubsection{BEGIN Example}
In an all-electron calculation on MgNa$_6$, there are 70 electrons in
doubly occupied core orbitals. The first valence MO thus carries the
number 36, and the data line

{
\footnotesize
\begin{verbatim}
          BEGIN 36
\end{verbatim}
}
starts the orbital counting with this MO.
%
\subsection{RPA Data - QZ}
Usually the RPA calculation is carried out for the ground state of a molecule.
In this case the symmetric RPA matrix is positive definite, and the projected
generalized eigenvalue problems can be reduced to ordinary eigenvalue
problems. This is the default strategy of the iterative algorithm.
If, however, the user wishes to perform a calculation on a state different
from the ground state, or if the RPA matrix is extremely ill-conditioned,
she/he can resort to the QZ algorithm of Moler and Stewart which treats
generalized indefinite eigenvalue problems. This path detects and discards
complex eigenvalues. It is initiated by presentation of the single data line

{
\footnotesize
\begin{verbatim}
          QZ
\end{verbatim}
}
Note that the QZ algorithm is considerably slower than the standard method
for positive definite matrices.
The occurrence of complex eigenvalues during the standard reduction is usually
indicated by the message

{
\footnotesize
\begin{verbatim}
          PROBLEM IN ITERATION STEP 1: ERED2 IS NOT POSITIVE DEFINITE
\end{verbatim}
}
which terminates the RPA procedure for that particular symmetry. In this
case the user should first check if she/he is really calculating on the
Hartree-Fock ground state.
Note that the QZ algorithm is the default in Direct RPA calculations.
%
\subsection{Auxiliary files}
The RPA program automatically generates a file named {\tt rpa\_spectrum}
(and/or {\tt tda\_spectrum}, if a TDA calculation is performed),
which contains a table of the excitation energies (in a.u.) and
corresponding oscillator strengths. This file may serve as input for
a suitable plotting program.
If the user wishes to keep this
file for plotting, she/he should place a corresponding command line
at the end of the shell script file which moves {\tt rpa\_spectrum} to the user's
permanent directory.

The files {\tt tda\_table.tex} and {\tt rpa\_table.tex} contain the
\LaTeX\ input for a list of the excited states computed with TDA and RPA,
respectively, with excitation energies, oscillator strengths and
most important single excitations.

If a calculation is performed on the states of symmetry $i\;
(1\leq i\leq 8)$, a file {\tt tm4$i$} is generated
which contains the input for a plot of the RPA analogue of the 
transition density matrix.

Finally, we list the complete input data for an RPA plus polarisability 
calculation on the water molecule, using a 4-31G basis:
%
\subsubsection{SCF calculation}
{
\footnotesize
\begin{verbatim}
          TITLE
          H2O 4-31G BASIS 
          SUPER OFF NOSYM
          ZMAT
          O
          H 1 1.809 
          H 1 1.809 2 104.52 
          END
          BASIS 4-31G
          ENTER
\end{verbatim}
}
%
\subsubsection{Integral transformation}
{
\footnotesize
\begin{verbatim}
          RESTART
          TITLE
          H2O 4-31G BASIS + INTEGRAL TRANSFORMATION
          BYPASS SCF
          SUPER OFF NOSYM
          ZMAT
          O
          H 1 1.809 
          H 1 1.809 2 104.52 
          END
          BASIS 4-31G
          RUNTYPE TRANSFORM
          ENTER
\end{verbatim}
}
%
\subsubsection{RPA calculation}
{
\footnotesize
\begin{verbatim}
          RESTART
          TITLE
          H2O 4-31G BASIS RPA CALCULATION
          BYPASS TRANSFORM
          SUPER OFF NOSYM
          ZMAT
          O
          H 1 1.809 
          H 1 1.809 2 104.52 
          END
          BASIS SV 4-31G
          RUNTYPE RESPONSE RPA
          TDA
          SYMM 1 1 TO 5
          SYMM 2 1 TO 5
          SYMM 3 1 TO 2
          MAXRED 25
          MAXIT 20
          ANALYSE
          THRESH
           EIGEN    0.001
           EQSYS    0.0001
           TABLE    0.25
           ANALYSIS 0.05
          END
          POLA 0.0 0.1 0.2
          ENTER
\end{verbatim}
}
%
\section{Direct RPA calculations}

For large atomic orbital basis sets, the integral transformation step
in conventional RPA calculations can become prohibitive. In this case
it is possible to resort to a ``Direct SCF'' like implementation of the
RPA procedure which breaks up the four--index transformation into
two two--index transformation whenever the RPA matrix acts on a trial
vector. The Direct RPA module is started by the input lines

{
\footnotesize
\begin{verbatim}
          RUNTYPE RESPONSE RPA DIRECT
\end{verbatim}
}
in the input file. In this case the only preparatory run is a 
closed shell SCF calculation which may be conventional or direct,
and in which the integrals may be generated in supermatrix or {\tt 2E}
format. Only the Dumpfile of the SCF calculation must be kept.
All directives that are available for conventional RPA calculations
can also be used for the Direct RPA case, with two exceptions:
The polarisability module has not yet been implemented, and the 
QZ directive is redundant ({\em vide supra}).
The following additional directives are available:
%
\subsection{PREF}
This directive controls the prefactor tolerance for the integral
generator. It consists of a single data line with variables
TEXT, EXP using format (A,I), where TEXT is set to the character string
PREF and EXP is a positive integer, setting the prefactor tolerance
to $10^{-\mbox{\scriptsize\sf EXP}}$. The default value for EXP is 7.
%
\subsection{MAXVEC}
The direct RPA procedure is organized in such a way that a maximal number
of trial vectors, with respect to the main memory available,
is contracted with the integrals generated in one batch.
It may, however, sometimes be necessary to reduce this number since
valuable memory is needed for other purposes, e.g., to increase the
maximal size of the reduced matrices during the iterations. The
MAXVEC directive allows to limit the number of trial vectors which are
treated in one batch to a specific value M.
It consists of a single data line
with the variables TEXT, M using format (A,I), 
where TEXT is set to the character string
MAXVEC and M is the above mentioned integer.
%
\subsection{Dumping and restoring trial vectors}
Since Direct RPA calculations on larger systems are rather 
time-consuming, it is desirable to have the possibility to interrupt
a calculation and restart it at a later time.
Dumping and restoring intermediate results is possible with the DUMP
and RESTORE directives.
The user may dump vectors by specifying the data line
\begin{flushleft}
\mbox{\hphantom{{\tt ABCDEFGHIJ}}}{\tt DUMP} {\em file}
\end{flushleft}
where {\em file} is (the complete path of) a file in some permanent
directory that may be chosen freely. In case the RPA (or TDA)
calculation aborts in the middle of the iterative process
(either by runtime problems or by a user's operation), this file
contains the approximate
eigenvectors of the last iteration step that was completed.
These vectors may be used in a later job to resume the calculation
at that iteration step. This is accomplished by the data line
\begin{flushleft}
\mbox{\hphantom{{\tt ABCDEFGHIJ}}}{\tt RESTORE RPA} {\em file}
\end{flushleft}
where {\em file} is the above mentioned file containing the saved vectors.
Note that the keyword {\tt RPA} has to be replaced by {\tt TDA} if
the job was interrupted during a TDA calculation.
Note also that the restart run must begin with the particular irrep
in which the termination occurred, and that the
same number of roots must be specified.
%
\section{Multiconfigurational linear response calculations}
A multiconfigurational linear response (MCLR) calculation
\cite{ref:46}, also known under
the term time--dependent multiconfigurational SCF, is performed by
presenting the data line

{
\footnotesize
\begin{verbatim}
          RUNTYPE RESPONSE MCLR
\end{verbatim}
}
in the input file. A necessary condition for performing an MCLR
calculation is the successful completion of a corresponding
multiconfigurational SCF calculation with the GAMESS--UK MCSCF module.
The Dumpfile (ED3) and the transformed integral
file (ED6) must be saved. The MCLR calculation is then performed
as a restart job. In the following we list the directives which are
available in the MCLR module. Note that the first four are obligatory.
%
\subsection{MCLR Data - ORBITAL}
The set of active orbitals must be specified by means of the ORBITAL
directive. The individual lines of this directive must be identical 
to those presented in the preceding MCSCF calculation.
Since this directive is described in detail in the
MCSCF part (and is usually copied into the MCLR job
from the MCSCF job), we refer the reader to the corresponding
section.
%
\subsection{MCLR Data - SECTIONS}
In order to perform an MCLR calculation, several vectors have to be
retrieved from the Dumpfile. The SECTIONS directive specifies in which
sections of the Dumpfile the corresponding vectors are stored. Note that
these sections must still be specified even if the default sections have
been used at vector generation time (in the absence of explicit section
specification on the corresponding ENTER directive).

The data lines

{
\footnotesize
\begin{verbatim}
          SECTIONS
             SCF        1
             MCSCF      8
             CANONICAL  10
             CIVEC      9
\end{verbatim}
}
instruct the program to read the SCF eigenvectors from section 1,
the MCSCF MOs from section 8, the pseudocanonical MCSCF orbitals from
section 10 and the MCSCF CI vector from section 9 of the Dumpfile. Note
that with the possible exception of CANONICAL (as specified by the CANONICAL MCSCF
directive), these section numbers correspond to the defaults in place
at vector generation time in the closed-shell SCF and MCSCF module.
Note that the default section for the MCSCF NOs is now section 10, and
it is this section number that should be specified in the data above.
The indentation is, of course, not necessary but convenient for better
reading.

%
\subsection{MCLR Data - SYMM}
This directive controls the calculation of excited states. 
It consists of the variables TEXT, ISYM, IHIGH using format (A,2I).
TEXT should be set to the character string SYMM. The integer ISYM
indicates the irreducible representation, and IHIGH is the number 
of roots to be calculated in the irrep ISYM. Note that the syntax
of this directive is different from the corresponding directive
in the RPA module. In particular, it is not possible to calculate
an interval [ILOW,IHIGH] of roots with ILOW different from 1.
%
\subsection{MCLR Data - END}
This directive which consists of the single keyword END terminates
the input which controls the MCLR calculation. It must always be
present.
%
\subsection{MCLR Data - REDUCE}
The default algorithm for solving the small generalized eigenvalue
problems during the iterative MCLR calculation is the QZ algorithm
(see the description of the QZ directive in the RPA module).
The REDUCE directive which consists of a single data line with the
keyword REDUCE forces the program to reduce the generalized
eigenvalue problems to standard eigenvalue problems.
%
\subsection{MCLR Data - SPLIT}
This directive initiates the use of split trial vectors, as described
in \cite{ref:47}. It consists of variables TEXT, ITER using format (A,I),
where TEXT is set to the character string SPLIT and ITER is an integer
indicating after which iteration step split trial vectors are to be used.
Usually there is no loss of speed of convergence when split trial vectors
are used from the very beginning. Setting ITER to zero is therefore
recommended with this directive.
%
\subsection{MCLR Data - Further directives}
The directives MAXRED, MAXIT, BEGIN, POLA and THRESH are also 
available in MCLR calculations.
They are identical to the corresponding directives in the RPA module,
with one exception: the THRESH directive must {\em not} be terminated
by the keyword END since this is reserved for signaling the end of
the MCLR directives.
The reader is referred to the corresponding sections for a
detailed description of the directives.

We list below the complete input for an MCLR calculation on the 
water molecule, using a TZVP basis and performing a full valence 
space calculation. We start with the MCSCF job, omitting the SCF job(s):
%
\subsubsection{MCSCF calculation}
{
\footnotesize
\begin{verbatim}
          RESTART
          TITLE
          H2CO TZVP MCSCF
          BYPASS
          ZMAT
          O
          H 1 1.809 
          H 1 1.809 2 104.52 
          END
          BASIS TZVP
          RUNTYPE SCF
          SCFTYPE MCSCF
          MCSCF
          ORBITAL
          DOC1 DOC1 DOC3 DOC1 DOC2 UOC1 UOC3
          END
          ENTER
\end{verbatim}
}
%
\subsubsection{MCLR calculation}
{
\footnotesize
\begin{verbatim}
          RESTART
          TITLE
          H2CO TZVP MCSCF
          BYPASS MCSCF
          ZMAT
          O
          H 1 1.809 
          H 1 1.809 2 104.52 
          END
          BASIS TZVP
          SCFTYPE MCSCF
          MCSCF
          ORBITAL
          DOC1 DOC1 DOC3 DOC1 DOC2 UOC1 UOC3
          END
          RUNTYPE RESPONSE MCLR
          ORBITAL
          DOC1 DOC1 DOC3 DOC1 DOC2 UOC1 UOC3
          END
          SECTIONS
             SCF       1
             MCSCF     8
             CANONICAL 10
             CIVEC     9
          SYMM 1 5
          SYMM 2 5
          SYMM 3 5
          SPLIT 0
          END
          VECTORS 2 3
          ENTER 2 3 
\end{verbatim}
}
\clearpage

\begin{thebibliography}{10}

\bibitem{ref:46} C. Fuchs, V. Bona\v{c}i\'{c}-Kouteck\'{y} and J. Kouteck\'{y},
       J.\ Chem.\ Phys.\ {\bf 98} (1993) 3121, \doi{10.1063/1.464086}.

\bibitem{ref:47} P.~J\o rgensen, H.\,J.\,Aa.~Jensen, and J.~Olsen,
       J.\ Chem.\ Phys.\ {\bf 89} (1988) 3654, \doi{10.1063/1.454885}.

\end{thebibliography}

\end{document}
