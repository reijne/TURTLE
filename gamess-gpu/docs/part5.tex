\documentclass[11pt,fleqn]{article} 

\usepackage{hyperref}

% package HTML requires Latex2HTML to be installed for html.sty
\usepackage{html}
\newcommand{\doi}[1]{doi:\href{http://dx.doi.org/#1}{#1}}
\begin{htmlonly}
\renewcommand{\href}[2]{\htmladdnormallink{#2}{#1}}
\end{htmlonly}
\hypersetup{colorlinks,
            %citecolor=black,
            %filecolor=black,
            %linkcolor=black,
            %urlcolor=black,
            bookmarksopen=true,
            pdftex}

\addtolength{\textwidth}{1.0in}
\addtolength{\oddsidemargin}{-0.5in}
\addtolength{\topmargin}{-0.5in}
\addtolength{\textheight}{1.0in}

\newcommand{\formaldehyde}{\mbox{H$_{2}$CO}}
\newcommand{\formion}{\mbox{H$_{2}$CO$^{+}$}}
\newcommand{\bstate}{\mbox{$^{2}$B$_{2}$}}
\newcommand{\water}{\mbox{H$_{2}$O}}
\newcommand{\dinfh}{\mbox{D$_{\infty h}$}}
\newcommand{\dtwoh}{\mbox{D$_{2h}$}}
\newcommand{\cinfv}{\mbox{C$_{\infty v}$}}
\newcommand{\ctwov}{\mbox{C$_{2v}$}}
\newcommand{\silane}{\mbox{SiH$_{4}$}}
\newcommand{\phosphine}{\mbox{PH$_{3}$}}
\newcommand{\cucl}{\mbox{CuCl}}
\newcommand{\nitrog}{\mbox{N$_{2}$}}
\newcommand{\cah}{\mbox{CaH$_{2}$}}
\newcommand{\astate}{\mbox{$^{1}$A$_{1}$}}
\newcommand{\xastate}{\mbox{X$^{1}$A$_{1}$}}

\pagestyle{headings}
\pagenumbering{roman}
\begin{document}
\sf
\parindent 0cm
\parskip 1ex
\begin{flushleft}
 
Computing for Science (CFS) Ltd.,\\CCLRC Daresbury Laboratory.\\[0.30in]
{\large Generalised Atomic and Molecular Electronic Structure System }\\[.2in]
\rule{150mm}{3mm}\\
\vspace{.2in}
{\huge G~A~M~E~S~S~-~U~K}\\[.3in]
{\huge USER'S GUIDE~~and}\\[.2in]
{\huge REFERENCE MANUAL}\\[0.2in]
{\huge Version 8.0~~~June 2008}\\ [.2in]
{\large PART 5. INTEGRAL TRANSFORMATION \& DIRECT-CI}\\
\vspace{.1in}
{\large M.F. Guest, J. Kendrick, J.H. van Lenthe and P. Sherwood}\\[0.2in]
 
Copyright (c) 1993-2008 Computing for Science Ltd.\\[.1in]
This document may be freely reproduced provided that it is reproduced\\
unaltered and in its entirety.\\
\vspace{.2in}
\rule{150mm}{5mm}\\
\end{flushleft}


\tableofcontents

\newpage

\pagenumbering{arabic}

\section[Introduction]{Introduction}

In this chapter we turn our attention to the post-Hartree Fock modules
within GAMESS--UK, considering initially the integral transformation
routines and associated data input, and then the Direct-CI module.
Note that the transformation module acts not only as a precursor to
Direct-CI, but finds more widespread usage in, for example, both OVGF
and TDA Green's function calculations, and in the semi-direct Table-CI
module.

\section[Integral Transformation]{Integral Transformation}

Before detailing the directives associated with
the transformation of 1 and 2-electron integrals
over atomic orbitals to the corresponding set over molecular
orbitals, we describe briefly the data sets
used. Note that the algorithm employed is basically that due to
Yoshimine  \cite{ref:40}.
In addition to the Mainfile, Dumpfile and Scratchfile,
the following files will be used, with the associated space
requirements considered below:
\begin{itemize}
\item Sortfile: A dataset assigned to SORT will be used as a 
scratchfile in sorting operations. The
space requirements are slightly more than twice the length of the
Mainfile if a single pass sorting is adopted. 
\item Secondary Mainfile: Partially transformed 2-electron 
integrals are output to a dataset referred to 
as the Secondary Mainfile, and the user
may direct this dataset to any of the files, ED0--ED19 and MT0--MT19.
The LFN ED4 is used in default: the SFILE directive described in 
Part 3 may be used to re-assign Secondary Mainfile output.  Care
should be taken if the user assigns the Secondary Mainfile to the same
file as the Mainfile. The Secondary Mainfile should not be
allowed to overwrite the Mainfile except where the integrals sort has
been completed in one pass (NPASS=1). The Mainfile should not be over-
written if it is wished to perform a 2-index transformation of a Fock
operator. Generally it is advisable to route the Secondary Mainfile
to a different file, rather than to use the Mainfile.
\item  Transformed Integral file: Fully transformed 2-electron 
integrals over the molecular orbitals are output to datasets 
referred to as the Transformed Integral File, written in default to ED6.
The user may re-direct this dataset to any file, ED0--ED19 
and MT0--MT19 using the FFILE directive of Part 3.
The Transformed Integral file 
may overwrite the Mainfile, but
should not be allowed to overwrite the Secondary Mainfile except where
the second integrals sort has been completed in one pass (NPASS2=1).
Again overwriting of the Mainfile is not allowed if a Fock operator is
to be transformed. It is advisable to 
route the Transformed Integral file to a
different file than is allocated to either the Mainfile and the
Secondary Mainfile.
\end{itemize} 
The following approximation may prove useful in the considering the
space requirements of the Secondary Mainfile, Transformed Integral 
File and Sortfile. Given NBASIS basis functions and NACT active
orbitals, and defining,

{
\footnotesize
\begin{verbatim}
          M = NBASIS * (NBASIS+1) / 2.

          N = NACT * (NACT+1) / 2.

          L = Number of blocks in the Mainfile (1 block = 512 words).
\end{verbatim}
}
Then   approximate space requirements are given by:
{
\footnotesize
\begin{verbatim}
          Secondary Mainfile         = S = (2*L*N) / M

      Transformed Integral file      = (L*N*N) / (M*M)

          Sortfile                   = (2*L) / NPASS1 or S / NPASS2
                                  (whichever is the greater)
\end{verbatim}
}

\section[Directives Controlling Integral Transformation]{Directives Controlling Integral Transformation}

\subsection[TRACC]{TRACC}

The TRACC directive consists of a single line read to variables TEXT, K
using the format (A,I).
\begin{itemize}
\item  TEXT should be set to the character string TRACC.
\item  The integer K is used to compute the threshold factor 
ACC = 10$^{-K}$, and if the
absolute value of a transformed 2-electron integral is
less than ACC, that integral will not be output to the
Transformed Integral file. A factor ACC1 = ACC*10$^{-2}$ is also
computed, and if the absolute value of a partially trans-formed 
2-electron integral is less than ACC1, that
integral will not be output to the Secondary Mainfile.
\end{itemize}
The TRACC directive may be omitted, when the default value K = 10 is
assumed. The smaller the value of K, the shorter will be the size of
the Secondary and Transformed Integral file, 
and the shorter the computation time
in the second phase of the 4-index transformation.

\subsection[PASS]{PASS}

The PASS directive consists of a single dataline read to variables 
TEXT, NPASS1, NPASS2 using format (A,2I).
\begin{itemize}
\item TEXT should be set to the character string PASS.
\item NPASS1 is an integer specifying the minimum number of passes of
the Mainfile in the first phase of the 4-index transformation.
\item NPASS2 is an integer specifying the minimum number of passes of
the Secondary Mainfile in the second phase of the 4-index
transformation.
\end{itemize}
In the absence of a PASS directive the module will calculate the
minimum number of passes required, which will depend on the basis set
size and the amount of memory available to the module. The user should
note the following:
\begin{itemize}
\item The size of the Sortfile is usually inversely proportional to
either NPASS1 or NPASS2. To reduce the size of the Sortfile
multi-passing of the Mainfile and Secondary Mainfile must be employed.
\item The program forms a dump enabling a restart of the 4-index
transformation process at the end of each pass. The more passes, the
shorter time interval between dumps.
\end{itemize}
{\bf Example:}
{
\footnotesize
\begin{verbatim}
           PASS 3 2
\end{verbatim}
}
Specifies a 3 and 2 pass sort of the Mainfile and Secondary Mainfile
respectively.

\subsection[ACTIVE]{ACTIVE}

This directive specifies those members of the molecular orbital set
which are deemed 'active' in the integral transformation, so that
integrals of the form $<$ij/kl$>$ will be computed and output to the 
Transformed Integral File if
all four orbitals are specified using the ACTIVE directive.
The first data field consists of the
character string ACTIVE in the first data field. Subsequent data lines
are read to an array (IACTIV(I),I=1,NACT) using free I-format. The last
data field presented should be the character string END.\\

{\bf Example 1}
{
\footnotesize
\begin{verbatim}
          ACTIVE
          10 11 12 13 14 15 16 17 18 20
          END
\end{verbatim}
}
Molecular Orbitals 10 to 18 and 20 are made active, and will 
be re-indexed 1 to 10  respectively, with the Direct-CI module, 
for example,   referring to the orbitals in this re-indexed convention.\\

{\bf Example 2}
{
\footnotesize
\begin{verbatim}
          ACTIVE
          10 TO 18 20
          END
\end{verbatim}
}
This example shows the use of the string TO to abbreviate consecutive
sequences of integers, and is equivalent to example 1.


\subsection[CORE]{CORE}

The CORE directive allows the user to:
\begin{itemize}
\item route a 2-index transformed Fock
operator to  a specific section of the Dumpfile. 
\item  factor frozen doubly occupied orbitals (such orbitals retain 
their double occupation
in all configurations generated by the Direct-CI module, for
example) into the Fock Operator (F) where:

{
\footnotesize
\begin{verbatim}
          F = H + 2 J[R] - K[R]
\end{verbatim}
}
H denotes the usual one-electron operator (sum of
kinetic and nuclear attraction), R denotes the CORE shells
density matrix, and J and K are coulomb and exchange matrices
constructed therefrom.
\end{itemize}
When employing the CORE directive in conjunction with factoring out
molecular orbitals, it must be remembered that the molecular orbitals
must be doubly occupied and no partial occupied orbitals can be factored
out in this manner.

The first line is read to TEXT,NSECT in format (A,I).
\begin{itemize}
\item TEXT should be set to the character string 
CORE, although ONELEC is also acceptable.
\item NSECT specifies the section number on the
Dumpfile where the transformed 1-electron integrals
are to be placed. If omitted the integrals are routed to
section 466. If specified, NSECT must lie between 1 and 350.
\end{itemize}
Subsequent lines specify the frozen doubly occupied
orbitals as a sequence of integers, corresponding to the MO ordering
that came from the SCF module.  The sequence can be abbreviated
using the string TO, and
is terminated by a line containing the string END in the first
datafield.\\

{\bf Example 1}
{
\footnotesize
\begin{verbatim}
          CORE  200
          1 2
          3 4 5 6 7 8 9 10
          END
\end{verbatim}
}
This example routes the transformed 1-electron integrals to
section 200 of the Dumpfile. Molecular Orbitals 1 to 10
have been declared to be doubly occupied and frozen.\\

{\bf Example 2}
{
\footnotesize
\begin{verbatim}
          CORE 200
          1 TO 10
          END
\end{verbatim}
}
This example shows the use of the string TO to shorten the data input,
and is equivalent to example 1.\\

{\bf Example 3}
{
\footnotesize
\begin{verbatim}
          CORE
          1 TO 20 
          END
\end{verbatim}
}
This example assumes default routing of the transformed 1-electron
integrals, to section 466. MOs 1 to 20 will be frozen.\\

{\bf Example 4}\\

The CORE directive can be omitted, when no MOs will be frozen, and
the transformed 1-electron integrals will be routed to section 466
of the Dumpfile. Omission is equivalent to:

{
\footnotesize
\begin{verbatim}
           CORE
           END
\end{verbatim}
}


\section[RUNTYPE and Restarting the Transformation]{RUNTYPE and Restarting the Transformation}

In most applications, the transformation module will be run as part of 
either Direct-CI or Green's function calculations, under control
of RUNTYPE~CI, RUNTYP~GF or RUNTYPE~TDA specification. 
In some circumstances 
it may be necessary to generate the transformed integrals only, and
a specific RUNTYPE, code-named TRANSFORM, has been provided for
this purpose. Note that TRANSFORM processing includes both the
SCF step and subsequent integral transformation, with restarts
possible in both steps. 
For  such restarting, the Mainfile, Secondary Mainfile and
Transformed Integral file 
should have been permanent files in the startup job,
exactly the same files being presented to the restart job, and, if used,
identical MFILE, SFILE and FFILE directives should be used in the 
startup and restart jobs. 
The Sortfile need not be preserved between jobs.\\

{\bf Example}\\

In this example we are transforming the integrals from a TZVP
calculation on \formaldehyde. The first data file represents
the startup job, the second the restart data, assuming the
processing in the startup job did not complete in
the allocated time.\\

{\bf The Startup Job}
{
\footnotesize
\begin{verbatim}
         TITLE
         H2CO - TZVP BASIS - TRANSFORMATION
         SUPER OFF NOSYM
         ZMATRIX ANGSTROM
         C
         O 1 1.203
         H 1 1.099 2 121.8
         H 1 1.099 2 121.8 3 180.0
         END
         BASIS TZVP
         RUNTYPE TRANSFORM
         PASS 2 2
         TRACC 9
         ENTER
\end{verbatim}
}
{\bf The Restart Job}
{
\footnotesize
\begin{verbatim}
         RESTART TRANSFORM
         TITLE
         H2CO - TZVP BASIS - TRANSFORMATION
         SUPER OFF NOSYM
         ZMATRIX ANGSTROM
         C
         O 1 1.203
         H 1 1.099 2 121.8
         H 1 1.099 2 121.8 3 180.0
         END
         BASIS TZVP
         RUNTYPE TRANSFORM
         PASS 2 2
         TRACC 9
         ENTER
\end{verbatim}
}


\section[Direct-CI Calculations]{Direct-CI Calculations}

The Direct-CI module performs  general multi-reference singles and
doubles configuration interaction (CI) calculations. 
The method (direct or
integral driven) used in the package is described in \cite{ref:19}.
For optimal running the main memory allocation should
be at least 3 times longer than the number of configuration state
functions (CSFs) in the CI expansion. The module will execute
in less memory, but at the cost of increased
disc input/output, and higher overall job cost.
 The following data sets will be used by the program.
\begin{itemize}
\item Transformed Integral File: Integrals over the molecular
orbitals (MOs) will be read from this file. The user may direct
that this dataset be read from any  file except ED5 or ED8
by means of the FFILE directive (see above). The default is to use ED6.
\item Direct-CI File: The file ED5 is used to 
hold control information, accommodate the partial
Hamiltonian matrix elements and to
store the update vectors created by the Davidson diagonalization
procedure. Twice the CI expansion length will be added at each
iterative cycle. The DIAGMODE directive (see below) may be used to
reduce the maximum size of the Davidson sub-space, thus limiting the
ultimate size of ED5, possibly at the expense of an inferior rate of
convergence. If it is thought possible that the
diagonalization procedure may not converge in one job, ED5 must be
retained to allow for restarts.
\item The P-Sortfile: A dataset normally assigned using the 
local file name (LFN) PSORT will be used as a scratchfile 
in a pre-sort of the transformed
molecular integrals. 
The space requirements of the P-Sortfile are about 1.5 times that of the
Transformed integral File  produced by
the transformation module.
\item the Sortfile: A dataset normally assigned using the LFN
SORT will be used as a scratchfile in a post-sort of the transformed
molecular integrals. The maximum
space requirements of the SORT FILE are about twice that of the
Transformed integral File produced by the
transformation module, although this will be much reduced in
high symmetry, by an inverse factor approaching the order of the point group
involved.
\item Overflow File: During the construction of the 
partial matrix elements
a scratchfile allocated as file ED8 may be required.
This is particularly
likely to occur if a large number of reference CSFs are
specified, while its likelyhood is decreased if a large main memory
allocation is used. ED8 is usually of the order 1000 to 4000 blocks
long if it is required.
\end{itemize}

Data input characterising the CI calculation commences with the DIRECT
data line, and is typically followed by a sequence of directives,
terminated by presenting a valid {\em Class 2} directive, such as VECTORS
or ENTER.  The directives may be presented in any order, although in
some  cases directives are inter-related and care should be taken when
presenting them, since the order in which the directives are presented
is in such cases often significant; this is particularly the case for
the EXCIT, CONF, REFGEN and CASGEN directives.

Before describing each of the directives in detail, we cater for those
users who wish to "fast forward" through the directive descriptions by
outlining how to perform "default" single reference CISD calculations.
This provides a set of default attributes that bypasses the requirement
for explicit data specification; while of somewhat limited
applicability, it does provide a starting point for users, and a route
to subsequent, more extensive calculations.

\subsection[Direct-CI - Default CISD Calculations]{Direct-CI - Default CISD Calculations}

In order to simplify the process of configuration specification and
data preparation, the Direct-CI module now provides a set of default
options that require little or no data input.  To illustrate this
default working of the module, we consider below a number of example
calculations based on those that will be described in more detail in
the subsequent sections.

\subsubsection{Closed-shell Systems}

A Direct-CI calculation is to performed on the formaldehyde molecule.
Given the following data sequence:

{
\footnotesize
\begin{verbatim}
          TITLE
          H2CO - 3-21G DEFAULT DIRECT-CI CISD OPTION
          ZMAT ANGSTROM
          C
          O 1 1.203
          H 1 1.099 2 121.8
          H 1 1.099 2 121.8 3 180.0
          END
          RUNTYPE CI
          ENTER
\end{verbatim}
}
then the calculation undertaken will be based on the following;
\begin{enumerate}
\item The format of the 2e-integral file will be automatically set to
the required "SUPER OFF NOSYM", triggered by the presence of the
CI runtype.
\item Integral transformation will use the set of orbitals from section
1, the default section for output of the closed-shell SCF eigenvectors.
All orbitals will be deemed ACTIVE in the transformation.
\item The Direct-CI module is the default module loaded under
RUNTYPE CI control, so that the DIRECT directive is not required.
\item The division of the molecular orbital space into an internal and
external space, typically specified by the DIRECT directive, is now
handled automatically, with the internal space comprising all doubly
occupied SCF MOs orbitals, the external space all SCF virtual MOs. All
electrons will be deemed active in the CI.
\item The SYMMETRY and SPIN of the CI wavefunction are taken to be
those of the SCF wavefunction.
\item A single reference configuration will be employed, just
the SCF configuration; the final configuration space will include
all single and double excitations from this SCF reference configuration.
\item The spinfree natural orbitals will be written to section
11 of the Dumpfile.
\end{enumerate}

The full data specification corresponding to the defaults generated
from the above data file is shown below; the role of each of the
directives will be described in later sections.

{
\footnotesize
\begin{verbatim}
          TITLE
          H2CO - 3-21G  CISD DIRECT-CI CALCULATION
          SUPER OFF NOSYM
          ZMATRIX ANGSTROM
          C
          O 1 1.203
          H 1 1.099 2 121.8
          H 1 1.099 2 121.8 3 180.0
          END
          RUNTYPE CI
          DIRECT 16 8 14
          CONF
          2 2 2 2 2 2 2 2
          NATORB 11 0 PRINT
          ENTER
\end{verbatim}
}

\subsubsection{Open-shell Systems}

Let us now consider a Direct-CI calculation  on the \bstate\  state of
\formion, again using default options available within the module. A
valid data sequence for performing such a calculation is shown below:

{
\footnotesize
\begin{verbatim}
          TITLE
          H2CO+ 2B2 3-21G - DEFAULT CISD DIRECT-CI OPTION
          MULT 2
          CHARGE 1
          ZMAT ANGSTROM
          C
          O 1 1.203
          H 1 1.099 2 121.8
          H 1 1.099 2 121.8 3 180.0
          END
          RUNTYPE CI
          ENTER
\end{verbatim}
}
As with the closed-shell run above, no explicit data is required to
define the nature of the CI calculation. In practice the defaults
adopted correspond to the following:
\begin{enumerate}
\item The CI will be based on the high-spin open-shell RHF calculation.
\item The set of vectors used in the transformation will be
the energy-ordered SCF orbitals from section 5 of the Dumpfile, the
default section in the absence of section specification on the ENTER
directive.
\item The symmetry and spin of the CI wavefunction will be deduced
from the preceding SCF calculation i.e. a doublet CI wavefunction of B$_{2}$
symmetry (corresponding to SPIN 2).
\item The number of active electrons in the CI will be set to be those
involved in the SCF calculation (i.e. 15).
\item The reference configuration to be employed will be just the
open-shell SCF configuration. The internal space comprises the doubly plus
singly occupied SCF orbitals, with the external space comprising the SCF
virtual orbitals. All electrons will be deemed active in the CI.
\item The spinfree natural orbitals will be written to section
11 and the spin natural orbitals to section 12 of the Dumpfile.
\end{enumerate}

The full data specification corresponding to the defaults generated
from the above data file is shown below; the role of each of the
directives will again be described in later sections.

{
\footnotesize
\begin{verbatim}
          TITLE
          H2CO+ - 2B2 - 3-21G  CISD DIRECT-CI CALCULATION
          SUPER OFF NOSYM
          CHARGE 1
          MULT 2
          ZMATRIX ANGSTROM
          C
          O 1 1.203
          H 1 1.099 2 121.8
          H 1 1.099 2 121.8 3 180.0
          END
          RUNTYPE CI
          OPEN 1 1
          DIRECT 15 8 14
          SPIN DOUBLET
          CONF
          2 2 2 2 2 2 2 1
          NATORB 11 12 PRINT
          ENTER

\end{verbatim}
}

\section[Direct-CI Data Input]{Direct-CI Data Input}

\subsection[DIRECT]{DIRECT}

DIRECT consists of one line read to variables TEXT, NELEC, NINT, NEXT using
format (A,3I).
\begin{itemize}
\item TEXT should be set to the character string DIRECT.
\item  NELEC specifies the number of electrons in the CI calculation.
Notice that any inner shell electrons frozen out using the
CORE directive of the transformation module  should not
be included.
\item  NINT specifies the number of internal MOs. These will be used to
construct reference CSFs. If a MO is not occupied
in any reference CSF it should not ordinarily be classified
as internal, unless high levels of internal excitation are contemplated
(see EXCIT directive below).
The internal MOs normally correspond to that set capable of producing
a qualitatively correct wavefunction.
Notice that NINT*2 must be greater than or equal
to NELEC.
\item  NEXT specifies the number of external MOs. Such MOs will be
unoccupied in all reference CSF. Single and double
excitations from the internal to the external MOs cause
the latter to contribute to the CI expansion.
Notice that NINT+NEXT must be less than or equal to the number of
active MOs as specified under control of the ACTIVE directive.
If less than, then some active MOs will not take part in the CI,
and in the absence of a REORDER directive (see below), these will be
the highest indexed active MOs.
\end{itemize}

\subsection[THRESH]{THRESH}

This directive consists of a single line read to variables
TEXT, C, K using format (A,F,I).
\begin{itemize}
\item TEXT should be set to the character string THRESH.
\item  C,K: The diagonalization is converged to a threshold (T)
such that T=C/(10**K). If K is not set, it will be
given the value 0. The lowest value to which T may be set is
1E-8, and this minimum will be selected if the user attempts
to set a smaller T value.
\end{itemize}
The THRESH directive may be omitted, when T will be set to 3E-4.\\

{\bf Example}
{
\footnotesize
\begin{verbatim}
          THRESH 2 5
    
          THRESH 2E-5
    
          THRESH 0.00002
\end{verbatim}
}
are equivalent, causing T to be set to 2E-5.


\subsection[MAXCYC]{MAXCYC}

This directive consists of one line read to variables TEXT, MAXC
using format (A,I).
\begin{itemize}
\item  TEXT should be set to the character string MAXCYC.
\item  MAXC specifies the maximum number of iterative cycles to be
carried out by the Davidson  diagonalizer.
\end{itemize}
The directive may be omitted, when MAXC will take the default value 50.

\subsection[SHIFT]{SHIFT}

 This directive consists of one line read to variables TEXT, SHIF using
format (A,F).
\begin{itemize}
\item  TEXT should be set to the character string SHIFT.
\item  SHIF should be set to the desired value of the level shifter to
be used in the CI diagonalization phase.
\end{itemize}
If the SHIFT directive is omitted, the default SHIF=0.0 will be taken.
For ground states small values (between 0.0 and 0.2) provide an optimal
rate of convergence, and usually there is little point in using
the SHIFT directive. For excited states, the rate of convergence may
sometimes be markedly improved by using a SHIF value of between 0.3 and
0.5, particularly if the ALTERNAT directive (see below) is used.

\subsection[ALTERNAT]{ALTERNAT} 

 This directive consists of one line which should contain the
character string ALTERNAT in the first data field. If presented,
the directive causes the sign of the SHIF parameter (see
the SHIFT directive above) to be altered at each iterative cycle
of the CI diagonalization, and this may improve the convergence
rate for excited states. We do not recommend use of ALTERNAT
except in cases where severe convergence problems are encountered.

\subsection[DIAGMODE]{DIAGMODE}

This directive consists of a single data line read to variables
TEXT, ATEXT, NDAVID using format (2A,I).
\begin{itemize}
\item  TEXT should be set to the character string DIAGMODE.
\item  ATEXT should be set to one of the character strings EMIN, VMIN or
LOCK. EMIN causes the program to unconditionally minimize the total
energy, and is normally the best option for ground states.
VMIN causes the program to minimize the variance (sum of squares
of residuals in the secular problem), and is usually the best option if
convergence to an excited state is required.
LOCK causes the program to seek a solution to the CI problem
looking most like the trial wavefunction (see TRIAL directive below),
and is therefore another way of
trying to converge onto an excited state.
\item  NDAVID specifies the maximum size of the sub-space to be used
in the Davidson  diagonalization procedure. If omitted, the default
NDAVID=50 will be taken. It may be necessary to use a smaller
sub-space in order to limit the size of ED5, as explained above.
The largest possible sub-space is 50; attempts to set NDAVID
larger than 50 will cause the program to use this maximum.
\end{itemize}
The DIAGMODE directive may be omitted, when the defaults
ATEXT=EMIN and NDAVID=50 will be taken. It is possible to omit
the TEXT parameter of this directive.\\

{\bf Example}
{
\footnotesize
\begin{verbatim}
          DIAGMODE VMIN 50

          VMIN 50

          VMIN
\end{verbatim}
}

all have an equivalent effect, causing the variance minimization
option to be selected, with a maximum sub-space of 50.

\subsection[PRINTVAR]{PRINTVAR}

This directive consists of a single line, whose first field should
contain one of the character strings PRINTVAR or VARPRINT.
The EMIN and LOCK modes of the diagonalizer (see DIAGMODE directive above)
do not ordinarily compute the variance, and therefore do not output this
quantity. The PRINTVAR directive
may be used to turn such printing on, and is redundant if the VMIN
option of the DIAGMODE directive is used, since the variance is
always printed in this case.

\subsection[TRIAL]{TRIAL}

This directive may be used to define the trial CI wavefunction as
a linear combination of CSFs. There are two ways to accomplish this:
\begin{enumerate}
\item Specify the coefficients of the CSFs in the input file. In this
case the first data line should contain the
character string TRIAL in the first data field.
There may be up to 20 lines following this directive initiator,
each being read to variables ICSF, CCSF using format (I,F).
\begin{itemize}
\item ICSF should be set to the index of a CSF in the
CI wavefunction. Normally, such an index will only be known
after running the program once, so that the configuration generator
output can be studied. The user should understand the order in which
the program generates spin states belonging to the same space occupancy
pattern (see below) before using this directive, particularly
where CSFs involving large numbers of singly occupied MOs are involved.
\item  CCSF should be set to the coefficient of the CSF in the trial
wavefunction. The trial wavefunction will be a linear combination of the
indicated CSFs with coefficients given by the CCSF parameters.
\end{itemize}

{\bf Example}
{
\footnotesize
\begin{verbatim}
          TRIAL
          1  0.5
          2 -0.5
\end{verbatim}
}

The trial wavefunction will consist of a linear combination of the first
and second CSF in the CI expansion, with coefficients 0.5 and -0.5
respectively. It will be subsequently normalized by the program.
 
\item Select a subspace of CSFs and compute the eigenvector of the
matrix corresponding to the subspace.  In this case the syntax is:

{
\footnotesize
\begin{verbatim}
          TEXT TEXTA [TEXTB [NCSF] or TEXTC] [ TEXTD ISTATE ] [ TEXTE ]
\end{verbatim}
}
where
\begin{itemize}
\item TEXT   should be the literal string TRIAL.
\item TEXTA  should be the literal string DIAG.
\item TEXTB  may be set to SELECT or to FIRST. If the string SELECT is 
supplied a number of CSFs with the lowest energies will be selected. If 
the string FIRST is supplied a number of CSFs that are the first in 
order will be selected. Only a maximum of 100 CSFs can be selected.
SELECT is default.
\item TEXTC may be set to 'ref' or 'vac' selecting the whole reference 
space or vacuum space respectively. 
\item TEXTD may be set to 'state' in which case ISTATE is an integer 
specifying which eigenvector from the subspace matrix should be used. 
The default value is 1 i.e., the eigenvector with the lowest 
eigenvalue will be selected.
\item TEXTE  may be set to the literal string PRINT. If supplied the
trial vector will be printed. By default the trial vector will NOT be
printed. if PRINT is given twice the selected H-matrix is also printed
\end{itemize}
{\bf Example}
{
\footnotesize
\begin{verbatim}
          TRIAL DIAG SELECT 25
\end{verbatim}
}
This will cause the program to select the 25 CSFs with the lowest 
eigenvalues. The matrix in this basis is solved for the eigenvector to
obtain the trial vector.
\end{enumerate}

{\bf Default:}\\
The TRIAL directive may be omitted, when the trial wavefunction will be
selected from the reference space.

\subsection[JACDAV]{JACDAV}

The JACDAV directive sets the controls of the Jacobi-Davidson 
preconditioning method. The syntax of this directive is 

{
\footnotesize
\begin{verbatim}
         JACDAV SUBDIR [ SUBDIR [ .. ] ]
\end{verbatim}
}
where JACDAV is a literal string, acting as the directive initiator,
and SUBDIR is a valid subdirective. Each subdirective consists of a 
literal string eventually followed by an integer, real, or string 
argument. The supported subdirectives, OFF, ON, SHIFT, THRESH, MAXCYC
and PRINT are detailed below:
\begin{enumerate}
\item  {\bf OFF}:
This is a literal string, it switches the Jacobi-Davidson preconditioner
off. By default the preconditioner is switched on. 
\item  {\bf ON}:
This is a literal string, that switches the Jacobi-Davidson
preconditioner on. This is the default. 
\item {\bf SHIFT DYNAMIC / RSHIFT}:
This subdirective sets the level shifter for the preconditioner. Here
SHIFT is a literal string that is followed either by the literal string
DYNAMIC or a real value for RSHIFT. If a real value RSHIFT is
supplied then that value will be used for the level shifter. If the string
DYNAMIC is supplied the level shifter will be automatically adjusted
to force convergence. The latter is the default.
\item  {\bf THRESH RTHRSH}:
This subdirective sets the convergence threshold for the
preconditioner. Here THRESH is a literal string and RTHRSH is a
real value. By default the threshold is set to half the CI threshold at
time of calling JACDAV. 
\item {\bf MAXCYC IMAXC}:
This subdirective sets the maximum number of cycles for the
preconditioner. Here MAXCYC is a literal string and IMAXC is an
integer value. By default IMAXC is set to 100. 
\item  {\bf PRINT}:
This subdirective controls the output level of the preconditioner. Here
PRINT is a literal string. Every time this string is supplied the output
level is increased, causing the preconditioner to generate a more
detailed output. By default the printing is off. 
\end{enumerate}
NOTE: Because the default threshold is determined from the CI threshold at time of
calling JACDAV interchanging the order of the THRESH directive and the
JACDAV directive may lead to different convergence behaviour. 

\subsection[VPRINT]{VPRINT}


 This directive consists of a single data line read to variables
TEXT, NPR, TPR, ATEXT, BTEXT, CTEXT using format (A,I,F,3A), and is used
to control printing of the CI wavefunction.
\begin{itemize}
\item TEXT should be set to the character string VPRINT.
\item  NPR specifies the maximum number of CI coefficients to be printed.
\item  CI coefficients less than TPR in absolute magnitude will not be
printed.
\item  ATEXT, BTEXT, CTEXT may each be set to one of the character strings
VAC, N-1, or N-2. If VAC is specified, CI coefficients of all the
vacuum states (those with no electrons in the external MO space) will
be printed, irrespective of the values of NPR and TPR. Similarly
the parameters N-1 and N-2 control printing of those CSFs with one or
two electrons in the external space respectively.
\end{itemize}
 This directive may be omitted, when the defaults NPR=100 and
TPR=1E-7 will be taken.\\

{\bf Example }
{
\footnotesize
\begin{verbatim}
          VPRINT 50 0.02 VAC
\end{verbatim}
}

will cause all vacuum state coefficients to be printed.
All coefficients greater in absolute magnitude than 0.02 will be printed,
unless there are more than 50 of these, in which case only the largest
50 will be printed.

\subsection[SPIN]{SPIN}

This directive consists of one line read to variables TEXT, NSPIN
using format (A,I).
\begin{itemize}
\item  TEXT should be set to the character string SPIN.
\item  NSPIN is used to specify the spin degeneracy 
of the CI wavefunction,
using the values 1,2,3 etc. for singlet, doublet, triplet states etc.
respectively. It is also possible to use one of the character strings
SINGLET, DOUBLET, TRIPLET, QUARTET, QUINTET, SEXTET, SEPTET, OCTET and
NONET to specify NSPIN.
\end{itemize}
The SPIN directive may be omitted, when the program will set NSPIN
to 1 or 2 if NELEC is even or odd respectively.\\

{\bf Example}
{
\footnotesize
\begin{verbatim}
          SPIN 4

          SPIN QUARTET
\end{verbatim}
}

are equivalent; the wavefunction will be four-fold spin degenerate.

\subsection[NATORB]{NATORB}

The NATORB directive consists of a single dataline read to variables
TEXT, KSPACE, KSPIN, ATEXT using format (A,2I,A).
\begin{itemize}
\item  TEXT should be set to the character string NATORB.
\item  KSPACE is an integer (between 0 and 350 inclusive) specifying the
section number of the Dumpfile where the spin-free NOs are to be placed.
If KSPACE=0, spin-free NOs will not be routed to the Dumpfile.
\item KSPIN is an integer (between 0 and 350 inclusive) specifying the
section number of the Dumpfile where the spin NOs are to be placed.
If KSPIN=0, spin NOs will not be routed to the Dumpfile. Notice
that spin NOs will not be produced for singlet wavefunctions (see
SPIN directive above) because they have an occupation number of
zero in this case.
\item  ATEXT may be set to the character string PRINT, when the NOs 
will be printed. If ATEXT is omitted, the NOs will 
not be sent to the output.
\end{itemize}

{\bf Example}
{
\footnotesize
\begin{verbatim}
          NATORB 12 14 PRINT
\end{verbatim}
}

The spin-free and spin NOs are output to sections 12
and 14 respectively of the Dumpfile, and routed to the output.

{\bf Note:} In the absence of the NATORB directive (Version 6.3 onwards)
both spin-free and, if appropriate, spin natural orbitals will
be generated in default, and routed to sections 10 and 11 of
the Dumpfile. This default thus corresponds to presenting the
data line:
{
\footnotesize
\begin{verbatim}
          NATORB 10 11 PRINT
\end{verbatim}
}
so that the NATORB directive now need only be presented to override
these defaults.

\subsection[EXCIT]{EXCIT}

This directive is used to define the excitation pattern allowed for
a set of reference CSFs defined using the CONF directive (see
below). The EXCIT directive
may be used more than once in the data input to allow the user to
specify different excitation patterns for different reference CSFs,
and normally consists of a single dataline, in which the first
data field should contain the character string EXCIT.
The second data field may also be read in A format, and if
so used should be set to one of the character strings OCTAL or BINARY.
If this second A format field is omitted, the program takes OCTAL as 
default.
 Subsequent data fields are read in I-format, and should contain either
octal numbers (valid range 0 to 7) or binary numbers (valid range 0 to
1), according to the character string contained in the second data
field. The octal or binary integers may be continued onto subsequent
lines if necessary.
 We now explain the significance of these integers if OCTAL
input mode is selected.
 The first octal integer specifies the external excitation pattern for the
reference CSFs, (where no internal excitations have
been applied). This octal number should be translated into a binary
format, such that:

{
\footnotesize
\begin{verbatim}
          OCTAL NUMBER          BINARY NUMBER
          ============          =============
              0         =       0     0     0
              1         =       0     0     1
              2         =       0     1     0
              3         =       0     1     1
              4         =       1     0     0
              5         =       1     0     1
              6         =       1     1     0
              7         =       1     1     1
\end{verbatim}
}

 The left-most binary integer defines a double external excitation,
where two electrons are promoted from the internal to the
external space. If it is 0 or 1 the process is forbidden or allowed
respectively.
The middle binary integer defines the single external
excitation process. If this is set to 1, the process is allowed, if it
is 0, single external excitations are forbidden.
The right-most binary integer
corresponds to a no external excitation process, the reference
CSFs being left as they are.
If set to 0 or 1 the reference CSFs will be eliminated from or retained
in the final list of CSFs for the CI. Note
that CSFs may also be eliminated from the CI list because they are of
the wrong spin/space symmetry.\\

{\bf Example 1}
{
\footnotesize
\begin{verbatim}
          EXCIT OCTAL 7
\end{verbatim}
}

This excitation mask will cause the reference CSFs and the
single and double external excitations generated from them to be
included in the final CI list of CSFs.\\

{\bf Example 2}
{
\footnotesize
\begin{verbatim}
          EXCIT OCTAL 5
\end{verbatim}
}

Will cause all the reference CSFs and all double external
excitations generated therefrom to be added to the
list of CSFs for the CI.\\

{\bf Example 3}
{
\footnotesize
\begin{verbatim}
          EXCIT OCTAL 4
\end{verbatim}
}

 Will cause the double external excitations of the reference states to
appear in the final list of CSFs. Note the reference and single
external excited CSFs are excluded from the CI list.
If a second octal integer is defined, the reference CSFs have
undergone a single internal excitation process. That is, a transfer of
an electron from one internal MO to another. This second integer
defines the external excitation mask on these newly constructed
internal CSFs. A third octal integer defines the external
excitation pattern after a double internal excitation. Additional octal
integers may be presented, up to a maximum to 21 integers. Thus it is
possible to define a state which is up to 20 fold internally excited
and up to doubly externally excited.\\

{\bf Example 4}
{
\footnotesize
\begin{verbatim}
          EXCIT OCTAL 7 3 1
\end{verbatim}
}

corresponds to a CI list containing the reference CSFs plus
all single and double excitations (internally and externally).
If the BINARY option is chosen, then the full binary patterns of each
equivalent octal integer must be given.\\

{\bf Example 5}
{
\footnotesize
\begin{verbatim}
          EXCIT OCTAL 7 3 1

          EXCIT BINARY  1 1 1  0 1 1  0 0 1
\end{verbatim}
}

 are equivalent.
 It is possible to give different excitation patterns to different
reference CSFs.\\

{\bf  Example 6}
{
\footnotesize
\begin{verbatim}
          EXCIT OCTAL 7 3 1
          CONF
          .
          .
          .
          .
          EXCIT OCTAL 5 0 1
          CONF
          .
          .
          .
\end{verbatim}
}

The first set of reference CSFs are associated with the excitation
pattern 7 3 1, while the second set have an excitation
pattern 5 0 1.
If the EXCIT directive is not invoked the excitation pattern will default
to the setting 7 3 1, corresponding to all single and double excitations.
An EXCIT directive presented without parameters will cause restoration
of the 7 3 1 default.


\begin{table}
 \caption{\label{table:1}\ Resolution of the \cinfv\ Species into the \ctwov\ Species}

 \begin{centering}
 \begin{tabular}{llr}
 \\ \hline\hline
\multicolumn{2}{c}{Orbital} &  IRrep  \\
         \cline{1-2}
    $\cinfv$ & $\ctwov$ & Sequence No. \\ \cline{1-3}

 $\sigma$         &   a$_{1}$         &    1 \\
 $\delta_{x2-y2}$ &                   &      \\
 $\pi_{x}$        &   b$_{1}$         &    2 \\
 $\pi_{y}$        &   b$_{2}$         &    3 \\
 $\delta_{xy}$    &   a$_{2}$         &    4 \\  \hline\hline
\end{tabular}

\end{centering}
\end{table}

\begin{table}
 \caption{\label{table:2}\ Resolution of the \dinfh\ Species into the \dtwoh\ Species}

 \begin{centering}
 \begin{tabular}{llr}
 \\ \hline\hline
\multicolumn{2}{c}{Orbital} &  IRrep  \\
         \cline{1-2}
    $\dinfh$ & $\dtwoh$ & Sequence No. \\ \cline{1-3}

 $\sigma_{g}$            &   a$_{g}$         &    1 \\
 $\delta_{g,x2-y2}$ &                   &      \\
 $\pi_{u,x}$        &   b$_{3u}$         &    2 \\
 $\pi_{u,y}$        &   b$_{2u}$         &    3 \\
 $\delta_{g,xy}$    &   b$_{1g}$         &    4 \\
 $\sigma_{u}$      &   b$_{1u}$         &    5 \\
 $\delta_{u,x2-y2}$ &                   &      \\
 $\pi_{g,x}$        &   b$_{2g}$         &    6 \\
 $\pi_{g,y}$        &   b$_{3g}$         &    7 \\
 $\delta_{u,xy}$    &   a$_{u}$         &     8 \\  \hline\hline
\end{tabular}

\end{centering}
\end{table}


\subsection[CONF]{CONF}

 The CONF directive is used to specify the reference CSFs for
the CI expansion. CONF may be presented more than once in the data
input, usually in conjunction with a different excitation pattern 
(see the
EXCIT directive above) acting on the reference CSFs. The first
line of the CONF directive is set to the character string CONF. 
Each subsequent
line specifies a reference CSF by giving the number of electrons
(0,1 or 2) in each internal MO. Thus each reference CSF is
defined by NINT numbers, the ordering of which should conform to the
order of the internal MOs as specified under control of the ACTIVE
directive.  If necessary, CSF defining
data may be carried over to further lines. 
The CONF directive may also be used to determine a reference set based
on CARDS data dumped by a preceeding CASSCF, e.g.  CONF CARDS, see below,
We now illustrate CONF usage through a series of examples that will
be subsequently used in Part 6 when describing configuration
input for the Table-CI module.\\

{\bf Example 1}\\

Consider performing a valence-CI calculation on the \phosphine\ molecule
using a 6-31G(*)  basis. While the molecular symmetry is C$_{3v}$,
the symmetry adaptation  and subsequent direct-CI will be conducted in
the C$_{s}$ point group. An examination of the SCF output reveals the
following orbital analysis.

{
\footnotesize
\begin{verbatim}
               =============================
               IRREP  NO. OF SYMMETRY ADAPTED
                      BASIS FUNCTIONS
               =============================
                 1          18
                 2           7
               =============================
\end{verbatim}
}
and the following orbital assignments characterising the closed--shell
SCF configuration:
\begin{equation}
  1a_{1}^{2}  2a_{1}^{2}  1e^{4}  3a_{1}^{2}  4a_{1}^{2}  2e^{4}  5a_{1}^{2}
\end{equation}
or, in the C$_{s}$ symmetry representation:
\begin{equation}
  1a'^{2}  2a'^{2}  3a'^{2}  1a''^{2}  4a'^{2}  5a'^{2}  6a'^{2} 2a''^{2} 7a'^{2}
\end{equation}
{
\footnotesize
\begin{verbatim}
           ===============================================
            M.O. IRREP  ORBITAL ENERGY   ORBITAL OCCUPANCY
           ===============================================
              1     1    -79.93661395           2.0000000
              2     1     -7.48916431           2.0000000
              3     1     -5.38319410           2.0000000
              4     2     -5.38319405           2.0000000
              5     1     -5.38149104           2.0000000
              6     1     -0.85610769           2.0000000
              7     1     -0.52191424           2.0000000
              8     2     -0.52191424           2.0000000
              9     1     -0.38579686           2.0000000
             10     1      0.16819544           0.0000000
             11     2      0.16819544           0.0000000
             12     1      0.26587776           0.0000000
             13     1      0.46072690           0.0000000
             14     2      0.46072690           0.0000000
             15     1      0.47871033           0.0000000
             16     1      0.56106989           0.0000000
             17     1      0.89229884           0.0000000
             18     2      0.89229885           0.0000000
             19     2      0.91131383           0.0000000
             20     1      0.91131383           0.0000000
             21     1      0.93118300           0.0000000
             22     1      1.17900613           0.0000000
             23     2      1.45058658           0.0000000
             24     1      1.45058658           0.0000000
             25     1      3.78674557           0.0000000
           ===============================================
\end{verbatim}
}
Assume that we wish to freeze the five inner shell orbitals:
\begin{equation}
  1a'^{2}  2a'^{2}  3a'^{2}  1a''^{2}  4a'^{2}  
\end{equation}
requiring the following data lines for the transformation
{
\footnotesize
\begin{verbatim}
          CORE
          1 TO 5 END
          ACTIVE
          6 TO 25 END
\end{verbatim}
}
To perform an 8-electron valence-CI calculation,
involving the SCF configuration and  two  degenerate (1e)' to (2e)'
doubly-excited  configurations 
\begin{equation}
    5a'^{2}  8a'^{2} 2a''^{2} 7a'^{2}
\end{equation}
and
\begin{equation}
    5a'^{2}  6a'^{2} 3a''^{2} 7a'^{2}
\end{equation}
would require the following CONF data:
{
\footnotesize
\begin{verbatim}
          CONF
          2 2 2 2  0 0
          2 0 2 2  2 0
          2 2 0 2  0 2
\end{verbatim}
}
where there are six orbitals in the internal space, the four doubly
occupied valence SCF MOs, and the two components of the (2e)' virtual
orbital. Note that no re-ordering of the MOs is required since the (2e)'
orbitals are the two lowest unoccupied VMOs.  The complete data file
for performing the SCF and subsequent CI would then be as follows:

{
\footnotesize
\begin{verbatim}
          TITLE
          PH3 * 6-31G*  VALENCE-CI 3M/1R
          ZMAT 
          P
          H 1 RPH
          H 1 RPH 2 THETA
          H 1 RPH 2 THETA 3 THETA  1
          VARIABLES
          RPH 2.685   
          THETA 93.83  
          END
          BASIS 6-31G*
          RUNTYPE CI
          CORE
          1 TO 5 END
          ACTIVE
          6 TO 25 END
          DIRECT 8 6 14
          CONF
          2 2 2 2  0 0
          2 0 2 2  2 0
          2 2 0 2  0 2
          ENTER
\end{verbatim}
}
Note that the "SUPER OFF NOSYM" constraint on the two-electron
integral file generated during the initial SCF will be automatically
imposed by virtue of the nominated RUNTYPE. The SUPER directive need
only be presented in an initial SCF calculation that is driven under
RUNTYP SCF control, when the user plans to subsequently access this
file in a separate CI step under control of the BYPASS directive.

{\bf Example 2}\\

In this example we wish to perform a valence-CI calculation on the \cucl\
molecule using a 3-21G  basis. While the molecular symmetry
is \cinfv, the symmetry adaptation  and subsequent CI 
will be conducted in the \ctwov\ point group. 
The resolution of the \cinfv\ into the \ctwov\ orbital species is given in
Table~\ref{table:1}.  An examination of the SCF output
reveals the following orbital analysis.

{
\footnotesize
\begin{verbatim}
           =============================
           IRREP  NO. OF SYMMETRY ADAPTED
                  BASIS FUNCTIONS
           =============================
             1          22
             2           9
             3           9
             4           2
           =============================
\end{verbatim}
}
and the following orbital assignments from the converged closed shell SCF:
{
\footnotesize
\begin{verbatim}
           ===============================================
            M.O. IRREP  ORBITAL ENERGY   ORBITAL OCCUPANCY
           ===============================================
              1     1   -326.84723972           2.0000000
              2     1   -104.02836336           2.0000000
              3     1    -40.71695637           2.0000000
              4     1    -35.46377378           2.0000000
              5     3    -35.45608069           2.0000000
              6     2    -35.45608068           2.0000000
              7     1    -10.42193940           2.0000000
              8     1     -7.88512031           2.0000000
              9     2     -7.88222844           2.0000000
             10     3     -7.88222844           2.0000000
             11     1     -5.07729175           2.0000000
             12     1     -3.38247056           2.0000000
             13     3     -3.35978308           2.0000000
             14     2     -3.35978307           2.0000000
             15     1     -1.01099628           2.0000000
             16     3     -0.53702948           2.0000000
             17     2     -0.53702947           2.0000000
             18     4     -0.49640067           2.0000000
             19     1     -0.49640067           2.0000000
             20     1     -0.44715317           2.0000000
             21     3     -0.39988537           2.0000000
             22     2     -0.39988537           2.0000000
             23     1     -0.35127248           2.0000000
             24     1      0.00023285           0.0000000
             25     3      0.06300102           0.0000000
             26     2      0.06300102           0.0000000
             27     1      0.12855448           0.0000000
             28     1      0.19287013           0.0000000
             29     3      0.25729975           0.0000000
             30     2      0.25729975           0.0000000
             31     1      0.39720201           0.0000000
             32     1      0.86197727           0.0000000
             33     2      0.88942618           0.0000000
             34     3      0.88942618           0.0000000
             35     1      1.01877167           0.0000000
             36     1      2.16694989           0.0000000
             37     3      3.96181512           0.0000000
             38     2      3.96181512           0.0000000
             39     4      3.98212497           0.0000000
             40     1      3.98212497           0.0000000
             41     1      4.08851360           0.0000000
             42     1     24.51368240           0.0000000
           ===============================================
\end{verbatim}
}
Assume that we wish to freeze the first 14 inner shell orbitals:
\begin{equation}
 1\sigma^{2}  2\sigma^{2}   3\sigma^{2}  4\sigma^{2}  1\pi^{4}  5\sigma^{2} 6\sigma^{2}   2\pi^{4}  7\sigma^{2}  8\sigma^{2}  3\pi^{4}  
\end{equation}
requiring the following data lines for the transformation
{
\footnotesize
\begin{verbatim}
          CORE
          1 TO 14 END
          ACTIVE
          15 TO 42 END
\end{verbatim}
}
To perform an 18-electron valence-CI calculation,
based on the SCF configuration 
\begin{equation}
 9\sigma^{2}  4\pi^{4}   1\delta^{4} 10\sigma^{2}  5\pi^{4}  11\sigma^{2} 
\end{equation}
would require the following CONF data:
{
\footnotesize
\begin{verbatim}
          CONF
          2 2 2 2 2 2 2 2 2
\end{verbatim}
}
The complete data file for performing the
SCF and subsequent CI would then be as follows:
{
\footnotesize
\begin{verbatim}
          TITLE\CUCL .. 3-21G
          ZMAT ANGSTROM\CU\CL 1 CUCL\
          VARIABLES\CUCL 2.093 \END 
          BASIS 3-21G
          RUNTYPE CI
          CORE
          1 TO 14 END
          ACTIVE
          15 TO 42 END
          DIRECT 18 9 19
          CONF
          2 2 2 2 2 2 2 2 2 
          ENTER
\end{verbatim}
}
The inclusion of a second reference configuration corresponding to
the doubly excited configuration
\begin{equation}
 9\sigma^{2}  4\pi^{4}   1\delta^{4} 10\sigma^{2}  5\pi^{4}  12\sigma^{2} 
\end{equation}
would require incorporating the 12$\sigma$ orbital into the internal
space, leading to 10 internal and 18 external MOs. The CONF data
would then appear as follows:

{
\footnotesize
\begin{verbatim}
          CONF
          2 2 2 2 2 2 2 2 2 0
          2 2 2 2 2 2 2 2 0 2 
\end{verbatim}
}
and the overall data file,
{
\footnotesize
\begin{verbatim}
          RESTART NEW
          TITLE\CUCL .. 3-21G
          BYPASS SCF
          ZMAT ANGSTROM\CU\CL 1 CUCL\
          VARIABLES\CUCL 2.093 \END 
          BASIS 3-21G
          RUNTYPE CI
          CORE
          1 TO 14 END
          ACTIVE
          15 TO 42 END
          DIRECT 18 10 18
          CONF
          2 2 2 2 2 2 2 2 2 0
          2 2 2 2 2 2 2 2 0 2
          ENTER
\end{verbatim}
}
where we have, assuming the Mainfile and Dumpfile to have
been retained, by-passed the SCF and modified the DIRECT, CONF and
VECTORS data.

{\bf Example 3}\\

Consider performing a valence-CI calculation on the \silane\ molecule
using a 6-31G(*)  basis. While the molecular symmetry is T$_{d}$, the
symmetry adaptation  and subsequent CI will be conducted in the C$_{2v}$
point group. An examination of the SCF output reveals the following
orbital analysis.

{
\footnotesize
\begin{verbatim}
           =============================
           IRREP  NO. OF SYMMETRY ADAPTED
                  BASIS FUNCTIONS
           =============================
             1           9
             2           6
             3           6
             4           6
           =============================
\end{verbatim}
}
and the following orbital assignments from
the converged closed shell SCF:
{
\footnotesize
\begin{verbatim}
           ===============================================
            M.O. IRREP  ORBITAL ENERGY   ORBITAL OCCUPANCY
           ===============================================
              1     1    -68.77130710           2.0000000
              2     1     -6.12943325           2.0000000
              3     2     -4.23503117           2.0000000
              4     3     -4.23503117           2.0000000
              5     4     -4.23503117           2.0000000
              6     1     -0.73046864           2.0000000
              7     4     -0.48480821           2.0000000
              8     3     -0.48480821           2.0000000
              9     2     -0.48480821           2.0000000
             10     2      0.16291387           0.0000000
             11     3      0.16291387           0.0000000
             12     4      0.16291387           0.0000000
             13     1      0.25681257           0.0000000
             14     1      0.33606346           0.0000000
             15     3      0.37087856           0.0000000
             16     2      0.37087856           0.0000000
             17     4      0.37087856           0.0000000
             18     1      0.79946861           0.0000000
             19     1      0.79946861           0.0000000
             20     4      0.86232544           0.0000000
             21     3      0.86232544           0.0000000
             22     2      0.86232544           0.0000000
             23     1      1.23833149           0.0000000
             24     4      1.44033091           0.0000000
             25     3      1.44033091           0.0000000
             26     2      1.44033091           0.0000000
             27     1      3.13181655           0.0000000
           ===============================================
\end{verbatim}
}
Assume that we wish to freeze the first 5 silicon inner shell orbitals
requiring the following CORE and ACTIVE directives:

{
\footnotesize
\begin{verbatim}
          CORE
          1 TO 5 END
          ACTIVE
          6 TO 27 END
\end{verbatim}
}
To perform a 8-electron valence-CI calculation,
based on the SCF configuration would require the following CONF data:

{
\footnotesize
\begin{verbatim}
          CONF
          2 2 2 2  
\end{verbatim}
}
The complete data file for performing the
SCF and subsequent CI would then be as follows:
{
\footnotesize
\begin{verbatim}
          TITLE
          SIH4 * 6-31G* DIRECT VALENCE-CI 1M/1R
          ZMAT 
          SI
          H 1 SIH
          H 1 SIH 2 109.471
          H 1 SIH 2 109.471 3 120.0
          H 1 SIH 2 109.471 4 120.0
          VARIABLES
          SIH 2.80   
          END
          BASIS 6-31G*
          RUNTYPE CI
          CORE
          1 TO 5 END
          ACTIVE
          6 TO 27 END
          DIRECT 8 4 18
          CONF
          2 2 2 2
          ENTER
\end{verbatim}
}

{\bf Example 4}\\

In this example we wish to perform a valence-CI calculation 
on the \nitrog\ molecule using a 4-31G(*)  basis. While the molecular symmetry
is \dinfh, the symmetry adaptation  and subsequent CI will be conducted in the
\dtwoh\ point group. The resolution of the \dinfh\ into the \dtwoh\
orbital species is given in Table~\ref{table:2}.
An examination of the SCF output reveals the following orbital analysis.

{
\footnotesize
\begin{verbatim}
           =============================
           IRREP  NO. OF SYMMETRY ADAPTED
                  BASIS FUNCTIONS
           =============================
             1           8
             2           3
             3           3
             4           1
             5           8
             6           3
             7           3
             8           1
           =============================
\end{verbatim}
}
and the following orbital assignments from
the converged closed shell SCF:
{
\footnotesize
\begin{verbatim}
           ===============================================
            M.O. IRREP  ORBITAL ENERGY   ORBITAL OCCUPANCY
           ===============================================
              1     1    -15.65951533           2.0000000
              2     5    -15.65474750           2.0000000
              3     1     -1.50615941           2.0000000
              4     5     -0.75782277           2.0000000
              5     1     -0.63244925           2.0000000
              6     3     -0.63135826           2.0000000
              7     2     -0.63135826           2.0000000
              8     6      0.20154861           0.0000000
              9     7      0.20154861           0.0000000
             10     5      0.63883097           0.0000000
             11     1      0.82491489           0.0000000
             12     3      0.89634343           0.0000000
             13     2      0.89634343           0.0000000
             14     1      0.91812387           0.0000000
             15     7      1.10036132           0.0000000
             16     6      1.10036132           0.0000000
             17     5      1.17625689           0.0000000
             18     5      1.66995008           0.0000000
             19     4      1.70518236           0.0000000
             20     1      1.70518236           0.0000000
             21     3      1.91001614           0.0000000
             22     2      1.91001614           0.0000000
             23     8      2.29436539           0.0000000
             24     5      2.29436539           0.0000000
             25     1      2.84356916           0.0000000
             26     7      3.00847817           0.0000000
             27     6      3.00847817           0.0000000
             28     5      3.37447679           0.0000000
             29     1      3.71753400           0.0000000
             30     5      4.09917273           0.0000000
           ===============================================
\end{verbatim}
}
Assume that we wish to freeze the two N1s inner shell orbitals, thus
{
\footnotesize
\begin{verbatim}
          CORE
          1 2 END
          ACTIVE
          3 TO 30 END
\end{verbatim}
}
To perform a 10-electron valence-CI calculation,
based on the SCF configuration  
\begin{equation}
 2\sigma_g^{2}  2\sigma_u^{2}  3\sigma_g^2  1\pi_u^{4}
\end{equation}
and associated $\pi$ to $\pi^{*}$ excitations
\begin{equation}
 2\sigma_g^{2}  2\sigma_u^{2}  3\sigma_g^2  1\pi_{u,x}^2 1\pi_{g,y}^2
\end{equation}
\begin{equation}
 2\sigma_g^{2}  2\sigma_u^{2}  3\sigma_g^2  1\pi_{u,x}^2 1\pi_{g,x}^2
\end{equation}
\begin{equation}
 2\sigma_g^{2}  2\sigma_u^{2}  3\sigma_g^2  1\pi_{u,y}^2 1\pi_{g,x}^2
\end{equation}
\begin{equation}
 2\sigma_g^{2}  2\sigma_u^{2}  3\sigma_g^2  1\pi_{u,y}^2 1\pi_{g,y}^2
\end{equation}
\begin{equation}
 2\sigma_g^{2}  2\sigma_u^{2}  3\sigma_g^2  (1\pi_{u,x} 1\pi_{g,x}) (1\pi_{u,y} 1\pi_{g,y})
\end{equation}
would require the following CONF data:
{
\footnotesize
\begin{verbatim}
          CONF
          2 2 2 2 2  0 0
          2 2 2 0 2  0 2
          2 2 2 0 2  2 0
          2 2 2 2 0  2 0
          2 2 2 2 0  0 2
          2 2 2 1 1  1 1
\end{verbatim}
}
with an internal space of 7 orbitals, an external space of 21; note
again that the ordering of the virtual MOs is such that
no reordering is required within the ACTIVE directive.
The complete data file for performing the
SCF and subsequent CI would then be as follows:

{
\footnotesize
\begin{verbatim}
          TITLE\N2 .. 4-31G*
          ZMAT ANGS\N\N 1 NN
          VARIABLES\NN 1.05 \END
          BASIS 4-31G*
          RUNTYPE CI
          CORE
          1 2 END
          ACTIVE
          3 TO 30 END
          DIRECT 10 7 21
          CONF
          2 2 2 2 2  0 0
          2 2 2 0 2  0 2
          2 2 2 0 2  2 0
          2 2 2 2 0  2 0
          2 2 2 2 0  0 2
          2 2 2 1 1  1 1
          NATORB 10 0 PRINT
          ENTER
\end{verbatim}
}

{\bf Example 5}\\

In this example we wish to perform a valence-CI calculation 
on the \cah\
molecule using a 3-21G  basis. While the molecular symmetry
is D$_{\infty h}$, the symmetry adaptation  and subsequent CI 
will be conducted in the
D$_{2h}$ point group. An examination of the SCF output
reveals the following orbital analysis.

{
\footnotesize
\begin{verbatim}
           =============================
           IRREP  NO. OF SYMMETRY ADAPTED
                  BASIS FUNCTIONS
           =============================
             1           7
             2           4
             3           4
             5           6
           =============================
\end{verbatim}
}
and the following orbital assignments from
the converged closed shell SCF:
{
\footnotesize
\begin{verbatim}
          ===============================================
           M.O. IRREP  ORBITAL ENERGY   ORBITAL OCCUPANCY
          ===============================================
             1     1   -148.37173884           2.0000000
             2     1    -16.76521275           2.0000000
             3     3    -13.55586861           2.0000000
             4     2    -13.55586861           2.0000000
             5     5    -13.55460610           2.0000000
             6     1     -2.26357685           2.0000000
             7     3     -1.36160958           2.0000000
             8     2     -1.36160958           2.0000000
             9     5     -1.35089927           2.0000000
            10     1     -0.34923025           2.0000000
            11     5     -0.31649941           2.0000000
            12     2      0.02334207           0.0000000
            13     3      0.02334207           0.0000000
            14     1      0.04980631           0.0000000
            15     5      0.09478404           0.0000000
            16     1      0.12395484           0.0000000
            17     3      0.13549605           0.0000000
            18     2      0.13549605           0.0000000
            19     5      0.28345574           0.0000000
            20     1      1.32404002           0.0000000
            21     5      1.45900204           0.0000000
          ===============================================
\end{verbatim}
}
Assume that we wish to freeze the nine Ca inner shell orbitals, thus
{
\footnotesize
\begin{verbatim}
          CORE
          1 TO 9 END
          ACTIVE
          10 TO 21 END
\end{verbatim}
}
To perform a 4-electron valence-CI calculation,
based on the SCF configuration would simply require 
the following CONF data:

{
\footnotesize
\begin{verbatim}
          CONF
          2 2
\end{verbatim}
}
The complete data file for performing the
SCF and subsequent CI would then be as follows:
{
\footnotesize
\begin{verbatim}
          TIME 60 
          TITLE\CAH2 .. 3-21G
          ZMAT ANGS\CA\X 1 1.0\ H 1 CAH 2 90.0\H 1 CAH 2 90.0 3 THETA
          VARIABLES\CAH 2.148 \THETA 180.0 \END
          BASIS 3-21G
          RUNTYPE CI
          CORE
          1 TO 9 END
          ACTIVE
          10 TO 21 END
          DIRECT 4 2 10
          CONF
          2 2
          ENTER
\end{verbatim}
}

\subsection[CONF-CARDS]{CONF-CARDS }

The directive consists of s single line, where the first 2 character strings are either
CONF CARDS or CONF FILE or CONF ASCII .  On the same line sub-directives
may be specified,
\begin{itemize}
\item FILE  string

A 44 charactrer string may be supplied, specifying the file to read the configuration
information from. The file must be as generated by the CARDS CASSCF directive i.e. 
fiexed format like
\begin{verbatim}
      1   222220000                    .9644755897
\end{verbatim}
\item COEF value

Specify a minimum absolute value for a coefficient of a configuration to be 
included. Different spin-paths are combined.
\item WEIGHT value

Specify the minimum weight of a configurations (including all its spin possibilities)
to be included.
\item DOC ndoc

Specify the number of doubly occupied orbitals not in the configurations in the list
to be prepended to each configuration.
\item NFRZ nfrz

Specify the number of orbitals to be frozen and kept doubly occupied.
NFRZ and NDOC can cancel each other.
\end{itemize}
The reference configurations are printed, so the working of this directive is easily checked.

\subsection[REFGEN]{REFGEN }

 The first line should consist of the character string REFGEN in the first
data field. Subsequent lines are read using I-format, to paired integers
IA and IC. As many such lines as required may be presented.
IA and IC define annihilation and creation operators respectively,
which will be allowed to operate on the set of reference CSFs 
in existence
at the time when REFGEN is called. Thus at least one CONF directive must
have been presented before using REFGEN, and the IA ,IC integers refer to
internal MOs. The result of the annihilation/creation process will be
further reference CSFs, whose excitation mask will be that of the 
most recently issued EXCIT directive.\\

{\bf  Example 1}
{
\footnotesize
\begin{verbatim}
          EXCIT OCTAL 7 3 1
          CONF
          2 2 0 0
          EXCIT OCTAL 5
          REFGEN
          1 3
          2 4
\end{verbatim}
}

will result in a further two reference states, to give three in all,
of the form:
{
\footnotesize
\begin{verbatim}
          CSFS       EXCITATION MASK        ORIGIN 
         =======     ===============        =============
         2 2 0 0     7    3    1            CONF directive
         1 2 1 0     5                      REFGEN directive
         2 1 0 1     5                      REFGEN directive
\end{verbatim}
}


{\bf Example 2}
{
\footnotesize
\begin{verbatim}
          EXCIT OCTAL 7 3 1
          CONF
          2 2 0 0
          EXCIT OCTAL 5
          REFGEN
          1 3 2 4
\end{verbatim}
}

is equivalent to example 1; more than one IA/IC pair may be given on
a single line.\\



{\bf Example 3}
{
\footnotesize
\begin{verbatim}
          EXCIT OCTAL 7 3 1
          CONF
          2 2 0 0
          EXCIT OCTAL 5
          REFGEN
          1 3
          EXCIT OCTAL 3
          REFGEN
          2 4
\end{verbatim}
}
will produce 4 reference CSFs, of the form:
{
\footnotesize
\begin{verbatim}
          CSFS        EXCITATION MASK        ORIGIN 
          =======     ===============        =====================
          2 2 0 0     7 3 1                  CONF directive
          1 2 1 0     5                      First REFGEN directive
          2 1 0 1     3                      Second REFGEN directive
          1 1 1 1     3                      Second REFGEN directive
\end{verbatim}
}


{\bf Example 4}
{
\footnotesize
\begin{verbatim}
          EXCIT 7 3 1
          CONF
          2 2 0 0
          REFGEN
          1 3 1 4 2 3 2 4
          REFGEN
          1 3 1 4 2 3 2 4
          REFGEN
          1 3 1 4 2 3 2 4
          REFGEN
          1 3 1 4 2 3 2 4
\end{verbatim}
}

will produce a reference space consisting of all possible CSFs
that can be generated by distributing 4 electrons in 4 MOs.

\subsection[CASGEN]{CASGEN }

The CASGEN directive is on 1 card and is meant to produce a
selective Complete Active Space reference function. More than
one CASGEN directives may be given. Each one working with respect
to the previous one. The first word on the card is the
textstring CASGEN (A).

The following may be specified on the same card :
\begin{itemize}
\item DOC NDOC (A,I) : Specifies number of orbitals to remain 
                       doubly occupied
\item SDOC ND1 ND2 .. (A,I,I,): As DOC but per symmetry; The number
                                of integers is equal to the number 
                                of representations.
      DOC and SDOC are mutually exclusive
\item NORB NORBC (A,I): Number of orbitals taken into account
\item MAXEX MAX (A,I): Maximum excitation level with respect to the
                       current reference set
\item EXCIT (A): Use current excitation allowance for the newly
                 generated configurations only, Often sensible if 
                 restrictions (like DOC) are employed. The default is
                 the excitation mask 7b (no internal excitations), which
                 is appropriate for a real CAS.
\item NOSYM (A): Do not select the configurations on symmetry (default)
\item SYM   (A): Select configurations on symmetry
\item SPIN  (A): Select configurations on spin-symmetry
\item PRINT (A): Print the generated reference configurations
\end{itemize}

To generate a 4-orbital casscf wavefunction for water, having the
3 inner orbitals doubly occupied in the reference function,
and doing a Single-Double CI from this reference function one specifies
{
\footnotesize
\begin{verbatim}
          EXCIT OCTAL 7 3 1
          CASGEN DOC 3 NORB 7 EXCIT
\end{verbatim}
}

\subsection[SCREEN]{SCREEN}

This directive consists of one line whose first data field should
contain the character string SCREEN. If this directive is presented
all reference configurations will be checked to see if they are
of the same spin/space symmetry as that of the required CI wavefunction.
If they are not, they will be eliminated from the reference space,
and will take no part in the excitation process. Normally, this directive
may only be required if a REFGEN directive is presented, since
reference states of undesired properties will not usually be presented
under control of the CONF directive.


\subsection[RESTRICT]{ RESTRICT}

 This directive provides a means of selectively eliminating CSFs from
the CI expansion by specifying a minimum and maximum number of
electrons which user specified sets of internal MOs may carry. The
first line should contain the character string RESTRICT in the first
data field.
Subsequent lines are read to variables TEXT, MIN, MAX,(IORBS(i),i=1,m),
using format (A,2I,mI).
 TEXT should be set to one of the character strings REF, VAC, N-1
or N-2. If REF is chosen the restrictions will apply to the reference
CSFs; should such a state fail to comply with the restriction applied
it will take no part in the excitation process used to generate the CI
list, and will most probably be used when REFGEN directives are used.
The character strings VAC, N-1 and N-2 refer to CSFs with 0,1 or 2
electrons in the external space respectively.
 MIN,MAX specify the minimum and maximum number of electrons to be
allowed to populate the internal MOs defined by the IORBS parameters.
 IORBS A sequence of internal MO indices terminated by the integer 0.
This data may be carried over to subsequent lines if necessary,
and the character string TO may be used to shorten the data if desired.
These parameters refer to the reordered MO list if a REORDER
directive (see above) has been presented.\\

{\bf Example}
{
\footnotesize
\begin{verbatim}
          REFGEN
          VAC 7 8 1 TO 4 0
          N-1 7 8 1 TO 4 0
          N-2 7 8 1 TO 4 0
\end{verbatim}
}

 Internal MOs 1 to 4 inclusive are allowed to carry either 7 or 8 electrons
in vacuum, N-1 and N-2 CSFs.


\subsection[PRCONF]{PRCONF}

This directive consists of one line, read to variables TEXT, IPR
using format (A,I).
\begin{itemize}
\item  TEXT should be set to the character string PRCONF.
\item  IPR specifies that every IPR'th occupation pattern
generated by the configuration generator is to be printed.
If IPR=1, all occupation patterns will be printed.
\end{itemize}

 The directive may be omitted, when no occupation patterns generated
by the configuration generator will be printed.
The main use of the PRCONF directive is to generate a detailed
occupation pattern listing.\\

{\bf Example}
{
\footnotesize
\begin{verbatim}
          PRCONF 1
\end{verbatim}
}

\subsection[CEPA]{CEPA}

 The CEPA directive allows the user to calculate the unlinked cluster
correction to the final CI energy. This correction factor is more
accurate than the Davidson correction factor, as it is done in 
an iterative way and the correction (shift) may be different for  
different n-2 states. Both the classical CEPA variants (0,1,2) for
closed shell single determinant reference states and Multi-Reference
variants  are provided.

This directive consists of a single data line read to the variables 
TEXT, CEPA-variant (A,A) followed by optional options.  
TEXT should specify the character string CEPA and
CEPA-variant is the text-string specifying which CEPA (or
approximated Coupled Cluster) approach is requested. Options are:
\begin{itemize}
\item 0    : CEPA(0) (closed shell; shift is correlation energy)
      Cf.    \cite{ref:41}.
\item 1    : CEPA(1) (closed shell; recommended;default)
      Cf.    \cite{ref:41}.
\item 2    : CEPA(2) (closed shell; not invariant for mixing of 
             occupied orbitals) Cf. \cite{ref:41}.
\item MR0  : Straight multi-reference variant of CEPA(0)
\item ACPF : Averaged Coupled Pair Functional (Shift is 
             modified correlation energy) \cite{ref:ACPF}
             The correlation energy is (default) the difference with the
             (variationally determined) reference function.
\item AQCC : Averaged Quadratic Coupled Cluster (Shift is 
             modified correlation energy) \cite{ref:AQCC}
             The correlation energy is (default) the difference with the
             (variationally determined) reference function.
\item MRD  : Multi Reference CEPA (taking Variationally Included
             (VI) terms into account) \cite{ref:51}
\item MR1  : Multi Reference CEPA (taking VI and EPV terms 
             into account) \cite{ref:Szalay,ref:MRCEPA1}
             The EPV terms may be determined for the "inactive"
             space (DOC) or for all occupied orbitals (ALL; default);
             The choice may be may by specifying the additional
             string DOC or ALL.
\end{itemize}

Additional options, applicable to some or all of the variants 
mentioned above may be specified on the same card :
\begin{itemize}
\item For the approaches that use the (modified) correlation energy
      one may specify how this is calculated. Choices are PROJECT,
      where the expectation value of the "CI"-function projected onto
      the reference space is used as reference, VARIA, where
      the energy of the optimised reference space is employed or
      PAIRS, computing the correlation energy as sum of pair energies
      (this is (~) project).
      The default is VARIA, which is Ahlrichs' choice.
\item Using the keywords SIN or NOSIN, one may specify if single
      excitations should be shifted (SIN) or not. The default is
      that singles are shifted, except for single reference CEPA2.
      For the multi-reference variant, the option is dubious.
\item Using the texstring IT one may specify
      after what iteration the CEPA mode may start, using 
      IT ITCEPA (A,I).  Default is 3.
\item Using the textstring CRIT, one may specify at which 
      particular threshold determined by the TESTER in the
      diagonalisation phase of the CI calculation the CEPA is switched
      on. The format is : CRIT CRITC (A,F). Default of CRITC is 0.01.
\item PRINT requests intermediate printing within the CEPA mode.
\item PAUL refers only to the CEPA 2 mode, and invokes an 
      unpublished EPV correction formula due to P.J.A. Ruttinck.
\item MICRO allows one to control the CEPA micro iterations, when
      the CI vector and correlation energy and shifts are updated
      without a matrix-vector product in between. The format is
      MICRO MCYC CRIT (A,I,F). MCYC is the maximum number of micro
      iterations and CRIT the relative convergence criterion.
      Defaults are  3 and 0.01.
\end{itemize}

The directive can be invoked without parameters, which will result 
in the following and crash for a multi-reference case:
{
\footnotesize
\begin{verbatim}
          CEPA 1 SIN IT 3 CRIT 0.01 MICRO 3 0.01 
\end{verbatim}
}

\subsection[MP]{MP}

The MP directive allows one to perform multi-reference M{\o}ller-Plesset
calculations \cite{ref:48,ref:49,ref:50}. To generate the reference wavefunction
one should run an MCSCF calculation on the required state first to obtain the
correct orbitals and use the TRIAL DIAG directive to rebuild the MCSCF 
wavefunction. Details on this process will be given in an example in the
Direct-CI - Multi-reference MP section of Part 2.

Once the reference wavefunction
has been constructed the MP directive should be used to control the perturbation
theory applied to it. The syntax is
{
\footnotesize
\begin{verbatim}
          TEXT [ TEXTA ] [ TEXTB TEXTC or IMODEL ] [ TEXTD M E ]
\end{verbatim}
}
where
\begin{itemize}
\item TEXT should specify the character string MP
\item TEXTA specifies the order of perturbation theory required. One can choose
      from '2' and '3'. 
\item TEXTB should specify the character string MODEL to choose the form of
      the zeroth order Hamiltonian. This Hamiltonian can be specified by name
      using 
      \begin{itemize}
      \item 'RUTTINK' to select $H^{(0)}_{S,D-minimal}$ in \cite{ref:48}
            in honour of his introduction of the excitation classes in MRCEPA
            \cite{ref:51}
      \item 'PULAY' to select $H^{(0)}_{S,D}$ in \cite{ref:48} in honour
            of his work on MRMP methods \cite{ref:50}
      \item 'ANDERSSON' to select the Pulay zeroth order Hamiltonian but 
            removing the excitations within the reference space. The name was
            chosen in honour of her work on CASPT2 \cite{ref:52}
      \end{itemize}
      Alternatively the model may be specified by a number where
      \begin{itemize}
      \item the units stand for; 1 using a projector operator on the combined
            single and double excitation space, 2 same as 1 but eliminating
            all single excitations, 3 using projector operators onto the space
            of single and doubly excitations separately
      \item the decades stand for; 0 no modification, 1 removing all excitations
            within the reference space
      \item the sign stands for; + no modification, - removing all parts of 
            the zeroth order Hamiltonian that connect different excitation
            classes.
      \end{itemize}
      i.e. 'RUTTINK' can also be entered as -1, 'PULAY' as 1, and 'ANDERSSON'
      as 11. 
\item TEXTD can be used to specify algorithm to orthogonalise the set of 
      single and double excitations. One can choose from 'SCHMIDT' for 
      modified Gramm-Schmidt orthogonalisation, 'LOWDIN' for orthogonalising
      by diagonalisation of the overlap matrix, or 'HOUSEHOLDER' for applying 
      the House-Holder method. The floating point number M and the integer E
      specify the orthogonalisation accuracy $M*10^{-E}$.
\end{itemize}
The default settings can be specified by either of the 2 following
{
\footnotesize
\begin{verbatim}
          MP
          MP 2 RUTTINK HOUSEHOLDER 1. 8
\end{verbatim}
}

\subsection[Spin Functions]{Spin Functions}

 It may be necessary for the user to understand the nature and order
of spin functions associated with a given occupation pattern.
The program makes use of Yamanouchi-Kotani genealogical spin functions,
the coupling order being such that higher indexed MOs are coupled before
lower indexed MOs. The ordering of the MOs is as defined using the ACTIVE
directive of the transformation module (see above).
 Use the digits 0 and 1 to denote down and up spin coupling respectively.
Proceeding from the highest to the lowest indexed singly occupied MO,
write down the digitized representation of the possible spin functions, the
digits being written from left to right. The resultant binary number defines
the lexical ordering of the members of the spin canonical set, the higher
the number, the higher the lexical index.\\

{\bf Example}\\

Consider 5 doubly occupied MOs coupled to a doublet.
The possible spin functions in digitized representation are, in order of
increasing lexical index:
\begin{itemize}
\item    10101   -   Spin function 1
\item    10110   -   Spin function 2
\item    11001   -   Spin function 3
\item    11010   -   Spin function 4
\item    11100   -   Spin function 5

\end{itemize}

\subsection[Using GVB Orbitals]{Using GVB Orbitals}

In this section we briefly outline specification of a multi-reference
direct-CI based on a GVB-1/PP wavefunction.  Consider again the example
of section 4.4.2, where a 4-pair GVB/PP calculation on \formaldehyde\
is described, in which the two C-H bonds and two C-O orbitals are treated
within the perfect pairing approximation.  The sequence of calculations
included:
\begin{itemize}
\item performing  the closed shell SCF calculation;
\item localising  the set of valence SCF orbitals;
\item performing the GVB calculation using the set of LMO
input under control of the VECTORS option NOGEN.
\end{itemize}
An examination of the GVB output reveals the following orbital
assignments, with orbitals 5, 7, 9 and 11 corresponding to the strongly
occupied orbitals, and orbitals 6, 8, 10 and 12 to the weakly occupied
orbitals of the GVB pairs.

{
\footnotesize
\begin{verbatim}
          ===============================================
           M.O. IRREP  ORBITAL ENERGY   ORBITAL OCCUPANCY
          ===============================================
             1     1    -20.48204464           2.0000000
             2     1    -11.25090140           2.0000000
             3     1     -1.12106354           2.0000000
             4     1     -0.55607328           2.0000000
             5     1     -1.42285552           1.9843054
             6     1     -0.02615928           0.0156946
             7     1     -1.42285519           1.9843052
             8     1     -0.02615932           0.0156948
             9     1     -1.93008469           1.9904691
            10     1     -0.02325080           0.0095309
            11     1     -1.07958389           1.9032736
            12     1     -0.09164423           0.0967264
            13     1      0.62048723           0.0000000
            14     1      0.87018377           0.0000000
            15     1      0.87310154           0.0000000
            16     1      0.92519148           0.0000000
            17     1      1.03015198           0.0000000
            18     1      1.38072957           0.0000000
            19     1      1.79683989           0.0000000
            20     1      1.81440429           0.0000000
            21     1      1.97615975           0.0000000
            22     1      3.26852884           0.0000000
          ===============================================
\end{verbatim}
}
Assume that we wish to freeze the O1s and C1s orbitals, thus
{
\footnotesize
\begin{verbatim}
          CORE\1 2 \END
          ACTIVE\3 TO 22\END
\end{verbatim}
}
To perform a 12-electron valence-CI calculation  based on the
leading term in the GVB expansion, together with those
doubly excited configurations corresponding to each GVB pair, 
would  require  the following CONF data:

{
\footnotesize
\begin{verbatim}
          CONF
          2 2  2 0 2 0 2 0 2 0
          2 2  0 2 2 0 2 0 2 0
          2 2  2 0 0 2 2 0 2 0
          2 2  2 0 2 0 0 2 2 0
          2 2  2 0 2 0 2 0 0 2
\end{verbatim}
}
with 10 orbitals in the internal space and 10 in the external space,
given that the inner shell orbitals have been frozen.  The complete
date file for performing the GVB-CI would then appear as follows, where
the canonicalised GVB orbitals are restored from the default section,
section 5, of the Dumpfile.

{
\footnotesize
\begin{verbatim}
          RESTART NEW
          TITLE
          H2CO - 3-21G  4PAIR GVB
          SUPER OFF NOSYM
          BYPASS SCF
          ADAPT OFF
          ZMATRIX ANGSTROM
          C
          O 1 1.203
          H 1 1.099 2 121.8
          H 1 1.099 2 121.8 3 180.0
          END
          RUNTYPE CI
          SCFTYPE GVB 4
          CORE\1 2 \END
          ACTIVE\3 TO 22\END
          DIRECT 12 10 10
          CONF
          2 2  2 0 2 0 2 0 2 0
          2 2  0 2 2 0 2 0 2 0
          2 2  2 0 0 2 2 0 2 0
          2 2  2 0 2 0 0 2 2 0
          2 2  2 0 2 0 2 0 0 2
          ENTER
\end{verbatim}
}
We show below the output from the resulting CI calculation.
{
\footnotesize
\begin{verbatim}
             TOTAL ENERGY  -113.4467990858

   0.95233628        1   VACUUM    0   0        1  -- 2220202020
  -0.08236057        1   VACUUM    0   0        2  -- 2202202020
  -0.08236123        1   VACUUM    0   0        3  -- 2220022020
  -0.05407166        1   VACUUM    0   0        4  -- 2220200220
  -0.16888657        1   VACUUM    0   0        5  -- 2220202002
  -0.03606575        1   VACUUM    0   0      163  -- 2121202011
  -0.03606465        1   VACUUM    0   0      224  -- 2120212011
  -0.07796058        1   VACUUM    0   0      279  -- 2220201111
 
 
 SUM OF SQUARES OF VACUUM CSF CI COEFFICIENTS=   0.973137161834E+00
 SUM OF SQUARES OF     5-MAIN CI COEFFICIENTS=   0.951957435390E+00
\end{verbatim}
}


\subsection[FP Geometry Optimisations]{FP Geometry Optimisations}

Energy-only optimisation for direct-CI wavefunctions may be performed 
using a variant of the RUNTYPE OPTIMIZE directive. The data line

{
\footnotesize
\begin{verbatim}
          RUNTYPE OPTIMIZE CI
\end{verbatim}
}
requests use of the Fletcher Powell (FP) optimiser, with subsequent
data used to characterize the direct-CI wavefunction to be
employed in the energy calculation.  We illustrate such usage below
for the case of a direct-CI calculation on the \formaldehyde\
cation, performing the calculation in several steps.
The first two steps carry out an RHF open shell geometry
optimisation, and the third the corresponding CI optimisation.\\

{\bf Runs I and II: The SCF Optimisation}
{
\footnotesize
\begin{verbatim}
          TITLE
          H2CO - TZVP - CLOSED SHELL STARTUP
          ZMATRIX ANGSTROM
          C
          O 1 CO
          H 1 CH 2 HCO
          H 1 CH 2 HCO 3 180.0
          VARIABLES
          CO 1.203
          CH 1.099
          HCO 121.8
          END
          BASIS TZVP
          ENTER
\end{verbatim}
}
The first step is merely used to generate a suitable set of
MOS for initiating the SCF geometry optimisation on the
ion below.

{
\footnotesize
\begin{verbatim}
          RESTART NEW
          TITLE
          H2CO+ - TZVP - GEOMETRY OPTIMISATION SCF
          MULT 2
          CHARGE 1
          ZMATRIX ANGSTROM
          C
          O 1 CO
          H 1 CH 2 HCO
          H 1 CH 2 HCO 3 180.0
          VARIABLES
          CO 1.203
          CH 1.099
          HCO 121.8
          END
          BASIS TZVP
          RUNTYPE OPTIMIZE 
          OPEN 1 1
          ENTER 
\end{verbatim}
}

{\bf Run III: The Direct-CI Calculation}
{
\footnotesize
\begin{verbatim}
          RESTART NEW
          TITLE
          H2CO+ - TZVP - CI/1M AT RHF GEOMETRY
          MULT 2
          CHARGE 1
          ZMATRIX ANGSTROM
          C
          O 1 CO
          H 1 CH 2 HCO
          H 1 CH 2 HCO 3 180.0
          VARIABLES
          CO        1.2063534 HESS     .792378
          CH        1.0876643 HESS     .681619
          HCO     117.8503752 HESS     .709462
          END
          BASIS TZVP
          RUNTYPE OPTIMIZE CI
          OPEN 1 1 
          DIRECT 15 8 34
          CONF
          2 2 2 2 2 2 2 1 
          ENTER
\end{verbatim}
}

Note again that some care must be taken when reducing the orbital space
in FP CI optimisations. In open shell calculations, the CI step will
derive the orbital set at each point from the second section specified
on the ENTER directive i.e., the energy ordered MOs. If this ordering
varies from point to point in the FP optimisation, and symmetry is used
in minimising the configuration space, it is quite likely that this space
will vary during successive points, with disastrous consequences on the
optimisation pathway. As a general rule, the user should only consider
freezing or discarding orbitals that are well separated from those MOs
included in the CI space i.e. inner shell or inner-shell complement MOs.

\subsection[Calculating the \astate\ states of \water]{Calculating the \astate\ states of \water}

To conclude our discussion of the Direct-CI module, we work through
a typical example of using the Direct-CI method in calculating
the energetics and properties of the three low lying $^{1}$A$_{1}$
states of the \water\ molecule. The basis set employed is the
TZVP triple-zeta plus polarisation set; this is augmented
with a diffuse s- and p-orbital on the oxygen to provide
a reasonable description of the known Rydberg character of
the states of interest.
The computation is split into a number of separate jobs, in which we
\begin{enumerate}
\item perform the initial SCF;
\item carry out an initial CI for each state, 
where the reference set employed
acts to provide at least a qualitative description of the states
of interest;
\item based on the output from the initial CIs, we augment the
reference set to provide a quantitative description of the
first three states.
\end{enumerate}
We now consider various aspects of each job in turn.\\

{\bf Job 1: The SCF}
{
\footnotesize
\begin{verbatim}
          TITLE  
          ****  H2O  TZVP + DIFFUSE S,P SCF *
          SUPER OFF NOSYM
          ZMAT ANGSTROM
          O
          H 1 0.951
          H 1 0.951 2 104.5
          END
          BASIS 
          TZVP O
          TZVP H
          S O
          1.0 0.02
          P O
          1.0 0.02
          END
          ENTER
\end{verbatim}
}
The only point to note here is the use of the SUPER directive in
suppressing skeletonisation, given the user wishes to access this file
in a subsequent CI step under BYPASS control.\\

{\bf Jobs 2-4: The Initial CIs}\\

An examination of the SCF output reveals the following orbital analysis.
{
\footnotesize
\begin{verbatim}
          =============================
          IRREP  NO. OF SYMMETRY ADAPTED
                 BASIS FUNCTIONS
          =============================
            1          18
            2           6
            3          10
            4           2
          =============================
\end{verbatim}
}
and the following orbital assignments characterising the closed--shell
SCF configuration:
\begin{equation}
  1a_{1}^{2}  2a_{1}^{2}  1b_{2}^{2}  3a_{1}^{2}  1b_{1}^{2}
\end{equation}
{
\footnotesize
\begin{verbatim}
          ===============================================
           M.O. IRREP  ORBITAL ENERGY   ORBITAL OCCUPANCY
          ===============================================
             1     1    -20.56084959           2.0000000
             2     1     -1.35696939           2.0000000
             3     3     -0.72200122           2.0000000
             4     1     -0.58247942           2.0000000
             5     2     -0.50858566           2.0000000
             6     1      0.02724259           0.0000000
             7     3      0.04894440           0.0000000
             8     2      0.05589681           0.0000000
             9     1      0.06133571           0.0000000
            10     1      0.20403420           0.0000000
            11     3      0.22824210           0.0000000
            12     3      0.53700802           0.0000000
            13     1      0.56235022           0.0000000
            14     2      0.58645643           0.0000000
            15     1      0.66887228           0.0000000
            16     3      0.74805617           0.0000000
            17     1      1.07690608           0.0000000
            18     1      1.88545053           0.0000000
            19     4      1.92243836           0.0000000
            20     2      2.12944874           0.0000000
            21     3      2.20541910           0.0000000
            22     1      2.34202871           0.0000000
            23     3      2.39946430           0.0000000
            24     3      2.69788310           0.0000000
            25     1      2.72651832           0.0000000
            26     2      2.73832720           0.0000000
            27     1      3.07664215           0.0000000
            28     3      3.26840142           0.0000000
            29     2      3.54616570           0.0000000
            30     1      3.58631019           0.0000000
            31     4      3.59701772           0.0000000
            32     1      3.84174131           0.0000000
            33     1      4.84610143           0.0000000
            34     3      5.14220270           0.0000000
            35     1      7.73115986           0.0000000
            36     1     47.56758932           0.0000000
          ===============================================
\end{verbatim}
}
Assuming that we wish to freeze the O1s inner shell orbitals and discard
the inner shell complement orbital, the following data lines
should be presented in the transformation:

{
\footnotesize
\begin{verbatim}
          CORE\1\END
          ACTIVE\2 TO 35\END
\end{verbatim}
}
Note that the virtual SCF MOs dominated by the diffuse oxygen
basis functions are the 4a$_{1}$, the 2b$_{2}$, the 2b$_{1}$ and the
5a$_{1}$, with SCF sequence numbers 6,7,8 and 9 respectively. 
The  re-ordered sequence numbers, allowing for the
effective removal of the two a$_{1}$ orbitals, are 5,6,7 and 8
respectively.
To perform a balanced valence-CI treatment of the three states 
of interest will require a three-root 8-electron  reference set,
based on the SCF configurations of the ground and 
excited Rydberg states, involving the single excitations
(1b$_{1}$ to 2b$_{1}$) and (3a$_{1}$ to 4a$_{1}$). This specification
will require the following CONF data:

{
\footnotesize
\begin{verbatim}
          CONF
          2 2 2 2 0 0 0 0
          2 2 2 1 0 0 1 0
          2 2 1 2 1 0 0 0
\end{verbatim}
}
In contrast to the Table-CI module (see Part 6), where 
the three-state CI may be performed in a single job, the Direct-CI
treatment will require three separate jobs, each job looking
to describe a specific state. This is achieved in the case
of the excited states through use of the
TRIAL directive, which identifies the particular state
under investigation. The data files for these three jobs 
are given below:\\

{\bf Job 2: Direct-CI Treatment of the \xastate}
{
\footnotesize
\begin{verbatim}
          RESTART NEW
          TITLE  
          ****  H2O  TZVP + DIFFUSE S,P DIRECT-CI 3M X1A1*
          SUPER OFF NOSYM
          BYPASS SCF
          ZMAT ANGSTROM
          O
          H 1 0.951
          H 1 0.951 2 104.5
          END
          BASIS 
          TZVP O
          TZVP H
          S O
          1.0 0.02
          P O
          1.0 0.02
          END
          RUNTYPE CI
          CORE\1\END
          ACTIVE\2 TO 35\END
          DIRECT 8 8 26
          CONF
          2 2 2 2 0 0 0 0
          2 2 2 1 0 0 1 0
          2 2 1 2 1 0 0 0
          ENTER
\end{verbatim}
}
The following points should be noted:
\begin{itemize}
\item the SCF computation is BYPASS'ed;
\item the CORE and ACTIVE directives act to
freeze and discard  the two a$_{1}$ MOs;
\item the parameters on the DIRECT data line reflect the number
of active electrons (8), number of internal orbitals (8) and
the number of external orbitals (26);
\item in this job we are describing the \xastate, so that the
default diagonalisation controls will prove satisfactory.
\end{itemize}
{\bf Job 3: Direct-CI Treatment of the 1$^{1}$A$_{1}$ state}
{
\footnotesize
\begin{verbatim}
          RESTART CI
          TITLE  
          ****  H2O  TZVP + DIFFUSE S,P DIRECT-CI 3M 1A1*
          SUPER OFF NOSYM
          BYPASS TRANSFORM
          ZMAT ANGSTROM
          O
          H 1 0.951
          H 1 0.951 2 104.5
          END
          BASIS 
          TZVP O
          TZVP H
          S O
          1.0 0.02
          P O
          1.0 0.02
          END
          RUNTYPE CI
          CORE\1\END
          ACTIVE\2 TO 35\END
          DIRECT 8 8 26
          CONF
          2 2 2 2 0 0 0 0
          2 2 2 1 0 0 1 0
          2 2 1 2 1 0 0 0
          SHIFT 0.5\ALTERNATE
          TRIAL
          2 1.0
          VMIN
          ENTER
\end{verbatim}
}
The following points should be noted:
\begin{itemize}
\item the SCF and Transformation are BYPASS'ed: this assumes
that the Transformed Integral file from the initial CI has
been saved;
\item in this job we are describing an excited state, the 
1$^{1}$A$_{1}$, so that the
default diagonalisation controls will no longer prove satisfactory.
The specification of
SHIFT, ALTERNATE and VMIN is typical in such calculations.
The TRIAL directive
is now specifying a starting vector with the second configuration
in the CONF list as the dominant term.
\end{itemize}
{\bf Job 4: Direct-CI Treatment of the 2$^{1}$A$_{1}$ state}
{
\footnotesize
\begin{verbatim}
          RESTART CI
          TITLE  
          ****  H2O  TZVP + DIFFUSE S,P DIRECT-CI 3M 2A1*
          SUPER OFF NOSYM
          BYPASS TRANSFORM
          ZMAT ANGSTROM
          O
          H 1 0.951
          H 1 0.951 2 104.5
          END
          BASIS 
          TZVP O
          TZVP H
          S O
          1.0 0.02
          P O
          1.0 0.02
          END
          RUNTYPE CI
          CORE\1\END
          ACTIVE\2 TO 35\END
          DIRECT 8 8 26
          CONF
          2 2 2 2 0 0 0 0
          2 2 2 1 0 0 1 0
          2 2 1 2 1 0 0 0
          TRIAL
          3  1.0
          SHIFT 0.5\ALTERNATE
          VMIN
          ENTER
\end{verbatim}
}
The following points should be noted:
\begin{itemize}
\item the SCF and Transformation are again BYPASS'ed;
\item we are again describing an excited state, the 
2$^{1}$A$_{1}$, so that 
SHIFT, ALTERNATE and VMIN are again specified.
The TRIAL directive
is now specifying a starting vector with the third configuration
in the CONF list as the dominant term.
\end{itemize}

{\bf The Final 16-Reference CI Jobs}\\

An examination of the output from the initial CI calculations reveals that
the dominant configurations have, as expected, been included.  We show
below the final CI vectors for each of the states: not surprisingly
the ground state is more accurate, by virtue of its SCF MOs having been
employed. Augmenting the reference set to improve the description of the
two excited states follows straightforwardly from the statistics below: \\

{\bf Description of the X$^{1}$A$_{1}$ state}
{
\footnotesize
\begin{verbatim}
             TOTAL ENERGY  -76.2725934815

  ***********************************************
  COEFFICIENT INTERNAL EXTERNAL EXTERNAL INTERNAL
                  SPIN     SPIN      MOS     CONF  -- OCC.
  *******************************************************************

     SPIN-COUPLING REFERS TO REORDERED ORBITALS
  *******************************************************************

   0.97419222        1   VACUUM    0   0        1  -- 22220000
  -0.03737671        1  SINGLET   13  13      201  -- 22200000
  -0.03086972        1  SINGLET   10  15      202  -- 20220000
  -0.03143494        1  SINGLET   15  15      202
   0.03223099        1  SINGLET   14  15      218  -- 21120000

 SUM OF SQUARES OF VACUUM CSF CI COEFFICIENTS=   0.949366016991E+00
 SUM OF SQUARES OF     3-MAIN CI COEFFICIENTS=   0.949241228068E+00
\end{verbatim}
}
{\bf Description of the 1$^{1}A_{1}$ state}
{
\footnotesize
\begin{verbatim}
             TOTAL ENERGY  -75.9018772961

  ***********************************************
  COEFFICIENT INTERNAL EXTERNAL EXTERNAL INTERNAL
                  SPIN     SPIN      MOS     CONF  -- OCC.
  *******************************************************************

     SPIN-COUPLING REFERS TO REORDERED ORBITALS
  *******************************************************************

  -0.91631377        1   VACUUM    0   0        2  -- 22210010
  -0.25186711        1   VACUUM    0   0        3  -- 22121000
  -0.05309578        1   VACUUM    0   0       12  -- 22200020
  -0.08694410        1  DOUBLET    0   9       87  -- 22120000
   0.03713591        1  DOUBLET    0   9       88  -- 12210010
   0.04438742        1  DOUBLET    0  12       89  -- 22110010
  -0.04467943        1  DOUBLET    0  14       89
   0.03994355        2  DOUBLET    0   9       89
  -0.04976149        2  DOUBLET    0  12       89
   0.05925473        2  DOUBLET    0  14       89
   0.05097226        1  DOUBLET    0  13      124  -- 22210000
   0.10693810        1  DOUBLET    0  13      125  -- 22200010
   0.08035163        1  DOUBLET    0  10      153  -- 21210010
   0.07987526        1  DOUBLET    0  15      153
 
 SUM OF SQUARES OF VACUUM CSF CI COEFFICIENTS=   0.909083899548E+00
 SUM OF SQUARES OF     3-MAIN CI COEFFICIENTS=   0.903281844886E+00
\end{verbatim}
}
{\bf Description of the 2$^{1}$A$_{1}$ state}
{
\footnotesize
\begin{verbatim}
             TOTAL ENERGY  -75.8837522007

  ***********************************************
  COEFFICIENT INTERNAL EXTERNAL EXTERNAL INTERNAL
                  SPIN     SPIN      MOS     CONF  -- OCC.
  *******************************************************************

     SPIN-COUPLING REFERS TO REORDERED ORBITALS
  *******************************************************************
   0.26477982        1   VACUUM    0   0        2  -- 22210010
  -0.88348347        1   VACUUM    0   0        3  -- 22121000
  -0.08640544        1   VACUUM    0   0        6  -- 22120001
  -0.04120749        1   VACUUM    0   0        9  -- 22111010
   0.03104762        1   VACUUM    0   0       15  -- 22022000
  -0.23849192        1  DOUBLET    0   9       87  -- 22120000
   0.04891462        1  DOUBLET    0   9       91  -- 22021000
  -0.05841301        1  DOUBLET    0  12       91
   0.07064985        1  DOUBLET    0  14       91
  -0.03000718        1  DOUBLET    0  13      125  -- 22200010
   0.09148369        1  DOUBLET    0  13      126  -- 22111000
   0.07828915        1  DOUBLET    0  10      154  -- 21121000
   0.07660116        1  DOUBLET    0  15      154
 
 SUM OF SQUARES OF VACUUM CSF CI COEFFICIENTS=   0.863430795611E+00
 SUM OF SQUARES OF     3-MAIN CI COEFFICIENTS=   0.850807372991E+00
\end{verbatim}
}
Taking as the criterion for inclusion a coefficient of 0.05, the final
reference set to be employed is constructed based on both
\begin{itemize}
\item the appropriate external MOs in the above coefficient lists,
which now must be assigned internal orbital status.
\item the internal configurations specified in the
ci-vector output above. 
\end{itemize}
This information is now provided directly in 
printing the final CI-vector;  it may also  be derived
from the print of the occupation patterns, assuming the data
line PRCONF~1 had been presented in the 3-reference job.

{
\footnotesize
\begin{verbatim}
 *** VACUUM STATES (# CONF      85   # STATES     127) ***

       1     (#         1       )    2 2 2 2 0 0 0 0
       2     (#         1       )    2 2 2 1 0 0 1 0
       3     (#         1       )    2 2 1 2 1 0 0 0
       6     (#         1       )    2 2 1 2 0 0 0 1
      12     (#         1       )    2 2 2 0 0 0 2 0

 *** DOUBLET STATES (# CONF     112   # STATES    1738) ***


 SYM  1  ONE     # CONF      38   EXTERNAL DIMENSION      12
      87     (#         1       )    2 2 1 2 0 0 0 0
      89     (#         2       )    2 2 1 1 0 0 1 0
      90     (#         2       )    1 2 1 2 1 0 0 0
      91     (#         1       )    2 2 0 2 1 0 0 0

 SYM  2  TWO     # CONF      28   EXTERNAL DIMENSION       4
     124     (#         1       )    2 2 2 1 0 0 0 0
     125     (#         1       )    2 2 2 0 0 0 1 0
     126     (#         2       )    2 2 1 1 1 0 0 0

 SYM  3  THREE   # CONF      28   EXTERNAL DIMENSION       8
     153     (#         2       )    2 1 2 1 0 0 1 0
     154     (#         2       )    2 1 1 2 1 0 0 0
\end{verbatim}
}
The final 16-reference-CI jobs are shown below. Since the ordering of the
orbitals has now changed, we repeat the transformation with a revised
ACTIVE list designed to incorporate the external orbitals referenced
in the coefficient lists above into the internal space. Note that the
external MO indexing within the CI module does not take into account
the inner-shell frozen in the transformation: adding one to the indices
referenced in the CI will provide the integers for specification in the
revised ACTIVE data i.e., 9,10,11,13,14,15 and 16, thus

{
\footnotesize
\begin{verbatim}
          ACTIVE\2 TO 11 13 TO 16 
          12 17 TO 35\END
\end{verbatim}
}
{\bf Final-CI Treatment of the \xastate}
{
\footnotesize
\begin{verbatim}
          RESTART NEW
          TITLE  
          ****  H2O  TZVP + DIFFUSE S,P DIRECT-CI 16M X1A1*
          SUPER OFF NOSYM
          BYPASS SCF
          ZMAT ANGSTROM
          O
          H 1 0.951
          H 1 0.951 2 104.5
          END
          BASIS 
          TZVP O
          TZVP H
          S O
          1.0 0.02
          P O
          1.0 0.02
          END
          RUNTYPE CI
          CORE\1\END
          ACTIVE\2 TO 11 13 TO 16 
          12 17 TO 35\END
          DIRECT 8 14 20
          CONF
          2 2 2 2 0 0 0 0 0 0 0 0 0 0
          2 2 2 1 0 0 1 0 0 0 0 0 0 0
          2 2 1 2 1 0 0 0 0 0 0 0 0 0
          2 2 1 2 0 0 0 1 0 0 0 0 0 0
          2 2 2 0 0 0 2 0 0 0 0 0 0 0
          2 2 1 2 0 0 0 0 1 0 0 0 0 0
          2 2 1 1 0 0 1 0 0 0 0 0 1 0
          2 2 0 2 1 0 0 0 0 0 1 0 0 0
          2 2 0 2 1 0 0 0 0 0 0 0 1 0
          2 2 2 1 0 0 0 0 0 0 0 1 0 0
          2 2 2 0 0 0 1 0 0 0 0 1 0 0
          2 2 1 1 1 0 0 0 0 0 0 1 0 0
          2 1 2 1 0 0 1 0 0 1 0 0 0 0
          2 1 2 1 0 0 1 0 0 0 0 0 0 1
          2 1 1 2 1 0 0 0 0 1 0 0 0 0
          2 1 1 2 1 0 0 0 0 0 0 0 0 1
          ENTER
\end{verbatim}
}
The following points should be noted:
\begin{itemize}
\item There are now 14 orbitals in the internal space and
20 in the external space.
\item We again assume that the Transformed Integral File
is retained from the above job, enabling the integral
transformation to be bypassed in the jobs below.
\item We show below the data file for the final CI
on the 1$^{1}$A$_{1}$ state: that on the third state
follows in straightforward fashion.
\end{itemize}
{\bf Final-CI Treatment of the 1$^{1}$A$_{1}$ state}
{
\footnotesize
\begin{verbatim}
          RESTART CI
          TITLE  
          ****  H2O  TZVP + DIFFUSE S,P DIRECT-CI 16M 1A1*
          SUPER OFF NOSYM
          BYPASS TRANSFORM
          ZMAT ANGSTROM
          O
          H 1 0.951
          H 1 0.951 2 104.5
          END
          BASIS 
          TZVP O
          TZVP H
          S O
          1.0 0.02
          P O
          1.0 0.02
          END
          RUNTYPE CI
          CORE\1\END
          ACTIVE\2 TO 11 13 TO 16 
          12 17 TO 35\END
          DIRECT 8 14 20
          CONF
          2 2 2 2 0 0 0 0 0 0 0 0 0 0
          2 2 2 1 0 0 1 0 0 0 0 0 0 0
          2 2 1 2 1 0 0 0 0 0 0 0 0 0
          2 2 1 2 0 0 0 1 0 0 0 0 0 0
          2 2 2 0 0 0 2 0 0 0 0 0 0 0
          2 2 1 2 0 0 0 0 1 0 0 0 0 0
          2 2 1 1 0 0 1 0 0 0 0 0 1 0
          2 2 0 2 1 0 0 0 0 0 1 0 0 0
          2 2 0 2 1 0 0 0 0 0 0 0 1 0
          2 2 2 1 0 0 0 0 0 0 0 1 0 0
          2 2 2 0 0 0 1 0 0 0 0 1 0 0
          2 2 1 1 1 0 0 0 0 0 0 1 0 0
          2 1 2 1 0 0 1 0 0 1 0 0 0 0
          2 1 2 1 0 0 1 0 0 0 0 0 0 1
          2 1 1 2 1 0 0 0 0 1 0 0 0 0
          2 1 1 2 1 0 0 0 0 0 0 0 0 1
          SHIFT 0.5\ALTERNATE
          TRIAL
          2 1.0
          VMIN
          ENTER
\end{verbatim}
}

\clearpage

\begin{thebibliography}{10}

\bibitem{ref:19}
P.E.M. Siegbahn, J. Chem. Phys. {\bf 72} (1980) 1647, \doi{10.1063/1.439365};
V.R. Saunders and J.H. van Lenthe, Mol. Phys. {\bf 48} (1983) 923,
\doi{10.1080/00268978300100661}.


\bibitem{ref:40}   M. Yoshimine, J. Comp. Phys. 11 (1973) 449;
P.S. Bagus, B. Liu, A.D. McLean and M. Yoshimine,
Energy, Structure and Reactivity, edited by D.W. Smith and
W.B. McRae, (Wiley) (1973) 130;
G.H.F. Diercksen,
  Theor. Chim. Acta {\bf 33} (1974) 1, \doi{10.1007/BF00527620};
S.T. Elbert, Numerical Algorithms in Chemistry : Algebraic
Methods-LBL 8158, edited by C. Moler and I. Shavitt,
(Lawrence Berkeley Laboratory, University of California,
Berkeley) (1978)  1298

\bibitem{ref:41} C. Zirz and R. Ahlrichs, in 'Electron Correlation:
Proceedings of the Daresbury Study Weekend', eds. M.F. Guest and
S. Wilson (Daresbury Laboratory Report DL/SCI/R14) (1980) 83.

\bibitem{ref:48}
H.J.J. van Dam, J.H. van Lenthe, and P.J.A. Ruttink,
'Exact size consistency of multi-reference M{\o}ller-Plesset perturbation
theory', Int. J. Quant. Chem. {\bf 72} (1999) 549--558.
% No doi available.

\bibitem{ref:49}
H.J.J. van Dam, J.H. van Lenthe, and P. Pulay,
'The size consistency of multi-reference M{\o}ller-Plesset perturbation theory',
Mol. Phys. {\bf 93} (1998) 431--439, \doi{10.1080/002689798169122}.

\bibitem{ref:50}
K. Wolinski, H.L. Sellers, and P. Pulay,
  Chem. Phys. Lett. {\bf 140} (1987) 225, \doi{10.1016/0009-2614(87)80448-7}.

\bibitem{ref:51}
P.J.A. Ruttink, J.H. van Lenthe, R. Zwaans, and G.C. Groenenboom,
J. Chem. Phys. {\bf 94} (1991) 7212--7220, \doi{10.1063/1.460204}.

\bibitem{ref:52}
K. Andersson, P.-\AA. Malmqvist, and B.O. Roos,
  J. Chem. Phys. {\bf 96} (1992) 1218, \doi{10.1063/1.462209}.

\bibitem{ref:AQCC}
P.G. Szalay, R.J. Bartlett,
  Chem. Phys. Lett. {\bf 214} (1993) 481, \doi{10.1016/0009-2614(93)85670-J}.

\bibitem{ref:ACPF}
R. Gdanitz, R. Ahlrichs,
  Chem. Phys, Lett. {\bf 143} (1988) 413, \doi{10.1016/0009-2614(88)87388-3}.

\bibitem{ref:Szalay}
L. Fusti-Molnar, P.G. Szalay,
  J. Phys. Chem. {\bf 100} (1996) 6288, \doi{10.1021/jp952840j}.

\bibitem{ref:MRCEPA1}
P.J.A. Ruttink, J.H. van Lenthe, P. Todorov,
  Mol. Phys. {\bf 103} (2005) 2497, \doi{10.1080/00268970500180725}.

\end{thebibliography}
\end{document}
