\section{Tool Programs}
Most of the following tool programs are called from within the 
{\tt diesel} main driver program. So you usually do not need to interact
with them directly. 

\subsection{Diesel Results ({\tt dr})}
This program collects the energy results of the calculation within one irrep,
performs several extrapolation schemes, and prints the results.

\bigskip
command line:
{\tt dr [-R=$n$] [-T=$t$] [-C=$r$,$c$] [-h] [-w]}

\begin{enumerate}
\item -R: restrict output to root $n$
\item -T: restrict output to threshold $t$
\item -C: print row $r$ and column $c$ for each root block only
\item -h: surpress table headers
\item -w: surpress wave function output
\end{enumerate}

These options are especially useful to grep a certain number out of
the output in order to generate a potential surface for example.


\subsection{Getting Project Information ({\tt getDirInfo})}
{\tt getDirInfo} evaluates the directory structure of a calculation
and prints information about it.

For example:
\begin{verbatim}
             Project = c6
            Geometry = linear
               Basis = C.DUN...5S3P1D
        Multiplicity = 3
               IrRep = 3
\end{verbatim}


\subsection{Grepping the Dominating Configurations ({\tt grepImp})}
\bigskip

command line:
{\tt grepImp {\it threshold} <diag.out.{\it thresh}}

Print the configurations in the wave function having a coefficient
greater than {\it threshold}.


\subsection{Calculation of excitation statistics ({\tt confStat})}
\bigskip

command line:
{\tt confStat ConfTree.dat {\it ref} }

Print the excitation levels in tree "{\tt ConfTree.dat}" relative
to configuration in file {\it ref}.


\subsection{Symmetrization of selected configurations ({\tt symTree})}
\bigskip

command line:
{\tt symTree "equivalence" ["fort.31"-file]  }

\subsection{Symmetrization of references ({\tt refsym})}

command line:
{\tt refsym <equivalences irrep sel.in.all-files}

The file "equivalences" contains equivalent MOs per line. For example if
1 2 3 were $\sigma$- and (10,20), (11,21) were $\pi$-orbitals this
file would look like:
{\obeylines
{\tt
1
2
3
10 20
11 21
}}
The references in "sel.in.all" are transformed to the representation
of the degenerated point group. By transforming this orbital representation
back to the original non degenerated point group the program completes
or discards whole classes of configurations with respect to the full
point group. 
%The decision of discarsion or completion is made from the 
%amount 
%%!!!!!!!!!!!!!!!!!!

\subsection{Listing the Configuration Tree ({\tt lstconfs})}
\bigskip

command line:
{\tt lstconfs ConfTree.dat}

For example:
\begin{verbatim}
lstconfs (Part of DIESEL-MR-CI), Version 1.08pre, 22. Jan 1999

intern-0: 125
intern-1: 10
intern-2: 3

number of reference configurations          : 112
number of CSFs from reference configurations: 216
total number of configurations              : 138
total number of CSFs                        : 254

  ref.   2 38 58 1-3 39-40 59-60 83-85 121-122 141-142
  ref.   2 38 58 1-3 39-40 59 83-85 121-123 141-142
  ref.   2 38 58 1-3 39-40 59 83-85 121-122 141-143
  ref.   2 38 58 1-3 39 59-60 83-85 121-123 141-142
  ref.   2 38 58 1-3 39 59-60 83-85 121-122 141-143
.
.
.
\end{verbatim}

\subsection{Set Operations ({\tt setops})}
The program calculates merge, intersection or difference sets
of given configurations and writes them to the standard output
in tree format.

\bigskip
command line:
{\tt setops -{c|s|t} -{m|i|d} file1 file2 ...}

\begin{enumerate}
\item -c: configurations in plain format
\item -s: configurations to be read from selector input
\item -t: configurations in tree format (as produced by the selector)
\end{enumerate}

\begin{enumerate}
\item -m: perform merge
\item -i: perform intersection
\item -d: perform difference
\end{enumerate}


\subsection{Fort.31 File Format Conversion ({\tt f31endian})}
The program {tt f31endian} is capable of converting the "fort.31" integral file
format between little (e. g. Intel, Transputer, VAX) and 
big endian (e. g. RISC, HP, IBM, Sun) notation.

command line:
{\tt f31endian {l2b|b2l} {old|new|tradpt} input-filename output-filename}

\begin{enumerate}
\item l2b: little $\to$ big endian conversion
\item b2l: big $\to$ little endian conversion
\end{enumerate}

\begin{enumerate}
\item old: fort.31 format from the HONDO program suite
\item new: fort.31 format from the MOLCAS program suite
\item tradpt: fort.31 format from the TRADPT program
\end{enumerate}



\subsection{One Electron Density Matrices ({\tt dens})}
\label{Density}
The program {\tt dens} calculates the (transition) one electron density matrices.
It depends on the configuration tree and the eigenvector files from
a previous MR-CI calculation.

\bigskip
command line:

{\tt dens motra-input-file Thresh\\
{\#Lstate1[-\#LstateN], all} Ldir {[\#Rstate1[-\#RstateM], all}}  [Rdir]]


{\it state} means the nth state calculated in an MR-CI calculation.
Arguments in square brackets are optional. If missing they default to
corresponding arguments given first. The output is written
to a file "Density.dat".
%The matrix is echoed to stdout in formatted long line format.




\subsection{Natural Orbitals ({\tt natorb})}
\label{Natural}
The program {\tt natorb} calculates natural orbitals from one electron 
density matrices.

\bigskip
command line:

{\tt natorb DensityMatrix1 ... [-weight w1 ... ] <InputOrbitals >OutputOrbitals}

Several density matrices may be weighted by the option {\it weight}.
The input and output orbitals are in MOLCAS format.







\subsection{Properties ({\tt prop})}

The program {\tt prop} calculates one electron operator properties
based on the one electron density matrices generated with the program
described in \ref{Density} and on the one electron integral file
(ONEINT) generated by the MOLCAS program package. The result is
written to stdout.

\bigskip
command line:

{\tt prop} {\it operator component IntegralPath OrbitalPath DensityPath $\ldots$}

with

\begin{center}
\begin{tabular}{rl}
{\it operator}: &\{ "Mltpl1", "Kinetic", "OneHam" \}\\
{\it component}: &\{ 1, 2, $\ldots$ \}\\
{\it IntegralPath}: &path to MOLCAS ONEINT File\\
{\it OrbitalPath}: &path to MOLCAS orbitals file (e.g. "SCFORB", "RASORB")\\
{\it DensityPath}: &path to CI density matrix(ces) (generated with "dens")\\
\end{tabular}
\end{center}



\subsubsection{Driver to calculate and nicely print several Properties ({\tt prettyProp})}

The program {\tt prettyProp} is a driver for the {\tt prop}-program.

calculates one electron operator properties
based on the one electron density matrices generated with the program
described in \ref{Density} and on the one electron integral file
(ONEINT) generated by the MOLCAS program package. The result is
written to stdout.

\bigskip
command line:

%{\tt prettyProp} {\it -global options  -printing modes}
%
%with the following global options:

%\begin{center}
%\begin{tabular}{rl}
%-unit \{au, eV, kcal/mol\}: &set printing unit, default: au\\
%-mode \{E\underbar{ }CI, E\underbar{ }extpolCI, E\underbar{ }Dav1, E\underbar{ }Dav2\}: 
%&which energy type to be used,\\
%&default: E\underbar{ }Dav2\\
%-thresh {\it value}: &threshold value\\
%-intpath {\it IntegralPath}: &path to MOLCAS ONEINT file\\
%&default: ../ONEINT\\
%-orbpath {\it OrbitalPath}: &path to MOLCAS orbitals file\\
%&(e.g. "SCFORB", "RASORB")\\
%&default: ../SCFORB\\
%\end{tabular}
%\end{center}

%and the following printing modes:
%\begin{center}
%\begin{tabular}{rl}
%-e: & print excitation energies\\
%-prop {\it operator component}: &calculate properties (depends on density matrices)\\
% & the {\it operator} and {\it component} are the same as\\
% &for the {\tt prop} program\\
%-osc: &calculate oscillator strengths (depends on density matrices)\\
%\end{tabular}
%\end{center}


