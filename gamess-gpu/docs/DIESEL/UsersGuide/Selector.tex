\subsection{Selector}




\subsubsection{Invocation}
{\tt sel} {\it arguments}

\subsubsection{Command Line Arguments}
\begin{itemize}
\item {\tt -r }: recalculation mode: perform the perturbation estimation
on a given MR-CI tree (useful after an applied merge operation).
\end{itemize}





\subsubsection{Files}


\begin{tabular}{r|r|r|p{7cm}}
status &name &type &content \cr
\TspaceII
\noalign{\hrule}
\TspaceII
in & stdin & ASCII & user input 
(reference configurations, selection thresholds, etc.), see \ref{SelInput}\cr
\TspaceII
in & "fort.31"%\footnote{or whatever has been 
%specified in the "MOIntegralFilename"-assignment} 
&binary&integrals in MO basis (AO$\to$MO transformation performed),
point symmetry,
number of orbitals in certain irreducible representation\cr
\TspaceII
\TspaceII
out & stdout & ASCII &program output\cr
&&& (containing protocol, eigenvalues, statistics, etc.,\cr
&&& depending on verbosity level)\cr
out& "ConfTree.dat" & ASCII &tree of selected configurations\cr
\end{tabular}




\subsubsection{User Input and Keywords}
\label{SelInput}
\label{SelectorKeywords}
\begin{supertabular}{p{3cm}|c|p{1.5cm}|p{1.5cm}|p{5cm}}
&sta- &de-&argument &\\
keyword	&tus&fault&type &description\\[5pt]
\hline&&&&\\[-9pt]
{\tt Multiplicity}	&req. &-- &\key{num} &
multiplicity of state ($=2S+1$, $S$: spin quantum number) \tnl
{\tt Selection\-Thresholds}	&req. &-- &\key{floatSet} &
a set of thresholds in Hartree for selection procedure\tnl
{\tt RefConfs}	&req. &-- &\key{confSet} &
set of configurations spanning the multi reference space
(used as basis for excitations)\tnl
{\tt NumberOf\-Electrons}	&opt. & conf. input&\key{num} & \tnl
{\tt ExcitationLevel}	&opt. & 2 &\key{num} &
maximum excitation level used in MR-CI\tnl
{\tt selectInternal}	&opt. & no &\key{bool} &
completely select internal space\tnl
{\tt selectNth\-Excitation}	&opt. &\{\} &\key{numSet} &
set of excitation levels to be completely selected\tnl
{\tt AnnihilatorSpace}	&opt. &inactive &\key{numSet} &
orbitals from which electrons may be excited\tnl
{\tt CreatorSpace}	&opt. &inactive &\key{numSet} &
orbitals to which electrons may be excited\tnl
{\tt ActiveSpace\-ExcitationLevel}	&opt. &1 &\key{num} &
excitation level to generate references\tnl
{\tt maxRefOpenShells}	&opt. &4 &\key{num} &
maximum number of open shells for references\tnl
{\tt FirstGuessConfs}	&opt. &3 &\key{num} &
number of configurations increased by the number of ordered roots
to be used from the first guess\tnl
{\tt MORestrictions}	&opt. &none &\key{MO\-Restrict} &
restrictions on the MO occupation pattern to be applied
on the generated configuration space\tnl
{\tt MOEquivalence}	&opt. &none &\key{MOEqui\-valence} &
equivalent (degenerated) MO list; 
if a selected configuration contains an equivalent MO
the same configuration with equivalent MOs substituted is
selected also\tnl
{\tt MOStatistics}	&opt. &no &\key{boolean} &
print out MO statistics\tnl
{\tt EstimationMode}	&opt. &Epstein\-Nesbet &\key{Estima\-tionMode} &
determines the method to estimate the energy contribution of a certain
configuration\tnl
{\tt StorePTEnergy}	&opt. &no &\key{bool} &
flag if perturbational energies are stored in configuration tree\tnl
{\tt StorePTCoef}	&opt. &no &\key{bool} &
flag if perturbational CI coefficients are stored in configuration tree\tnl
{\tt PTRefConfs}	&opt. &{\tt RefConfs} &\key{confSet} &
set of configurations to be used as zero order
wave function by perturbation theory\tnl
\end{supertabular}

\bigskip

If the reference configurations are given explicitly they are checked for
consistency among each other and the {\tt IrRep}-keyword becomes optional. Any
specified number of electrons or irreducible representation is checked against
the given references.


\subsubsection{Automatic Initial Reference Space Guess}
Figure \ref{SpaceGuess} shows the procedure that is used to generate
a first reference configuration guess. It works pretty well for different
irreducible representations and several roots.
This feature is very useful as it saves
one from the error intensive handling with the MO numbers.
This procedure is quite reliable. There may be problems if many roots
are ordered and the wave function is "diffuse". In that case
the selection due to the diagonal element may be not really optimal.
\begin{figure}[h]
\begin{center}
\input XFig/SpaceGuess.pstex
\end{center}
\caption{steps to generate first reference space guess}
\label{SpaceGuess}
\end{figure}


\subsubsection{Input Examples}

\begin{description}
\item[first example:]
\ 

\begin{footnotesize}
\begin{verbatim}
# this is a comment
VerbosityLevel          = { RefGuess RefMatEigenValues IterationBlocks WaveFunction }
MORestrictions          = { 21-24>0 30,31=1 25<2 50,51,52<=2 }
MOIntegralFilename      = fort.31
MOIntegralFileFormat    = auto
NumberOfElectrons       = 19
Multiplicity            = 2
IrRep                   = 2
ExcitationLevel         = 2
SelectionThresholds     = { 0.5 }
Roots                   = { 1 2 5 }
selectInternal          = no
selectNthExcitation     = { }
RefConfs                = {
1 20 1 2 3 4 5 6 7 8 9
# another comment
1 21 1 2 3 4 5 6 7 8 9
3 10 15 25 1 2 3 4 5 6 7 8
}
\end{verbatim}
\end{footnotesize}

\item[second example:]
\ 

\begin{footnotesize}
\begin{verbatim}
#this works too!
MULTIPLICITY = 2;excitationlevel=2
NumberOfElectrons       = 19
 SelectionThresholds= { 0.5 }
RefConfs                = auto
\end{verbatim}
\end{footnotesize}
Note that the starting configurations are generated automatically so there
is no need to worry about the sophisticated process of MO-renumbering.

The reference configuration guess is performed by the following scheme:
\begin{enumerate}
\item The closed shell ground state occupation pattern is calculated.
This is done by calculating the diagonal elements of the Hamilton Matrix
whith any possible occupation pattern matching the irreducible representations
and the given number of electrons. The configuration lowest in energy is
selected as a base for the following step. If the number of given electrons is
odd the calculation is done with the positive ion.
\item A certain set of MOs is chosen to be active. This set consists of
the HOMO$\ldots$HOMO$-$3 and the LUMO$\ldots$LUMO$+$5 in every irreducible
representation.
\item Within the active space single, double, triple, and quadruple
excitations are performed and the diagonal hamilton matrix elements are
calculated. The 3+{\it number of roots} configurations lowest in energy
are chosen to be the reference configuration set.
\end{enumerate}

This scheme is rather reliable especially for ground state occupation patterns.
There may be problems if there are many roots ordered which can be described
properly by several configurations only. Usually the following
iterative space generation process is able to handle this problem.

\end{description}
