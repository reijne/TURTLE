\section{Examples}


\subsection{C$_6$ with automatic reference generation and property
calculation}


\begin{prog}
\begin{verbatim}
#!/bin/bash

export DIESEL_EXE_DIR=somepath
export MOLCAS_EXE_DIR=somepath
cd whereEver

cat <<! >diesel.in
# MO integral file
MOIntegralFileFormat             = New
MOLCASRootDir                    = `pwd`/..

# MOs
MOEquivalence                    = auto

# electrons / state
NumberOfElectrons                = 16
Multiplicities                   = { 1 3 }
IrReps                           = { 0 1 3 }
Roots                            = { 1 2 3 4 }

# references
RefConfs                         = auto

#selection
SelectionThresholds              = { 1e-3 1e-4 1e-5 }

#properties
propertyThresholds               = { 1e-3 1e-4 1e-5 }
orbitalFile                      = INPORB

MaxHamiltonStorageMem            = 500MB
!

$DIESEL_EXE_DIR/diesel <diesel.in 1>diesel.out 2>diesel.prot.out
\end{verbatim}
\vspace{-9pt}
\caption{\label{ListExampleRefIterInput} Listing of a simple diesel driver 
job file for C$_6$ with 20 core electrons using automatic reference
generation}
\end{prog}



\begin{prog}
\begin{verbatim}
*******************************************************************************
*                                                                             *
*                                diesel protocol                              *
*                                                                             *
*******************************************************************************


multiplicity=1
    irrep=0
        creating reference space:
            iteration #1
            iteration #2
            iteration #3
            iteration #4
            iteration #5
        reference space generation completed
        performing selection on given thresholds
        diagonalization steps
            threshold 1e-3
            threshold 1e-4
            threshold 1e-5
        diagonalization finished

    irrep=1
        creating reference space:
            iteration #1
            iteration #2
            iteration #3
        reference space generation completed
        performing selection on given thresholds
        diagonalization steps
            threshold 1e-3
            threshold 1e-4
            threshold 1e-5
        diagonalization finished

    irrep=3
        creating reference space:
            iteration #1
            iteration #2
            iteration #3
        reference space generation completed
        performing selection on given thresholds
        diagonalization steps
            threshold 1e-3
            threshold 1e-4
            threshold 1e-5
        diagonalization finished

    calculating one particle density matrices
        threshold=1e-3
            iirrep=0
                jirrep=1
                jirrep=3
            iirrep=1
                jirrep=3
            iirrep=3
            calculating properties
            property calculation finished
        threshold=1e-4
            iirrep=0
                jirrep=1
                jirrep=3
            iirrep=1
                jirrep=3
            iirrep=3
            calculating properties
            property calculation finished
        threshold=1e-5
            iirrep=0
                jirrep=1
                jirrep=3
            iirrep=1
                jirrep=3
            iirrep=3
            calculating properties
            property calculation finished
    one particle density matrices calculation finished
\end{verbatim}
\vspace{-9pt}
\caption{\label{ListExampleRefIterProtOut} Protocol output (stderr)}
\end{prog}

\begin{prog}
\begin{verbatim}
multiplicity=3
    irrep=0
        creating reference space:
            iteration #1
            iteration #2
            iteration #3
        reference space generation completed
        performing selection on given thresholds
        diagonalization steps
            threshold 1e-3
            threshold 1e-4
            threshold 1e-5
        diagonalization finished

    irrep=1
        creating reference space:
            iteration #1
            iteration #2
            iteration #3
            iteration #4
        reference space generation completed
        performing selection on given thresholds
        diagonalization steps
            threshold 1e-3
            threshold 1e-4
            threshold 1e-5
        diagonalization finished

    irrep=3
        creating reference space:
            iteration #1
            iteration #2
            iteration #3
        reference space generation completed
        performing selection on given thresholds
        diagonalization steps
            threshold 1e-3
            threshold 1e-4
            threshold 1e-5
        diagonalization finished

    calculating one particle density matrices
        threshold=1e-3
            iirrep=0
                jirrep=1
                jirrep=3
            iirrep=1
                jirrep=3
            iirrep=3
            calculating properties
            property calculation finished
        threshold=1e-4
            iirrep=0
                jirrep=1
                jirrep=3
            iirrep=1
                jirrep=3
            iirrep=3
            calculating properties
            property calculation finished
        threshold=1e-5
            iirrep=0
                jirrep=1
                jirrep=3
            iirrep=1
                jirrep=3
            iirrep=3
            calculating properties
            property calculation finished
    one particle density matrices calculation finished
\end{verbatim}
\vspace{-9pt}
\caption{\label{ListExampleRefIterProtOut} Protocol output (stderr, continued)}
\end{prog}

\begin{prog}
\begin{verbatim}
user@machine>ls -R  
1                =Multiplicity    diesel.in        diesel.prot.out
3                CI.job           diesel.out

1:
0              3              Densities      dens.out.1e-4  fort.31        prop.out.1e-4
1              =Irrep         dens.out.1e-3  dens.out.1e-5  prop.out.1e-3  prop.out.1e-5

1/0:
ConfTree.dat           Eigenvectors.dat.1e-5  genspace.0             sel.in.2
ConfTree.dat.0.001     diag.in                genspace.1             sel.in.3
ConfTree.dat.1e-3      diag.in.RefGen         genspace.2             sel.in.4
ConfTree.dat.1e-4      diag.out.1e-3          genspace.3             sel.in.all
ConfTree.dat.1e-5      diag.out.1e-4          genspace.4             sel.out.all
Eigenvectors.dat.1e-3  diag.out.1e-5          sel.in.0
Eigenvectors.dat.1e-4  fort.31                sel.in.1

1/1:
ConfTree.dat           Eigenvectors.dat.1e-4  diag.out.1e-5          sel.in.1
ConfTree.dat.0.001     Eigenvectors.dat.1e-5  fort.31                sel.in.2
ConfTree.dat.1e-3      diag.in                genspace.0             sel.in.all
ConfTree.dat.1e-4      diag.in.RefGen         genspace.1             sel.out.all
ConfTree.dat.1e-5      diag.out.1e-3          genspace.2
Eigenvectors.dat.1e-3  diag.out.1e-4          sel.in.0

1/3:
ConfTree.dat           Eigenvectors.dat.1e-4  diag.out.1e-5          sel.in.1
ConfTree.dat.0.001     Eigenvectors.dat.1e-5  fort.31                sel.in.2
ConfTree.dat.1e-3      diag.in                genspace.0             sel.in.all
ConfTree.dat.1e-4      diag.in.RefGen         genspace.1             sel.out.all
ConfTree.dat.1e-5      diag.out.1e-3          genspace.2
Eigenvectors.dat.1e-3  diag.out.1e-4          sel.in.0

1/Densities:
Density.dat.I0R1_I0R1.1e-3  Density.dat.I0R3_I1R2.1e-3  Density.dat.I1R2_I1R4.1e-3
.
.
.

3:
0              3              Densities      dens.out.1e-4  fort.31        prop.out.1e-4
1              =Irrep         dens.out.1e-3  dens.out.1e-5  prop.out.1e-3  prop.out.1e-5

3/0:
ConfTree.dat           Eigenvectors.dat.1e-4  diag.out.1e-5          sel.in.1
ConfTree.dat.0.001     Eigenvectors.dat.1e-5  fort.31                sel.in.2
ConfTree.dat.1e-3      diag.in                genspace.0             sel.in.all
ConfTree.dat.1e-4      diag.in.RefGen         genspace.1             sel.out.all
ConfTree.dat.1e-5      diag.out.1e-3          genspace.2
Eigenvectors.dat.1e-3  diag.out.1e-4          sel.in.0

3/1:
ConfTree.dat           Eigenvectors.dat.1e-4  diag.out.1e-5          sel.in.0
ConfTree.dat.0.001     Eigenvectors.dat.1e-5  fort.31                sel.in.1
ConfTree.dat.1e-3      diag.in                genspace.0             sel.in.2
ConfTree.dat.1e-4      diag.in.RefGen         genspace.1             sel.in.3
ConfTree.dat.1e-5      diag.out.1e-3          genspace.2             sel.in.all
Eigenvectors.dat.1e-3  diag.out.1e-4          genspace.3             sel.out.all

3/3:
ConfTree.dat           Eigenvectors.dat.1e-4  diag.out.1e-5          sel.in.1
ConfTree.dat.0.001     Eigenvectors.dat.1e-5  fort.31                sel.in.2
ConfTree.dat.1e-3      diag.in                genspace.0             sel.in.all
ConfTree.dat.1e-4      diag.in.RefGen         genspace.1             sel.out.all
ConfTree.dat.1e-5      diag.out.1e-3          genspace.2
Eigenvectors.dat.1e-3  diag.out.1e-4          sel.in.0

3/Densities:
Density.dat.I0R1_I0R1.1e-3  Density.dat.I0R3_I1R2.1e-3  Density.dat.I1R2_I1R4.1e-3
.
.
.
\end{verbatim}
\vspace{-9pt}
\caption{\label{ListExampleDir} Recursive directory contents}
\end{prog}


The property output is written to the files {\tt prop.out.}{\it thresh}.


\begin{table}
\begin{tabular}{r|r|p{5cm}}
{\bf programs}
& motra&\\
& form31&\\[5pt]
\hline&\\[-9pt]
{\bf read files}
& motra&\\
& form31&\\[5pt]
\hline&\\[-9pt]
{\bf written files}
& motra&\\
& form31&
\end{tabular}
\caption{MOLCAS-dependencies}
\end{table}


\begin{prog}
\begin{verbatim}
============================================
             Project = c6
            Geometry = linear
               Basis = C.DUN...5S3P1D
            Orbitals = 2_2.CAS
                Core = 20e
       ReferenceType = RefIter
        Multiplicity = 1
               Irrep = 0
============================================



reference configurations:
0 1-2 23-24 35-36 50 72        # 1
.
.
.
4 35-36 84-85 1-2 23-24 50 72   # 32


number of configurations: 32
dimension of ci matrix  : 42


selected roots within reference space:
root #1: -226.910979
root #2: -226.901841
root #3: -226.725320
root #4: -226.721146


total generated configurations/CSFs:

                confs.        CSFs
intern-0:          352         603
intern-1:        23423       70382
intern-2:       531645     2215655
          ------------------------
total   :       555420     2286640



++++++++
root #1:
reference energy: -226.91098
character (at threshold 1.00e-02 mH, ci^2>0.01): 
        0.0126213162       -0.1123446312:   ref.   0 1-2 23-24 35-36 50 72  # 1
        0.0120907020        0.1099577283:   ref.   0 1-2 23-24 35-36 50 84  # 2
        0.4144999165        0.6438166793:   ref.   0 1-2 23-24 35 50 72 84  # 4
        0.3989897564       -0.6316563594:   ref.   0 1-2 23 35-36 50 72 84  # 10

   threshold  sel. confs   sel. CSFs   CI energy    ref ci^2     overlap      PT(EN)      PT(EN)
         /mH                                  /H                     max         /mH weighted/mH
----------------------------------------------------------------------------------------------------
           1          18          28   -226.9351  0.97464208  0.99998533    -434.763  -434.76363
         0.1         648        1014  -227.04893  0.90646824  0.99992052    -299.391  -299.39962
        0.01        8279       14719  -227.19895  0.87097902  0.99995333    -100.904  -100.90425


extrapolation to full MRCI (all Energies in H):

   threshold _______________ EN ________________
         /mH      E(l=1)           l        E(l)
------------------------------------------------
           1  -227.36987         ---         ---
         0.1  -227.34833  0.84093363  -227.30071
        0.01  -227.29985  0.75575772  -227.27521


extrapolation to full CI (all Energies in H):

         /mH   ____________________ EN _____________________
   threshold   ____ Davidson 1 _____   ____ Davidson 2 _____
                  E(l=1)        E(l)      E(l=1)        E(l)
------------------------------------------------------------
           1   -227.3815         ---  -227.38181         ---
         0.1  -227.38924  -227.33716  -227.39346  -227.34092
        0.01  -227.35003   -227.3222  -227.35746  -227.32916
++++++++
.
.
.
\end{verbatim}
\vspace{-9pt}
\caption{\label{ListExampleRefIterOut} Report output (stdout)}
\end{prog}



\subsection{C$_6$ with active space references and root homing}



\begin{prog}
\begin{verbatim}
#!/bin/bash

export DIESEL_EXE_DIR=somepath
export MOLCAS_EXE_DIR=somepath
cd whereEver

cat <<! >diesel.in
# MO integral file
MOIntegralFileFormat             = New
MOLCASRootDir                    = `pwd`/..

# MOs
MOEquivalence                    = auto

# electrons / state
NumberOfElectrons                = 16
Multiplicities                   = { 1 3 }
IrReps                           = { 0 1 3 }
Roots                            = { 1 2 3 4 }

# references
RefConfs                         = {
0 1-2 23-24 35 50 72 84   
0 1-2 23 35-36 50 72 84   
2 24 36 1-2 23 35 50 72 84
}
AnnihilatorSpace                 = { 2 23 24 35 36 50 72 84 }
CreatorSpace                     = { 23 24 35 36 72 73 84 85 }
activeSpaceExcitationLevel       = 2
maxRefOpenShells                 = 6

RootHoming                       = yes

#selection
SelectionThresholds              = { 1e-3 1e-4 1e-5 }

MaxHamiltonStorageMem            = 500MB
!

$DIESEL_EXE_DIR/diesel <diesel.in 1>diesel.out 2>diesel.prot.out
\end{verbatim}
\vspace{-9pt}
\caption{\label{ListExampleRefActiveInput} Listing of a simple diesel driver 
job file for C$_6$ with 20 core electrons using active space references}
\end{prog}


If you later decide you need one more threshold simply add it in the input
(s. Listing \ref{ListExampleAppendThresh}) and restart.
\begin{prog}
\begin{verbatim}
SelectionThresholds              = { 1e-3 1e-4 5e-5 1e-5 }
\end{verbatim}
\vspace{-9pt}
\caption{\label{ListExampleAppendThresh} 
Input for an additional threshold}
\end{prog}
As you see in Listing \ref{ListExampleAppendThreshProt} only the
necessary steps are executed.


\begin{prog}
\begin{verbatim}
multiplicity=1
        irrep=0
                using active space references
                performing selection on given thresholds
                diagonalization steps
                        threshold 1e-3: already done
                        threshold 1e-4: already done
                        threshold 5e-5
                        threshold 1e-5: already done
                diagonalization finished
.
.
.
\end{verbatim}
\vspace{-9pt}
\caption{\label{ListExampleAppendThreshProt} Protocol output (stderr)}
\end{prog}

The same way you may append additional irreducible representations or
multiplicities. The output of results will contain all information as if
you originally started with the whole set of parameters
(s. Listing \ref{ListExampleRefActiveOut}). If you want to change something
different from that described above (e. g. different fort.31-file, 
reference configurations, convergence criterions, ...) please delete the
calculation tree before restarting the calculation in order to get 
what you want.


\begin{prog}
\begin{verbatim}
============================================
             Project = c6
            Geometry = linear
               Basis = C.DUN...5S3P1D
            Orbitals = 2_2.CAS
                Core = 20e
       ReferenceType = RefActive
        Multiplicity = 1
               Irrep = 0
============================================



reference configurations:
0 1-2 23-24 35-36 50 72         # 1
.
.
.
4 72-73 84-85 1-2 23 35-36 50   # 72


number of configurations: 72
dimension of ci matrix  : 100


selected roots within reference space:
root #1: -226.926197
root #2: -226.917450
root #3: -226.743097
root #4: -226.738563


total generated configurations/CSFs:

                confs.        CSFs
intern-0:          412         751
intern-1:        41684      133864
intern-2:      1133336     5392979
          ------------------------
total   :      1175432     5527594



++++++++
root #1:
reference energy: -226.9262
character (at threshold 1.00e-02 mH, ci^2>0.01): 
        0.0121657647       -0.1102985255:   ref.   0 1-2 23-24 35-36 50 72  # 1
        0.0122560845        0.1107072015:   ref.   0 1-2 23-24 35-36 50 84  # 2
        0.4050904682        0.6364671777:   ref.   0 1-2 23-24 35 50 72 84  # 5
        0.4078713680       -0.6386480784:   ref.   0 1-2 23 35-36 50 72 84  # 15

   threshold  sel. confs   sel. CSFs   CI energy    ref ci^2     overlap      PT(EN)      PT(EN)
         /mH                                  /H                     max         /mH weighted/mH
----------------------------------------------------------------------------------------------------
           1          10          14  -226.93881  0.98552683  0.99987399     -427.22  -427.22388
         0.1         603         953  -227.04791  0.91783744  0.99999991     -301.27  -301.27001
        0.05        1501        2383  -227.09915  0.90225781  0.99994973    -235.914  -235.92348
        0.01        8088       14344  -227.19816  0.88045173  0.99997603    -106.048  -106.05273


extrapolation to full MRCI (all Energies in H):

   threshold _______________ EN ________________
         /mH      E(l=1)           l        E(l)
------------------------------------------------
           1  -227.36603         ---         ---
         0.1  -227.34918  0.86624541  -227.30889
        0.05  -227.33508  0.78410956  -227.28414
        0.01  -227.30422  0.76237506  -227.27902


extrapolation to full CI (all Energies in H):

         /mH   ____________________ EN _____________________
   threshold   ____ Davidson 1 _____   ____ Davidson 2 _____
                  E(l=1)        E(l)      E(l=1)        E(l)
------------------------------------------------------------
           1   -227.3724         ---  -227.37249         ---
         0.1  -227.38394  -227.34033  -227.38705  -227.34315
        0.05  -227.37504  -227.31913  -227.37937  -227.32292
        0.01  -227.34941  -227.32119  -227.35554  -227.32692
++++++++
.
.
.
\end{verbatim}
\vspace{-9pt}
\caption{\label{ListExampleRefActiveOut} Report output (stdout)}
\end{prog}

\subsection{Using natural orbitals}

\begin{prog}
\begin{verbatim}
# MO integral file
#MOIntegralFilename               = fort.31
MOIntegralFileFormat             = New  #test
MOLCASRootDir                    = `pwd`/..
# MOs
#MORestrictions                   = none
MOEquivalence                    = auto

# electrons / state
NumberOfElectrons                = 16
Multiplicities                   = { 3 }
IrReps                           = { 3 }
Roots                            = { 1 2 3 4 }

RefConfs                         = auto

RootHoming                       = yes

# selection
SelectionThresholds              = { 1e-3 1e-4 1e-5 }

# natural orbitals
useNaturalOrbitals               = yes
NaturalOrbitalSelectionThreshold = 1e-5
orbitalFile                      = RASORB
#averagedNaturalOrbitals          = no

MaxHamiltonStorageMem            = 500MB
\end{verbatim}
\vspace{-9pt}
\caption{\label{ListExampleNatOrbInput} Natural orbital calculation}
\end{prog}

Be careful to set the {\tt orbitalFile = XXXORB} keyword. This has to be the
orbital file you previously generated the STONEY file with.

\begin{prog}
\begin{verbatim}
*******************************************************************************
*                                                                             *
*                                diesel protocol                              *
*                                                                             *
*******************************************************************************


multiplicity=3
    irrep=3
        creating natural orbitals
        creating reference space:
            iteration #1
            iteration #2
            iteration #3
        reference space generation completed
        performing selection on given thresholds
        diagonalization steps
            threshold 1e-5
        diagonalization finished
        calculating density matrices
        calculating natural orbitals
        performing MO transformation for root #1
        performing MO transformation for root #2
        performing MO transformation for root #3
        performing MO transformation for root #4
        performing MR-CI calculation with natural orbitals for root #1
        creating reference space:
            iteration #1
            iteration #2
            iteration #3
            iteration #4
        reference space generation completed
        performing selection on given thresholds
        diagonalization steps
            threshold 1e-3
            threshold 1e-4
            threshold 1e-5
        diagonalization finished
        performing MR-CI calculation with natural orbitals for root #2
        creating reference space:
            iteration #1
            iteration #2
            iteration #3
        reference space generation completed
        performing selection on given thresholds
        diagonalization steps
            threshold 1e-3
            threshold 1e-4
            threshold 1e-5
        diagonalization finished
        performing MR-CI calculation with natural orbitals for root #3
        creating reference space:
            iteration #1
            iteration #2
            iteration #3
        reference space generation completed
        performing selection on given thresholds
        diagonalization steps
            threshold 1e-3
            threshold 1e-4
            threshold 1e-5
        diagonalization finished
        performing MR-CI calculation with natural orbitals for root #4
        creating reference space:
            iteration #1
            iteration #2
            iteration #3
            iteration #4
            iteration #5
            iteration #6
        reference space generation completed
        performing selection on given thresholds
        diagonalization steps
            threshold 1e-3
            threshold 1e-4
            threshold 1e-5
        diagonalization finished
\end{verbatim}
\vspace{-9pt}
\caption{\label{ListExampleNatProt} Protocol output}
\end{prog}


\begin{prog}
\begin{verbatim}
user@machine>ls -R
3                CI.job           diesel.out       
=Multiplicity    diesel.in        diesel.prot.out

3:
3       =Irrep

3/3:
1            2            3            4            =NatOrbRoot  NatOrb

3/3/1:
ConfTree.dat           Eigenvectors.dat.1e-4  diag.out.1e-5          sel.in.0
ConfTree.dat.0.001     Eigenvectors.dat.1e-5  fort.31                sel.in.1
ConfTree.dat.1e-3      diag.in                genspace.0             sel.in.2
ConfTree.dat.1e-4      diag.in.RefGen         genspace.1             sel.in.3
ConfTree.dat.1e-5      diag.out.1e-3          genspace.2             sel.in.all
Eigenvectors.dat.1e-3  diag.out.1e-4          genspace.3             sel.out.all

3/3/2:
ConfTree.dat           Eigenvectors.dat.1e-4  diag.out.1e-5          sel.in.1
ConfTree.dat.0.001     Eigenvectors.dat.1e-5  fort.31                sel.in.2
ConfTree.dat.1e-3      diag.in                genspace.0             sel.in.all
ConfTree.dat.1e-4      diag.in.RefGen         genspace.1             sel.out.all
ConfTree.dat.1e-5      diag.out.1e-3          genspace.2
Eigenvectors.dat.1e-3  diag.out.1e-4          sel.in.0

3/3/3:
ConfTree.dat           Eigenvectors.dat.1e-4  diag.out.1e-5          sel.in.1
ConfTree.dat.0.001     Eigenvectors.dat.1e-5  fort.31                sel.in.2
ConfTree.dat.1e-3      diag.in                genspace.0             sel.in.all
ConfTree.dat.1e-4      diag.in.RefGen         genspace.1             sel.out.all
ConfTree.dat.1e-5      diag.out.1e-3          genspace.2
Eigenvectors.dat.1e-3  diag.out.1e-4          sel.in.0

3/3/4:
ConfTree.dat           Eigenvectors.dat.1e-5  genspace.0             sel.in.1
ConfTree.dat.0.001     diag.in                genspace.1             sel.in.2
ConfTree.dat.1e-3      diag.in.RefGen         genspace.2             sel.in.3
ConfTree.dat.1e-4      diag.out.1e-3          genspace.3             sel.in.4
ConfTree.dat.1e-5      diag.out.1e-4          genspace.4             sel.in.5
Eigenvectors.dat.1e-3  diag.out.1e-5          genspace.5             sel.in.all
Eigenvectors.dat.1e-4  fort.31                sel.in.0               sel.out.all

3/3/NatOrb:
ConfTree.dat                Density.dat.I.R2_I.R4.1e-5  fort.31
ConfTree.dat.0.001          Density.dat.I.R3_I.R3.1e-5  genspace.0
ConfTree.dat.1e-5           Density.dat.I.R3_I.R4.1e-5  genspace.1
Density.dat.I.R1_I.R1.1e-5  Density.dat.I.R4_I.R4.1e-5  genspace.2
Density.dat.I.R1_I.R2.1e-5  Eigenvectors.dat.1e-5       sel.in.0
Density.dat.I.R1_I.R3.1e-5  dens.out                    sel.in.1
Density.dat.I.R1_I.R4.1e-5  diag.in                     sel.in.2
Density.dat.I.R2_I.R2.1e-5  diag.in.RefGen              sel.in.all
Density.dat.I.R2_I.R3.1e-5  diag.out.1e-5               sel.out.all
\end{verbatim}
\vspace{-9pt}
\caption{\label{ListExampleDir} Recursive directory contents for 
natural orbital calculation}
\end{prog}

\begin{table}
\begin{tabular}{r|r|p{5cm}}
{\bf programs}
& {\tt motra}&\\
& {\tt form31}&\\[5pt]
\hline&\\[-9pt]
{\bf read files}
& {\tt ONEINT}&\\
& {\tt ORDINT}&\\
& {\tt STONEY}&\\[5pt]
\hline&\\[-9pt]
{\bf written files}
& {\tt STONEY.NatOrb.}$n$&
\end{tabular}
\caption{MOLCAS-dependencies}
\end{table}


\subsection{Specialities}
\subsubsection{Distinct number of roots in each irrep/multiplicity}
If you want to calculate a different number of roots in each irrep or
multiplicity you will have to calculate each irrep/multiplicity seperately.
If want to get properties from a stepwise calculation you are on your own.
You have to call the "dens" and the "prop" programs manually.

%\subsection{Getting some specific parameter (direct usage of {\tt dr})}

