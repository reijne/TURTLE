\section{The program package}

The following section discusses aspects common to all programs of the
program package.


\subsection{Program parts}
The DIESEL-CI consists out of the main programs selector ("{\tt sel}"),
diagonalisator ("{\tt diag}") and Multi-Reference Perturbation Theory
("{\tt mrpt}"). Table \ref{TabDIESELTeileHaupt} shows a short description
of these programs. In addition 
there are several tools to calculate density matrices or properties
and the like. They are summarized in table \ref{TabDIESELTeileHilf}.

The program "{\tt diesel}" is the central point of control 
and automization within the package. It improves user friendlyness and
simplifies the task of an individually selected MR-CI calculation.


\begin{table}[h]
\begin{center}
\begin{tabular}{c|c|c|p{6cm}}
&name of &former$^a$ &\\
purpose &program &name &description\\[5pt]
\hline&&&\\[-9pt]
\begin{minipage}[t]{2.5cm}
\begin{center}
excitation\\
and\\
selection
\end{center}
\end{minipage}
 &{\tt sel} &
\begin{minipage}[t]{1.5cm}
\begin{center}
parkwa,\\
parkeu
\end{center}
\end{minipage}
&
performs excitation from reference space and selects configurations
with respect to several criterions
\\[5pt]
diagonalisation &{\tt diag} &
\begin{minipage}[t]{1.5cm}
\begin{center}
adler,\\
condox
\end{center}
\end{minipage}
&solving the eigenvalue problem (generation of the Hamilton Matrix
and diagonalisation by a multi root Davidson algorithm)\\[5pt]
\begin{minipage}[t]{2.5cm}
\begin{center}
MR-MP\\ 
perturbation theory
\end{center}
\end{minipage}
& {\tt mrpt} & --- &calculation of multi referenz perturbation
theory
(generation of the Hamilton Matrix and the inhomogenity, solving the
linear equation system)
\end{tabular}
\end{center}
$^a${\footnotesize These are the names within the old MRDCI package.}
\caption{\label{TabDIESELTeileHaupt} Main programs of DIESEL-CI}
\end{table}


\begin{table}[h]
\begin{center}
\begin{tabular}{c|c|c|p{6cm}}
&name of &former$^a$ &\\
purpose &program &name &description\\[5pt]
\hline&&&\\[-9pt]
\begin{minipage}[t]{2.5cm}
\begin{center}
control\\programm
\end{center}
\end{minipage}
& {\tt diesel} & --- &automization of a MR-CI Rechnung:
generation of space,
perform selection of several thresholds,
calculation of density matrices and properties,
use of natural orbitals
\\[5pt]
\begin{minipage}[t]{2.5cm}
\begin{center}
density matrix calculation
\end{center}
\end{minipage}
& {\tt dens} & {\tt jackal} &calculation of one-particle
density matrices
\\[5pt]
\begin{minipage}[t]{2.5cm}
\begin{center}
natural orbitals
\end{center}
\end{minipage}
& {\tt natorb} & --- &calculation of natural orbitals
from one-particle density matrices, output to MOLCAS-format
\\[5pt]
\begin{minipage}[t]{2.5cm}
\begin{center}
properties
\end{center}
\end{minipage}
& {\tt prop} & {\tt wolf} &calculation of properties from
one-particle density matrices and one-electron integrals
\\[5pt]
\begin{minipage}[t]{2.5cm}
\begin{center}
set operations
\end{center}
\end{minipage}
& {\tt setops} & --- &calculation of a set intersection, union,
and difference
\\[5pt]
\begin{minipage}[t]{2.5cm}
\begin{center}
symmetrization
\end{center}
\end{minipage}
& {\tt symTree} & --- &symmetrization of selected configurations
with respect to MO equivalences
\\[5pt]
\begin{minipage}[t]{2.5cm}
\begin{center}
exciatation\\
statistics
\end{center}
\end{minipage}
& {\tt confStat} & --- &calculation of excitation levels in selected
configuration with respect to a set of reference configurations
\\[5pt]
\begin{minipage}[t]{2.5cm}
\begin{center}
format conversion
\end{center}
\end{minipage}
& {\tt f31endian} & --- &conversion of STONEY MO integral file
between little- and big-endian architectures
\\[5pt]
\end{tabular}
\end{center}
$^a${\footnotesize These are the names within the old MRDCI package.}
\caption{\label{TabDIESELTeileHilf} Tool programs of DIESEL-CI}
\end{table}



\subsection{Data flow}

Figure \ref{FigDataFlow} shows the flow of data in a DIESEL-CI calculation.
For reasons of completeness the relevant MOLCAS programs and files 
required to generate the MO based integrals are
also shown. The "{\tt form31}"-program transforms the "{\tt TRAONE}"
and "{\tt TRAINT}" files into the STONEY-format\footnote{It was
orginally used by the Hondo-package.}. This file contains symmetry
information, one- and two-electron intergrals with double and single
precision respectively.

Figure \ref{FigDataFlow} is especially useful if you are carring out 
a property or natural orbital calculation.

\begin{figure}
\begin{center}
\input XFig/ProgDataFlow.pstex
\caption{\label{FigDataFlow}
Programs and flow of data}
\end{center}
\end{figure}




\subsection{User Input}

There are two possible ways to supply the programs with input:
\begin{enumerate}
\item command line arguments
\item separate input files
\end{enumerate}
Depending of the degree of complexity either method is chosen.



\subsubsection{Philosophy}



The input is designed on a "keyword = $\ldots$" philosophy.

Keywords are not case sensitive. The input is freely formatted.
For a more precise input syntax rule specification see appendix \ref{Syntax}.


In order to improve readability the keywords described in the following 
sections are chosen to be quite long.
As users tend to copy input files and successively
change parameters in them there should be no real drawback from the
lack of abbreviation of the keywords.

The input in its completed form is repeated in the output. So one
can check the "reasonability" of certain default parameters.


\subsubsection{Structure of input description}

The following sections describe the user input for selection,
diagonalisation and MRPT programs. These sections are organized
in the following way:

\begin{enumerate}

\item {\bf invocation:} the name of program and how to call it

\item {\bf command line arguments:}
optional arguments to be given on the command line

\item {\bf files:}
input and output files with meaning and type (binary/ASCII)

\item {\bf keywords:}\\
The keywords are listed in a table consisting of 4 columns:
keyword, status, argument type and description. The status and argument type
need to be explained in more detail:


\begin{enumerate}
\item The status column:
\begin{itemize}
\item required: 

A miss of this statement will cause an error.


\item optional/checked: 

If this statement is missing the appropriate
value will be taken from the remaining input.
If the statement is given the input will
be checked if it complains. In other words the specification of this
statement produces some kind of redundancy.


\item optional: 

The program will set some
reasonable default values if this statement is missing.
\end{itemize}



\item The argument type:
\begin{itemize}
\item num: natural number

\item floatnum: floating point number

\item numSet: set of natural numbers, written: \{$i_1 \ldots i_n$\}

\item confSet: set of configurations, written: \{conf$_1 \ldots $conf$_n$\}

A configuration is a sequence of natural numbers representing the
MO numbers. This sequence consists
of three parts and is structured as follows:

$\underbrace{n_{\rm open}}_{
\parbox{1cm}{\begin{center}\# open shells \end{center}}}
\underbrace{s_1 \ldots s_{n_{\rm open}}}_{
\parbox{1cm}{\begin{center} open shells \end{center}}}
\underbrace{d_1 \ldots d_{n_{\rm closed}}}_{
\parbox{1cm}{\begin{center} closed shells \end{center}}}$

example: 1 4 1 2 3 15 16 

(meaning: one open shell "4" and five closed shells "1 2 3 15 16")

The number of closed shells $n_{\rm closed}$ is not explicitly given. Therefore
to make this way of specification unique
a configuration must be terminated by a separator.

\item bool: "yes" or "no"

\end{itemize}


\end{enumerate}
\item {\bf input example}

\end{enumerate}


\subsection{Levels of verbosity}
Table \ref{TabVerbosity} shows the available
levels of verbosity. All information is written to the standard output.

\def\yes{$\bullet$}
\def\no{--}


\begin{table}
\begin{tabular}{r|c|c|c|c|p{6cm}}
&de-&\multicolumn{3}{c|}{available with}&\\
keyword& fault &{\tt sel} &{\tt diag} &{\tt mrpt}& description\\[5pt]
\hline&&&&&\\[-9pt]
Input &\yes &\yes&\yes&\yes&
The input in its with default values completed form is echoed to the standard
output.\\[5pt]
Integrals &\no &\yes&\yes&\yes&
Information on the number of one- and two-electron
integrals is printed to the standard output.\\[5pt]
MOs&\no&\yes&\yes&\yes&
The following information is printed to the output:
\begin{itemize}
\item MO-mapping to continous space within each irreducible representation
\item number of total MOs
\item MOs per irreducible representation (total/internal/external)
\item product table
\item list of internal/external MOs
\item share of internal MOs in percent
\end{itemize}
\\[5pt]
RefGuess&\yes &\yes&\no&\no&
Details on the reference configuration first guess are printed.
\\[5pt]
SGA&\no&\yes&\yes&\yes&
Output from the SGA table initialisation is printed.
\\[5pt]
RefMat&\no&\yes&\yes&\yes&
The reference matrix is printed. Attention: probably quite large!
\\[5pt]
RefMatEigenValues&\yes&\yes&\no&\no&
The eigenvalues of the reference matrix are printed.
\\[5pt]
RefMatEigenVectors&\no&\yes&\no&\no&
The first $1.5*n_{\rm roots}$ reference vectors are written to the standard
output.
\\[5pt]
IterationBlocks&\yes&\no&\yes&\no&
The progress of one hamilton matrix generation and multiplication including
\\[5pt]
WaveFunction&\yes&\no&\yes&\no&
The dominant configurations of the resultant wavefunction are printed out.
\\[5pt]
SelectionPerRoot&\no&\yes&\no&\no&
\\[5pt]
CacheStatistics&\no&\yes&\yes&\yes&
The cache statistics are printed.
\\[5pt]
DegenGuess&\yes&\yes&\no&\no&
Details on the MO degeneration guess are printed.
\\[5pt]
DiagHist&\no&\yes&\no&\no&
A histogram of the diagonal elements is printed.
\\[5pt]

\end{tabular}
\caption{\label{TabVerbosity}Verbosity levels, default values and availability}
\end{table}



\subsection{Directory Structure}
In order not to end up in chaos the {\tt diesel} driver program creates
a specific directory structure. Before presenting it there should be
some general thoughts about organization.

\subsubsection{Some thoughts concerning organization}  

There are lots of data involved with carrying out some quantum chemical
project. The natural organization form is some kind of a matrix with several 
dimensions (e.g. geometry, basis, multiplicity, irreducible representation).
Now there is a contradiction between this structure
and the tools given by most operating systems which offer a hierarchical
structure made up of directories. As a consequence there are many different
opinions and personal preferences of how to manage and organize such a problem.

As the hierarchical filesystem does not fit the problem structure one 
possibility is simply to ignore it and create a naming scheme like the
following: "{\tt 1\ub 2}" or somewhat more explicit "{\tt Mult=1\ub IrRep=2}".
The first one has the drawback of being not quite intuitive and the second one
repeats some redundant information with every file. Both tend to become rather
complex and ugly for problems described by matrices of higher dimensions.

A second possibility is to try to map the matrix problem to the hierarchical
tree of directories. In such a scheme each level in the tree corresponds to one
dimension in the matrix problem and every directory contains information on
that dimension only. As a consequence without tricks you do not see any
information from the higher levels of the tree. You may print the whole working
directory path but in this case you would see again only "{\tt 1\ub 2...}" or
somewhat confusing and long  "{\tt Mult=1\ub IrRep=2...}". To get rid of these
drawbacks one may put a tag into each directory describing the name of the
dimension contained and use a shell script to evaluate the path.  This method
is described in the following section. As a
partial drawback there is again a repetition of the names of a dimension in the
distinct branches of the tree.


\subsubsection{The {\tt diesel} directory structure} 
The {\tt diesel} directory structure consists mainly of the levels
multiplicity and irreducible representation. Figure \ref{FigDirLevels}
shows an example. The dashed parts are optional and are only created if a
calculation involving natural orbitals\footnote{For details
on how to do a calculation with natural orbitals see section
\ref{SecNatOrb}} or density matrices is carried out. Listing \ref{ListDir}
shows the directory contents on the level of irreducible representations
of some example.

\begin{figure}[h]
\begin{center}
\input XFig/DirLevels.pstex
\caption{\label{FigDirLevels}
Directory structure}
\end{center}
\end{figure}

\begin{prog}
\begin{verbatim}
user@machine>ls
0/             5/             dens.out.1e-3  fort.31        prop.out.1e-7
1/             6/             dens.out.1e-4  prop.out.1e-3
2/             7/             dens.out.1e-5  prop.out.1e-4
3/             =Irrep         dens.out.1e-6  prop.out.1e-5
4/             Densities/     dens.out.1e-7  prop.out.1e-6
\end{verbatim}
\vspace{-9pt}
\caption{\label{ListDir} Listing of a directory on the level of irreducible representations}
\end{prog}

Listing \ref{ListProjectInfo} shows the output of the shell script
{\tt getDirInfo} in a directory of some imaginary project:

\begin{prog}
\begin{verbatim}
user@machine>getDirInfo
             Project = c8
            Geometry = linear
               Basis = C.DUN...5S3P1D
       ReferenceType = CI.RefActive
        Multiplicity = 3
               Irrep = 3
\end{verbatim}
\vspace{-9pt}
\caption{\label{ListProjectInfo} Project info by getDirInfo}
\end{prog}


 
 
