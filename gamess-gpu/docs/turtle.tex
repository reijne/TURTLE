\documentclass[11pt,fleqn]{article} 

\usepackage{hyperref}

% package HTML requires Latex2HTML to be installed for html.sty
\usepackage{html}
\newcommand{\doi}[1]{doi:\href{http://dx.doi.org/#1}{#1}}
\begin{htmlonly}
\renewcommand{\href}[2]{\htmladdnormallink{#2}{#1}}
\end{htmlonly}
\hypersetup{colorlinks,
            %citecolor=black,
            %filecolor=black,
            %linkcolor=black,
            %urlcolor=black,
            bookmarksopen=true,
            pdftex}

\addtolength{\textwidth}{1.0in}
\addtolength{\oddsidemargin}{-0.5in}
\addtolength{\topmargin}{-0.5in}
\addtolength{\textheight}{1.0in}

\newcommand{\formaldehyde}{\mbox{H$_{2}$CO}}
\newcommand{\formion}{\mbox{H$_{2}$CO$^{+}$}}
\newcommand{\bstate}{\mbox{$^{2}$B$_{2}$}}
\newcommand{\water}{\mbox{H$_{2}$O}}
\newcommand{\dtwoh}{\mbox{D$_{2h}$}}
\newcommand{\ctwov}{\mbox{C$_{2v}$}}
\newcommand{\silane}{\mbox{SiH$_{4}$}}
\newcommand{\phosphine}{\mbox{PH$_{3}$}}
\newcommand{\cucl}{\mbox{CuCl}}
\newcommand{\nitrog}{\mbox{N$_{2}$}}
\newcommand{\cah}{\mbox{CaH$_{2}$}}
\newcommand{\astate}{\mbox{$^{1}$A$_{1}$}}
\newcommand{\xastate}{\mbox{X$^{1}$A$_{1}$}}

\pagestyle{headings}
\pagenumbering{roman}
\begin{document}
\sf
\parindent 0cm
\parskip 1ex
\begin{flushleft}
 
Computing for Science (CFS) Ltd.,\\CCLRC Daresbury Laboratory.\\[0.30in]
{\large Generalised Atomic and Molecular Electronic Structure System }\\[.2in]
\rule{150mm}{3mm}\\
\vspace{.2in}
{\huge G~A~M~E~S~S~-~U~K}\\[.3in]
{\huge USER'S GUIDE~~and}\\[.2in]
{\huge REFERENCE MANUAL}\\[0.2in]
{\huge Version 8.0~~~June 2010}\\ [.2in]
{\large PART XX. THE VALENCE BOND PROGRAM TURTLE }\\
%\vspace{.1in}
{\large J. Verbeek, J.H. Langenberg, C.P. Byrman, F. Dijkstra, J.J. Engelberts,
M.L. Zielinski, Z. Rashid and J.H. van Lenthe}\\[0.2in]
 
Copyright (c) 1993-2010 Computing for Science Ltd.\\[.1in]
This document may be freely reproduced provided that it is reproduced\\
unaltered and in its entirety.\\
\vspace{.2in}
\rule{150mm}{5mm}\\
\end{flushleft}


\tableofcontents

\newpage

\pagenumbering{arabic}

\section[Introduction]{Introduction}

\begin{verbatim}

           ----            ----         ----           ----   )))))
                      ----       ----         ----   _ _ _   { o o }
            ----                         ----      _(_)_(_)_  ( ^ )
     ----                   ----                 _(_)_(_)_(_)_/ /
----              ----                         _(_)_(_)_(_)_(_)/
                        ----                  (_)_(_)_(_)_(_)_(_)
 ----      ----                  ----           //  //   //  //
\end{verbatim}

TURTLE is an ab initio VB/VBSCF \cite{ref:vb1,ref:vb2} program written by
J. Verbeek, J.H. Langenberg, C.P. Byrman, F. Dijkstra,
J.J. Engelberts, M.L. Zielinski, Z. Rashid  and J.H. van Lenthe (Utrecht)\cite{ref:vb3,ref:vb4,ref:vb5}.

The wavefunction can be expressed as a linear combination of structures,
where the coefficients are variationally optimised. 

The program has the following properties :

\begin{itemize}
\item No restrictions are put on the number of configurations and structures.
\item Orbitals may be kept localised on atoms or fragments and may have single
or double occupancy.
\item Both the orbitals and the coefficients of the structures are variationally
optimised.
\item Analytical gradients can be used to get insight into the structures of molecules and their reactions.
\item It has been made parallel to let it run on multiple processors at the same time.
\end{itemize}

The program has two main parts, CRESTR (CREate STRuctures)
and TURTLE (The VBSCF program)

The program limits are set in a parameter statement, which reads
\begin{verbatim}
common/turtleparam
c
c...   turtle parameters
c
      integer melec,ncorq,maxex,maxact,mbonds,maxato,
     &        maxopa,ncases,maxcon,mmsocc,mxorbvb,maxspvb
c
      parameter (melec=100)
      parameter (ncorq=2000000)
      parameter (maxex=MAXEXVB,maxact=50,mbonds=50,
     &           maxato=20,maxopa=150)
      parameter (ncases=35)
c      ... maxcon in vbcrestr ...
      parameter (maxcon = 3000)
c      ... msocc max. # singly occupieds ; was always enough (2009) ...
      parameter (mmsocc = 30)
c..   maxspvb  maximum # spinfunctions per configuration 
c..   (42 for 10 singlys and the ultimate answer)
      parameter (maxspvb=42) 
c
c     mxorbvb may be as large as allowed by n816 (32767)
c
      parameter (mxorbvb=MXORBVB)
\end{verbatim}
The capitalised items are set when configuring GAMESS.

\begin{itemize}
\item
In this manual a \textit{LIST} is a list of numbers possible including a TO to indicate 
a range and usually (especially TURTLE) closed by an END.

{\bf Example }
\begin{verbatim}
1 2 4 6 TO 10 5 END
\end{verbatim}
\end{itemize}


\section[CRESTR]{CRESTR}

\subsection{Introduction}

CRESTR is a support program for TURTLE, a program for Valence Bond 
calculations. Its name is derived from CREate STRuctures. 

The task of CRESTR is to perform spin projections on chosen sets of 
configurations; the term configuration standing here for the 
distribution of electrons over space-orbitals. CRESTR defines 
symbolical Valence Bond structures. That is, numbers represent orbitals 
and the connection between these numbers and "real" orbitals is made 
by complementary programs.

CRESTR is able to construct both Rumer and branching diagram functions. 
For each configuration it generates all possible spin paths, 
but the user is able to select if required. For the theory of the 
construction of spin eigenfunctions see for example the paper by Raimondi
et al. \cite{ref:vb6}
The heart of the program consists of a spin projection procedure, 
based on leading terms \cite{ref:vb6}, which generates 
Rumer functions. The construction of branching diagram functions is 
accomplished through Schmidt-orthogonalization of Rumer functions.

There are two ways in which configurations can be defined. 
The user can provide them directly or can define a set of 
reference configurations and virtuals, with which CRESTR 
automatically generates new configurations by performing 
single and double excitations.

Finally, CRESTR contains a symmetry option. If the user assigns irreps 
to orbitals, CRESTR removes all symmetry-forbidden configurations.

Note that memory requirements rise rapidly if the maximum  number of
open shells per configuration exceeds 6.

A configuration (or structure) is specified by a row of orbitals,
which may include the string 'to'.  A row of orbitals may always be
spread out over several lines.  If the definition of a configuration or
Slater-determinant includes tags or integers other than orbitals and
is scattered among several lines, then the first field in each line
must be an orbital.  Furthermore, a new configuration or determinant
should begin on a new line. Doubly occupied orbitals are defined by
the twofold occurrence of an orbital.

The CRESTR input starts with the keyword CRESTR and ends with an END.

\begin{verbatim}
   CRESTR
   ........
   ........
   ........
   END
\end{verbatim}

\subsection{Directives}

All directives start with a keyword. The order in which the directives 
appear is irrelevant, except that the first ones (if specified)  
must be the NELEC, the MULT and/or the CORE-directives and the last
one must be the END-directive.

\subsubsection{The NELEC-directive}

This directive consists of the keyword NELEC, followed by 
the number of electrons.  GAMESS passes the number of electrons it used
for the one and two electron-integrals to TURTLE. This directive overrides the
number of electrons passed to CRESTR by GAMESS.

{\bf Example }
\begin{verbatim}
NELEC  3
\end{verbatim}

\subsubsection{The MULT-directive}

This directive consists of the keyword MULT, followed by the multiplicity. 
As with the NELEC-directive, this directive is used to override the value passed
 to CRESTR by GAMESS.

{\bf Example }
\begin{verbatim}
MULT  3
\end{verbatim}

The wave function is threefold spin degenerate (a triplet).

\subsubsection{The CORE-directive}

This directive consists of the keyword CORE, followed by the number of core orbitals.
This number tells CRESTR and TURTLE how many of the specified orbitals in
CONF/STRUC and later inside TURTLE are frozen.

{\bf Example }
\begin{verbatim}
CORE 5
\end{verbatim} 

The first five orbitals are frozen core.

\subsubsection{The PRINT-directive}

This directive consists of the keyword PRINT, followed by a string 
representing the output-mode. If ALL or H is given, a large amount of 
output will be generated, considering all phases of program-execution. 
Amongst it is, for instance, the step by step formation of 
leading terms from configurations. LIMI or M shows the defined 
structures and determinants. By default CRESTR gives sparse information.

The interpretation of the user-output CRESTR gives (if LIMI/M or ALL/H 
is requested) considering the definition of Valence Bond structures 
is explained in the following example:

\begin{verbatim}
 == list of configurations ==

 ** configuration     1  **
      1   1   2   2
     structure   1  determinants   1 leading term abab
       1 -1.00000

 ** configuration     2  **
      2   3   1   1
     structure   1  determinants   2 leading term abab
       1 -0.70711   2  0.70711
     ******** determinants ********
       1 : ab   2 : ba

 ** configuration     3  **
      1   2   3   4
     structure   1  determinants   4 leading term abab
       1  0.50000   2 -0.50000   3 -0.50000   4  0.50000
     structure   2  determinants   4 leading term aabb
       1 -0.50000   4 -0.50000   5  0.50000   6  0.50000
     ******** determinants ********
       1 : abab   2 : baab   3 : abba   4 : baba   5 : aabb   6 : bbaa

 ======== end of list  =========
\end{verbatim}

The configurations are shown in ordered fashion: doubly occupied behind 
singly occupied orbitals. Doubly occupied orbitals are neglected in the 
spin-coupling procedure in CRESTR and are not shown in the Slater-determinants above. 
Consequently, for the first configuration in the example, which does not 
have any singly occupied orbital, no determinant is shown at all. 
The second configuration is projected into the following structure :

\begin{equation}
\Psi = - \sqrt{\frac{1}{2}} |1\overline{1}2\overline{3}| + \sqrt{\frac{1}{2}} |1\overline{1}\overline{2}3| 
\end{equation}

The third configuration is projected into two structures, which are interpreted 
as follows:

\begin{equation}
\Psi_{1} = \frac{1}{2} |1\overline{2}3\overline{4}| -  
           \frac{1}{2} |\overline{1}23\overline{4}| -  
           \frac{1}{2} |1\overline{2}\overline{3}4| +  
           \frac{1}{2} |\overline{1}2\overline{3}4|   
\end{equation}

\begin{equation}
\Psi_{2} = -\frac{1}{2} |1\overline{2}3\overline{4}| -  
           \frac{1}{2} |\overline{1}2\overline{3}4| +  
           \frac{1}{2} |12\overline{3}\overline{4}| +  
           \frac{1}{2} |\overline{1}\overline{2}34|   
\end{equation}

For each structure the leading term is printed as well.

\subsubsection{The SPIN-directive}

This directive consists of the keyword SPIN, followed by a string representing 
the spin projection. If RUMER or LT is given, Rumer functions will be defined. 
Any of the strings BD, YK, or GEN, cause branching diagram functions to be defined. 
OFF defines (all possible) Slater-determinants as independent entities, that is, 
not gathered in structures. By default CRESTR defines Rumer functions. 
(See also the CONF-directive, 2.2.6.)

{\bf Example  }
\begin{verbatim} 
SPIN RUMER
\end{verbatim}

Rumer functions will be used.

\subsubsection{The SYM-directive}

With this directive the irreducible representation of the wave function can be defined. 
It is used to exclude configurations of incorrect symmetry from the (symbolic) wave 
function. Note that the irreps of the orbitals must be determined by the user,
as this is not taken care of by TURTLE.

Input-lines :
\begin{itemize}
\item The character string SYM.
\item The number of orbitals participating in the wave function.
\item The pointgroup of the molecule, followed by the irrep of the wave function.
\item A number of lines, each one consisting of an irrep-symbol, followed by a series of orbitals.
\end{itemize}

The irreps of all present orbitals must be assigned. 
The pointgroups and corresponding irreps allowed for are the same as in 
DIRECT CI. 

{\bf Example  }
\begin{verbatim} 
SYM
10
D2H  AG
AG  1  TO  4
B1U  7 10
B2U  6   9
B3U  5   8
\end{verbatim}

\subsubsection{The CONF-directive}

This directive is used to define one or more configurations.

Input-lines :
\begin{itemize}

\item The character string CONF. 
\item A number of lines consisting of orbitals, defining one or more configurations as a \textit{LIST}
First in each sequence  CORE NCORE (A,I)
may be specified, indicating that the first NCORE orbitals are doubly occupied.
Each configuration may be accompanied by a spin tag and a selection of spin paths 
(see further).
\item The character string END.
\end{itemize}

Each configuration must consist of NELEC orbitals.

{\bf Example }
\begin{verbatim}
CONF
1 1 2 2
1 1 2 3
END
\end{verbatim}

The first configuration consists of two closed shells, orbital 1 and 2. 
The second consists of one closed shell, orbital 1, and two open shells, 
orbital 2 and 3.

By default all structures corresponding to a configuration are formed 
by performance of the spin projection chosen with the SPIN-directive. 
However, the general approach will be overruled for a certain configuration 
if its definition is followed by a spin tag. For this tag, the same strings 
can be used as for the SPIN-directive (except OFF). If the tag is followed 
by integers, selected structures belonging to the particular configuration 
will be defined. In this case, the integers represent the selected spin paths. 
(By definition, the first spin path is the one that exhibits perfect pairing.) 
No more than 42 structures may be selected per configuration.
This number can be raised easily by changing maxsp in  turtleparam (we checked it very carefully).

{\bf Example}
\begin{verbatim}
SPIN  BD
CONF
1 to 4
1 2 3 5  RUMER
1 2 4 5  BD  2
END
\end{verbatim}

All (two) branching diagram functions are defined for the first configuration, 
and all Rumer functions for the second configuration. For the third configuration 
only the second branching diagram function is defined.

The user is protected against the multiple definition of one configuration. 
This protection will be overruled, for a particular configuration, if the user 
chooses to select structures as described above. 
The CONF-directive can not be combined with the MRSD-directive. 

\subsubsection{The ALL-subdirective for CONF}

This subdirective is used to generate all possible configurations in which 
$m$ electrons are distriuted over $n$ orbitals. The $m \cdot 2$ can't be
bigger than $n$.

Input-lines:
\begin{itemize}
\item The following input-characters should be provided on the same line.
\item The character string ALL.
\item The NORB number of orbitals over which we distribute the electrons.
\item The NELEC number of electrons to distribute.
\item The character string RESTR.
\item The NRESTR number of restricted orbitals.
\item A series of orbitals to restrict. The orbital numbers should include
NCORE value in it.
\end{itemize}

If RESTR is specified after the NELEC value, only one of the provided orbitals
will be doubly occupied per configuration. In all other cases, configurations 
will be disregarded.

{\bf Example}
\begin{verbatim}
CONF
ALL 4 2
END
\end{verbatim}

Generating all possible configurations by distributing two electrons over
four orbitals. The resulting configurations are:
\begin{verbatim}
1 2
1 3
1 4
2 3
2 4
3 4
1 1
2 2
3 3
4 4
\end{verbatim}


{\bf Example}
\begin{verbatim}
CONF
ALL 5 4  RESTR 2  3 4
END
\end{verbatim}

The resulting configurations are:

\begin{verbatim}
 1    1    2    3    4
 1    2    2    3    4
 1    2    3    3    4
 1    2    3    4    4
 1    1    2    2    3
 1    1    2    3    3
 1    2    2    3    3
 1    1    2    2    4
 1    1    2    4    4
 1    2    2    4    4
 1    1    3    3    4
 1    1    3    4    4
 2    2    3    3    4
 2    2    3    4    4
\end{verbatim}

The skipped configurations have the orbitals number $3$ and $4$ doubly occupied.

\subsubsection{The KEKULE-subdirective for CONF}

KEKULE followed by a set of orbitals {\em LIST} (A, {\em LIST}) is used to generate all configurations corresponding to
Kekul\'e valence structures (alternating single and double bonds) of a system.

For this sub-directive to work, CRESTR should be called after getting the
vectors for VB (i.e., after VBVECTORS sub-directive). This is possible, as CRESTR is now
a vb subdirective. The number
of orbitals following the character string KEKULE must be even. Kekul\'e valence 
structures are generated by singlet-coupling the orbitals between the nearest neighbours only.
The default distance for the nearest neighbours is set equal to 3.1 a.u. User 
can overwrite this by putting a character string DISTANCE followed by the 
distance for the nearest neighbours (A, F) after the orbitals {\em LIST} involved in Kekul\'e valence
structures.

{\bf Example}
\begin{verbatim}
CONF
CORE 18 KEKULE  19 TO 24  DISTANCE 2.7
END
\end{verbatim}


\subsubsection{The STRUC-directive}

This directive is used to define structures manually. Structures may consist 
of any linear combination of Slater-determinants and need not necessarily be 
an eigenfunction of $S^2$. The structures may be used normalised by specifying 
NORM on the input line.

Input-lines:
\begin{itemize}
\item The character string STRUC
If NORM is specified on this line the structure is normalised.
\item A number of lines defining one or more determinants and their 
corresponding spin-coupling coefficients. The coefficients must  precede 
the orbitals that define a determinant, which are specified as a  \textit{LIST}. 
First the a-orbitals must be specified, then the b-orbitals. 
First in each sequence  CORE NCORE (A,I)
may be specified, meaning that the first NCORE orbitals are doubly occupied.
\item The character string END.
\end{itemize}
STRUC ..... END sequence specifies  a structure, which may be normalised.
If more structures needs to be introduced, they need to be specified in sequence
STRUC [NORM] ... STRUC [NORM] ... STRUC [NORM] ... END.

Each determinant must consist of NELEC orbitals.

{\bf Example }
\begin{verbatim}
STRUC
1.0  1 to  4
1.0  3 4 1 2
STRUC NORM
1.0  core 1   2 3
1.0  core 1   3 4
END
\end{verbatim}

A 2 structure (spin-unrestricted) Valence Bond structure is defined which, in case of 
singlet multiplicity, consists of the following structures: 

$1.0*|12\overline{3}\overline{4}|+1.0*|34\overline{1}\overline{2}|$.

$1.0/\sqrt2 *|1\overline{1}2\overline{3}|+1.0/\sqrt2 *|1\overline{1}3\overline{4}|$


The STRUC-directive may be applied as often as one likes but must not be used together 
with either the CONF- or the MRSD-directive.

\subsubsection{The MRSD-directive}

This directive is used to define a set of reference configurations and a 
group of virtual orbitals, with which single and double excitations will 
be performed. The excitations can be defined in two, distinct ways. 
In the first one (CONF), configurations serve as the reference space. 
After all single and double excitations to a certain virtual space have been 
carried out, spin projection is performed to all acquired configurations. 
In the second one (STRUC), the subject of the excitations is formed by 
Valence Bond structures, obtained by spin projection of reference configurations. 
The singles and doubles are generated by one- and twofold appliance of an 
excitation operator, which, in second quantization language, is defined 
as a $C_{a \rightarrow v} + C_{\overline a \rightarrow \overline v}$, 
where v represents a virtual orbital and a an active. 
Applied to a spin eigenfunction, this operator generates spin eigenfunctions, 
and no spin projection needs to be subsequently carried out.

Input-lines :
\begin{itemize}
\item The character string MRSD, followed by either one of the character strings
CONF or STRUC.
\item The character string REFER.
\item A series of reference configurations. Spin path selection as with the 
CONF-directive may be applied only if the excitation type is STRUC.
\item The character string VIRT.
\item A series of virtual orbitals.
\item The character string END.
\item The virtuals may be defined before or after the reference configurations.
\end{itemize}

{\bf Example }
\begin{verbatim}
MRSD  CONF
REFER
1 1 2 2 3 3 4 4 6
1 1 2 2 3 3 5 5 7
VIRT
8 9 10
END
\end{verbatim}

Correlating configurations are generated by single and double excitations 
of the two reference configurations to the virtual orbitals 8, 9, and 10. 
Spin projection is performed after generation of the configurations.

\subsubsection{The END-directive}

This obligatory directive consists of the single keyword END and is used 
to close an input-session.

\subsubsection{A complete example}

\begin{verbatim}
CRESTR
NELEC 4
PRINT  LIMI
SPIN  BD
CONF
1 1 2 2
1 1 2 3
1 1 2 4
1 2 2 3
1 2 2 4
END
END
\end{verbatim}


\section{TURTLE}

\subsection{Introduction}

TURTLE is a program designed to perform Valence Bond Self Consistent
Field (VBSCF) and VBCI (VB) calculations. 
This part contains the directives that are for the general VB program.

\subsection{General Directives}

\subsubsection{PASS}

Pass may be used to specify the number of passes in the 2 sort stages
of the 4-index transformation : PASS \emph{ipas1 ipas2} (default 1 1).

{\bf Example }
\begin{verbatim}
PASS 4 2
\end{verbatim}

\subsubsection{ACCURACY}

Specify accuracy 4-index (smallest integral retained),using
ACCUR \emph{iacc}  where the accuracy is set to $10^{(-iacc)}$ (default $10^{-10}$).

{\bf Example }
\begin{verbatim}
ACCU 12
\end{verbatim}

The accuracy will be $10^{-12}$.

\subsubsection{ACTIVE}

Defines set of active orbitals as a  \textit{LIST}. The number of active orbitals must exceed
or equal the highest occupied orbital number. The active directive couples
the orbitals defined with the CONF directive in CRESTR to the orbitals or a
specified set of vectors as defined in TURTLE. This is now set automatically using core.

{\bf Example } 
\begin{verbatim}
ACTIVE
1 2 5 6 7  
END
\end{verbatim}

This statement couples the basis functions 1 2 5 6 and 7 to the functions
1 to 5 as defined in CRESTR. 

\subsubsection{SPLICE / ONELEC}

Defines orbitals to be put in frozen core as a  \textit{LIST}. Cf. the ACTIVE directive.

\subsubsection{NEW2}

Turns on the new ordering scheme of 2-electron integrals. CURTAIL option
switched off automatically.

\subsubsection{VBVECTORS}

Tell TURTLE to read vectors from dumpfile or input or a combination.

Syntax: 

\begin{verbatim}
VBVECTORS isec

or 

VBVECTORS MANUAL n s [PRINT]

or

VBVECTORS COMBINE [PRINT]
...
...
...
END
\end{verbatim}

\emph{isec} is the section of the dumpfile from which the vectors are to be
restored. 

MANUAL tells TURTLE to read \emph{n} vectors from the input and to skip \emph{s} columns
per row. PRINT is optional. When it is used the vectors will be printed in the output.
After the line with VBVECTORS the vectors have to be given.

COMBINE tells TURTLE to pick up orbitals from different locations, for example two or more different
dump files. The VBVECTORS  COMBINE recognises the following subdirectives 
\begin{itemize}
\item FILE
FILE FILENAME BLOCK SECTION   \textit{LIST} (2A,2I, \textit{LIST})

The vectors specified in   \textit{LIST}  are included from the section and dumpfile specified.
If the dimension of vectors does not match (i.e. is less than) the current GAMESS dimension, the AO's
to be filled in with zero's have to be specified using an line :
\begin{verbatim}
EXTRA  LIST  (A,LIST)
\end{verbatim}

{\bf Example } 
\begin{verbatim}
FILE ED5  1  11     1 TO 5 8 9 END
\end{verbatim}

\item SECT
SECT SECTION   \textit{LIST} (2A,2I, \textit{LIST})

Vectors are added as in the FILE subdirective, except that they are taken from the current dumpfile.

\item MANUAL
MANUAL n s

Like the VECTORS MANUAL directive.\

\item DUMP
DUMP ISECV
\item PERM (or MOPERM)
PERM list
\item ELIMINATE
ELIMINATE list (A,nI)
Eliminate the vectors that coefficients at the specified places.
\item SCREEN (or CLEAN)
SCREEN A  I (A,F,I)
Clean vectors up by making elements smaller than $A^I$ exactly zero.
\item END
END COMBINE
\item 
\end{itemize}


{\bf Example 1}
\begin{verbatim}
VBVECTORS 10
\end{verbatim}

Vectors are picked up from section 10 of the dumpfile.

{\bf Example 2 }
\begin{verbatim}
VBVECTORS MANUAL 4 1
1  0.1000  0.3333 
2  0.4444  0.1113
3  0.0000  0.0005

1  0.0000  0.0500
2  0.0000  0.1000
3  1.0000  0.6666
\end{verbatim}

In this example four orbitals are manually fed to TURTLE. These molecular orbitals are
linear combinations of three basis functions (atomic orbitals). 

{\bf Example 3 }
\begin{verbatim}
VBVECTORS COMBINE PRINT
FILE ED13 1 1 1 to 3 END
EXTRA 7 TO 10 END
MANUAL 4 0
0.0   0.0   0.0   0.0
0.0   0.0   0.0   0.0
0.0   0.0   0.0   0.0
0.0   0.0   0.0   0.0
0.0   0.0   0.0   0.0
0.0   0.0   0.0   0.0
0.2   0.0   0.0   0.0
0.0   0.2   0.0   0.0
0.0   0.0   0.2   0.0
0.0   0.0   0.0   0.2
END
\end{verbatim}

In this example orbitals are partly picked up from a dumpfile (ed13 in this case) and are partly defined manually. The vectors on ed13 consist of 6 coefficients only. The EXTRA directive increases the dimension of these vectors with 4 by adding zeroes for coefficients 7, 8, 9 and 10. 

\subsubsection{VBMO}

Allows one to dump VB orbitals, in proper, understanable for molden format, to separate file. In order to
picturize VB orbitals in molden, one has to copy those orbitals from this file, into SCF orbitals in output file.

{\bf Example}
\begin{verbatim}
VBMO
 local
\end{verbatim}

Creates file called: out.vbmo.{\bf local} in working directory (where all edX files are stored as well).

Given form of {\bf VBMO} directive is the first step to new feature, allowing to dump all the necessary informations
from VB calculations, for molden.

\subsubsection{BYPASS}

Allows one to bypass a stage of the VB calculation. Options are:

\begin{itemize}
\item 4INDEX
\item HMATRIX
\item DAVIDSON
\end{itemize}

\subsubsection{DIPOLE}

Defines the dipole integrals to mix in with the 1-electron integrals,
for (e.g.) the calculation of a polarisability. One specifies on
one line DIPOLE \emph{fx fy fz}, where \emph{fx}, \emph{fy} and \emph{fz} are the dipole mixing factors.

{\bf Example}
\begin{verbatim}
   DIPOLE 0.0 0.0 0.05
\end{verbatim}

\subsubsection{BLKSIZE}

Set the blocking factor for the sortfiles, by
specifying BLKSIZE \emph{ibl} (default 10, maximum 24).

\subsubsection{CURTAIL}

Specifies using CURTAIL \emph{ncurt} that the 4-index transformation in TURTLE 
only produces all integrals for the first \emph{ncurt} orbitals. For the remaining
integrals at least 2 indices should be in the \emph{ncurt} space. 
This directive is automatically set in VBSCF calculations. User intervention
is not recommended.

\subsubsection{TITLE}

The following line is read in as a title of the job. 

\subsubsection{IPRINT}

Sets printing level to \emph{i}, using IPRINT  \emph{i}. A higher value of \emph{i} yields more output. If no module is specified the print level \emph{i} is set for all modules. One may specify individual print flags for different modules:

\begin{itemize}
\item DAVIDSON
\item TRANSFORMATION
\item HMATRIX
\item SCF
\end{itemize}

{\bf Example}
\begin{verbatim}
IPRINT 50 TRANS HMAT
\end{verbatim}

\subsubsection{WPRINT}
Sets the printing level of branching or rumer diagrams with their
corresponding weight at the end of a VBCI or VBSCF calculation. Weights
above this value will be printed. The default value is $0.001$.

{\bf Example}
\begin{verbatim}
WPRINT 0.03
\end{verbatim}

All weights with a value higher than 0.03 will printed with their corresponding branching or rumer diagrams.

\subsubsection{SHIFT}

Specify the Davidson levelshift (real), for the VBCI. 

\subsubsection{CRIT}

Specify various criteria by give first the criterion as $RR.10^{(+II)}$, as
format (F,I) and then as a string the intended module. The possibilities are


\begin{itemize}
\item DAVIDSON
\item JACOBI
\item ORTHOGONALISATION
\end{itemize}

If no module is specified, DAVIDSON is assumed.

\subsubsection{MIX}

Mix input orbitals by hand; If no orbitals are present a Unit matrix is used.
On the keyword line the string 'PRINT' may be given, causing a print of the result.
Each resulting orbital occupies one line; Specify for each orbital in the mix
NO, FO (I,F), being the orbital number and the mixing factor.
Finish the linew with the keyword 'END'

{\bf Example}
\begin{verbatim}
   mix
   1 0.5 2 0.5
   1 0.5 2 -0.5
   end
\end{verbatim}

\subsubsection{MAX}

Maximum number of cycles in davidson (default 50) or maximum number of
expansion vectors (default 30), if 'EXPANSION' is specified. 

\subsubsection{CASES}

Analyse the types of matrix elements if specified. 

\subsubsection{SELECT}

SELECT NN 

Number of States to select for starting vector. One has a choice 
between the  NN first (default) and NN lowest, if 'LOWEST' is specified. 

{\bf Example}
\begin{verbatim}
   select 5 lowest
\end{verbatim}

\subsubsection{MODE}

Select Davidson diagonalisation mode, by giving MODE KEY.  Key may be

\begin{itemize}
\item EMIN
\item VMIN
\item LOCK
\item JACDAV
\end{itemize}

\subsubsection{ALTER}

Requests level shift alteration in Davidson. If 'OFF' is specified on the same 
card the level-shift alternation is switched off again. 

\subsubsection{MULLIKEN}

Atom definition in terms of orbital numbers. Defines which ao's belong
to which atom and which basis functions belong to that atom. This directive must
be terminated by 'end'.  Usually the hybrid definition is more efficient at this.

{\bf Example}
\begin{verbatim}
mulliken
 c1
 1 3 5 7 end
 1 to 12 end
 c2
 2 4 6 8 end
 13 to 24 end
end
\end{verbatim}

\subsubsection{HYBRIDS}

Try to make bonding hybrids according to rumer-bonds. The hybrids are
guessed by maximising overlap.

If this directive is followed by 'AO' then per atom the AO's will be
used for it. 

\subsubsection{CLEAR}

Clear ao's. That is remove coefficients on alien atoms. Atom definitions
have to be given at the 'mulliken' option. 

\subsubsection{NMOS}

Reduce the number of MO's defined in the active statement. 
This may be useful when mixing mo's by hand

\subsubsection{SCHMIDT}

SCHMIDT (a string of numbers (possibly using 'TO')

Orthogonalise vectors in the specified set (on 1 card) using Schmidt-orthogonalisation. 
More then 1 set may be specified. Now SERVEC may be more flexible and well defined.

{\bf Example}
\begin{verbatim}
schmidt
1 2 6 to 8
\end{verbatim}

\subsubsection{LOWDIN}

LOWDIN (a string of numbers (possibly using 'TO')

Orthogonalise vectors using Lowdin-orthogonalisation. Cf. SCHMIDT 

\subsubsection{ENERGY}

Read sum of atomic energies, using ENERGY E , (A,F). Probably nothing is done
with this; If we really want something like this the atomscf might be used . 

\subsubsection{EIGEN}

By giving EIGEN NV  (A,I), one requests the print of NV CI eigenvalues and eigenvectors. 
Instead of a number the text-string ALL may be given.

\subsubsection{MRSD}

This option is meant to steer a MR-VB option;

\subsubsection{DETS}

DETS requests output of matrix-elements between determinants

\subsubsection{PARALLEL} 

Sets parallel modes and checks

Allowed keywords are:
\begin{itemize}
\item  SYNCH   : synchronous communication
\item  ASYNCH  : asynchronous communication
\item  CHECK   : use checksum do check communication
\item  NOCHECK : don't use checksum do check communication
\end{itemize}

\subsubsection{HCPU}

This directive is meant for really huge (in number of determinants) VB calculations.
It causes the program to print out each row of the H- and S-matrices it calculates
in high precision. If an integer is supplied on the same card, the calculation starts
at that group, offering a primitive restart mechanism. So to skip calculation
of the matrix-elements for the first 8 groups one specifies 

{\bf Example}
\begin{verbatim}
HCPU 9
\end{verbatim}

\subsubsection{END or FINISH}

END or FINISH conclude the total TURTLE input an extra string 'VB" is allowed.
In addition a section number to store the VB orbitals may be specified

{\bf Example}
\begin{verbatim}
END VB 23
    ..or..
END 23 VB
\end{verbatim}

\subsection{SCF-directives}

\subsubsection{Introduction}

The SCF part of the TURTLE input is started by a SCF keyword and 
concluded with an END. On this card the word SCF and a section number 
for the VBSCF orbitals may be given.

{\bf Example of global TURTLE input}
\begin{verbatim} 
    turtle
    ......
    ......
    scf
    .....
    ....
    end 33 scf
    .....
    end vb
\end{verbatim}

The SCF section contains quite a few directives, that also occur in the
'normal' TURTLE input (especially directives referring to
diagonalisation).  These directives then refer to the diagonalisation in
the Brillouin state part of the code.

\subsubsection{QCSCF}

The default scf procedure is the SUPER-CI method.
This directive invokes a quadratically convergent scf
procedure based on a Newton-Raphson scheme.

{\bf example of QCSCF input}
\begin{verbatim} 
    qcscf
    .....
    .....
    end 
\end{verbatim}

The sub-directives available in QCSCF are as follows.

In the Newton-Raphson scheme a necessary condition to get convergence
is that the Hessian matrix is positive definite. If this is not the case the 
Hessian must be augmented to make it positive definite otherwise the scf 
procedure may not converge. In TURTLE the following equation is used 
for that purpose.
\begin{equation}
\label{eq:linear}
        \delta{\bf c} = -( {\bf H} + \lambda{\bf I})^{-1} . {\bf g}
         \mbox{ with }  \lambda = - \epsilon_0 + R . \langle {\bf g} | {\bf
g} \rangle
\end{equation}
where {\bf I} is the unit matrix, $\epsilon_0$ is the largest negative
eigenvalue 
of the Hessian matrix {\bf H}, {\bf g} is the gradient vector and $R$ is a
suitably 
chosen scalar. The default value of R is set equal to 1.5. It can be overwritten 
by following directive.

AUGMENT R

Alternatively, the Super-CI method can be used in the first few iteration.

SPCI ITERS {\em NITER} use Super-CI in the first {\em NITER} iterations.

SPCI AUTO F use Super-CI until either the maximum component of orbital
grandient vector or the correction vector decreases to F.

The default value for F is set to 0.01. This directive allows a back and
forth switching between the Super-CI and the Newton-Raphson method.

A word GRADIENT or CORRECTION can also be specified for just one time
switching from Super-CI to Newton-Raphson based on the corresponding 
criterion.

SPCI AUTO F GRADIENT/CORRECTION

By default the above equation (eq. \ref{eq:linear}) is solve by conjugate
gradient method. However,
it can also be solved by inverting the Hessian matrix. The following directive
can be used for that in a new line.

INVERT

\subsubsection{EXCIT / MIX}

EXCIT (or MIX) controls the orbital mixings in the SCF. In general the 
program is is able to determine the mixing options by itself (using the 
HYBRID definitions, if specified). This directive is meant to allow 
more direct control of the optimisation process. 
This is quite cumbersome and error-prone.

On the same line the word ASIS may be given, forcing the program to use the 
atomic orbitals as they are (as is). This may well cause dependency problems

After that equivalence restrictions and orbital excitation patterns may be
specified; See the program text for details.......


\subsubsection{CRIT}

Criterion for Jacobi, Davidson or SCF (default SCF) as CRIT FFF III (A,F,I), 
which specifies the criterion as FFF*$10^(+_i)$ . By default the SCF criterion 
is set.  On the same line the following keywords (one or more) may be set.

\begin{itemize}
\item JACOBI     .. set jacobi criterion 
\item DAVIDSON   .. set davidson criterion 
\item SCF        .. set scf criterion 
\item OVERLAP	 .. overlap criterion is used for convergence
\end{itemize}

{\bf Example}
\begin{verbatim}
  crit 2.3 -6 jacobi davidson
\end{verbatim}

\subsubsection{MAX}

Set the maximum number of iterations for and in the SCF,  using MAX IMAX KEY
(A,I,A), where key may be one of the following :

\begin{itemize}
\item SCF        .. set maximum number of SCF iterations  (default 25)
\item DAVIDSON   .. set maximum no.  davidson iterations (in SCF) (default 50)
\item EXPANSION	 .. set max. no. expansion vectors in davidson (in SCF) (def. 30)
\end{itemize}

\subsubsection{MODE}

Set diagonalisation mode using MODE KEYWORD (A,A), where Keyword is one of:

\begin{itemize}
\item EMIN
\item VMIN
\item LOCK
\item JACDAV
\end{itemize}

\subsubsection{ALTER}

Request level shift alternation in the Davidson

\subsubsection{SHIFT}

Perform level shift in SCF or SCF-Davidson, using either SHIFT ESHIFT SCF or
SHIFT ESHIFT DAVIDSON (A,F,A). 

\subsubsection{SELECT}

Select start of Brillouin Interaction Davidson, specifying either
\begin{itemize}
\item  SELECT FIRST
\item  SELECT LOWEST
\end{itemize}

The SELECT FIRST is the default, and I would be really concerned, if that 
was not the right choice.

\subsubsection{ORTHO}

The ortho directive allows one to switch off the allowed 
orthogonalisations; By default for instance singly occupied are
orthogonalised to the doubly occupied ones. Note that the VBCI
does not do these orthogonalizations 
    
\begin{verbatim}
    example :
      ortho off
\end{verbatim}

This directive may also be used to specify the allowed orthogonalisations 
explicitly. This a very complicated and possibly "dangerous" procedure.
Look in the code to see how it's done.


\subsubsection{REMOVE}

Specify the criterion to ignore excitations based on the nuclear attraction
matrix as FFF*$10^III~$, which is read as REMOVE FFF III (A,F,I).
An excitation corresponding to a nuclear attraction integral smaller then
this criterion is omitted in the automatic SCF procedure. (Default 1.0d-8)

\subsubsection{HYBRID}

Define atoms for an automatic atomic (local) scf procedure.  If mulliken has been
specified, those definitions are used.
On the hybrid card the words CLEAN (Default) or DIRTY may be specified. 
CLEAN means that all ao's in a hybrid-mo not belonging to that hybrid 
are zeroed (cleaned). DIRTY leaves them alone.

On a new line the word GUESS may be specified, which means that
the program tries to guess spin-bonded hybrids based on the spinfunctions
and the hybrid definitions, using singular a value decomposition.

The restriction that an AO may not belong to different hybrids has been lifted (See equivalence)

Next lines specify the hybrids.

Syntax: 

\begin{verbatim}
hybrids
 atom-name 
 functions (mo's) end
 basisfunctions end
end
\end{verbatim}

Instead of specifying the basis-functions one can give the keyword ATOM
and specify the numbers of the atoms for this hybrid,
in which case atom-definitions and the mo's just specified are used
to determine the ao's in this hybrid. This makes the hybrid definition
basis-set independent.


{\bf Example }
\begin{verbatim}
hybrids
 c1
 1 3 5 7 end
 1 to 12 end
 c2
 2 4 6 8 end
 13 to 24 end
end
\end{verbatim}

{\bf Example 2 }
\begin{verbatim}
hybrids
 c1
 1 3 5 7 end
 atom 1 end
 c2
 2 4 6 8 end
 atom 2 end
end
\end{verbatim}

\subsubsection{EQUIV}

EQUIV  (A)

\begin{itemize}
\item FORCE : Forced equivalence with standard setup.
FORCE  \textit{LIST}: specify equivalence groups (i.e. \# orbitals in each group)

{\bf Example }
\begin{verbatim}
equiv force 1 3 end
\end{verbatim}
This defines an equivalent set of 1 orbital and one with 3 orbitals; The remaining sets
all contain 1 orbital; The remaining keywords detemine how the orbital mixing is set up.
\begin{itemize}
\item asis : do excitation selection as the program used to do
\item h      : select the excitation corresponding to the largest h-matrix element
\item n      : do not have a specific selection
\item internal : specify the internal excitations explicitly. A  \textit{LIST} of pairs specifies the excitations

{\bf Example }
\begin{verbatim}
internal 2   2 3  3 4 4 2    2 4  3 2  4 3 end
\end{verbatim}
\end{itemize}

\item PRIMITIVE (virt)
Primitive equivalences (only one set); if virt is specified the virtuals are made equivalent.
An MO and for each MO a set of AO's is specified (like in HYBRID). So we have MO/AO's pairs.
You may use the same AOs in different combinations; The restriction that an AO may not
appear in different hybrids is also lifted. See there. Use of super hybrid is recommended.\
Syntax: 

\begin{verbatim}
equivalence 
 functions (mo's) end
 basisfunctions end
end
\end{verbatim}

{\bf Example }
\begin{verbatim}
? resonating BLW for H6
  hybrid  
  h1h2
   1 end   
   atom 1 2 end 
  h3h4
   2 end   
   atom 3 4 end 
  h5h6
   3 end   
   atom 5 6 end 
  h2h3
   4 end   
   atom 2 3 end 
  h4h5
   5 end   
   atom 4 5 end 
  h6h1
   6 end   
   atom 6 1 end 
  end
  equivalence virt
   1 end   
    1 2 3 4 end 
   2 end   
    5 6 7 8 end 
   3 end   
    9 10 11 12 end
   4 end   
    3 4 5 6 end 
   5 end   
    7 8 9 10 end
       6 end   
    11 12 1 2 end 
  end equivalence
  super hybrid orth
\end{verbatim}
\end{itemize}
 

\subsubsection{NOSYMM}

Do not use spatial symmetry. If CALC is specified, do not use
symmetry in the calculation of matrix elements

\subsubsection{NOSPIN}

Do not use symmetry due to spin in matrix element evaluation. 

\subsubsection{SUPER}

Do not allow internal excitations. These are replaced by excitations
to ao's. This may be very beneficial in cases, where the active
orbitals are nearly (or exactly) identical. When convergence is
troublesome in e.g. a Breathing Orbital \cite{ref:vb7} calculation
try SUPER.
In addition some super options/approaches may be specified.
Not all of these are well tested or understood. They are :
\begin{itemize}
\item HYBRID : determine mixing completely by HYBRID definition. One
can ask for the last IGNO AO's to be ignored, by giving IGNORE IGNO (A,I).
\item CAS    : allow no active-active mixing (CASSCF assumed)
\item ACTIVE : allow no excitations to virtual orbitals
\end{itemize} 

Attention, SUPER HYBRID has to appear after the HYBRIDS definition

\subsubsection{FORBI}

FORBID IPAIR1 IPAIR2 ... (a,I,I,I,I,...)

This directive forbids (usual internal) excitations at an orbital level.
It is meant for eliminations of undetected redundancies.

{\bf Example}
\begin{verbatim}
FORBID 1 2 1 3
\end{verbatim}

\subsubsection{FOCK}
Control Fock matrix generation; If made the Fock matrix is used, where applicable
\begin{itemize}
\item on (or yes) : compute Fock matrix (default)
\item off (or no)   : do not compute Fock matrix
\item diag : use a Fock type approximation for the diagonal
\item nodiag : calculate the diagonal exactly
\end{itemize}

\subsubsection{OPTIMISE}

OPTIMISE TYPE [IDELP IFIRSP SHIFTP]

This directive is used to specify the way orbitals are optimised. When this directive is not given
the orbitals are optimised using all possible Brillouin states, which is called Super CI. Building
the Hamiltonian in the Brillouin basis is the rate determining step in most VBSCF calculation. With this
directive it is possible to specify certain parts of the Brillouin matrix which will be approximated,
or not calculated at all, as determined by TYPE:
\begin{itemize}
\item BRIL  specify optimisation per Brillouin state with KEY LIST, as many times as needed,
 with KEY is 
\begin{itemize}
\item PERT : treat the Brillouin states with perturbation theory
\item VARIATIONAL : use SuperCI (default)
\item END   : end directive
\end{itemize}
\item ORBITAL : specify optimisation per orbital similar to BRIL
\item KIND : specify optimisation by kind of Brillouin state with
KEY FROM TO

KEY is
\begin{itemize}
\item PERT : treat the set  with perturbation theory
\item VARIATIONAL : use SuperCI (default)
\item AUTOMATIC : (recommended) switch automatically between variation and perturbation, based on the 
expected Brillouin-state coefficient from perturbation theory. One may specify a switching criterium
 at the end of the same line (default 0.0001)
\item END  :  end directive
\end{itemize}
FROM and TO is the kind of orbital we excite from and to
\begin{itemize}
\item DOC : doubly occupieds (not for TO)
\item VOC : variably occupied
\item UOC : empty (not for FROM)
\item ALL  : all of the above
\end{itemize}
\end{itemize}

{\bf Example 1}
\begin{verbatim}
OPTIMISE KIND
 PERT DOC UOC
 PERT DOC VOC
 PERT VOC VOC
 PERT VOC UOC
END OPTIMISE
\end{verbatim}

In this example the orbitals are grouped by their occupation in the wave function (KIND). 
Perturbation theory is used for excitations from doubly to unoccupied orbitals (PERT DOC UOC), for excitations
from doubly to variably occupied orbitals (PERT DOC VOC), for excitations from variably to variably
occupied orbitals (PERT VOC VOC) and for excitations from variably to unoccupied orbitals (PERT VOC UOC).
Note that the parameters IDELP IFIRSP and SHIFTP are optional. In this case perturbation theory is switched
on from the beginning.

{\bf Example 2}
\begin{verbatim}
OPTIMISE KIND
 AUTO ALL ALL 0.001
END OPTIMISE
\end{verbatim}

The auto option is used for all, starting from the beginning.


{\bf Example 3}
\begin{verbatim}
OPTIMISE BRILLOUIN -1 5 0
 PERT 4 5 10 23 END
END OPTIMISE
\end{verbatim}

In this example one can select certain Brillouin states. To use this option it is useful to know the way Brillouin
states are numbered. An earlier run of Turtle can provide this information. The perturbation option is switched
on after the DEL-value has dropped below -1, but not sooner than iteration 5 for Brillouin states 4, 5, 10 and 23.


\subsubsection{EVEC}


EVEC IVEC (A,I)

Optimize orbitals for excited state IVEC. 
If the state order changes during optimisation, you're out of luck

\subsubsection{FREEZE}
 
FREEZE ISTRING

Freeze (i.e. do not optimise) the orbitals in ISTRING.

\subsubsection{VIRTUAL} 

VIRTUAL specifies the way to determine the virtuals in the SCF procedure.
Options are :

\begin{itemize}
\item CANONICALISE : Canonicalise over the nuclear attraction matrix
\item LOCALISE     : Use localised virtuals ; using NIT or ITER a maximum
number of iterations may be specified.
\item IDEMPOTENT   : The  projection operator used is idempotent.
\item AOS          : Use AO's as virtuals.
\end{itemize}

If the word NOT is specified before an option, the option is negated.

subsubsection{CANONCALISE} 
Determine canonicalisation in (at beginning of) the VBSCF cycle
\begin{itemize}
\item off           : no canonicalisation
\item h             : use the 1-electron operator
\item fock        : use the fock operator
\item localise : localise the orbitals; If nit or iter is specified at the end of the line, the \# iterations may be given (A,I)	
\end{itemize}



\subsubsection{END}

END concludes the SCF-TURTLE input. An extra string 'SCF" is allowed.
In addition a section number to store the VBSCF orbitals may be specified

\clearpage

\begin{thebibliography}{10}

\bibitem{ref:vb1} J.H. van Lenthe and G.G. Balint-Kurti, Chem Phys.
Lett. {\bf 76}, 138 (1980)

\bibitem{ref:vb2}
J.H. van Lenthe and G.G. Balint-Kurti,
  J. Chem. Phys.  {\bf 78}, 5699 (1983), \doi{10.1063/1.445451}.

\bibitem{ref:vb3} J. Verbeek, Nonorthogonal orbitals in {\it ab initio}
many-electron wavefunctions, Ph.D. Thesis, Utrecht University (1990)

\bibitem{ref:vb4} C.P. Byrman, Nonorthogonal orbitals in Chemistry,
Ph.D. Thesis, Utrecht University (1995)

\bibitem{ref:vb5} F. Dijkstra, Valence Bond theory, implementation and
use of analytical gradients, Ph.D. Thesis, Utrecht University (2000)

\bibitem{ref:vb6} M.Raimondi, M. Simonetta and G.F. Tantardini, 
Ab initio Valence Bond Theory, Computer Physics Reports, Vol {\bf 2}, e 171 (1985) 

\bibitem{ref:vb7} P.C. Hiberty, S.Humbel, C.P. Byrman, J.H. van Lenthe,
J. Chem. Phys. {\bf 101}, 5969 (1994), \doi{10.1063/1.468459}.

\end{thebibliography}

\end{document}
