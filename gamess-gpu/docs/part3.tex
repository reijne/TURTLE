\documentclass[11pt,fleqn]{article}

\usepackage{hyperref}

% package HTML requires Latex2HTML to be installed for html.sty
\usepackage{html}
\newcommand{\doi}[1]{doi:\href{http://dx.doi.org/#1}{#1}}
\begin{htmlonly}
\renewcommand{\href}[2]{\htmladdnormallink{#2}{#1}}
\end{htmlonly}
\hypersetup{colorlinks,
            %citecolor=black,
            %filecolor=black,
            %linkcolor=black,
            %urlcolor=black,
            bookmarksopen=true,
            pdftex}

 
\addtolength{\textwidth}{1.0in}
\addtolength{\oddsidemargin}{-0.5in}
\addtolength{\topmargin}{-0.5in}
\addtolength{\textheight}{1.0in}
\newcommand{\water}{\mbox{H$_{2}$O}}
\newcommand{\waterp}{\mbox{H$_{2}$O$^{+}$}}
\newcommand{\degree}[1]{\mbox{$#1^{o}$}}
\newcommand{\dinfh}{\mbox{D$_{\infty h}$}}
\newcommand{\dtwoh}{\mbox{D$_{2h}$}}
\newcommand{\cinfv}{\mbox{C$_{\infty v}$}}
\newcommand{\ctwov}{\mbox{C$_{2v}$}}
\newcommand{\formaldehyde}{\mbox{H$_{2}$CO}}
\newcommand{\re}{\mbox{$r_{e}$}}
\newcommand{\silane}{\mbox{SiH$_{4}$}}

\pagestyle{headings}
\pagenumbering{roman}
\begin{document}
\sf
\parindent 0cm
\parskip 1ex
\begin{flushleft}
Computing for Science (CFS) Ltd.,\\CCLRC Daresbury Laboratory.\\[0.30in]
{\large Generalised Atomic and Molecular Electronic Structure System }\\[.2in]
\rule{150mm}{3mm}\\
\vspace{.2in}
{\huge G~A~M~E~S~S~-~U~K}\\[.3in]
{\huge USER'S GUIDE~~and}\\[.2in]
{\huge REFERENCE MANUAL}\\[0.2in]
{\huge Version 8.0~~~June 2008}\\ [.2in]
{\large PART 3. DATA INPUT - Pre-directives and CLASS 1 Directives}\\
\vspace{.1in}
{\large M.F. Guest, J. Kendrick, J.H. van Lenthe, P. Sherwood and
  J.M.H Thomas}\\[0.2in]

Copyright (c) 1993-2008 Computing for Science Ltd.\\[.1in]
This document may be freely reproduced provided that it is reproduced\\
unaltered and in its entirety.\\
\vspace{.2in}
\rule{150mm}{3mm}\\
\end{flushleft}
 
\tableofcontents

\pagenumbering{arabic}

\newpage

\section[Introduction]{Introduction}

In this and the following chapters we provide a fairly detailed
description of the role and format of the 
directives within the GAMESS--UK data specification responsible
for defining the molecular system under consideration and
the required computation. In addition to defining the system
geometry (through the ZMATRIX or GEOMETRY directive) and basis
set (the BASIS directive), the {\em Class 1} directives
provide control over file usage, printed output (IPRINT etc.)
and the integral generation routines (SUPER, ACCURACY etc.).

Before the discussion of the {\em Class 1} directives, we firstly
describe the ''pre-directives'' available within GAMESS-UK that should
appear before any {\em Class 1} directives. The pre-directives may be
required to set and/or modify the environment, defining for example
memory requirements or job cpu time, or assisting in file assignments.

The following points should be noted:
\begin{itemize}
\item With certain exceptions noted below, the {\em Class 1}
directives may be presented in any order.
\item All {\em Class 1} directives must appear in the input
stream before any {\em Class 2} directive (see Part 4).
\item Any pre-directives must appear before any {\em Class 1} directives
\end{itemize}

\section{The pre-directives}
The program is capable of processing a set of `pre-directives', each
such directive extending over one data line, and appearing as input
before the program specific data. These pre-directives allow the user
to define or modify, through data input, certain characteristics of
the job environment e.g., time allocations, routing of output, file
allocations, memory requirements etc.

\subsection{The MEMORY Pre-directive}
This pre-directive provides a mechanism for specifying the
dynamic core to be associated with the present run of the code.
Memory requirements are in general a function of the
RUNTYPE requested in the data input:  allocation can be 
modified through MEMORY data specification. The pre-directive
consists of a single data line, the first data field
being set to the character string memory or core, the second
to an integer defining the number of words of memory.
Presenting the data line

{
\footnotesize
\begin{verbatim}
          memory 8000000
\end{verbatim}
}
will yield an allocation of 64 MBytes. The default allocation of
4,000,000 words will prove adequate for most runs
involving both SCF and CI wavefunctions.

\subsection{The TIME Pre-directive}
GAMESS--UK monitors the CPU time available
at intervals, and if it is found that insufficient time remains to
usefully continue, will send restart control information
to the Dumpfile, and terminate execution. The time pre-directive is
used to specify the time limit for the job in CPU minutes, e.g.,

{
\footnotesize
\begin{verbatim}
          time 120
\end{verbatim}
}
will allocate 2 hours of CPU time to the job. In the absence
of the time pre-directive, a default allocation of 600 minutes
will be in effect.


\subsection{The FILE Pre-directive}

The FILE predirective may be used to override the default mode,
in which each of the direct access data sets used by GAMESS--UK
will be deleted on termination of the job. Such data sets are 
allocated to the program using logical file names (LFNs) in the 
range ED0, ED1, ..., ED19 and MT0, MT1, ..., MT19, and
retaining such a data set on job completion
requires the user to associate the LFN with a specified filename 
through the FILE directive. This is best shown through an
example. Assume that we wish to retain the Dumpfile, which in 
default is routed to ED3 (and is crucial to the program, controlling
all restart activities). This might involve the following
pre-directive specification:

{
\footnotesize
\begin{verbatim}
          FILE ED3 DUMPFILE
\end{verbatim}
}
in the startup job, and {\em all subsequent} jobs that make
use of this file. In similar fashion, retaining both the
Dumpfile and two-electron integral Mainfile between jobs 
might involve the following specifications:

{
\footnotesize
\begin{verbatim}
          FILE ED2 MAINFILE
          FILE ED3 DUMPFILE
\end{verbatim}
}
There is one standard use of the file directive that the
user need employ when accessing the various GAMESS--UK 
library data sets that reside in the directory:

{
\footnotesize
\begin{verbatim}
          GAMESS-UK/libs.
\end{verbatim}
}

When using the library of ECPs (with the specification
`pseudo nonlocal'), the file pre-directive

{
\footnotesize
\begin{verbatim}
         FILE ED0 ECPLIB
\end{verbatim}
}
must be specified when using RUN\_GAMESS.

In addition to the direct access data sets described above
GAMESS--UK will, in certain phases of computation e.g., Table-CI, 
make use of the more conventional FORTRAN unformatted sequential
data sets. In the majority of cases this usage is
restricted to scratch activities, and the user need not be 
concerned with, for example, FORTRAN unit specification. In some
instances, however, it may be necessary to `keep' the files
in question between jobs. This activity is again handled
through the FILE pre-directives.

\section{Class I directives}
The following sections describe the CLASS 1 directives available
within the programme.

\section[The DUMPFILE Directive]{The DUMPFILE Directive}

The DUMPFILE directive may be used to redefine Dumpfile output
from the default setting and consists of a single data line read to
variables TEXT, LFNAME, IBLOCK using format (2A,I)
\begin{itemize}
\item TEXT should be set to the character string DUMPFILE
\item LFNAME should be set to the LFN used to assign the
data set for output of the Dumpfile. The LFN can be one of
the files ED0 -- ED19, with the exception of the Scratchfile, ED7.
\item IBLOCK is the integer used to specify the starting block
at which Dumpfile output will commence.
\end{itemize}
{\bf Example }\\

Presenting the data line
{
\footnotesize
\begin{verbatim}
         DUMPFILE ED4 100
\end{verbatim}
}
would route Dumpfile output to ED4, commencing at block 100.
\vspace{0.10in}
The following points should be noted:
\begin{itemize}
\item The DUMPFILE directive may be omitted, when Dumpfile
information will be routed to block 1 of the data set assigned
using the LFN ED3. This default thus corresponds to 
presenting the data line

{
\footnotesize
\begin{verbatim}
         DUMPFILE ED3 1 
\end{verbatim}
}
\item If  used, the DUMPFILE directive {\em MUST} be the first
of the {\em Class 1} directives presented
in the data stream, immediately following 
any pre-directives (see Parts 12-16 of the manual).
\end{itemize}

\section[The RESTART Directive]{The RESTART Directive}

Execution and control of the program is 
classified into various tasks, as nominated by the  RUNTYPE directive
(see Part 4, section 2).
Once a calculation has been initiated, and
the files saved, all subsequent restart jobs {\em must}  specify the
RESTART directive. The directive has various roles;
\begin{enumerate}
\item  to indicate that the Dumpfile for the case in hand exists i.e., that
a previous calculation is to be continued;
\item  to restart the computation associated with a specific task;
\item  to abort the computation currently in progress, prior to
the definition of new activity under control of the RUNTYPE directive;
\item  when instigating a new task under RUNTYPE control, the RESTART
directive may be used to request that the input geometry be that
currently resident on the Dumpfile (ie from the previous completed
task) rather than that specified in the card input stream.
\end{enumerate}

In general the  format of the directive is dependent on the particular
role above. The directive consists of a single data line
read to the variables TEXT, TYPE, TREG using format (3A).
\begin{itemize}
\item TEXT should be set to the character string RESTART.
\item There are two general specifications for TYPE;
\begin{enumerate}
\item TYPE may be set to a character string 
identifying the task (RUNTYPE) to be
restarted. The following strings are recognised when restarting
some particular computation which had terminated successfully
in a prior job:
\begin{itemize}
\item  RESTART OPTIMIZE - continue a geometry optimization 
as initiated by a RUNTYPE OPTIMIZE directive.
\item  RESTART OPTXYZ - continue a geometry optimization 
as initiated by a RUNTYPE OPTXYZ directive.
\item RESTART SADDLE  - continue a saddle point optimisation, 
as initiated by a RUNTYPE SADDLE directive.
\item  RESTART SCF - continue an SCF, GVB, M\o ller Plesset,
CASSCF or MCSCF calculation, as initiated by a RUNTYPE SCF directive.
\item  RESTART FORCE - continue a numerical force constant 
calculation as initiated by a RUNTYPE FORCE directive.
\item  RESTART HESSIAN - continue an analytical force constant 
calculation as initiated by a RUNTYPE HESSIAN directive.
\item  RESTART POLARISABILITY - continue a polarisability
calculation as initiated by a RUNTYPE POLARISABILITY directive.
\item  RESTART MAGNET - continue a magnetisability
calculation as initiated by a RUNTYPE MAGNET directive.
\item  RESTART HYPER - continue a hyperpolarisability
calculation as initiated by a RUNTYPE HYPER directive.
\item  RESTART INFRARED - continue an Infra--red intensity
calculation as initiated by a RUNTYPE INFRARED directive.
\item  RESTART RAMAN - continue a Raman intensity
calculation as initiated by a RUNTYPE RAMAN directive.
\item  RESTART CI - continue  a CI calculation as initiated
by a RUNTYPE CI directive.
\item  RESTART GF or RESTART TDA  - continue  a Green's function
calculation as initiated by the corresponding  RUNTYPE directive.  
\item RESTART ANALYSE - while the analysis modules within GAMESS--UK
are not in general restartable, it is possible to restart the
computation of a grid of electrostatic potentials, as 
initiated by the RUNTYPE ANALYSE, GRAPHICS and GTYPE
directives (see Part 8).
\end{itemize}
\item TYPE may be set to the character string NEW or may be
left unspecified.  Let us assume that the computation 
associated with a given RUNTYPE has
been completed, and the user wishes to begin another phase of the
computation. This may be achieved by specifying the data line

{
\footnotesize
\begin{verbatim}
           RESTART NEW
\end{verbatim}
}
where the NEW keyword indicates that the molecular geometry for
the new calculation is to be taken from the data input stream.
An alternative procedure is available on presenting simply the
line

{
\footnotesize
\begin{verbatim}
           RESTART
\end{verbatim}
}
in which case the molecular geometry used will be that currently
resident on the Dumpfile.
\end{enumerate}
\item TREG may be set to the character string REGEN, in which case 
all integral files required to successfully continue the
computation in progress will be recomputed. This option is
provided to allow for instances when files have been  inadvertently
lost or corrupted. Thus presenting a data line of the form

{
\footnotesize
\begin{verbatim}
           RESTART OPTIMIZE REGEN
\end{verbatim}
}
to restart a geometry optimisation would, if the startup job
terminated in the course of the SCF iterations, lead to
the two-electron integral list being recomputed.
\end{itemize}
The following points on use of the RESTART directive should 
be noted:
\begin{itemize}
\item  Corruption or loss of the Dumpfile is
viewed at present as a terminal error condition.
\item In all cases except RESTART and RESTART NEW the 
character string specified
on both the RESTART and RUNTYPE directives must agree.
\item {\bf Note :- At present the restriction is imposed 
in RESTART mode 
under control of RUNTYPE OPTIMIZE or RUNTYPE SADDLE,
that use is made of the RESTART NEW specification}
ie it is not possible to read the optimised geometry for, say, one state
of the molecule from the Dumpfile in instigating a second
optimisation on, say, some other state.
\end{itemize}

\section[File specification: file identifiers, MAXBLOCK and MINBLOCK]{File specification: \protect \\file identifiers, MAXBLOCK and MINBLOCK}

In default, each of the direct access files associated with the
storage of intermediate quantities, such as two-electron integrals,
transformed integrals, symbolic formula etc. is associated with
a particular logical file name (LFN), in the range ED0--ED19. With
the exception of the Dumpfile, output to each file will commence
at block 1 and continue in uninterrupted fashion until the
entire list is complete. The file identifiers, and the MAXBLOCK and MINBLOCK
directives may be used to modify this default action by,
\begin{itemize}
\item changing the default LFN allocation;
\item fragmenting the output over multiple files;
\item commencing output to some specified position on a   given file.
\end{itemize}
\subsection[file identifiers]{file identifiers}
The file directives may be used to change the LFN associated
with a given output file, and to fragment the output to
multiple files. The directive consists of a single data line,
the first data field of which comprises the file identifier,
a character string identifying the particular output file
whose attributes are to be changed. Valid file identifiers are
shown in Table~\ref{table:1}. 

\begin{table}
 \caption{\label{table:1}\  Specification of the file identifiers and the default associated logical file names}
 
 \begin{centering}
 \begin{tabular}{lcr}
\\ \hline\hline
  File         &     File Identifier &     Default LFN\\ \cline{1-3}
\\
    Mainfile     &     MFILE or MAINFILE  &              ED2\\
    Symmetry Adapted Mainfile  &  AFILE & ED1\\
    Secondary Mainfile  &    SFILE or SECOND & ED4\\
    Direct-CI file &  CIFILE  & ED5\\
    Transformed Integral File &   FFILE or TFILE & ED6\\
    Loop Formulae Tape  &    LOOP &  ED9\\
    Reordered Formulae Tape &  RLOOP  &   ED10\\
    Density Matrix File & DMFILE or TWOPDM  & ED11\\
    Hamiltonian File & HFILE   & ED12\\
\hline\hline
 \end{tabular}
 
 \end{centering}
\end{table}

Subsequent data field(s) are read
in free A-format and are used to specify output reassignment by
nominating the direct access files involved. Thus
simply reassigning output to a different LFN is accomplished by a
data line of the form

{
\footnotesize
\begin{verbatim}
           MFILE ED4
\end{verbatim}
}
whereby Mainfile output is routed to the data set assigned with the LFN
ED4.  Fragmentation of file output may be achieved by multiple LFN
specification, so that splitting the Mainfile over three data sets,
assigned with the LFNs ED0, ED1 and ED2 would require the specification

{
\footnotesize
\begin{verbatim}
           MFILE ED0 ED1 ED2
\end{verbatim}
}
in conjunction with appropriate MINBLOCK and MAXBLOCK directives
defining the starting and terminal blocks of each data set (see below).

\subsection[MINBLOCK and MAXBLOCK]{MINBLOCK and MAXBLOCK}
These directives may be used to reassign the starting and terminating
blocks available for output on a given file, and are provided for
use with the FILE directive(s) in fragmenting output. The MINBLOCK
directive consists of a single data line read to the variables
TEXT, LFNAME, MINBLOK using format (2A,I),
\begin{itemize}
\item TEXT should be set to the character string MINBLOCK;
\item LFNAME should be set to the LFN of the file (ED0--ED19),
output to which is to be modified;
\item MINBLOK is an integer specifying the starting block
at which output is to commence.
\end{itemize}
Thus the data line
{
\footnotesize
\begin{verbatim}
          MINBLOCK ED2 1000
\end{verbatim}
}
would result in output to ED2 commencing at block 1000. The MAXBLOCK
directive is of similar syntax, being read to the variables
TEXT, LFNAME, MAXBLOK using format (2A,I).
\begin{itemize}
\item TEXT should be set to the character string MAXBLOCK;
\item LFNAME should be set to the LFN of the file (ED0--ED19)
output to which is to be modified;
\item MAXBLOK is an integer specifying the  block
at which output is to terminate.
\end{itemize}
Thus the data line
{
\footnotesize
\begin{verbatim}
          MAXBLOCK ED2 10000
\end{verbatim}
}
would result in output to ED2 terminating at block 10000.\\

{\bf Example}
{
\footnotesize
\begin{verbatim}
           MFILE ED0 ED1 ED2
           MAXBLOCK ED0 9000
           MAXBLOCK ED1 9000
           MINBLOCK ED2 2000
\end{verbatim}
}
The above data sequence
would result in Mainfile output being routed initially to ED0 (blocks
1-9000), then to ED1 (blocks 1-9000) and finally to ED2, commencing
at block 2000.

\subsection[In--core SCF Calculations]{In--core SCF Calculations}

In default, the two--electron integral list is written to the
data set on disk nominated by the MFILE directive (ED2 in default),
with the format controlled by SUPER specification. It is possible
to override this default, and avoid disk storage and the associated
I/O overheads, by mapping the integral list directly to memory. This
{\em modus operandi} is requested through a modification to the
MFILE directive, thus

{
\footnotesize
\begin{verbatim}
          MFILE MEMORY
\end{verbatim}
}
The following points should be noted;
\begin{enumerate}
\item The program employs a rather conservative algorithm in
deciding whether sufficient memory is available
to house the integral list prior to integral evaluation. If the
host machine does not have the available memory, the program aborts.
The estimated memory may not be sufficient, in which case one may override 
the default allocation in either blocks or in MBytes or GBytes.
For parallel runs, this allocation is 'per processor'.
{\bf Example}
{
\footnotesize
\begin{verbatim}
           MFILE MEMORY 500MB
\end{verbatim}
}
One may specify an overflow unit, in case the memory allocated is not sufficient.
This may be any valid file with the exception of ED2 (in memory), ED3 or ED7 (for scf).
Of course, one should be sure that this file is not in use in another stage of
the calculation for another purpose.
{\bf Example}
{
\footnotesize
\begin{verbatim}
           MFILE MEMORY 500MB ED12
\end{verbatim}
}
Now, if the memory (in this case 500 MByte) is exhausted the program switches to 
conventional integral storage, routing the remainder of the integral list to ED12.

\item This memory is independent of that specified under 
control of the CORE or MEMORY pre--directive (see Parts 12-16 of the manual),
and the latter specification should be made independently of the
present usage.

\item At present the memory resident integral list is not
written to disk on job termination, so that it is not
possible to RESTART in--core SCF computations.

\item This routing of integrals to memory is available for
all SCFTYPEs and integral formats, regardless of the RUNTYPE
in effect, and may thus be used, for example, in geometry 
optimisations.

\item In--core integral storage is targeted to those machines
with large amounts of central memory (ie 128~MBytes and above),
and is at present only available on those machines with
a UNIX operating system. It is useful on large parallel machines
with a lot of memory per processor, where the aggregated memory may amount
to hundreds of Gigabytes.
\end{enumerate}

\section[Directives Controlling Printed Output]{Directives Controlling Printed Output}

\subsection[NOPRINT]{NOPRINT}

A single data line containing the keyword NOPRINT in the
first data field will act to
suppress all intermediate output.
Greater selectivity in minimising the quantity of output
is achieved by appending  any combination
of the following parameters to the NOPRINT command:


\begin{tabular}{ll} 
\\ \hline 
HISTORY & Suppress printing of the history file during optimisation \\ 
VECTORS & Suppress printing of the eigenvectors. \\
BASIS   & Suppress printing of the basis set \\
ADAPT   & Suppress adaptation output \\
DISTANCE& Prevent the distance matrix from being printed \\
HESS    & Suppress printing of the force constant matrix during optimisation. \\
ANAL    & Suppress printing of the wavefunction analysis \\ \hline
\end{tabular}


\subsection[IPRINT]{IPRINT}
The IPRINT directive may be used to increase
the default program output, and consists of a single data line with
the character string IPRINT in the first data field. Subsequent
data fields may comprise any combination of the following parameters:

\begin{tabular}{ll}
\\ \hline
S       & requests printing of the overlap matrix in the a.o. basis\\
T       & requests printing of the kinetic energy matrix in the a.o. basis\\
T+V     & requests printing of the 1-e hamiltonian matrix in the a.o. basis\\
X       & requests printing of the x-dipole moments integrals\\
Y       & requests printing of the y-dipole moments integrals\\
Z       & requests printing of the z-dipole moments integrals\\
SCF     & Full SCF output during optimisation \\
OPTIMISE& Increased diagnostics during optimisation \\
VECTORS & Only print vectors upon completion of optimisation. \\
MULLIK  & Increased output from Mulliken analysis \\
LOWDIN  & Increased output from Lowdin wavefunction analysis \\
GUESS   & Requests printing of trial vectors \\
FOCK    & requests printing of the Fock matrix at SCF convergence \\
ADAPT   & printing of the symmetry adapted basis functions \\
DIIS    & monitoring of the solution of the DIIS equations  \\
SYMM    & increased symmetry diagnostics  \\
DIST    & requests printing of the matrix of internuclear distances \\
DSCF    & additional printing in the direct-SCF module \\
INERTIA & print moments of inertia etc. at optimised geometry \\
AOPR    & additional printing of 1-e property integrals in a.o. basis \\ 
MOPR    & additional printing of 1-e property integrals in m.o. basis \\ \hline
\end{tabular}

\subsection[FORMAT]{FORMAT}

The FORMAT directive may be used to modify the format for
printing eigenvectors and eigenvalues. The directive comprises a
single data line read to the variables TEXT, FORM using format (2A)
\begin{itemize}
\item TEXT should be set to the character string FORMAT
\item FORM may have the setting HIGH or LOW. In the former case
the vectors will be printed to 7 significant figures, with 
all orbitals (occupied and virtual) output. In the latter case
only the occupied and 5 lowest virtuals will be printed, 
and then to only 4 significant figures.
\end{itemize}
The default setting corresponds to presenting the data line
{
\footnotesize
\begin{verbatim}
          FORMAT LOW
\end{verbatim}
}

\section[Directives Controlling Integral Generation]{Directives Controlling Integral Generation}

\subsection[INTEGRAL]{INTEGRAL}

This directive may be used to specify the mode for 2-electron integral
and derivative integral evaluation. The program incorporates the
rotated axes and Gauss-Rys algorithms for 2-electron integral evaluation,
and the Schlegel algorithm \cite{schlegel}
for s,p derivative integrals.
In default the program will use the significantly faster s,p routines
whenever possible, handling only those integrals involving d- and f-
functions by the Gauss-Rys method.  This default mode may be 
overridden by the INTEGRAL directive, which has the form:

{
\footnotesize
\begin{verbatim}
           INTEGRAL HIGH 
\end{verbatim}
}
requesting   use of  the Gauss-Rys routines for all integral evaluation.


\subsection[ACCURACY]{ACCURACY}
This directive may be used to control the accuracy of the
integral routines, and  comprises a single data line
read to the variables TEXT, ITOL, ILOAD using format (A,2I).
\begin{itemize}
\item TEXT should be set to the character string ACCURACY.
\item ITOL is an integer used in computing a threshold according
to which integrals may be neglected. Specifically, integrals
involving products of primitives whose pre-exponential 
factor is less than 10$^{-ITOL}$ are skipped. Default value is
10$^{-20}$.
\item ILOAD is an integer used in computing the threshold for
loading two-electron integrals to the Mainfile. Specifically,
integrals less than 10$^{-ILOAD}$ are 
not loaded to the integral file. Default value is 10$^{-9}$.
\end{itemize}

\noindent
Experience suggests that the following setting,
{
\footnotesize
\begin{verbatim}
           ACCURACY 20 7
\end{verbatim}
}
can lead to a significant reduction in SCF cycle time without undue
loss of accuracy.


\subsection[SUPER]{SUPER}

This directive may be used to control the format and structure
of the two-electron integral file (Mainfile), and override the
default format imposed by the program. This default is in
general a function of both the SCFTYPE and RUNTYPE, and while
the program will, in default, impose the appropriate format associated
with the requested TYPEs, the user need be aware of these
when considering, for example, trying to re-use, rather than
regenerate, the Mainfile produced under control of a given
RUNTYPE/SCFTYPE combination in some subsequent calculation (under
control of the BYPASS directive). First, let us consider
the three possible Mainfile formats available, and how to
request each type; these are,
\begin{enumerate}
\item P-supermatrix format (2J-K), for use in closed--shell SCF 
calculations, where it is the default format;
\item J and K supermatrix format (in fact 2J-K and K), the default
in open--shell RHF, UHF and GVB calculations and usable in
closed--shell SCF calculations;
\item conventional 2--electron integral list format, which may
be used in all calculations, and as such is the most general.
\end{enumerate}
The SUPER directive allows the user to override the format
imposed by the program, and comprises a single data line with
the character string SUPER in the first data field. The integral format
required is controlled by the second keyword, with three possible 
settings;
\begin{itemize}
\item   {\bf  SUPER ON}: This will request supermatrix format (the 
default setting), with the format chosen (P or J + K)
dictated by the specified SCFTYPE i.e., P-matrix for closed-shell
SCF calculations, J + K matrices for GVB and open shell  calculations;
\item  {\bf SUPER OFF}: Produces conventional two-electron 
integral format.
\item  {\bf SUPER FORCE}: May be used to force J and K matrices
to be used, even in closed--shell calculations. This would typically
be introduced in a closed--shell SCF calculation, with the aim
of re-using the Mainfile in subsequent open shell calculations (under
control of the BYPASS directive).
\end{itemize}
Having nominated the integral format, we need next consider
whether the Mainfile is to comprise just the `skeletonised' list of
integrals (i.e only the symmetry distinct elements), or whether
the 'full' list is to be output. This choice may be controlled by the
third parameter of the SUPER directive; presenting
the optional keyword NOSYM will act to suppress the default 
skeletonisation, producing the complete list.\\

{\bf Examples}\\

To produce a complete list of 2--electron integrals requires
the data line
{
\footnotesize
\begin{verbatim}
           SUPER OFF NOSYM
\end{verbatim}
}
while generation of a skeletonised list in J~+~K supermatrix format
requires the data line
{
\footnotesize
\begin{verbatim}
           SUPER FORCE 
\end{verbatim}
}
Now let us consider the default formats imposed by the 
program (based on both computational efficiency and complexity). We
have already given in Table~1 of Part 2 an initial list of these,
together with the possible formats, as a function of SCFTYPE.
Note however that the defaults specified in the Table  are those
appropriate to RUNTYPE~SCF, and do not necessarily apply
for all RUNTYPEs; note also that Table~1 of Part 2 gives no 
consideration to the skeletonised nature of the list. We
summarise below the various formats in common use;
\begin{enumerate}
\item While for small and medium sized molecules P--supermatrix
usage is probably the most efficient, this may only be used for
closed--shell SCF calculations. Thus if we wish to perform
an initial closed--shell calculation, and then re-use the
Mainfile in some subsequent open--shell calculation, the
following data structure is typical of that required for the
closed shell SCF;

{
\footnotesize
\begin{verbatim}
           TITLE
           CLOSED SHELL SCF
           SUPER FORCE
           ZMATRIX
             .
             .
             .
           END
           ENTER
\end{verbatim}
}
with the open-shell data structure as follows;
{
\footnotesize
\begin{verbatim}
           RESTART
           TITLE
           OPEN-SHELL TRIPLET SCF
           SUPER FORCE
           BYPASS
           MULT 3
           ZMATRIX
             .
             .
             .
           END
           ENTER
\end{verbatim}
}
where the SUPER~FORCE data line in the closed-shell data
provides a usable Mainfile in the open--shell run, allowing BYPASS'ing
of integral generation.

\item Conventional integral list format, with no skeletonisation,
must be employed in all SCF calculations conducted under
control of the TRANSFORM, CI, GF and TDA RUNTYPEs, a requirement that
is also present in all CASSCF and MCSCF calculations. This corresponds
to presenting the data line

{
\footnotesize
\begin{verbatim}
           SUPER OFF NOSYM
\end{verbatim}
}
Note again that this format will be automatically employed in
any run with the above SCFTYPEs and RUNTYPEs in the data stream;
the user need only present the SUPER directive when changing these
defaults, in anticipation of some future use of the generated Mainfile.
Now let us consider performing the calculations above, followed by 
a CI calculation. The easiest way to proceed is to present the
data sequence

{
\footnotesize
\begin{verbatim}
           RESTART
           TITLE
           OPEN-SHELL TRIPLET DIRECT-CI
           MULT 3
           ZMATRIX
             .
             .
             .
           END
           RUNTYPE CI
           DIRECT
            .
            .
           ENTER
\end{verbatim}
}
when the integral list will be regenerated as part of the
RUNTYPE~CI processing in the appropriate format. However, in
order to avoid regenerating the integrals, we need to have
requested the appropriate Mainfile format from the outset,
with suitable changes to both the closed-- and open--shell
SCF data sets, thus;\\

{\bf Closed--shell SCF Data}
{
\footnotesize
\begin{verbatim}
           TITLE
           CLOSED SHELL SCF - SUPPRESS SKELETONISATION
           SUPER OFF NOSYM
           ZMATRIX
             .
             .
             .
           END
           ENTER
\end{verbatim}
}
{\bf Open--shell SCF Data}
{
\footnotesize
\begin{verbatim}
           RESTART
           TITLE
           OPEN-SHELL TRIPLET SCF -- NO SKELETONISATION
           SUPER OFF NOSYM
           BYPASS
           MULT 3
           ZMATRIX
             .
             .
             .
           END
           ENTER
\end{verbatim}
}

{\bf Open--shell Direct--CI Data}
{
\footnotesize
\begin{verbatim}
           RESTART
           TITLE
           OPEN-SHELL TRIPLET DIRECT-CI - BYPASS SCF
           MULT 3
           BYPASS SCF
           ZMATRIX
             .
             .
             .
           END
           RUNTYPE CI
           DIRECT
            .
            .
           ENTER
\end{verbatim}
}
\item Conventional integral format, with skeletonisation of
the integral list, must be employed in all SCF calculations
conducted under control of the HESSIAN, POLARISABILITY, HYPER,
MAGNET, RAMAN, INFRARED RUNTYPE's, and in all M\o ller Plesset
calculations, corresponding to the explicit specification

{
\footnotesize
\begin{verbatim}
           SUPER OFF
\end{verbatim}
}
\end{enumerate}
\begin{table}
 \caption{\label{table:2}\  Parameters of the BYPASS Directive}
 
 \begin{centering}
 \begin{tabular}{ll}
\\ \hline\hline
  PHASE        &    Excluded Computation \\ \cline{1-2}
\\
ONE & One-electron Integral Evaluation\\
TWO  & Two-electron Integral Evaluation \\
ADAPT   & Generation of Symmetry Adapted Integrals \\
SCF   & Integral Evaluation plus SCF Computation \\
TRAN & Integral Transformation \\
OLD    & See Text \\
HF    & SCF computation \\  \hline \hline
 \end{tabular}
 
 \end{centering}
\end{table}

\subsection[BYPASS]{BYPASS}

The BYPASS directive may be used to exclude various phases
of computation associated with a given RUNTYPE. For example,
when performing multiple SCFs at
a given nuclear geometry, it is clearly only necessary to 
evaluate the 1- and 2-electron list in the first SCF
computation. If performing multiple direct-CI calculations
under RUNTYPE CI control, it is again convenient to bypass
the integral transformation step.

The directive consists of a single data line, read to the
variables TEXT, PHASE using format (2A):
\begin{itemize}
\item TEXT should be set to the character string BYPASS;
\item PHASE should be set to a character string identifying
the particular phase of computation to be excluded from the
present run of the program. Valid character strings, together
with the associated computation are shown in
Table~\ref{table:2}).
\end{itemize}
The following points should be noted:
\begin{itemize} 
\item Simply presenting the data line BYPASS will
act to bypass both 1- and 2-electron integral
evaluation i.e., is equivalent to the data line
BYPASS ONE TWO.
\item  It is of course assumed that the integral files 
associated with the excluded steps are available 
from some previous run of the program. Thus when 
excluding integral generation, it is assumed that a Mainfile
compatible with the particular SCFTYPE requested has been created
by some previous job under control of the current Dumpfile. As such
the directive is usually presented only in RESTART mode.
\item An exception to this rule occurs in cases when the Dumpfile 
has been corrupted, but the 
user retains access to, for example,
the Mainfile. Clearly there is then the need  to associate the
two-electron integral file with the computation instigated with a 
`new' Dumpfile. Assuming the Mainfile is allocated to the
program with the default ED2 LFN, this may be achieved
through specifying the keyword OLD on the BYPASS directive.
\end{itemize}
{\bf Example}\\

We illustrate this point by considering an SCF
computation on \formaldehyde. Assuming the Dumpfile in Run I
below had not been correctly saved on termination of the job, then
Run II may recreate the Dumpfile, and use the Mainfile
from the first job (assuming of course that this had been correctly saved).\\

{\bf Run I}
{
\footnotesize
\begin{verbatim}
           TITLE
           H2CO - 3-21G DEFAULT BASIS - CLOSED SHELL SCF
           ZMATRIX ANGSTROM
           C
           O 1 1.203
           H 1 1.099 2 121.8
           H 1 1.099 2 121.8 3 180.0
           END
           ENTER
\end{verbatim}
}
{\bf Run II}
{
\footnotesize
\begin{verbatim}
           TITLE
           H2CO - 3-21G DEFAULT BASIS - CLOSED SHELL SCF
           BYPASS OLD
           ZMATRIX ANGSTROM
           C
           O 1 1.203
           H 1 1.099 2 121.8
           H 1 1.099 2 121.8 3 180.0
           END
           ENTER
\end{verbatim}
}

\subsection[ADAPT]{ADAPT}
In default the SCF, MCSCF and CI modules are performed
internally in a symmetry adapted basis set. This involves
the automatic generation of a symmetry adapted list of functions
driven from the input molecular geometry and basis specified, 
and subsequent characterisation of the molecular
orbitals in terms of the generated list.
The ADAPT directive consists of a single data line comprising
two data fields, the first of which should be set
to the character string ADAPT. The second data field may be used
to either,
\begin{itemize}
\item suppress symmetry adaptation by presenting the character
string OFF;
\item  specify the section number on the
Dumpfile where the adapted 1-electron integrals
are to be placed. If omitted the integrals are routed to
section 482. If specified, the section  must lie between 1 and 350.
\end{itemize}
{\bf Example 1}
{
\footnotesize
\begin{verbatim}
          ADAPT  200
\end{verbatim}
}
This example routes the symmetry adapted 1-electron integrals to
section 200 of the Dumpfile. 

{\bf Example 2}
{
\footnotesize
\begin{verbatim}
          ADAPT OFF
\end{verbatim}
}
This example suppresses symmetry adaptation.\\

{\bf Note:-}\\

There is one specific case where suppression of symmetry
adaptation will be required, and this is in SCF or GVB
calculations which use localised orbitals as the starting
point. During the localisation process, under control of
RUNTYPE~ANALYSE, adaptation is automatically switched off to
enable orbitals of different irreducible representation to
mix (although the total wavefunction remains, of course,
a unitary transformation of the SCF wavefunction). Assuming
these orbitals are to be used in a subsequent SCF or GVB
calculation, via the VECTORS directive, then the ADAPT~OFF
data line {\em must} be presented in any such job
that utilises the LMOs.

\section[Directives Defining the Molecular System]{Directives Defining the Molecular System}

\subsection[TITLE]{TITLE}

This directive may be used to  define an 80 character title for the 
run, and extends over two lines. The first line consists of the
character string TITLE in the first data field, the second 
line comprises the title.\\

{\bf Example}
{
\footnotesize
\begin{verbatim}
           TITLE
           H2CO - 3-21G CALCULATION
\end{verbatim}
}

\subsection[MULTIPLICITY]{MULTIPLICITY}

This directive defines the spin multiplicity of the system, and
consists of one line read to variables TEXT, MULT
using format (A,I).
\begin{itemize}
\item  TEXT should be set to the character string MULT.
\item  MULT is used to specify the spin degeneracy 
of the SCF wavefunction,
using the values 1,2,3 etc. for singlet, doublet, triplet states etc.
respectively.  It is also possible to use one of the character strings
SINGLET, DOUBLET, TRIPLET, QUARTET, QUINTET, SEXTET, SEPTET, OCTET and
NONET to specify MULT.
\end{itemize}
The MULT directive may be omitted, when the program will set MULT
to 1.\\

{\bf Example}
{
\footnotesize
\begin{verbatim}
          MULT 4

          MULT QUARTET
\end{verbatim}
}
\noindent
are equivalent; the wavefunction will be four-fold spin degenerate.

\subsection[CHARGE]{CHARGE}

This directive defines the net charge of the system, and
consists of one line read to variables TEXT, ICHA
using format (A,I).
\begin{itemize}
\item  TEXT should be set to the character string CHARGE.
\item  ICHA is used to specify the  net system charge
using the values 1,2,3 etc. for net single, double, triple  etc.
positive charges respectively.
\end{itemize}
The CHARGE directive may be omitted, when the program will set ICHA
to 0, the default thus referencing a neutral system.\\

{\bf Example}
\noindent
{
\footnotesize
\begin{verbatim}
          CHARGE 2
\end{verbatim}
}
\noindent
would be required, for example, in calculations on a di-cationic
species.

\subsection[ELECTRONS]{ELECTRONS}

This directive defines the number of electrons in the system, and
consists of one line read to variables TEXT, NELEC
using format (A,I).
\begin{itemize}
\item  TEXT should be set to the character string ELECTRONS.
\item  NELEC is used to specify the number of electrons required.
\end{itemize}

The ELECTRONS directive may be omitted, when the program will set NELEC
to the sum of the nuclear charges, adjusted by the charge if specified 
by the CHARGE directive. The ELECTRONS directive will be required whenever
additional nuclear charges are included in the system but should not
affect the electron count, for example those used to create a model potential.

The ELECTRONS and CHARGE directives are incompatible.

\subsection[ZMATRIX and GEOMETRY]{ZMATRIX and GEOMETRY}

Two directives may be used to define the molecular geometry,
ZMATRIX and GEOMETRY. The ZMATRIX directive plays a far more general
and versatile role in this specification, and may be used to,
\begin{itemize}
\item  define a system of internal coordinates required by the
geometry optimisation facilities in GAMESS--UK. This is achieved through
definition of the so-called VARIABLES. Initial values are attributed
to VARIABLES on the VARIABLE definition lines, and as such are used
to define an initial geometry;
\item  define the starting Hessian or force-constant matrix to
be used in geometry or transition-state optimisation;
\item control the point group symmetry employed in subsequent
integral evaluation.
\end{itemize}
The GEOMETRY directive may be used for direct input of cartesian
coordinates, but offers far less control over subsequent
optimisation studies.
Given the fundamental importance of both directives, 
we present a detailed description of the associated data input below.


\section[The ZMATRIX Directive]{The ZMATRIX Directive}


The first data line, the directive initiator, is read to variables
TEXTA, TEXTB using format 2A.

\begin{itemize}
\item  TEXTA should be set to the character string ZMATRIX or ZMAT

\item  TEXTB may be set to one of the strings A.U., AU, ANGSTROM or ANGS and
is used to define the units in which the relative positions of the
nuclei will be specified. If TEXTB is omitted
atomic units will be assumed in default.
\end{itemize}

The last line of the ZMATRIX directive, the directive terminator, consists
of the text END in the first data field. Lines appearing between the
directive initiator and terminator are of the following type :
\begin{itemize}
\item  Z-matrix definition lines, responsible for specifying the
relative positions of the nuclei. Most of these will be real
nuclei, although dummy nuclei may also be introduced to
assist in the geometry specification (see section 8.3)

\item  VARIABLE and CONSTANT definition lines.

\item  COORDINATE definition lines.

\item  CHARGES definition lines.
\end{itemize}
\subsection[Z-Matrix Definition Lines]{Z-Matrix Definition Lines}

 Each nucleus (including dummies) is numbered sequentially and specified
on a single data line, so that the (N+1)th line of the ZMATRIX directive
is used to specify the nature and location of the Nth nucleus in terms of
the positions of the (N-1) nuclei as determined by previous lines.
The specification of the Nth nucleus is contained 
in up to eight items, read to variables

{
\footnotesize
\begin{verbatim}
           TAGN, N1, R1, N2, ANG12, N3, ANG123, ITYPE
\end{verbatim}
}
using format (A,I,F,I,F,I,F,I)
\begin{itemize}
\item  TAGN is used to give the nucleus a name by which it will be subsequently
known, and to convey the chemical nature of the nucleus. Typically it
consists of just the chemical symbol, such as `H' or `SI' for
hydrogen or silicon. Alternatively it may be an alphanumeric
string, commencing with the chemical symbol, and followed
immediately by a secondary identifying integer to reflect, perhaps, a
different environment of the atoms involved e.g., H1 and H2.
TAGN may be up to 8 characters long, and must not
include the space character. We describe below the role of the TAGN
settings on the subsequent determination of molecular point group
symmetry.
\item  N1 is  an integer specifying a previously defined nucleus for
which the internuclear length R(N-N1) will be given.
\item  R1 is the internuclear length R(N-N1) in the appropriate
units.
\item  N2 is  an integer specifying a second nucleus, N2, different from
N1, for which the angle   (N,N1,N2) will be given.
\item  ANG12    is the value of   (N,N1,N2), the internuclear angle at N1
between N and N2, in degrees.

\item  N3 and ANG123  - the significance of N3 and ANG123 depends upon
the value assigned to ITYPE, the last item in the list. We may identify
the following two cases :

\begin{enumerate}
\item  ITYPE is omitted (or set to zero), then
\begin{itemize}
\item N3    is an integer specifying a nucleus for which the
internuclear dihedral angle   (N,N1,N2,N3) will be defined as ANG123.
\item ANG123        is the internuclear dihedral angle   (N,N1,N2,N3)
specified in degrees.
It is the angle between the planes (N,N1,N2) and (N1,N2,N3), whose
sign is governed by considering the movement of the vector $N1\Rightarrow N$
towards the vector $N2\Rightarrow N3$. The sign is positive if this
movement involves a righthand screw motion, and negative if this
movement involves a left hand motion.
\end{itemize}

\item  ITYPE is set to 1 or -1, then
\begin{itemize}
\item N3            specifies a nucleus for which a second internuclear
angle,  (N,N1,N3), will be defined by
\item ANG123        the value of the second internuclear angle,  (N,N1,N3)
in degrees.
\end{itemize}
\end{enumerate}
\end{itemize}
Given the above specification of the two internuclear angles, ANG12
and ANG123, we must differentiate between the two possible positions
arising for nucleus N: this is achieved through the sign specified
for ITYPE. If the triple vector product
\begin{equation}
          ( N1 \Rightarrow N) . ( N1 \Rightarrow N2  \times  N1 \Rightarrow N3)
\end{equation}
is positive, then ITYPE is set to 1: if the vector 
product is negative, ITYPE should be set to -1.

Having dealt with the general format of a z-matrix definition line
specifying the N'th nucleus, let us consider the specification of the
first three nuclei, which requires only a subset of the data required
in the general case.
\begin{enumerate}
\item For the {\em first}  nucleus, the definition line comprises
just a single data field, specifying the nucleus name and read to
variable TAG1 using format (A).
\item  For the {\em second}  nucleus specified, the definition line
comprises three data fields, read to variables TAG2, N1, R12 using
format (A,I,F), and is used to specify the internuclear length with
the first centre.
\item  For the {\em third}  nucleus specified, the definition line 
comprises five
data fields, read to variables TAG3, N1, R1, N2, ANG12 using format
(A,I,F,I,F) , and is used to specify an internuclear length and angle
with the two previously specified nuclei.
\end{enumerate}

Note that for consistency with data input of other codes, the Z-matrix 
connectivity may now also be defined using atom names directly. In this
case the specification of the Nth nucleus is still contained in up to 
eight items, read to variables:

{
\footnotesize
\begin{verbatim}
           TAGN, AN1, R1, AN2, ANG12, AN3, ANG123, ITYPE
\end{verbatim}
}
where AN1, AN2 and AN3 are now either integers or the names of previously 
defined centres.

Clearly, to avoid ambiguities, each centre must be defined using a unique name.
Failure to do so may result in the generation of an incorrect geometry:
GAMESS-UK will print out a warning when ambiguities are detected but it is up to
the user to ensure that the geometry is correctly defined. Bearing in mind the
discussion above detailing the use of the atom tags in defining symmetry, we
see that the use of unique atom names required in this form of the Z-matrix
definition lines will, in many cases, lead to a reduction in the 
molecular symmetry and should be avoided when defining the geometry of highly 
symmetric molecules.

Let us now consider a few examples to illustrate the specification so far\\

{\bf Example 1}\\

A simple C$_{2v}$ system, the water molecule, with r(O-H) =
0.952 {\AA} and angle HOH = \degree{104.5}.
{
\footnotesize
\begin{verbatim}
          ZMAT ANGS
          O
          H 1 0.951
          H 1 0.951 2 104.5
          END
\end{verbatim}
}
Note that it also possible to separate the individual data fields of
each Z-matrix definition line by a `comma', instead of a blank field,
so that the sequence above is equivalent to presenting

{
\footnotesize
\begin{verbatim}
          ZMAT ANGS
          O
          H,1,0.951
          H,1,0.951,2,104.5
          END
\end{verbatim}
}
The connectivity for the same structure could also be defined using atom
names directly, as follows

{
\footnotesize
\begin{verbatim}
          ZMAT ANGS
          O1
          H1 O1 0.951
          H2 O1 0.951 H1 104.5
          END
\end{verbatim}
}
However, in this case, the use of unique atom names will lead to lower
symmetry (C$_{s}$).\\

{\bf Example 2}\\

Specification of a simple D$_{3h}$ system, the CH$_{3}$
radical
(r(C-H) = 1.120 {\AA}), either 
\begin{itemize}
\item by definition of a dihedral angle
{
\footnotesize
\begin{verbatim}
          ZMAT ANGS
          C
          H 1 1.120
          H 1 1.120 2 120.0
          H 1 1.120 2 120.0 3 180.0
          END
\end{verbatim}
}
when the ITYPE parameter is omitted on the final specification line, or
\item by specification of a second internuclear angle, with the ITYPE
parameter specified, thus
{
\footnotesize
\begin{verbatim}
          ZMAT ANGS
          C
          H 1 1.120
          H 1 1.120 2 120.0
          H 1 1.120 2 120.0 3 120.0 1
          END
\end{verbatim}
}
\end{itemize}
{\bf Example 3}\\

Specification of a Td system, the CH$_{4}$ molecule  (C-H=1.083 {\AA}).
Again we may either make use of dihedral angle specification, as follows

{
\footnotesize
\begin{verbatim}
          ZMAT ANGS
          C
          H 1 1.083
          H 1 1.083 2 109.471
          H 1 1.083 2 109.471 3 120.0
          H 1 1.083 2 109.471 4 120.0
          END
\end{verbatim}
}
or the specification of two internuclear angles, thus
{
\footnotesize
\begin{verbatim}
          ZMAT ANGS
          C
          H 1 1.083
          H 1 1.083 2 109.471
          H 1 1.083 2 109.471 3 109.471  1
          H 1 1.083 2 109.471 3 109.471 -1
          END
\end{verbatim}
}
Note that it is not necessary to specify the tetrahedral angle to more
than 3 decimal places - the program will replace the value specified with
the full precision value. Again, we may use a `comma' for the field
separators on each data line, thus:
{
\footnotesize
\begin{verbatim}
          ZMAT ANGS
          C
          H,1,1.083
          H,1,1.083,2,109.471
          H,1,1.083,2,109.471,3,109.471,1
          H,1,1.083,2,109.471,3,109.471,-1
          END
\end{verbatim}
}

{\bf Example 4}\\

Specification of a system of D$_{4h}$  symmetry, planar CH$_{4}$ (
r(C-H) = 1.083 {\AA}).
{
\footnotesize
\begin{verbatim}
          ZMAT ANGS
          C
          H 1 1.083
          H 1 1.083 2 90.0
          H 1 1.083 3 90.0 2 180.0
          H 1 1.083 4 90.0 3 180.0
          END
\end{verbatim}
}
{\bf Example 5}\\

Specification of a C$_{3v}$ system, the methyl fluoride molecule,
CH$_{3}$F, with r(C-F)=1.384 {\AA}, r(C-H)=1.097 {\AA}, and angle HCF =
\degree{110.6}.

{
\footnotesize
\begin{verbatim}
          ZMAT ANGS
          C
          F 1 1.384
          H 1 1.097 2 110.6
          H 1 1.097 2 110.6 3 120.0
          H 1 1.097 2 110.6 3 -120.0
          END
\end{verbatim}
}
{\bf Example 6}\\

Specification of a D$_{6h}$ system, the benzene molecule
with r(C-C) = 1.387 {\AA}  and r(C-H) = 1.082 {\AA}.

{
\footnotesize
\begin{verbatim}
          ZMAT ANGS
          C
          H 1 1.082
          C 1 1.387 2 120.0
          H 3 1.082 1 120.0 2 0.0
          C 3 1.387 1 120.0 4 180.0
          H 5 1.082 3 120.0 4 0.0
          C 5 1.387 3 120.0 6 180.0
          H 7 1.082 5 120.0 6 0.0
          C 7 1.387 5 120.0 8 180.0
          H 9 1.082 7 120.0 8 0.0
          C 9 1.387 7 120.0 10 180.0
          H 11 1.082 9 120.0 10 0.0
          END
\end{verbatim}
}
\subsection[VARIABLES and CONSTANTS Specification]{VARIABLES and CONSTANTS Specification}

In the examples and specification to date we have assumed that the
relative positions of the nuclei are defined using actual values for
the bond lengths and bond angles i.e., for R1, ANG12 and ANG123 on the
z-matrix definition lines. An alternative specification
allows for these quantities to be replaced by symbolic names
on the definition lines, and to be subsequently assigned values on the
so-called VARIABLE and CONSTANT definition lines.

 While this mode of specification is optional in the context of a
single point calculation, it {\em must} be adopted in the framework of
geometry optimisations (under control of RUNTYPE OPTIMIZE)
and transition state location (under control of
RUNTYPE SADDLE). In such calculations
distances and angles specified as VARIABLES will be allowed
to vary throughout the optimisation, while those read as CONSTANTS
will remain fixed at the nominated value, as will those actually specified by
value in the z-matrix definition lines.

Note that optimisation studies require at least one variable to be
nominated in the z-matrix.
The revised format of a z-matrix definition line when using VARIABLES
and/or CONSTANTS to specify the relative positions of the nuclei is
as follows:

{
\footnotesize
\begin{verbatim}
          TAGN, N1, TEXT1, N2, TEXT2, N3, TEXT3, ITYPE
\end{verbatim}
}
using format (A,I,A,I,A,I,A,I).

The definition of TAGN, N1, N2, N3 and ITYPE are as specified previously
in section 8.
TEXT1, TEXT2 and TEXT3 are now alphanumeric strings: each is read in
A-format and may be up to 8 characters in length, but must not include the
space character.
\begin{itemize}
\item TEXT1 is symbolic name for the internuclear length R(N-N1).
\item TEXT2 is a symbolic name for the internuclear bond angle at N1
between N and N2.
\item TEXT3 is a symbolic name for either the internuclear dihedral angle
(N,N1,N2,N3) (when ITYPE is omitted), or for a second internuclear
angle (N,N1,N3) (when ITYPE is set to 1 or -1).
\end{itemize}
 Each of the symbolic names so defined must subsequently be assigned a
value on a CONSTANT or  VARIABLE definition line, to be presented after the
z-matrix definition lines, prior to the ZMATRIX terminator.

\subsubsection{VARIABLES Specification}
The specification of all
symbolic names to be assigned as VARIABLEs during optimisation
is initiated by presenting a data line containing the text VARIABLES
in the first data line. Subsequent data lines may be used to,
\begin{itemize}
\item assign an initial value to the variables,
one data line per name, which will subsequently be allowed
to vary during optimisation; 
\item assist in the specification of the initial Hessian
in geometry or transition state calculations.
In default mode the program provides an estimate of the diagonal
force constant matrix, based on a 
look-up table of bond-stretches, bending angle
etc. involving the component nuclei of the molecule.
While these defaults are, in most cases, perfectly adequate in
equilibrium geometry optimisation, they will not be in
transition state optimisations, and should  be overridden by
providing additional information on the corresponding 
VARIABLE definition lines. 
\end{itemize}
The first two data fields on each definition line are read to the
variables TEXT, VALUE using format (A,F).
\begin{itemize}
\item TEXT should be set to the character string of the 
symbolic variable
to be assigned as a VARIABLE, as specified in the z-matrix.
\item VALUE assigns an initial value to the symbolic  name TEXT.
\end{itemize}
Additional data fields are used in Hessian specification, with
two alternative formats possible. Assigning a value to the
diagonal element(s) of the Hessian corresponding to the variable
in question is achieved by presenting the character string
HESSIAN followed by the value to be assigned. Thus
in the \water\ examples above, the default value for the
O-H bond  variable may be replaced by the data line

{
\footnotesize
\begin{verbatim}
          OH 0.951 HESSIAN 1.1
\end{verbatim}
}
whereby the diagonal force constant for the OH variable is set to 1.1.
Note that the value specified is cumulative, and should be the
sum of terms arising from {\em all} occurrences of the variable. Had
the molecule in question been HOF, then the definition line might have
read

{
\footnotesize
\begin{verbatim}
          OH 0.951 HESSIAN 0.55
\end{verbatim}
}

The second specification requesting explicit computation of the Hessian
may be requested through specification of the TYPE keyword on the
VARIABLE definition lines. In such cases the corresponding part
of the Hessian will be evaluated numerically, prior to
commencing optimisation. Two settings are possible
\begin{itemize}
\item  TYPE 2 ; requests calculation of the diagonal force constant
and involves an additional energy calculation.
\item  TYPE 3 ; requests calculation of the diagonal force
constant and all off-diagonal elements involving the variable. This
requires an additional energy-plus-gradient calculation for
each variable nominated.
\end{itemize}


\subsubsection{CONSTANTS Specification}
If a ZMATRIX parameter is to be assigned a CONSTANT value during
optimisation, then 
a data line containing the text CONSTANTS 
in the first data field should be presented 
after the final VARIABLES definition line.
Subsequent data lines assign a value to each symbolic name, one data line
per name, which is to remain fixed throughout optimisation. Such a
data line is read to the variables TEXT, VALUE using format (A,F).
\begin{itemize}
\item  TEXT should be set to the character string of the symbolic name
specified in the z-matrix.
\item  VALUE assigns a CONSTANT value to the symbolic name TEXT.
\end{itemize}


We illustrate the use of constant and variable specification below, drawing upon
some of the examples given previously.\\

{\bf Example 1}~~ A rather trivial example is bond length optimisation 
in a diatomic molecule. A single point calculation at a nominated bond 
length would be
accomplished, in obvious fashion, by the following ZMATRIX

{
\footnotesize
\begin{verbatim}
          ZMAT ANGS
          F
          H 1 0.954
          END
\end{verbatim}
}
The corresponding data specifying the bond length as a VARIABLE is 
shown below, so that 
in a bond length optimisation the initial calculation would be
performed at r(H-F) = 0.954 {\AA}.

{
\footnotesize
\begin{verbatim}
          ZMAT ANGSTROM
          F
          H 1 HF
          VARIABLES
          HF 0.954
          END
\end{verbatim}
}
{\bf Example 2}~~The \water\ molecule, with the bond length and angle
represented by the symbolic names OH and HOH.

{
\footnotesize
\begin{verbatim}
          ZMAT ANGS
          O
          H 1 OH
          H 1 OH 2 HOH
          VARIABLES
          OH 0.951
          HOH 104.5
          END
\end{verbatim}
}
Presenting the above data in a geometry optimisation would lead to
optimisation of both the OH and HOH variables. The two ZMATRIX 
directives shown below
are equivalent, and would, in an optimisation run, result in variation
of the OH bond length  
with the HOH angle held fixed at \degree{104.5}.

{
\footnotesize
\begin{verbatim}
    ZMATRIX ANGS                              ZMATRIX ANGS
    O                                         O
    H 1 OH                                    H 1 OH
    H 1 OH 2 104.5                            H 1 OH 2 HOH
    VARIABLES                                 VARIABLES
    OH 0.951                                  OH 0.951
    END                                       CONSTANTS
                                              HOH 104.5
                                              END           
\end{verbatim}
}
{\bf Example 3}~~CH$_{3}$ using the dihedral angle specification.
{
\footnotesize
\begin{verbatim}
           ZMAT ANGS
           C
           H 1 CH
           H 1 CH 2 HCH
           H 1 CH 2 HCH 3 180.0
           VARIABLES
           CH 1.120
           HCH 120.0
           END
\end{verbatim}
}

\subsection[Specification and Role of DUMMY Centres]{Specification and Role of DUMMY Centres}

Up to this point we have confined our attention to the specification
of real nuclei within the z-matrix definition lines, with the
appropriate TAGs commencing with the chemical symbol of the atom in
question. It is frequently useful to introduce `dummy nuclei' to
assist in the geometry specification, but which are ignored subsequently
throughout the calculation. The z-matrix definition lines for such
DUMMY centres are identical in format to that specified in 3.8.1, with
the TAGs set either to the symbol `X' or to `-', the minus sign.
 The real value of z-matrix specification with dummy nuclei lies in
the ability to avoid possible numerical instability that may
arise in geometry optimisation studies. In particular, the
present optimisation techniques exhibit such instabilities when
dealing with internuclear angles close to \degree{180}. It is always possible
in such cases to locate a dummy atom on the bond angle bisector, and
thus remove the problem.

Note that it is {\em not} possible to site basis functions on dummy
centres; the so-called `BQ' centres are provided for this 
purpose (see section 8.4)\\

{\bf Example 1}: The water molecule
{
\footnotesize
\begin{verbatim}
           ZMAT ANGSTROM
           O
           X 1 1.0
           H 1 OH 2 90.0
           H 1 OH 2 90.0 3 HOH
           VARIABLES
           HOH 104.5
           OH 0.951
           END
\end{verbatim}
}
In this case the specification of the HOH angle is as a dihedral
angle, rather than as an internuclear angle as in Example 1 of section 8.1.
Note that the setting for the O-X length is arbitrary, and is set to 1.0
in this and all subsequent examples.\\

{\bf Example 2} The CH$_{3}$ radical.
{
\footnotesize
\begin{verbatim}
           ZMAT ANGSTROM
           C
           X 1 1.0
           H 1 1.120 2 90.0
           H 1 1.120 2 90.0 3 120.0
           H 1 1.120 2 90.0 3 -120.0
           END
\end{verbatim}
}
Again the internuclear HCH angle is specified as a dihedral angle
through the use of the dummy centre.\\

{\bf Example 3} The linear HCN molecule. 
{
\footnotesize
\begin{verbatim}
           ZMAT ANGSTROM
           C
           X 1 1.0
           N 1 CN 2 90.0
           H 1 CH 2 90.0 3 180.0
           VARIABLES
           CN 1.1506
           CH 1.0532
           END
\end{verbatim}
}
Note that optimisation
studies of linear species, or molecules containing potential
linear fragments should always be conducted with a z-matrix
of the  type shown above.
Let us consider an equivalent Z-MATRIX for the linear HNC isomer.
Following the above example, we would write

{
\footnotesize
\begin{verbatim}
           ZMAT ANGSTROM
           N
           X 1 1.0
           C 1 CN 2 90.0
           H 1 NH 2 90.0 3 180.0
           VARIABLES
           CN 1.170
           NH 1.011
           END
\end{verbatim}
}
This would be perfectly adequate for a calculation on this isomer. In
some cases, however, we wish to be able to write a single Z-MATRIX
connecting, as in this case, two minima on a potential surface to
provide a suitable framework for locating the transition state
connecting the 2 minima i.e., to have a single Z-MATRIX capable of
describing all three points through the specification of common
VARIABLES. We shall return to this concept
in more detail in Part~4  when describing the
synchronous transit data requirements, but note here that
the required z-matrix would be specified thus

{
\footnotesize
\begin{verbatim}
           ZMAT ANGS
           C
           X 1 1.0
           N 1 CN 2 90.0
           H 1 CH 2 90.0 3 PHI
\end{verbatim}
}
where the following variables would be specified for the 2 minima
{
\footnotesize
\begin{verbatim}
                    HCN                 HNC
    -------------------------------------------------------
    CN            1.151                 1.170
    CH            1.053                 2.281 (ie CN + NH)
    PHI           180.0                 0.0
\end{verbatim}
}
while a suitable starting point for the transition state
location would be derived from the `average' geometry, thus

{
\footnotesize
\begin{verbatim}
           ZMAT ANGSTROM
           C
           X 1 1.0
           N 1 CN 2 90.0
           H 1 CH 2 90.0 3 PHI
           VARIABLES
           CN 1.160
           CH 1.617
           PHI 90.0
           END
\end{verbatim}
}

\subsection[Specification of Ghost Centres]{Specification of Ghost Centres}
In the discussion to date, TAG specification on the z-matrix
definition lines has been used to convey the chemical nature
of the nucleus. An additional requirement here is the
ability to specify ghost centres i.e., centres with no electrons, but
with basis functions, for use in either,
\begin{itemize}
\item basis set superposition estimates (Counterpoise calculations);
\item specification of bond-centred functions for incorporating
either polarisation effects, or for introducing Rydberg character
through the addition of diffuse functions.
\end{itemize}
Such centres may be introduced in the ZMATRIX using the characters
BQ as the first two characters of the TAG. We illustrate this
specification below, showing also the citing of basis functions
on such centres.\\

{\bf Example}
{
\footnotesize
\begin{verbatim}
          TITLE
          HCN  DUNNING DZ + BOND(S,P) 
          ZMAT ANGSTROM
          C
          BQ 1 RCN2
          X 2 1.0 1 90.0
          N 2 RCN2 3 90.0 1 180.0
          X 1 1.0 2 90.0 3 0.0
          H 1 RCH 5 90.0 4 180.0
          VARIABLES
          RCN2 0.580
          RCH 1.056
          END
          BASIS
          DZ H
          S BQ
          1.0 1.0
          P BQ
          1.0 0.7
          DZ C
          DZ N
          END
          RUNTYPE OPTIMIZE
          ENTER
\end{verbatim}
}

For counterpoise calculations the ghosts may be specified by denoting
the nuclei involved in a separate GHOST directive. They will retain their 
basis sets, but their charge will be set to 0.0 and the corresponding 
electrons will be deleted. e.g.
\begin{verbatim}
          GHOST H1 O1 END
\end{verbatim}
A * may serve as wild character, so specifying *1 will make all nuclei
whose name ends with 1 into ghosts, as in the following example 

{
\footnotesize
\begin{verbatim}
          TITLE 
          BSSE TEST
          GHOST *1 END
          GEOMETRY
           0.00000000  -1.10092542 -1.43475395  1.0 H1  
           0.00000000  -1.10092542  1.43475395  1.0 H1  
           0.00000000   0.00000000  0.00000000  8.0 O1  
           3.24201636   2.02583666  0.00000000  1.0 H2  
           4.24693920   4.71362490  0.00000000  1.0 H2  
           4.77568401   2.98417857  0.00000000  8.0 O2  
          END 
          BASIS SV 4-31G 
          ENTER
\end{verbatim}
}

\subsection[Specification of Point Charges]{Specification of Point Charges}

As we have seen above, centres specified within the Z-matrix
definition lines fall into three categories;
\begin{itemize}
\item atomic centres, with the element type characterised by the
first, and when necessary, the second character of the TAG;
\item dummy centres, characterised by the X specification;
\item ghost centres, characterised by the BQ specification.
\end{itemize}
There remains the need to be able to specify point charges
within the ZMATRIX framework; this capability is provided
through the COORDINATES and CHARGES sub-directives. 
The first data line of the COORDINATES, the sub-directive initiator, 
contains the character string COORDINATES in the first data field.
Subsequent data lines are the `point charge definition'
lines, each line defining a given  centre, and is read to
variables TAG, X, Y, Z, and CHARGE using format (A,4F).
\begin{itemize}
\item  TAG is used to give the centre a name by which it will be
subsequently known. TAG may be up to 8 characters long,
and should not include the `space' character. Note that the TAG name
{\em must} commence with the two characters `BQ'.
\item  X, Y, Z are the cartesian co-ordinates of the given centre, in
the appropriate units.
\item  CHARGE is the charge of a given nucleus, the units being such
that the charge of the proton is unity. Negative, zero
and fractional charges are allowed, as well as more usual
positive integer values. 
\end{itemize}
If specified, the COORDINATE definition lines must be presented
{\em after} any CONSTANT and VARIABLES specification.

Alternatively, one may reassign the charges to previously defined 
centre's using the CHARGES sub-directive. The first data line of this
subsection contains the character string CHARGES.
Subsequent lines are the charge definition lines, each line
defining the charge for a centre and is read to variables TAG CHARGE
using format (A,F).
\begin{itemize}
\item  TAG should be a previously defined centre name in either zmatrix
(internal or cartesian) or coordinates.
Note that TAG usually commences with the two characters `BQ'.
\item  CHARGE is the charge which will be assigned to all
centre's with that name, overwriting previously defined values.
The same conditions as for the charge in coordinates and cartesian apply.
\end{itemize}
If specified, the CHARGES definition lines must be presented {\em last}.

It is important to realise that by default the number of electrons are
estimated from the total overall charge (as specified by the CHARGE
directive) and the sum of the atomic charges. If the sum of atom
charges, including the `BQ' centres, is not integral, or not equal to
the sum of the overall charge plus the required number of electrons,
then the number of electrons must be specified by the ELECTRONS
directive and the CHARGE directive omitted.  

When more than one point charge is included, the computed nuclear
energy will exclude the interaction between the point charges. If this
term is required, use the class 1 directive BQBQ.\\

{\bf Example}\\

In the data file below the COORDINATE data lines are being used
to place a \water\ molecule above a  grid of charged ions.

{
\footnotesize
\begin{verbatim}
          TITLE
          H2O -- C2V -- SURROUNDING POINT CHARGES
          ZMAT ANGSTROM
          X
          O 1 D
          H 2 OH 1 HOX
          H 2 OH 1 HOX 3 180.0
          VARIABLES
          OH 0.956   HESSIAN 1.4
          HOX 133.05  HESSIAN 0.5
          CONSTANTS
          D 3.0
          COORDINATES
          BQ 0.0 0.0 0.0 2.0
          BQ 0.0 -2.106 0.0 -2.0
          BQ 0.0 2.106 0.0 -2.0
          BQ 2.106 0.0 0.0 -2.0
          BQ -2.106 0.0 0.0 -2.0
          BQ 0.0 0.0 -2.106 -2.0
          BQ 2.106 2.106 0.0 2.0
          BQ -2.106 -2.106 0.0 2.0
          BQ 2.106 -2.106 0.0 2.0
          BQ -2.106 2.106 0.0 2.0
          END
          RUNTYPE OPTIMIZE
          ENTER
\end{verbatim}
}

{\bf Example}\\

{
\footnotesize
\begin{verbatim}
          TITLE
          CO ON SMALL CRYSTAL
          ZMATRIX ANGSTROM
          C
          O 1 r
          CARTESIANS
          BQ1  -3.0 0.0 0.0 
          BQ1   +3.0 0.0 0.0 
          BQ2   0.0 +3.0 0.0 
          BQ2   0.0 -3.0 0.0 
          VARIABLES
          R 1.08
          CHARGES 
          BQ1  0.1
          BQ2 -0.1
          END
          ENTER 1
\end{verbatim}
}

\subsection[Examples of Z-matrix input]{Examples of Z-matrix input}
\begin{enumerate}
\item Triatomic C$_{2v}$
\begin{itemize}
\item utilising a dummy:
{
\footnotesize
\begin{verbatim}
          ZMAT ANGSTROM
          O
          X 1 1.0
          H 1 R 2 90.0
          H 1 R 2 90.0 3 THETA
          VARIABLES
          THETA 104.5
          R 0.951
          END
\end{verbatim}
}
\item no dummy, no variables
{
\footnotesize
\begin{verbatim}
          ZMAT ANGSTROM
          O
          H 1 0.951
          H 1 0.951 2 104.5
          END
\end{verbatim}
}
\end{itemize}
 \item Polyatomic C$_{2v}$ - Fe(CO)$_{2}(NO)_{2}$
{
\footnotesize
\begin{verbatim}
          ZMAT ANGSTROM
          FE
          X 1 1.0
          C 1 FEC 2 CFEX
          C 1 FEC 2 CFEX 3 180.0
          X 3 1.0 1 90.0 2 180.0
          O 3 CO 5 90.0 1 180.0
          X 4 1.0 1 90.0 2 180.0
          O 4 CO 7 90.0 1 180.0
          X 1 1.0 2 90.0 4 180.0
          X 1 1.0 9 90.0 2 180.0
          N 1 FEN 10 NFEX 9 90.0
          X 11 1.0 1 90.0 10 180.0
          O 11 NO 12 90.0 1 180.0
          N 1 FEN 10 NFEX 9 -90.0
          X 14 1.0 1 90.0 10 180.0
          O 14 NO 15 90.0 1 180.0
          VARIABLES
          FEC 1.84
          CFEX 53.829
          CO 1.15
          FEN 1.77
          NFEX 55.922
          NO 1.12
          END
\end{verbatim}
}
\item Tetra-atomic H$_{3}$O C$_{3v}$, in terms of the out-of-plane
bending angle
{
\footnotesize
\begin{verbatim}
          ZMAT ANGSTROM
          O
          X 1 1.0
          H 1 R 2 ANG
          H 1 R 2 ANG 3 120.0
          H 1 R 2 ANG 3 -120.0
          VARIABLES
          ANG 70.0 HESSIAN 0.6
          R 0.964 HESSIAN 1.5
          END
\end{verbatim}
}
\item Tetra-atomic PH$_{3}$ C$_{3v}$, in terms of the bond
angles
{
\footnotesize
\begin{verbatim}
          ZMAT 
          P
          H 1 RPH
          H 1 RPH 2 THETA
          H 1 RPH 2 THETA 3 THETA  1
          VARIABLES
          RPH 2.685   HESSIAN 0.7
          THETA 93.83  HESSIAN 0.2
          END
\end{verbatim}
}
\item Tetra-atomic D$_{3h}$
{
\footnotesize
\begin{verbatim}
          ZMAT ANGSTROM
          O
          X 1 1.0
          H 1 R 2 90.0
          H 1 R 2 90.0 3 120.0
          H 1 R 2 90.0 3 -120.0
          VARIABLES
          R 0.964 HESSIAN 1.5
          END
\end{verbatim}
}
\item NH$_{4}$  Td
{
\footnotesize
\begin{verbatim}
          ZMAT ANGSTROM
          N
          H 1 R
          H 1 R 2 109.471
          H 1 R 2 109.471 3 120.0
          H 1 R 2 109.471 4 120.0
          VARIABLES
          R 1.16 HESSIAN 2.0
          END
\end{verbatim}
}
\item Cr(NO)$_{4}$  T$_{d}$
{
\footnotesize
\begin{verbatim}
          ZMAT ANGSTROM
          CR
          N 1 CRN
          N 1 CRN 2 109.471
          N 1 CRN 2 109.471 3 120.0
          N 1 CRN 2 109.471 4 120.0
          X 2 1.0 1 90.0 3 180.0
          O 2 NO 6 90.0 1 180.0
          X 3 1.0 1 90.0 2 180.0
          O 3 NO 8 90.0 1 180.0
          X 4 1.0 1 90.0 5 180.0
          O 4 NO 10 90.0 1 180.0
          X 5 1.0 1 90.0 4 180.0
          O 5 NO 12 90.0 1 180.0
          VARIABLES
          CRN 1.79
          NO 1.16
          END
\end{verbatim}
}
\item NH$_{4}$ D$_{4h}$
{
\footnotesize
\begin{verbatim}
          ZMAT ANGSTROM
          N
          H 1 R
          H 1 R 2 90.0
          H 1 R 3 90.0 2 THETA
          H 1 R 4 90.0 3 THETA
          VARIABLES
          R 1.16 HESSIAN 2.0
          THETA 180.0
          END
\end{verbatim}
}
\item NH$_{4}$ D$_{2h}$
{
\footnotesize
\begin{verbatim}
          ZMAT ANGSTROM
          N
          H 1 R
          H 1 S 2 90.0
          H 1 R 3 90.0 2 180.0
          H 1 R 4 90.0 3 180.0
          VARIABLES
          R 1.339 HESSIAN 1.0
          S 0.991 HESSIAN 1.0
          END
\end{verbatim}
}
\item NbCl$_{5}$ D$_{3h}$
{
\footnotesize
\begin{verbatim}
          ZMAT ANGSTROM
          NB
          CL  1  REQ
          X  2  1.0    1  90
          CL  1  REQ   2  120  3 180
          CL  1  REQ   2  120  3   0
          CL  1  RAX   2   90  3  90
          CL  1  RAX   2   90  3 -90
          CONSTANTS
          REQ 2.338
          RAX 2.362
          END
\end{verbatim}
}
\item Ni(PH$_{3}$)$_{2}$ D$_{3d}$
{
\footnotesize
\begin{verbatim}
          ZMAT ANGSTROM
          X
          NI   1  1.0
          P    2  NIP     1   PNIX
          P    2  NIP     1   PNIX     3 180.0
          H    3  PH      2   HPNI     1 180.0
          H    3  PH      2   HPNI     1 -60.0
          H    3  PH      2   HPNI     1  60.0
          H    4  PH      2   HPNI     1   0.
          H    4  PH      2   HPNI     1 120.
          H    4  PH      2   HPNI     1 240.
          VARIABLES
          NIP  2.1801684 HESSIAN 2.178 
          PNIX 90.0 HESSIAN 0.1
          HPNI 120.6751816 HESSIAN 1.824
          PH 1.4353437 HESSIAN 1.787
          END
\end{verbatim}
}
\item Fe(CO)$_{5}$ D$_{3h}$
{
\footnotesize
\begin{verbatim}
          ZMAT ANGSTROM
          FE
          C  1  RCEQ
          X  2  1.00   1   90
          O  2  RCO    3   90  1 180
          C  1  RCEQ   2  120  3 180
          X  5  1.00   1   90  2 180
          O  5  RCO    6   90  1 180
          C  1  RCEQ   2  120  3   0
          X  8  1.00   1   90  2 180
          O  8  RCO    9   90  1 180
          C  1  RCAX   2   90  3  90
          X 11  1.00   1   90  2 180
          O 11  RCO   12   90  1 180
          C  1  RCAX   2   90  3 -90
          X 14  1.00   1   90  2 180
          O 14  RCO   15   90  1 180
          CONSTANTS
          RCEQ 1.8273000
          RCAX 1.8068000
          RCO  1.1520
          END
\end{verbatim}
}
\item MoF$_{6}$ O$_{h}$
{
\footnotesize
\begin{verbatim}
          ZMAT ANGSTROM
          MO
          F 1 MOF
          F 1 MOF 2 90.0
          F 1 MOF 2 90.0 3 90.0
          F 1 MOF 2 90.0 3 180.0
          F 1 MOF 2 90.0 3 -90.0
          F 1 MOF 3 90.0 2 180.0
          VARIABLES
          MOF 1.814
          END
\end{verbatim}
}
\end{enumerate}

\subsection[Controlling the Point Group Symmetry]{Controlling the Point Group Symmetry}

In some instances the user need consider lowering the point group
determined in default by the program, particularly in the
case of degenerate point groups, which for some SCFTYPEs and
RUNTYPEs must be a subset of the D$_{2h}$ group. Specifically
the appearance of the message

{
\footnotesize
\begin{verbatim}
          *****************************************************
          * The molecular point group prohibits use of either *
          * the requested SCFTYPE or RUNTYPE. Reduce the      *
          * molecular symmetry by modifying the nuclear TAGs  *
          *     (see chapter 2 of the User Manual)            *
          *****************************************************
\end{verbatim}
}
requires remedial action, involving a simple modification
of the TAGs used on the ZMATRIX definition lines.
The symmetry handling routines within GAMESS--UK assume that 
{\em any centres with differing TAGs are  not related by symmetry}.
The user may thus control the point group actually adopted in the
calculation though appropriate TAG specification. This is
demonstrated below , where we consider the  examples of section 8.6, and
through TAG modification lower the point group symmetry to a 
subset of D$_{2h}$.
\begin{enumerate}
\item Tetra-atomic PH$_{3}$, may be reduced from  C$_{3v}$ symmetry to
C$_{s}$ by modifying the first hydrogenic TAG, thus

{
\footnotesize
\begin{verbatim}
          ZMAT 
          P
          H1 1 RPH
          H  1 RPH 2 THETA
          H  1 RPH 2 THETA 3 THETA  1
          VARIABLES
          RPH 2.685   HESSIAN 0.7
          THETA 93.83  HESSIAN 0.2
          END
\end{verbatim}
}
\item Tetra-atomic OH$_{3}$ may be reduced from D$_{3h}$ symmetry
to C$_{2v}$ by again modifying the first hydrogen, thus

{
\footnotesize
\begin{verbatim}
          ZMAT ANGSTROM
          O
          X  1 1.0
          H1 1 R 2 90.0
          H  1 R 2 90.0 3 120.0
          H  1 R 2 90.0 3 -120.0
          VARIABLES
          R 0.964 HESSIAN 1.5
          END
\end{verbatim}
}
\item NH$_{4}$  Td; Note that simply changing the first hydrogenic
TAG will not produce the desired effect, as this will still lead to
a degenerate C$_{3v}$ point group. Changes to the first two H TAGs
will lead to C$_{2v}$ symmetry, thus;

{
\footnotesize
\begin{verbatim}
          ZMAT ANGSTROM
          N
          H1 1 R
          H1 1 R 2 109.471
          H  1 R 2 109.471 3 120.0
          H  1 R 2 109.471 4 120.0
          VARIABLES
          R 1.16 HESSIAN 2.0
          END
\end{verbatim}
}
\item NH$_{4}$ D$_{4h}$; Changing the first hydrogenic TAG will yield a
C$_{2v}$ point group, while changes to the first and third  TAG
will yield a D$_{2h}$ point group, the preferred option, thus;

{
\footnotesize
\begin{verbatim}
          ZMAT ANGSTROM
          N
          H1 1 R
          H  1 R 2 90.0
          H1 1 R 3 90.0 2 THETA
          H  1 R 4 90.0 3 THETA
          VARIABLES
          R 1.16 HESSIAN 2.0
          THETA 180.0
          END
\end{verbatim}
}
\item NbCl$_{5}$ D$_{3h}$; changing the first equatorial
chlorine TAG will yield a C$_{2v}$ point group, thus;

{
\footnotesize
\begin{verbatim}
          ZMAT ANGSTROM
          NB
          CL1 1  REQ
          X  2  1.0    1  90
          CL  1  REQ   2  120  3 180
          CL  1  REQ   2  120  3   0
          CL  1  RAX   2   90  3  90
          CL  1  RAX   2   90  3 -90
          CONSTANTS
          REQ 2.338
          RAX 2.362
          END
\end{verbatim}
}
\item Ni(PH$_{3}$)$_{2}$ D$_{3d}$; The following TAG modification
will lower the symmetry to C$_{2h}$;
{
\footnotesize
\begin{verbatim}
          ZMAT ANGSTROM
          X
          NI   1  1.0
          P    2  NIP     1   PNIX
          P    2  NIP     1   PNIX     3 180.0
          H1   3  PH      2   HPNI     1 180.0
          H    3  PH      2   HPNI     1 -60.0
          H    3  PH      2   HPNI     1  60.0
          H1   4  PH      2   HPNI     1   0.
          H    4  PH      2   HPNI     1 120.
          H    4  PH      2   HPNI     1 240.
          VARIABLES
          NIP  2.1801684 HESSIAN 2.178 
          PNIX 90.0 HESSIAN 0.1
          HPNI 120.6751816 HESSIAN 1.824
          PH 1.4353437 HESSIAN 1.787
          END
\end{verbatim}
}
\item MoF$_{6}$ O$_{h}$; Simply changing the first two fluorine TAGs
will produce a C$_{2v}$ group; a more effective lowering may be
achieved through introducing three fluorine TAGS, F, F1 and F2, which
can lead to a D$_{2h}$ lowering, thus

{
\footnotesize
\begin{verbatim}
          ZMAT ANGSTROM
          MO
          F  1 MOF
          F1 1 MOF 2 90.0
          F2 1 MOF 2 90.0 3 90.0
          F1 1 MOF 2 90.0 3 180.0
          F2 1 MOF 2 90.0 3 -90.0
          F  1 MOF 3 90.0 2 180.0
          VARIABLES
          MOF 1.814
          END
\end{verbatim}
}
\end{enumerate}

\section[The GEOMETRY Directive]{The GEOMETRY Directive}

It is  possible to define the molecular geometry through
simple specification of the coordinates, charge and `tag' of
the component atoms under control of the GEOMETRY directive.
This mode of operation has been significantly extended in the
present release of the  program, so that it is now possible to
either:
\begin{itemize}
\item perform a geometry optimisation in the
cartesian space of the molecule under control of the
RUNTYPE OPTXYZ specification. Note that automatic symmetry handling
is now incorporated in GEOMETRY usage.
\item generate the z-matrix and associated variables from the
input list of cartesian coordinates, hence permitting use
of the standard internal-coordinate based optimisation 
techniques within the program.
\end{itemize}
Let us first consider the format of the GEOMETRY directive when
used simply as a mechanism for co-ordinate specification.
The first data line, the directive initiator, 
is read to variables TEXTA, TEXTB using format (2A).
\begin{itemize}
\item  TEXTA should be set to the string GEOMETRY.
\item  TEXTB may be set to one of the strings AU (or A.U.) or ANGSTROM,
and is used to define the units in which the Cartesian
co-ordinates will be specified. If TEXTB is omitted, the
units will be assumed to be in atomic units.
\end{itemize}
The last line of the GEOMETRY directive, the directive terminator,
consists of the text END in the first data field. Lines appearing
between the initiator and terminator are the `nucleus definition'
lines. Each line defines a given nuclear centre, and is read to
variables X, Y, Z, CHARGE and TAG using format (4F,A).
\begin{itemize}
\item  X, Y, Z are the Cartesian co-ordinates of the given centre, in
the appropriate units.
\item  CHARGE is the charge of a given nucleus, the units being such
that the charge of the proton is unity. Negative, zero
and fractional charges are allowed, as well as more usual
positive integer values. 
\item  TAG is used to give the centre a name by which it will be
subsequently known. TAG may be up to 8 characters long,
and should not include the `space' character. Note that the TAG name
is again used internally within the program to allocate, for example,
basis functions to the centre in question. The user should follow
certain conventions, as in the ZMATRIX directive, when specifying
TAG. Typically  the first, and where appropriate the second, character
should be used to specify the chemical symbol of the centre. Subsequent
characters may be used to provide more specific labelling
information e.g., H1, SI2 etc. (see however the Examples below  on the
impact of TAG specification on subsequent symmetry handling
by the program).  
\end{itemize}
{\bf Example 1}
{
\footnotesize
\begin{verbatim}
          GEOMETRY
           0.0000000      0.0000000      0.0000000  14.0 SI
          -1.6165808      1.6165808     -1.6165808  1.0 H
           1.6165808     -1.6165808     -1.6165808  1.0 H
          -1.6165808     -1.6165808      1.6165808  1.0 H
           1.6165808      1.6165808      1.6165808  1.0 H
          END
\end{verbatim}
}
{\bf Example 2}
{
\footnotesize
\begin{verbatim}
          GEOMETRY
           0.0000000      0.0000000      0.0000000  14.0 SI
          -1.6165808      1.6165808     -1.6165808  1.0 H1
           1.6165808     -1.6165808     -1.6165808  1.0 H2
          -1.6165808     -1.6165808      1.6165808  1.0 H3
           1.6165808      1.6165808      1.6165808  1.0 H4
          END
\end{verbatim}
}
The difference between these two examples depicting the
GEOMETRY data for \silane\ is the TAG
parameters. Thus the hydrogen centres which were user simply TAGged H
in example 1, are TAGged H1,H2,H3 and H4 respectively by the user
in example 2; this will have a drastic effect on the symmetry 
handling routines, which assume that {\em any centres with differing
TAGs are  not related by symmetry}! Thus in the first example the
calculation will proceed in T$_{d}$ symmetry while in the second
C$_{1}$ i.e., (no symmetry) will be recognised, with a corresponding
increase in integral evaluation and SCF time. 
The user may control the point group actually adopted in the
calculation though appropriate TAG specification. This is
demonstrated in the data below, which will lead to
the calculation being performed in C$_{2v}$ symmetry.

{
\footnotesize
\begin{verbatim}
          GEOMETRY
           0.0000000      0.0000000      0.0000000  14.0 SI
          -1.6165808      1.6165808     -1.6165808  1.0 H1
           1.6165808     -1.6165808     -1.6165808  1.0 H2
          -1.6165808     -1.6165808      1.6165808  1.0 H1
           1.6165808      1.6165808      1.6165808  1.0 H2
          END
\end{verbatim}
}

\subsection[Cartesian-based Optimisation]{Cartesian-based Optimisation}

We show below three examples of simple GEOMETRY specification, with
single-point energy evaluation in Example 1, geometry optimisation
requested under control of the OPTXYZ specification in Example 2, and
the "freezing" of atomic positions when performing geometry optimisation
under OPTXYZ control in examples 3.  \\

{\bf Example 1}
{
\footnotesize
\begin{verbatim}
         TITLE
         H2O - EXPLICIT GEOMETRY SPECIFICATION - TZVP BASIS
         GEOMETRY
         0.0 0.0 -0.2212037 8.0 O
         0.0  1.4284429 0.884815 1.0 H
         0.0 -1.4284429 0.884815 1.0 H
         END
         BASIS TZVP
         ENTER
\end{verbatim}
}

Note that in this case the coordinates are specified in
atomic units: the data line GEOMETRY~ANGSTROM would be used
for specification in angstrom.\\


{\bf Example 2}
{
\footnotesize
\begin{verbatim}
          TITLE
          H2CO GEOMETRY TEST
          GEOMETRY  
           0.0000000      0.0000000      0.9998722  6 C
           0.0000000      0.0000000     -1.2734689  8 O
           0.0000000      1.7650653      2.0942591  1 H
           0.0000000     -1.7650653      2.0942591  1 H
          END
          BASIS  STO3G
          RUNTYPE OPTXYZ
          ENTER 
\end{verbatim}
}

{\bf Example 3}
{
\footnotesize
\begin{verbatim}
          TITLE
          H2CO GEOMETRY TEST
          GEOMETRY  
           0.0000000      0.0000000      0.9998722  6 C
           0.0000000      0.0000000     -1.2734689  8 O
           0.0000000      1.7650653      2.0942591  1 H NOOPT
           0.0000000     -1.7650653      2.0942591  1 H NOOPT
          END
          BASIS  STO3G
          RUNTYPE OPTXYZ
          ENTER 
\end{verbatim}
}

A further variation of the `nucleus definition' lines may be used to
request the freezing of one or more atoms when performing geometry
optimisation in cartesian space. In such cases appending the character
string NOOPT after the centre TAG will act to freeze the centre in
question throughout; thus in the example above the H atoms will remain
fixed throughout the geometry optimisation.


\subsection[Z-Matrix Construction via GEOMETRY]{Z-Matrix Construction via GEOMETRY}

Finally, it is possible to generate a z-matrix from the list
of supplied x-, y- and z-coordinates under control of the GEOMETRY
directive. This will then permit usage of the other optimisation
procedures, in addition to OPTXYZ e.g., transition-state location, which
rely on the definition of internal co-ordinate variables inherent
in the z-matrix specification. This mode of operation is requested
by the specification of additional keywords on the first data line
of the GEOMETRY directive, with the remainder of the GEOMETRY data as
defined above.
The  directive initiator, 
is now read to variables TEXTA, TEXTB, TEXTC, TEXTD and TEXTE and TEXTF SCALE
using format (5A,A,F).
\begin{itemize}
\item  TEXTA should be set to the string GEOMETRY.
\item  TEXTB may be set to one of the strings AU (or A.U.) or ANGSTROM,
and is used to define the units in which the Cartesian
co-ordinates will be specified. If TEXTB is omitted, the
units will be assumed to be in atomic units.
\item  TEXTC may be set to the character string BOND, when all
internuclear distances generated in the z-matrix  will be 
assigned a variable name.
\item  TEXTD may be set to the character string ANGLE, when all
bond angles generated in the z-matrix  will be 
assigned a variable name.
\item  TEXTE may be set to the character string TORSION, when all
dihedral angles generated in the z-matrix  will be 
assigned a variable name.
\end{itemize}
Thus presenting the following data line as the GEOMETRY directive
initiator,
{
\footnotesize
\begin{verbatim}
          GEOMETRY ANGSTROM BOND ANGLE TORSION
\end{verbatim}
}
will assign variable status to all internal co-ordinates of the
generated z-matrix. The following points should be noted
\begin{itemize}
\item Z-matrix construction is activated by the presence of
any one of the three additional keywords, BOND, ANGLE or TORSION
\item An abbreviated form of the data line requesting all three
keywords is possible by specifying the character string ALL. Thus
the data line above may be presented as

{
\footnotesize
\begin{verbatim}
          GEOMETRY ANGSTROM ALL
\end{verbatim}
}
\item The classification of the internal coordinates as BOND,
ANGLE and TORSION provides a rather limited  mechanism for constraining
any  geometry optimisation. Only those variables
specifically requested through GEOMETRY will be treated
as variables in subsequent optimisation.
\item Sometimes the algorithm to assign bonds to specific atom-pairs may 
fail. Changing the criterion using for TEXTF SCALE e.g SCALE 0.9 may help.
\end{itemize}
{\bf Example 1}
{
\footnotesize
\begin{verbatim}
           TITLE\H2O DZ GEOMETRY SPECIFICATION
           GEOMETRY ALL
           0.0    0.0    -0.2212037 8      O
           0.0     1.4284429     0.8848150 1      H
           0.0    -1.4284429     0.8848150 1      H
           END
           BASIS DZ
           RUNTYPE OPTIMIZE
           ENTER
\end{verbatim}
}
{\bf Example 2}
{
\footnotesize
\begin{verbatim}
           TITLE\H2CO DZ GEOMETRY SPECIFICATION
           GEOMETRY  BOND
           0.0000000      0.0000000      0.9998722   6.0      C
           0.0000000      0.0000000     -1.2734689   8.0      O
           0.0000000      1.7650653      2.0942591   1.0      H
           0.0000000     -1.7650653      2.0942591   1.0      H
           END
           RUNTYPE OPTIMIZE\ENTER
\end{verbatim}
}
{\bf Example 3}
{
\footnotesize
\begin{verbatim}
           TITLE
           AZO-THIOPHENE DERIVATIVE
           ACCURACY 20 7
           NOPRINT VECTORS
           GEOMETRY  ANGSTROM BOND
           -8.879100   -0.405200   -0.722100   8       O
           -9.410100    1.638600   -1.156700   8       O
           -3.144600    2.045000   -1.672400   7       N
           -2.310300    1.077200   -1.875100   7       N
            3.034600    2.388300   -2.473700   7       N
           -8.598200    0.738400   -1.028100   7       N
           -4.461900    1.616700   -1.528300   6       C
           -4.925400    0.303600   -1.471300   6       C
           -6.290900    0.011800   -1.326500   6       C
           -7.176900    1.043200   -1.201600   6       C
           -6.781200    2.356000   -1.242700   6       C
           -5.434700    2.625100   -1.402300   6       C
           -1.007400    1.465800   -2.038600   6       C
           -0.563700    2.785400   -1.972400   6       C
            0.753100    3.102500   -2.153400   6       C
            1.719200    2.084000   -2.368300   6       C
            1.266300    0.755700   -2.467100   6       C
           -0.063300    0.431200   -2.279100   6       C
            4.083900    1.366900   -2.686200   6       C
            4.427300    1.253500   -4.104900   6       C
            3.526400    3.769300   -2.331300   6       C
            3.791900    4.129700   -0.898100   6       C
           -6.572200   -0.941100   -1.297800   1       H
           -7.415400    3.100900   -1.057700   1       H
            1.020200    3.997000   -2.073400   1       H
            1.832900    0.038400   -2.734900   1       H
            3.831500    0.343600   -2.299600   1       H
            4.959500    1.538200   -2.165800   1       H
            3.587800    1.020300   -4.637800   1       H
            5.259500    0.581700   -4.119400   1       H
            4.495300    2.043500   -4.561900   1       H
            2.933100    4.391300   -2.839600   1       H
            4.355900    3.794400   -2.887700   1       H
            2.914400    4.032800   -0.339900   1       H
            4.605000    3.512700   -0.603100   1       H
            4.221000    4.898400   -0.947400   1       H
           -5.005500    4.009600   -1.441700   6       C
           -4.662100    5.117100   -1.473200   7       N
           -3.974500   -0.787200   -1.563900   6       C
           -3.213800   -1.659800   -1.637900   7       N
           -1.289600    3.580200   -1.885500   1       H
           -0.512600   -1.020000   -2.328900   6       C
            0.348000   -1.660300   -2.522400   1       H
           -1.245900   -1.146100   -3.125400   1       H
           -0.962100    -1.294100   -1.374400  1       H
           END
           BASIS STO3G
           RUNTYPE OPTIMIZE
           THRESH 6
           XTOL 0.005
           ENTER
\end{verbatim}
}

\subsection[Mixed Z-matrix and Cartesian Input]{Mixed Z-matrix and Cartesian Input}

Structures may also be input using a combination of cartesian coordinates 
and z-matrices using the ZMATRIX directive. The sub-directive, 
CARTESIANS, is used to define a set of cartesian atoms. Each line of the
CARTESIANS directive comprises four data fields, with the atom name
followed by the three cartesian coordinates. The latter may be input
as either constants or variables (which may be optimised). 
If a fifth data-field is provided it's contents is taken to be the
charge of the center, like in the COORDINATES section. The sub-directive 
INTERNALS terminates cartesian input and allows the user to define the 
remaining atoms using standard z-matrix notation. The example below 
illustrates how an initial structure for ethylene might be defined
using mixed cartesian/zmatrix input.

{
\footnotesize
\begin{verbatim}
           TITLE
           ETHYLENE - MIXED Z-MATRIX/CARTESIAN INPUT
           ZMATRIX ANGSTROM
           CARTESIANS
           C   0.000    0.000    0.000
           C   0.000    0.000    CC
           H   WIDTH    0.000   -DEPTH
           H  -WIDTH    0.000   -DEPTH
           INTERNALS
           H  2  CH  1  CCH  3  TWIST
           H  2  CH  1  CCH  5  180.0
           VARIABLES
           CC 1.4
           CH 1.0
           WIDTH 0.8
           DEPTH 0.5
           CCH 120.0
           TWIST 10.0
           END
           RUNTYPE OPTIMISE
           ENTER
\end{verbatim}
}
Note the following restrictions; if the first atom in mixed z-matrix 
input/cartesian mode is a cartesian, then each coordinate (x,y,z) 
must be input as a constant. If the second atom is also specified as cartesian, 
then the x and y coordinates must also be constants. Similarly, the y
coordinate of the third atom must be a constant. Specifying any of these
coordinates as a variable will lead to an error.

Structures can also be input as alternating combinations of z-matrices and
coordinates, e.g.
{
\footnotesize
\begin{verbatim}
           TITLE
           ETHENE
           ZMATRIX ANGSTROM
           CARTESIANS
           C   0.000    0.000    0.000
           C   0.000    0.000    CC
           H   WIDTH    0.000   -DEPTH
           INTERNALS
           H   2  CH  1  CCH  3  TWIST
           CARTESIANS
           H  -WIDTH    0.000   -DEPTH
           INTERNALS
           H   2  CH  1  CCH  4  180.0
           VARIABLES
           CC 1.4
           CH 1.0
           WIDTH 0.8
           DEPTH 0.5
           CCH 120.0
           TWIST 10.0
           END
           RUNTYPE OPTIMISE
           ENTER
\end{verbatim}
}


\section[The WEIGHTS Directive]{The WEIGHTS Directive}

A number of modules (runtypes HESSIAN, FORCE, INFRARED and 1-electron 
properties under RUNTYPE ANALYSE) are able to output results
for a number of isotopic substitution patterns.  The masses used
for these calculations are defined using the WEIGHTS directive.

The user should note that the atomic mass tables are reset to their
default values at the start of each RUNTYPE, so the following data
should be presented as part of the data for the RUNTYPE that will use
it.  By default, a single vector of atomic masses is defined,
containing the mass of the most abundant isotope of the element.  This
mass vector may be modified using the VECTOR keyword within the WEIGHTS
data block, as follows:

{
\footnotesize
\begin{verbatim}
          TITLE
          H2O2 MODIFIED MASSES FOR PROPERTIES PACKAGE
          ZMAT ANGS
          O
          X  1  1.
          H  1  OH    2  90.0
          O  1  OO    3  OOH   2  0.0
          H  4  OH    1  OOH   2  HOOH
          VARI
           OH             1.0010897
           OO             1.3962308
           OOH          101.1195323
           HOOH         124.9861586
          END
          BASIS STO3G
          RUNTYPE SCF
          ENTER
          RUNTYPE ANALYSE
          WEIGHTS
          VECTOR 16 1 17 2
          END
          PROP
          6 5
          END
          VECTORS 1
          ENTER
\end{verbatim}
}
will modify the isotopic substitution pattern of the system. One
integer for each atom is required, corresponding to the number of
nucleons in the required nucleus. The ordering of masses provided
should correspond to the input order of the atoms.

The specification:
{
\footnotesize
\begin{verbatim}
          WEIGHTS
          VECTOR CREATE M1 M2 M3 ...
          END
\end{verbatim}
}
is similar, but will create an additional vector of atomic masses,
thus causing an extra  set of frequencies and thermochemical data 
(in the case of RUNTYPE INFRARED) to be output.

Using a value of  0 in an isotope vector is shorthand for the
most abundant isotope of the given element, and -1 can be used 
to request the average of natural abundances.
If the integers are omitted, natural abundance for all atoms is assumed.

Alternatively, a single element of the atomic mass vector 
may be substituted using the SUBSTITUTE keyword:

{
\footnotesize
\begin{verbatim}
          WEIGHTS
          SUBST 2 ISOTOPE 2
          END
\end{verbatim}
}
If atom 2 is an H atom, this will result on the mass of deuterium
being assigned.  For the (rarer) cases where an arbitrary mass is to
be used, the form

{
\footnotesize
\begin{verbatim}
          WEIGHTS
          SUBST 2 MASS 1.989
          END
\end{verbatim}
}

can be adopted.  An alternative form of the substitute keyword allows
the substitution to be performed for all centres with labels matching a
specific tag:

{
\footnotesize
\begin{verbatim}
          WEIGHTS
          SUBST C1 ISOTOPE 13
          END
\end{verbatim}
}

If the VECTOR CREATE and SUBSTITUTE keywords appear in the same
input, substitute will act on the {\em most recently created} vector of
masses. However, input of an explicit mass (rather than an integral
isotope specification) is only possible for the first vector. These
directives may be used to request a series of substitution patterns in
a concise manner, by labelling the symmetry inequivalent atoms to be
substituted. The following example will compute the frequencies for
non-deuterated chloroethene and the 3 mono-deuterated forms:

{
\footnotesize
\begin{verbatim}
          TITLE
          C2H3F - DEUTERATED FREQUENCIES
          INTEG HIGH
          ZMAT  ANGSTROM
          C
          C   1    CC
          F   1    CF   2   FCC
          H1  1    CH1  2   CCH1   3   180.0
          H2  2    CH2  1   CCH2   3     0.0
          H3  2    CH3  1   CCH3   3   180.0
          VARI
          CC   1.34
          CF  1.6
          CH1  1.0
          CH2  1.0
          CH3  1.0
          FCC  120.0
          CCH1  120.0
          CCH2  120.0
          CCH3  120.0
          END
          BASIS  6-31G
          SCFTYPE RHF
          RUNTYPE OPTIMISE
          XTOL   0.00001
          ENTER
          RUNTYPE INFRARED
          WEIGHTS
          VECT CREATE
          SUBST H1 ISOTOPE 2
          VECT CREATE
          SUBST H2 ISOTOPE 2
          VECT CREATE
          SUBST H3 ISOTOPE 2
          END
          ENTER
\end{verbatim}
}

The internal data tables contain average masses and  the abundant 
isotopes for z $<$54. More comprehensive tables are included for
z $<$ 18. If other isotopes are needed it is possible to augment the
internal tables, the isotope keyword may be used

{
\footnotesize
\begin{verbatim}
          ISOTOPE AG 108.904756
\end{verbatim}
}

will define 109-Ag.  If the nearest integer to the mass entered 
corresponds to an isotope stored within the program, the new mass will
replace the stored one.

It is also possible to input the masses for each atom in the geometry block, 
as follows:
{
\footnotesize
\begin{verbatim}
          TITLE
          H2O2  EXPLICIT MASSES WITH GEOMETRY
          GEOMETRY
            .7541095    -1.0813168     -.2371497   8 O 16.999131
           -.7541095     1.0813168     -.2371497   8 O  
           -.2202484    -2.2328936      .9081618   1 H  2.0140
            .2202484     2.2328936      .9081618   1 H  
          END
          BASIS STO3G
          RUNTYPE INFRARED
          ENTER
\end{verbatim}
}

In this case, only a single set of masses may be provided, and the
user cannot make use of the internal table of nuclide masses.

\section[Basis Set Specification]{Basis Set Specification}

The program incorporates a variety of `built-in' basis sets which
may be invoked through the BASIS directive. Prior to outlining the
variety of available basis sets, specified under control of BASIS,
we consider the role of the HARMONIC directive in requesting usage
of spherical-harmonic angular functions and the DEPENDENCY  directive
which allows control over projecting out nearly dependent parts of the basis.

\subsection[The HARMONIC Directive]{The HARMONIC Directive}

The default Cartesian angular functions (6~d, 10~f, 15~g) used throughout
GAMESS-UK may be overridden under control of the HARMONIC directive.
This provides the option of using spherical-harmonic (5~d, 7~f, 9~g)
angular functions.  Note that such usage is implemented internally
through appropriate transformations, and not by computing integrals or
derivative integrals over the spherical functions.

The HARMONIC directive comprises a single data line
read to variables TEXTA, TEXTB, TEXTC using format (3A).
\begin{itemize}
\item  TEXTA should be set to the string HARMONIC
\item  TEXTB may be set to one of the strings ON or OFF to 
activate or de-activate the use of spherical harmonic functions.
\item TEXT may be set to the character string PRINT to provide
additional diagnostic information.
\end{itemize}
Typical usage will involve just presenting the string HARMONIC.
The following points should be noted:
\begin{itemize}
\item The HARMONIC directive, if present, should appear before 
the BASIS directive.
\item A primary use of spherical functions is to help to 
eliminate problems with linear dependence.
\item It is not possible in the present release of the code to
employ the HARMONIC option in Table-CI calculations.
\item The correlation-consistent basis sets together with a number
of the ECP families of basis sets (see below) were 
designed using spherical harmonics and to use these the HARMONIC 
directive should be present. 
\end{itemize}

\subsection[The DEPENDENCY Directive]{The DEPENDENCY Directive}

When linear dependency occur in the basis, as generally signified by 
ridiculous energies and wildly oscillating convergence behaviour, it may 
be necessary to project out the dependent parts of the basis set. The 
HARMONIC directive should have been tried first.
The overlap matrix is diagonalised and any vector with a too small
eigenvalue is removed from the calculation.  This is implemented 
internally through appropriate transformations.
By default all vectors with eigenvalues smaller then 10$^{-7}$ are
eliminated and warnings are printed if eigenvalues fall below 10$^{-5}$.
The DEPENDENCY directive allows one to change this behaviour.

The directive comprises a single data line starting with
the keyword DEPENDENCY followed by one or more of three subdirectives.
\begin{itemize}
\item  CRIT - 
       This directive allows one to specify the criterion used for
       eliminating vectors, by specifying CRIT II using format (A,I). 
       The criterion is then set to 10$^{-II}$. Specifying CRIT 0
       amounts to a criterion of 10$^{-20}$.
\item  NUMBER or QUANTITY - 
       By specifying NUMBER NN using format (A,I) one requests the 
       elimination of NN vectors with the lowest eigenvalues of the S-matrix.
\item  PRINT -  
       If PRINT is specified using format(A), the eigenvalue analysis of 
       the S-matrix is always printed.
\end{itemize}

If both NUMBER and CRIT are specified, the criterion that eliminates 
most vectors applies. Note that the eigenvalues of the S-matrix, correspond
to a norm$^2$, so that 10$^{-7}$ is close to the machine precision 
(double precision).

{\bf Example}
To eliminate all vectors with an eigenvalue under 10$^{-6}$ and
at least 4 of them, specify 

{
\footnotesize
\begin{verbatim}
        DEPENDENCY CRIT 6 NUMBER 4 
\end{verbatim}
}

\subsection[The BASIS Directive]{The BASIS Directive}

Broadly speaking, the available basis sets may be classified into the
following categories;
\begin{itemize}
\item Minimal Basis Sets
\item Split-valence Basis Sets
\item Double-Zeta Basis Sets
\item Triple-zeta and Extended Basis Sets
\item ECP Basis Sets
\item Polarisation Basis Sets
\item Correlation consistent Basis Sets
\item Density Functional DFT Basis Sets
\end{itemize}
The specification of each category, referenced by a suitable codename,
is outlined below.

\subsubsection{Minimal Basis Sets}
\begin{itemize}
\item The STO-nG minimal basis due to Pople et al \cite{hehre}.
Basis sets are stored for H-Xe, and are code-named STO3G, STO4G etc.
\item The MINI basis sets due to Huzinaga \cite{huzinaga}. Two
sets are available. The MINI-1 set features three gaussian expansions 
of each atomic orbital:  since the exponents and contraction
coefficients are optimized for each element, and s and p
exponents are not constrained to be equal,
these bases give much lower energies than does STO-3G.
This set represents the most extensive coverage of the
elements within GAMESS, with the code-named MINI1 basis available
for all elements from Hydrogen to Radon (Z=86).
A second set, code-named MIDI4, features four gaussian
orbitals for each inner shell function.
\end{itemize}

\subsubsection{Split-valence Basis Sets}
Code-named SV, these include:
\begin{itemize}
\item  The split-valence set due to Dunning and Hay (code-named DUNNING) \cite{dunning};
and Binning and Curtis \cite{binning}, in the following contractions;

{
\footnotesize
\begin{verbatim}
                     <4s/2s>  for hydrogen
                 <9s5p/3s2p>  for first row atoms
                <11s7p/6s4p>  for second row atoms
           <14s11p5d/6s4p1d>  for 3rd row atoms
\end{verbatim}
}
\item  The n-m1G split-valence basis sets
due to Pople and co-workers \cite{binkley}
These include the 3-21G (H-Xe), 4-31G and 6-31G sets, and are
code-named as such.
\item The MIDI basis sets due to Huzinaga. These are  
derived from the MINI sets by
floating the outermost primitive in each valence orbitals,
and renormalizing the remaining 2 gaussians.  MIDI bases
are not scaled by GAMESS--UK.  The transition metal basis sets are
taken from s$^{1}$d$^{n}$ states. The MIDI bases are code-named
MIDI1 and MIDI4.
\item the split-valence set due to Schafer, Horn and Ahlrichs \cite{schafer}
(code named AHLRICHS) in the following contractions

{
\footnotesize
\begin{verbatim}
                      <4s/2s>  for hydrogen
                  <7s4p/3s2p>  for first row atoms
                 <10s7p/4s3p>  for second row atoms
             <14s8p6d/5s2p2d>  for 1st row transition metal atoms
            <14s10p5d/5s4p2d>  for 3rd row atoms
\end{verbatim}
}

\end{itemize}
A summary of the split-valence basis sets available is given in
Table~\ref{table:3}.

\subsubsection{Double-Zeta Basis Sets}
Code-named DZ, these include:
\begin{itemize}
\item The Dunning-Huzinaga full double-zeta sets for
H-Cl and Hay in the following contractions \cite{dunning1};

{
\footnotesize
\begin{verbatim}
           <4s/2s>      for hydrogen
           <9s5p/4s2p>  for first row atoms
           <11s7p/6s4p> for second row atoms
\end{verbatim}
}
\item the double-zeta sets due to Schafer, Horn and Ahlrichs \cite{schafer}
in the following contractions

{
\footnotesize
\begin{verbatim}
           <4s/2s>      for hydrogen
           <8s4p/4s2p>  for first row atoms
           <11s7p/6s4p> for second row atoms
           <14s9p6d/8s5p3d> for 1st row transition metal atoms
           <14s11p5d/8s6p2d> for 3rd row atoms
\end{verbatim}
}
\end{itemize}

\subsubsection{Triple-zeta and Extended Basis Sets}
\begin{itemize}
\item The triple-zeta valence (code-named TZV) basis sets due
to Dunning \cite{dunning2}, McClean \cite{mcclean} and Curtis \cite{binning}
 in the following contractions;

{
\footnotesize
\begin{verbatim}
           <5s/3s>           for hydrogen
           <10s6p/5s3p>      for first-row atoms
           <12s9p/6s5p>      for second-row atoms
           <14s9p5d/10s8p3d> for first-row transition metals [32]
           <14s11p5d/9s6p2d> for 3rd row atoms
\end{verbatim}
}
\item the triple-zeta set due to Schafer, Horn and 
Ahlrichs \cite{schafer} for H-Ar in the following contractions

{
\footnotesize
\begin{verbatim}
           <5s/3s>      for hydrogen
           <10s6p/6s3p>  for first row atoms
           <12s9p/7s5p> for second row atoms
\end{verbatim}
}
\item The 6-311G basis set of Pople and co-workers, code-named
6-311G.
\end{itemize}
A summary of the double-zeta and extended  basis sets available 
is given in Table~\ref{table:4}.

\subsubsection{ECP Basis Sets} 
Version 6.2 of GAMESS-UK contained just two of the standard literature
basis sets for use in performing valence-only calculations, namely;
\begin{itemize}
\item that due to Hay and Wadt \cite{hay}, now code-named LANL, covering
the elements Na-Bi.
\item the Compact Effective Potentials (code-name CEP or SBKJC) due to
(i) Stevens et al \cite{stevens} for the elements Li-Ar, (ii) Stevens
at al \cite{stevens2} for the elements K-Rn, and (iii) Cundari et al
\cite{cundari} for the Lanthanides.
\end{itemize}
The variety of available ECP basis sets has now been extended through
the addition of a further five basis set families, code-named as follows:
\begin{enumerate}
\item LANL2 -- The double-zeta Hay and Wadt ECP basis sets, with the
inner-valence forms used for transition metals etc. These are as provided
in the Gaussian and NWChem suite of programs \cite{hay}.
\item CRENBL -- The large ECP orbital basis for use with the small core 
potentials due to Christiansen et al \cite{christiansen}.
\item CRENBS -- The small ECP orbital basis for use with the averaged
relativistic, large core ECPs due to Ermler and co-workers \cite{ermler}.
\item STRLC -- The Stuttgart relativistic, large core ECP basis sets
due to Preuss at al. \cite{preuss1}.
\item STRSC --  The Stuttgart relativistic, small core ECP basis sets
due to Preuss at al. \cite{preuss2}.
\end{enumerate}
A full list of the elements for which ECP basis sets are available for
each of the seven library sets above is given in Table~\ref{table:5}.
Note that each of the five basis sets above are assumed to be spherical
harmonic, and not cartesian (in contrast to the CEP/SBKJC basis). Their
usage should thus be accompanied by the presentation of the HARMONIC
directive, although this is not enforced by the code.

Additional contractions of the two of the above sets remain available
for the LANL and CEP basis sets (note these contractions are NOT available
for the LANL2, CRENBL, CRENBS, STRLC or STRSC basis sets, and are
retained at present for consistency with previous versions of the code).
For the LANL basis the user may request minimal and double-zeta
contractions: for the CEP set minimal, double-zeta plus
extended (triple-zeta) contractions are available for Li-Ar, but
only double-zeta for K-Rn and the Lanthanides.

The code names for these contractions are, in obvious notation,
ECPMIN, ECPDZ and ECPTZV for minimal, double-zeta and triple-zeta
respectively. The program will load the LANL set where possible, only
loading the CEP set when
\begin{itemize}
\item the Hay set is not available e.g., for the 1st row elements,
Li-Ne, or
\item when requested by the user, through specification of the
CEP keyword (see below)
\end{itemize}

\subsubsection{Polarisation Basis Sets}
Single first polarisation functions may
be appended to several of the standard basis sets above through the
use of suitably modified code-names. Specifically:
\begin{itemize}
\item For the split-valence Dunning, Ahlrichs  and Pople basis sets, 
SVP will result in the split-valence set plus one set of d-functions on
the heavy atoms and one set of p-functions on hydrogen.
\item For Double-zeta basis sets, DZP will
result in the DZ set plus one set of d-functions on
the heavy atoms and one set of p-functions on hydrogen.
\item For Triple-zeta basis sets, TZVP will
result in the extended set plus one set of d-functions on
the heavy atoms and one set of p-functions on hydrogen.
\item Note that when appending polarisation functions to the
standard GAUSSIAN basis sets e.g., 4-31G, 6-311G etc, the usual
notation `*' (append to heavy-atoms only) and `**' (append
to both heavy and hydrogen atoms) may be employed. Thus
the code-name 6-311G** will yield the 6-311G set with polarisation
functions on all heavy plus all hydrogen atoms.
\item For the LANL and CEP ECP basis sets, ECPTZVP will provide one set of
d-functions on the heavy atoms (only available for the CEP basis for Li-Ar).
\end{itemize}

\subsubsection{Correlation consistent Basis Sets}
A subset of the
correlation consistent basis sets due to Dunning and co-workers
\cite{dunning3} are available. These include the double-zeta (code
named CC-PVDZ), triple-zeta sets (code name CC-PVTZ), quadruple-zeta
sets (code name CC-PVQZ), and quintuple-zeta sets (code name CC-PV5Z).
With the exception of magnesium, all four basis sets available for all
the H-Ar and Ga-Kr series.

\subsubsection{DFT Basis Sets}
The polarized DFT orbital basis sets due to Godbout et al.~\cite{godbout},
originally developed for the DGauss and DeMon DFT packages. Three such
sets are available, the DZVP, DZVP2 and TZVP sets.


\begin{table}
 \caption{\label{table:3}\ Library of Split-valence Basis Sets}
 
 \begin{centering}
 \begin{tabular}{lcccccccccc}
 \\ \hline\hline
Element &  \multicolumn{10}{c}{Basis Set}\\
         \cline{2-11}
   & 3-21G   & 4-21G   & 6-21G   & 4-31G   & 5-31G   & 6-31G    & MIDI-1    & MIDI-4   & DUNNING  & AHLRICHS \\ \cline{2-11}

H  & $\surd$ & $\surd$ & $\surd$ & $\surd$ & $\surd$ & $\surd$  &  $\surd$  & $\surd$  &  $\surd$ & $\surd$  \\
He & $\surd$ & $\surd$ & $\surd$ &         &         &          &           &          &          & $\surd$ \\
Li & $\surd$ & $\surd$ & $\surd$ & 5-21G   & 5-21G   & $\surd$  &  $\surd$  & $\surd$  &  $\surd$ & $\surd$  \\
Be & $\surd$ & $\surd$ & $\surd$ & 5-21G   & 5-21G   & $\surd$  &  $\surd$  & $\surd$  &  $\surd$ & $\surd$ \\
B  & $\surd$ & $\surd$ & $\surd$ & $\surd$ &         & $\surd$  &  $\surd$  & $\surd$  &  $\surd$ & $\surd$ \\
C  & $\surd$ & $\surd$ & $\surd$ & $\surd$ & $\surd$ & $\surd$  &  $\surd$  & $\surd$  &  $\surd$ & $\surd$  \\
N  & $\surd$ & $\surd$ & $\surd$ & $\surd$ & $\surd$ & $\surd$  &  $\surd$  & $\surd$  &  $\surd$ & $\surd$ \\
O  & $\surd$ & $\surd$ & $\surd$ & $\surd$ & $\surd$ & $\surd$  &  $\surd$  & $\surd$  &  $\surd$ & $\surd$ \\
F  & $\surd$ & $\surd$ & $\surd$ & $\surd$ & $\surd$ & $\surd$  &  $\surd$  & $\surd$  &  $\surd$ & $\surd$ \\
Ne & $\surd$ & $\surd$ & $\surd$ & $\surd$ &         & $\surd$  &  $\surd$  & $\surd$  &  $\surd$ & $\surd$  \\
\\ 
Na & $\surd$ &         & $\surd$ &         &         & $\surd$  &  $\surd$  & $\surd$  &         & $\surd$  \\
Mg & $\surd$ &         & $\surd$ &         &         & $\surd$  &  $\surd$  & $\surd$  &         & $\surd$  \\
Al & $\surd$ &         & $\surd$ &         &         & $\surd$  &  $\surd$  & $\surd$  & $\surd$ & $\surd$ \\
Si & $\surd$ &         & $\surd$ &         &         & $\surd$  &  $\surd$  & $\surd$  & $\surd$ & $\surd$ \\
P  & $\surd$ &         & $\surd$ & $\surd$ &         & $\surd$  &  $\surd$  & $\surd$  & $\surd$ & $\surd$ \\
S  & $\surd$ &         & $\surd$ & $\surd$ &         & $\surd$  &  $\surd$  & $\surd$  & $\surd$ & $\surd$ \\
Cl & $\surd$ &         & $\surd$ & $\surd$ &         & $\surd$  &  $\surd$  & $\surd$  & $\surd$ & $\surd$ \\
Ar & $\surd$ &         & $\surd$ &         &         & $\surd$  &  $\surd$  & $\surd$  &         & $\surd$  \\
\\
 K-Ca & $\surd$ &      &         &         &         & $\surd$  &  $\surd$  & $\surd$  &         & $\surd$  \\
Sc-Cu & $\surd$ &      &         &         &         & $\surd$  &  $\surd$  & $\surd$  & $\surd$ & $\surd$ \\
Zn    & $\surd$ &      &         &         &         & $\surd$  &  $\surd$  & $\surd$  &         & $\surd$ \\
Ga-Kr & $\surd$ &      &         &         &         &          &  $\surd$  & $\surd$  & $\surd$ & $\surd$ \\
Cs-Xe & $\surd$  \\
\\ \hline \hline
\end{tabular}

\end{centering}
\end{table}
\clearpage

\begin{table}
 \caption{\label{table:4}\ Library of Double-Zeta, Triple-Zeta
                and Extended Basis Sets}
 
 \begin{centering}
 \begin{tabular}{lcccccc}
 \\ \hline\hline
Element &  \multicolumn{6}{c}{Basis Set}\\
         \cline{2-7}
      & DZ/DZP  & DZ/DZP & TZV/TZVP & TZV/TZVP & 6-311G & 6-311G*   \\ 
      & Dunning & Ahlrichs & Dunning & Ahlrichs&        &           \\ \cline{2-7}

H     & $\surd$ & $\surd$ & $\surd$ & $\surd$ & $\surd$ & $\surd$    \\
He    &         & $\surd$ &         & $\surd$ & $\surd$ & $\surd$    \\
Li    & $\surd$ & $\surd$ &         & $\surd$ & $\surd$ & $\surd$    \\
Be    & $\surd$ & $\surd$ & $\surd$ & $\surd$ & $\surd$ & $\surd$    \\
B     & $\surd$ & $\surd$ & $\surd$ & $\surd$ & $\surd$ & $\surd$    \\
C     & $\surd$ & $\surd$ & $\surd$ & $\surd$ & $\surd$ & $\surd$    \\
N     & $\surd$ & $\surd$ & $\surd$ & $\surd$ & $\surd$ & $\surd$    \\
O     & $\surd$ & $\surd$ & $\surd$ & $\surd$ & $\surd$ & $\surd$    \\
F     & $\surd$ & $\surd$ & $\surd$ & $\surd$ & $\surd$ & $\surd$    \\
Ne    & $\surd$ & $\surd$ & $\surd$ & $\surd$ & $\surd$ & $\surd$    \\
\\ 
Na    &         & $\surd$ &         & $\surd$ & $\surd$ & $\surd$    \\
Mg    &         & $\surd$ &         & $\surd$ & $\surd$ & $\surd$    \\
Al    & $\surd$ & $\surd$ & $\surd$ & $\surd$ & $\surd$ & $\surd$    \\
Si    & $\surd$ & $\surd$ & $\surd$ & $\surd$ & $\surd$ & $\surd$    \\
P     & $\surd$ & $\surd$ & $\surd$ & $\surd$ & $\surd$ & $\surd$    \\
S     & $\surd$ & $\surd$ & $\surd$ & $\surd$ & $\surd$ & $\surd$    \\
Cl    & $\surd$ & $\surd$ & $\surd$ & $\surd$ & $\surd$ & $\surd$    \\
Ar    & $\surd$ & $\surd$ & $\surd$ & $\surd$ & $\surd$ & $\surd$    \\
\\
K-Ca  &         & $\surd$ &         &                                \\
Sc-Zn & $\surd$ & $\surd$ & $\surd$ & $\surd$ &         &       \\
Ga-Kr &         & $\surd$ & $\surd$ & $\surd$ &                   \\
\hline \hline
\end{tabular}

\end{centering}
\end{table}

\subsection[Global Specification]{Global Specification}

At the simplest level, it is assumed that each element specified in the
z-matrix is to be described by functions belonging to the same category
of basis set. In this case the BASIS directive comprises a single data
line read to variables TEXT, CODENAME using format (2A).
\begin{itemize}
\item  TEXT should be set to  the character string BASIS
\item  CODENAME  should be set to the code-name of the
required basis to be sited on each element specified
in the ZMATRIX or GEOMETRY directive
(not a DUMMY centre).
\end{itemize}
Note that in some cases two separate strings may be required in
defining the CODENAME.
Valid CODENAME strings include the following:
\begin{itemize}
\item {\bf STO3G, STO4G} etc.  - invokes the minimal STO-nG minimal basis
due to Pople et al \cite{hehre}

\item {\bf MINI1} - invokes the MINI-1 basis due to Huzinaga.
\item {\bf MINI4} - invokes the MINI-4 basis due to Huzinaga.

\item  {\bf SV } - invokes the split-valence set due to Dunning and Hay.
\item  To  invoke the split-valence basis sets due to Ahlrichs and
co-workers \cite{schafer} append the keyword AHLRICHS thus:
  {\bf SV AHLRICHS} 

\item  To  invoke the split-valence basis sets due to Pople and
co-workers \cite{binkley} append the appropriate keyword thus:
  {\bf SV 3-21G} ,  {\bf SV 4-31G}, {\bf SV 6-31G}

\item {\bf MIDI1} - invokes the MIDI-1 basis due to Huzinaga.
\item {\bf MIDI4} - invokes the MIDI-4 basis due to Huzinaga.

\item  {\bf DZ}  - invokes the double-zeta basis sets due
to Dunning \cite{dunning2} and McClean \cite{mcclean}.
\item  To  invoke the double-zeta basis sets
due to Ahlrichs and co-workers \cite{schafer}
append the keyword AHLRICHS thus:
  {\bf DZ  AHLRICHS} 

\item {\bf CC-PVDZ} - invokes the double-zeta polarised
correlation consistent basis sets due to Dunning \cite{dunning3}.

\item  {\bf TZV}  - invokes the triple-zeta basis sets due
to Dunning \cite{dunning2} and McClean \cite{mcclean}.

\item  To  invoke the triple-valence basis sets due to Ahlrichs and
co-workers \cite{schafer} append the keyword AHLRICHS thus:
  {\bf TZV AHLRICHS}

\item  {\bf TZVP}  - the TZV basis, augmented with polarisation
functions (p on H, d on first- and second-row atoms \cite{ahlrichs})

\item  To  invoke the triple-zeta basis sets augmented with polarisation
functions due to Ahlrichs and co-workers \cite{schafer}
append the keyword AHLRICHS thus: {\bf TZVP AHLRICHS} 

\item {\bf CC-PVTZ} - invokes the triple-zeta polarised
correlation consistent basis sets due to Dunning \cite{dunning3}.

\item {\bf CC-PVQZ} - invokes the quadruple-zeta polarised
correlation consistent basis sets due to Dunning \cite{dunning3}.

\item {\bf CC-PV5Z} - invokes the quintuple-zeta polarised
correlation consistent basis sets due to Dunning \cite{dunning3}.

\item {\bf ECP} - invokes one of the family of seven available ECP basis
sets. An additional keyword is now required to specify the family code-name,
using one of the character strings LANL, CEP or SBKJC, LANL2, CRENBL, CRENBS,
STRLC or STRSC (see above). Thus to invoke the Stuttgart RLC ECP basis set,
append the keyword STRLC, thus; \\
  {\bf BASIS ECP STRLC} 

As in Version 6.2 of the code, minimal, double- and triple-zeta
contractions of the LANL and CEP basis sets remain available. These are
requested in different fashion to the above generic specification, involving
one of the following keywords on the BASIS directive;
\begin{itemize}
\item {\bf BASIS ECPMIN} - ECP valence-only basis in minimal contraction.
\item {\bf BASIS ECPDZ} - ECP valence-only basis in a double-zeta contraction.
\item {\bf BASIS ECPTZV}- ECP valence-only basis in a triple-zeta contraction.
\item {\bf BASIS ECPTZVP} - ECP valence-only basis in a triple-zeta contraction,
augmented with polarisation functions.
\end{itemize}
Note again that these contractions are only available for the LANL and
CEP basis sets, and not for the LANL2, CRENBL, CRENBS, STRLC or STRSC
families.
\end{itemize}

{\bf Examples}
\begin{enumerate}
\item The 4-31G basis of Pople and co-workers may be requested
through either of the two data lines:

{
\footnotesize
\begin{verbatim}
        BASIS SV 4-31G       or          BASIS 4-31G
\end{verbatim}
}
To append polarisation functions to just the heavy atoms, use
{
\footnotesize
\begin{verbatim}
                  BASIS  4-31G*
\end{verbatim}
}
and to both heavy and hydrogen atoms,
{
\footnotesize
\begin{verbatim}
                  BASIS  4-31G**
\end{verbatim}
}
with no blank characters preceding the '*'.
\item A minimal STO-3G basis is requested thus,
{
\footnotesize
\begin{verbatim}
                  BASIS  STO3G
\end{verbatim}
}
\item The double-zeta plus polarisation basis of Dunning is
requested thus,
{
\footnotesize
\begin{verbatim}
                  BASIS  DZP
\end{verbatim}
}
\item The double-zeta plus polarisation basis of Ahlrichs  is
requested thus,
{
\footnotesize
\begin{verbatim}
                  BASIS  DZP AHLRICHS
\end{verbatim}
}
\item The correlation consistent cc-pvdz basis of Dunning is
requested thus,
{
\footnotesize
\begin{verbatim}
                  BASIS  CC-PVDZ
\end{verbatim}
}
\item The correlation consistent cc-pvdz spherical harmonics basis of 
Dunning is requested through specification of the HARMONIC option, thus
{
\footnotesize
\begin{verbatim}
                  HARMONIC
                  BASIS  CC-PVDZ
\end{verbatim}
}
\item The CEP ECP basis set due to Stevens et al \cite{stevens} 
is requested thus,
{
\footnotesize
\begin{verbatim}
                  BASIS  ECP SBKJC
 or
                  BASIS  ECP CEP
\end{verbatim}
}
\item The ECP spherical harmonic basis due to Hay \cite{hay} 
is requested through additional specification of the HARMONIC option, thus
{
\footnotesize
\begin{verbatim}
                  HARMONIC
                  BASIS  ECP LANL2
\end{verbatim}
}
\end{enumerate}

\subsection[Hybrid Specification]{Hybrid Specification}

In the more general case, it is possible to request basis functions
from different categories of the available `built-in' sets. In such
cases the User must specify the basis to be sited on each element of
the z-matrix which had been defined with a unique tag. 
A more general form of the
BASIS directive is again used to define the basis set.
The first line, the directive initiator, consists of the character
string BASIS in the first data field. The last line of the
directive, the directive terminator, consists of the text END in the
first data field. Lines between the initiator and terminator define the
basis set, and consist of `basis definition' 
lines. Each  basis definition line is read to variables TYPE, TAGA,
TYPEE using format (3A).
\begin{itemize}
\item  TYPE should be set to the code-name of the required
set of basis functions.
\item TAGA should be set to a TAG labelling one or more of one of the centres 
(not a DUMMY centre)
defined in the ZMATRIX or  GEOMETRY directive. The
group of basis functions will be sited on the nominated centre(s).
\item  TYPEE should be used when necessary to uniquely define 
the requested set of basis functions.
\end{itemize}
Let us assume
that in the formaldehyde examples of Part 2 we wished to include
polarisation functions {\em only}  on the oxygen atom. This could
be achieved by the following data sequence:

{
\footnotesize
\begin{verbatim}
           BASIS
           TZV H
           TZV C
           TZVP O
           END
\end{verbatim}
}
To nominate the 6-311G extended set of Pople would require use of
the third field on each definition line, thus

{
\footnotesize
\begin{verbatim}
           BASIS
           TZV H 6-311G
           TZV C 6-311G
           TZVP O 6-311G
           END
\end{verbatim}
}
while requesting the corresponding set of Ahlrichs would be
achieved by the following data lines:
{
\footnotesize
\begin{verbatim}
           BASIS
           TZV H AHLRICHS
           TZV C AHLRICHS
           TZVP O AHLRICHS
           END
\end{verbatim}
}
Similarly , locating a split-valence 3-21G basis on hydrogen and a
4-31G basis on carbon and oxygen would be achieved thus, where
again the TYPEE field is used to uniquely define the required
basis

{
\footnotesize
\begin{verbatim}
           BASIS
           SV H 3-21G
           SV O 4-31G
           SV C 4-31G
           END
\end{verbatim}
}
One clear example where this hybrid specification will be required is when
performing ECP calculations where the ECP basis is only to be deployed
on a sub-set of the atoms specified in the z-matrix.  Thus to locate an
all electron DZP basis set on hydrogen and the CRENBL ECP basis set on
oxygen and carbon would be achieved as follows, where the TYPEE field is
now used to define the required ECP basis,

{
\footnotesize
\begin{verbatim}
           BASIS
           DZP H 
           ECP O CRENBL
           ECP C CRENBL
           END
\end{verbatim}
}

\subsection[Scaling Basis Functions]{Scaling Basis Functions}
Under hybrid specification, it is possible to input scaling factors
for the exponents of the built-in basis sets to override the standard
values, so that
\begin{equation}
    \zeta^{i} = \zeta_{0}^{i} \times scale^{2}
\end{equation}
where scale is the input value, and i indicates the i'th
shell on the centre of interest. This is achieved by appending the
scaling factors to the 'basis definition' lines. As such a line
is in fact responsible for loading multiple shells on a given
centre, the user need be aware of the loading order when attempting
to override the default values. Considering, for example, the STO-nG
functions, the loading order internally is 1s, 2sp, 3sp, 3d, 4sp, 
4d and 5sp. Thus if one wishes to modify the 3d scaling factor for
vanadium (default value of 2.55), the input factor needs point to
the fourth shell. In specifying such values, the program recognises
an input value of zero as requesting the default, so that in the
present example, the data line

{
\footnotesize
\begin{verbatim}
          STO3G  V  0.0  0.0  0.0  3.0
\end{verbatim}
}
will modify the 3d scale factor from 2.55 to 3.0. To replace the
default hydrogenic scaling factor of 1.24 by 1.30 
would require the specification

{
\footnotesize
\begin{verbatim}
          STO3G  H  1.30
\end{verbatim}
}
For DZ, TZV and
associated polarisation basis sets, this scaling is limited to
hydrogen. For the DZ basis, the default scaling factors are 1.20
and 1.15 for the hydrogenic s-functions and 1.0 for the 2p (DZP).
To override the s factors by unit scale would require the data
line

{
\footnotesize
\begin{verbatim}
          DZ  H  1.0  1.0
\end{verbatim}
}
A full list of default factors and more details on the 
loading orders is available from the author on request. Note also
that any request to load further basis sets to the built-in 
library will be handled as swiftly as possible.

\subsection[General Basis Specification]{General Basis Specification}

Finally it is possible to define explicitly the basis set through
the exponents and coefficients of the component primitive
functions .

The first line, the directive initiator, again consists of the character
string BASIS   in the first data field. The last line of the
directive, the directive terminator, consists of the text END in the
first data field. Lines between the initiator and terminator define the
basis set, and consist of 'group definition' and 'primitive definition'
lines. The data for each group of contracted functions is introduced
in turn, as follows:

\begin{itemize}
\item The group definition line is read to variables TYPE, TAGA
using (2A).
\begin{itemize}
\item  TYPE should be set to one of the characters S,P,D or F.
These symbols
defines the type of group of basis functions being introduced.
\item TAGA should be set to a TAG of one or more of the centres 
defined in a GEOMETRY or ZMATRIX directive. The
group of basis functions will be sited on the nominated centre(s).
\end{itemize}

\item The primitive definition lines follow the group definition line,
and are used to define the contraction coefficients and orbital
exponents of the primitive associated with the group. If NTERM
primitives are to be used (NTERM $\leq$ 25), NTERM primitive definition
lines are required, each read to CTRAN, ZETA using format (2F).
\begin{itemize}
\item  CTRAN the contraction coefficient of the primitive.
\item  ZETA   the orbital exponent of the primitive.
\end{itemize}

\end{itemize}

The following example defines a double zeta basis set for pyrimidine,
with the addition of diffuse s,p and d basis basis functions at
a bond centre. Note that basis functions are cited on those 
with unique TAGs in the z-matrix i.e C, N and H and the mid-point, BQ.

{
\footnotesize
\begin{verbatim}
           TITLE
           PYRIMIDINE  DZ  AT EQUIL GEOMETRY 
           ZMAT ANGSTROM
           C
           H 1 CH2
           X 1 1.0 2 90.
           X 1 1.0 2 90. 3 90.
           C 1 C1C4 3 90. 2 180.
           X 5 1.0 1 90. 3 0.0
           X 5 1.0 1 90. 4 0.0
           H 5 CH4 6 90. 1 180.
           N 1 CN1 2 NCH2 3 180.
           N 1 CN1 2 NCH2 3 0.0
           C 9 CN2 1 CNC 2 180.
           C 10 CN2 1 CNC 2 180.
           H 11 CH35 9 NCH3 1 180.
           H 12 CH35 10 NCH3 1 180.
           BQ  1 1.34249115 3 90. 6 0.0
           VARIABLES
           CH2 1.0789117
           CH35 1.0810722
           CH4 1.0793701
           CN1 1.3350902
           CN2 1.3385252
           C1C4 2.6849823
           NCH2 117.525610
           CNC 117.642581
           NCH3 116.792492
           END
           BASIS
           S H
           0.032828 13.3615
           0.231208  2.0133
           0.817238  0.4538
           S H
           1.0  0.1233
           S N
           0.0020040  5909.44
           0.0153100   887.451
           0.0742930   204.749
           0.2533640    59.8376
           0.6005760    19.9981
           0.2451110     2.686
           S N
           1.0 7.1927
           S N
           1.0  0.7
           S N
           1.0  .2133
           P N
           0.018257   26.786
           0.116407    5.9564
           0.390111    1.7074
           0.637221    0.5314
           P N
           1.0  0.1654
           S C
           0.002090 4232.61
           0.015535  634.882
           0.075411  146.097
           0.257121   42.4974
           0.596555   14.1892
           0.242517    1.9666
           S C
           1.0  5.1477
           S C
           1.0 0.4962
           S C
           1.0  0.1533
           P C
           0.018534  18.1557
           0.1154420  3.9864
           0.3862060  1.1429
           0.6400890  0.3594
           P C
           1.0  0.1146
           S BQ
           1.0 0.021
           S BQ
           1.0 0.008
           S BQ
           1.0 0.0025
           P BQ
           1.0 0.017
           P BQ
           1.0 0.009
           D BQ
           1.0 0.015
           D BQ
           1.0 0.008
           END
\end{verbatim}
}


Note that there is no restriction as to the ordering of the groups in
the input stream. However the program will internally re-organise the
data to produce a list of atom-ordered functions.  Note that the User may
mix such `external' basis sets with the available `built-in' functions.
In the following example we wish to add a single diffuse s-function and
polarisation f-function to the TZVP carbon basis in formaldehyde;

{
\footnotesize
\begin{verbatim}
           BASIS
           TZVP H
           TZVP C
           TZVP O
           S C
           1.0 0.02
           F C
           1.0 1.0
           END
\end{verbatim}
}
In the pyrimidine example above, use of the built-in DZ basis set
would reduce the BASIS data, thus

{
\footnotesize
\begin{verbatim}
           BASIS
           DZ H
           DZ N
           DZ C
           S BQ
           1.0 0.021
           S BQ
           1.0 0.008
           S BQ
           1.0 0.0025
           P BQ
           1.0 0.017
           P BQ
           1.0 0.009
           D BQ
           1.0 0.015
           D BQ
           1.0 0.008
           END
\end{verbatim}
}
Finally, we note that it is possible to cite 'built-in' basis
sets from the library onto BQ centres, intended for use in basis
set superposition studies. Consider the HCN example presented
in the previous discussion of ghost centres.

{
\footnotesize
\begin{verbatim}
          TITLE
          HCN  DUNNING DZ + BOND(S,P) 
          ZMAT ANGSTROM
          C
          BQ 1 RCN2
          X 2 1.0 1 90.0
          N 2 RCN2 3 90.0 1 180.0
          X 1 1.0 2 90.0 3 0.0
          H 1 RCH 5 90.0 4 180.0
          VARIABLES
          RCN2 0.580
          RCH 1.056
          END
          BASIS
          DZ H
          S BQ
          1.0 1.0
          P BQ
          1.0 0.7
          DZ C
          DZ N
          END
          RUNTYPE OPTIMIZE
          ENTER
\end{verbatim}
}
To cite the DZ basis functions at the hydrogenic position (denoted BQH
below) while performing an open shell SCF computation on the
$^{2}\Sigma^{+}$ state of the CN radical would be achieved thus, where
the data line 

{
\footnotesize
\begin{verbatim}
          DZ BQH H
\end{verbatim}
}
is placing the H DZ functions on the BQH ghost centre:
{
\footnotesize
\begin{verbatim}
          TITLE
          CN  DUNNING DZ + BOND(S,P) 
          MULT 2
          ZMAT ANGSTROM
          C
          BQ 1 RCN2
          X 2 1.0 1 90.0
          N 2 RCN2 3 90.0 1 180.0
          X 1 1.0 2 90.0 3 0.0
          BQH 1 RCH 5 90.0 4 180.0
          VARIABLES
          RCN2 0.5991
          RCH 1.056
          END
          BASIS
          DZ BQH H
          S BQ
          1.0 1.0
          P BQ
          1.0 0.7
          DZ C
          DZ N
          END
          ENTER
\end{verbatim}
}

\subsection[NWChem and G94 Basis Specification]{NWChem and G94 Basis Specification}

Finally, in order to provide more compatibility with other ab initio
packages, in particular NWChem and Gaussian, it is possible to invert
the field ordering on both Group definition (Centre and Group tags) and
Primitive definition lines (the exponents and coefficients of the
component primitive functions).

The first line, the directive initiator, again consists of the
character string BASIS in the first data field, with either the
character string NWCHEM or G94 now presented as the second data field.
The last line of the directive, the directive terminator, consists of
the text END in the first data field. Lines between the initiator and
terminator define the basis set, consisting  of 'group definition' and
'primitive definition' lines. The data for each group of contracted
functions is introduced in turn, with the data fields in reverse
ordering to the General Basis lines above, as follows:

\begin{itemize}
\item The group definition line is now read to variables TAGA, TYPE
using (2A).
\begin{itemize}
\item TAGA should be set to a TAG of one or more of the centres 
defined in a GEOMETRY or ZMATRIX directive. The
group of basis functions will be sited on the nominated centre(s).
\item  TYPE should be set to one of the characters S,P,D or F.
These symbols defines the type of group of basis functions being introduced.
\end{itemize}

\item The primitive definition lines follow the group definition line,
and are used to define the contraction coefficients and orbital
exponents of the primitive associated with the group. If NTERM
primitives are to be used (NTERM $\leq$ 25), NTERM primitive definition
lines are required, each read to ZETA, CTRAN using format (2F).
\begin{itemize}
\item  ZETA   the orbital exponent of the primitive.
\item  CTRAN the contraction coefficient of the primitive.
\end{itemize}

\end{itemize}

The following example is that given above, defining a double zeta basis
set for pyrimidine, with the addition of diffuse s,p and d basis basis
functions at a bond centre.

{
\footnotesize
\begin{verbatim}
           TITLE
           PYRIMIDINE  DZ  AT EQUIL GEOMETRY 
           ZMAT ANGSTROM
           C
           H 1 CH2
           X 1 1.0 2 90.
           X 1 1.0 2 90. 3 90.
           C 1 C1C4 3 90. 2 180.
           X 5 1.0 1 90. 3 0.0
           X 5 1.0 1 90. 4 0.0
           H 5 CH4 6 90. 1 180.
           N 1 CN1 2 NCH2 3 180.
           N 1 CN1 2 NCH2 3 0.0
           C 9 CN2 1 CNC 2 180.
           C 10 CN2 1 CNC 2 180.
           H 11 CH35 9 NCH3 1 180.
           H 12 CH35 10 NCH3 1 180.
           BQ  1 1.34249115 3 90. 6 0.0
           VARIABLES
           CH2 1.0789117
           CH35 1.0810722
           CH4 1.0793701
           CN1 1.3350902
           CN2 1.3385252
           C1C4 2.6849823
           NCH2 117.525610
           CNC 117.642581
           NCH3 116.792492
           END
           BASIS G94
           H S
           13.3615 0.032828 
            2.0133 0.231208 
            0.4538 0.817238 
           H S
           0.1233  1.0
           N S
           5909.44  0.0020040
           887.451  0.0153100
           204.749  0.0742930
           59.8376  0.2533640
           19.9981  0.6005760
            2.686   0.2451110
           N S
           7.1927  1.0
           N S
           0.7     1.0
           N S
           0.2133  1.0
           N P
           26.786  0.018257
           5.9564  0.116407
           1.7074  0.390111
           0.5314  0.637221
           N P
           0.1654  1.0
           C S
           4232.61  0.002090
           634.882  0.015535
           146.097  0.075411
           42.4974  0.257121
           14.1892  0.596555
           1.9666   0.242517
           C S
           5.1477  1.0
           C S
           0.4962  1.0
           C S
           0.1533  1.0
           C P
           18.1557  0.018534 
           3.9864   0.1154420
           1.1429   0.3862060
           0.3594   0.6400890
           C P
           0.1146  1.0
           BQ S
           0.021   1.0
           BQ S
           0.008   1.0
           BQ S
           0.0025  1.0
           BQ P
           0.017   1.0
           BQ P
           0.009   1.0
           BQ D
           0.015   1.0
           BQ D
           0.008   1.0
           END
\end{verbatim}
}


Note that the User may still mix 
`external' basis sets with the available `built-in' functions. 
In the following example we wish to add a single diffuse s-function 
and polarisation f-function to the
TZVP carbon basis in formaldehyde;

{
\footnotesize
\begin{verbatim}
           BASIS NWCHEM
           H TZVP
           C TZVP
           O TZVP
           C S
           0.02 1.0 
           C F
           1.0 1.0
           END
\end{verbatim}
}
In the pyrimidine example above, use of the built-in DZ basis set
would reduce the BASIS data, thus

{
\footnotesize
\begin{verbatim}
           BASIS NWCHEM
           H DZ 
           N DZ 
           C DZ 
           BQ S
           0.021 1.0
           BQ S
           0.008 1.0
           BQ S
           0.0025 1.0
           BQ P
           0.017 1.0
           BQ P
           0.009 1.0
           BQ D
           0.015 1.0
           BQ D
           0.008 1.0
           END
\end{verbatim}
}

\subsection[Shell-structured Basis Specification]{Shell-structured Basis Specification}

A modification to the data input specification above is required
if the user wishes to explicitly input sp-type shells i.e., one s-type
function and three p functions, while retaining the computational
advantages associated with such basis sets. Two extensions have been
made to the group definition and primitive definition lines, namely
\begin{enumerate}
\item The TYPE variable on the group definition line should be set to the
character L.
\item Each primitive definition line is now read to the variables
CTRANS, ZETA, CTRANP using format (3F)
\begin{itemize}
\item CTRANS is the contraction coefficient of the s-primitive
\item ZETA is the common orbital exponent
\item CTRANP is the contraction coefficient of the p-primitive.
\end{itemize}
\end{enumerate}
{\bf Example}\\

The following data contains explicit specification of the
3-21G basis for the \formaldehyde\ molecule in a closed--shell
SCF calculation.

{
\footnotesize
\begin{verbatim}
          TITLE
          H2CO - 3-21G BASIS - EXPLICIT BASIS SPECIFICATION
          ZMATRIX ANGSTROM
          C
          O 1 1.203
          H 1 1.099 2 121.8
          H 1 1.099 2 121.8 3 180.0
          END
          BASIS
          S C       
          0.061767   172.256000
          0.358794    25.910900
          0.700713     5.533350
          L C
          -0.395897    3.664980     0.236460  
           1.215840    0.770545     0.860619  
          L C
          1.000    0.195857         1.000
          S O       
          0.059239   322.037000
          0.351500    48.430800 
          0.707658    10.420600
          L O
          -0.404453    7.402940     0.244586  
           1.221560    1.576200     0.853955  
          L O
          1.000   0.373684  1.000
          S  H       
          0.156285     5.447178 
          0.904691     0.824547
          S  H
          1.000    0.183192 
          END
          ENTER 
\end{verbatim}
}

\subsection[PSEUDO]{PSEUDO}

This directive is used to request a  valence-only rather than all-electron
treatment of the molecular system in hand, to specify the nature of this
valence-only treatment and to define those centres within the system to
be treated by the pseudo-orbital method.  The first data line is read
to the variables TEXT, TYPE using format (2A).
\begin{itemize}
\item TEXT is set the character string PSEUDO \item TYPE is used to
define the nature of the pseudopotential treatment. When requesting
the non-local formalism, TYPE should be set to the character string
NONLOCAL. If TYPE is set to the character string ECP, or is left blank,
one of the available semi-local ECPs will be employed. Seven such
semi-local sets are now available, code-named as follows;
\begin{enumerate}
\item CEP or SBKJC -- available in Version 6.2 of the code, the
the Compact Effective Potentials (CEPs) are due to (i) Stevens et al
\cite{stevens} for the elements Li-Ar, (ii) Stevens at al \cite{stevens2}
for the elements K-Rn, and (iii) Cundari et al \cite{cundari} for
the Lanthanides.
\item LANL -- available in version 6.2 of the code, the LANL ECPs
are due to Hay and Wadt \cite{hay} and cover the elements Na-Bi.
\item LANL2 -- The Hay and Wadt ECPs, with the inner-valence forms
used for transition metals etc. These are as provided in the
Gaussian and NWChem suite of programs.
\item CRENBL -- The small core potential due to Christiansen et al
\cite{christiansen}.  These ECPs are sometimes referred to as shape
consistent because they maintain the shape of the atomic orbitals in
the valence region.
\item CRENBS -- The averaged relativistic, large core ECPs due to Ermler
and co-workers \cite{ermler}.
\item STRLC -- The Stuttgart relativistic, large core ECP due to Preuss
at al. \cite{preuss1}.
\item STRSC --  The Stuttgart relativistic, small core ECP due to Preuss
at al. \cite{preuss2}.
\end{enumerate}
A full list of the elements for which ECPs are available for each of the 
seven library sets above is given in Table~\ref{table:5} and Table~\ref{table:6}.
\end{itemize}

\begin{table}
 \caption{\label{table:5}\ Library of Available ECPs with Number of Frozen-core Electrons (Li-Xe)}
 
 \begin{centering}
 \begin{tabular}{lcccccccc}
 \\ \hline\hline
Element & Z & \multicolumn{7}{c}{ECP Library}\\
         \cline{3-9}
        &     &    SBKJC &  LANL  &    LANL2  &  CRENBL  &   CRENBS &  STRLC &   STRSC \\ \hline
Li- Ne & 3 - 10 &  2   &    2    &      -    &    2    &      -    &    2   &    -  \\ \hline
Na   &  11   &     10  &    10   &     10    &    10   &      -    &    10  &    -  \\
Mg   &  12   &     10  &    10   &     10    &    10   &      -    &    10  &    -  \\
Al   &  13   &     10  &    10   &     10    &    10   &      -    &    10  &    -  \\
Si   &  14   &     10  &    10   &     10    &    10   &      -    &    10  &    -  \\
P    &  15   &     10  &    10   &     10    &    10   &      -    &    10  &    -  \\
S    &  16   &     10  &    10   &     10    &    10   &      -    &    10  &    -  \\
Cl   &  17   &     10  &    10   &     10    &    10   &      -    &    10  &    -  \\
Ar   &  18   &     10  &    10   &     10    &    10   &      -    &    10  &    - \\ \hline
K    &  19   &     18  &    18   &     18    &    10   &      -    &    18  &    10  \\
Ca   &  20   &     18  &    18   &     18    &    10   &      -    &    18  &    10 \\ \hline
Sc   &  21   &     10  &    18   &     10    &    10   &      18   &     -  &    10  \\
Ti   &  22   &     10  &    18   &     10    &    10   &      18   &     -  &    10  \\
V    &  23   &     10  &    18   &     10    &    10   &      18   &     -  &    10  \\
Cr   &  24   &     10  &    18   &     10    &    10   &      18   &     -  &    10  \\
Mn   &  25   &     10  &    18   &     10    &    10   &      18   &     -  &    10  \\
Fe   &  26   &     10  &    18   &     10    &    10   &      18   &     -  &    10  \\
Co   &  27   &     10  &    18   &     10    &    10   &      18   &     -  &    10  \\
Ni   &  28   &     10  &    18   &     10    &    10   &      18   &     -  &    10  \\
Cu   &  29   &     10  &    18   &     10    &    10   &      18   &     -  &    10  \\
Zn   &  30   &     10  &    18   &     18    &    10   &      18   &    28  &    10 \\ \hline
Ga   &  31   &     10  &    28   &     28    &    18   &       -   &    28  &    -  \\
Ge   &  32   &     28  &    28   &     28    &    18   &       -   &    28  &    -  \\
As   &  33   &     28  &    28   &     28    &    18   &       -   &    28  &    -  \\
Se   &  34   &     28  &    28   &     28    &    18   &       -   &    28  &    -  \\
Br   &  35   &     28  &    28   &     28    &    18   &       -   &    28  &    -  \\
Kr   &  36   &     28  &    28   &     28    &    18   &       -   &    28  &    - \\ \hline
Rb   &  37   &     36  &    36   &     28    &    28   &       -   &    36  &    28  \\
Sr   &  38   &     36  &    36   &     28    &    28   &       -   &    36  &    28 \\ \hline
Y    &  39   &     28  &    36   &     28    &    28   &      36   &     -  &    28  \\
Zr   &  40   &     28  &    36   &     28    &    28   &      36   &     -  &    28  \\
Nb   &  41   &     28  &    36   &     28    &    28   &      36   &     -  &    28  \\
Mo   &  42   &     28  &    36   &     28    &    28   &      36   &     -  &    28  \\
Tc   &  43   &     28  &    36   &     28    &    28   &      36   &     -  &    28  \\
Ru   &  44   &     28  &    36   &     28    &    28   &      36   &     -  &    28  \\
Rh   &  45   &     28  &    36   &     28    &    28   &      36   &     -  &    28  \\
Pd   &  46   &     28  &    36   &     28    &    28   &      36   &     -  &    28 \\
Ag   &  47   &     28  &    36   &     28    &    28   &      36   &     -  &    28 \\
Cd   &  48   &     28  &    36   &     36    &    28   &      36   &     -  &    28 \\ \hline
In   &  49   &     28  &    46   &     46    &    36   &       -   &    46  &    -  \\
Sn - Xe  &  50 - 54  &     46  &    46   &     46    &    36   &       -   &    46  &    -  \\ \hline \hline
\end{tabular}

\end{centering}
\end{table}

\begin{table}
 \caption{\label{table:6}\ Library of Available ECPs with Number of Frozen-core Electrons (Cs-Lw)}
 
 \begin{centering}
 \begin{tabular}{lcccccccc}
 \\ \hline\hline
Element & Z & \multicolumn{7}{c}{ECP Library}\\
         \cline{3-9}
        &     &    SBKJC &  LANL  &    LANL2  &  CRENBL  &   CRENBS &  STRLC &   STRSC \\ \hline
Cs - Ba & 55 - 56 & 54 &    54   &     46    &    46   &       -   &    54  &    46 \\ \hline
La   &  57   &     46  &    54   &     46    &    46   &      54   &     -  &    - \\ \hline
Ce   &  58   &     46  &    -    &     -     &    54   &       -   &     -  &    28  \\
Pr   &  59   &     46  &    -    &     -     &    54   &       -   &     -  &    28  \\
Nd   &  60   &     46  &    -    &     -     &    54   &       -   &     -  &    28  \\
Pm   &  61   &     46  &    -    &     -     &    54   &       -   &     -  &    28  \\
Sm   &  62   &     46  &    -    &     -     &    54   &       -   &     -  &    28  \\
Eu   &  63   &     46  &    -    &     -     &    54   &       -   &     -  &    28  \\
Gd   &  64   &     46  &    -    &     -     &    54   &       -   &     -  &    28  \\
Tb   &  65   &     46  &    -    &     -     &    54   &       -   &     -  &    28  \\
Dy   &  66   &     46  &    -    &     -     &    54   &       -   &     -  &    28  \\
Ho   &  67   &     46  &    -    &     -     &    54   &       -   &     -  &    28  \\
Er   &  68   &     46  &    -    &     -     &    54   &       -   &     -  &    28  \\
Tm   &  69   &     46  &    -    &     -     &    54   &       -   &     -  &    28  \\
Yb   &  70   &     46  &    -    &     -     &    54   &       -   &     -  &    28  \\
Lu   &  71   &     46  &    -    &     -     &    54   &       -   &     -  &    - \\ \hline
Hf- Au & 72 - 79 & 60  &    68   &     60    &    60   &      68   &     -  &    60  \\
Hg   &  80   &     60  &    68   &     -     &    -    &       -   &    78  &    60 \\ \hline
Ll   &  81   &     60  &    68   &     -     &    -    &       -   &    78  &    -  \\
Pb   &  82   &     78  &    78   &     -     &    68   &      78   &    78  &    -  \\
Bi   &  83   &     78  &    78   &     -     &    68   &      78   &    78  &    -  \\
Po   &  84   &     78  &    -    &     -     &    68   &      78   &    78  &    -  \\
At   &  85   &     78  &    -    &     -     &    68   &      78   &    78  &    -  \\
Rn   &  86   &     78  &    -    &     -     &    68   &      78   &    78  &    -  \\
Fr   &  87   &     -   &    -    &     -     &    78   &       -   &     -  &    -  \\
Ra   &  88   &     -   &    -    &     -     &    78   &       -   &     -  &    - \\ \hline
Ac   &  89   &     -   &    -    &     -     &    78   &       -   &    78  &    60  \\
Th   &  90   &     -   &    -    &     -     &    78   &       -   &    78  &    60  \\
Pa   &  91   &     -   &    -    &     -     &    78   &       -   &    78  &    60  \\
U    &  92   &     -   &    -    &     78    &    78   &       -   &    78  &    60  \\
Np   &  93   &     -   &    -    &     78    &    78   &       -   &    78  &    60  \\
Pu   &  94   &     -   &    -    &     -     &    78   &       -   &    78  &    60  \\
Am   &  95   &     -   &    -    &     -     &    78   &       -   &    78  &    60  \\
Cm   &  96   &     -   &    -    &     -     &    78   &       -   &    78  &    60  \\
Bk   &  97   &     -   &    -    &     -     &    78   &       -   &    78  &    60  \\
Cf   &  98   &     -   &    -    &     -     &    78   &       -   &    78  &    60  \\
Es   &  99   &     -   &    -    &     -     &    78   &       -   &    78  &    60  \\
Fm   & 100   &     -   &    -    &     -     &    78   &       -   &    78  &    60  \\
Md   & 101   &     -   &    -    &     -     &    78   &       -   &    78  &    60  \\
No   & 102   &     -   &    -    &     -     &    78   &       -   &    78  &    60  \\
Lw   & 103   &     -   &    -    &     -     &    78   &       -   &    78  &    60  \\ \hline \hline
\end{tabular}

\end{centering}
\end{table}

The subsequent data input is, in general a function of whether the
non-local or semi-local ECPs are requested. We consider the data
input for each type below.

\subsubsection{Non-local ECPs}

When performing non-local treatments (TYPE=NONLOCAL) two
additional data fields may be presented to override the
default location of the Library File. These are read to
the variables LIBNAME, IBLLIB using format (A,I). LIBNAME should
be set to the LFN used to assign the Library File (ED0-ED19) and
IBLLIB to the starting block of the Library. If these two fields
are omitted, pseudo-orbital information will be input from
block 1 of the data set assigned using the LFN ED0.

Subsequent data lines, the `ECP definition' lines, are 
used to assign a specific ECP to centres specified in the
ZMATRIX or GEOMETRY directive. The first data field of such a line
is read to the variable ECPTAG in A-format, specifying the name or
TAG of an ECP within the Library. Subsequent data fields are also
read in A-format, and specify the unique TAGs of all centres
(as specified by the ZMATRIX or GEOMETRY) to be allocated ECPTAG.
The definition lines are continued until all centres to
be treated as valence-only have been allocated an appropriate ECP.

The following points should be noted.
\begin{itemize}
\item The standard ECPs as input from the library file
are labelled with just the
element name. Thus the TAG for the ECP for nickel is simply NI.
\item Presenting the data line
{
\footnotesize
\begin{verbatim}
          PSEUDO NONLOCAL ED0 1
\end{verbatim}
}
corresponds to the default when requesting a non-local treatment.
\end{itemize}
{\bf Example}\\

The following data sequence would be required in performing a non-local
ECP calculation on the \formaldehyde\ molecule:
{
\footnotesize
\begin{verbatim}
          TITLE
          H2CO - 1A1 - NON-LOCAL ECP CALCULATION
          ZMATRIX ANGSTROM
          C
          O 1 1.203
          H 1 1.099 2 121.8
          H 1 1.099 2 121.8 3 180.0
          END
          BASIS ECPDZ
          PSEUDO NONLOCAL
          O O
          C C
          ENTER 
\end{verbatim}
}
The following points should be noted:
\begin{itemize}
\item The non-local pseudopotential is requested on the PSEUDO
directive with the library ECPs for carbon and oxygen,
tagged C and O respectively , allocated 
to the unique nuclei tags specified in the z-matrix.
\item the ECPDZ tag on the BASIS line requests use of 
a double zeta contraction of the 
appropriate library of ECP basis sets  \cite{stevens,stevens2,cundari}.
\end{itemize}


\subsubsection{Local ECPs}

Data lines after the PSEUDO directive, the `ECP definition' lines,
are used to assign a specific ECP to centres specified in the ZMATRIX
or GEOMETRY directive. The first data field of such a line is read to
the variable ECPTAG in A-format, specifying the name or TAG of an ECP
within the Library. The second data field is the code name to identify
the required family of ECP, and should comprise one of the tags CEP
(or SBKJC), LANL, LANL2, CRENBL, CRENBS, STRLC or STRSC (see above).
Subsequent data fields are also read in A-format, and specify the unique
TAGs of all centres (as specified by the ZMATRIX or GEOMETRY) to be
allocated ECPTAG.  The definition lines are continued until all centres
to be treated as valence-only have been allocated an appropriate ECP.

Note that the standard ECPs within the nominated program-resident
library are labelled with just the element name. Thus the TAG for the
ECP for nickel is simply NI.

{\bf Example}\\

The following data sequence would be required in performing a local
ECP calculation on the \formaldehyde\ molecule using the CRENBL ECP for both
oxygen and carbon:
{
\footnotesize
\begin{verbatim}
          TITLE
          H2CO - 1A1 -  CRENBL ECP CALCULATION
          ZMATRIX ANGSTROM
          C
          O 1 1.203
          H 1 1.099 2 121.8
          H 1 1.099 2 121.8 3 180.0
          END
          BASIS 
          ECP C CRENBL
          ECP O CRENBL
          DZ H
          END
          PSEUDO 
          O CRENBL O
          C CRENBL C
          ENTER 
\end{verbatim}
}
The following points should be noted:
\begin{itemize}
\item The CRENBL library ECPs for carbon and oxygen,
tagged C and O respectively , are allocated 
to the unique nuclei tags specified in the z-matrix.
\item The BASIS directive defines the CRENBL basis set for
C and O, and a DZ basis on H.
\end{itemize}

{\bf Example 2}\\

The following data sequence would be required in performing a local
ECP geometry optimisation calculation on SiO using the SBKJC/CEP potentials
of \cite{stevens}. The basis sets have been augmented with d--functions.

{
\footnotesize
\begin{verbatim}
           TITLE
           SIO SBKJC-31G+D
           MULT 1
           ZMAT ANGSTROM\O\SI 1 SIO\VARIABLES\SIO 1.582\END
           BASIS
           ECP SI SBKJC
           D SI
           1.0 0.35
           ECP O SBKJC
           D O
           1.0 0.90
           END
           PSEUDO ECP
           SI SBKJC SI
           O SBKJC O
           RUNTYPE OPTIMIZE
           LEVEL  2.0
           MAXCYC 30
           ENTER 
\end{verbatim}
}
The local ECP is requested through the ECP field on the PSEUDO
directive: the subsequent data lines of this directive again
allocate the appropriate SBKJC ECP from the program resident library
to the atoms specified in the z-matrix through TAG specification. 

\subsubsection{Compatibility with Version 6.2 of GAMESS-UK}

Users familiar with previous versions of GAMESS-UK will have noted that
differences in the above format for presenting the local ECP data. Note that
the previous format for the CEP and LANL basis sets is, however, still supported.
The ECP definition lines for these two cases may be presented as
previously; the following points should be noted.
\begin{itemize}
\item The standard ECPs within the nominated program-resident library
are labelled with just the element name. Thus the TAG for the ECP for
nickel is simply NI.  The CEP pseudopotentials for the elements, Na--Rn,
may be TAGged with the three characters CEP appended to the element name,
so that the CEP TAG for the sulphur ECP is SCEP.
\end{itemize}

{\bf Example}\\

The following data sequence is equivalent to Example 2 above, and may
be used in performing a local ECP geometry optimisation calculation on
SiO using the CEP potentials of \cite{stevens}.  The basis sets have
been augmented with d--functions.

{
\footnotesize
\begin{verbatim}
           TITLE
           SIO CEP-31G+D
           MULT 1
           ZMAT ANGSTROM\O\SI 1 SIO\VARIABLES\SIO 1.582\END
           BASIS
           ECPDZ SI CEP
           D SI
           1.0 0.35
           ECPDZ O CEP
           D O
           1.0 0.90
           END
           PSEUDO ECP
           SICEP SI
           O O
           RUNTYPE OPTIMIZE
           LEVEL  2.0
           MAXCYC 30
           ENTER 
\end{verbatim}
}
The local ECP is requested through the ECP field on the PSEUDO directive:
the subsequent data lines of this directive again allocate an ECP from
the program resident library to the atoms specified in the z-matrix
through TAG specification.  The library ECP for silicon is tagged SICEP.
\section[Relativistic Calculations]{Relativistic Calculations}

Relativistic calculations with the Zeroth Order Regular Approximation
(ZORA) \cite{chang:1986,heully:1986,vanLenthe:1993,faas:1995,faas:2000}
are available. The implemented formalism can be used to treat both the
effects due to electron mass changes and the average spin-orbit
interaction (scalar effects) and that due to
spin-orbital coupling.

The formalism is effected by incorporating the relativistic components in the
1-electron integrals. This means that scalar relativistic calculations can be
performed within every formalism available in the program \cite{faas2:2000}. 

\subsection{ZORA}

The relativistic modifications of the 1-electron integrals are controlled 
through the ZORA directive. The ZORA directive may be followed by one or more
subdirectives. The ZORA directive may be repeated on various lines each time
with a selection the subdirectives. In this case the subdirectives are applied
in the sequence provided and accumulated settings will take effect in the 
calculation.

An important aspect of ZORA calculations is the requirement for an internal 
basis set to represent the relativistic kinetic energy~\cite{faas:2000}. This
internal basis set has to be complete to the extent that all couplings between
it and the AO basis set over the impulse operator are represented accurately, 
i.e. 
\begin{equation}
   \sum_\nu \langle \chi^{AO}_\mu | \vec{p} | \chi^{internal}_\nu \rangle
            \langle \chi^{internal}_\nu | 
   \approx \langle \chi^{AO}_\mu | \vec{p}
\end{equation}
The internal basis should also be able to represent the density accurately.
The program automatically generates this internal basis set by copying the
AO basis set and adding functions generated by applying the momentum operator
to primitive Gaussians of the AO basis set. This way a primitive function with
angular momentum $l$ and exponent $e$ contributes a function of angular momentum
$l+1$ and exponent $e$ to the internal basis. To avoid the internal basis 
becoming linearly dependent the new $l+1$ function is kept only if $e$ is higher
than any exponent of a $l+1$ primitive function already present in the AO basis.
However, the generation of this internal basis set may be modified by some
of the ZORA subdirectives described below.

Another important aspect is that the ZORA kinetic matrix involves the inverse
of the Coulomb matrix represented in the internal basis \cite{faas:2000}. 
In principle this means that this Coulomb matrix should be computed and inverted
in every SCF cycle. In practice various approximations are available, the selection
of which may be controlled by some of the ZORA subdirectives.

If the ZORA directive appears without any subdirectives then
{
\footnotesize
\begin{verbatim}
     ZORA LIMIT G SCALE ON GET ATOM COULOMB ATOM EXCHANGE OFF 
     ZORA NITERATION 100 INTERNAL AUTO C 137.0359895 
\end{verbatim}
}
is used. 
This invokes a strictly atomic ZORA approach~\cite{vanLenthe:2000}. 
This is in general a superior approximation, although it might be
less suitable in cases, where the molecular charge distribution 
deviates significantly (near the nuclei) from the atomic charge
densities, because it
\begin{itemize}
\item is free from the problems with gauge invariance, 
      that plague normal ZORA.
\item yields exact gradients and second derivatives.
\end{itemize}
The atomic scaling is realised through an effective
one-electron operator (like the ZORA corrections themselves). 

The available subdirectives are described below.

\subsection{SPIN}

The subdirective SPIN should be used to request inclusion of spin-orbit coupling
terms in the ZORA formalism \cite{faas3:2000}. This involves a switch to a complex 2-component formalism. 
It is currently only available at the Hartree-Fock level and
gradients have not been tested.

\subsection{MOLECULAR}
Switches from the strictly atomic approximation (the default) 
to a molecular ZORA.
A  normal scalar non-approximate scaled ZORA is thus obtained by using
{
\footnotesize
\begin{verbatim}
     ZORA MOLECULAR
\end{verbatim}
}
The subdirective MOLECULAR is equivalent to
{
\footnotesize
\begin{verbatim}
     GET NONE COULOMB FULL SCALE ON
\end{verbatim}
}
noting that the scaling is now molecular in nature.

\subsection{SCALE}

The ZORA equation differs from the Hartree-Fock equation in the form of the
kinetic energy operator but also in the presence of a scale factor for the
equation for each orbital (see \cite{faas:2000} Eq.(7)).
The SCALE subdirective can be used to choose between different choices for 
calculating the scale factors in the ZORA 1-electron equations.
It is read to the variables TEXT, TEXTA using the format (2A),
where
\begin{itemize}
\item TEXT should be set to the string SCALE
\item TEXTA can be set to 
    \begin{itemize}
    \item ON (default)  in which case the scale factors will be 
          evaluated ; The scaling will be either strictly atomic 
          (if GET ATOM, the default, is in effect), or applied to 
          the molecular SCF orbital energies. (ZORA MOLE)
    \item OFF in which case the scale factors are assumed to be 1.
    \item IORA in which case the metric will be changed to perform infinite
          order scaling as proposed by Kenneth Dyall and Erik van Lenthe
          (in an experimental and not the published form).
    \item MOLECULAR in which case scaling is applied to the converged
          molecular orbital energies even if GET ATOM (the default) 
          is in effect. This is also the default behaviour for molecular
          scaled ZORA calculations. *Note* that this scaling spoils
          the exactness of gradients, even for GET ATOM.
    \end{itemize}
\end{itemize}

\subsection{Directives controlling the calculation of the Coulomb matrix}

\subsubsection{GET}

The GET subdirective is used to determine the density, which is used 
to calculate the Coulomb matrix needed in the ZORA formalism. This
density may be read from the dumpfile (DENSITY) or calculated from
natural orbitals (VECTORS), or the densities of the constituent
atoms may be used in an by itself iterative ZORA procedure as is done
in the default "GET ATOM' behaviour.
If GET is not specified in a molecular ZORA calculation (ZORA MOLECULAR),
the density of the Hartree-Fock calculation in that iteration is used
(cf NITERATION).

The directive is read to the variables TEXT, TEXTA, ISECT using the format
(2A,I), where
\begin{itemize}
\item TEXT should be set to the string GET
\item TEXTA should be set to either 
    \begin{itemize}
    \item VECTORS in which case ISECT should be set to a vectors dumpfile 
          section
    \item DENSITY is which case ISECT should be set to a density dumpfile 
          section
    \item ATOM (which shouldn't be followed by an integer)
    \item ATOI in which case NIT should be provided, giving the 
          number of atomic zora scfs to be performed
          (default (ATOM) results in NIT=3)
    \item NONE (which shouldn't be followed by an integer and resets get)
    \end{itemize}
\end{itemize}

\subsubsection{COULOMB}

The COULOMB subdirective should be used to choose among the various possible 
approximations for the Coulomb matrix in the ZORA kinetic energy 
expression~\cite{faas:2000}. In the default case only ATOMIC and NONE
have a meaning. The subdirective is read to the variables TEXT, 
TEXTA using the format (2A), where
\begin{itemize}
\item TEXT should be set to the string COULOMB
\item TEXTA can be set to 
    \begin{itemize}
    \item FULL in which case the complete Coulomb matrix in the internal basis 
               will be used without any approximations.
    \item ATOMIC in which case only intra-atomic components of the Coulomb 
               matrix are taken into account. 
    \item NONE in which case all 2-electron contributions to the Coulomb matrix
               will be neglected (bare nuclei).
    \item SMALL in which case the Coulomb matrix will be evaluated in the AO
               basis and subsequently projected onto the internal basis before
               inverting it.
    \end{itemize}
\end{itemize}

\subsubsection{EXCHANGE}

It is read to the variables TEXT, TEXTA using the format (2A),
where
\begin{itemize}
\item TEXT should be set to the string EXCHANGE
\item TEXTA can be set to 
    \begin{itemize}
    \item OFF (default): No exchange is included in the Coulomb potential
    \item ON : Exchange is included in the Coulomb potential (not justifiable)
    \end{itemize}
\end{itemize}

\subsubsection{NITERATION}

In principle the Coulomb matrix in the internal basis should be computed 
every SCF cycle. In practice it turns out to be sufficient to compute it far 
less often and assume it to be constant for a number of SCF cycles.
The NITERATION subdirective can be used to set the maximum number of SCF cycles
between recomputing the Coulomb matrix in the internal basis.
The directive is read to the variables TEXT, NITER using the format (A,I), where
\begin{itemize}
\item TEXT should be set to the string NITERATION
\item NITER should be set to the number of iterations required
\end{itemize}

\subsection{Directives controlling the internal basis}

\subsubsection{NOCOPY}

The subdirective NOCOPY is read to variable TEXT using the format (A).
If given it signifies, that the internal basis is to be generated, 
without copying the external basis first.

\subsubsection{INTERNAL}

The INTERNAL subdirective should be used to select a mechanism for defining
the internal basis. The subdirective INTERNAL is read to the variables 
TEXT, TEXTA using format (2A).
\begin{itemize}
\item TEXT should be set to the string INTERNAL
\item TEXTA should be set to either
    \begin{itemize}
    \item AUTO in which case the internal basis will be constructed 
          automatically.
    \item NONE or OFF in which case no internal basis will be used.
    \item MANUAL in which case the internal basis should be entered explicitly.
    \end{itemize}
\end{itemize}

\subsubsection{LIMIT}

The LIMIT subdirective can be used to limit the maximum angular momentum 
of the functions included in the internal basis in addition to the AO basis
functions. 

The subdirective is read to the variables TEXT, TEXTA using the format
(2A). Where
\begin{itemize}
\item TEXT should be set to the string LIMIT
\item TEXTA should be one of the strings S, P, D, or F.
\end{itemize}

{\bf Example}: Consider an AO basis containing S and P functions, then
by default the internal basis set would consist of the AO basis
functions plus newly generated P and D functions. In this case one
could use LIMIT P to restrict the additional functions to P functions
only and suppress the generation of the D functions.

\subsubsection{CONTRACTED}

The CONTRACTED subdirective attempts to create contracted $l+1$ 
angular momentum basis functions to extend the internal basis set 
instead of creating uncontracted ones. This should help to limit 
the size of the internal basis.

\subsection{Miscellaneous}

\subsubsection{PRINT}

The PRINT subdirective can be used to turn on additional printing.
Its probably most important effect is to print the internal ZORA basis which
used to represent the inverse of the Coulomb matrix~\cite{faas:2000}.

\subsubsection{PRI2}

The PRI2 subdirective can be used to invoke a debug level of output for
ZORA related entities.

\subsubsection{C}

The C subdirective can be used to explicitly specify the speed of light in 
atomic units. It is read to the variables TEXT, LIGHTSPEED using the format
(A,F), where
\begin{itemize}
\item TEXT should be set to the character C
\item LIGHTSPEED is the light speed in atomic units
\end{itemize}
By default the light speed will be set to 137.0359895 a.u.

\subsubsection{GAUGE}

The subdirective GAUGE allows one to investigate gauge-invariance errors in the
ZORA formalism. Using the subdirective results in adding IVAL times the 
overlap matrix to the potential. If ZORA was exactly gauge-invariant this would
result in a change of the total energy by IVAL times the number of electrons.
The deviation from this gives an indication of the gauge-variation error in
ZORA.
The subdirective GAUGE is read to the variables TEXT, IVAL using the format 
(A,I), where
\begin{itemize}
\item TEXT should be set to the string GAUGE
\item IVAL should be set to an integer, which is used to add IVAL/NEL
      time the S-matrix to the 1-electron (h) matrix, where NEL is the
      number of electrons. If a method is GAUGE-invariant, this should
      result in an energy shift of exactly IVAL Hartree.
\end{itemize}
This directive might be handy for non-ZORA calculations also, to test
the Gauge-invariance.

\subsubsection{OFF}

The subdirective OFF may be used to switch the use of the ZORA 
formalism off explicitly.

\clearpage

\begin{thebibliography}{10}

\bibitem{schlegel}
H.B. Schlegel, J. Chem. Phys. {\bf 77} (1982) 3676, \doi{10.1063/1.444270}.

\bibitem{hehre}
W.J. Hehre, R.F. Stewart and J.A. Pople, 
  J. Chem. Phys. {\bf 51} (1969) 2657, \doi{10.1063/1.1672392};
W.J. Hehre, R. Ditchfield, R.F. Stewart and J.A. Pople,
  J. Chem. Phys {\bf 52} (1970) 2769, \doi{10.1063/1.1673374};
W.J. Pietro and W.J. Hehre,
  J. Comp. Chem. {\bf 4} (1983) 241, \doi{10.1002/jcc.540040215};
J.B. Collins, P.van R. Schleyer, J.S. Binkley and J.A. Pople,
  J. Chem. Phys. {\bf 64} (1976) 5142, \doi{10.1063/1.432189};
H. Tatewaki and S. Huzinaga,
  J. Chem. Phys.  {\bf 72} (1980) 399, \doi{10.1063/1.438863}.

\bibitem{huzinaga} S. Huzinaga, J. Andzelm, M. Klobukowski, 
E. Radzio-Andzelm,
Y. Sakai and  H. Tatewaki, `Gaussian Basis Sets for 
Molecular Calculations', Elsevier, Amsterdam, 1984.

\bibitem{dunning} T.H. Dunning Jr. and P.J. Hay in 
`Modern Theoretical Chemistry',
Vol. 3, ed. H.F. Schaefer, Plenum, New York (1977) 1.;
T.H. Dunning Jr., J. Chem. Phys. {\bf 66} (1977) 1382, \doi{10.1063/1.434039}.

\bibitem{binning}
R.C. Binning and L.A. Curtiss, 
  J. Comp. Chem. {\bf 11} (1990) 1206--1216, \doi{10.1002/jcc.540111013}.

\bibitem{binkley}
J.S. Binkley, J.A. Pople and W.J. Hehre, 
  J. Am. Chem. Soc., {\bf 102} (1980) 939, \doi{10.1021/ja00523a008};
M.S. Gordon, J.S. Binkley, J.A. Pople, W.J. Pietro and W.J. Hehre,
  J. Am. Chem. Soc., {\bf 104} (1982) 2797, \doi{10.1021/ja00374a017};
M.J. Frisch, J.A. Pople and J.S. Binkley,
  J. Chem. Phys. 80 (1984) 3265, \doi{10.1063/1.447079},
  and references cited therein;
K.D. Dobbs and W.J. Hehre,
  J. Comp. Chem. {\bf 7} (1986) 359--378, \doi{10.1002/jcc.540070313};
  {\em ibid} {\bf 8} (1987) 861--879, \doi{10.1002/jcc.540080614},
  880--893, \doi{10.1002/jcc.540080615}.

\bibitem{schafer}
A. Schafer, H. Horn and R. Ahlrichs,
  J. Chem. Phys. {\bf 97} (1992) 2571, \doi{10.1063/1.463096}.

\bibitem{dunning1}
T.H. Dunning Jr., J. Chem. Phys. {\bf 53} (1970) 2823, \doi{10.1063/1.1674408}.

\bibitem{dunning2}
T.H. Dunning Jr., J. Chem. Phys. {\bf 55} (1971) 716, \doi{10.1063/1.1676139}.

\bibitem{mcclean}
A.D. McLean and G.S. Chandler, 
  J. Chem. Phys. {\bf 72} (1980) 5639, \doi{10.1063/1.438980}.

\bibitem{hay}
P.J. Hay and W.R. Wadt, J. Chem. Phys. {\bf 82} (1985)
270 \doi{10.1063/1.448799},
284 \doi{10.1063/1.448800},
299 \doi{10.1063/1.448975}.

\bibitem{stevens}
W.J. Stevens, H. Basch and M. Krauss,
J. Chem. Phys. {\bf 81} (1984) 6026, \doi{10.1063/1.447604}.

\bibitem{stevens2}
W.J. Stevens, P.G. Jasien, M. Krauss, and H. Basch,
  Can. J. Chem. {\bf 70} (1992) 612, \doi{10.1139/v92-085}.

\bibitem{cundari}
T.R. Cundari and W.J. Stevens, J. Chem. Phys. {\bf 98} (1993) 5555,
\doi{10.1063/1.464902}.

\bibitem{christiansen} (H) - T.H. Dunning, Jr. and P.J. Hay, Methods of
Electronic Structure Theory, Vol. 3, H. F.  Schaefer III, Ed. Plenum
Press (1977);
(Li-Ne, Na-Ar) - L. F. Pacios and P. A. Christiansen,
 J. Chem. Phys.  {\bf 82} (1985) 2664, \doi{10.1063/1.448263};
(K-Ca, Sc-Zn, Ga-Kr) - M. M. Hurley et al.
 J. Chem. Phys. {\bf 84} (1986) 6840, \doi{10.1063/1.450689};
(Rb-Sr,Y-Cd, In) - L. A. LaJohn et al.
 J. Chem. Phys., {\bf 87} (1987) 2812, \doi{10.1063/1.453069};
(Xe) - M. M. Hurley et al.,
 J. Chem. Phys. {\bf 84} (1986) 6840, \doi{10.1063/1.450689};
(Cs, La, Hf-Hg, Tl-Rn) - R.B. Ross, W.C. Ermler, P.A. Christiansen et al.
 J. Chem. Phys. {\bf 93} (1990) 6654, \doi{10.1063/1.458934},
 erratum: \doi{10.1063/1.468517};
(Ba, Ce-Lu) - R.B. Ross, W.C. Ermler, S. Das, Unpublished;
(Fr-Ra, Ac-Pu) - W.C. Ermler, R.B. Ross, P.A. Christiansen,
 Int. J. Quant. Chem {\bf 40} (1991) 829, \doi{10.1002/qua.560400611}.

\bibitem{ermler}
(Sc - Co, Cu-Zn) - M.M. Hurley et al.,
 J. Chem. Phys., {\bf 84} (1986) 6840, \doi{10.1063/1.450689};
(Ni, Y-Cd) -  L.A. LaJohn et al.,
 J. Chem. Phys., {\bf 87} (1987) 2812, \doi{10.1063/1.453069};
(La, Hf-Hg, Tl-Tn) - R.B. Ross, W.C. Ermler, P.A. Christiansen et al.
 J. Chem. Phys. {\bf 93} (1990) 6654, \doi{10.1063/1.458934},
 erratum: \doi{10.1063/1.468517}.

\bibitem{preuss1}
(Li - Be, Na) - P. Fuentealba, H. Preuss, H. Stoll, L. v. Szentpaly,
  Chem. Phys. Lett.  {\bf 89} (1982) 418, \doi{10.1016/0009-2614(82)80012-2};
(B-Ne) -  A. Bergner, M. Dolg, W. K\"uchle, H. Stoll, H. Preuss,
  Mol. Phys. {\bf 80} (1993) 1431, \doi{10.1080/00268979400100024};
(Mg) - P. Fuentealba, L. v. Szentpaly, H. Preuss, H. Stoll,
  J. Phys. B {\bf 18} (1985) 1287, \doi{10.1088/0022-3700/18/7/010};
(Al) - G. Igel-Mann, H. Stoll, H. Preuss,
  Mol. Phys. {\bf 65} (1988) 1321, \doi{10.1080/00268978800101811};
(Hg-Rn) -  W. K\"uchle, M. Dolg, H. Stoll, H. Preuss,
  Mol. Phys. {\bf 74} 1245 (1991), \doi{10.1080/00268979100102941};
(Ac-Lr) W. K\"uchle, M. Dolg, H. Stoll, H. Preuss,
  J. Chem. Phys. {\bf 100} (1994) 7535, \doi{10.1063/1.466847}.


\bibitem{preuss2} 
(K) - A. Bergner, M. Dolg, W. K\"uchle, H. Stoll, H. Preuss, 
  Mol. Phys. {\bf 80} (1993) 1431, \doi{10.1080/00268979400100024};
(Ca) - M. Kaupp, P. v. R. Schleyer, H. Stoll, H. Preuss,
  J. Chem. Phys. {\bf 94} (1991) 1360, \doi{10.1063/1.459993};
(Rf - Db) - M. Dolg, H. Stoll, H. Preuss, R.M. Pitzer,
  J. Phys. Chem. {\bf 97} (1993) 5852, \doi{10.1021/j100124a012}.

\bibitem{dunning3}
T.H. Dunning, Jr.,
  J. Chem. Phys. {\bf 90} (1989) 1007, \doi{10.1063/1.456153};
D.E. Woon and T.H. Dunning, Jr.,
  J. Chem. Phys. {\bf 98} (1993) 1358, \doi{10.1063/1.464303};
D.E.  Woon and T.H. Dunning, Jr.,
  J. Chem. Phys. {\bf 100} (1994) 2975, \doi{10.1063/1.466439};
A.K. Wilson, D.E. Woon, K.A. Peterson, T.H. Dunning, Jr.,
  J. Chem. Phys. {\bf 110} (1999) 7667, \doi{10.1063/1.478678}.

\bibitem{godbout}
N. Godbout, D. R. Salahub, J. Andzelm and E. Wimmer,
  Can. J. Chem. {\bf 70} (1992) 560, \doi{10.1139/v92-079}.

\bibitem{ahlrichs}
R. Ahlrichs and P.R. Taylor, 
J. Chim. Phys., 78 (1981) 315.
% This journal ceased to exist in 1999.

\bibitem{durand}
Ph. Durand and J.-C.Berthelat,
  Theoret. Chim. acta, {\bf 38} (1975) 283, \doi{10.1007/BF00963468}.

% \bibitem{ref:36} D.M. Hood, R.M. Pitzer and H.F. Schaefer, 
% J. Chem. Phys. 71 705.

\bibitem{chang:1986}
Ch. Chang, M. Pelissier, and Ph. Durand,
  Phys. Scr. {\bf 34} (1986) 394, \doi{10.1088/0031-8949/34/5/007}.

\bibitem{heully:1986}
J.-L. Heully, I. Lindgren, E. Lindroth, S. Lundqvist,
  and A.-M. M{\aa}rtensson-Pendrill,
  J. Phys. {\bf B19} (1986) 2799, \doi{10.1088/0022-3700/19/18/011}.

\bibitem{vanLenthe:1993}
E. van Lenthe, E.J. Baerends, and J.G. Snijders,
  J. Chem. Phys. {\bf 99} (1993) 4597, \doi{10.1063/1.466059}.

\bibitem{faas:1995}
S. Faas, J.G. Snijders, J.H. van Lenthe, E. van Lenthe, and E.J. Baerends,
  Chem. Phys. Lett. {\bf 246} (1995) 632, \doi{10.1016/0009-2614(95)01156-0}.

\bibitem{dyall:1999}
Kenneth G. Dyall, and Erik van Lenthe,
  J. Chem. Phys. {\bf 111} (1999) 1366, \doi{10.1063/1.479395}.

\bibitem{faas:2000} S. Faas, J.G. Snijders, and J.H. van Lenthe,
  {\it Scalar Relativistic Ab Initio ZORA Calculations} in: 
  {\it Quantum Systems in Chemistry and Physics}
  Vol. 1: {\it Basic Problems and Model Systems}, page 251
  Kluwer Academic Publishers (2000).

\bibitem{vanLenthe:2000}
J.H. van Lenthe, S. Faas and J.G. Snijders,
  Chem. Phys. Lett. {\bf 328} (2000) 107, \doi{10.1016/S0009-2614(00)00832-0}.

\bibitem{faas2:2000} S. Faas, J.H. van Lenthe and J.G. Snijders,
  Mol. Phys. {\bf 98} (2000) 1467--1472, \doi{10.1080/002689700417574}.

\bibitem{faas3:2000} S.Faas, J.H. van Lenthe, A.C. Hennum and J.G. Snijders,
  J. Chem. Phys. {\bf 113} (2000) 4052, \doi{10.1063/1.1288387}.

\end{thebibliography}

\end{document}
