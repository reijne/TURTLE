\documentclass[11pt,fleqn]{article}

\usepackage{hyperref}

% package HTML requires Latex2HTML to be installed for html.sty
\usepackage{html}
\newcommand{\doi}[1]{doi:\href{http://dx.doi.org/#1}{#1}}
\begin{htmlonly}
\renewcommand{\href}[2]{\htmladdnormallink{#2}{#1}}
\end{htmlonly}
\hypersetup{colorlinks,
            %citecolor=black,
            %filecolor=black,
            %linkcolor=black,
            %urlcolor=black,
            bookmarksopen=true,
            pdftex}
 
\addtolength{\textwidth}{1.0in}
\addtolength{\oddsidemargin}{-0.5in}
\addtolength{\topmargin}{-0.5in}
\addtolength{\textheight}{1.0in}
\newcommand{\water}{\mbox{H$_{2}$O}}
\newcommand{\waterp}{\mbox{H$_{2}$O$^{+}$}}
\newcommand{\namgplus}{\mbox{Na$_{7}$Mg$^{+}$}}
\newcommand{\nitrobz}{\mbox{C$_{6}$H$_{5}$NO$_{2}$}}
\newcommand{\tnt}{\mbox{C$_{6}$H$_{2}$(NO$_{2}$)$_{3}$CH$_{3}$}}
\newcommand{\ammonia}{\mbox{NH$_{3}$}}
\newcommand{\formaldehyde}{\mbox{H$_{2}$CO}}
\newcommand{\nico}{\mbox{Ni(CO)$_{4}$}}
\newcommand{\fcubane}{\mbox{C$_{4}$F$_{4}$}}
\newcommand{\nimeth}{\mbox{NiCCH$_{2}$}}
\newcommand{\ethene}{\mbox{C$_{2}$H$_{4}$}}
\newcommand{\pyridine}{\mbox{C$_{5}$H$_{5}$N}}

\pagestyle{headings}
\pagenumbering{roman}
\begin{document}
\sf
\parindent 0cm
\parskip 1ex
\begin{flushleft}
 
Computing for Science (CFS) Ltd.,\\CCLRC Daresbury Laboratory.\\[0.30in]
{\large Generalised Atomic and Molecular Electronic Structure System }\\[.2in]
\rule{150mm}{3mm}\\
\vspace{.2in}
{\huge G~A~M~E~S~S~-~U~K}\\[.3in]
{\huge USER'S GUIDE~~and}\\[.2in]
{\huge REFERENCE MANUAL}\\[0.2in]
{\huge Version 8.0~~~June 2008}\\ [.2in]
{\large PART 12. GAMESS-UK under UNIX}\\
\vspace{.1in}
{\large M.F. Guest, J. Kendrick, J.H. van Lenthe and P. Sherwood}\\[0.2in]
 
Copyright (c) 1993-2008 Computing for Science Ltd.\\[.1in]
This document may be freely reproduced provided that it is reproduced\\
unaltered and in its entirety.\\
\vspace{.2in}
\rule{150mm}{3mm}\\
\end{flushleft}
 
\tableofcontents

\pagenumbering{arabic}
\newpage


\section[Introduction]{Introduction}

In the present section we consider initially  execution of GAMESS--UK
on a Linux workstation, although these notes are intended as a
more general guide to running the code under the Unix operating
system. Throughout this description we make the following assumptions:

\begin{itemize}
\item the user will in general submit jobs from a
disk partition different to that where the direct access files
used by GAMESS--UK will be cited: typically the latter will be striped. 
For the purpose of the present notes we assume that the various files
involved will be routed to the partition /scr1/user, where the
directory 'user' has been created prior to job submission (using
the command {\em mkdir}), thus
{
\footnotesize
\begin{verbatim}
          mkdir /scr1/user
\end{verbatim}
}

\item  that all job inputs reside in the
user's home directory (or subdirectory thereof), and that all
job output is to be routed back to that directory.
\item We assume the user is submitting jobs directly to the
background. 
\item In Part 13  we discuss execution of GAMESS--UK
using the prepared script {\em rungamess}
\item all of the commands in these notes belong to the C shell.
\end{itemize}
Note that  all data input for the Unix implementation of GAMESS--UK
may be presented
in either {\bf upper} or {\bf lower case}. In the examples below 
we have used {\em lower case} input throughout. Note that it is
a trivial matter to convert data from under control of the {\em dd}
command. Thus if upper case data input resides in the file {\bf upper}, 
the following command\\

 dd~~if={\bf upper}~~of={\bf lower}~~conv=lcase\\

will perform the necessary conversion to leave the required lower
case input in the file {\bf lower}. Note that the executable for GAMESS--UK
is assumed to reside in the file /scr1/wab/GAMESS-UK/gamess.
The following examples will be considered:

\begin{enumerate}
\item Single point SCF calculation of water using the default basis.
\item Sequence of calculations on water, including starting and
restarting geometry optimisation of the neutral molecule,
an RHF calculation on the $^{2}$B$_{1}$ cation, boys localisation
and 2-pair GVB calculation of the neutral molecule. 
\item Closed-shell SCF calculations on \fcubane, \nitrobz and \tnt.
\item STO-3G calculation of \namgplus, together with an RHF and UHF
calculation of the triplet state.
\item Extended basis set calculation on \namgplus, with SCF calculation
of the singlet state preceding an RHF calculation of the triplet state.
\item ECP calculation on \namgplus in a double-zeta valence
basis, with SCF calculation
of the singlet state preceding an RHF calculation of the triplet state.
\item 2-pair GVB calculation on 
1-imino-2,4-pentadiene, CH$_{2}$(CH)$_{4}$NH$_{2}^{+}$
\item Graphical analysis of the X$^{1}$A$_{1}$ state of \nico.
\item STO-3G direct-SCF calculations on C$_{11}$O$_{10}$NPH$_{18}$
and the geometry optimisation of Be(C$_{5}$H$_{5}$)$_{2}$.
\item Determination of the transition state for the HCN/HNC
isomerisation reaction. This example demonstrates usage of the 
three saddle point algorithms available, the default trust-region
method plus the synchronous transit and Simons-Jorgensen algorithms.
\item Location of the transition state for the HPSi/HSiP
isomerisation and subsequent force constant evaluation.
\item Specification of bond-centred functions, located at the
midpoint of the C-N bond in the HCN/HNC transition state.
\item SCF geometry optimisation and analytic force constants
of \ethene.
\item MP2 geometry optimisation and analytic force constants
of \ethene.
\item MP2 geometry optimisation and polarisability of \ethene.
\item Direct-MP2 calculation of \pyridine\ in a 6-31G* basis.
\item CASSCF geometry optimisation of the X$^{1}$A$_{1}$ state of \water.
\item CASSCF + 2nd-order CI calculations of the X$^{1}\Sigma^{+}$ 
state of BeO.
\item MCSCF + 2nd-order CI calculations of the X$^{1}\Sigma^{+}$ 
state of BeO.
\item  Table-CI calculation of the X$^{2}$A$_{1}$ and 1$^{2}$A$_{1}$ 
states of the ammonia cation, NH$_{3}^{+}$ cation.
\item  ECP calculation on \nimeth\, with CASSCF and Direct-CI
calculations of the lowest triplet state wavefunction.
\item  Table-CI calculations typical of those
performed in the calculation of electronic spectra.
In this case we are studying the disposition of the lowest
$^{1}$A$_{1}$ and 1$^{1}$A$_{2}$ states of pyridine.
The sequence of calculations involve the determination of the
lowest 10 states of each category, performed in a DZ plus rydberg
basis.
\item Full-CI calculations of  the X$^{1}$A$_{1}$ state of \water.
\end{enumerate}

\section[Single Point SCF Calculation]{Single Point SCF Calculation}

Let us consider a simple SCF run for \water\ in a 3-21G basis,
and describe the various mechanisms available for running the
job in terms of the citing of direct access files involved and
the routing of job output. Note at the outset that GAMESS--UK reads from
{\em stdin} and writes to {\em stdout}, so that both input and 
output may be controlled by the standard means.
\begin{itemize}
\item Let us assume that the following lines reside in the
file myjob,
{
\footnotesize
\begin{verbatim}
          /scr1/wab/GAMESS-UK/bin/gamess << EOF
          title\h2o 3-21g scf
          zmat angstrom\o\h 1 roh\h 1 roh 2 theta
          variables\roh 0.956 hess 0.7\theta 104.5 hess 0.2 \end
          enter
          EOF
\end{verbatim}
}
and are submitted to the background through the command
{
\footnotesize
\begin{verbatim}
          myjob >& myjob.log &
\end{verbatim}
}
This will result in the files used by GAMESS--UK, in this case ED2, ED3 and
ED7 being created in the user's own directory for the duration of the
run, and deleted on job termination. The output from the job and any
system diagnostics will be routed to the file myjob.log. Note that 
the user may  assign execute attributes to the file myjob,
using the {\em chmod} command, thus

{
\footnotesize
\begin{verbatim}
          chmod 700 myjob
\end{verbatim}
}
\item Note that
it is also possible to read the data from a separate file, so that
myjob would comprise the single line

{
\footnotesize
\begin{verbatim}
          /scr1/wab/GAMESS-UK/bin/gamess < datain
\end{verbatim}
}
where datain contains just the GAMESS--UK data, thus
{
\footnotesize
\begin{verbatim}
          title\h2o 3-21g scf
          zmat angstrom\o\h 1 roh\h 1 roh 2 theta
          variables\roh 0.956 hess 0.7\theta 104.5 hess 0.2 \end
          enter
\end{verbatim}
}
In this case the job might equally well be submitted directly with
the following command,
{
\footnotesize
\begin{verbatim}
          /scr1/wab/GAMESS-UK/bin/gamess < datain >& myjob.log &
\end{verbatim}
}
\item Now consider the changes to myjob required to route the
direct access files from the user's own directory to the
disk partition /scr1, in particular to the directory
/scr1/user. Assuming this directory has been created by the user, thus

{
\footnotesize
\begin{verbatim}
          mkdir /scr1/user
\end{verbatim}
}
this may be achieved either by using the {\em cd} command
at the outset of myjob, so as execute the job in the required 
directory, thus

{
\footnotesize
\begin{verbatim}
          cd /scr1/user
          /scr1/wab/GAMESS-UK/bin/gamess << EOF
          title\h2o 3-21g scf
          zmat angstrom\o\h 1 roh\h 1 roh 2 theta
          variables\roh 0.956 hess 0.7\theta 104.5 hess 0.2 \end
          enter
          EOF
\end{verbatim}
}
in which case the direct access files will be deleted on job 
termination, or through the {\em setenv} command, thus

{
\footnotesize
\begin{verbatim}
          #!/bin/csh -f
          setenv ed2 /scr1/user/ed2
          setenv ed3 /scr1/user/ed3
          setenv ed7 /scr1/user/ed7
          /scr1/wab/GAMESS-UK/bin/gamess << EOF
          title\h2o 3-21g scf
          zmat angstrom\o\h 1 roh\h 1 roh 2 theta
          variables\roh 0.956 hess 0.7\theta 104.5 hess 0.2 \end
          enter 
          EOF
\end{verbatim}
}
In this case the files created in /scr1/user {\em will not} be
deleted on job termination.
\end{itemize}
This provides, of course, the mechanism for keeping files between
separate runs of the program, with the file names assigned though
the {\em setenv} values. Thus to keep the
mainfile and dumpfile with file names h2omain and h2odump would
require the following job input:

{
\footnotesize
\begin{verbatim}
          #!/bin/csh -f
          setenv ed2 /scr1/user/h2omain
          setenv ed3 /scr1/user/h2odump
          setenv ed7 /scr1/user/ed7
          /scr1/wab/GAMESS-UK/bin/gamess << EOF
          title\h2o 3-21g scf
          zmat angstrom\o\h 1 roh\h 1 roh 2 theta
          variables\roh 0.956 hess 0.7\theta 104.5 hess 0.2 \end
          enter 
          EOF
\end{verbatim}
}
or, if using the {\em cd} command,
{
\footnotesize
\begin{verbatim}
          #!/bin/csh -f
          cd /scr1/user
          setenv ed2 h2omain
          setenv ed3 h2odump
          /scr1/wab/GAMESS-UK/bin/gamess << EOF
          title\h2o 3-21g scf
          zmat angstrom\o\h 1 roh\h 1 roh 2 theta
          variables\roh 0.956 hess 0.7\theta 104.5 hess 0.2 \end
          enter 
          EOF
\end{verbatim}
}
with the scratchfile (ed7) omitted from the {\em setenv} commands.

Alternatively, instead of setting the names of the files through the
setenv command, it is possible to use ''file'' pre-directives within the
body of the GAMESS-UK input. Thus the following:

{
\footnotesize
\begin{verbatim}
          #!/bin/csh -f
          cd /scr1/user
          /scr1/wab/GAMESS-UK/bin/gamess << EOF
          title\h2o 3-21g scf
          file ed2 h2omain
          file ed3 h2odump
          zmat angstrom\o\h 1 roh\h 1 roh 2 theta
          variables\roh 0.956 hess 0.7\theta 104.5 hess 0.2 \end
          enter 
          EOF
\end{verbatim}
}
achieves the same as using the {\em setenv} command. For all the
following examples, either {\em setenv} commands run in the shell, or
{\em file} directives in the GAMESS-UK input can be used interchangably.

\section[Sequence of Calculations on Water]{Sequence of Calculations on Water}

In this example we include a sequence of job files for
performing various calculations based on the \water\ example given above.
\begin{enumerate}
\item This run utilises the vectors made above. Both bond length and
bond angle are to be optimised.

{
\footnotesize
\begin{verbatim}
          #!/bin/csh -f
          setenv ed2 /scr1/user/h2omain
          setenv ed3 /scr1/user/h2odump
          setenv ed7 /scr1/user/ed7
          /scr1/wab/GAMESS-UK/bin/gamess << EOF
          restart new
          title
          water optimisation at scf level 3-21g basis set
          zmat angstrom
          o
          h 1 oh
          h 1 oh 2 hoh
          variables
          oh 0.956
          hoh 104.5
          end
          runtype optimise
          enter
          EOF
\end{verbatim}
}
\item This example would be used
to complete, if necessary, the run started in  1.
{
\footnotesize
\begin{verbatim}
          #!/bin/csh -f
          setenv ed2 /scr1/user/h2omain
          setenv ed3 /scr1/user/h2odump
          setenv ed7 /scr1/user/ed7
          /scr1/wab/GAMESS-UK/bin/gamess << EOF
          restart optimise
          title
          water optimisation at scf level 3-21g basis set
          zmat angstrom
          o
          h 1 oh
          h 1 oh 2 hoh
          variables
          oh 0.956
          hoh 104.5
          end
          runtype optimise
          enter
          EOF
\end{verbatim}
}
\item This example performs an open-shell RHF calculation 
at the optimised geometry from 2. (note the RESTART usage).

{
\footnotesize
\begin{verbatim}
          #!/bin/csh -f
          setenv ed2 /scr1/user/h2omain
          setenv ed3 /scr1/user/h2odump
          setenv ed7 /scr1/user/ed7
          /scr1/wab/GAMESS-UK/bin/gamess << EOF
          restart
          title
          h2o+ doublet b1 state - ground state geometry
          charge 1
          mult 2
          zmat angs
          o
          h 1 oh
          h 1 oh 2 hoh
          variables
          oh 0.956
          hoh 104.5
          end
          enter
          EOF
\end{verbatim}
}
\item The valence shell scf mos are to be localised. For the purpose
of the example below it is assumed that orbitals 2 and 3 are the bond
orbitals.  The CHARGE and MULT directives are required to override the
values set in the preceding job on the doublet cation.

{
\footnotesize
\begin{verbatim}
          #!/bin/csh -f
          setenv ed2 /scr1/user/h2omain
          setenv ed3 /scr1/user/h2odump
          setenv ed7 /scr1/user/ed7
          /scr1/wab/GAMESS-UK/bin/gamess << EOF
          restart
          title
          lmos for h2o--  ground state geometry
          charge 0
          mult 1
          zmat angs
          o
          h 1 oh
          h 1 oh 2 hoh
          variables
          oh 0.956
          hoh 104.5
          end
          runtype analyse
          local
          2 to 5
          end
          vectors 1
          enter 5
          EOF
\end{verbatim}
}
\item The localised orbitals are restored from section 5 and the NOGEN
facility used to generate the virtual pairs of the two bond orbitals. The
SWAP directive has been used to move the localised bond orbitals to the
top of the occupied orbital list. The NOGEN facility reorders the orbitals
so that the GVB pairs occur together.  Note the ADAPT~OFF specification,
now required when using localised orbitals as the input orbital set.

{
\footnotesize
\begin{verbatim}
          #!/bin/csh -f
          setenv ed2 /scr1/user/h2omain
          setenv ed3 /scr1/user/h2odump
          setenv ed7 /scr1/user/ed7
          /scr1/wab/GAMESS-UK/bin/gamess << EOF
          restart
          title
          water gvb calculation using localised orbitals
          adapt off
          zmat ang
          o
          h 1 oh
          h 1 oh 2 hoh
          variables
          oh 0.956
          hoh 104.5
          end
          scftype gvb 2
          vectors nogen 5
          swap
          2 5
          3 4
          end
          enter
          EOF
\end{verbatim}
}
\end{enumerate}
\section[SCF Calculations on Cubane, Nitrobenzene and Trinitrotoluene]{SCF Calculations on Cubane, Nitrobenzene and Trinitrotoluene}
We show below the job files for three straightforward closed-shell SCF
calculations, providing more examples of z-matrix specification and
reliance on the default options in such calculations. In each case the
direct access files will be deleted on job completion.\\

{\bf Closed shell SCF Job for \fcubane}
{
\footnotesize
\begin{verbatim}
          #!/bin/csh -f
          cd /scr1/user
          /scr1/wab/GAMESS-UK/bin/gamess << EOF
          title  
          **** c4f4  3/21g ****
          zmat angs                   
          x
          c 1  r1
          c 1  r2  2  90.
          c 1  r1  3  90.   2 180.
          c 1  r2  4  90.   3 180.
          x 2 1.   1 90.    3 0.
          f 2 r3   6 90.    3 180.
          x 4 1.   1 90.    3 0.
          f 4 r3   8 90.    3 180.
          x 3 1.   1 90.    4 0.
          f 3 r3   10 90.   4 180.
          x 5 1.   1 90.    4 0.
          f 5 r3   12 90.   4 180.
          variables
          r1 1.2 
          r2  1.3 
          r3 1.313
          end
          enter
          EOF
\end{verbatim}
}
{\bf Closed shell SCF Job for \nitrobz}
{
\footnotesize
\begin{verbatim}
          #!/bin/csh -f
          cd /scr1/user
          /scr1/wab/GAMESS-UK/bin/gamess << EOF
          title        
          c6h5.no2  3-21g
          accuracy 20 7
          noprint
          zmat angstrom
          c
          n 1 rcn
          x 2 1.0 1 90.0
          c 1 rcc1 2 t1 3 p1
          c 1 rcc1 2 t1 3 -p1
          c 4 rcc2 1 t2 2 p2
          c 5 rcc2 1 t2 2 p2
          c 7 rcc3 5 t3 1 p3
          o 2 rno1 1 t5 3 -90.0
          o 2 rno1 1 t5 3  90.0
          h 4 rch1 1 t6 2 p5
          h 5 rch1 1 t6 2 p5
          h 6 rch2 4 t7 11 p6
          h 7 rch2 5 t7 12 p6
          h 8 rch3 7 t8 14 p7
          variables
          rcn 1.49
          rcc1 1.37
          rcc2 1.43
          rcc3 1.37
          rno1 1.21
          rch1 1.084
          rch2 1.084
          rch3 1.084
          t1 120.0
          t2 120.0
          t3 120.0
          t5 120.0
          t6 120.0
          t7 120.0
          t8 120.0
          p1 90.0
          p2 180.0
          p3 0.0
          p5 0.0
          p6 0.0
          p7 0.0
          end
          maxcyc 20
          enter
          EOF
\end{verbatim}
}
{\bf Closed shell SCF Job for \tnt}
{
\footnotesize
\begin{verbatim}
          #!/bin/csh -f
          cd /scr1/user
          /scr1/wab/GAMESS-UK/bin/gamess << EOF
          title
          2,4,6 tri-nitro-toluene. 3-21g basis.
          noprint distance basis vectors hessian
          zmat angstrom
          x
          c    1    r1
          c    1    r2    2    a1
          c    1    r2    2    a1    3    180.
          c    1    r3    2    a2    3      0.
          c    1    r3    2    a2    3    180.
          x    1    1.    2    90.   3    180.
          c    1    r4    7    90.   2    180.
          n    3    r5    2    a3    1    180.
          o    9    r6    3    a4    2      0.
          o    9    r7    3    a5    2    180.
          n    4    r5    2    a3    1    180.
          o   12    r6    4    a4    2      0.
          o   12    r7    4    a5    2    180.
          x    8    1.    1    90.   6      0.
          n    8    r8   15    90.   1    180.
          o   16    r9    8    a6    5      0.
          o   16    r9    8    a6    5    180.
          x    2    1.    1    90.   4      0.
          c    2   r10   19    90.   1    180.
          h   20   r11    2    a7   19     90.
          h   20   r12    2    a8   21    120.
          h   20   r12    2    a8   21   -120.
          h    5   r13    8    a9    6    180.
          h    6   r13    8    a9    5    180.
          variables
          r1 1.431\r2 1.367\r3 1.397\r4 1.395
          a1 60.38\a2 120.67
          r5 1.521\r8 1.505\r6 1.277\r7 1.278\r9 1.277
          a3 123.52\a4 121.19\a5 116.17\a6 117.53
          r10 1.529\r11 1.087\r12 1.082
          a7 110.07\a8 109.7
          r13 1.086\a9 121.26
          end
          enter
          EOF
\end{verbatim}
}

\section[STO-3G Calculations on \namgplus]{STO-3G Calculations on \namgplus}
In this calculation we perform an STO-3G calculation on \namgplus,
followed by an RHF and then
UHF calculation on the triplet state. Note the use of the SUPER
directive to ensure an integral file format compatible with the
use of BYPASS in the subsequent RHF and UHF calculations.\\

{\bf Closed shell SCF Job}
{
\footnotesize
\begin{verbatim}
          #!/bin/csh -f
          setenv ed2 /scr1/user/namgmain
          setenv ed3 /scr1/user/namgdump
          setenv ed7 /scr1/user/ed7
          /scr1/wab/GAMESS-UK/bin/gamess << EOF
          title
          * na7mg+ * sto-3g * closed shell * scf-energy= -1314.828516
          mult 1
          super force
          charge 1
          zmat angs
          mg
          na 1 r1 
          na 1 r2 2 90.
          na 1 r2 2 90. 3 72.
          na 1 r2 2 90. 4 72.
          na 1 r2 2 90. 5 72.
          na 1 r2 2 90. 6 72.
          na 1 r1 3 90. 2 180.
          variables
          r1 3.0286740
          r2 3.194799
          end
          basis sto3g
          level 1.5   10 1.0 
          maxcyc 40
          enter
          EOF
\end{verbatim}
}
{\bf Open shell RHF Job}
{
\footnotesize
\begin{verbatim}
          #!/bin/csh -f
          setenv ed2 /scr1/user/namgmain
          setenv ed3 /scr1/user/namgdump
          setenv ed7 /scr1/user/ed7
          /scr1/wab/GAMESS-UK/bin/gamess << EOF
          restart
          title
          * na7mg+ * sto-3g * triplet * scf  energy=-1314.900851
          super force
          bypass
          mult 3
          charge 1
          zmat angs
          mg
          na 1 r1 
          na 1 r2 2 90.
          na 1 r2 2 90. 3 72.
          na 1 r2 2 90. 4 72.
          na 1 r2 2 90. 5 72.
          na 1 r2 2 90. 6 72.
          na 1 r1 3 90. 2 180.
          variables
          r1 3.0286740
          r2 3.194799
          end
          basis sto3g
          maxcyc 40
          enter
          EOF
\end{verbatim}
}
{\bf UHF Job}
{
\footnotesize
\begin{verbatim}
          #!/bin/csh -f
          setenv ed2 /scr1/user/namgmain
          setenv ed3 /scr1/user/namgdump
          setenv ed7 /scr1/user/ed7
          /scr1/wab/GAMESS-UK/bin/gamess << EOF
          restart
          super force
          title
          * na7mg+ * sto-3g * triplet * uhf  energy=-1314.901919
          mult 3
          charge 1
          bypass
          zmat angs
          mg
          na 1 r1 
          na 1 r2 2 90.
          na 1 r2 2 90. 3 72.
          na 1 r2 2 90. 4 72.
          na 1 r2 2 90. 5 72.
          na 1 r2 2 90. 6 72.
          na 1 r1 3 90. 2 180.
          variables
          r1 3.0286740
          r2 3.194799
          end
          basis sto3g
          scftype uhf
          vectors 5
          enter
          EOF
\end{verbatim}
}
Note that we are using the energy ordered open-shell RHF eigenvectors to
initiate the UHF calculation (as written to the default section 5 by the
Open shell RHF Job). Had this section not been specified using the VECTORS
directive, then the closed-shell SCF MOs would been used in default.

\section[Extended Basis Set Calculations of \namgplus]{Extended Basis Set Calculations of \namgplus}
In this example we use the STO-3G calculation on \namgplus, performed
above as a starting point for a more extensive basis set calculation.
In particular the set of closed-shell vectors is restored under control
of GETQ, with the STO-3G Dumpfile used as a `foreign' Dumpfile. We then
perform an RHF calculation on the open-shell singlet state of \namgplus, using the
integrals calculated in the closed-shell case. Note that the OPEN directive
is now required for this low-spin state.\\

{\bf Closed shell SCF Job}
{
\footnotesize
\begin{verbatim}
          #!/bin/csh -f
          setenv ed2 /scr1/user/namgmain
          setenv ed3 /scr1/user/namgdump
          setenv ed7 /scr1/user/ed7
          /scr1/wab/GAMESS-UK/bin/gamess << EOF
          dumpfile ed3 500
          title
          * na7mg+ (4s3p//4s3p1d)  scf energy=-1330.808742
          mult 1
          charge 1
          super off
          noprint vectors
          zmat angs
          mg
          na 1 r1 
          na 1 r2 2 90.
          na 1 r2 2 90. 3 72.
          na 1 r2 2 90. 4 72.
          na 1 r2 2 90. 5 72.
          na 1 r2 2 90. 6 72.
          na 1 r1 3 90. 2 180.
          variables
          r1 3.0286740
          r2 3.194799
          end
          basis
          s mg
            .005004   5609.67
            .037083   841.969
            .171495   191.263
            .444597   53.2621
            .480060   16.6003
          s mg
            .352170   2.97082
            .692921   1.00728
          s mg
            1.00000   .113641
          s mg
            1.000000  .044678
          p mg
            .039884   50.9665
            .223321   11.4364
            .514536   3.21935
          p mg
            1.00000   0.914433
          p mg
            1.00000   0.16
          d mg
            1.00000   0.175
          s na 
              .003064  6902.67
              .022198  1059.04
              .095576  255.445
              .280448  77.3172  
               .452587 26.8224
              .29313   10.0718
          s na
              1.000000  2.17902
          s na
              1.000000 .689482
          s na 
            1.0       .040274
          p na
            .042422   38.9438
            .229433   8.71012
            .509774   2.42053
          p na
            1.00000   .661896
          p na
          1.0        .065
          end
          maxcyc 40
          vectors getq ed3 1 1
          enter
          EOF
\end{verbatim}
}
{\bf Open shell RHF Job}
{
\footnotesize
\begin{verbatim}
          #!/bin/csh -f
          setenv ed2 /scr1/user/namgmain
          setenv ed3 /scr1/user/namgdump
          setenv ed7 /scr1/user/ed7
          /scr1/wab/GAMESS-UK/bin/gamess << EOF
          dumpfile ed3 500
          restart
          # open shell scf using closed shell vectors from above   *
          # bypass integral evaluation                             *
          title
          * na7mg+ (4s3p//4s3p1d)  triplet rhf *  energy=-1330.766950
          super off
          bypass
          mult 1
          charge 1
          zmat angs
          mg
          na 1 r1 
          na 1 r2 2 90.
          na 1 r2 2 90. 3 72.
          na 1 r2 2 90. 4 72.
          na 1 r2 2 90. 5 72.
          na 1 r2 2 90. 6 72.
          na 1 r1 3 90. 2 180.
          variables
          r1 3.0286740
          r2 3.194799
          end
          basis
          s mg
            .005004   5609.67
            .037083   841.969
            .171495   191.263
            .444597   53.2621
            .480060   16.6003
          s mg
            .352170   2.97082
            .692921   1.00728
          s mg
            1.00000   .113641
          s mg
            1.000000  .044678
          p mg
            .039884   50.9665
            .223321   11.4364
            .514536   3.21935
          p mg
            1.00000   0.914433
          p mg
            1.00000   0.16
          d mg
            1.00000   0.175
          s na 
              .003064  6902.67
              .022198  1059.04
              .095576  255.445
              .280448  77.3172  
               .452587 26.8224
              .29313   10.0718
          s na
              1.000000  2.17902
          s na
              1.000000 .689482
          s na 
            1.0       .040274
          p na
            .042422   38.9438
            .229433   8.71012
            .509774   2.42053
          p na
            1.00000   .661896
          p na
          1.0        .065
          end
          open 1 1 1 1
          maxcyc 40
          enter 
          EOF
\end{verbatim}
}

\section[ECP calculations of \namgplus]{ECP calculations of \namgplus}

In this example  on \namgplus, we perform a local ECP calculation, using
the Hay-Wadt ECP's, together with the associated double zeta basis sets,
augmented by a d-function on Mg.  Having carried out the closed-shell
SCF calculation, we  perform an RHF calculation on the singlet state
of \namgplus, using the integrals calculated in the closed-shell case.
Note that we are re-using the files from the previous example.\\

{\bf Closed shell SCF Job}
{
\footnotesize
\begin{verbatim}
          #!/bin/csh -f
          setenv ed2 /scr1/user/namgmain
          setenv ed3 /scr1/user/namgdump
          setenv ed7 /scr1/user/ed7
          /scr1/wab/GAMESS-UK/bin/gamess << EOF
          title
          na7mg+ ecp LANL ecp /closed shell singlet
          charge 1
          super off
          zmat angs
          mg
          na 1 r1 
          na 1 r2 2 90.
          na 1 r2 2 90. 3 72.
          na 1 r2 2 90. 4 72.
          na 1 r2 2 90. 5 72.
          na 1 r2 2 90. 6 72.
          na 1 r1 3 90. 2 180.
          variables
          r1 3.0286740
          r2 3.194799
          end
          basis 
          ecpdz na 
          ecpdz mg
          d mg
          1.0 0.175
          end
          ecp
          na na
          mg mg
          level  1.0
          enter
          EOF
\end{verbatim}
}
{\bf Open shell RHF Job}
{
\footnotesize
\begin{verbatim}
          #!/bin/csh -f
          setenv ed2 /scr1/user/namgmain
          setenv ed3 /scr1/user/namgdump
          setenv ed7 /scr1/user/ed7
          /scr1/wab/GAMESS-UK/bin/gamess << EOF
          restart
          title
          na7mg+ ecp /open-shell singlet rhf
          mult 1
          charge 1
          super off
          bypass
          zmat angs
          mg
          na 1 r1 
          na 1 r2 2 90.
          na 1 r2 2 90. 3 72.
          na 1 r2 2 90. 4 72.
          na 1 r2 2 90. 5 72.
          na 1 r2 2 90. 6 72.
          na 1 r1 3 90. 2 180.
          variables
          r1 3.0286740
          r2 3.194799
          end
          basis 
          ecpdz na 
          ecpdz mg
          d mg
          1.0 0.175
          end
          ecp
          na na
          mg mg
          runtype scf
          open 1 1 1 1
          level  0.3 1.0
          enter
          EOF
\end{verbatim}
}
\section[Graphical analysis of nickel tetracarbonyl]{Graphical analysis of nickel tetracarbonyl}
 
The following  example illustrates features of the  Graphical Analysis
module, in analysing the ground state SCF wavefunction of \nico. Let us
assume the following job has been used in constructing this wavefunction.

{
\footnotesize
\begin{verbatim}
          #!/bin/csh -f
          setenv ed2 /scr1/user/nicomain
          setenv ed3 /scr1/user/nicodump
          setenv ed7 /scr1/user/ed7
          /scr1/wab/GAMESS-UK/bin/gamess << EOF
          title\ni(co)4 .. 3-21g / SCF total energy  -1947.868687
          zmat angstrom
          ni
          c 1 nic
          c 1 nic 2 109.471
          c 1 nic 2 109.471 3 120.0
          c 1 nic 2 109.471 4 120.0
          x 2 1.0 1 90.0 3 180.0
          o 2 co 6 90.0 1 180.0
          x 3 1.0 1 90.0 2 180.0
          o 3 co 8 90.0 1 180.0
          x 4 1.0 1 90.0 5 180.0
          o 4 co 10 90.0 1 180.0
          x 5 1.0 1 90.0 4 180.0
          o 5 co 12 90.0 1 180.0
          variables
          nic 1.831
          co 1.131
          end
          level 1.5
          enter
          EOF
\end{verbatim}
}
Examination of the output reveals the following symmetry
designation:
{
\footnotesize
\begin{verbatim}
                                  ******************
                                  MOLECULAR SYMMETRY
                                  ******************
    
     MOLECULAR POINT GROUP    TD      
     ORDER OF PRINCIPAL AXIS   0
    
     SYMMETRY POINTS :
    
     POINT 1 :    0.0000000   0.0000000   0.0000000
     POINT 2 :    0.0000000   0.0000000   1.0000000
     POINT 3 :    0.0000000   1.0000000   0.0000000
\end{verbatim}
}
and the following atomic coordinates:
{
\footnotesize
\begin{verbatim}
           0.0000000      0.0000000      0.0000000       NI
          -1.9976836      1.9976836     -1.9976836       C
           1.9976836     -1.9976836     -1.9976836       C 
          -1.9976836     -1.9976836      1.9976836       C 
           1.9976836      1.9976836      1.9976836       C 
          -3.2316433      3.2316433     -3.2316433       O
           3.2316433     -3.2316433     -3.2316433       O
          -3.2316433     -3.2316433      3.2316433       O
           3.2316433      3.2316433      3.2316433       O
\end{verbatim}
}
The following job may be used to construct a total density plot of the SCF
wavefunction in a plane containing the Ni atom and two carbonyl groups,
with the Ni at the centre of the plot: a contour plot will be generated
on line printer output.

{
\footnotesize
\begin{verbatim}
          #!/bin/csh -f
          setenv ed2 /scr1/user/nicomain
          setenv ed3 /scr1/user/nicodump
          setenv ed7 /scr1/user/ed7
          /scr1/wab/GAMESS-UK/bin/gamess << EOF
          restart
          title\ni(co)4 .. 3-21g / SCF total energy  -1947.868687
          zmat angstrom
          ni
          c 1 nic
          c 1 nic 2 109.471
          c 1 nic 2 109.471 3 120.0
          c 1 nic 2 109.471 4 120.0
          x 2 1.0 1 90.0 3 180.0
          o 2 co 6 90.0 1 180.0
          x 3 1.0 1 90.0 2 180.0
          o 3 co 8 90.0 1 180.0
          x 4 1.0 1 90.0 5 180.0
          o 4 co 10 90.0 1 180.0
          x 5 1.0 1 90.0 4 180.0
          o 5 co 12 90.0 1 180.0
          variables
          nic 1.831
          co 1.131
          end
          runtype analyse
          graphics
          gdef
          type 2d
          title
          square 2d grid ni(co)4 - total density
          calc
          type dens 
          title
          ni(co)4 - total density
          section 151
          plot
          type line 
          title
          ni(co)4 - total density
          vectors 1
          enter
          EOF
\end{verbatim}
}

\section[Two-Pair GVB Calculation on 1-imino-2,4-pentadiene]{Two-Pair GVB Calculation on 1-imino-2,4-pentadiene}

{\bf Closed shell SCF Job}
{
\footnotesize
\begin{verbatim}
          #!/bin/csh -f
          setenv ed2 /scr1/user/gvbmain
          setenv ed3 /scr1/user/gvbdump
          setenv ed7 /scr1/user/ed7
          /scr1/wab/GAMESS-UK/bin/gamess << EOF
          # open shell rhf sto3g                                     *
          # check super force                                        *
          title
          1-imino-2,4-pentadiene *  energy=-245.073477
          super force nosym
          mult 1
          accuracy 20 7
          charge 1
          zmat angs
          c
          c 1 r1
          c 2 r2 1 a1
          c 3 r3 2 a2 1 cx
          c 4 r4 3 a3 2 c2
          n 1 r5 2 a4 3 c2
          h 1 r6 2 a5 3 c1
          h 2 r7 3 a6 7 c2
          h 3 r8 4 a7 8 c2
          h 4 r9 5 a8 9 c2
          h 5 r10 4 a9 10 c1
          h 5 r11 4 a10 11 c2
          h 6 r12 1 a11 7 c1 
          h 6 r13 1 a12 13 c2
          variables
          r1     1.3463261 
          r2     1.4632685
          r3     1.3961480 
          r4     1.3568062 
          r5     1.3416773 
          r6     1.0748794
          r7     1.0669518
          r8     1.0811641
          r9     1.0707072
          r10    1.0729928
          r11    1.0747118
          r12    0.9971261
          r13    0.9985437
          a1   118.2539432
          a2   122.8419092
          a3   119.3229511
          a4   125.9857684
          a5   119.9766875
          a6   122.1120978
          a7   116.3317717
          a8   120.9014063
          a9   121.5606571
          a10  121.8124933
          a11  120.9303349
          a12  121.8324395
          constants
          c1 0.
          c2 180.
          cx  90.
          end
          basis sto3g
          scftype rhf
          open 1 1 1 1
          enter
          EOF
\end{verbatim}
}
{\bf Two-Pair GVB Job}
{
\footnotesize
\begin{verbatim}
          #!/bin/csh -f
          setenv ed2 /scr1/user/gvbmain
          setenv ed3 /scr1/user/gvbdump
          setenv ed7 /scr1/user/ed7
          /scr1/wab/GAMESS-UK/bin/gamess << EOF
          # gvb-scf  with two pairs                                   *
          # bypass integral evaluation                                *
          restart
          bypass
          title
          1-imino-2,4-pentadiene *   energy=-245.1146628
          mult 1
          accuracy 20 7
          super force nosym
          charge 1
          zmat angs
          c
          c 1 r1
          c 2 r2 1 a1
          c 3 r3 2 a2 1 cx
          c 4 r4 3 a3 2 c2
          n 1 r5 2 a4 3 c2
          h 1 r6 2 a5 3 c1
          h 2 r7 3 a6 7 c2
          h 3 r8 4 a7 8 c2
          h 4 r9 5 a8 9 c2
          h 5 r10 4 a9 10 c1
          h 5 r11 4 a10 11 c2
          h 6 r12 1 a11 7 c1 
          h 6 r13 1 a12 13 c2
          variables
          r1     1.3463261 
          r2     1.4632685
          r3     1.3961480 
          r4     1.3568062 
          r5     1.3416773 
          r6     1.0748794
          r7     1.0669518
          r8     1.0811641
          r9     1.0707072
          r10    1.0729928
          r11    1.0747118
          r12    0.9971261
          r13    0.9985437
          a1   118.2539432
          a2   122.8419092
          a3   119.3229511
          a4   125.9857684
          a5   119.9766875
          a6   122.1120978
          a7   116.3317717
          a8   120.9014063
          a9   121.5606571
          a10  121.8124933
          a11  120.9303349
          a12  121.8324395
          constants
          c1 0.
          c2 180.
          cx  90.
          end
          basis sto3g
          runtype scf
          scftype gvb 2
          enter
          EOF
\end{verbatim}
}

\section[Direct-SCF Calculations]{Direct-SCF Calculations}

In the first example below we show a direct-SCF calculation
in which the input geometry in cartesian coordinates
is converted to z--matrix representation; the second calculation
features geometry optimisation of Be(C$_{5}$H$_{5}$)$_{2}$
using the direct-SCF module.

{
\footnotesize
\begin{verbatim}
          #!/bin/csh -f
          setenv ed3 /scr1/user/dscfdump
          setenv ed7 /scr1/user/ed7
          /scr1/wab/GAMESS-UK/bin/gamess << EOF
          # direct scf generate zmatrix sto3g-basis              *
          title
          * test2a * energy=-1550.28679356
          charge -2
          geometry  distance angles torsions       all
          -2.9512196   -0.1547624    -2.3287565    8.0    o
          -1.0815728   -1.9700376    -1.5385361    6.0    c
          -1.1603816   -3.5363478    -2.8467927    1.0    h
           1.6367703   -0.8475562    -1.5624978    6.0    c
           2.1566895   -0.4797763    -3.4976155    1.0    h
           3.5011251   -2.6669663    -0.4827273    7.0    n
           2.9474874   -3.6433853     1.0782554    1.0    h
           1.7359097    1.7242063    -0.1090391    6.0    c
           1.3150929    1.4125382     1.8702785    1.0    h
           4.1957289    2.9182102    -0.4054688    8.0    o
          -0.3417469    3.5022051    -1.1895255    6.0    c
           0.0532604    3.8272799    -3.1681491    1.0    h
          -0.3292201    5.9011224     0.0783133    8.0    o
          -1.4442971    7.1126500    -0.6802213    1.0    h
          -2.9721287    2.1997957    -0.9534701    6.0    c
          -3.3279629    1.8049798     1.0197182    1.0    h
          -5.1750163    3.9122805    -1.8751357    6.0    c
          -4.8159917    4.4971112    -3.8035825    1.0    h
          -6.8815215    2.7942026    -1.8646838    1.0    h
          -5.4889849    6.1460386    -0.3379417    8.0    o
          -7.5612331    6.0739323     1.4363625    6.0    c
          -7.4278090    4.3945431     2.5896385    1.0    h
           5.7357642    3.1285958     1.6502323    6.0    c
           5.5713092    1.7094915     3.4875662    8.0    o
           7.8017770    5.1679189     1.7463198    6.0    c
           7.0224823    6.8328410     2.6479587    1.0    h
           8.3530583    5.6784964    -0.1498644    1.0    h
           5.9321304   -2.9668945    -1.2970693    6.0    c
           6.7897779   -1.7121829    -3.0617741    8.0    o
           7.7021385   -4.8380024     0.0564767    6.0    c
           7.1302965   -6.7576698    -0.3499893    1.0    h
           7.6050313   -4.5640081     2.0812894    1.0    h
          -1.5715523   -2.9124352     0.9658559    8.0    o
          -3.9184431   -4.8124121     0.9101853   15.0    p
          -3.1659952   -7.1913356    -0.3438675    8.0    o
          -6.0342521   -3.6456571    -0.4943459    8.0    o
          -4.4900093   -5.1460285     3.8664887    8.0    o
           9.4719556   -4.5440252    -0.5381892    1.0    h
           9.3293820    4.5891096     2.6957586    1.0    h
          -9.1872029    6.0381561     0.4734880    1.0    h
          -7.5522104    7.5798978     2.5757500    1.0    h
          end
          basis sto3g
          scftype direct
          enter
          EOF
\end{verbatim}
}
{\bf Direct-SCF Geometry Optimisation of Be(C$_{5}$H$_{5}$)$_{2}$}

{
\footnotesize
\begin{verbatim}
          #!/bin/csh -f
          /scr1/wab/GAMESS-UK/bin/gamess << EOF
          title
          be(c5h5)2 sto3g optimised total energy  = -394.2789851 au
          zmatrix angstrom
          x 
          x  1  fxa
          c  2  xc  1  xxc
          c  2  xc  1  xxc  3  cxc
          c  2  xc  1  xxc  4  cxc
          c  2  xc  1  xxc  3  -cxc
          c  2  xc  1  xxc  6  -cxc
          x  2  xx  3  xxc  4  -xxc
          h  2  hx  8  hxx  3  hcx
          h  2  hx  8  hxx  4  hcx
          h  2  hx  8  hxx  5  hcx
          h  2  hx  8  hxx  6  hcx
          h  2  hx  8  hxx  7  hcx
          x  2  fxt 3  xxc  4  xxc
          c 14  xc  1  xxc  3  cxxc
          c 14  xc  1  xxc 15  cxc
          c 14  xc  1  xxc 16  cxc
          c 14  xc  1  xxc 15  -cxc
          c 14  xc  1  xxc 18  -cxc
          x 14  xx  15 xxc 16  -xxc
          h 14  hx  20 hxx 15  hcx
          h 14  hx  20 hxx 16  hcx
          h 14  hx  20 hxx 17  hcx
          h 14  hx  20 hxx 18  hcx
          h 14  hx  20 hxx 19  hcx
          be 2  fxa  3 xxc  4  xxc
          variables
          fxa   1.47
          fxt   3.37
          xc    1.22
          hx    2.12 
          hxx  88.4
          constants
          cxc  72.0
          cxxc  36.0
          xxc  90.0 
          hcx   0.0
          xx    1.0
          end
          basis sto3g
          runtype optimize
          scftype direct 
          level 2.0 10 1.4
          enter
          EOF
\end{verbatim}
}

\section[HCN/HNC Transition State Location]{HCN/HNC Transition State Location}
Transition state calculation  for the HCN,HNC isomerisation process.
The first job uses the default trust region algorithm, the second the
synchronous transit algorithm, and the 
third the Jorgensen-Simons algorithm.

{
\footnotesize
\begin{verbatim}
          #!/bin/csh -f
          cd /scr1/user
          /scr1/wab/GAMESS-UK/bin/gamess << EOF
          title
          hcn 4-31G saddle point
          zmat angs
          c
          x 1 1.0
          n 1 cn 2 90.0
          h 1 ch 2 90.0 3 hcn
          variables
          cn 1.1484 type 3
          ch 1.5960 type 3
          hcn 90.0 type 3
          end
          basis 4-31g
          runtype saddle
          enter
          EOF
\end{verbatim}
}
Saddle point for HCN using the synchronous transit algorithm - note the
definition of the minima required (on the variable definition lines) and
the LSEARCH directive. The default saddle point method does nor
require minima definition (see above)

{
\footnotesize
\begin{verbatim}
          #!/bin/csh -f
          cd /scr1/user
          /scr1/wab/GAMESS-UK/bin/gamess << EOF
          title
          hcn saddle point - synchronous transit
          zmat angs
          c
          x 1 1.0
          n 1 cn 2 90.0
          h 1 ch 2 90.0 3 hcn
          variables
          cn 1.1484 minima 1.1371 1.1597
          ch 1.5960 minima 1.0502 2.1429
          hcn 90.0   minima 180.0  0.0
          end
          basis 4-31g
          runtype saddle
          lsearch 0 4
          enter
          EOF
\end{verbatim}
}
Saddle point for HCN using the Jorgensen-Simons algorithm.
{
\footnotesize
\begin{verbatim}
          #!/bin/csh -f
          cd /scr1/user
          /scr1/wab/GAMESS-UK/bin/gamess << EOF
          title                
          hcn/hnc ts search . jorgensen-simons
          zmat angs
          c
          x 1 1.0
          n 1 cn 2 90.0
          h 1 ch 2 90.0 3 hcn
          variables
          cn 1.1484 type 3
          ch 1.5960 type 3
          hcn 90.0   type 3
          end
          basis 4-31g
          runtype saddle jorgensen 
          powell
          maxjor 55
          recalc off
          rfo off
          cutoffs
          optprint on
          xtol 0.0018
          enter           
          EOF 
\end{verbatim}
}
\section[HSiP/HPSi Transition State Location]{HSiP/HPSi Transition State Location}
This example is concerned with locating the transition
state in the HPSi, HSiP isomerisation process, and calculating
the associated vibrational frequencies. We provide sample
jobs using both numerical and analytical techniques
in the transition state location and
subsequent force constant evaluation. Note that the latter example is
computationally the most efficient, and should certainly
be adopted for small--medium sized molecules.

\subsection[Numerical Force Constants]{Numerical Force Constants}

In the first step we perform an initial SCF for subsequent use in the
saddle point calculation.\\

{\bf Closed shell SCF Job}
{
\footnotesize
\begin{verbatim}
          #!/bin/csh -f
          setenv ed2 /scr1/user/hpsimain
          setenv ed3 /scr1/user/hpsidump
          setenv ed7 /scr1/user/ed7
          /scr1/wab/GAMESS-UK/bin/gamess << EOF
          title
          psih saddle point
          zmat ang
          p
          x 1 1.0
          si 1 psi 2 90.0
          h 1 ph 2 90.0 3 hpsi
          variables
          psi 2.053 type 3
          ph  2.44  type 3
          hpsi  51.02 type 3
          end
          enter
          EOF
\end{verbatim}
}
In the subsequent location of the transition state, note
the use of TYPE 3 which causes the program
to calculate the complete force constant matrix numerically
before commencing the search for the saddle point, and the
use of XTOL to provide  more stringent optimisation criteria
in view of the subsequent force constant evaluation.
LOCK is used to retain the initial SCF  configuration 
throughout the search.\\

{\bf Transition State Job}
{
\footnotesize
\begin{verbatim}
          #!/bin/csh -f
          setenv ed2 /scr1/user/hpsimain
          setenv ed3 /scr1/user/hpsidump
          setenv ed7 /scr1/user/ed7
          /scr1/wab/GAMESS-UK/bin/gamess << EOF
          restart new
          bypass
          title
          psih <-> hpsi saddle point
          zmat angs
          p
          x 1 1.0
          si 1 psi 2 90.0
          h 1 ph 2 90.0 3 hpsi
          variables
          psi 2.053 type 3
          ph  2.44  type 3
          hpsi  51.02 type 3
          end
          runtype saddle
          lock
          xtol 0.0005
          enter 2
          EOF
\end{verbatim}
}
Finally we present the job for numerical evaluation of the
force constants at the optimised geometry. Note the use
of restart in requesting usage of the geometry from the
Dumpfile, rather than from the data file.\\

{\bf Numerical Force Constant Job}
{
\footnotesize
\begin{verbatim}
          #!/bin/csh -f
          setenv ed2 /scr1/user/hpsimain
          setenv ed3 /scr1/user/hpsidump
          setenv ed7 /scr1/user/ed7
          /scr1/wab/GAMESS-UK/bin/gamess << EOF
          restart 
          title
          psih <-> hpsi saddle point numerical fcm
          zmat angs
          p
          x 1 1.0
          si 1 psi 2 90.0
          h 1 ph 2 90.0 3 hpsi
          variables
          psi 2.053 type 3
          ph  2.44  type 3
          hpsi  51.02 type 3
          end
          runtype force
          vectors 2
          lock
          enter 3
          EOF
\end{verbatim}
}

\subsection[Analytic Force Constants]{Analytic Force Constants}

In the first step we perform the computation of the trial
hessian under RUNTYPE~HESSIAN control for subsequent use in the
saddle point calculation.\\

{\bf Computing the trial Hessian}
{
\footnotesize
\begin{verbatim}
          #!/bin/csh -f
          cd /scr1/user
          setenv ed2 hpsimain
          setenv ed3 hpsidump
          /scr1/wab/GAMESS-UK/bin/gamess << EOF
          title
          psih trial hessian / SCF 
          zmat ang
          p
          x 1 1.0
          si 1 psi 2 90.0
          h 1 ph 2 90.0 3 hpsi
          variables
          psi 2.053 
          ph  2.44  
          hpsi  51.02 
          end
          runtype hessian
          enter
          EOF
\end{verbatim}
}
In the subsequent location of the transition state, note
the use of the FCM keyword on the RUNTYPE data line
to restore the trial hessian computed in the first job
and the use of XTOL to provide  more stringent optimisation criteria
in view of the subsequent force constant evaluation.
LOCK is used to retain the initial SCF  configuration 
throughout the search.\\

{\bf Transition State Job}
{
\footnotesize
\begin{verbatim}
          #!/bin/csh -f
          cd /scr1/user
          setenv ed2 hpsimain
          setenv ed3 hpsidump
          /scr1/wab/GAMESS-UK/bin/gamess << EOF
          restart new
          title
          psih <-> hpsi saddle point location / using trial hessian
          zmat angs
          p
          x 1 1.0
          si 1 psi 2 90.0
          h 1 ph 2 90.0 3 hpsi
          variables
          psi 2.053
          ph  2.44  
          hpsi  51.02 
          end
          runtype saddle fcm
          xtol 0.0005
          vectors 1
          lock
          enter 2
          EOF
\end{verbatim}
}
Finally we present the job for analytic computation of the
force constants at the optimised geometry under control
of runtype hessian. Note the use
of restart in requesting usage of the geometry from the
Dumpfile, rather than from the data file.\\

{\bf Analytic Force Constant Job}
{
\footnotesize
\begin{verbatim}
          #!/bin/csh -f
          cd /scr1/user
          setenv ed2 hpsimain
          setenv ed3 hpsidump
          /scr1/wab/GAMESS-UK/bin/gamess << EOF
          restart
          title
          psih <-> hpsi saddle point / force constants
          zmat angs
          p
          x 1 1.0
          si 1 psi 2 90.0
          h 1 ph 2 90.0 3 hpsi
          variables
          psi 2.053
          ph  2.44  
          hpsi  51.02 
          end
          runtype hessian
          vectors 2
          lock
          enter 2
          EOF
\end{verbatim}
}

\section[Use of Bond-centred Functions]{Use of Bond-centred Functions}
In this example we demonstrate the use
of bond-centred functions (s,p), cited at  the mid-point
of the C-N bond in the HCN, HNC transition state.

{
\footnotesize
\begin{verbatim}
          #!/bin/csh -f
          cd /scr1/user
          /scr1/wab/GAMESS-UK/bin/gamess << EOF
          title
          hcn-hnc basis - dunning (9s5p-3s2p) bond(s,p) + p(h)
          zmat angstrom
          c
          bq 1 rcn2
          x 2 1.0 1 90.0
          n 2 rcn2 3 90.0 1 180.0
          x 1 1.0 2 90.0 3 0.0
          h 1 rch 5 90.0 4 phi
          variables
          rcn2 0.5991
          rch 1.2128
          phi 71.2
          end
          basis
          sv h
          p h
          1.0 0.7
          s bq
          1.0 1.0
          p bq
          1.0 0.7
          sv c
          sv n
          end
          enter
          EOF
\end{verbatim}
}
\section[SCF Analytic Force Constants for Ethylene]{SCF Analytic Force Constants for Ethylene}

We consider below computing the analytic force constants for
\ethene, initially optimising the molecule at the SCF level, followed
by the force constant calculation. Note the use of the XTOL
directive in the optimisation job to ensure a higher degree
of optimisation than that derived using the default XTOL.\\

{\bf Geometry Optimisation}
{
\footnotesize
\begin{verbatim}
          #!/bin/csh -f
          cd /scr1/user
          setenv ed2 c2h4main
          setenv ed3 c2h4dump
          /scr1/wab/GAMESS-UK/bin/gamess << EOF
          title 
          ethylene 6-31g**  geometry optimisation
          zmatrix angstrom
          c
          c 1 cc
          h 1 ch 2 hcc
          h 1 ch 2 hcc 3 180.0
          h 2 ch 1 hcc 3 0.0
          h 2 ch 1 hcc 3 180.0
          variables
          cc 1.40
          ch 1.10
          hcc 118.0
          end
          basis 6-31g**
          runtype optimize
          xtol 0.0001
          enter
          EOF
\end{verbatim}
}
{\bf Analytic Force Constants}
{
\footnotesize
\begin{verbatim}
          #!/bin/csh -f
          cd /scr1/user
          setenv ed2 c2h4main
          setenv ed3 c2h4dump
          /scr1/wab/GAMESS-UK/bin/gamess << EOF
          restart
          title 
          ethylene 6-31g** ground state vibrational frequencies
          zmatrix angstrom
          c
          c 1 cc
          h 1 ch 2 hcc
          h 1 ch 2 hcc 3 180.0
          h 2 ch 1 hcc 3 0.0
          h 2 ch 1 hcc 3 180.0
          variables
          cc 1.40
          ch 1.10
          hcc 118.0
          end
          basis 6-31g**
          runtype hessian
          enter
          EOF
\end{verbatim}
}
\section[MP2 Analytic Force Constants for Ethylene]{MP2 Analytic Force Constants for Ethylene}

We consider below computing the analytic force constants for
\ethene, initially optimising the molecule at the MP2 level, followed
by the force constant calculation. Note again the use of the XTOL
directive in the optimisation job to ensure a higher degree
of optimisation than that derived using the default XTOL.\\

{\bf MP2 Geometry Optimisation}
{
\footnotesize
\begin{verbatim}
          #!/bin/csh -f
          cd /scr1/user
          setenv ed2 c2h4main
          setenv ed3 c2h4dump
          /scr1/wab/GAMESS-UK/bin/gamess << EOF
          title 
          ethylene 6-31g**  MP2/ optimised total energy  =  -78.3272309
          zmatrix angstrom
          c
          c 1 cc
          h 1 ch 2 hcc
          h 1 ch 2 hcc 3 180.0
          h 2 ch 1 hcc 3 0.0
          h 2 ch 1 hcc 3 180.0
          variables
          cc 1.40
          ch 1.10
          hcc 118.0
          end
          basis 6-31g**
          runtype optimize
          scftype mp2
          xtol 0.0001
          enter
          EOF
\end{verbatim}
}
{\bf MP2 Analytic Force Constants}
{
\footnotesize
\begin{verbatim}
          #!/bin/csh -f
          cd /scr1/user
          setenv ed2 c2h4main
          setenv ed3 c2h4dump
          /scr1/wab/GAMESS-UK/bin/gamess << EOF
          restart
          title 
          ethylene MP2/6-31g** ground state vibrational frequencies
          #freq    847.7,  941.6,  994.1, 1091.6, 1267.4,  1414.5,  
          #freq   1525.2, 1729.7, 3241.3, 3259.2, 3336.6,  3359.8
          zmatrix angstrom
          c
          c 1 cc
          h 1 ch 2 hcc
          h 1 ch 2 hcc 3 180.0
          h 2 ch 1 hcc 3 0.0
          h 2 ch 1 hcc 3 180.0
          variables
          cc 1.40
          ch 1.10
          hcc 118.0
          end
          basis 6-31g**
          runtype hessian
          scftype mp2
          enter
          EOF
\end{verbatim}
}

\section[MP2 Polarisability for Ethylene]{MP2 Polarisability for Ethylene}

We consider below computing the molecular polarisability of
\ethene, initially optimising the molecule at the MP2 level, followed
by the property calculation. Note again the use of the XTOL
directive in the optimisation job to ensure a higher degree
of optimisation than that derived using the default XTOL.\\

{\bf MP2 Geometry Optimisation}
{
\footnotesize
\begin{verbatim}
          #!/bin/csh -f
          cd /scr1/user
          setenv ed2 c2h4main
          setenv ed3 c2h4dump
          /scr1/wab/GAMESS-UK/bin/gamess << EOF
          title 
          ethylene 6-31g**  MP2/ optimised total energy  =  -78.3272309
          zmatrix angstrom
          c
          c 1 cc
          h 1 ch 2 hcc
          h 1 ch 2 hcc 3 180.0
          h 2 ch 1 hcc 3 0.0
          h 2 ch 1 hcc 3 180.0
          variables
          cc 1.40
          ch 1.10
          hcc 118.0
          end
          basis 6-31g**
          runtype optimize
          scftype mp2
          xtol 0.0001
          enter
          EOF
\end{verbatim}
}
{\bf MP2 Polarisability}
{
\footnotesize
\begin{verbatim}
          #!/bin/csh -f
          cd /scr1/user
          setenv ed2 c2h4main
          setenv ed3 c2h4dump
          /scr1/wab/GAMESS-UK/bin/gamess << EOF
          restart
          title 
          ethylene MP2/6-31g** ground state polarisability
          zmatrix angstrom
          c
          c 1 cc
          h 1 ch 2 hcc
          h 1 ch 2 hcc 3 180.0
          h 2 ch 1 hcc 3 0.0
          h 2 ch 1 hcc 3 180.0
          variables
          cc 1.40
          ch 1.10
          hcc 118.0
          end
          basis 6-31g**
          runtype polarisability
          scftype mp2
          enter
          EOF
\end{verbatim}
}
\section[Direct-MP2 Calculation of Pyridine]{Direct-MP2 Calculation of Pyridine}

We show below the data for performing a direct--MP2 calculation
on the \pyridine\ molecule, conducted in a 6--31G* basis set. Note
the use of the MEMORY pre-directive in requesting
a memory allocation of 4 MWords.

{
\footnotesize
\begin{verbatim}
          #!/bin/csh -f
          cd /scr1/user
          setenv ed3 pyred3
          /scr1/wab/GAMESS-UK/bin/gamess << EOF
          title
          pyridine 6-31g* direct-mp2
          zmat angstrom
          n
          x 1 1.0
          x 1 1.0 2 90.
          x 1 1.0 2 90. 3 90.
          c 1 c4n 3 90. 2 180.
          x 5 1.0 1 90. 3 0.0
          x 5 1.0 1 90. 4 0.0
          h 5 ch4 6 90. 1 180.
          c 1 c2n 2 c2nz 3 180.
          c 1 c2n 2 c2nz 3 0.0
          c 9 c2c3 1 ccn 2 180.
          c 10 c2c3 1 ccn 2 180.
          h 9 c2h6 1 nch2 2 0.0
          h 10 c2h6 1 nch2 2 0.0
          h 11 c3h5 9 c2c3h 1 180.
          h 12 c3h5 10 c2c3h 1 180.
          variables
          c4n 2.7845546
          ch4 1.0823078
          c2n 1.3372389
          c2nz 120.641858
          c2c3 1.3944571
          ccn 122.662269
          c2h6 1.0814291
          c3h5 1.0809550
          nch2 116.400433
          c2c3h 120.158516
          end
          basis 6-31g*
          scftype direct mp2
          enter
          EOF
\end{verbatim}
}

\section[CASSCF Geometry Optimisations]{CASSCF Geometry Optimisations}
We consider below a CASSCF calculation on the X$^{1}$A$_{1}$ state
of \water, using a full valence criterion in specifying the
active space so that the formally vacant SCF virtual MOs, 4a$_{1}$
and 2b$_{2}$, are permitted variable occupancy. This example
utilises the vectors from the closed shell SCF calculation of
Example 1.

{
\footnotesize
\begin{verbatim}
          #!/bin/csh -f
          cd /scr1/user    
          setenv ed2 h2omain
          setenv ed3 h2odump
          /scr1/wab/GAMESS-UK/bin/gamess << EOF
          restart new
          title
          water  at casscf level 3-21g basis set
          zmat angstrom
          o
          h 1 oh
          h 1 oh 2 hoh
          variables
          oh 0.956
          hoh 104.5
          end
          scftype casscf
          config print
          doc 1 to 5
          uoc 6 7
          end
          superci 1 to 8
          newton 9 to 20
          hessian 9 to 20
          simul 9 to 20
          enter
          EOF
\end{verbatim}
}
The following points should be noted:
\begin{itemize}
\item  It is not possible to use BYPASS in the above, given the data
for the SCF job of Example 1. This would have resulted in
generation of a P-supermatrix which is not usable in
a CASSCF run (see Part 2, Table~1).
\item  CASSCF calculations require two scratch FORTRAN data sets, FT01
and FT02. 
\end{itemize}
In the above we have assumed that the CASSCF calculation completes in
the time allocated, with the associated direct-access files allocated in
default scratch status.  The following job is typical of that required
if restarts of the CASSCF step are envisaged:

{
\footnotesize
\begin{verbatim}
          #!/bin/csh -f
          cd /scr1/user
          setenv ed1 h2oed1
          setenv ed2 h2omain
          setenv ed3 h2odump
          setenv ed4 h2oed4
          setenv ed6 h2oed6
          setenv ed9 h2oed9
          setenv ed10 h2oed10
          /scr1/wab/GAMESS-UK/bin/gamess << EOF
          restart new
          title
          water at casscf level 3-21g basis set
          zmat angstrom
          o
          h 1 oh
          h 1 oh 2 hoh
          variables
          oh 0.956
          hoh 104.5
          end
          scftype casscf
          config print
          doc 1 to 5
          uoc 6 7
          end
          superci 1 to 8
          newton 9 to 20
          hessian 9 to 20
          simul 9 to 20
          enter
          EOF
\end{verbatim}
}
Assuming the above job terminated prior to convergence, the
calculation might be restarted as follows:

{
\footnotesize
\begin{verbatim}
          #!/bin/csh -f
          cd /scr1/user
          setenv ed1 h2oed1
          setenv ed2 h2omain
          setenv ed3 h2odump
          setenv ed4 h2oed4
          setenv ed6 h2oed6
          setenv ed9 h2oed9
          setenv ed10 h2oed10
          /scr1/wab/GAMESS-UK/bin/gamess << EOF
          restart scf
          title
          water at casscf level 3-21g basis set
          zmat angstrom
          o
          h 1 oh
          h 1 oh 2 hoh
          variables
          oh 0.956
          hoh 104.5
          end
          scftype casscf
          config bypass
          doc 1 to 5
          uoc 6 7
          end
          superci 1 to 2
          newton 3 to 20
          hessian 3 to 20
          simul 3 to 20
          vectors 6 7
          enter 6 7
          EOF
\end{verbatim}
}
where the default sections housing the CASSCF vectors and
ci coefficients (sections 6 and 7 respectively)  created
in the startup job are explicitly declared above, and CONFIG
processing is bypass'ed.
Having completed the single point calculation, the
following might be used to perform a geometry optimisation
at the CASSCF level. Note that it is now necessary to SAVE
the data set associated with ED11 if restarts are envisaged.

{
\footnotesize
\begin{verbatim}
          #!/bin/csh -f
          cd /scr1/user
          setenv ed1 h2oed1
          setenv ed2 h2omain
          setenv ed3 h2odump
          setenv ed4 h2oed4
          setenv ed6 h2oed6
          setenv ed9 h2oed9
          setenv ed10 h2oed10
          setenv ed11 h2oed11
          /scr1/wab/GAMESS-UK/bin/gamess << EOF
          restart new
          title
          water - geometry optimisation at casscf level
          zmat angstrom
          o
          h 1 oh
          h 1 oh 2 hoh
          variables
          oh 0.956
          hoh 104.5
          end
          runtype optimise
          scftype casscf
          config bypass
          doc 1 to 5
          uoc 6 7
          end
          superci 1 to 5
          newton 6 to 20
          hessian 6 to 20
          simul 6 to 20
          enter
          EOF
\end{verbatim}
}
Again the default CASSCF vector and CI coefficient sections from the
initial energy calculation will be used by default in the optimisation
job.

\section[CASSCF + 2nd-order CI Calculations on BeO]{CASSCF + 2nd-order CI Calculations on BeO}

First, we consider below a CASSCF calculation 
on the X$^{1}\Sigma^{+}$ state
of BeO, characterised by the configuration
\begin{equation}
  1\sigma^2 2\sigma^2 3\sigma^2 4\sigma^2 1\pi^4 
\end{equation}
The initial closed--shell SCF using a DZP basis was conducted with the
following job:
{
\footnotesize
\begin{verbatim}
          #!/bin/csh -f
          cd /scr1/user
          setenv ed2 beomain
          setenv ed3 beodump
          /scr1/wab/GAMESS-UK/bin/gamess << EOF
          super off nosym
          title\beo .. dzp
          zmat angstrom\be\o 1 beo
          variables\beo 1.300 hessian 0.7\end
          basis dzp
          enter
          EOF
\end{verbatim}
}
An examination of the closed-shell SCF output reveals the
following symmetry adapted basis information

{
\footnotesize
\begin{verbatim}
          =============================
          IRREP  NO. OF SYMMETRY ADAPTED
                 BASIS FUNCTIONS
          =============================
            1          17
            2           6
            3           6
            4           2
          =============================
\end{verbatim}
}
and the SCF MO ordering shown below:
{
\footnotesize
\begin{verbatim}
          ===============================================
          M.O.  IRREP  ORBITAL ENERGY   ORBITAL OCCUPANCY
          ===============================================
            1     1    -20.45769692           2.0000000
            2     1     -4.72825831           2.0000000
            3     1     -1.15792230           2.0000000
            4     1     -0.46629250           2.0000000
            5     3     -0.39257378           2.0000000
            6     2     -0.39257378           2.0000000
            7     1     -0.05704423           0.0000000
            8     2      0.09936415           0.0000000
            9     3      0.09936415           0.0000000
           10     1      0.15573442           0.0000000
           11     1      0.25156032           0.0000000
           12     3      0.29518660           0.0000000
           13     2      0.29518660           0.0000000
           14     1      0.57290305           0.0000000
           15     4      0.66957865           0.0000000
           16     1      0.66957865           0.0000000
           17     2      0.84348942           0.0000000
           18     3      0.84348942           0.0000000
           19     1      1.01733643           0.0000000
           20     2      1.05237116           0.0000000
           21     3      1.05237116           0.0000000
           22     1      1.24301610           0.0000000
           23     1      1.45883277           0.0000000
           24     1      1.80402496           0.0000000
           25     4      2.36534068           0.0000000
           26     1      2.36534068           0.0000000
           27     2      2.62307878           0.0000000
           28     3      2.62307878           0.0000000
           29     1      3.16009549           0.0000000
           30     1      4.37857805           0.0000000
           31     1     45.42979631           0.0000000
          ===============================================
\end{verbatim}
}
We wish to perform a CASSCF calculation in which the inner shell and
O2s orbitals (the 1$\sigma$--3$\sigma$) remain doubly occupied, with
the active space including the formally vacant SCF virtual MOs,
the 5$\sigma$ and 2$\pi$. This example utilises the vectors from the
closed shell SCF calculation.  We wish to perform the CASSCF calculation
under RUNTYPE CI specification where, having performed the 6 electrons
in 6 orbital CASSCF, we use the natural orbitals in carrying out a
second-order CI using the Direct-CI module. Specifically we aim to use a
reference space in the CI consisting of all CSFs which can be generated
by distributing 6 electrons in 6 MOs i.e. the CASSCF  space.  This may
be achieved in a single run through the following job specification:

{
\footnotesize
\begin{verbatim}
          #!/bin/csh -f
          cd /scr1/user
          setenv ed1 beoed1
          setenv ed2 beomain
          setenv ed3 beodump
          setenv ed4 beoed4
          setenv ed5 beoed5
          setenv ed6 beoed6
          setenv ed9 beoed9
          setenv ed10 beoed10
          /scr1/wab/GAMESS-UK/bin/gamess << EOF
          restart new
          title\beo .. dzp casscf+ci (6 electrons in 6 mos)
          bypass
          zmat angstrom\be\o 1 beo
          variables\beo 1.300 hessian 0.7\end
          basis dzp
          runtype ci
          active\4 to 31\end
          core\1 to 3\end
          scftype casscf
          thresh 4
          config print
          fzc 1 to 3
          doc 4 to 6
          uoc 7 to 9
          end
          superci 1 to 8
          newton 9 to 20
          hessian 9 to 20
          simul 9 to 20
          direct 6 6 22
          conf 
          2 2 2 0 0 0
          refgen
          1 4 1 5 1 6  2 4 2 5 2 6  3 4 3 5 3 6
          refgen
          1 4 1 5 1 6  2 4 2 5 2 6  3 4 3 5 3 6
          refgen
          1 4 1 5 1 6  2 4 2 5 2 6  3 4 3 5 3 6
          refgen
          1 4 1 5 1 6  2 4 2 5 2 6  3 4 3 5 3 6
          refgen
          1 4 1 5 1 6  2 4 2 5 2 6  3 4 3 5 3 6
          refgen
          1 4 1 5 1 6  2 4 2 5 2 6  3 4 3 5 3 6
          enter
          EOF
\end{verbatim}
}

\section[MCSCF + 2nd-order CI Calculations on BeO]{MCSCF + 2nd-order CI Calculations on BeO}

First, we consider below a MCSCF calculation 
on the X$^{1}\Sigma^{+}$ state
of BeO, characterised by the configuration
\begin{equation}
  1\sigma^2 2\sigma^2 3\sigma^2 4\sigma^2 1\pi^4 
\end{equation}
The initial closed--shell SCF using a DZP basis was conducted with the
following job:
{
\footnotesize
\begin{verbatim}
          #!/bin/csh -f
          cd /scr1/user
          setenv ed2 beomain
          setenv ed3 beodump
          /scr1/wab/GAMESS-UK/bin/gamess << EOF
          super off nosym
          title\beo .. dzp
          zmat angstrom\be\o 1 beo
          variables\beo 1.300 hessian 0.7\end
          basis dzp
          enter
          EOF
\end{verbatim}
}
An examination of the closed-shell SCF output reveals the
following symmetry adapted basis information

{
\footnotesize
\begin{verbatim}
          =============================
          IRREP  NO. OF SYMMETRY ADAPTED
                 BASIS FUNCTIONS
          =============================
            1          17
            2           6
            3           6
            4           2
          =============================
\end{verbatim}
}
and the SCF MO ordering shown below:
{
\footnotesize
\begin{verbatim}
          ===============================================
          M.O.  IRREP  ORBITAL ENERGY   ORBITAL OCCUPANCY
          ===============================================
            1     1    -20.45769692           2.0000000
            2     1     -4.72825831           2.0000000
            3     1     -1.15792230           2.0000000
            4     1     -0.46629250           2.0000000
            5     3     -0.39257378           2.0000000
            6     2     -0.39257378           2.0000000
            7     1     -0.05704423           0.0000000
            8     2      0.09936415           0.0000000
            9     3      0.09936415           0.0000000
           10     1      0.15573442           0.0000000
           11     1      0.25156032           0.0000000
           12     3      0.29518660           0.0000000
           13     2      0.29518660           0.0000000
           14     1      0.57290305           0.0000000
           15     4      0.66957865           0.0000000
           16     1      0.66957865           0.0000000
           17     2      0.84348942           0.0000000
           18     3      0.84348942           0.0000000
           19     1      1.01733643           0.0000000
           20     2      1.05237116           0.0000000
           21     3      1.05237116           0.0000000
           22     1      1.24301610           0.0000000
           23     1      1.45883277           0.0000000
           24     1      1.80402496           0.0000000
           25     4      2.36534068           0.0000000
           26     1      2.36534068           0.0000000
           27     2      2.62307878           0.0000000
           28     3      2.62307878           0.0000000
           29     1      3.16009549           0.0000000
           30     1      4.37857805           0.0000000
           31     1     45.42979631           0.0000000
          ===============================================
\end{verbatim}
}
We wish to perform a CASSCF calculation in which the inner shell and
O2s orbitals (the 1$\sigma$--3$\sigma$) remain doubly occupied, with
the active space including the formally vacant SCF virtual MOs,
the 5$\sigma$ and 2$\pi$. This example utilises the vectors from the
closed shell SCF calculation.

{
\footnotesize
\begin{verbatim}
          #!/bin/csh -f
          cd /scr1/user
          setenv ed2 beomain
          setenv ed3 beodump
          /scr1/wab/GAMESS-UK/bin/gamess << EOF
          restart new
          super off nosym
          title\beo .. dzp
          bypass
          zmat angstrom\be\o 1 beo
          variables\beo 1.300 hessian 0.7\end
          basis dzp
          scftype mcscf
          thresh 4
          mcscf
          orbital\3cor1 doc1 doc3 doc2  uoc1 uoc2 uoc3 \end
          enter
          EOF
\end{verbatim}
}
The following points should be noted:
\begin{itemize}
\item  The MCSCF Natural orbitals will be routed to section 10
of the Dumpfile on convergence, the default section used for NO output.
\item  Integral evaluation has been bypassed as the
initial SCF job specified the necessary integral format
for the subsequent SCF.
\end{itemize}
In the above we have assumed that the MCSCF calculation completes in
the time allocated, with the associated direct-access files allocated in
default scratch status.  The following job is typical of that required
if restarts of the MCSCF step are envisaged:

{
\footnotesize
\begin{verbatim}
          #!/bin/csh -f
          cd /scr1/user
          setenv ed2 beomain
          setenv ed3 beodump
          setenv ed4 beoed4
          setenv ed6 beoed6
          setenv ed13 beoed13
          /scr1/wab/GAMESS-UK/bin/gamess << EOF
          restart new
          super off nosym
          title\beo .. dzp
          bypass
          zmat angstrom\be\o 1 beo\
          variables\beo 1.300 hessian 0.7\end
          basis dzp
          scftype mcscf
          thresh 4
          mcscf
          orbital\3cor1 doc1 doc3 doc2  uoc1 uoc2 uoc3 \end
          enter
          EOF
\end{verbatim}
}
Note that the symmetry adapted list of integrals are
sorted at the outset of processing to ED13, and this file
should be preserved across restart jobs, given that the
DONT SORT data line is presented to the restart job.
Assuming the above job terminated prior to convergence, the
calculation might be restarted as follows:

{
\footnotesize
\begin{verbatim}
          #!/bin/csh -f
          cd /scr1/user
          setenv ed2 beomain
          setenv ed3 beodump
          setenv ed4 beoed4
          setenv ed6 beoed6
          setenv ed13 beoed13
          /scr1/wab/GAMESS-UK/bin/gamess << EOF
          restart scf
          super off nosym
          title\beo .. dzp
          zmat angstrom\be\o 1 beo
          variables\beo 1.300 hessian 0.7\end
          basis dzp
          scftype mcscf
          thresh 4
          mcscf
          orbital\3cor1 doc1 doc3 doc2  uoc1 uoc2 uoc3 \end
          dont sort
          enter
          /EOF
\end{verbatim}
}
Now let us consider performing the MCSCF calculation under
RUNTYPE CI specification where, having performed the 6 electrons in
6 orbital CASSCF, we use the natural orbitals in carrying out
a second-order CI using the Direct-CI module. Specifically we
aim to use a reference space in the CI consisting of all CSFs which
can be generated by distributing 6 electrons in 6 MOs i.e. the
CASSCF  space.  This may be achieved in a single run through the
following job specification:

{
\footnotesize
\begin{verbatim}
          #!/bin/csh -f
          cd /scr1/user
          setenv ed2 beomain
          setenv ed3 beodump
          /scr1/wab/GAMESS-UK/bin/gamess << EOF
          restart new
          super off nosym
          title\beo .. dzp mcscf+2nd-order ci (6 electrons in 6 mos)
          zmat angstrom\be\o 1 beo
          variables\beo 1.300 hessian 0.7\end
          basis dzp
          runtype ci
          active\4 to 31\end
          core\1 to 3\end
          scftype mcscf
          thresh 4
          mcscf
          orbital\3cor1 doc1 doc3 doc2  uoc1 uoc2 uoc3 \end
          direct 6 6 22
          conf 
          2 2 2 0 0 0
          refgen
          1 4 1 5 1 6  2 4 2 5 2 6  3 4 3 5 3 6
          refgen
          1 4 1 5 1 6  2 4 2 5 2 6  3 4 3 5 3 6
          refgen
          1 4 1 5 1 6  2 4 2 5 2 6  3 4 3 5 3 6
          refgen
          1 4 1 5 1 6  2 4 2 5 2 6  3 4 3 5 3 6
          refgen
          1 4 1 5 1 6  2 4 2 5 2 6  3 4 3 5 3 6
          refgen
          1 4 1 5 1 6  2 4 2 5 2 6  3 4 3 5 3 6
          vectors 1\enter 20 21
          EOF
\end{verbatim}
}
Note that the vectors specification is requesting that the closed-shell
SCF eigenvectors be used to initiate the MCSCF calculation. The MCSCF
natural orbitals, routed to section 10 of the Dumpfile in default, will
be used as the orbitals for the Direct-CI calculation.

\section[Table-CI calculations on the Ammonia Cation]{Table-CI calculations on the Ammonia Cation}
We consider below a Table-CI calculation of the X$^{2}$A$_{1}$ state and
1$^{2}$A$_{1}$ state of the ammonia cation.  In the first instance we
consider performing the calculation in two steps, initially the open-shell
SCF calculation followed by the MRD-CI 2-reference state calculation. We
then sub-divide the CI calculation into 5 separate steps, performing the
symmetry adaption + integral transformation, followed by configuration
selection, construction of the CI Hamiltonian, the diagonalisation and,
finally, the subsequent analysis of the CI wavefunctions.\\

{\bf Open-shell SCF Job}
{
\footnotesize
\begin{verbatim}
          #!/bin/csh -f
          cd /scr1/user
          setenv ed2 nh3main
          setenv ed3 nh3dump
          /scr1/wab/GAMESS-UK/bin/gamess << EOF
          title
          nh3+ * 3-21g * scf-energy=-55.53325817 hartree
          super off nosym
          charge 1
          mult 2
          zmat angstrom
          n
          h 1 roh
          h 1 roh 2 theta
          h 1 roh 2 theta 3 theta  1
          variables
          roh 1.03   hessian 0.7
          theta 104.2  hessian 0.2
          end
          enter
          EOF
\end{verbatim}
}
{\bf Table-CI Job}
{
\footnotesize
\begin{verbatim}
          #!/bin/csh -f
          cd /scr1/user
          setenv ed2 nh3main
          setenv ed3 nh3dump
          setenv table TABLE
          /scr1/wab/GAMESS-UK/bin/gamess << EOF
          restart
          title
          * nh3+ * 3-21g * mrdci-energies 1r -55.6393336 2r -55.4116210
          bypass scf
          charge 1
          mult 2
          zmat angstrom
          n
          h 1 roh
          h 1 roh 2 theta
          h 1 roh 2 theta 3 theta  1
          variables
          roh 1.03   hessian 0.7
          theta 104.2  hessian 0.2
          end
          runtype ci
          mrdci
          adapt 
          tran 
          table
          select  
          symmetry 1
          spin 2
          cntrl 9
          conf
          1 4 1 2 3  12
          1 3 1 2 4  12
          roots 2
          thresh 5 5
          ci 
          diag 
          extrap 3
          dthr 0.0001 0.0001
          natorb
          cive 1 2
          prop
          cive 1 2
          1 4  1 2 3  12
          1 3  1 2 4  12
          moment
          36 1 36 2 1
          enter
          EOF
\end{verbatim}
}
The orbitals employed in the CI calculation will be taken from the
default section associated with the open-shell RHF module, section 5,
that containing the energy-ordered canonicalised open-shell vectors
written on termination of the SCF process.  Note that the table keyword
will activate generation of the Table-CI data base, to be used in the
subsequent steps below. In this case  the data base will be written to
the file TABLE in /scr1 for subsequent use. A copy of this data base is
available on the DEC PW433AU, and may be accessed directly, thus:

{
\footnotesize
\begin{verbatim}
          setenv table /scr1/wab/GAMESS-UK/libs/TABLE
\end{verbatim}
}
Now let us consider dividing the above CI calculation. The following
points should be noted in this division:
\begin{itemize}
\item The SETENV lines indicate those FORTRAN
data sets that must be retained between separate runs of the 
program. We assume below that the TABLE data-base is available
from the previous job, thus omitting the TABLE step.
\item Note the use of the BYPASS keyword on the various steps
comprising the Table-CI procedure. Such a keyword is required
on {\em both} those steps already completed and those steps
to be handled in a subsequent run of the program.
\end{itemize}
{\bf Table-CI Data I. Symmetry Adaption and Integral Transformation}
{
\footnotesize
\begin{verbatim}
          #!/bin/csh -f
          cd /scr1/user
          setenv ed2 nh3main
          setenv ed3 nh3dump
          setenv table TABLE
          setenv ftn031 nh3tran
          /scr1/wab/GAMESS-UK/bin/gamess << EOF
          restart
          title
          * nh3+ * 3-21g * mrdci-energies 1r -55.6393336 2r -55.4116210
          bypass scf
          charge 1
          mult 2
          zmat angstrom
          n
          h 1 roh
          h 1 roh 2 theta
          h 1 roh 2 theta 3 theta  1
          variables
          roh 1.03   hessian 0.7
          theta 104.2  hessian 0.2
          end
          runtype ci
          mrdci
          adapt 
          tran 
          select  bypass
          symmetry 1
          spin 2
          cntrl 9
          conf
          1 4 1 2 3  12
          1 3 1 2 4  12
          roots 2
          thresh 5 5
          ci bypass
          diag bypass
          extrap 3
          dthr 0.0001 0.0001
          enter
          EOF
\end{verbatim}
}
{\bf Table-CI Job II. Configuration Selection}
{
\footnotesize
\begin{verbatim}
          #!/bin/csh -f
          cd /scr1/user
          setenv ed2 nh3main
          setenv ed3 nh3dump
          setenv table TABLE
          setenv ftn031 nh3tran
          setenv ftn033 nh3sel01
          setenv ftn034 nh3sel02
          /scr1/wab/GAMESS-UK/bin/gamess << EOF
          restart ci
          title
          * nh3+ * 3-21g * mrdci-energies 1r -55.6393336 2r -55.4116210
          bypass scf
          charge 1
          mult 2
          zmat angstrom
          n
          h 1 roh
          h 1 roh 2 theta
          h 1 roh 2 theta 3 theta  1
          variables
          roh 1.03   hessian 0.7
          theta 104.2  hessian 0.2
          end
          runtype ci
          mrdci
          adapt bypass
          tran bypass
          select  
          symmetry 1
          spin 2
          cntrl 9
          conf
          1 4 1 2 3  12
          1 3 1 2 4  12
          roots 2
          thresh 5 5
          ci bypass
          diag bypass
          extrap 3
          dthr 0.0001 0.0001
          enter
          EOF
\end{verbatim}
}
{\bf Table-CI Job III. CI Hamiltonian Construction}
{
\footnotesize
\begin{verbatim}
          #!/bin/csh -f
          cd /scr1/user
          setenv ed2 nh3main
          setenv ed3 nh3dump
          setenv table TABLE
          setenv ftn031 nh3tran
          setenv ftn033 nh3sel01
          setenv ftn034 nh3sel02
          setenv ftn035 nh3ham
          /scr1/wab/GAMESS-UK/bin/gamess << EOF
          restart ci
          title
          * nh3+ * 3-21g * mrdci-energies 1r -55.6393336 2r -55.4116210
          bypass scf
          charge 1
          mult 2
          zmat angstrom
          n
          h 1 roh
          h 1 roh 2 theta
          h 1 roh 2 theta 3 theta  1
          variables
          roh 1.03   hessian 0.7
          theta 104.2  hessian 0.2
          end
          runtype ci
          mrdci
          adapt bypass
          tran bypass
          select bypass
          symmetry 1
          spin 2
          cntrl 9
          conf
          1 4 1 2 3  12
          1 3 1 2 4  12
          roots 2
          thresh 5 5
          ci
          diag bypass
          extrap 3
          dthr 0.0001 0.0001
          enter
          EOF
\end{verbatim}
}
{\bf Table-CI Job IV. Diagonalisation}
{
\footnotesize
\begin{verbatim}
          #!/bin/csh -f
          cd /scr1/user
          setenv ed2 nh3main
          setenv ed3 nh3dump
          setenv ftn035 nh3ham
          setenv ftn036 nh3diag
          /scr1/wab/GAMESS-UK/bin/gamess << EOF
          restart ci
          title
          * nh3+ * 3-21g * mrdci-energies 1r -55.6393336 2r -55.4116210
          bypass scf
          charge 1
          mult 2
          zmat angstrom
          n
          h 1 roh
          h 1 roh 2 theta
          h 1 roh 2 theta 3 theta  1
          variables
          roh 1.03   hessian 0.7
          theta 104.2  hessian 0.2
          end
          runtype ci
          mrdci
          adapt bypass
          tran  bypass
          select bypass
          symmetry 1
          spin 2
          cntrl 9
          conf
          1 4 1 2 3  12
          1 3 1 2 4  12
          roots 2
          thresh 5 5
          ci bypass
          diag 
          extrap 3
          dthr 0.0001 0.0001
          enter
          EOF
\end{verbatim}
}
{\bf Table-CI Job V. CI Wavefunction Analysis}
{
\footnotesize
\begin{verbatim}
          #!/bin/csh -f
          cd /scr1/user
          setenv ed2 nh3main
          setenv ed3 nh3dump
          setenv ftn036 nh3diag
          /scr1/wab/GAMESS-UK/bin/gamess << EOF
          restart ci
          title
          * nh3+ * 3-21g * mrdci-energies 1r -55.6393336 2r -55.4116210
          bypass scf
          charge 1
          mult 2
          zmat angstrom
          n
          h 1 roh
          h 1 roh 2 theta
          h 1 roh 2 theta 3 theta  1
          variables
          roh 1.03   hessian 0.7
          theta 104.2  hessian 0.2
          end
          runtype ci
          mrdci
          adapt bypass
          tran bypass
          select bypass
          symmetry 1
          spin 2
          cntrl 9
          conf
          1 4 1 2 3  12
          1 3 1 2 4  12
          roots 2
          thresh 5 5
          ci bypass
          diag  bypass
          extrap 3
          dthr 0.0001 0.0001
          natorb
          cive 1 2
          prop
          cive 1 2
          1 4  1 2 3  12
          1 3  1 2 4  12
          moment
          36 1 36 2 1
          enter
          EOF
\end{verbatim}
}
\section[ECP, CASSCF and Direct-CI Calculations on \nimeth]{ECP, CASSCF and Direct-CI Calculations on \nimeth}
This example illustrates the use of CASSCF and Direct-CI calculations in
the framework of ECP studies. The molecular system under investigation
is \nimeth, with a 5-reference direct-CI calculation performed using a
CASSCF wavefunction for the lowest triplet state. Five data files are
presented below:
\begin{enumerate}
\item Start--up closed--shell SCF calculation for
the $^{1}$A$_{1}$ state. Note the SUPER directive
for compatibility with the subsequent open--shell calculation.
The ECP library file \protect \\/scr1/wab/GAMESS-UK/libs/ecplib is allocated to LFN ed0, 
with the
NIHAY and C non-local ECPs requested under control
of the PSEUDO directive.
\item Restart SCF job, with appropriate use of the
SWAP directive to converge the closed--shell SCF.
\item RHF calculation for the $^{3}$A$_{1}$ state.
\item CASSCF calculation for the $^{3}$A$_{1}$ state.
\item 5-reference Direct-CI  calculation for the $^{3}$A$_{1}$ state.
\end{enumerate}
{\bf Closed-shell SCF Start-up Job}
{
\footnotesize
\begin{verbatim}
          #!/bin/csh -f
          cd /scr1/user
          setenv ed2 mainnips
          setenv ed3 dumpnips
          setenv ed0 /scr1/wab/GAMESS-UK/libs/ecplib
          /scr1/wab/GAMESS-UK/bin/gamess << EOF
          title
          ni(cch2) 1a1 rhf ,hay's ni, bar's nm ecp's
          mult 1
          super force
          zmat angstrom
          ni
          c 1 nica
          x 2 1.0 1 90.0
          c 2 cacb 3 90.0 1 180.0
          x 4 1.0 2 90.0 3 0.0
          h 4 hcb 2 hcc 5 90.0
          h 4 hcb 2 hcc 5 -90.0
          variables
          nica 2.0895
          cacb 1.3604
          hcb 1.1047
          hcc 122.646
          end
          basis
          s h
          0.032828     13.3615
          0.231208      2.0133
          0.817238      0.4538
          s h
          1.000000      0.1233
          s c
          1.000000      0.4962
          s c
          1.000000      0.1533
          p c
          0.018534     18.1557
          0.115442      3.9864
          0.386206      1.1429
          0.640089      0.3594
          p c
          1.000000      0.1146
          s ni
          -0.4372528 0.6778
          1.1889453 0.1116
          s ni
          1.0000000 0.0387
          p ni
          1.0000000 0.0840
          p ni
          1.0000000 0.0240
          d ni
          0.0360414 42.7200
          0.1938645 11.7600
          0.4596238 3.8170
          0.5599305 1.1690
          d ni
          1.0000000 0.2836
          end
          pseudo nonlocal
          nihay  ni
          c c
          maxcyc 30
          level 2.5 15 1.0
          vectors hcore
          enter
          EOF
\end{verbatim}
}
{\bf Closed-shell SCF Restart Data}
{
\footnotesize
\begin{verbatim}
          #!/bin/csh -f
          cd /scr1/user
          setenv ed2 mainnips
          setenv ed3 dumpnips
          setenv ed0 /scr1/wab/GAMESS-UK/libs/ecplib
          /scr1/wab/GAMESS-UK/bin/gamess << EOF
          restart new
          title
          ni(cch2) 1a1 rhf restart ,hay's ni, bar's nm ecp's
          mult 1
          bypass
          super force
          zmat angstrom
          ni
          c 1 nica
          x 2 1.0 1 90.0
          c 2 cacb 3 90.0 1 180.0
          x 4 1.0 2 90.0 3 0.0
          h 4 hcb 2 hcc 5 90.0
          h 4 hcb 2 hcc 5 -90.0
          variables
          nica 2.0895
          cacb 1.3604
          hcb 1.1047
          hcc 122.646
          end
          basis
          s h
          0.032828     13.3615
          0.231208      2.0133
          0.817238      0.4538
          s h
          1.000000      0.1233
          s c
          1.000000      0.4962
          s c
          1.000000      0.1533
          p c
          0.018534     18.1557
          0.115442      3.9864
          0.386206      1.1429
          0.640089      0.3594
          p c
          1.000000      0.1146
          s ni
          -0.4372528 0.6778
          1.1889453 0.1116
          s ni
          1.0000000 0.0387
          p ni
          1.0000000 0.0840
          p ni
          1.0000000 0.0240
          d ni
          0.0360414 42.7200
          0.1938645 11.7600
          0.4596238 3.8170
          0.5599305 1.1690
          d ni
          1.0000000 0.2836
          end
          pseudo nonlocal
          nihay  ni
          c c
          maxcyc 40
          level 3.0 10 1.0
          swap
          4 5
          6 8
          8 10
          10 11
          end
          enter
          EOF
\end{verbatim}
}
{\bf Open-shell SCF Job}
{
\footnotesize
\begin{verbatim}
          #!/bin/csh -f
          cd /scr1/user
          setenv ed2 mainnips
          setenv ed3 dumpnips
          setenv ed0 /scr1/wab/GAMESS-UK/libs/ecplib
          /scr1/wab/GAMESS-UK/bin/gamess << EOF
          restart new
          title
          ni(cch2) 3a1 rhf ,hay's ni, bar's nm ecp's
          mult 3
          super force
          bypass
          zmat angstrom
          ni
          c 1 nica
          x 2 1.0 1 90.0
          c 2 cacb 3 90.0 1 180.0
          x 4 1.0 2 90.0 3 0.0
          h 4 hcb 2 hcc 5 90.0
          h 4 hcb 2 hcc 5 -90.0
          variables
          nica 2.0895
          cacb 1.3604
          hcb 1.1047
          hcc 122.646
          end
          basis
          s h
          0.032828     13.3615
          0.231208      2.0133
          0.817238      0.4538
          s h
          1.000000      0.1233
          s c
          1.000000      0.4962
          s c
          1.000000      0.1533
          p c
          0.018534     18.1557
          0.115442      3.9864
          0.386206      1.1429
          0.640089      0.3594
          p c
          1.000000      0.1146
          s ni
          -0.4372528 0.6778
          1.1889453 0.1116
          s ni
          1.0000000 0.0387
          p ni
          1.0000000 0.0840
          p ni
          1.0000000 0.0240
          d ni
          0.0360414 42.7200
          0.1938645 11.7600
          0.4596238 3.8170
          0.5599305 1.1690
          d ni
          1.0000000 0.2836
          end
          pseudo nonlocal
          nihay ni
          c c
          scftype gvb
          maxcyc 50
          level 2.0 3.0 15 1.0 1.0
          open 2 2
          swap
          11 12
          14 15
          end
          enter
          EOF
\end{verbatim}
}
Eigenvector utilisation in the above job will drive off the default
sections of the Dumpfile, with the open-shell SCF module using the
closed-shell SCF vectors from section 1 to initiate the SCF process,
and writing the SCF open-shell orbitals to sections 4 and 5 (the
energy-ordered SCF MOs). These latter orbitals will be used below
to instigate the CASSCF processing.\\

{\bf CASSCF Job}
{
\footnotesize
\begin{verbatim}
          #!/bin/csh -f
          cd /scr1/user
          setenv ed2 mainnips
          setenv ed3 dumpnips
          setenv ed0 /scr1/wab/GAMESS-UK/libs/ecplib
          setenv ed6 trannips
          setenv ed4 lfornips
          /scr1/wab/GAMESS-UK/bin/gamess << EOF
          restart new
          title
          ni(cch2) 3a1 c2v cas at opt (3 2 2 0), hay's ni, bar's nm ecp's
          mult 3
          super off nosym
          zmat angstrom
          ni
          c 1 nica
          x 2 1.0 1 90.0
          c 2 cacb 3 90.0 1 180.0
          x 4 1.0 2 90.0 3 0.0
          h 4 hcb 2 hcc 5 90.0
          h 4 hcb 2 hcc 5 -90.0
          variables
          nica 2.0895
          cacb 1.3604
          hcb 1.1047
          hcc 122.646
          end
          basis
          s h
          0.032828     13.3615
          0.231208      2.0133
          0.817238      0.4538
          s h
          1.000000      0.1233
          s c
          1.000000      0.4962
          s c
          1.000000      0.1533
          p c
          0.018534     18.1557
          0.115442      3.9864
          0.386206      1.1429
          0.640089      0.3594
          p c
          1.000000      0.1146
          s ni
          -0.4372528 0.6778
          1.1889453 0.1116
          s ni
          1.0000000 0.0387
          p ni
          1.0000000 0.0840
          p ni
          1.0000000 0.0240
          d ni
          0.0360414 42.7200
          0.1938645 11.7600
          0.4596238 3.8170
          0.5599305 1.1690
          d ni
          1.0000000 0.2836
          end
          pseudo nonlocal
          nihay  ni
          c c
          scftype casscf
          config print
          fzc 1 to 7
          doc 8 to 9
          alp 10 to 11
          uoc 12 to 14
          end
          superci 1 to 12
          newton 13 to 20
          hessian 13  to 20
          simul 15 to 20
          vectors 5
          swap
          14 15
          end
          enter
          EOF
\end{verbatim}
}
{\bf Direct-CI Job}
{
\footnotesize
\begin{verbatim}
          #!/bin/csh -f
          cd /scr1/user
          setenv ed2 mainnips
          setenv ed3 dumpnips
          setenv ed0 /scr1/wab/GAMESS-UK/libs/ecplib
          setenv ed5 cinips
          setenv ed6 trannips
          /scr1/wab/GAMESS-UK/bin/gamess << EOF
          restart new
          title
          ni(cch2) 3a1 mrsdci (3 2 2 0) ,hay's ni, bar's nm ecp's
          mult 3
          super off nosym
          bypass scf
          zmat angstrom
          ni
          c 1 nica
          x 2 1.0 1 90.0
          c 2 cacb 3 90.0 1 180.0
          x 4 1.0 2 90.0 3 0.0
          h 4 hcb 2 hcc 5 90.0
          h 4 hcb 2 hcc 5 -90.0
          variables
          nica 2.0895
          cacb 1.3604
          hcb 1.1047
          hcc 122.646
          end
          basis
          s h
          0.032828     13.3615
          0.231208      2.0133
          0.817238      0.4538
          s h
          1.000000      0.1233
          s c
          1.000000      0.4962
          s c
          1.000000      0.1533
          p c
          0.018534     18.1557
          0.115442      3.9864
          0.386206      1.1429
          0.640089      0.3594
          p c
          1.000000      0.1146
          s ni
          -0.4372528 0.6778
          1.1889453 0.1116
          s ni
          1.0000000 0.0387
          p ni
          1.0000000 0.0840
          p ni
          1.0000000 0.0240
          d ni
          0.0360414 42.7200
          0.1938645 11.7600
          0.4596238 3.8170
          0.5599305 1.1690
          d ni
          1.0000000 0.2836
          end
          pseudo nonlocal
          nihay  ni
          c c
          runtype ci
          core
          end
          active
          1 to 40
          end
          scftype casscf
          config print
          fzc 1 to 7
          doc 8 to 9
          alp 10 to 11
          uoc 12 to 14
          end
          superci 1 to 12
          newton 13 to 20
          hessian 13  to 20
          simul 15 to 20
          direct 20 14 26
          spin triplet
          conf
          2 2 2 2 2 2 2 2 2 1 1 0 0 0
          2 2 2 2 2 2 2 2 1 2 0 0 1 0
          2 2 2 2 2 2 2 2 1 1 1 0 1 0
          2 2 2 2 2 2 2 2 0 1 1 0 2 0
          2 2 2 2 2 2 2 1 2 2 0 1 0 0
          2 2 2 2 2 2 2 1 2 1 1 1 0 0
          vprint 100 0.01
          maxcyc 10
          enter
          EOF
\end{verbatim}
}
\section[Table-CI Calculations of the Electronic Spectra of Pyridine]{Table-CI Calculations of the Electronic Spectra of Pyridine}
This example demonstrates the use of the Table-CI module in the
calculation of the low-lying states of Pyridine.  Specifically, we are
involved in determining the disposition of the first ten $^{1}$A$_{1}$ and
$^{1}$A$_{2}$ states, using a common set of orbitals (the X$^{1}$A$_{1}$
SCF-MOs) in a DZ plus Rydberg basis set of 91 functions.  Five job files
are presented below:
\begin{enumerate}
\item Start--up closed--shell SCF calculation for
the X$^{1}$A$_{1}$ state. Note the SUPER directive
for compatibility with the subsequent CI calculation.
Note also the particular syntax for siting the DZ basis on H: the
third and fourth data fields are to provide an unscaled
hydrogen basis, since the default specification will scale the
two components by 1.2 (the more contracted) and 1.15 (the more
diffuse component)
\item 1M/1R Table-CI calculation of the X$^{1}$A$_{1}$ state.
\item 6M/1R Table-CI calculation of the X$^{1}$A$_{1}$ state.
\item 21M/10R Table-CI calculation of the ten lowest $^{1}$A$_{1}$ states.
\item 19M/10R Table-CI calculation of the ten lowest $^{1}$A$_{2}$ states.
\end{enumerate}
{\bf 1.~~Closed-shell SCF Job}
{
\footnotesize
\begin{verbatim}
          #!/bin/csh -f
          cd /scr1/user
          setenv ed2 pyred2
          setenv ed3 pyred3
          /scr1/wab/GAMESS-UK/bin/gamess << EOF
          title
          pyridine dz+bond-centred functions
          super off nosym
          zmat angstrom
          n
          x 1 1.0
          x 1 1.0 2 90.
          x 1 1.0 2 90. 3 90.
          c 1 c4n 3 90. 2 180.
          x 5 1.0 1 90. 3 0.0
          x 5 1.0 1 90. 4 0.0
          h 5 ch4 6 90. 1 180.
          c 1 c2n 2 c2nz 3 180.
          c 1 c2n 2 c2nz 3 0.0
          c 9 c2c3 1 ccn 2 180.
          c 10 c2c3 1 ccn 2 180.
          h 9 c2h6 1 nch2 2 0.0
          h 10 c2h6 1 nch2 2 0.0
          h 11 c3h5 9 c2c3h 1 180.
          h 12 c3h5 10 c2c3h 1 180.
          bq 1 1.39 3 90. 2 180.
          variables
          c4n 2.7845546
          ch4 1.0823078
          c2n 1.3372389
          c2nz 120.641858
          c2c3 1.3944571
          ccn 122.662269
          c2h6 1.0814291
          c3h5 1.0809550
          nch2 116.400433
          c2c3h 120.158516
          end
          basis
          dz h 1.0 1.0
          dz n
          dz c
          s bq
          1.0 0.021
          s bq
          1.0 0.008
          s bq
          1.0 0.0025
          p bq
          1.0 0.017
          p bq
          1.0 0.009
          d bq
          1.0 0.015
          d bq
          1.0 0.008
          end
          enter
          EOF
\end{verbatim}
}
{\bf 2.~~1M/1R Table-CI Job for the X$^{1}$A$_{1}$ State}\\

An examination of the SCF output reveals the following symmetry adapted
basis functions,  
given the C$_{2v}$ geometry and DZ plus Rydberg basis set:

{
\footnotesize
\begin{verbatim}
          =============================
          IRREP  NO. OF SYMMETRY ADAPTED
                 BASIS FUNCTIONS
          =============================
            1          45
            2          12
            3          28
            4           6
          =============================
\end{verbatim}
}
Thus the orbital reordering performed by the Table-CI module
will yield the sequence numbers 1--45 for the a$_{1}$ MOs,
46--57 for the b$_{1}$ MOs, 58--85 for the b$_{2}$ MOs and 
86--91 for the a$_{2}$ MOs. We are both freezing and discarding
orbitals in the subsequent CI calculations. Confining this to
both a$_{1}$ and b$_{2}$ MOs, the first six orbitals of a$_{1}$ 
symmetry and first four orbitals of b$_{2}$ symmetry are to be frozen, 
and the eight highest energy orbitals of a$_{1}$ symmetry 
and six highest of b$_{2}$ are to be discarded.
Thus the final sequence numbers for the active 
orbitals in the CI are 1--31 for the a$_{1}$ MOs,
32--43 for the b$_{1}$ MOs, 44--61 for the b$_{2}$ MOs and 
62--68 for the a$_{2}$ MOs.  The CONF data line specifying
the reference configuration is based on the associated
sequence numbers of these active orbitals.

{
\footnotesize
\begin{verbatim}
          #!/bin/csh -f
          cd /scr1/user
          setenv ed2 pyred2
          setenv ed3 pyred3
          setenv ftn031 pyrtran
          setenv table /scr1/wab/GAMESS-UK/libs/TABLE
          /scr1/wab/GAMESS-UK/bin/gamess << EOF
          restart
          title
          pyridine dz+bond-centred functions
          super off nosym
          bypass scf
          zmat angstrom
          n
          x 1 1.0
          x 1 1.0 2 90.
          x 1 1.0 2 90. 3 90.
          c 1 c4n 3 90. 2 180.
          x 5 1.0 1 90. 3 0.0
          x 5 1.0 1 90. 4 0.0
          h 5 ch4 6 90. 1 180.
          c 1 c2n 2 c2nz 3 180.
          c 1 c2n 2 c2nz 3 0.0
          c 9 c2c3 1 ccn 2 180.
          c 10 c2c3 1 ccn 2 180.
          h 9 c2h6 1 nch2 2 0.0
          h 10 c2h6 1 nch2 2 0.0
          h 11 c3h5 9 c2c3h 1 180.
          h 12 c3h5 10 c2c3h 1 180.
          bq 1 1.39 3 90. 2 180.
          variables
          c4n 2.7845546
          ch4 1.0823078
          c2n 1.3372389
          c2nz 120.641858
          c2c3 1.3944571
          ccn 122.662269
          c2h6 1.0814291
          c3h5 1.0809550
          nch2 116.400433
          c2c3h 120.158516
          end
          basis
          dz h 1.0 1.0
          dz n
          dz c
          s bq
          1.0 0.021
          s bq
          1.0 0.008
          s bq
          1.0 0.0025
          p bq
          1.0 0.017
          p bq
          1.0 0.009
          d bq
          1.0 0.015
          d bq
          1.0 0.008
          end
          runtype ci
          mrdci
          adapt
          tran freeze discard
          6 0 4 0
          1 to 6 1 to 4
          8 0 6 0
          38 to 45 23 to 28
          select
          cntrl 22
          spin singlet
          symmetry 1
          conf
          0 1 2 3 4 5 32 33 44 45 46 62
          roots 1 1
          thresh 30 10
          ci
          diag
          title
          pyridine 1m1r ground state   dz + 3s2p2d rydberg basis
          extrap 3
          natorb
          civec 1
          putq aos 2
          enter
          EOF
\end{verbatim}
}
{\bf 3.~~6M/1R Table-CI Job for the X$^{1}$A$_{1}$ State}
{
\footnotesize
\begin{verbatim}
          #!/bin/csh -f
          cd /scr1/user
          setenv ed2 pyred2
          setenv ed3 pyred3
          setenv ftn031 pyrtran
          setenv table /scr1/wab/GAMESS-UK/libs/TABLE
          /scr1/wab/GAMESS-UK/bin/gamess << EOF
          restart
          title
          pyridine dz+bond-centred functions
          super off nosym
          bypass scf
          zmat angstrom
          n
          x 1 1.0
          x 1 1.0 2 90.
          x 1 1.0 2 90. 3 90.
          c 1 c4n 3 90. 2 180.
          x 5 1.0 1 90. 3 0.0
          x 5 1.0 1 90. 4 0.0
          h 5 ch4 6 90. 1 180.
          c 1 c2n 2 c2nz 3 180.
          c 1 c2n 2 c2nz 3 0.0
          c 9 c2c3 1 ccn 2 180.
          c 10 c2c3 1 ccn 2 180.
          h 9 c2h6 1 nch2 2 0.0
          h 10 c2h6 1 nch2 2 0.0
          h 11 c3h5 9 c2c3h 1 180.
          h 12 c3h5 10 c2c3h 1 180.
          bq 1 1.39 3 90. 2 180.
          variables
          c4n 2.7845546
          ch4 1.0823078
          c2n 1.3372389
          c2nz 120.641858
          c2c3 1.3944571
          ccn 122.662269
          c2h6 1.0814291
          c3h5 1.0809550
          nch2 116.400433
          c2c3h 120.158516
          end
          basis
          dz h 1.0 1.0
          dz n
          dz c
          s bq
          1.0 0.021
          s bq
          1.0 0.008
          s bq
          1.0 0.0025
          p bq
          1.0 0.017
          p bq
          1.0 0.009
          d bq
          1.0 0.015
          d bq
          1.0 0.008
          end
          runtype ci
          mrdci
          adapt bypass
          tran freeze discard bypass
          6 0 4 0
          1 to 6 1 to 4
          8 0 6 0
          38 to 45 23 to 28
          select
          cntrl 22
          spin singlet
          symmetry 1
          conf
          0 1 2 3 4 5 32 33 44 45 46 62
          0 1 2 3 4 5 32 38 44 45 46 62
          0 1 2 3 4 5 32 33 44 45 46 65
          4 32 33 38 39  1 2 3 4 5  44 45 46 62
          4 32 39 62 65  1 2 3 4 5  33 44 45 46 
          4 33 38 62 65  1 2 3 4 5  32 44 45 46 
          roots 1 1
          thresh 30 10
          ci
          diag
          title
          pyridine 6m1r ground state   dz + 3s2p2d rydberg basis
          extrap 3
          natorb
          civec 1
          putq aos 2
          enter
          EOF
\end{verbatim}
}
{\bf 4.~~21M/10R Table-CI Job for the $^{1}$A$_{1}$ States}
{
\footnotesize
\begin{verbatim}
          #!/bin/csh -f
          cd /scr1/user
          setenv ed2 pyred2
          setenv ed3 pyred3
          setenv ftn031 pyrtran
          setenv table /scr1/wab/GAMESS-UK/libs/TABLE
          /scr1/wab/GAMESS-UK/bin/gamess << EOF
          restart
          title
          pyridine   dz+rydberg basis
          super off nosym
          bypass scf
          zmat angstrom
          n
          x 1 1.0
          x 1 1.0 2 90.
          x 1 1.0 2 90. 3 90.
          c 1 c4n 3 90. 2 180.
          x 5 1.0 1 90. 3 0.0
          x 5 1.0 1 90. 4 0.0
          h 5 ch4 6 90. 1 180.
          c 1 c2n 2 c2nz 3 180.
          c 1 c2n 2 c2nz 3 0.0
          c 9 c2c3 1 ccn 2 180.
          c 10 c2c3 1 ccn 2 180.
          h 9 c2h6 1 nch2 2 0.0
          h 10 c2h6 1 nch2 2 0.0
          h 11 c3h5 9 c2c3h 1 180.
          h 12 c3h5 10 c2c3h 1 180.
          bq 1 1.39 3 90. 2 180.
          variables
          c4n 2.7845546
          ch4 1.0823078
          c2n 1.3372389
          c2nz 120.641858
          c2c3 1.3944571
          ccn 122.662269
          c2h6 1.0814291
          c3h5 1.0809550
          nch2 116.400433
          c2c3h 120.158516
          end
          basis
          dz h 1.0 1.0
          dz n
          dz c
          s bq
          1.0 0.021
          s bq
          1.0 0.008
          s bq
          1.0 0.0025
          p bq
          1.0 0.017
          p bq
          1.0 0.009
          d bq
          1.0 0.015
          d bq
          1.0 0.008
          end
          runtype ci
          mrdci
          adapt bypass
          tran freeze discard bypass
          6 0 4 0
          1 to 6 1 to 4
          8 0 6 0
          38 to 45 23 to 28
          select
          cntrl 22
          spin singlet
          symmetry 1
          conf
          2 62 65 1 2 3 4 5 32 33 44 45 46
          2 33 34 1 2 3 4 5 32 44 45 46 62
          2 5 8 1 2 3 4 32 33 44 45 46 62
          2 62 63 1 2 3 4 5 32 33 44 45 46
          2 33 38 1 2 3 4 5 32 44 45 46 62
          2 33 35 1 2 3 4 5 32 44 45 46 62
          2 5 7 1 2 3 4 32 33 44 45 46 62
          2 33 37 1 2 3 4 5 32 44 45 46 62
          2 5 9 1 2 3 4 32 33 44 45 46 62
          2 62 64 1 2 3 4 5 32 33 44 45 46
          2 33 36 1 2 3 4 5 32 44 45 46 62
          2 5 11 1 2 3 4 32 33 44 45 46 62
          2 33 39 1 2 3 4 5 32 44 45 46 62
          2 32 34 1 2 3 4 5 33 44 45 46 62
          2 32 38 1 2 3 4 5 33 44 45 46 62
          4 5 8 33 38 1 2 3 4 32 44 45 46 62
          4 33 38 62 65 1 2 3 4 5 32 44 45 46
          4 32 33 34 38 1 2 3 4 5 44 45 46 62
          4 33 34 62 65 1 2 3 4 5 32 44 45 46
          4 32 38 62 63 1 2 3 4 5 33 44 45 46
          4 5 7 33 38 1 2 3 4 32 44 45 46 62
          roots 10 1 2 3 4 5 6 7 8 9 10
          thresh 30 10
          ci
          diag
          title
          pyridine 21m10r 1a1    dz + 3s2p2d rydberg basis
          extrap 3
          enter
          EOF
\end{verbatim}
}
{\bf 5.~~19M/10R Table-CI Job for the $^{1}$A$_{2}$ States}
{
\footnotesize
\begin{verbatim}
          #!/bin/csh -f
          cd /scr1/user
          setenv ed2 pyred2
          setenv ed3 pyred3
          setenv ftn031 pyrtran
          setenv table /scr1/wab/GAMESS-UK/libs/TABLE
          /scr1/wab/GAMESS-UK/bin/gamess << EOF
          restart
          title
          pyridine  dz+rydberg basis
          super off nosym
          bypass scf
          zmat angstrom
          n
          x 1 1.0
          x 1 1.0 2 90.
          x 1 1.0 2 90. 3 90.
          c 1 c4n 3 90. 2 180.
          x 5 1.0 1 90. 3 0.0
          x 5 1.0 1 90. 4 0.0
          h 5 ch4 6 90. 1 180.
          c 1 c2n 2 c2nz 3 180.
          c 1 c2n 2 c2nz 3 0.0
          c 9 c2c3 1 ccn 2 180.
          c 10 c2c3 1 ccn 2 180.
          h 9 c2h6 1 nch2 2 0.0
          h 10 c2h6 1 nch2 2 0.0
          h 11 c3h5 9 c2c3h 1 180.
          h 12 c3h5 10 c2c3h 1 180.
          bq 1 1.39 3 90. 2 180.
          variables
          c4n 2.7845546
          ch4 1.0823078
          c2n 1.3372389
          c2nz 120.641858
          c2c3 1.3944571
          ccn 122.662269
          c2h6 1.0814291
          c3h5 1.0809550
          nch2 116.400433
          c2c3h 120.158516
          end
          basis
          dz h 1.0 1.0
          dz n
          dz c
          s bq
          1.0 0.021
          s bq
          1.0 0.008
          s bq
          1.0 0.0025
          p bq
          1.0 0.017
          p bq
          1.0 0.009
          d bq
          1.0 0.015
          d bq
          1.0 0.008
          end
          runtype ci
          mrdci
          adapt bypass
          tran freeze discard bypass
          6 0 4 0
          1 to 6 1 to 4
          8 0 6 0
          38 to 45 23 to 28
          select
          cntrl 22
          spin singlet
          symmetry 4
          conf
          2 5 65 1 2 3 4 32 33 44 45 46 62
          2 33 47 1 2 3 4 5 32 44 45 46 62
          2 5 63 1 2 3 4 32 33 44 45 46 62
          2 33 48 1 2 3 4 5 32 44 45 46 62
          2 8 62 1 2 3 4 5 32 33 44 45 46
          2 7 62 1 2 3 4 5 32 33 44 45 46
          2 33 50 1 2 3 4 5 32 44 45 46 62
          2 5 64 1 2 3 4 32 33 44 45 46 62
          2 33 49 1 2 3 4 5 32 44 45 46 62
          2 6 62 1 2 3 4 5 32 33 44 45 46
          4 5 33 38 65 1 2 3 4 32 44 45 46 62
          4 5 33 38 63 1 2 3 4 32 44 45 46 62
          4 33 47 62 65 1 2 3 4 5 32 44 45 46
          4 5 33 34 65 1 2 3 4 32 44 45 46 62
          4 7 33 38 62 1 2 3 4 5 32 44 45 46
          4 33 47 62 63 1 2 3 4 5 32 44 45 46
          4 8 33 38 62 1 2 3 4 5 32 44 45 46
          4 5 33 34 63 1 2 3 4 32 44 45 46 62
          4 33 48 62 63 1 2 3 4 5 32 44 45 46
          roots 10 1 2 3 4 5 6 7 8 9 10
          thresh 30 10
          ci
          diag
          title
          pyridine 19m10r 1a2  dz + 3s2p2d rydberg basis
          extrap 3
          enter
          EOF
\end{verbatim}
}

\section[Full-CI calculations]{Full-CI calculations}

We consider below  Full-CI calculations of the X$^{1}$A$_{1}$ state of
the \water\ molecule.  In the first instance we consider correlating
all electrons.  We then perform a valence-only calculation, freezing the
O1s orbital through the ACTIVE and CORE directives, specifying a total
of 8 electrons on the FULLCI data line.\\

{\bf All-electron Job}
{
\footnotesize
\begin{verbatim}
          #!/bin/csh -f
          cd /scr1/user
          setenv ed2 h2omain
          setenv ed3 h2odump
          setenv ed6 h2otran
          setenv ftn008 file8
          /scr1/wab/GAMESS-UK/bin/gamess << EOF
          title
          h2o - DZ basis - full-ci
          super off nosym
          zmat angstrom\o\h 1 roh\h 1 roh 2 theta
          variables\roh 0.956 hess 0.7\theta 104.5 hess 0.2 \end
          basis dz
          runtype ci\fullci 14 5 5
          enter 
          EOF
\end{verbatim}
}
Assuming the above job did not complete in the time allocated,
and dumped to disk in a controlled fashion, the following job
would act to continue the processing, assuming that the
FORTRAN file file8 had been saved, along with the Mainfile,
Dumpfile and Transformed integral file.\\

{\bf Restarting the Full-CI job}
{
\footnotesize
\begin{verbatim}
          #!/bin/csh -f
          cd /scr1/user
          setenv ed2 h2omain
          setenv ed3 h2odump
          setenv ed6 h2otran
          setenv ftn008 file8
          /scr1/wab/GAMESS-UK/bin/gamess << EOF
          restart ci
          title
          h2o - DZ basis - restart full-ci
          super off nosym
          zmat angstrom\o\h 1 roh\h 1 roh 2 theta
          variables\roh 0.956 hess 0.7\theta 104.5 hess 0.2 \end
          basis dz
          runtype ci\fullci 14 5 5
          enter
          EOF
\end{verbatim}
}

{\bf Valence-electron Job}
{
\footnotesize
\begin{verbatim}
          #!/bin/csh -f
          cd /scr1/user
          setenv ed2 h2omain
          setenv ed3 h2odump
          setenv ftn008 file8
          /scr1/wab/GAMESS-UK/bin/gamess << EOF
          core 8000000
          restart new
          title
          h2o - DZ basis - valence full-ci
          super off nosym
          bypass
          zmat angstrom\o\h 1 roh\h 1 roh 2 theta
          variables\roh 0.956 hess 0.7\theta 104.5 hess 0.2 \end
          basis dz
          runtype ci
          active\2 to 14 end\core\1\end
          fullci 13 4 4
          enter
          EOF
\end{verbatim}
}

The following points should be noted:
\begin{itemize}
\item the use of setenv to allocate the FORTRAN stream ftn008;
\item the use of the core pre-directive to specify
memory requirements.
\end{itemize}

\end{document}
