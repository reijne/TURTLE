\subsection{Driver}

The diesel-driver was designed to simplify the application of an individually
selecting MR-CI calculation.
It handles the following tasks:
\begin{enumerate}
\item management of calculation of several multiplicities and irreps
within a well defined directory structure
\item iterative generation of the MR-CI space (selection of References)
\item run selector and diagonalisator automatically on given thresholds
\item calculation of density matrices and properties
\item prepare results in a condensed form.
\end{enumerate}



\subsubsection{Invocation}
{\tt diesel} 

\subsubsection{Command Line Arguments}
\begin{itemize}
\item none
\end{itemize}





\subsubsection{Files}


\begin{tabular}{r|p{3cm}|r|p{5cm}}
status &name &type &content \cr
\TspaceII
\noalign{\hrule}
\TspaceII
in & stdin & ASCII & user input \cr
\TspaceII
in & "\$MOLCASRootDir\-/fort.31"%\footnote{or whatever has been 
%specified in the "MOIntegralFilename"-assignment} 
&binary&integrals in MO basis (AO$\to$MO transformation performed),
point symmetry, number of orbitals in certain irreducible representation\cr
\TspaceII
\TspaceII
out & stderr & ASCII &protocol output, progress indication\cr
out& stdout & ASCII &results (wave function, energies, $\ldots$)\cr
\end{tabular}




\subsubsection{User Input and Keywords}
\label{dieselInput}


\label{DriverKeywords}
\begin{tabular}{p{3cm}|c|p{1.5cm}|p{1.5cm}|p{5cm}}
&sta- &de-&argument &\\
keyword	&tus&fault&type &description\\[5pt]
\hline&&&&\\[-9pt]

%\begin{tabular}{r|p{1.5cm}|p{2cm}|p{5cm}}
%keyword	&status &argument &description\cr
%&&type &\cr
%\TspaceII
%\noalign{\hrule}
%\TspaceII
%{\tt Precision}	&optional &"float", "double" &
%precision of matrix elements and vectors 
%(float: 32 bit\footnote{6.9 significant digits with IEEE floating point arithmetic},
%double: 64 bit\footnote{15.5 significant digits with IEEE floating point arithmetic}),
%using 32 bit-precision will result in less memory consumption (especially
%when using stored Hamilton Matrices), and slightly faster execution, but
%may cause convergence problems. 
%\cr
%&& &if missing, defaults to "double"\cr
%\TspaceII
%{\tt IterationMode}	&optional &"CI", "ACPF", &
%iteration mode\cr
%&&"AQCC" &if missing, defaults to "CI"\cr
%\end{tabular}






{\tt Multiplicities}	&req. &-- &\key{numSet} &
multiplicities to be calculated\tnl
{\tt IrReps}	&req. &-- &\key{numSet} &
irreducible representations to be calculated\tnl
{\tt fullMRCI\-Extrapolation}	&req. & &\key{Extrapola\-tionSet} &
methods to be used for extrapolation\tnl
{\tt useNatural\-Orbitals}	&req. &no &\key{boolean} &\tnl
{\tt NaturalOrbital\-Selection\-Threshold}	&opt. &--  &\key{floatNum} &
selection threshold to be used in natural orbital calculation\tnl
{\tt averagedNatural\-Orbitals}	&opt. &no  &\key{boolean} &
use state averaged natural orbitals\tnl
{\tt property\-Thresholds}	&opt. & \{\} &\key{floatSet} &
thresholds to be used in property calculation\tnl
{\tt orbitalFile}	&req. &  &\key{name} &
use state averaged natural orbitals\\
\end{tabular}
\bigskip

Additionally the following keywords are passed to the called subprograms.
They are grouped by the subprograms.
\begin{itemize}
\item{\bf Selector:}\\
{\tt SelectionThresholds},
{\tt NumberOfElectrons},
{\tt ExcitationLevel},
{\tt selectInternal},
{\tt selectNthExcitation},
{\tt MORestrictions},
{\tt MOEquivalence},
{\tt SelectionEstimationMode}

\item{\bf Diagonalisator:}\\
{\tt ReferenceThreshold},
{\tt PTReferenceThreshold},
{\tt ConvergenceEnergyChange},
{\tt ConvergenceEigenvectorChange},
{\tt MaxIters},
{\tt MaxHamiltonStorageMem}

\item{\bf MR-PT:}\\
{\tt MRPTInhomogenityThreshold},
{\tt MRPTSelectionThresholds}
\end{itemize}

For further information on these keywords please see sections 
\ref{SelectorKeywords},
\ref{DiagonalisatorKeywords} and
\ref{MRPTKeywords} respectively.

\subsubsection{Input Example}


\begin{footnotesize}
\begin{verbatim}
# MO integral file
MOIntegralFilename               = fort.31
MOIntegralFileFormat             = New
MOLCASRootDir                    = /XYZ
# MOs
MORestrictions                   = none
MOEquivalence                    = auto

# electrons / state
NumberOfElectrons                = 24
Multiplicities                   = { 1 3 }
IrReps                           = { 0 1 2 3 4 5 6 7 }
Roots                            = { 1 2 3 4 5 6 7 8 }

# selection
SelectionThresholds              = { 1e-3 1e-4 1e-5 1e-6 1e-7 }
SelectionEstimationMode          = EpsteinNesbet
selectInternal                   = no
selectNthExcitation              = { }
ExcitationLevel                  = 2

# reference generation
ReferenceThreshold               = 0.004
maxRefGenIters                   = 6


# diagonalization / convergence
MaxDavidsonIters                 = 40
ConvergenceEnergyChange          = 1
ConvergenceEigenvectorChange     = 1

# natural orbitals
#useNaturalOrbitals               = yes
#NaturalOrbitalSelectionThreshold = 1e-6
#averagedNaturalOrbitals          = no

# extrapolation
fullMRCIExtrapolation            = { EpsteinNesbet }
#MRPTInhomogenityThreshold        = 1e-4
MRPTSelectionThresholds          = { 1e-3 1e-4 1e-5 1e-6 }

# properties
propertyThresholds               = { 1e-3 1e-4 1e-5 1e-6 }
orbitalFile                      = RASORB

MaxHamiltonStorageMem            = 300MB
\end{verbatim}
\end{footnotesize}

