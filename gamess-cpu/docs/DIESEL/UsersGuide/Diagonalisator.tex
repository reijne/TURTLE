\subsection{Diagonalisator}

\subsubsection{Invocation}
{\tt diag} {\it arguments}


\subsubsection{Command Line Arguments}
\begin{itemize}
\item {\tt -p procs}: number of parallel processes to be used in 
Davidson diagonalisation
\item {\tt -i}: read input from standard input (otherwise no user input
is read)
\item {\tt -s} {\it thresh}: 
read start vectors from a previous smaller calculation at threshold {\it thresh}.
This depends on the existence of the files "ConfTree.dat.{\it thresh}" and
"Eigenvectors.dat.{\it thresh}"
\item {\tt -r}: restart a previously aborted calculation.
This depends on the existence of the files "Davidson\_b.dat" and
"Davidson\_Ab.dat".
\item {\tt -w}: write Hamiltonmatrix to file "MatrixStorageWrite.dat".
No Davidson iteration is performed.
\end{itemize}
 

\subsubsection{Files}



\begin{tabular}{r|r|r|p{7cm}}
status &name &type &content \cr
\TspaceII
\noalign{\hrule}
\TspaceII
in & stdin & ASCII & user input 
(roots, reference selection threshold, etc.), see \ref{DiagInput}\cr
\TspaceII
in & "ConfTree.dat" & ASCII &tree of selected configurations\cr
\TspaceII
in & "fort.31"%\footnote{or whatever has been 
%specified in the "MOIntegralFilename"-assignment} 
&binary&integrals in MO basis (AO$\to$MO transformation performed),
point symmetry,
number of orbitals in certain irreducible representation\cr
\TspaceII
\TspaceII
out & stdout & ASCII &program output\cr
&&& (containing protocol, eigenvalues, etc.,\cr
&&& depending on verbosity level)\cr
\TspaceII
out & "Eigenvectors.dat" & binary &eigenvectors of ci-matrix\cr
\TspaceII
\TspaceII
temporary & "Davidson\underbar{ }b.dat" & binary &basis vectors in Davidson diagonalisation\cr
temporary & "Davidson\underbar{ }Ab.dat" & binary &mapped basis vectors\cr

\end{tabular}

\subsubsection{User Input and Keywords}
\label{DiagInput}

\label{DiagonalisatorKeywords}
\begin{supertabular}{p{3cm}|c|p{1.5cm}|p{1.5cm}|p{5cm}}
&sta- &de-&argument &\\
keyword	&tus&fault&type &description\\[5pt]
\hline&&&&\\[-9pt]
{\tt Reference\-Threshold}	&opt. & 0.004 &\key{floatnum} &
threshold for automatic reference space generation\tnl
{\tt PTReference\-Threshold}	&opt. & 0.004 &\key{floatnum} &
threshold for automatic generation 
of 0th-order wave function used in perturbation theory\tnl
{\tt Convergence\-EnergyChange}	&opt. &1e-5 &\key{floatnum} &
energy change that is sufficient for convergence of a certain root\tnl
{\tt Convergence\-EigenvectorChange}	&opt. &1 &\key{floatnum} &
eigenvector change that is sufficient for convergence of a certain root\tnl
{\tt MaxIters}	&opt. &20 &\key{intnum} &
maximum number of iterations\tnl
{\tt Roots}	&opt. & selector output &\key{numSet} &
set of roots to be used in the diagonalisation procedure\tnl
{\tt RootHoming}	&opt. & no &\key{boolean} &
freeze roots to characters given by the reference space\tnl
{\tt StorePTEnergy}	&opt. &no &\key{bool} &
flag if perturbational energies are stored in configuration tree\tnl
{\tt StorePTCoef}	&opt. &no &\key{bool} &
flag if perturbational CI coefficients are stored in configuration tree\tnl
{\tt MaxHamilton\-StorageMem}	&opt. &0 &\key{intnum} \{GB, MB, KB\} &
maximum amount of memory available to store Hamilton Matrix;
if there is not enough memory calculation falls back to direct mode\tnl
{\tt Precision}	&opt. &double &"float", "double" &
precision of matrix elements and vectors 
(float: 32 bit\footnote{6.9 significant digits with IEEE floating point arithmetic},
double: 64 bit\footnote{15.5 significant digits with IEEE floating point arithmetic}),
using 32 bit-precision will result in less memory consumption (especially
when using stored Hamilton Matrices), and slightly faster execution, but
may cause convergence problems. \tnl
{\tt IterationMode}	&opt. &"CI" &"CI", "ACPF", "AQCC" &
iteration mode\tnl
\end{supertabular}
\bigskip

Since all input keywords are optional, the input may be missing completely.
Therefore by default no input is read. To make the program
read any user input from stdin an explicit flag ({\tt -i}) must be specified.
