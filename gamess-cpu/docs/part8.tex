\documentclass[11pt,fleqn]{article}

\usepackage{hyperref}

% package HTML requires Latex2HTML to be installed for html.sty
\usepackage{html}
\newcommand{\doi}[1]{doi:\href{http://dx.doi.org/#1}{#1}}
\begin{htmlonly}
\renewcommand{\href}[2]{\htmladdnormallink{#2}{#1}}
\end{htmlonly}
\hypersetup{colorlinks,
            %citecolor=black,
            %filecolor=black,
            %linkcolor=black,
            %urlcolor=black,
            bookmarksopen=true,
            pdftex}
 
\addtolength{\textwidth}{1.0in}
\addtolength{\oddsidemargin}{-0.5in}
\addtolength{\topmargin}{-0.5in}
\addtolength{\textheight}{1.0in}
\newcommand{\formaldehyde}{\mbox{H$_{2}$CO}}
\newcommand{\trstate}{\mbox{$^{3}$A$^{2}$}}

\pagestyle{headings}
\pagenumbering{roman}
\begin{document}
\sf
\parindent 0cm
\parskip 1ex
\begin{flushleft}
Computing for Science (CFS) Ltd.,\\CCLRC Daresbury Laboratory.\\[0.30in]
{\large Generalised Atomic and Molecular Electronic Structure System }\\[.2in]
\rule{150mm}{3mm}\\
\vspace{.2in}
{\huge G~A~M~E~S~S~-~U~K}\\[.3in]
{\huge USER'S GUIDE~~and}\\[.2in]
{\huge REFERENCE MANUAL}\\[0.2in]
{\huge Version 8.0~~~June 2008}\\ [.2in]
{\large PART 8. WAVEFUNCTION ANALYSIS and MODELS for SOLVATION}\\
\vspace{.1in}
{\large M.F. Guest, J. Kendrick, J.H. van Lenthe, P. Sherwood and A.H. de Vries}\\[0.2in]
 
Copyright (c) 1993-2008 Computing for Science Ltd.\\[.1in]
This document may be freely reproduced provided that it is reproduced\\
unaltered and in its entirety.\\
\vspace{.2in}
\rule{150mm}{3mm}\\
\end{flushleft}

% 
\tableofcontents

\newpage

\pagenumbering{arabic}

\section[Introduction]{Introduction}

In the first part of this chapter we provide a description of the analysis
options and associated data input available within GAMESS-UK, including (i)
the calculation of 1-electron properties and localised molecular orbitals,
(ii) the graphical analysis of wavefunctions by the calculation of charge
densities, molecular orbitals, atom difference densities and electrostatic
potentials on a grid of points, (iii) the calculation of potential derived
charges with electrostatic potential data calculated using the graphics
module to generate least-squares fitted point charges at the nuclei,
and (iv) performing both Mulliken and Distributed Multipole Analyses.

We then describe the capabilities and data input associated with
the (Direct) Reaction Field (DRF) model for solvation. This model,
developed at the University of Groningen \cite{vries,duijnen}, is an
embedding technique enabling the computation of the interaction between
a quantum-mechanically described molecule and its classically described
surroundings.

\section[Analysis Modules - Introduction]{Analysis Modules - Introduction}

The analysis modules of GAMESS--UK are requested under control of the
RUNTYPE ANALYSE directive. Note that at present only one
mode of analysis may be carried out in a given step, when it is assumed
that the vectors to be analysed are resident in the section nominated
on the VECTORS directive. The following points on eigenvector specification
should be noted:
\begin{itemize}
\item While the SCF modules now support default eigenvector section
usage (removing the need for explicit section specification on both
the VECTORS and ENTER directives in such jobs), the user is strongly advised
to specify the section containing the eigenvectors to be analysed under
RUNTYPE ANALYSE.
\item With the exception of Localised Orbital generation, the analysis
routines do not output a separate set of eigenvectors, so that no section
specification is required on the ENTER directive. In such cases the following
sequence is typical of that required:

{
\footnotesize
\begin{verbatim}
          VECTORS 1
          ENTER
\end{verbatim}
}
with the eigenvectors to be analysed resident in section 1. In the case
of localised orbital generation, the sequence

{
\footnotesize
\begin{verbatim}
          VECTORS 1
          ENTER 10
\end{verbatim}
}
will act to route the localised orbitals to section 10 of the Dumpfile.
\end{itemize}

Each mode of analysis has an associated set of sub-directives, which
are described below.

\section[One-electron Properties]{One-electron Properties}

This section deals with the data input used to drive the 1-electron
properties module. Note that the module can at present only 
run with basis sets comprising s,p and d basis functions
(see however the simplified property specification described below).

\subsection[PROPERTY]{PROPERTY}

This directive is used to specify which molecular 1-electron properties
are to be computed. The first data line consists of the character string
PROPERTY in the first data field. Subsequent data lines, the `property
definition' lines, are read to variables NPROP,TAGA,ISECT using format
(I,A,I).  NPROP is a code number of the property to be computed (see 
Table~\ref{table:5} of Part 2)
TAGA should be set to the TAG of one of the centres defined in
in the ZMATRIX or GEOMETRY directive, 
or in a CENTRES directive.
ISECT is the section number of the Dumpfile where the computed
property integrals are to be stored. If ISECT is omitted,
the integrals are not stored on the Dumpfile.
 The final data line consists of the character string END in the first
data field. Only 100 property lines may be presented in any one job.

\subsection[CENTRES]{CENTRES}

 This directive permits the specification of additional (non-nuclear)
centres at which the properties are to be evaluated. The first data
line consists of the character string CENTRES in the first data field.
Subsequent data lines are read to variables X,Y,Z,TAG using format
(3F,A), where X,Y and   Z are the Cartesian 
co-ordinates of an additional centre (in atomic units).
TAG is a name (up to 8 non-blank characters) by which the
centre will subsequently be known. TAG may be omitted,
when the system will supply an ordinal default.
 The final line consists of the character string END in the first data
field. \\

{\bf Example}
{
\footnotesize
\begin{verbatim}
          CENTRES
          -1.0 1.0 1.0 ADDC
          END
\end{verbatim}
}

\subsection[NUCLIDIC]{NUCLIDIC}

This directive is used to re-define the nuclidic mass by the program,
which by default corresponds to the most abundant isotope. The first
data line consists of the character string NUCLIDIC in the first data
field. Subsequent lines are read to variables TAGB,CENMAS using format
(A,F).  TAGB should be set to the TAG of a previously 
defined atomic centre, while CENMAS should be set to the 
value of the nuclidic mass to be used for this centre.
The final line consists of the character string END in the first data
field.\\

{\bf Example}
{
\footnotesize
\begin{verbatim}
          NUCLIDE
          OXYGEN 17
          END
\end{verbatim}
}

\section[Simplified Property Specification]{Simplified Property Specification}

In the section above we have assumed that property evaluation is to be
conducted under control of RUNTYPE ANALYSE, with explicit specification
of the required one-electron properties. A simplified mechanism for
property evaluation can be requested through presenting the data line

{
\footnotesize
\begin{verbatim}
         PROPERTY ATOMS
\end{verbatim}
}
after RUNTYPE and SCFTYPE specification. This will result in the
default wavefunction analysis conducted after RUNTYPE processing
being augmented with the computation of certain one-electron properties.
The following points should be noted:
\begin{itemize}
\item the properties evaluated include the electrostatic potential,
electric field, electric field gradient, and electron density at each of
the atomic centres, plus the dipole, second moment, quadrupole moment,
third and octupole moments, at the computed centre of mass of the
system under study. In addition the spin densities will also be computed
in the case of open shell systems.
\item this analysis, if requested, is available on completion
of SCF, OPTIMIZE, OPTXYZ, SADDLE, and CI processing.
\item in contrast to the detailed property evaluation performed under
RUNTYPE ANALYSE control, this default evaluation may be conducted 
with basis sets comprising s,p,d and f basis functions.
\end{itemize}
We illustrate below a number of considerations that arise as a function
of RUNTYPE and SCFTYPE processing when invoking this default property
specification.

\subsection[SCF Calculations]{SCF Calculations}

The following data sequence would be required to generate the
default list of properties on completion of an SCF calculation
of the formaldehyde molecule.

{
\footnotesize
\begin{verbatim}
          TITLE
          H2CO - 3-21G BASIS - SCF + DEFAULT 1-E PROPERTIES
          ZMATRIX ANGSTROM
          C
          O 1 1.203
          H 1 1.099 2 121.8
          H 1 1.099 2 121.8 3 180.0
          END
          RUNTYPE SCF
          PROPERTY ATOMS
          ENTER
\end{verbatim}
}
In this example the set of MOs to be used in the property evaluation
will be retrieved from the default section of the Dumpfile written to
by the closed-shell SCF module i.e. section 1.

\subsection[UHF Calculations]{UHF Calculations}

A somewhat different approach may be required when computing the
one-electron properties derived from a wavefunction with more than
one set of MOs (e.g., a UHF wavefunction), or in cases where only the
total density matrix, and not an associated set of MOs, is available
(e.g., in an MP2 calculation).  In both cases, the user may need to
ensure that the associated set of spinfree natural orbitals and, where
relevant SPIN natural orbitals, are generated by specification of the
NATORB directive(s), used to route the NOs to a nominated section on
the Dumpfile.

We illustrate this effect by first considering the data requirements when
performing a UHF wavefunction.  The following data sequence  would be
required when evaluating the properties based on a direct-UHF calculation,
with the alpha- and beta-UHF MOs routed to the default sections 2
and 3 of the Dumpfile in the absence of explicit section specification 
on the ENTER directive.

{
\footnotesize
\begin{verbatim}
          TITLE
          H2CO - 3A2 UHF PROPERTIES - 3-21G BASIS
          MULT 3
          ZMATRIX ANGSTROM
          C
          O 1 1.203
          H 1 1.099 2 121.8
          H 1 1.099 2 121.8 3 180.0
          END
          SCFTYPE DIRECT UHF
          PROPERTY ATOMS
          ENTER
\end{verbatim}
}
The same calculation may be performed based on the spinfree and spin
natural orbitals of the UHF wavefunction; in this case the NATORB data
lines will be used to route the spinfree and spin natural orbitals to
sections 10 and 11 of the Dumpfile respectively, and these orbitals will
be used in computing the 1-electron properties, thus:

{
\footnotesize
\begin{verbatim}
          TITLE
          H2CO - 3A2 UHF NO-BASED PROPERTIES - 3-21G BASIS
          MULT 3
          ZMATRIX ANGSTROM
          C
          O 1 1.203
          H 1 1.099 2 121.8
          H 1 1.099 2 121.8 3 180.0
          END
          SCFTYPE DIRECT UHF
          PROPERTY ATOMS
          NATORB 10 
          NATORB SPIN 11
          ENTER
\end{verbatim}
}
The following data sequence would be required if the user wished to
compute the properties of the annihilated UHF wavefunction:

{
\footnotesize
\begin{verbatim}
          TITLE
          H2CO - 3A2 annihilated UHF properties 3-21G BASIS 
          MULT 3
          ZMATRIX ANGSTROM
          C
          O 1 1.203
          H 1 1.099 2 121.8
          H 1 1.099 2 121.8 3 180.0
          END
          SCFTYPE DIRECT UHF
          PROPERTY ATOMS
          NATORB 10 ANNIHILATE
          NATORB SPIN 11 ANNIHILATE
          ENTER
\end{verbatim}
}
Note again that the NOs of the UHF and AUHF wave function are in fact identical,
the only difference lying in the occupation numbers.

\subsection[MP2 Calculations]{MP2 Calculations}

Now let us consider the date requirements when computing properties
at the optimum geometry derived from an MP2 calculation.  

{
\footnotesize
\begin{verbatim}
          TITLE
          H2CO - X1A1 - MP2 DZ BASIS - PROPERTIES
          ZMATRIX ANGSTROM
          C
          O 1 CO
          H 1 CH 2 HCO
          H 1 CH 2 HCO 3 180.0
          VARIABLES
          CO 1.203
          CH 1.099
          HCO 121.8
          END
          BASIS DZ
          RUNTYPE OPTIMISE
          PROPERTY ATOMS
          SCFTYPE MP2
          NATORB 20 
          ENTER
\end{verbatim}
}
Having generated the MP2 optimised geometry, the spinfree natural orbitals
will be routed to section 20 on the Dumpfile, and used in the subsequent
properties calculation.

\subsection[CI Calculations]{CI Calculations}

Computing the default set of one-electron properties at completion of
CI processing may be readily accomplished through the addition of the
PROPERTY ATOMS data line. Note that any such calculation requires
the NATORB directive to specify the routing of the spinfree, and where
relevant, the spin NOS to specified sections on the Dumpfile. Note also
that property evaluation under PROPERTY ATOMS control is only available
for Direct-CI calculations (and not Full-CI or coupled cluster
calculations).

{
\footnotesize
\begin{verbatim}
          TITLE
          H2CO - 3-21G  CISD DCI + PROPERTIES CALCULATION
          SUPER OFF NOSYM
          ZMATRIX ANGSTROM
          C
          O 1 1.203
          H 1 1.099 2 121.8
          H 1 1.099 2 121.8 3 180.0
          END
          RUNTYPE CI
          PROPERTY ATOMS
          DIRECT 16 8 14
          CONF
          2 2 2 2 2 2 2 2
          NATORB 10 0 PRINT
          ENTER
\end{verbatim}
}
To further illustrate property evaluation for CI wavefunctions, let
us consider a CI calculation on the \trstate\  state of \formaldehyde.
First, the data for the open-shell SCF calculation;

{
\footnotesize
\begin{verbatim}
          TITLE
          H2CO - DZ BASIS - 3A2 GRHF  TOTAL ENERGY = -113.73954029 AU
          MULT 3
          SUPER OFF NOSYM
          ZMATRIX ANGSTROM
          C
          O 1 1.203
          H 1 1.099 2 121.8
          H 1 1.099 2 121.8 3 180.0
          END
          BASIS DZ
          OPEN 1 1 1 1
          ENTER
\end{verbatim}
}
Having generated the SCF wavefunction, the following data sequence
would be used for a single reference CI calculation; routing the
spinfree and spin natural orbitals to sections 10 and 11 of the
Dumpfile will permit subsequent property generation, requested by
presenting the data line PROPERTY ATOMS.

{
\footnotesize
\begin{verbatim}
          RESTART NEW
          TITLE
          H2CO - DZ BASIS - 3A2 CISD DIRECT-CI -113.934177537 AU
          MULT 3
          SUPER OFF NOSYM
          BYPASS SCF
          ZMATRIX ANGSTROM
          C
          O 1 1.203
          H 1 1.099 2 121.8
          H 1 1.099 2 121.8 3 180.0
          END
          BASIS DZ
          RUNTYPE CI
          PROPERTY ATOMS
          OPEN 1 1 1 1
          DIRECT 16 9 15
          SPIN 3
          CONF
          2 2 2 2 2 2 2 1 1
          NATORB 10 11 PRINT
          ENTER
\end{verbatim}
}
Note that properties could also have been calculated after the CI job by
specifying the appropriate natural orbitals under RUNTYPE ANALYSE. The
data below would compute the isotropic ESR coupling constants (property
index 19) at carbon, oxygen and hydrogen, where the spin NOS are nominated
on the VECTORS line.

{
\footnotesize
\begin{verbatim}
          RESTART NEW
          TITLE
          H2CO - DZ - 3A2 UHF SPIN DENSITIES
          MULT 3
          ZMATRIX ANGSTROM
          C
          O 1 1.203
          H 1 1.099 2 121.8
          H 1 1.099 2 121.8 3 180.0
          END
          BASIS DZ
          RUNTYPE ANALYSIS
          PROPERTY
          19 C
          19 O
          19 H
          END
          VECTORS 11
          ENTER
\end{verbatim}
}
\section[Localised Molecular Orbitals]{Localised Molecular Orbitals}

The purpose of this module is the localisation of
molecular orbitals according to either:
\begin{itemize}
\item the criterion of Foster and Boys \cite{foster};
\item the overlap-based criterion due to  Pipek and Mezey \cite{pipek}.
\end{itemize}
The particular technique to be employed, together with the specification
of the orbitals involved, is requested by presenting the LOCAL directive:

\subsection[LOCAL]{LOCAL}

This directive, which may comprise one or more data lines, is used to 
define those molecular orbitals which take
part in the localisation process, and the localisation technique
to be employed. The first data line comprises one or more data fields;
\begin{itemize}
\item The first field consists of the character string LOCAL;
\item If specified, the second data field may be used to nominate the 
localisation technique to be employed, and comprises the character string 
BOYS (for the Foster-Boys technique) or OVERLAP (for the Pipek-Mezey 
overlap-based method). In the absence of this data field, the default
Foster-Boys method will be used.
\item If specified, the third data field may be used to define a default
set of orbitals to be incorporated in the localisation process. Presenting
the data field DEFAULT instructs the program to consider just the set of 
valence molecular orbitals, omitting all core orbitals from the process.
The presence of the DEFAULT data character string signals the termination
of LOCAL input.
\end{itemize}
An alternative to utilising the DEFAULT option above is to include
additional data lines that explicitly nominate the molecular orbitals 
to be incorporated. Such data lines are read to an array (LMO(I),I=1,NACT) 
using free I-format. When specifying such an orbital set, the last
data field presented should be the character string END.\\

{\bf Example 1}\\

The single data line
{
\footnotesize
\begin{verbatim}
          LOCAL OVERLAP
          1 2 3 4 5 6 7 8 9 END
\end{verbatim}
}
requests use of the overlap-based localisation technique due to
Pipek and Mezey, with molecular orbitals 1-9 to be included.

{\bf Example 2}\\

The single data line
{
\footnotesize
\begin{verbatim}
          LOCAL OVERLAP DEFAULT
\end{verbatim}
}
requests use of the overlap-based localisation technique due to
Pipek and Mezey, with only the set of valence  molecular orbitals to 
be included.\\

{\bf Example 3}
{
\footnotesize
\begin{verbatim}
          LOCAL
          2 3 4 5 6 7 9 END
\end{verbatim}
}
Declares molecular orbitals 2 to 7 inclusive, plus molecular orbital 9
to be active in the Foster-Boys localisation process.\\

{\bf Example 4}
{
\footnotesize
\begin{verbatim}
          LOCAL
          2 3 4 5 6 7
          9
          END
\end{verbatim}
}
This sequence has an equivalent effect to that of Example 3.\\

{\bf Example 5}
{
\footnotesize
\begin{verbatim}
          LOCAL
          2 TO 7 9 END
\end{verbatim}
}

The above sequence shows an abbreviated form of 
specifying the list of  molecular orbitals,
invoking the character string TO, to link together a sequence of
consecutive numbered active molecular orbitals.
This sequence is equivalent to examples 3 and 4.\\

{\bf Example 6}
{
\footnotesize
\begin{verbatim}
          LOCAL
          2 TO 7 9 TO 14
          END
\end{verbatim}
}

{\bf Note:-}\\

During the localisation process, under control of RUNTYPE~ANALYSE,
symmetry adaptation is automatically switched off to enable orbitals
of different irreducible representation to mix (although the total
wavefunction remains, of course, a unitary transformation of the SCF
wavefunction).  Assuming these orbitals are to be used in a subsequent
SCF or GVB calculation, via the VECTORS directive, then the ADAPT~OFF
data line {\em must} be presented in any such job that utilises the
LMOs. Failure to provide such a line will probably lead to an error
condition when restoring the vectors.

\section[Graphical Analysis]{Graphical Analysis}
 
\subsection{Introduction}

GAMESS--UK supports the graphical analysis of wavefunctions by the calculation of
charge densities, molecular orbitals, atom difference densities and electrostatic
potentials on a grid of points. Versions of GAMESS--UK which are linked to the GHOST
graphical library allow graphical display of those datasets which are 
calculated on regular two-dimensional grids as contour or surface plots. In general,
the data values will be written to the formatted punchfile (See Part 11),
and analysed using other visualisation software.

The module is invoked under control of RUNTYPE ANALYSE by presence of the
GRAPHICS directive. Unlike the graphical analysis code present in previous
versions of GAMESS--UK the GRAPHICS keyword may only appear once for each 
specification of RUNTYPE ANALYSE. 

Note that the original restrictions in running this module - to basis
sets comprising only s,p and d basis functions - have been lifted in
the calculation of both molecular densities and potentials, where the
full range of s,p,d,f and g-functions may now be used.

The subsequent directives are presented in groups with the following functions

\begin{enumerate}
\item Those that define the grid of points (a group of directives initiated by GDEF)
\item Those that request data calculation on a grid (a group of directives initiated by CALC)
\item The SURF directive for generation of molecular surfaces (a combination of
both grid definition and calculation)
\item The RESTORE directive to retrieve grid definitions or data from the dumpfile.
\item Those that generate graphical output (when available, initiated by PLOT).
\end{enumerate}

Presentation of each of the directives which appear as groups
are initiated by the specific directives GDEF, CALC and PLOT as given above. 
Directives within a group are terminated by the first directive of another group, 
by SURF or RESTORE, or by any other valid GAMESS--UK {\em Class 2}  directive.

The ordering of groups is significant (but the ordering of directives within a group
is not), as follows:
\begin{enumerate}
\item  A CALC directive must be preceded by a directives to generate (or restore) 
a grid definition, which will be used for the calculation. An exception is a calculation
of a combination grid (TYPE COMB) which must follow the definition of a calculation.
\item A PLOT directive must be preceded directives to generate or restore
a (suitable) array of data values to be plotted. 
\item A SURF directive must follow the generation or restoration 
of an array of scalar data points on a regular 3D grid.
\end{enumerate}

All directions and positions specified must be given in the molecular coordinate 
system, (after reorientation by the GAMESS--UK symmetry routines) in units of a.u.

At present, the number of grids, the number of calculations, the number 
of plot requests and the number of dumpfile restore operations
are each limited to 10 (this total including requests made implicitly by SURF 
directives, see below). Users who need to generate more data may use multiple 
sets of directives starting with RUNTYPE ANALYSE.

\subsection{Grid Definition Directives}

\subsection{Grid Definition - GDEF}
{
\footnotesize
\begin{verbatim}
          GDEF 
\end{verbatim} 
}
Grid definition mode is initiated by a GDEF directive. 
This may (optionally)
be followed by by a character string (8 characters or less) which will
be used in the output to reference the grid.
Grid definition mode is terminated by a GDEF, CALC, PLOT, REST, SURF 
or valid {\em Class 2}  directive.

\subsection{Grid Definition - TITLE}
Provide a title for the grid.
{
\footnotesize
\begin{verbatim}
          TITLE
          .. title string ...
\end{verbatim} 
}

\subsection{Grid Definition - TYPE}

The type of grid is specified by the TYPE directive. TYPE is followed by
a keyword specifying the type of grid to be generated, and other data as required
by the grid type. Valid keywords are given in Table~\ref{table:1} .


\begin{table}
 \caption{\label{table:1}\  Keywords of the Grid TYPE Directive}
 
 \begin{centering}
 \begin{tabular}{ll}
\\ \hline\hline
  keyword         &      Grid Type\\ \cline{1-2}
\\
 2D          &  Planar Rectangular 2D grid \\
 3D          &  Orthogonal 3D grid \\
 SPHERE      &  Spherical grid \\
 CARDS       &  User specified points \\
 CONTOUR     &  Generate points on an isovalue surface \\
 WRAP        &  Generate points on an iso-electron-density surface\\
 ATOM        &  Place grid points at the nuclear positions \\
\hline\hline
 \end{tabular}
 
 \end{centering}
\end{table}

  The individual grid generation modes are described below.

\subsubsection{TYPE 2D or TYPE 3D}

    Generate regular 2 or 3D grids. The origin (centre of the grid) is set using the
    ORIG directive, and the X and Y directives set the orientation.
    The number of points is set with the POINTS directive, and the edge lengths
    using SIZE (see below).

\subsubsection{TYPE SPHERE}
    
    Generate a grid of points on the surface of a sphere of given radius.
    The centre (default 0.0 0.0 0.0) can be set using the ORIG directive.
    The number of points is set with the POINTS directive.
    The radius is set using the SIZE directive. 

The SPHERE directive
may optionally be followed by one of the keywords RAND or SYMM.
RAND (which is the default) requests that random
starting latitude values be used for each ring of points at a given longitude.
SYMM requests that the generated grid preserve
axial rotation symmetry, and must be followed  by an integer specifying the
order of rotation symmetry required.

\subsubsection{TYPE CARDS ncards}
    Read a set of points from the input file. The number points (ncards) must be 
    specified. The directive is followed by ncards data records containing the x,y,
and z coordinates of a data point, in a.u., in the coordinate system of the molecule
after reorientation by the GAMESS--UK symmetry analysis routines.

\subsubsection{TYPE CONTOUR value }

    Generate points on an isovalue surface. The section number or calc-id refers 
    to an array of data values calculated on a regular 3D grid, which are to 
    be contoured.

\subsubsection{TYPE WRAP value }

    Points are generated on an isodensity surface (without storage of a 3D array
    of density points.  If this option is used, it must follow the specification
of a regular 3D grid, which defines the volume and mesh density for the contouring.

\subsection{Grid Definition - ORIG}

The ORIG directive specifies the origin (ie centre) of the grid, it 
applies only to the grid types 2D, 3D and SPHERE.

{
\footnotesize
\begin{verbatim}
          ORIG x y z
\end{verbatim} 
}

\subsection{Grid Definition - X and Y}
Set the direction of the plot axes for 2D and 3D grid types. By default, the 
first plot axis (referred to as X) lies along the +x direction of the molecular
coordinate system, and the second along +y. 

{
\footnotesize
\begin{verbatim}
          X xx xy xz
          Y yx yy yz
\end{verbatim} 
}

\subsection{Grid Definition - SIZE}

Set the grid size (edge length) for 2D and 3D grids, or the radius for spherical
grids.  If only one value is specified
it will be used for all dimensions, but extra values may be provided to set the y and
(for 3D) z axis lengths independently.

{
\footnotesize
\begin{verbatim}
          SIZE sx <sy <sz>>
\end{verbatim}
}

\subsection{Grid Definition - POINTS}

Set the mesh density. For 2 and 3D regular grids this sets the number of points along
the axes. For spherical grids the number of longitudinal
and latitudinal divisions should be given as a single integer argument.

{
\footnotesize
\begin{verbatim}
          POINTS nx <ny <nz>>
\end{verbatim}
}

\subsection{Grid Definition - SECTION}

Specify a section number (isect) on the dumpfile where the grid is to be stored. 
{
\footnotesize
\begin{verbatim}
          SECTION isect
\end{verbatim}
}
Storage of the grid is required in the following circumstances:
\begin{itemize}
\item the grid is to be restored later in the job, or in a later GAMESS--UK run
\item the grid is irregular (ie a type other than 2D or 3D) and is required to
appear on the punchfile along with data calculated on the grid. 
\end{itemize}

\subsection{Directives Requesting Data Calculation}

\subsection{Data Calculation - CALC}

The computation of data on a predefined grid is requested by the CALC 
directive. All subsequent directives up to the next GDEF, CALC, PLOT, SURF, REST or other 
{\em Class 2}  directive serve to specify details of the calculation. 

{
\footnotesize
\begin{verbatim}
          CALC 
\end{verbatim}
}

\subsection{Data Calculation - TITLE }

Specify a title for the data array
{
\footnotesize
\begin{verbatim}
          TITLE 
          ... title ...
\end{verbatim}
}

\subsection{Data Calculation - TYPE}
{
\footnotesize
\begin{verbatim}
          TYPE  keyword
\end{verbatim}
}
Specify the property to be calculated. Valid keywords are
given in Table~\ref{table:2}.

\begin{table}
 \caption{\label{table:2}\  Keywords of the Data TYPE Directive}
 
 \begin{centering}
 \begin{tabular}{ll}
\\ \hline\hline
  keyword         &      Data Type\\ \cline{1-2}
\\
  MO         &  Molecular Orbital\\
  DENS       &  electron density \\
  ATOM       &  atom difference electron density \\
  POTE       &  electrostatic potential \\
  GRAD DENS  &  gradient of the charge density \\
  GRAD MO    &  gradient of MO \\
  COMB       &  linear combination with the previous calculation \\
  VDW        &  Van der Waals function \\
  LVDW       &  log Van der Waals function \\
  GRAD VDW   &  gradient of Van der Waals function \\
  GRAD LVDW  &  gradient of Van der Waals function (sic)\\
\hline\hline
 \end{tabular}
 
 \end{centering}
\end{table}

A more detailed description of the functions available is given below, together
with information regarding any other input data required.\\

{\bf Electron Density Functions}\\
 
In depicting the spatial characteristics of the  density associated with one or
more  molecular orbitals,  the  program  computes  densities according  to  the
formula:
\begin{equation} 
          \rho (r) = \sum_{i} OCC_i \Theta_i (r)^2
\end{equation}
where $\Theta_i$ denotes  the i'th  molecular orbital and  OCC$_{i}$ its  occupation
number.  Plots of the amplitude of a single orbital, i:
\begin{equation} 
           a_i (r) = OCC_i \Theta_i (r)
\end{equation} 
may also be generated.\\

{\bf Density Difference Function}\\
 
The atomic density difference function is defined as:
\begin{equation} 
       \Delta \rho (r) = \rho_{mol} (r) -  \sum_{j}^{Atoms} \rho_j (r)
\end{equation} 
The first  term is the total electron density  associated with a molecule.  The
second  term represents the  sum of the  electron densities of  the atoms which
constitute the molecule, these being  placed at the same positions they  occupy
in the molecule, but which are  assumed to have undergone no interactions  with
each other, and  have remained undistorted, as in  the free state.   The atomic
density difference function provides an indication of the overall rearrangement
of density which occurs when the  atoms come together upon molecular formation.
The program incorporates  an atomic  SCF module,  so that  calculations on  the
ground states of the component atoms are  performed in line, with the basis set
of  each atom the same  as that used  in the parent molecule.   These plots are
known as `atomic-difference` plots;  the program is capable of  generating such
plots of molecular systems with component atoms up to, and including, zinc.
 
A more general form of  the density difference function, the molecular  density
difference:
\begin{equation} 
       \Delta \rho (r) = \sum_{i} \rho_i (r) -  \sum_{j} \rho_j (r)
\end{equation} 
 is used in  the construction  of molecular-difference plots,  which allows  the
user to display the density resulting from the addition or subtraction of up to
15 component density functions.  Two examples of the value of such plots are:
\begin{itemize}
\item Illustrating the effect on a molecular charge distribution, resulting from
an extension of the basis set, so that a typical plot would be constructed
using the function:
\begin{equation} 
       \Delta \rho (r) =  \rho_{mol}^{extended} (r) -  \rho_{mol}^{minimal} (r)
\end{equation} 
\item In depicting the rearrangement  of electron density which occurs  when the
component  ligands  and  metal atom  of  a  transition  metal complex  come
together to  form the  molecular system,  so that  for a  complex MX,  the
following difference function:
\begin{equation}                                                     
   \Delta \rho (r) = \rho_{complex} (r) - \rho_M (r) -  \sum_{i}^{n} \rho_{Xi} (r)
\end{equation} 
would be  constructed, this  being the  molecular analogue  of the  atomic
density difference function.
\end{itemize} 
{\bf Electrostatic Potential Function}\\
 
The value of the electrostatic potential created by the electronic distribution
and nuclear charge of a molecule, in the different regions of space surrounding
it, provides information  about possible  sites involved in  protonation or  in
reactions  with  electrophilic  agents.    The  interaction  energy  between  a
molecular distribution and an external unitary positive charge at a given point
i, is given by:
\begin{equation} 
   V(r_i ) = \sum_{\alpha}^{nucl}  \frac{Z_\alpha}{r_i\alpha} -  \int d\tau_1 \frac{\rho (1)}{r_{1i}}
\end{equation} 
where Z$_{\alpha}$ is the nuclear  charge of nucleus  alpha and $\rho (1)$ is  the first
order density  function.   The  program permits  the construction  of plots  of
electrostatic interaction energies  based on  the density distribution  arising
from wavefunctions constructed in Gaussian orbital basis sets.

\subsubsection{MO and GRAD MO }

The MO and GRAD MO keywords must be followed by an integer specifying the MO
of interest.

\subsubsection{COMB}

The COMB keyword is followed by specification of the data to be combined with the current
grid and a scale factor as follows;
{

\footnotesize
\begin{verbatim}
          TYPE COMB [lfn iblk]  isec scale
\end{verbatim}
}
If lfn and iblk are given, they determine the foreign dumpfile and start block
on which the data is to be found. If they are omitted the current dumpfile is used.
isec is the section number of the required data (as specified using the SECTION directive
when the data was generated)

The grid is multiplied by  scale before being added to the current grid.
If more than two grids are to be combined. Only one TYPE COMB directive may be present
in a given CALC group of directives.

\subsubsection{Van der Waals functions}

These functions are defined in such a way that contouring them leads to
VdW surfaces. The gradient functions generate unit normals
to the surface (ie not strict gradients, but normals to the surface which
are adequate for lighting calculations). 
The internal van der Waals radii are taken from Nuffield Advanced Science ``Book of
Data'', R. D Harrison (ed), 1988. Additional (or replacement) radii may be
provided using the RADII directive (see below).

The two functions available are VDW, defined by 

\noindent
max ( r$^{vdw}$$_i$  -r$_i$)

and LVDW, defined by

\noindent
  max(-ln(r$_i$/ r$^{vdw}$$_i$ ))

where r$_i$ is the distance from the test point to nucleus i, r$^{vdw}$$_i$ the van der Waals radius
of atom i, and the max function runs over all nuclei in the molecule.

Both functions have the property that the isovalue surface for value 0.0 is the
van der Waals surface. When contoured at a positive value ({\em v}), the VDW 
function gives rise to a surface inside the van der Waals surface, corresponding
to a surface with all radii reduced by {\em v}. Similarly negative contour heights
may be used to generate surfaces with a constant added to all radii. Although not
identical, these surfaces are similar to Connolly surfaces with a probe sphere
of radius {\em v}.  Contouring the LVDW function at non-zero values gives
rise to surfaces that correspond to scaled van der Waals radii. 
Positive values {\em v} thus lead to surfaces inside the van de Waals radius.

NB. It is important to note that at the moment the use of the LVDW function
to generate the surfaces with scaled radii is not fully tested - ie there 
are sets of radii, and/or choices of contour levels which will result
in a surface that does not correspond to that generated from the scaled radii.
%In the few cases tested, it has worked, but a mathematical definition
%of the range of radii/surfaces for which it is safe has so far escaped me.

\subsection{Data Calculation - OCCDEF}
The purpose  of this directive  is to allow the  user to define  the occupation
numbers for  the molecular  orbitals to be  analysed.   In the absence  of the
OCCDEF directive, the occupation numbers will be  taken from the section of the
Dumpfile specified on the VECTORS directive.
The first data  line contains the character  string, OCCDEF, in the  first data
field.  Following the directive initiator  are the occupation definition lines.
The first data field  of such lines is read  in F--format, and should contain  a
specified occupation number.  Subsequent data fields are read in I--format.  Let
the  value of an  integer specified in  such a field  be j,  the j'th molecular
orbital  will be assigned  the occupation  number specified  in the  first data
field of the line.  The following:

{
\footnotesize
\begin{verbatim} 
           2.0 1 2 3 4 5 7
\end{verbatim} 
}
comprises a valid occupation definition line.  Such lines may be  shortened, if
a sequence of  consecutive integers appear, by  means of the character  string,
TO.  Thus, the abbreviated form of the above line is:

{
\footnotesize
\begin{verbatim} 
           2.0 1 TO 5 7
\end{verbatim} 
}
The occupation  definition lines  are specified  until all  the orbitals  to be
assigned a finite  occupancy have been  declared.  A  data line containing  the
text,  END, in the first data field, must  be specified to terminate the OCCDEF
directive.

The following points must be noted:
\begin{itemize} 
\item Any orbital omitted from  the list specified on the  occupation definition
lines will be assigned zero occupancy, and thus will make no  contribution
to the grid of function values to be constructed.
\item It  is envisaged  that  the OCCDEF  directive will  not  be required  when
generating grids of total electron density, atomic density difference, and
interaction  potentials.   In these  three cases,  the  occupation numbers
should reflect the overall  orbital occupancy in the molecule,  and should
be just the  values calculated  during the construction  of the  molecular
orbitals and output to the Dumpfile.

\item The OCCDEF directive should be used when analysis of the electron density,
 associated with a certain subset of orbitals, is required.\\

{\bf Example}
{
\footnotesize
\begin{verbatim} 
           OCCDEF
           2.0 1 TO 5 7
           END
\end{verbatim} 
}
The grid  of values will  be generated assuming  the first  five molecular
orbitals,  together  with  orbital  7, are  doubly--occupied.    All  other
orbitals will be assigned zero occupancy.
\end{itemize} 
 
\subsection{Data Calculation - CONFIG}
The  CONFIG directive  is  only applicable  when  generating a  grid  of atomic
density difference, and may be used to  specify the configuration to be used in
computing the atomic density distribution, corresponding to the ground state of
the atoms.   In the absence of this  directive, spherically symmetric atoms are
chosen, with equal occupation of the degenerate open--shell orbitals.\\
The user should note that applying a CONFIG specification will not change the configuration
of the atom used in the atomic SCF calculation, but will result in modification of
the occupations of the atomic orbitals just prior to the computation of the density.
The computed density therefore does not correspond to a self-consistent
atomic calculation.\\

{\bf Example}\\

The atomic configuration chosen, by default, for the carbon atom would be:\\

(1s)$^{2}$ (2s)$^{2}$ (2p$_{x}$)$^{0.666667}$ (2p$_{y}$)$^{0.666667}$ (2p$_{z}$)$^{0.666667}$\\

and for the iron atom (d$^{6}$s$^{2}$ high spin):\\

(1s)$^{2}$ (2s)$^{2}$ (2p$_{x}$)$^{2}$ (2p$_{y}$)$^{2}$ (2p$_{z}$)$^{2}$ (3s)$^{2}$ (3p$_{x}$)$^{2}$ (3p$_{y}$)$^{2}$ (3p$_{z}$)$^{2}$ (4s)$^{2}$ 

(3d$_{xy}$)$^{1.2}$ (3d$_{xz}$)$^{1.2}$ (3d$_{yz}$)$^{1.2}$ (3d$_{x2-y2}$)$^{1.2}$ (3d$_{z2}$)$^{1.2}$ \\

The CONFIG directive consists of three types of data line.  The first line, the
directive initiator, consists  of the  character string, CONFIG,  in the  first
data field; the last line, the directive terminator, consists of  the character
string, END, in  the first data field.   Lines specified between  the directive
initiator and terminator, are  the `configuration definition' lines.   If there
are NAT atoms,  whose configuration  are to be  specified, NAT  `configuration'
lines are required.  Each line consists of  (NORB+1) data fields, where NORB is
the total number of doubly--occupied or partially--occupied orbitals in the atom.
The first data field is read to  the variable ALAB, using format A, whilst  the
remainder of the data line should contain real numbers, read  in F--format, to a
vector (OCC(I),  I=1,NORB).  ALAB should  be set to the label  parameter of the
nucleus, as specified  by the corresponding  `nucleus definition' lines in  the
GEOMETRY or ZMATRIX directives.
OCC(I) should be set to the occupation number of the i'th atomic orbital.
The latter must  be input  in order of  symmetry -- s,  p, d,  etc., -- with  the
partially--occupied orbitals  preceded by the  doubly--occupied orbitals,  within
each symmetry class.\\

{\bf  Example 1}\\

Suppose the  user wishes the   
(1s)$^{2}$ (2s)$^{2}$ (2p$_{x}$)$^{1}$ (2p$_{y}$)$^{1}$ configuration
for a  carbon atom, which
has been labelled as nucleus C1.
Then the following configuration definition line should be specified:

{
\footnotesize
\begin{verbatim} 
           C1 2.0 2.0 1.0 1.0 0.0
\end{verbatim} 
}
{\bf Example 2}\\

To specify the configuration
(1s)$^{2}$ (2s)$^{2}$ (2p)$^{6}$ (3s)$^{2}$ (3p$_{z}$)$^{1}$
for  an aluminium atom,  labelled AL by  the ZMATRIX directive,  the user must
specify the following configuration definition line:

{
\footnotesize
\begin{verbatim}
           AL 2.0 2.0 2.0 2.0 2.0 2.0 0.0 0.0 1.0
\end{verbatim} 
}
where  the first  three 2.0  give the  occupation of  the s  orbitals, and  the
remainder details the occupation of the p orbitals.\\

{\bf  Example 3}\\

To specify the configuration
(1s)$^{2}$ (2s)$^{2}$ (2p)$^{6}$ (3s)$^{2}$ (3p)$^{6}$ (4s)$^{2}$ (3d$_{xy}$)$^{2}$ (3d$_{xz}$)$^{2}$ (3d$_{yz}$)$^{2}$
for  an iron  atom, which  has been  labelled FE,  the user  should specify  the
following line:

{
\footnotesize
\begin{verbatim}
          FE 2.0 2.0 2.0 2.0 2.0 2.0 2.0 2.0 2.0 2.0 2.0 2.0 2.0 0.0 0.0
             < s orbitals  > <     p orbitals      > <    d orbitals   >
\end{verbatim} 
}
Note that the occupation numbers of the three p orbitals should be input in the
order (x,y,z), and those of the five d orbitals in the order:
d$_{xy}$, d$_{xz}$, d$_{yz}$, d$_{x2-y2}$, d$_{z2}$


\subsection{Data Calculation - SFAC}

The SFAC directive specifies a constant that will be used to multiply 
all calculated grid data values. 
The default scale factor is 1.0, except for electrostatic potential plots,
in which case it is 627.707, (to convert the values to kcal/mol).

\subsection{Data Calculation - RADII}

The RADII directive allows the user to input a set of radius values
for use in the van der Waals function calculation (see above).
it is followed by a series of records, each specifying an atom label
and the radius for all atoms with that label.
The input is terminated with a record containing the string END. By default
radii are expected in atomic units, angstroms may be used if the string
ANGS is added on  the RADII directive.

\subsection{Data Calculation - SECTION}

Specify a section number (isect) on the dumpfile where the calculated data is to be stored. 
{
\footnotesize
\begin{verbatim}
          SECTION isect
\end{verbatim}
}
Storage of the data is required is required in the following circumstances:
\begin{itemize}
\item the grid is to be written to the punchfile.
\item the data is on a 3D regular grid and is to be used in a contour grid 
generation step.
\end{itemize}

\subsection{Data Calculation - RESTORE}

The RESTORE Directive is used to bring a grid or data from a dumpfile into
memory. The data must have been written by a SECTION directive, (see above)
and the first keyword on the directive, DATA or GRID, is used to specify the
whether a grid definition or data is required. It is an error to specify DATA
if the dumpfile section actually contains a grid definition, and {\em vice-versa}.
The remaining data items are the lfn (eg ED3 for the current dumpfile), iblk is 
the starting block of the dumpfile (usually 1), and isec the dumpfile section as specified
on the SECTION directive.

{
\footnotesize
\begin{verbatim}
          RESTORE [ GRID | DATA] lfn iblk isec 
\end{verbatim}
}

\subsection{Data Calculation - SURF}

The SURF directive is used to generate one or more isovalue surfaces
from a 3D dataset, and calculate data at points on that surface. It is presented
after directives requesting the calculation of the 3D dataset.

The syntax of the surf directive is as follows:
{
\footnotesize
\begin{verbatim}
          SURF type isec [imo]  level1 [level2 ...]
\end{verbatim}
}

where
\begin{itemize}
\item type is the property to be calculated. It may be one of the valid keywords of the
calculation TYPE subdirective given in Table~\ref{table:2}, and may be followed by an MO index
where appropriate.
\item isec is the a section number on the current dumpfile for output of the first 
grids generated. Subsequent data and grids will be saved to consecutive
section numbers.
\item level1 level2 .. are a series of values at which the isovalue surface is to be generated.
\end{itemize}

The result is the same as if the user had made a series of 
grid definition requests (of type CONTOUR) and data 
calculation requests. 
For each level requested, GAMESS--UK will generate a surface grid and write
it to the dumpfile. If the 3D data being contoured is of type MO or DENS the gradient
of the field at every point will be calculated and stored. The requested property will 
then be calculated for every point.
When more than one level is required it is necessary to restore the 3D 
dataset prior to each contouring operation. This is performed automatically
by the program, but a consequence of this, that the 3D data must be stored on
the dumpfile, is the responsibility of the user.
The user should note that there is a limit, noted above, to the number of
of grid definitions, data calculations etc which may be requested for each
invocation of RUNTYPE ANALYSE, and the totals include requests generated implicitly
by SURFACE.

All grid definitions and data arrays are written to the dumpfile, with section numbers
counting from the isec value given, in the order in which they are generated. 
The user is responsible for ensuring that any data required by other software
is written to the punchfile. The grid definitions will be appear in the punchfile as part
of the property (but not gradient) datasets and are not requested explicitly.
The form of the punch directive required should be
clear form the examples below.

\subsubsection{Example Potential on an isodensity grid}

The following GAMESS--UK input assumes that a job to calculate the
closed-shell SCF wavefunction has already been completed

{
\footnotesize
\begin{verbatim}
          RESTART NEW
          PUNCH GRID 161 162 164 165
          TITLE
          FORMALDEHYDE SURFACE
          ZMAT ANGS
          C
          O 1 1.203
          H 1 1.099 2 121.8 
          H 1 1.099 2 121.8 3 180.0
          END
          BASIS DZ
          RUNTYPE ANALYSE
          GRAPHICS
          GDEF
          TITLE
          3D  GRID
          TYPE 3D
          POINTS 60
          SIZE 8.0
          CALC
          TYPE DENS
          SECTION 150
          SURF POTE 160 0.02 0.04
          VECTORS 1
          ENTER
\end{verbatim}
}
The result would be following job steps
\begin{enumerate}
\item A 3D grid, with edge length 8 a.u., centred on the origin is defined.
\item The density is calculated at all points on the grid and written to dumpfile 
section 150
\item A set of points on the surface of electron density 0.02 are generated and 
written to dumpfile section 160
\item The gradient of the density is calculated at each point,
 and written to dumpfile section 161
\item The potential is calculated at every point, and written to section 162
\item The 3D density grid is restored from the dumpfile
\item Steps 3,4, and 5 are repeated for a density value of 0.04, resulting in
the new grid definition being written to section 163, that density gradient to section
164 and the potential to section 165.
\item The data from sections 161 162 164 and 165 are written to the punchfile. 
The grid definitions (from sections 160 and 163) also appear on the dumpfile, although
not explicitly requested.
\end{enumerate}

\subsection{Plot Requests}

Note that  Users of versions of GAMESS--UK which support GHOST graphics 
can generate contour
plots (TYPE CONT) and relief plots (TYPE SURF). Otherwise only lineprinter 
plots are available. 

\subsection{Plot Requests - PLOT}

Plots are requested by a series of directives initiated by the PLOT directive.

\subsection{Plot Requests - TITLE}

Provide a title for the plot
{
\footnotesize
\begin{verbatim}
          TITLE
          ... title... 
\end{verbatim}
}

\subsection{Plot Requests - TYPE}

Specify the type of graphical output required, Valid directives are
given in Table~\ref{table:4}.

\begin{table}
 \caption{\label{table:4}\  Keywords of the Plot TYPE Directive}
 
 \begin{centering}
 \begin{tabular}{ll}
\\ \hline\hline
  keyword         &      Data Type\\ \cline{1-2}
\\
  LINE       &  lineprinter plot \\
  CONT       &  contour plot \\
  SURF      &   surface plot \\
\hline\hline
 \end{tabular}
 
 \end{centering}
\end{table}

\subsection{Plot Requests - CONT}
Set the contour heights for lineprinter and contour options. The directive
is followed by one or more records containing the values required, terminated by end.

{
\footnotesize
\begin{verbatim}
          CONT
          cont(1) cont(2) ... 
          ...
          END
\end{verbatim}
}
If the CONT directive is omitted, the default  set of contour values, given
in Table~\ref{table:5}  will be used.  

\begin{table}
 \caption{\label{table:5}\  Default Contour Values in Graphical Analysis}
 
 \begin{centering}
 \begin{tabular}{rrll}
\\ \hline\hline
 GTYPE setting  POTE &  DENS &   AMPLITUDE &  ATOM or DIFF\\ \cline{1-4}
      210.0   & 64.7837 & 1.0     &      0.8691\\
      180.0   & 16.1959 & 0.5     &      0.43455\\
      150.0   &  4.0490 & 0.25    &      0.21727\\
      120.0   &  1.0122 & 0.125   &      0.10864\\
       90.0   &  0.5061 & 0.0625  &      0.05432\\
       75.0   &  0.2531 & 0.03125 &      0.02716\\
       60.0   &  0.1265 & 0.01562 &      0.01358\\
       40.0   &  0.0633 & 0.00781 &      0.00697\\
       20.0   &  0.0316 & 0.00391 &      0.00339\\
       10.0   &  0.0158 & 0.00195 &      0.00170\\
        5.0   &  0.0079 & 0.00098 &      0.00085\\
        2.0   &  0.0040 & 0.0049  &      0.00042\\
        0.0   &  0.0020 & 0.0     &      0.0\\
              &  0.0010 \\ \hline\hline
 \end{tabular}

 \end{centering}
\end{table}
Contour  values  of   POTE,  AMPLITUDE,  ATOM,   and  DIFF  
setting  also   have corresponding negative values. 

\subsection{Plot Requests - VIEW}
 
The VIEW directive is only relevant when generating perspective plots, 
and may be used to specify the
angle  of view and  viewing distance.  The  grid of electron
densities or potentials define the z--values of the surface 
on a two--dimensional (x,y) grid of points covering the  
specified area of the molecular plane  under investigation.
 
The directive consists of a single  data line, read to variables TEXT,  THETAV,
THETAH, and DIST, using format (A,3F).
\begin{itemize} 
\item      TEXT      should be set to the character string, VIEW.
\item      THETAV    specifies the elevation  of the view  axis above the  horizontal
base--plane in degrees 
\item      THETAH    specifies the  rotation of the vertical axis  through the centre
of the grid  in a  clockwise direction, in  degrees .
For 0.0 $<$ THETAH $<$ 180.0, the surface appears  to rotate
in a clockwise direction as THETAH increases.
\item DIST      specifies the viewing distance  in Bohr.
\end{itemize} 
If the VIEW directive is omitted,  THETAV (thetav) and THETAH (thetah) will  be
given the value 30, and DIST will be set to the value of SIZE, specified by the
PLANE directive.\\

{\bf Example}\\

To view the surface edge-on along  the y-axis, from a distance of 10  Bohr, the
following data line should be specified:

{
\footnotesize
\begin{verbatim} 
           VIEW 0.0 0.0 10.0
\end{verbatim} 
}
\subsection{Plot Requests - SCALE}
This directive is only  relevant when generating perspective plots,  and may be
used to normalize the  stereo--graphic projection to certain values  of electron
density or potential function.  The  directive consists of a single data  line,
read to variables TEXT, SCAMAX, SCAMIN, and FACTOR, using format (A,4F).
\begin{itemize} 
\item TEXT           should be set to the character string, SCALE.
\item SCAMAX,FACTOR  the grid of  values to be plotted is scanned  to detect all
local maxima,  which will  appear as peaks  on the  projected
plot,  and  thus, to  determine  VMAX,  the  value  of  the
greatest  maximum  $<$ SCAMAX.   In  the  event that  no such
maximum  is  detected, VMAX  will be  set  to SCAMAX.   All
maxima with a greater value  than (VMAX*FACTOR) will appear
as beheaded peaks on the final plot.
\item SCAMIN,FACTOR  the grid of values  is scanned to detect all  local minima,
which will  appear as troughs  on the  projected plot,  and
thus, to determine VMIN, the value  of the lowest minimum $>$
SCAMIN.  In  the event  that no such  minimum is  detected,
VMIN will be set to SCAMIN.   All minima with a value  less
than (VMIN*FACTOR) will  appear as beheaded troughs  on the
final plot.
\end{itemize} 
The projected plot  will be  normalized to (VMAX*FACTOR--VMIN*FACTOR).   If  the
SCALE directive is omitted, SCAMAX is set to 0.7, SCAMIN to -0.7, and FACTOR to
1.2.
 
\subsection{Termination of GRAPHICS Input}
 
Data input for Graphical Analysis is terminated
by presenting a valid {\em Class 2} directive. This might
typically be the VECTORS directive, instructing the
analysis module as to the source of eigenvectors to be analysed. 
 
\section[Potential Derived Charges]{Potential Derived Charges}

The potential derived charges module uses electrostatic
potential data calculated using the graphics module to generate
least-squares fitted point charges at the nuclei.

The module is invoked by the POTFIT directive, under control
of RUNTYPE ANALYSE.

An example of the input data is given below
{
\footnotesize
\begin{verbatim}
          TITLE
          PDC CALCULATION
          ZMAT ANGS
          O
          H 1 1.0
          H 1 1.1 2 109.0
          END
          RUNTYPE SCF
          ENTER
          RUNTYPE ANALY
          GRAPHICS
          GDEF
          TYPE 3D
          POINTS 50
          SIZE 6
          SECTION 150
          CALC 
          TYPE DENS
          SECTION 151
          TITLE
          DENSITY ON 3D GRID
          SURFACE POTE 170 0.02 0.04
          VECTORS 1
          ENTER
          RUNTYPE ANALY
          POTF 172 175 CHAR 0.0
          VECTORS 1
          ENTER
\end{verbatim}
}
This job consists of three phases (each terminated by an
ENTER directive)

\begin{itemize}
\item perform an SCF calculation, writing vectors to Dumpfile section 1
\item generate the electron density on a 3D grid, and calculate the
electrostatic potential at the 0.02 and 0.04 isodensity surfaces. This data is written 
to the dumpfile on section 172 and 175
\item generate potential derived charges by fitting to the potential points
from 2. The total charge is constrained to 0.0
\end{itemize}

In addition to the dumpfile sections for the potential data, 
the following keywords may appear on the POTFIT directive
\begin{itemize}
\item CHARGE {\bf charge}   constrain charge
\item SYMMETRY   constrain symmetry equivalent atoms to have the same charge
\item DIPOLE {\bf dx dy dz}   constrain dipole (atomic units)
\item CUTOFF {\bf scale} exclude points closer than  {\bf scale} * the covalent radius
from a nucleus.
\end{itemize}

\section[Mulliken Analysis]{Mulliken Analysis}

The purpose of this module is to provide for an increased level
of analysis of a given set of molecular orbitals. The default SCF
options will typically provide a population analysis of the total SCF
wavefunction. In some instances it is useful to probe the individual
molecular orbitals , extracting quantities such as atom-atom overlap
populations and sub-dividing the orbital electrons into s,p and
d character (say) on the component atoms. Such an  analysis may be
requested under control of the MULLIK directive:

\subsection[MULLIK]{MULLIK}

This directive consists of a single data line, and is used to define 
those molecular orbitals for which
a detailed analysis is required, and to define the type of
analysis to be performed.

The first data field consists of the character string MULLIK.  One or two
data fields may then be read in A-format to define the type of  Mulliken
analysis required.  This analysis is performed by initially assigning
basis functions to 'groups', and then performing the Mulliken analysis
over these groups, rather than over individual basis functions. Two such
groups are recognised by the program, and may be activated by appropriate
keyword setting on the MULLIK directive. The groups, with appropriate
keyword settings, are as follows:

\begin{itemize}
\item Specifying the keyword ATOM will assign 
all basis functions sited on 
a given atomic centre to the same group, with the group
labelled by the centre name (based on that  specified by the ZMATRIX or
GEOMETRY directive). This will culminate in an atom-atom type of
Mulliken analysis.
\item Specifying the keyword ORBITAL will provide a 
more detailed analysis than
above, with the atomic groups further classified into sub-groups
of specific orbital (s, p, d or f) character.
\end{itemize}
Subsequent data fields are used to specify
those molecular orbitals whose analysis is to
be printed.  If NMO is the total number of
orbitals whose analysis is required, subsequent data fields should
contain NMO integer numbers read in free-I format. An abbreviated form
form of this data specification allows the user to introduce the
character string TO, read in free A-format, when specifying a sequence
of consecutive orbitals. 
The last data field presented should be the character string END.
Note that the integer and END data fields
may be omitted, in which case only the Mulliken
analysis of the total wavefunction will be printed.

The following notes may be helpful:
\begin{itemize}
\item Each  molecular orbital is individually analysed 
as if it contained
one electron. The occupation numbers (as retrieved from the 
Dumpfile section nominated on the VECTORS directive) 
are only used in the
evaluation of the Mulliken populations of the total wavefunction.
\item While all molecular orbitals regardless of occupation number, are
analysed, the detailed analysis of a given orbital will only be
printed if that orbital is referenced by the MULLIK directive.
\end{itemize}

{\bf Example 1}
{
\footnotesize
\begin{verbatim}
          MULLIK ATOM ORBITAL 2 3 4 5 6 7 9 END
\end{verbatim}
}
Requests an atom- and orbital-group based analysis for
molecular orbitals 2 to 7 inclusive, plus molecular orbital 9\\

{\bf Example 2}\\

The data line below has an equivalent effect to that of Example 1.
{
\footnotesize
\begin{verbatim}
          MULLIK ATOM ORBITAL 2 TO 7 9 END
\end{verbatim}
}
has an equivalent effect to that of Example 1.\\

{\bf Example 3}\\

The following data line will request an atom-grouped analysis
for the total wavefunction alone, with no printed analysis of the
individual orbitals.

{
\footnotesize
\begin{verbatim}
          MULLIK ATOM 
\end{verbatim}
}

\section[Distributed Multipole Analysis]{Distributed Multipole Analysis}

The DMA analysis \cite{stone} is instigated by the directive DMA.

By default, the DMA module will generate an expansion with multipoles
at the atomic sites, with a maximum l value (rank) of 10 for the poles at each site.
Each site is assigned a relative radius (see below) which is used in the partitioning
of the overlap density between the sites. By default, all relative radii
are set to 1.0.

A number of subdirectives may be presented after the DMA line, these
serve to modify the selection of DMA sites, and the distribution algorithms
for partitioning between the sites.

% c
% c      data word / 'add', 'dele', 'gaug',
% c     &     'limi', 'line', 'gene', 'atom', 'repo',
% c     &     'corr', 'nonu', 'shif', 'note',
% c     &     'scf',  'dens', 'mos', 'radi'/
% c
% c  atoms   (default) move all contributions to nearest atom.
% c
% c  report  print the multipole contributions of each pair of primitives
% c          as the calculation proceeds.
% c
% c  corrections
% c          use the perturbation correction to the density matrix
% c          instead of the scf density matrix. scf can be specified to
% c          use the scf density matrix, but this is the default.
% c
% c  density isec
% c          the density matrix is to be read from section isec of the
% c          dumpfile.
% c
% c  mos isecv
% c          the density matrix is in the m.o. basis , and the appropriate
% c          m.o.'s are in section isecv. default is section nominated
% c          on vectors directive
% c
% c
% 

\subsection{ADD}
{
\footnotesize
\begin{verbatim}
          ADD name x y z  <lmax <radius>>
\end{verbatim}
}
  Add a new site at (x,y,z) with the name specified.  the
  multipole rank is limited to lmax if a value is specified,
  and a relative radius can be specified also.

\subsection{DELETE}
{
\footnotesize
\begin{verbatim}
          DELETE name
\end{verbatim}
}
   Delete all sites of the given name.  DELETE ALL deletes
   all sites. DELETE CHARGE  deletes nuclear charges on the atoms.

\subsection{RADIUS}
{
\footnotesize
\begin{verbatim}
          RADIUS name radius
\end{verbatim}
}
 Specify a relative radius for all sites with the name given.
 The actual distances from an overlap centre to the sites are
 scaled by dividing by the relative radii of the sites, and
 the contributions are moved to the site which is closest, in
 terms of scaled distances, to the overlap centre. The default
 is that all sites have relative radius 1.0.

\subsection{LIMIT}
{
\footnotesize
\begin{verbatim}
          LIMIT name lmax
\end{verbatim}
}

    Limit the rank of multipoles on sites with the name given
    to lmax at most.  Contributions with higher ranks are
    moved to other sites. If no name is given the limit applies
    all sites. default (and maximum) is 20 for the linear
    version, 10 otherwise.

\subsection{SHIFT}
{
\footnotesize
\begin{verbatim}
          SHIFT  tshift
\end{verbatim}
}
 Distribute multipoles from an overlap contribution around
 several dma sites, using a Gaussian weighting function.
 tshift is a cutoff parameter; maximum 1, minimum 1e-6.
 A value of 1.0 (the default) means distribution is to the nearest 
 site only.

% \all\ \a\ \b\
%  only. all means that overlaps coincident with dma sites
%  are also distributed. a and b are different methods for
%  doing the distribution.

\subsection{LINEAR}

This directive invokes a faster version of the DMA program,
which is applicable when all the
atoms lie in a line parallel to the z axis and only the
z components of the multipoles are required.  In this
case the maximum rank is 20. The option is revoked if the
molecule is found not to be linear. If the molecule is subject
to an external field, or is not in a singlet sigma state, there
may be other non-vanishing multipole moments which will not be
calculated; however the ql0 will still be correct.

\subsection{GAUGE}
{
\footnotesize
\begin{verbatim}
          GAUGE ox oy oz
\end{verbatim}
}
The GAUGE directive resets the coordinate origin, and is 
followed by the coordinates of the new origin.

The distributed multipole analysis is not affected, but the
total multipoles are referred to the new origin. The sites 
(as printed out and in the punchfile) are specified with respect to
the {\em new} origin.

\subsection{NONUCLEAR}
    The nuclear contribution to the multipoles is not evaluated.

\clearpage


\section[The DRF Model for Solvation]{The DRF Model for Solvation}

The (Direct) Reaction Field model for solvation, developed at the University of
Groningen \cite{vries,duijnen}, is an embedding
technique enabling the computation of the interaction between a
quantum-mechanically described molecule and its
classically described surroundings. The classical surroundings may be
modelled in the following ways:
\begin{enumerate}
\item by point charges to model the electrostatic field due to the
surroundings
\item by polarizabilities to model the (electronic) response of the surroundings
\item by an enveloping dielectric to model bulk response (both static
and electronic) of the surroundings
\item by an enveloping ionic solution, characterized by its Debye
screening length
\end{enumerate}

The four representations may be combined freely
to model all aspects of the surroundings. The best results with this
model for solvation studies have been obtained by immersing the QM
solute by 2-3 layers of explicitly described (point charges and
polarizabilities) solvent molecules, enveloped by a surface defining
the boundary between the microscopic system and a dielectric with
bulk-solvent properties (dielectric constant).\cite{duijnen} The model has also been
applied to active sites in proteins.\cite{dijkman} Special care has to
be taken to avoid spurious electrostatic and
reaction-field interactions with nearby atoms when bonds between the
QM and classical systems are cut and
the QM system is capped by H-atoms to satisfy valence. These aspects of
embedding are the subject of ongoing research.

Having decided on the QM system and the representation of the
surroundings, the embedding may be treated at the following levels:
\begin{enumerate}
\item electrostatic potential as a perturbation

The QM density is
calculated as if the QM system were in vacuum. The interaction with
the point charges is then calculated with
the vacuum density.

\item electrostatic potential and reaction field as a perturbation

The QM density is
calculated as if the QM system were in vacuum. The interaction with
the point charges, polarizabilities, and dielectric is then calculated with
the vacuum density.

\item electrostatic potential self-consistently

The QM density is
calculated in the presence of the potential generated by the point charges
by including this field in the one-electron hamiltonian.

\item electrostatic potential self-consistently and reaction field as a
perturbation

The QM density is
calculated in the presence of the potential generated by the point charges
by including this field in the one-electron hamiltonian. The
interaction with
the polarizabilities and dielectric is then calculated with this
density.

\item electrostatic potential and reaction field self-consistently

The QM density is
calculated in the presence of the potential generated by the point
charges, and the reaction field due to induced dipoles at the
polarizabilities and surface polarization at the dielectric boundary,
by including these fields in the one- and two-electron parts of
hamiltonian, and Fock-matrix, respectively.
\end{enumerate}

For many systems, the difference in total energy between the fourth
and fifth levels is small; the self-consistent treatment of the
electrostatic field is often found to change results substantially from
a fully perturbative treatment.

The response of the surroundings is calculated through solving the
set of linear equations that result from coupling the moments induced
at the polarizabilities and at the boundary elements, the latter
resulting from the linearized Poisson-Boltzmann equation. This formally
requires the inversion of what is called the relay-matrix, which
contains all the couplings. The procedure implemented here is the
LU-decomposition of the relay-matrix. The LU-decomposition is the most
expensive step in the embedding procedure. Care should therefore be
taken to limit the number of polarizabilities and boundary
elements. The relay-matrix is square matrix of a maximum of (3*N$_{pol}$ +
N$_{be}$) X (3*N$_{pol}$ + N$_{be}$) elements, where N$_{pol}$ is the number of polarizabilities, and N$_{be}$ is
the number of boundary elements. In the case of specifying a non-zero
ionic strength of the dielectric, another N$_{be}$ is added to the
dimensionality of the relay-matrix.


\section{Total energy in the DRF model}

The total energy in the DRF model is may be partitioned in various
contributions:
\begin{enumerate}
\item{energy of the quantum system}

This is the energy of the
quantum-mechanical system, as calculated with the supported wave
functions. Any change in energy upon interaction with the classical
system is to be measured against a vacuum calculation on the
quantum-mechanical system.

\item{energy of the classical system}

This is the energy of the
classical system, calculated as if there were no QM system
present. The zero of energy depends on the specification of the
classical system; for example, atoms may be defined to be part of
molecules, excluding their interactions, making the infinitely separate
molecules defining the zero of energy, rather than infinitely separate
atoms.

In this energy interactions between classical
subsystems (e.g. molecules) are included. They may be electrostatic, dispersion,
repulsion, and induction interactions.

\item{interaction energy}

This is the sum of all separable interaction energies:
\begin{enumerate}

\item{electrostatic interactions}: interactions between point charges
in the classical system and the QM charge distribution (nuclei and electrons).
\item{Induction Interactions}, also called screening. This is the
interaction of one subsystem with the reaction field induced by
another subsystem. The interaction of the subsystem with its own
reaction field is also part of the interaction, and is used to
calculate the polarization energy, which is half the energy gain from
induction at equilibrium.
\item{dispersion interactions}: an estimate of the dispersion energy
between QM and classical subsystems,
based upon the Second-order Perturbation (SOP) expression for the
dispersion interaction, may be calculated (see also section \ref{estdisp}).
\item{model repulsion energy}: a molecular mechanics force-field
expression from CHARMM \cite{charmm} is used to model (Pauli) repulsion between
subsystems.

\end{enumerate}

\end{enumerate}

NOTE: Electrons do not 'feel' the repulsion and care has to be
taken to avoid close contacts, which may lead to electrons 'wandering
off' the QM system. Devices to lessen this effect are
available (see section \ref{dampfunc}) 

Care should be taken in comparing total energies that the proper
reference systems have been defined.

In a single calculation, one does not have access to the polarization
energy of the QM system: this is the change of the expectation value
of the vacuum hamiltonian of the QM distribution (and therefore internal QM
energy) upon interaction with the surroundings. A separate calculation in the absence of the surroundings (or
with the effect of the surroundings treated as a perturbation only) is necessary to
obtain this (interaction) energy.

The definition of the zero of energy for the classical system has been
discussed above. The total energy of QM + classical system is given
on output under: configuration total energy. Contributions have been
described in the Output section \ref{drfoutput}.


\section{Features of the DRF Model}


\subsection{Grouping of External Points}
\label{drfgroups}

Points that carry point charges and polarizabilities may be grouped
together. Interactions between members of the same group (e.g. a
molecule) may be excluded. Also, the dimensionality of the coupling
matrix may be reduced by using group polarizabilities, rather than
atomic polarizabilities. For instance, for a classical surroundings
comprising of any number water molecules, described by 3 atoms each, a
reduction of a factor 3 in the dimensionality of the coupling matrix
may be achieved by using one molecular polarizability for each water
molecule. For small molecules, the results from using group
polarizabilities do not differ considerably from using distributed
atomic polarizabilities if the molecules are distant enough from the
inducing electrostatic field.

Criteria for grouping may be supplied by the user, and groupings may be
suggested, which are then tested against the criteria. Grouping may
also be enforced.

Thole has shown that for small organic molecules group
polarizabilities from atom polarizabilities yield very good molecular
polarizabilities, compared to experiment.\cite{thole1} He derived a number of
parameter sets for the atomic polarizabilities, based on an
assumption about the charge-density distribution around an atom. The
shape of the charge density modifies the field at atoms within the
molecule due to dipoles induced at other atoms in the molecule. The
density--shape-functions implemented here are the homogeneous--conical and the exponentially
decreasing spherical shapes.


\subsection{Damping Functions}
\label{dampfunc}

Problems with numerical stability and/or unphysical behaviour may
arise when using the full Coulomb potentials and fields for the
calculation of interactions between subsystems that are nearby. Use of
full Coulomb potentials does not account for any overlap of the
charge-clouds that would damp the potential if both subsystems were
treated quantum-chemically.

Thole developed damping functions to account for overlap effects, and
they may be used to damp the potentials and fields between QM system
and classical systems and between classical subsystems.\cite{thole1} Two types of
damping function are available:
\begin{enumerate}
\item damping based on the assumption that the charge is distributed homogeneously
in a cone of certain width around the nucleus/expansion centre
\item damping based on the assumption that the charge is
exponentially decreasing radially from the nucleus/expansion centre
\end{enumerate}

Both models have been used to construct molecular polarizabilities
from atomic polarizabilities.\cite{thole1}

\subsection{CHARMM Model Repulsion}
\label{modrep}

The repulsion between atoms in the QM system and classical system, and
between the atoms in the classical subsystems is
treated by using the CHARMM force-field expression:

{\large
\begin{math}
E_{rep, CHARMM} = \sum_{i<j} {{ {3\over{4}} {\alpha_{i} \alpha_{j} (r_{i} +
r_{j})^6  \over {\sqrt{\alpha_{i}/n_{i}} + \sqrt{\alpha_{j}/n_{j}}}} r_{ij}^{-12}}}
\end{math}
}

where $\alpha_{i}$ is the polarizability of atom $i$, $r_{i}$ is the
radius of atom $i$, and $n_{i}$ is the
number of valence electrons of atom $i$, and $r_{ij}$ is the
distance between atoms $i$ and $j$.

The parameters used here do not come from the CHARMM force-field, but
are instead the DRF parameters. Several options for setting the radii
and polarizabilities are available.

A special treatment of H-bonds is possible, as in the CHARMM force-field. The radius of the H-atom
participating in a H-bond may be set differently from other H-atom radii.


\subsection{Estimate of the Dispersion Interaction}
\label{estdisp}

The interaction of the QM system with the polarizable environment is
calculated by taking expectation values of the reaction-field operator
over the wave function. The expectation values are of two kinds:\cite{thole2,thole3}
\begin{enumerate}
\item average: the interaction of the charge density as a whole with
the (dipole) moments induced by the charge distribution as a whole at
the polarizability; in formula:

{\large
\begin{math}
\langle0\vert F\dagger \vert0\rangle \alpha \langle0\vert F \vert0\rangle
\end{math}
}

where $\vert0\rangle$ is the wave function, $F$ the electric field operator, and $\alpha$
the polarizability.
\item direct: the screening of the self-interaction through the
interaction with the polarizable surrounding; in formula:

{\large
\begin{math}
\langle0\vert F\dagger \alpha F \vert0\rangle
\end{math}
}

\end{enumerate}

The availability of these two expectation values enables an estimate
of the dispersion interaction between QM systems and surroundings,
through a fluctuation expression:

{\large
\begin{math}
E_{disp, DRF} =  {1\over{2}} (\langle0\vert F\dagger \alpha F
\vert0\rangle - \langle0\vert F\dagger \vert0\rangle \alpha
\langle0\vert F \vert0\rangle)
\end{math}
}

The factor $1\over2$ comes in when accounting for the polarization costs.

This expression may be rewritten in the same form as the second-order
perturbation (SOP) expression for the dispersion between two fragments by
inserting the resolution of the identity in the first term. The $k=0$
terms in the first and second terms cancel, and:

{\large
\begin{math}
E_{disp, DRF} = {1\over{2}} {\sum_{k\not=0} \langle 0 \vert F\dagger
\vert k \rangle \alpha \langle k \vert F \vert 0 \rangle}
\end{math}
}

This expression is the same as the SOP expression in the Uns\"{o}ld
approximation, but for a factor

{\large
\begin{math}
{U_{S} \over U_{A} + U_{S}}
\end{math}
}

where $U_{S}$ and $U_{A}$ are mean excitation energies for the
polarizable and QM systems, respectively. This factor must be supplied
by the user. This factor may be estimated from e.g. ionization
energies (experimental or calculated). If $S$ and $A$ are identical,
the factor is $1\over2$. The user may specify this
factor to switch on this interaction. By default, this factor is set to
zero, and therefore the QM system is embedded in the Average Reaction
Field only.

The expressions above have been derived for a QM system interacting
with one polarizability. By substituting the effective response of
coupled polarizabilities and dielectric response for $\alpha$, the
estimate for the dispersion interaction with all the surroundings may
be made.

The dispersion interaction between classically described molecules is
given by the Slater-Kirkwood expression:

{\large
\begin{math}
E_{disp, clas} = {1\over{4}} { Tr(\alpha_{i} T_{ij}^{2} \alpha_{j}) \over
\sqrt{\alpha_{i}/n_{i}} + \sqrt{\alpha_{j}/n_{j}}}
\end{math}
}

where $\alpha_{i}$ is the polarizability of atom $i$, $n_{i}$ is the
number of valence electrons of atom $i$, and $T_{ij}$ is the
dipole--dipole interaction tensor between atoms $i$ and $j$. A minimum
number of parameters is introduced if one uses the same polarizabilities
defined for the reaction-field (induction) interaction in
this expression and the CHARMM repulsion as well.


\subsection{Expansion of Fields}

In this DRF implementation the electrostatic and reaction field
operators of the QM charge distribution are
expanded to second order, which greatly reduces the computational
effort because the coupling equations need to be solved only for a
limited number of source multipoles at each expansion centre, rather
than for every single overlap distribution. Each overlap distribution is assigned to an
expansion centre for which the static and reaction fields are calculated in
advance for a unit charge, dipole, and quadrupole. These formal interactions are
then simply multiplied by the overlap, dipole, and second moment
integrals of the charge distribution(s) in question to yield
one- and two-electron matrix elements.\cite{vries}

By default, the nuclei of the QM system are used as expansion centres,
but the user may add expansion centres by hand, or by automated procedures.

%The two-electron reaction-field matrix elements may be calculated in
%advance and added to the vacuum two-electron elements, either to the
%J- and K-matrices ({\tt SUPER OFF}), or the P- and K-supermatrices
%({\tt SUPER ON}), or may be calculated
%on the fly during the SCF-procedure. The
%value of the scaling-factor for the dispersion interaction (see above)
%is restricted to 1.0 with {\tt SUPER OFF}, because of the
%permutation symmetry exploited in the vacuum two-electron integrals in that scheme. The scaling-factor for
%the dispersion estimate may take any value in the other schemes.


Exact interaction with point charges may be obtained through using the
{\tt BQ} centres (see \S3.8.4) specified in the {\tt ZMATRIX} directive. As
far as the DRF model is concerned, these {\tt BQ} centres are part of
the QM system.


\subsection{Dielectric Response}

The dielectric response due to the continuum surrounding the molecular
system may be calculated separately for the static and optic
components in a single calculation. To this end, the two dielectric
constants may be given on input. If the Direct Reaction Field is
active, only the optic component of the dielectric response is coupled directly
to the charge distribution for calculation of the estimate of the
dispersion interaction, since this part reflects the electronic
part of the dielectric response; the static component is coupled
through the Average Reaction Field.

The properties of the dielectric continuum enveloping the molecular
system that may be specified are both the dielectric constant and the
ionic strength. This is due to the implementation of the
Poisson-Boltzmann equation-solver on the boundary surface, rather than
just the Poisson equation-solver. This option may be useful for
studying solvation effects in ionic solutions.

 

NOTE: At present, the analytical gradients (both on the QM system and on the classical
points (charge and/or polarizabilities)) are not available, limiting
geometry optimization capabilities of this module.


\section{DRF Directives}

The (Direct) Reaction Field model is invoked through specifying a
block of data marked by the single-line directives {\tt REACT} at the
start and {\tt END} at the end. Inside this block, the positions and
magnitude of the point charges and polarizabilities, the definition of
an enveloping surface to mark the beginning of a dielectric, and all 
DRF-options may be specified by their respective directives given
below. The definition of the QM system remains to be specified through
the {\tt ZMATRIX} directive. Any capping atoms to be introduced when
excising a QM system from a covalently bonded superstructure must be
specified with the QM system, since there is no automatic generation
for these atoms. Also, the user is responsible for avoiding close
contacts between QM (capping) atoms and classical atoms.


The {\tt REACT} subdirectives are given below.

\subsection{FIELD}

The {\tt FIELD} directive specifies the level of coupling between QM
system and surroundings.
The directive consists of a single data line, starting with {\tt
FIELD}, which may contain a number
of keywords followed by their values. The {\tt FIELD} keywords are:
\begin{enumerate}
\item {\tt STAT} specifies presence of static embedding potential,
followed by the desired level
\item {\tt REAC} specifies presence of response embedding potential,
followed by the desired level. The level of embedding may be:
\begin{enumerate}
\item {\tt NONE}: this embedding is not taken into account
\item {\tt PERT}: this embedding is added as a perturbation after
convergence of the wave function
\item {\tt SCF}: this embedding is treated self-consistently with the
wave function
\end{enumerate}
\end{enumerate}

The default is {\tt FIELD STAT NONE REAC NONE}


\subsection{DRFOUT}

The {\tt DRFOUT} directive serves to specify the level of output for
the DRF extension.
The directive consists of a single data line,
read to the variables {\tt TEXT, OUTOPT}, using format (2A)
\begin{enumerate}
\item {\tt TEXT} should be set to the character string {\tt DRFOUT}
\item {\tt OUTOPT} is the level of output for the DRF module. Valid
options are:
\begin{enumerate}
\item {\tt STANDARD}:
\item {\tt SOME}: Information on classical surroundings
\item {\tt MORE}: More information on classical surroundings,
printing of DRF settings, and flagging DRF-additions to hamiltonian
\item {\tt MATRICES}: {\tt MORE} plus DRF matrices produced
\item {\tt ONEEL}: {\tt MORE} plus one-electron information
\item {\tt TWOEL}: {\tt MORE} plus two-electron information
\end{enumerate}
\end{enumerate}

The default is {\tt DRFOUT STANDARD}.

NOTE: This directive must proceed the {\tt EXTERNAL} directive to be
effective there.

\subsection{UNITS}

The {\tt UNITS} directive specifies the unit of length used throughout
the {\tt REACT} input directives. The directive consists of a single data line
read to the variables {\tt TEXT, UNIT}, using format (2A),
\begin{enumerate}
\item {\tt TEXT} should be set to the character string {\tt UNITS}
\item {\tt UNIT} can be set to the character strings {\tt BOHR} or
{\tt ANGS}
\end{enumerate}

The default is {\tt UNITS BOHR}.

NOTE: All {\tt REACT} subdirectives will be assumed to be in the units
given here, e.g. {\tt DSTGRP 15.} will assign 15 Bohr to {\tt DSTGRP}
if {\tt UNITS BOHR}, but 15 Angstrom if {\tt UNITS ANGS} is
specified.

NOTE: This directive must proceed the {\tt EXTERNAL} directive to be
effective there.

\subsection{GAMDRF}

The {\tt GAMDRF} directive serves to specify the scaling factor to be
applied to the estimate of the dispersion energy (see also section \ref{estdisp}). A non-zero value
also triggers the use of the Direct, rather than Average
Reaction-Field coupling scheme.
The directive is
read to the variables {\tt TEXT, VALUE}, using format (A,F)
\begin{enumerate}
\item {\tt TEXT} should be set to the character string {\tt GAMDRF}
\item {\tt VALUE} is the scaling factor to be applied
\end{enumerate}

The default is {\tt GAMDRF 0.0}.

\subsection{INCLPOL}

The {\tt INCLPOL} directive serves to specify the strength of the
reaction field coupled to the QM system. The reaction field may either
be coupled back in full, or at half-strength only, in which case the
polarization energy is taken into account beforehand.
The directive is
read to the variables {\tt TEXT, OPTION}, using format (2A)
\begin{enumerate}
\item {\tt TEXT} should be set to the character string {\tt INCLPOL}
\item {\tt OPTION} may be either {\tt ON}, or {\tt OFF}. When {\tt
INCLPOL ON}, the polarization is taken into account beforehand 
\end{enumerate}

The default is {\tt INCLPOL ON}.

\subsection{EXPANDCM}

The {\tt EXPANDCM} directive specifies the use of the centre of
nuclear charge
of the QM partition as an extra expansion centre for the (D)RF
integrals. 
The directive consists of a single data line
read to the variables {\tt TEXT, EXPOPT}, using format (2A),
\begin{enumerate}
\item {\tt TEXT} should be set to the character string {\tt EXPANDCM}
\item {\tt EXPOPT} can be set to the character strings {\tt ON} or
{\tt OFF}
\end{enumerate}

The default is {\tt EXPANDCM OFF}.

\subsection{ASSIGN}

The {\tt ASSIGN} directive specifies the way the overlap distributions
are assigned to the expansion centres. The centre of a charge
distribution is defined by
$\langle i \vert \hat r \vert j \rangle / \langle i \vert j \rangle$.
The directive consists of a single data line
read to the variables {\tt TEXT, ASGNOPT}, using format (2A),
\begin{enumerate}
\item {\tt TEXT} should be set to the character string {\tt ASSIGN}
\item {\tt ASGNOPT} can be set to the following character strings:
\begin{enumerate}
\item {\tt DISTANCE}: assign overlap distributions on the basis of
distance of the centre of the overlap distribution to
the expansion centres. The overlap distribution is assigned to the
nearest expansion centre. If two or more centres are equally
close, the assignment is arbitrarily to the last centre encountered in
the search
\item {\tt OLAPTOCM}: assign all two-centre overlap distributions to
the expansion
centre at centre of nuclear charge (only with {\tt EXPANDCM ON})
\item {\tt EQUATOCM}: assign overlap distributions that are equally
close to one or more expansion centres to the expansion
centre at centre of nuclear charge (only with {\tt EXPANDCM ON})
\item {\tt MIDPTS}: add midpoints between atoms as expansion centres and assign as with
{\tt DISTANCE}
\item {\tt ALLDISTR}: expand each overlap distribution around its own
centre. If any expansion centre is close to one already defined, the expansion
centre will be shared
\end{enumerate}
\end{enumerate}

The default is {\tt ASSIGN DISTANCE}.

NOTE: Extra (other) expansion centres may be specified by using {\tt
BQ} centres in the {\tt ZMATRIX} directive. These centres will then be
used as expansion centres. Note that if the {\tt BQ} centres carry
charge, interaction with the QM atoms will be included in the QM
energy, interaction with classical atoms in the QM/classical interaction.



\subsection{GROUPING}

The {\tt GROUPING} directive specifies the attempt to construct group
polarizabilities from atomic polarizabilities on the basis of the
atoms' group name (see the {\tt EXTERNAL}
directive). It also serves to define the number of characters in the
group name of atoms used to exclude interactions between atoms.
The directive consists of a single data line
read to the variables {\tt TEXT, GROUPOPT, NGRNAM1, NGRNAM2}, using format (2A,2I)
\begin{enumerate}
\item {\tt TEXT} should be set to the character string {\tt GROUPING}
\item {\tt GROUPOPT} can be set to the character strings {\tt ON} or
{\tt OFF}
\item {\tt NGRNAM1} is the number of characters at the start of the group
name considered for exclusion of interactions
\item {\tt NGRNAM2} is the number of characters, after the name
considered for exclusion of interaction, considered for attempting to
group atom polarizabilities to a group polarizability
\end{enumerate}

The default is {\tt GROUPING OFF 2 2}.

NOTE: The {\tt GROUPING} directive and the following grouping criteria
are ignored if the user {\em forces} grouping of certain atoms. This
may be done within the {\tt EXTERNAL} data block by entering a
data-line reading {\tt GROUP} following a number of atoms. See also
the {\tt EXTERNAL} directive.

NOTE: {\tt NGRNAM1} specifies the number of characters considered
from the classical atoms' group name to
exclude electrostatic, dispersion, and repulsion interactions between
classical atoms. The {\tt NGRNAM2} characters of the
group name from {\tt NGRNAM1} onward are considered when deciding whether to construct a group
polarizability from the preceding
atom polarizabilities. Any remaining characters may be used for
additional labelling of the classical atoms. The energy does not
depend on these remaining characters.

NOTE: The names of the classical atoms are stored in a 16-character
string, the group name starting at character 7. There are thus a
maximum of 10 characters available for the group name, and {\tt NGRNAM1} may
not exceed 8 to leave 2 characters for grouping polarizabilities.

NOTE: This directive must proceed the {\tt EXTERNAL} directive to be
effective.

\subsection{DSTGRP}

The {\tt DSTGRP} directive specifies the minimum distance criterion for constructing group
polarizabilities from atomic polarizabilities under the control of the
{\tt GROUPING} directive. 
The directive consists of a single data line
read to the variables {\tt TEXT, DSTGRP}, using format (A,F)
\begin{enumerate}
\item {\tt TEXT} should be set to the character string {\tt DSTGRP}
\item {\tt DSTGRP} is the minimum distance any classical atom in the
prospective group must be separated from any QM atom for the grouping
to be considered. Grouping is not effected if any classical atom in
the prospective group is closer to any QM atom than {\tt DSTGRP} 
\end{enumerate}

The default is {\tt DSTGRP 15.0}.

NOTE: This directive must proceed the {\tt EXTERNAL} directive to be
effective.

\subsection{DSTMAX}

The {\tt DSTMAX} directive specifies the maximum distance criterion
for constructing group
polarizabilities from atomic polarizabilities under the control of the
{\tt GROUPING}
directive. The directive consists of a single data line
read to the variables {\tt TEXT, DSTMAX}, using format (A,F)
\begin{enumerate}
\item {\tt TEXT} should be set to the character string {\tt DSTMAX}
\item {\tt DSTMAX} is the maximum distance the nearest classical atom in the
prospective group should be from any QM atom to effect grouping. If that atom is
further removed, the whole group is discarded
\end{enumerate}

The default is {\tt DSTMAX 1000.0}.

NOTE: This directive must proceed the {\tt EXTERNAL} directive to be
effective.

\subsection{AGRPE}

The {\tt AGRPE} directive specifies the polarization energy difference
criterion for constructing group
polarizabilities from atomic polarizabilities under the control of the
{\tt GROUPING}
directive. The directive consists of a single data line
read to the variables {\tt TEXT, AGRPE}, using format (A,F)
\begin{enumerate}
\item {\tt TEXT} should be set to the character string {\tt AGRPE}
\item {\tt AGRPE} is the maximum allowed energy difference between the
reaction field interaction of a unit charge at the nearest QM atom as
calculated with and without the grouping. If the energy difference is
larger, grouping is not effected
\end{enumerate}

The default is {\tt AGRPE 0.001} (in Hartree).

NOTE: This directive must proceed the {\tt EXTERNAL} directive to be
effective.

\subsection{AGRPM}

The {\tt AGRPM} directive specifies the induced dipole difference
criterion for constructing group
polarizabilities from atomic polarizabilities under the control of the
{\tt GROUPING}
directive. The directive consists of a single data line
read to the variables {\tt TEXT, AGRPM}, using format (A,F)
\begin{enumerate}
\item {\tt TEXT} should be set to the character string {\tt AGRPM}
\item {\tt AGRPM} is the maximum allowed induced dipole difference between the
induced dipole due to a unit charge at the nearest QM atom as
calculated with and without the grouping. If the induced dipole difference is
larger, grouping is not effected
\end{enumerate}

The default is {\tt AGRPM 0.01} (in au).

NOTE: This directive must proceed the {\tt EXTERNAL} directive to be
effective.

\subsection{AGRPC}

The {\tt AGRPC} directive specifies the cosine
criterion for constructing group
polarizabilities from atomic polarizabilities under the control of the
{\tt GROUPING}
directive. The directive consists of a single data line
read to the variables {\tt TEXT, AGRPC}, using format (A,F)
\begin{enumerate}
\item {\tt TEXT} should be set to the character string {\tt AGRPC}
\item {\tt AGRPC} is the minimum allowed cosine of the angle between the
induced dipole moments due a unit charge at the nearest QM atom as
calculated with and without the grouping. If the cosine is
smaller, grouping is not effected
\end{enumerate}

The default is {\tt AGRPC 0.99}.

NOTE: This directive must proceed the {\tt EXTERNAL} directive to be
effective.

\subsection{DAMPING}

The {\tt DAMPING} directive serves to specify the type of
damping function and corresponding width-parameter to be used to damp
electrostatic fields by to avoid the so-called polarization
catastrophe (see section \ref{dampfunc}). It also controls the
selection of default atom polarizabilities.
The directive consists of a single data line, starting with {\tt
DAMPING}, which may contain a number
of keywords ({\tt AFCT} followed by its value). The {\tt DAMPING} keywords are:
\begin{enumerate}
\item {\tt OFF} to specify use of the full Coulomb operator
\item {\tt EXPO} to specify the damping associated with an exponentially decreasing
spherical shape function
\item {\tt CONE} to specify the damping associated with a uniform
conical shape function
\item {\tt AFCT} to specify the width parameter, the value following
the keyword. With {\tt EXPO}, the width parameter is the exponential
prefactor; with {\tt CONE}, the width parameter is the radius of the
cone. Beyond this radius, the charge
distribution behaves as a point charge
\end{enumerate}

The type of damping function used ({\tt CONE} or {\tt EXPO}) also
determines the {\em default atom polarizabilities} assigned to
classical atoms (see also \ref{drfgroups}), and possibly radii
assigned to both QM and classical atoms (see subdirectives {\tt
QMRADI} and {\tt CLASRADI}). If {\tt DAMPING OFF}, the polarizabilities belonging
to the damping function associated with the conical charge
distribution will be used.

The default is {\tt DAMPING OFF AFCT 1.662}.

NOTE: This directive must proceed the {\tt EXTERNAL} directive to be
effective.

\subsection{CLASDISP}

The {\tt CLASDISP} directive serves to specify the treatment of the
dispersion interaction between the classical atoms and groups.
The directive consists of a single data line read to the variables {\tt
TEXT, n(INTOPT)}, using format (nA)
\begin{enumerate}
\item {\tt TEXT} should be set to the character string {\tt CLASDISP}
\item {\tt INTOPT} may be either blank or one of following keywords:
\begin{enumerate}
\item {\tt GROUPPOL} or {\tt ATOMPOL} to switch between the use of
group polarizabilities and atom polarizabilities. If there are polarizable atoms that are
not used in the construction of any group polarizability, the
dispersion between those atom polarizabilities and any group 
polarizabilities is calculated if the {\tt GROUPPOL} option is
specified.
\item {\tt ORGPOL} or {\tt EFFPOL} to switch between the use of the
input atom polarizabilities and the effective atom polarizabilities
after group polarizability formation. This option affects atom
polarizabilities only
\item {\tt ISODIS} or {\tt NONISO} to switch between the use of
isotropic and anisotropic polarizabilities. Atom polarizabilities that
have not been used in the construction of any group polarizability
are always isotropic, and this option does not affect the dispersion
interaction between such atoms
\end{enumerate}
\end{enumerate}

The default is {\tt CLASDISP ATOMPOL ORGPOL ISODIS}.

NOTE: This directive must proceed the {\tt EXTERNAL} directive to be
effective in setting the correct polarizabilities.


\subsection{QMRADI}

The {\tt QMRADI} directive specifies the method for calculating the
radii of atoms in the QM partition for use in both construction of a
boundary surface marking the dielectric and the CHARMM QM/classical
repulsion, if appropriate.
The directive consists of a single data line
read to the variables {\tt TEXT, RADOPT}, using format (2A),
\begin{enumerate}
\item {\tt TEXT} should be set to the character string {\tt QMRADI}
\item {\tt RADOPT} is the option for selecting the radii. Valid
options are:
\begin{enumerate}
\item {\tt TABLE}: look-up from internal table (Bondi's van der Waals
radii)
\item {\tt CONEPOL}: calculated from Thole's polarizability optimized
for the damping according to a conical charge-density
\item {\tt EXPOPOL}: calculated from Thole's polarizability optimized
for the damping according to a exponential charge-density
\end{enumerate}
\end{enumerate}

The default is {\tt QMRADI TABLE}

NOTE: The radius is calculated as $r = f_{shape} \alpha^{1/3}$, with
$f_{shape}$ equal {\tt AFCT}, specified in subdirective {\tt DAMPING}.

\subsection{CLASRADI}

The {\tt CLASRADI} directive specifies the method for calculating the
radii of atoms in the classical partition for use in both construction of a
boundary surface marking the dielectric and the CHARMM QM/classical
repulsion, if appropriate.
The directive consists of a single data line
read to the variables {\tt TEXT, RADOPT}, using format (2A),
\begin{enumerate}
\item {\tt TEXT} should be set to the character string {\tt CLASRADI}
\item {\tt RADOPT} is the option for selecting the radii. Valid
options are:
\begin{enumerate}
\item {\tt TABLE}: look-up from internal table (Bondi's van der Waals
radii)
\item {\tt CONEPOL}: calculated from Thole's polarizability optimized
for the damping according to a conical charge-density
\item {\tt EXPOPOL}: calculated from Thole's polarizability optimized
for the damping according to a exponential charge-density
\item {\tt USERPOL}: calculated from assigned polarizability, whether
given on input (see {\tt EXTERNAL} subdirective) or through look-up
\end{enumerate}
\end{enumerate}

The default is {\tt CLASRADI TABLE}

NOTE: The radius is calculated as $r = f_{shape} \alpha^{1/3}$, with
$f_{shape}$ equal {\tt AFCT}, specified in subdirective {\tt DAMPING}.

NOTE: This directive must proceed the {\tt EXTERNAL} directive to be
effective.


\subsection{EXTERNAL}

The {\tt EXTERNAL} directive marks the beginning of the block input
for x-, y-, z-coordinates, charge, polarizability, and radius of the
atoms in the classically treated system. The block is closed with an
{\tt END} directive.

The attributes of each atom in the classically treated system are
input on a single data line, in the following order, but with free
format; first 2 character strings, then 6 numerics:

name (2 parts) -- charge -- x, y, z co-ordinate --
polarizability -- radius

\begin{enumerate}
\item name:

The name of a classical atom consists of two parts:

\begin{enumerate}
\item chemical symbol:

The chemical symbol is a maximum of 2 characters. Apart from the
elements, classical points may also be named {\tt XX}, {\tt QQ}, and
{\tt E}, which give them special attributes.
\begin{enumerate}
\item {\tt QQ}: point-charge only, no polarizability, no radius
(polarizability and radius given on the data line are ignored, nominal
unit polarizability for use in damping function only)
\item {\tt XX}: point-charge with default unit polarizability and unit
radius (polarizability and radius given on the data line are effective)
\item {\tt E }: point-charge and polarizability, no radius (radius
given on the data line is ignored)
\end{enumerate}

NOTE: The extra atoms without radius ({\tt QQ} and {\tt E}) are
ignored in the calculation of the model repulsion. {\tt XX}-atoms do
partake in the repulsion; they are assigned 1 valence electron. For
use in the Slater--Kirkwood formula for the dispersion interaction,
{\tt E}-atoms are assigned 1 valence electron.

\item group name:

The group name serves three purposes:
\begin{enumerate}
\item to exclude interactions between classical
atoms. The exclusions are controlled by the first part of the group
name; the first {\tt NGRNAM1} characters are considered when deciding
the exclusion of interactions
\item to define for which polarizabilities grouping to form a group
polarizability should be attempted. The {\tt NGRNAM2}
characters of the group name following the {\tt NGRNAM1} characters
are considered when deciding whether to attempt grouping the
preceding atom polarizabilities
\item label the classical atoms 
\end{enumerate}

\end{enumerate}

NOTE: {\tt NGRNAM1} and {\tt NGRNAM2} may be
specified on the {\tt GROUPING} directive (see above). 

NOTE: The names of the classical atoms are stored in a 16-character
string, the group name starting at character 7. There are thus a
maximum of 10 characters available for the group name, and {\tt NGRNAM1} may
not exceed 8, to leave at least 2 characters for the grouping name.

NOTE: A group is formed on the basis of the group name if the relevant
part of the group name changes from one atom to the next, and
formation of group polarizabilities is requested (see the {\tt
GROUPING} directive). 

NOTE: The specification of a group name is mandatory. If no use is to
be made for grouping purposes, the best way to ensure independent treatment of
the atoms is by using the group name to label the atoms uniquely.

NOTE: When grouping under the {\tt GROUPING} directive, a change in
group name at the {\tt NGRNAM2} characters governing grouping will
signify the attempt to group the preceding atoms. If further down the
list, the same group name for grouping is used again, the first set of
atoms will not be considered again.

NOTE: The group name of the resulting group
centre inherits the group name of the first atom in the newly formed
group; the group centre gets this atom's index, and this atom is then
put down as the last member of the group.

\item x-, y-, z-coordinates:

Cartesian co-ordinate input for classical atoms is supported only.

\item polarizability:

Atomic polarizabilities may be specified optionally. If not
specified, default values are used, according to the chemical symbol,
and the setting of the shape-function used for damping potentials (see
section \ref{dampfunc}).

NOTE: if the potentials are undamped, the polarizabilities associated
with the conical shape function will be used.

\item radius:

Atomic radii may be specified optionally. If not
specified, default values are used, according to the chemical
symbol, under control of the {\tt CLASRADI} directive (see above).

NOTE: the radii are used in the CHARMM repulsion expression and for
definition of a Juffer or Connolly surface if such a surface is
requested (see the {\tt DIELECTRIC} directive).

\end{enumerate}

Within the {\tt EXTERNAL} block one other directive may be specified
to force grouping of the preceding polarizabilities:

{\tt GROUP ANAL n}:

Forces construction of a group polarizability from the preceding
atoms, following the previous group. This directive supersedes the
formation of groups on the basis of group names (see above). Thus, a
{\tt GROUP} directive placed within a series of atoms with the same
second part of the group name will result in two, rather than one
group polarizabilities from those atoms.

The {\tt ANAL n}, where n is an integer (maximum 10) is an optional
addition to the forced grouping, causing analysis of the RF
contributions to the total energy to be split into groups. {\tt n}
provides an index into the analysis groups. In this
way, contributions from e.g. first and second solvent shells, or from
different residues in a protein may be separated.

{\bf EXAMPLE: 2 classical water molecules}

Here are a numer of examples of 2 classical water molecules:

1. {\tt EXTERNAL} part of input:

{\scriptsize
\begin{verbatim}
 external
o   w1      -0.796       3.459378         4.492394     0.0
h   w1       0.398       3.459378         5.595494    +1.4325
h   w1       0.398       3.459378         5.595494    -1.4325
group anal 1
o   w2      -0.796       3.459378        -4.492394     0.0      8.4  3.1
h   w2       0.398       3.459378        -5.595494    +1.4325   3.2  2.5
h   w2       0.398       3.459378        -5.595494    -1.4325   3.2  2.5
group anal 2
 end
\end{verbatim}
}

With all other directives at their default values, the following
attributes are given to the classical system (copied from output):

{\scriptsize
\begin{verbatim}
 induced dipole criterion for grouping not met
 WARNING: grouping forced by user, but grouping criteria not met
 grouping information may be obtained by specifying  the REACT subdirective DRFOUT MORE

   6 points found on input of which    6 polarisable
   constructed
      8 (charged) points
      0 point polarisabilities and
      2 group polarisabilities



 classical system specification
name                x         y         z         charge    radius alfa (b**3) effalf (b**3) analysis

 group w1gr        3.459378  5.092418  0.000000  0.000000    3.542     9.678     0.000         1
 h     w1          3.459378  5.595494  1.432500  0.398000    2.267     0.000     3.469         1
 h     w1          3.459378  5.595494 -1.432500  0.398000    2.267     0.000     3.469         1
 o     w1          3.459378  4.492394  0.000000 -0.796000    2.872     0.000     5.817         1
 group w2gr        3.459378 -4.969410  0.000000  0.000000    3.688    10.925     0.000         2
 h     w2          3.459378 -5.595494  1.432500  0.398000    2.500     0.000     3.200         2
 h     w2          3.459378 -5.595494 -1.432500  0.398000    2.500     0.000     3.200         2
 o     w2          3.459378 -4.492394  0.000000 -0.796000    3.100     0.000     8.400         2
\end{verbatim}
}

Points to note:

\begin{enumerate}
\item a warning is given that the grouping may not be of enough
accuracy to be justified. The criteria are given in the directives
{\tt AGRE}, {\tt AGRPM}, and {\tt AGRPC}.
\item the group centres head the group members; the first member of
the group on input is now the last atom in the group. The first part
of the centre's group name {\tt w1} is inherited from the first atom;
the second part {\tt gr} is added internally in the absence of further
group-name input.
\item the radius of the group centre is calculated from the group
polarizability, and depends on the settings of the directive {\tt DAMPING}.
\item the radii and polarizabilities of the first water molecule are
taken from internal data bases, and depend on the settings of the
directives {\tt CLASRADI}, and {\tt DAMPING}, respectively.
\item the radii and polarizabilities of the second water molecule are
taken from input, overriding any internal settings. The group
polarizability is, however, constructed under the settings of the {\tt
DAMPING} directive, as is the radius of the group centre.
\item the values of {\tt alfa} are the mean polarizabilities
used in calculation of the relay matrix to yield induction and
polarization terms. For group polarizabilities, the polarizability
tensor need not be (and in general isn't) isotropic.
\item the values of {\tt effalf} are the (mean) polarizabilities that
are used in calculation of the dispersion interaction between the classical
subsystems. The effective polarizabilities used need not be
isotropic. Their value and use depend on the settings of the {\tt
CLASDISP} directive. In this example, the isotropic atomic polarizabilities as
given on input are used. (The default is {\tt CLASDISP ATOMPOL ORGPOL ISODIS}.)
\item the water molecules are put into two analysis groups ({\tt 1}
and {\tt 2}). This will lead to final output in which the interaction
between the first and second molecule is separated out, as is true for
their respective interactions with the QM system.
\end{enumerate}



2. The following {\tt EXTERNAL} input is processed under the {\tt
GROUPING ON 2 3}, and {\tt DSTGRP 5.0} directives, with other
directives at their default values:

{\scriptsize
\begin{verbatim}
 external
o   w1gr1   -0.796       3.459378         4.492394     0.0
h   w1gr1    0.398       3.459378         5.595494    +1.4325
h   w1gr1    0.398       3.459378         5.595494    -1.4325
xx  w2gr2   -0.796       3.459378        -4.492394     0.0      8.4  3.1
xx  w2gr2    0.398       3.459378        -5.595494    +1.4325   3.2  2.5
xx  w2gr2    0.398       3.459378        -5.595494    -1.4325   3.2  2.5
 end
\end{verbatim}
}


The relevant part of the output is:

{\scriptsize
\begin{verbatim}
 induced dipole criterion for grouping not met
 - - - criteria for grouping not met, grouping atoms           5 to           7
  not effected

   6 points found on input of which    6 polarisable
   constructed
      7 (charged) points
      3 point polarisabilities and
      1 group polarisabilities



 classical system specification
name                x         y         z         charge    radius alfa (b**3) effalf (b**3) analysis

 group w1gr1       3.459378  5.092418  0.000000  0.000000    3.542     9.678     0.000         1
 h     w1gr1       3.459378  5.595494  1.432500  0.398000    2.267     0.000     3.469         1
 h     w1gr1       3.459378  5.595494 -1.432500  0.398000    2.267     0.000     3.469         1
 o     w1gr1       3.459378  4.492394  0.000000 -0.796000    2.872     0.000     5.817         1
 xx    w2gr2       3.459378 -4.492394  0.000000 -0.796000    3.100     8.400     8.400         1
 xx    w2gr2       3.459378 -5.595494  1.432500  0.398000    2.500     3.200     3.200         1
 xx    w2gr2       3.459378 -5.595494 -1.432500  0.398000    2.500     3.200     3.200         1
\end{verbatim}
}


Points to note:

\begin{enumerate}
\item grouping the first molecule has succeeded; all criteria are met,
and the resulting attributes are the same as for the forced grouping.
\item grouping of the second molecule was not effected because the
induced dipole criterion was not met, and the atoms are left as on input.
\item no separate analysis of the interactions of the two molecules
will be made (it is not possible to specify an analysis group without
forced grouping through the {\tt GROUP} directive.
\item the energy of the classical system and the interaction of the
classical system with the QM system will be different for cases 1. and
2. for the following type of interaction:
\begin{enumerate}
\item dispersion between classical subsystems: this is because an {\tt
XX} centre is assigned 1 valence electron. In case 1. one of the {\tt
XX} centres is an {\tt O} with 6, rather than 1 valence electron.
\item repulsion between classical subsystems: again this is due to the
assignment of just 1 valence electron to {\tt XX}.
\item induction and polarization energies: this is because in case 2.,
the polarizability of the second molecule is represented by a distributed, rather than group
polarizability. This will affect the response of the system.
\item dispersion energy between QM and classical system: this is again
because of the difference in response of the classical system.
\end{enumerate}
\end{enumerate}

\subsection{DIELECTRIC}

The {\tt DIELECTRIC} directive specifies the options for the dielectric
continuum. It consists of a block data structure terminated by an {\tt
END} directive.

Below follow the descriptions of the {\tt DIELECTRIC} subdirectives.

\subsubsection{SURFACE}

The {\tt SURFACE} directive serves to specify the type of surface
enveloping QM and classical atoms to be constructed.  The directive consists of a single data line
read to the variables {\tt TEXT, SURFOPT}, using format (2A)
\begin{enumerate}
\item {\tt TEXT} should be set to the character string {\tt SURFACE}
\item {\tt SURFOPT} is the type of surface to be constructed. The
following options are valid:
\begin{enumerate}
\item {\tt SPHERE}: The surface is a triangulated sphere around (0,0,0). The
radius and the number of surface points may be further specified by
the {\tt DIELECTRIC} subdirectives {\tt SRADIUS} and {\tt BEMLEV}, respectively
\item {\tt JUFFER}: The surface is generated by Juffer's method, which
distorts a triangulated sphere to envelop the atoms according to
the molecular shape. The surface is constructed by searching for the
atom furthest away from the centre of the initial sphere along a
number of spokes. Each spoke defines a cylinder within which an atom
is to be found. If no atom is found within the cylinder, the program
terminates. The number of surface points may be further specified by
the {\tt DIELECTRIC} subdirective {\tt BEMLEV}, and the location of
the surface w.r.t. the atom-positions by the subdirectives {\tt JUFFER}, and
{\tt CONNOLLY}
\item {\tt CONNOLLY}: The surface is generated by using M.L. Connolly's
surface program MSCON,\cite{connolly} which has been made part of GAMESS-UK. Further specification of the Connolly surface
may be given through the {\tt CONNOLLY} directive
\end{enumerate}
\end{enumerate}

The default is {\tt SURFACE SPHERE}.


\subsubsection{SRADIUS}

The {\tt SRADIUS} directive serves to specify the radius of a
triangulated sphere around (0,0,0) defining the dielectric boundary
(with {\tt SURFACE SPHERE}).
The directive is
read to the variables {\tt TEXT, SPHERAD}, using format (A,F)
\begin{enumerate}
\item {\tt TEXT} should be set to the character string {\tt SRADIUS}
\item {\tt SRADIUS} is the radius of the triangulated sphere
\end{enumerate}

The default is {\tt SRADIUS 1000.0}.


\subsubsection{BEMLEV}

The {\tt BEMLEV} directive serves to specify the number of elements of
the triangulated surface defining the dielectric boundary. This
directive is active when the {\tt SURFACE} keyword is either {\tt
SPHERE} or {\tt JUFFER}.
The directive is
read to the variables {\tt TEXT, BEMLEV}, using format (A,I)
\begin{enumerate}
\item {\tt TEXT} should be set to the character string {\tt BEMLEV}
\item {\tt BEMLEV} is the level of detail of the triangulated
sphere. Valid options are:
\begin{enumerate}
\item {\tt 0} : results in 60 boundary elements
\item {\tt 1} : results in 240 boundary elements
\item {\tt 2} : results in 960 boundary elements
\end{enumerate}
\end{enumerate}

The default is {\tt BEMLEV 0}.


\subsubsection{JUFFER}

The {\tt JUFFER} directive serves to specify the generation of a
Juffer triangulated surface enveloping QM (and classical, if present) atoms. The
directive consists of a single data line, starting with {\tt JUFFER}, which may contain a number of
keywords followed by their values. The {\tt JUFFER} keywords are:
\begin{enumerate}
\item {\tt CYLWDTH}: width of a cylinder around a spoke. The atom within
this cylinder which is furthest away from the centre of the molecular
system defines the position of the surface element associated with the
spoke. If {\tt CYLWDTH} is not set, or $<$ 0.0, it defaults to 2*maxrad/({\tt
BEMLEV}/2 +1) + {\tt RPROBE}, where maxrad is the largest atomic radius
of the QM and classical systems
\item {\tt SURFDIST}: distance of the surface-defining point to the
atom generating the surface. If {\tt SURFDIST} is not set, or $<$ 0.0, it
defaults to maxrad + {\tt RPROBE}, where maxrad is the largest atomic radius
of the QM and classical systems
\end{enumerate}

The default is {\tt JUFFER CYLWDTH -1.0 SURFDIST -1.0}.

NOTE: {\tt RPROBE} may be defined in subdirective {\tt CONNOLLY}.


\subsubsection{CONNOLLY}

The {\tt CONNOLLY} directive serves to specify the generation of a
Connolly surface enveloping QM (and classical, if present) atoms. The
directive consists of a single data line, starting with {\tt
CONNOLLY}, which may contain a number
of keywords followed by their values. The {\tt CONNOLLY} keywords are:
\begin{enumerate}
\item {\tt MXSURPTS}: maximum number of surface elements to be
generated by the Connolly algorithm. The algorithm is satisfied if the
number of surface points is between {\tt MNSURPTS} and {\tt MXSURPTS}
\item {\tt MNSURPTS}: minimum number of surface elements to be
generated by the Connolly algorithm. If unspecified, it is set equal
{\tt 0.8*MXSURPTS}
\item {\tt SPDENS}: Connolly surface-point density. Because the
surface-point density is optimized when setting {\tt MXSURPTS},
specification of both {\tt MXSURPTS} and {\tt SPDENS} conflicts and
results in an error message
\item {\tt RPROBE}: probe radius for defining solvent-accessible
surface. {\tt RPROBE 0.0} yields the Van der Waals surface
\end{enumerate}

The default is {\tt CONNOLLY MXSURPTS 200 RPROBE 1.0}.

NOTE: Care is required when specifying {\tt SPDENS}; the number of
surface points generated is unchecked, and a large relay-matrix to be
LU-decomposed may result.

NOTE: {\tt RPROBE} may also be relevant with {\tt SURFACE JUFFER}.

\subsubsection{SOLVENT}

The {\tt SOLVENT} directive serves to specify the solvent name
associated with the dielectric continuum. If the solvent is known,
solvent properties will be taken from the internal database. Values in
the internal database supersede any other input.
The directive is
read to the variables {\tt TEXT, SOLNAM}, using format (2A)
\begin{enumerate}
\item {\tt TEXT} should be set to the character string {\tt SOLVENT}
\item {\tt SOLNAM} is the name of the solvent associated with the
dielectric continuum (more than 8 characters are permissible). An
internal database storing solvent static and optic dielectric
constants, molecular masses, and densities is accessed through this
directive. Specification of an unknown solvent will result in
discontinuation of the calculation, except when {\tt SOLNAM X} is specified
\end{enumerate}

The default is {\tt SOLVENT X}.

\subsubsection{DIELTYP}

The {\tt DIELTYP} directive serves to specify the types of dielectric
response taken into account. Static and/or optic dielectric response may
be taken into account. If the estimate of the dispersion energy is
requested (see {\tt GAMDRF} directive), the appropriate direct
reaction field coupling of the electrons is only to the optic part of
the dielectric response, if the optic response is requested. 
The directive is
read to the variables {\tt TEXT, DIELOPT1 DIELOPT2}, using format (3A)
\begin{enumerate}
\item {\tt TEXT} should be set to the character string {\tt DIELTYP}
\item {\tt DIELOPT1} and {\tt DIELOPT2} are the types of response
requested. Either may be set to {\tt STAT} or {\tt OPT}, or left empty, requesting
static and/or optic response, and absence of response, respectively
\end{enumerate}

The default is {\tt DIELTYP STAT}.

\subsubsection{DIELOUT}

The {\tt DIELOUT} directive serves to specify the level of output
associated with the surface defining the dielectric boundary.
The directive is
read to the variables {\tt TEXT, OUTDIEL}, using format (2A)
\begin{enumerate}
\item {\tt TEXT} should be set to the character string {\tt DIELOUT}
\item {\tt OUTDIEL} is the level of output requested. The actual
output depends on the type of surface requested. Valid options
are:
\begin{enumerate}
\item {\tt STANDARD}: Standard output
\item {\tt SOME}: 
\item {\tt MODERATE}: 
\item {\tt EXTENDED}: 
\item {\tt ALL}: Maximum output
\end{enumerate}
\end{enumerate}

The default is {\tt DIELOUT STANDARD}.

\subsubsection{EPSSTAT}

The {\tt EPSSTAT} directive serves to specify the static dielectric
constant associated with the dielectric continuum ({\tt DIELTYP STAT}).
The directive is
read to the variables {\tt TEXT, EPS1}, using format (A,F)
\begin{enumerate}
\item {\tt TEXT} should be set to the character string {\tt EPSSTAT}
\item {\tt EPS1} is the static dielectric constant. This should be a
number larger than 1.0 to take effect. Values smaller than 1.0 are
ignored; a value of 1.0 effectively means the dielectric is a vacuum
and is therefore not active. (A finite ionic strength (see
subdirective {\tt KAPPAS}) may still be
specified, though.)
\end{enumerate}

The default is {\tt EPSSTAT 1.0}.

\subsubsection{EPSOPT}

The {\tt EPSOPT} directive serves to specify the optic dielectric
constant associated with the dielectric continuum ({\tt DIELTYP OPT}).
The directive is
read to the variables {\tt TEXT, EPS2}, using format (A,F)
\begin{enumerate}
\item {\tt TEXT} should be set to the character string {\tt EPSOPT}
\item {\tt EPS2} is the optic dielectric constant. This should be a
number larger than 1.0 to take effect. Values smaller than 1.0 are
ignored; a value of 1.0 effectively means the dielectric is a vacuum
and is therefore not active. (A finite ionic strength (see
subdirective {\tt KAPPAO}) may still be
specified, though.)
\end{enumerate}

The default is {\tt EPSOPT 1.0}.

\subsubsection{KAPPAS}

The {\tt KAPPAS} directive serves to specify the ionic strength
associated with the static dielectric
constant associated with the dielectric continuum ({\tt DIELTYP STAT}).
The directive is
read to the variables {\tt TEXT, KAPPAS}, using format (A,F)
\begin{enumerate}
\item {\tt TEXT} should be set to the character string {\tt KAPPAS}
\item {\tt KAPPAS} is the ionic strength associated with the static dielectric constant. This should be a
number larger than 0.0 to take effect. Values smaller than 0.0 are
ignored.
\end{enumerate}

The default is {\tt KAPPAS 0.0}.


\subsubsection{KAPPAO}

The {\tt KAPPAO} directive serves to specify the ionic strength
associated with the optic dielectric
constant associated with the dielectric continuum ({\tt DIELTYP OPT}).
The directive is
read to the variables {\tt TEXT, KAPPAO}, using format (A,F)
\begin{enumerate}
\item {\tt TEXT} should be set to the character string {\tt KAPPAO}
\item {\tt KAPPAO} is the ionic strength associated with the optic dielectric constant. This should be a
number larger than 0.0 to take effect. Values smaller than 0.0 are
ignored.
\end{enumerate}

The default is {\tt KAPPAO 0.0}.


\subsection{CLASCLAS}

The {\tt CLASCLAS} directive serves to specify the types of
interaction between classical subsystems to be calculated. The
directive consists of a single data line read to the variables {\tt
TEXT, n(INTOPT)}, using format (nA)
\begin{enumerate}
\item {\tt TEXT} should be set to the character string {\tt CLASCLAS}
\item {\tt INTOPT} may be either blank or one or more of the keywords
{\tt ELST}, {\tt NOELST}, {\tt REPN}, {\tt NOREPN}, {\tt DISP}, or {\tt NODISP}, to switch calculation
of the electrostatic, repulsion, and dispersion interactions between
the classical fragments on or off, respectively.
\end{enumerate}

The default is {\tt CLASCLAS ELST DISP REPN}.


\subsection{HBOND}

The {\tt HBOND} directive serves to specify the use of the special
CHARMM repulsion parameters for H-bonds.  The
directive consists of a single data line, starting with {\tt HBOND} which may contain a number
of keywords followed by their values. The {\tt HBOND} keywords are:
\begin{enumerate}
\item {\tt ON} or {\tt OFF} to switch special H-bond repulsion on or
off, respectively
\item {\tt RADI} to set the H-atom radius used in CHARMM-expression when within
H-bond distance
\item {\tt DIST} to set the distance within which H-bond radius is used
\end{enumerate}

The default is {\tt HBOND OFF}
Specifying {\tt HBOND ON} is equivalent to {\tt HBOND ON RADI 1.0 DIST 4.5}.


\subsection{DRFTWOEL}
The {\tt DRFTWOEL} directive serves to specify the treatment of the
2-electron reaction-field integrals.
The directive consists of a single data line read to the variables {\tt
TEXT, INTOPT}, using format (2A)
\begin{enumerate}
\item {\tt TEXT} should be set to the character string {\tt DRFTWOEL}
\item {\tt INTOPT} may be either blank or one of the keywords:
\begin{enumerate}
\item {\tt DIRECT}: 2-electron RF integrals are calculated on the fly
in each SCF cycle
\item {\tt DISK}: 2-electron RF integrals are added to the vacuum
2-electron integrals stored on disk
\end{enumerate}
\end{enumerate}

The default is {\tt DRFTWOEL DIRECT}

NOTE: This keyword only takes effect when the QM density is
treated self-consistently with the reaction field.

NOTE: {\tt DRFTWOEL DISK} is currently valid only 
in the following combination:{\tt SUPER OFF NOSYM} and {\tt GAMDRF 1.0}.


\section{DRF Output---Analysis of (D)RF Energies}
\label{drfoutput}

The (D)RF module produces additional output at the end of the normal
output. This is a full analysis of the (D)RF contributions to the
energy. The output is structured as follows:

\begin{enumerate}
\item energies associated with the quantum system (T: electron kinetic
energy operator; V$_{0}$: vacuum one-electron energy operator;
g$_{0}$: vacuum two-electron energy operator):
\begin{enumerate}
\item vacuum nuclear repulsion
\item one-electron energy: expectation value of T + V$_{0}$, using the final
density
\item electron kinetic energy: expectation value of T, using the final
density
\item nuclear attraction energy: expectation value of V$_{0}$, using the final
density
\item two-electron energy: expectation value of g$_{0}$, using the final
density
\item scf energy: converged total energy of the QM system. This may
include contributions from static and reaction fields if the QM density is
treated self-consistently with these fields
\item screening of the nuclear repulsion energy: interaction of all
nuclei with the RF induced by all nuclei
\item screening of the nuclear attraction energy: interaction of all
nuclei with the RF induced by all electrons plus interaction of all
electrons with the RF induced by all nuclei
\item screening of the two-electron energy: interaction of all
electrons with the RF induced by all electrons
\item screening of the electronic self energy: sum of interaction of 
each electron with the RF induced by itself (only with the DIRECT RF
option, see also the {\tt GAMDRF} directive)
\item molecular reaction field stabilisation: interaction of the QM
system with the RF induced by the QM system
\item equilibrium molecular polarisation energy: energy cost to induce
dipole moments and surface polarisation by the QM system
\end{enumerate}
\item energies associated with the classical system, collected per
analysis group:
\begin{enumerate}
\item vacuum classical electrostatic interaction: interaction between
point charges in the classical (sub)system(s)
\item vacuum dispersion energy estimate: polarizability-based dispersion interaction between
atoms in the classical (sub)system(s) (see also {\tt CLASDISP} directive)
\item vacuum repulsion energy estimate: CHARMM repulsion interaction between
atoms in the classical (sub)system(s) (see also the {\tt HBOND} directive)
\item vacuum classical interaction energy: sum of the first three
contributions 
\item screening of classical electrostatic energy: interaction of all
charges in analysis group A with the RF induced by all charges in
analysis group B, and {\em vice versa}
\item equilibrium classical polarisation energy: energy cost to induce
dipole moments and surface polarisation by the charges in the analysis
group (only when A=B)
\end{enumerate}
\item energies associated with the interaction between QM and classical system, collected per
analysis group:
\begin{enumerate}
\item nuclei/ external charge interaction: electrostatic interaction
between all QM nuclei (includes any {\tt BQ} centres with a charge)
and all charges in the analysis group
\item electrons/ external charge interaction: electrostatic interaction
between all electrons and all charges in the analysis group
\item screening of the ext. charge/nuclear int.n: interaction of all
nuclei (including {\tt BQ} charges) with the RF induced by all charges in the
analysis group  
\item screening of the ext. charge/electron int.n: interaction of all
electrons with the RF induced by all charges in the
analysis group
\item screening of the nuclei/ ext. charge int.n: interaction of all
charges in analysis group with the RF induced by all QM nuclei
(including {\tt BQ} charges)
\item screening of the electron/ext. charge int.n: interaction of all
charges in analysis group with the RF induced by all electrons
\item electrostatic qm-classical interaction: sum of first 2 terms
\item screening electrostatic qm-classical int.n: sum of last 4 terms
\item model repulsion energy: CHARMM model repulsion energy between QM
system and all atoms in analysis group
\end{enumerate}
\item total energies associated with the interaction between QM and classical system
\begin{enumerate}
\item as for individual analysis groups
\item estimate of the dispersion energy: the DRF estimate for
QM-classical dispersion, only in the Direct RF option (see section
\ref{estdisp} and the {\tt GAMDRF} directive)
\end{enumerate}
\item summary of the contributions:
\begin{enumerate}
\item total energy of quantum system: expectation value of the vacuum
hamiltonian (including nuclear repulsion) with the final density
\item total energy of classical system
\item quant. mech. /classical interaction energy: total interaction
energy of QM system with surroundings
\item equilibrium polarisation energy: total cost for inducing dipole
moments and surface polarization 
\item configuration total energy: energy of the QM + classical system
as a whole
\end{enumerate}
\end{enumerate}


\vspace{0.15in}

\clearpage
\begin{thebibliography}{10}

\bibitem{vries} 
A.H. de Vries, P.Th. van Duijnen, A.H. Juffer, J.A.C. Rullmann,
J.P. Dijkman, H. Merenga, and B.T. Thole,
  J. Comp. Chem. {\bf 16} (1995) 37--55, \doi{10.1002/jcc.540160105}
  erratum 1445--1446, \doi{10.1002/jcc.540161113}.

\bibitem{duijnen} 
P.Th. van Duijnen and A.H. de Vries,
  Int. J. Quant. Chem., {\bf 60} (1996) 1111,
  \doi{10.1002/(SICI)1097-461X(1996)60:6<1111::AID-QUA2>3.0.CO;2-2};

\bibitem{foster}
J.M. Foster and S.F. Boys,
  Rev. Mod. Phys. {\bf 32} (1960) 300, \doi{10.1103/RevModPhys.32.300}.

\bibitem{pipek}
J. Pipek and P.G. Mezey,
  J. Chem. Phys. {\bf 90} (1989) 4916, \doi{10.1063/1.456588}.

\bibitem{stone} 
A.J. Stone,
  Chem. Phys. Lett. 83 (1983) 233, \doi{10.1016/0009-2614(81)85452-8}.

\bibitem{dijkman} 
J.P. Dijkman and P.Th. van Duijnen, Int. J. Quant. Chem.,
Quant. Biol. Symp. {\bf 18} (1991) 49, \doi{10.1002/qua.560400710}.

\bibitem{charmm} 
B.R. Brooks, R.E. Bruccoleri, B.D. Olafson, D.J. States, S.J. Swaminathan,
and M. Karplus, J. Comp. Chem. {\bf 4} (1983) 187,
\doi{10.1002/jcc.540040211}.

\bibitem{thole1} 
B.T. Thole, Chem. Phys. {\bf 59} (1981) 341, \doi{10.1016/0301-0104(81)85176-2}.

\bibitem{thole2} 
B.T. Thole and P.Th. van Duijnen,
  Theor. Chim. Acta, {\bf 55} (1980) 307, \doi{10.1007/BF00549429}.

\bibitem{thole3} 
B.T. Thole and P.Th. van Duijnen,
  Chem. Phys., {\bf 71} (1982) 211, \doi{10.1016/0301-0104(82)87020-1}.

\bibitem{connolly} 
M.L. Connolly,
  Science, {\bf 221} (1983) 709, \doi{10.1126/science.6879170}.

\end{thebibliography}

\end{document}
