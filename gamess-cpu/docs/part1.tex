\documentclass[11pt]{article}

\usepackage{hyperref}

% package HTML requires Latex2HTML to be installed for html.sty
\usepackage{html}
\newcommand{\doi}[1]{doi:\href{http://dx.doi.org/#1}{#1}}
\newcommand{\http}[1]{\href{#1}{#1}}
\newcommand{\mailto}[1]{\href{mailto:#1}{#1}}
\begin{htmlonly}
\renewcommand{\href}[2]{\htmladdnormallink{#2}{#1}}
\end{htmlonly}
\hypersetup{colorlinks,
            %citecolor=black,
            %filecolor=black,
            %linkcolor=black,
            %urlcolor=black,
            bookmarksopen=true,
            pdftex}
 
\addtolength{\textwidth}{1.0in}
\addtolength{\oddsidemargin}{-0.5in}
\addtolength{\topmargin}{-0.5in}
\addtolength{\textheight}{1.0in}

\pagestyle{headings}
\begin{document}
\sf
\parindent 0cm
\parskip 1ex
\begin{flushleft}
Computing for Science (CFS) Ltd.,\\CCLRC Daresbury Laboratory.\\[0.30in]
{\large Generalised Atomic and Molecular Electronic Structure System }\\[.2in]
\rule{150mm}{3mm}\\
\vspace{.2in}
{\huge G~A~M~E~S~S~-~U~K}\\[.3in]
{\huge USER'S GUIDE~~and}\\[.2in]
{\huge REFERENCE MANUAL}\\[0.2in]
{\huge Version 8.0~~~June 2008}\\ [.2in]
{\large PART 1. INTRODUCTION and GENERAL FEATURES}\\
\vspace{.1in}
{\large M.F. Guest, J. Kendrick, J.H. van Lenthe, P. Sherwood and
  J.M.H. Thomas}\\[0.2in]

Copyright (c) 1993-2008 Computing for Science Ltd.\\[.1in]
This document may be freely reproduced provided that it is reproduced\\
unaltered and in its entirety.\\
\vspace{.2in}
\rule{150mm}{3mm}\\
\end{flushleft}

\pagenumbering{roman}

\section*{Introduction to GAMESS--UK}

GAMESS--UK is a general purpose {\em ab initio} molecular electronic
structure program for performing SCF-, DFT- and MCSCF-gradient
calculations, together with a variety of techniques for post Hartree
Fock calculations.

The program is derived from the original GAMESS code, obtained from
Michel Dupuis in 1981 (then at the National Resource for Computational
Chemistry, NRCC), and has been extensively modified and enhanced over
the past decade. This work has included contributions from numerous
authors (see below), and has been conducted largely at the former EPSRC
(Engineering and Physical Sciences Research Council), and now CCLRC
Daresbury Laboratory, under the auspices of Collaborative Computational
Project No. 1 (CCP1).  Other major sources that have assisted in the
on-going development and support of the program include various
academic funding agencies in the Netherlands, and ICI plc.

All publications resulting from use of this program  must include the
following acknowledgement:\\

{\em GAMESS-UK is a package of ab initio programs. See:
  \http{http://www.cfs.dl.ac.uk/gamess-uk/index.shtm}, M.F. Guest,
  I. J. Bush, H.J.J. van Dam, P. Sherwood, J.M.H. Thomas, J.H. van
  Lenthe, R.W.A Havenith, J. Kendrick, ''The GAMESS-UK electronic
  structure package: algorithms, developments and applications'',
  Molecular Physics, Vol. 103, No. 6-8, 20 March-20 April 2005,
  719-747, \doi{10.1080/00268970512331340592}.}\\

Depending on which modules are used, the following references
should be cited:\\
\begin{itemize}
\item {\sc MCSCF/CASSCF:}\\
P.J. Knowles, G.J. Sexton and N.C. Handy, Chem. Phys. 72 (1982) 337,\doi{10.1016/0301-0104(82)85131-8}:\\
P.J. Knowles and H.-J. Werner, Chem. Phys. Lett. 115 (1985) 259, \doi{10.1016/0009-2614(85)80025-7}.

\item {\sc M{\o}ller-Plesset Perturbation theory (MP2, MP3):}\\
J.E. Rice, R.D.Amos, N.C. Handy, T.J. Lee and H.F. Schaefer,
J. Chem. Phys. 85 (1986) 963, \doi{10.1063/1.451253}:\\
J.D. Watts and M. Dupuis, J. Comput. Chem. 9 (1988) 158, \doi{10.1002/jcc.540090208}.

\item {\sc Full CI (FCI):}\\
S. Zarrabian and R.J. Harrison,  Chem. Phys. Letts. 81 (1989), \doi{10.1016/0009-2614(89)87358-0 }

\item {\sc Direct-CI (DCI):}\\
V.R. Saunders and J.H. van Lenthe, Mol. Phys. 48 (1983) 923, \doi{10.1080/00268978300100661}.

\item {\sc Conventional CI (MRDCI):}\\
R.J. Buenker in `Studies in Physical and Theoretical Chemistry',
21 (1982) 17.

\item {\sc Multiple Active Space SCF (MASSCF):}\\
J.Ivanic, J.Chem.Phys. {\bf 119}, 9364-9376(2003), \doi{10.1063/1.1615954}.

\item {\sc Distributed Multipole Analysis (DMA):}\\
A.J. Stone, Chem. Phys. Letters 83 (1981) 233, \doi{10.1016/0009-2614(81)85452-8}.

\item {\sc Outer-valence Greens' Function (OVGF):}\\
L.S. Cederbaum and W. Domcke, Adv. Chem. Phys.  36 (1977) 205.

\item {\sc Tamm-Dancoff Method (TDA):}\\
J. Schirmer and L.S. Cederbaum, J. Phys. B11 (1978) 1889:\\
I.H. Hillier, K. Rigby, M. Vincent, M.F. Guest and W. von Niessen,
Chem. Phys. Lett. 134 (1987) 403, \doi{10.1016/0009-2614(87)87162-2}.

\item {\sc Effective Core Potentials (Local ECPs):}\\
P.J. Hay and W.R. Wadt, J. Chem. Phys. 82 (1985) 270, 284, 299, \doi{10.1063/1.448800}.

\item {\sc Effective Core Potentials (Non-local ECPs):}\\
Ph. Durand and J.-C.Berthelat, Theoret. Chim. Acta, 38 (1975) 283,
\doi{10.1007/BF00963468}:\\
N.A. Burton, I.H. Hillier, M.F. Guest and J. Kendrick,
Chem. Phys. Lett. 155 (1989) 195, \doi{10.1016/0009-2614(89)85348-5]}.

\item {\sc Infra-red and Raman Intensities:}\\
R.D. Amos, Adv. Chem. Phys. 67 (1987) 99, \doi{10.1002/9780470142936.ch2}.

\item {\sc Multi configuration Linear Response (MCLR):}\\
C. Fuchs, V. Bona\v{c}i\'{c}-Kouteck\'{y} and J. Kouteck\'{y},
J.  Chem.  Phys.  98 (1993) 3121, \doi{10.1063/1.464086}.

\item {\sc Coupled Cluster CCSD and CCSD(T):}\\
T.J. Lee, J.E. Rice and A.P. Rendell,
The TITAN Set of Electronic Structure Programs, 1991,

\item {\sc Density Functional Theory (DFT):}\\
The initial DFT module within GAMESS-UK was developed by Dr. P. Young
under the auspices of EPSRC's Collaborative Computational Project No. 1
(CCP1) (1995-1997). Subsequent developments have been undertaken by
staff at the Daresbury Laboratory.

\item {\sc Direct Reaction Field (DRF):}\\
A.H. de Vries, P.Th. van Duijnen, A.H. Juffer, J.A.C. Rullmann,
J.P. Dijkman, H. Merenga, and B.T. Thole,
  J. Comp. Chem. {\bf 16} (1995) 37--55, \doi{10.1002/jcc.540160105}
  erratum 1445--1446, \doi{10.1002/jcc.540161113};
P.Th. van Duijnen and A.H. de Vries,
  Int. J. Quant. Chem., {\bf 60} (1996) 1111,
  \doi{10.1002/(SICI)1097-461X(1996)60:6<1111::AID-QUA2>3.0.CO;2-2};

\end{itemize}
Additional key references and background material on each of the
modules are provided at appropriate points throughout this
manual.
\newpage
\tableofcontents

\pagenumbering{arabic}


\section{About This Manual}
This manual is divided into sixteen parts or `chapters', which may be
broadly classified as follows:

\begin{itemize}
\item {\em Part 1 (this part)} describes the general features of the program,
detailing the file system employed and the directive structure
associated with the free-format data input.
\item {\em Part 2} presents a series of example data files designed to
  provide a broad outline of the various capabilities of the
  program. Subsequent chapters provide a more detailed description of the data
 input formats. 
\item {\em Parts 3 and 4} describe the 
directives required in performing SCF, M\o ller Plesset, CASSCF, MCSCF,
DFT and associated gradient calculations.
\item {\em Part 5} deals with integral transformation and Direct-CI
calculations. 
\item {\em Part 6} covers conventional and semi-direct Table-CI calculations.
\item {\em Part 7} describes the data required in performing
RPA and MCLR calculations.
\item {\em Part 8} describes the various analysis capabilities of the code,
plus the directives and scope of the DRF solvation module.
\item {\em Part 9} presents a brief summary of possible problem areas in
running the code.
\item {\em Part 10}  provides a brief summary of the
utility functions associated with GAMESS--UK.
\item {\em Part 11} describes the way formatted output may be 
obtained from GAMESS--UK for use by other programs. 
\item {\em Part 12} presents a sequence of worked examples on a
UNIX machine, using the specially provided script {\em rungamess}
\item {\em Part 13} provides a series of worked examples invoking the
  GAMESS-UK binary directory from a UNIX shell (C-shell).
\item {\em Part 14} describes the capabilities and running of the
parallel version of GAMESS--UK on both MPP and SMP hardware.
\item {\em Turtle} describes how to run the GAMESS-UK Valence Bond module.
\item {\em Mopac} describes the GAMESS-UK/MOPAC interface.
\end{itemize}

The user who wishes to acquire rapid familiarity and a sufficient
subset of knowledge to use the program, albeit perhaps in a somewhat
restricted fashion, should concentrate initially on Parts 1 and 2,
together with any appropriate examples from Parts 12-14.

The example input files described in Part 2 are supplied with
GAMESS-UK and can can be found in the scripts labelled c2001\_a -
c2035 in the directory: {\bf GAMESS-UK/examples/chap2}.

NB: this documentation describes features available in the current release of
the code (Version 8.0); some of the features described herein may not
function as described in earlier releases of the program.


\section{General Features}

GAMESS--UK \cite{gamess} is a general purpose {\em ab initio} molecular
electronic structure program for performing SCF-, DFT- and MCSCF-gradient
calculations, together with a variety of techniques for post Hartree
Fock calculations. On-going development of the code is co-ordinated from
Daresbury, with the program currently available on a wide range of
machines, including:
\begin{itemize}
\item a wide variety of UNIX workstations, including those from SGI
(R4000-, R4400-, R8k- and R10k-Indigo$^2$, R5k-, R10k- and R12k-O2,
Challenge and Power Challenge, Origin200, Origin2000, Origin300 and
Origin3800), Hewlett Packard  (Model 9000/7xx,  PA8000, PA8200,
PA8500, PA8600 and PA8700), IBM RS/6000s (PowerPC, Power1, Power2,
P2SC, Power3, Power4 and Power5), Compaq/Digital (EV5-, EV6-, EV67- and
EV68-AXP under Tru64, OSF/1, OpenVMS and Linux), and SUN (Super-,
Hyper- and Ultra-SPARC); 
\item a number of IA-32, IA-64 and AMD PCs (e.g. Pentium, PentiumII,
Pentium III, Pentium 4, Xeon, EM64T, Itanium1 and Itanium2, Athlon and Opteron) under
Linux, FreeBSD and Windows NT, 2000 and XP;
\item Mac OSX running on Power 3, 4 and 5 processors.
\item Parallel versions of the code are available on both MPP and SMP
platforms. MPP versions include those for the IBM p-series (TN2, P2SC, WH2
and Regatta nodes), and a variety of IA32, IA64, EM64T, AMD64 and Alpha-based Beowulf
Systems. SMP versions are available for the SGI Origin and Altix,
Compaq AlphaServer SC and Cray XD1.
Ports to the Fujitsu VPP300 and NEC SX-5 are also available.
\item Versions for other machines of mainly historical interest include
the Intel iPSC/860, Amdahl VP, IBM-3090/VF, Convex~C2 and C3 series,
and Kendall Square KSR-2, Cray~Y-MP, C90 and J90, T3D and T3E
(Cray-T3E/900 and Cray/T3E-1200), VAX VMS and Hitachi SR2201.

\end{itemize}

The program utilises the Rys Polynomial or Rotation techniques
\cite{dupuis} to evaluate repulsion integrals over s,p,d,f and g type
Cartesian Gaussian orbitals. Open- and closed-shell SCF treatments are
available within both the RHF and UHF framework, with convergence
controls provided through a hybrid scheme of level shifters and the
DIIS method \cite{pulay}.  In addition generalised valence bond
\cite{bobrowicz}, complete active space SCF \cite{roos} and more
general MCSCF \cite{knowles}, MASSCF \cite{ormas} and M\o ller Plesset
(MP2 and MP3) \cite{handy} calculations may be performed. The original
limitation to cartesian basis sets is lifted through the provision of
spherical harmonic basis sets for all options within the
programme. The most recent additions to the code, available from
Version 6.0 onwards, include a full-featured Density Functional Theory
(DFT) module, with access to a wide variety of functionals (S-VWN,
B-LYP, B-P86, B3-LYP, HCTH, B97, B97-1, B97-2, FT97 etc). This module
is subject to on-going developments under the auspices of CCP1.

The analytic energy gradient is available for each class of
wavefunction above. Gradients for s and p Gaussians are evaluated using
the algorithm due to Schlegel \cite{schlegel}, while gradients involving d, f
and g Gaussians utilise the Rys Polynomial Method.

Geometry optimisation may be carried out using the DL-FIND module that
is included into GAMESS-UK (see Part 4 for information about DL-FIND).

Alternatively, GAMESS-UK has it's own optimisation capabilities using
a quasi-Newton rank-2 update method, while transition state location
is available through either a synchronous transit \cite{bell}, trust
region \cite{cerjan} or `hill-walking' \cite{simons,baker} method.
Force constants may be evaluated analytically \cite{pople} and by
numerical differentiation; coupled Hartree--Fock (CHF) calculations
provide for a range of molecular properties, including
polarisabilities and molecular hyperpolarisabilities \cite{stevens}
and, through the calculation of dipole moment and polarisability
derivatives, the computation of infra--red and Raman intensities
\cite{amos}.

{\em Ab initio} core potentials are provided in both semi-local
\cite{hay} and non-local \cite{durand} formalism for valence-only
molecular orbital treatments.

Conventional CI treatments using the table-driven selection algorithms
within the framework of MR-DCI calculations \cite{buenker} have been
extended to include an implementation of the semi-direct table-driven
MRDCI module. This provides more extensive capabilities in the
treatment of electronic spectra and related phenomena. The variety of
correlation treatments available include Direct-CI \cite{siegbahn},
Full-CI \cite{zarrabian}, and both CCSD \cite{lee,scuseria} and CCSD(T)
\cite{lee2,rendell} coupled-cluster calculations, the latter being
limited to closed-shell systems \cite{lee3}.  A size-consistent variant
of Multi-Reference MP2 theory, popularized in its CASPT2 form by Roos
et al., is also available~\cite{vanDam}. With no restriction to CAS
wavefunctions, the module also provides MR-MP3 capabilities.

The direct calculation of molecular valence ionization energies may be
performed though Green's function techniques, using either the
outer-valence Green's function (OVGF) \cite{cederbaum} or the
two-particle-hole Tamm-Dancoff method  (2ph-TDA) \cite{schirmer}.

Calculations of electronic transition energies and corresponding
oscillator strengths may also be performed using either the Random
Phase Approximation (RPA) method or the Multiconfigurational Linear
Response (MCLR) procedure \cite{fuchs}. The RPA calculations may be
performed either within the conventional approach where the
two--electron integrals are transformed or with a ``direct''
implementation.

A variety of wavefunction analysis methods are provided, including
population analysis, Natural Bond Orbital (NBO)~\cite{weinhold}
and distributed multipole analysis \cite{stone}, localised orbitals,
graphical analysis and calculation of 1-electron properties. An
interface to the AIMPAC code of Bader is provided \cite{aimpac}
for SCF, DFT, MCSCF, MP2, MP3, MRDCI and Direct-CI calculations.

The treatment of solvation effects is now based on the DRF (Direct
Reaction Field) model \cite{vries,duijnen}, an embedding technique
enabling the computation of the interaction between a quantum-mechanically
described molecule and its classically described surroundings.  The
classical surroundings may be modelled in a number of ways, (i) by point
charges to model the electrostatic field due to the surroundings (ii) by
polarizabilities to model the (electronic) response of the surroundings
(iii) by an enveloping dielectric to model bulk response (both static
and electronic) of the surroundings, and (iv) by an enveloping ionic
solution, characterized by its Debye screening length.

The treatment of relativistic effects is provided within the ZORA
scheme (Zeroth Order Regular Approximation~\cite{zora}), a two component
alternative to the full 4 component Dirac equation.  While much cheaper
than the latter, ZORA recovers  a large part of the relativistic effects.
The scalar (1-component) form is now available, with the full 2-component
implementation (including spin-orbit coupling) in progress.  The current
implementation will allow all usual Ab Initio (and  DFT) methods to be
performed, incorporating the major relativistic effects.

The restriction of a maximum of 255 basis functions when performing
conventional processing of two-electron integrals has been removed in
nearly all modules in Version 7.0 of the code.  The program structure is
open-ended in both direct-SCF, -DFT and direct-MP2 mode, so that
direct-SCF calculations up to 22,000 basis functions have been
performed.  Version 7.0 also features a significant extension to the
parallel capabilities of GAMESS--UK.

In addition to the functionality outlined above, the program provides
for (i) Hybrid QM/MM calculations, through an interface to the CHARMM
QM/MM code, developed in collaboration with B. Brooks and E. Billings,
of the NIH, and (ii) a Utilities module concerned with house-keeping,
library file creation, file manipulation etc.

The program also includes version 7.0 of the MOPAC \cite{mopac}
semi-empirical code, enabling the calculations with the semi-empirical
Hamiltonians MNDO, MINDO/3, AM1, PM3, PM5, and MNDO-d. The interface
between GAMESS-UK and MOPAC allows coordinates calculated with MOPAC
to be imported into GAMESS-UK for a subsequent {\em ab initio}
calculation.

A free Graphical User Interface (the CCP1GUI) has also been developed
to simplify the use of the code and to graphically display the results
of calculations. Further information on the CCP1GUI, as well as
downloadable packages for a number of different operating systems can
be found on the web at:

\http{http://www.cse.scitech.ac.uk/ccg/software/ccp1gui/}\\

Areas of application for GAMESS--UK include:

\begin{itemize}
\item Theoretical studies of reaction surfaces. 
SCF and MCSCF calculations
of equilibrium and transition state geometries, 
and the evaluation of force
constants and vibrational frequencies at these stationary points.
\item  Energy assignments  in the entire field of electronic
spectroscopy.
Calculation of spectroscopic properties of highly excited molecular
and ionic states. Evaluation of transition moments.
\item  Graphical and numerical analysis of 
Hartree Fock and post-Hartree Fock wavefunctions.
\item Generation of zero-order wavefunctions required in the more
extensive treatments of dynamical correlation energy.
Applicability to general systems with many electrons in this
treatment of dynamical correlation energy.
Benchmark treatments of correlation energy using Full-CI calculations.
\item  Theoretical mechanistic studies of chemisorption in heterogeneous
catalysis.
\item Treatment of bio--organic and related molecules
through Direct-SCF calculations and electrostatic potential
analysis.
\end{itemize}

\section{Units used in the manual}

Throughout the manual, use is made of the units of ''words'' and
''blocks'' to describe the sizes of files and/or memory. These are
defined as follows:

\subsection{WORD}
A word is defined as a FORTRAN DOUBLE PRECISION or INTEGER*8
datatype and is 8 bytes (64 bits) long.

\subsection{BLOCK}
The definition of BLOCK is derived from the ATMOL \cite{atmol} file system, where a
BLOCK is 512 words (4096 bytes) long.

\section{The File System}

In this section we provide a brief overview of the file system used by
GAMESS--UK. Note at this stage that both direct access and FORTRAN
sequential data sets may be used by the program, the exact usage being
a function of the particular activity in hand. Manipulation of these
files e.g., keeping, deleting etc is described in the
''pre-directives'' section of Part 3 of this manual and also under
the {\em rungamess} script in Part 13.

\subsection{Direct Access Data Sets}

Extensive use is made of the  system of disk and tape handling
originally developed within the ATMOL \cite{atmol} program system, with the
various files of integral data output to one or more direct access data
sets. Such data sets are allocated to the program using logical file
names (LFNs) in the range ED0, ED1, ..., ED19 and MT0, MT1, ..., MT19.
In Table~\ref{table:1} we list those files used by the program, and
define the default LFN associated with each file.  A full description
of data set management will not be given here:  for the present it
suffices to note the following;

\begin{enumerate}
\item  The Dumpfile, which in default is routed to ED3, is crucial to
the program and controls all restart activities, This file is organised
into variable length sections, with the user nominating various
sections for data storage e.g., eigenvectors. The sections are
characterised by integers (in the range 1 - 350), which are specified
by the user through data input.

\item  The Scratchfile, which  is routed to ED7, is used in all runs of
the program as a file for house-keeping activities. While this data set
need never be maintained across separate runs of the program, the user
should be aware of the role of ED7 and avoid any attempts to use the
data set in, for example, the re-routing of integral output.

\item  Each data file is block addressable (1 block = 512 words). In
default output of each file commences at block 1 of the associated
data set, and continues in uninterrupted fashion until output is
complete. With certain exceptions, the user may control
file usage in several ways;
\begin{itemize}
\item  redefine the LFN to be associated with a given data file
\item  redefine the starting block for the file, allowing several
files of data to be stored on the same data set
\item  partition output of a given file over several data sets, specifying
the associated LFNs, and the starting and terminal blocks of each LFN. This
feature is particularly useful in controlling the output of two-electron
integrals. It is not possible, however, to fragment the Dumpfile,
Scratchfile or Direct-CI file.

\begin{table}
 \caption{\label{table:1}\  Direct Access Files and Default LFNs used by GAMESS--UK}
 
 \begin{centering}
 \begin{tabular}{llr}
\\ \hline\hline
  File         &     Contents &     Default LFN\\ \cline{1-3}
\\
    Mainfile     &     Two-electron integrals  &              ED2\\
    Dumpfile     &     Eigenvectors, restart information & ED3\\
                 &     Hessians, CI-coefficients etc  & \\
    Scratchfile   &    Temporary file for program   &         ED7\\
                &            housekeeping & \\
    Libraryfile &   Pseudopotential Library File & ED0\\
    Symmetry Adapted   &  Two-electron integrals in a symmetry & ED1\\
    Mainfile           &     adapted basis & \\
    Secondary Mainfile  &    Partially transformed two-electron & ED4\\
       &      integrals  & \\
    Direct-CI file &  Direct-CI file for restarts   & ED5\\
       &      and symbolic hamiltonian etc  & \\
    Transformed Integral &   Transformed two-electron integrals & ED6\\
    File    & & \\
    Overflow file &  Direct-CI file for    & ED8\\
       &      symbolic hamiltonian   & \\
    Loop Formulae Tape  &    Symbolic formulae used in CASSCF &  ED9\\
    Reordered Formulae &  Reordered symbolic formulae  &   ED10\\
    Tape & & \\
    Density Matrix File & Two particle density matrix  & ED11\\
                        & used in CASSCF geometry optimisation & \\
    Hamiltonian File & Derivative Fock Operators   & ED12\\
    Reordered Integral &   Reordered two-electron integrals & ED13\\
    File    & used in MCSCF & \\
    Scratchfiles &   Temporary files used in  & ED16--ED19\\
            & derivative integral calculations & \\
\\
    Sortfile   &    Temporary file for sorting   &        SORT\\
                &            integrals & \\
    P-Sortfile   &    Temporary file for sorting molecular  &       PSORT\\
                &      integrals & \\
    Tablefile   &    Table-CI file for storing data-base   &       TABLE\\
                &      of pattern matrix elements & \\
\\
\hline\hline
 \end{tabular}
 
 \end{centering}
\end{table}


\item  while mainly of historical interest, we note that 
direct access files needed, on an IBM Mainframe (running the MVS or
VM/CMS operating system), to be preallocated. This
process assumed the user had some idea of the overall space requirements,
since any attempt to route file output beyond the terminal block of
a preallocated file was viewed as an unrecoverable error
condition. Preallocation was not required when using the
more-restrictive sequential  data sets, which may be accessed though
the LFNs MT0-MT19.
Note that these considerations refer only to GAMESS--UK usage 
on an IBM mainframe.  On other  machines no formal distinction
is made between direct (ED0-ED19) and sequential (MT0-MT19) files.

\end{itemize}
\end{enumerate}

\subsection{FORTRAN Data Sets}

In addition to the direct access data sets described above GAMESS--UK
will, in certain phases of computation e.g., Table-CI, make use of the
more conventional FORTRAN unformatted sequential data sets. In the
majority of cases this usage is restricted to scratch activities, and
the user need not be concerned with, for example, FORTRAN unit
specification. In some instances, however, it may be necessary to
`keep' the files in question between jobs. This activity should be
controlled through the appropriate FILE pre-directives (see Part 3 of
this manual).

\section{Data Input - The Directive Structure}

Data input for GAMESS--UK consists of a sequence of data lines, where
a GAMESS--UK data line consists of a number of data fields, the fields being
separated by at least one space character. The separating spaces are not
considered part of the data field.

\subsection{Data Lines}

Data lines normally end at column 72, assuming a `carriage return' is used
as the line terminator (although this can be altered
by means of the pre-directive WIDTH (see Part 3 of this manual).
In fact there are several valid terminators recognised for a data line. These
include
\begin{itemize}
\item the conventional `carriage-return' (CR).
\item the back-slash character ($\backslash$).
\item the colon character (:)
\end{itemize}
Thus the following sequence of data lines
{
\footnotesize
\begin{verbatim}
          TITLE
          H2CO - 3-21G DEFAULT BASIS - CLOSED SHELL SCF
          ZMATRIX ANGSTROM
          C
          O 1 1.203
          H 1 1.099 2 121.8
          H 1 1.099 2 121.8 3 180.0
          END
          ENTER
\end{verbatim}
}
\noindent
could be presented as:
{
\footnotesize
\begin{verbatim}
          TITLE\H2CO - 3-21G DEFAULT BASIS - CLOSED SHELL SCF
          ZMATRIX ANGSTROM\C\O 1 1.203\H 1 1.099 2 121.8
          H 1 1.099 2 121.8 3 180.0\END
          ENTER
\end{verbatim}
}
\noindent
Note also that a data line with 
one of the characters \#, $>$ or $<$ in the first column
will be treated by the program as `comment' lines, 
being merely echoed to the output file. It goes without saying
that the $\backslash$ character should be avoided in any TITLE
data.

\subsection{Data Fields}

A data field consists of a succession of non-space characters on a 
single line.  Three types of field are in use:
\begin{itemize}
\item {\em A-fields} are used for the introduction of textual material, 
in the form of character strings.
The maximum length of an A-field is 8 characters. A-fields of 
less than 8
characters are padded to the right with spaces, while if more than 8
characters are found, the field is truncated from the right, so that
only the leftmost 8 characters are significant.

\item {\em I-fields} are used for the introduction of numeric data in the
form of integers, and consist of a succession of decimal digits possibly
preceded by a + or - sign. The sign
character may be omitted, when the integer is assumed to be positive.
The following are examples of valid I-fields:

{
\footnotesize
\begin{verbatim}
           0
           762
           +4033
           -6149
\end{verbatim}
}
\item {\em F-fields} are used for the input of floating point numbers. 
They are
as I-fields except that a `decimal point' character may appear to
the right of the sign character. To introduce a decimal exponent a
concatenation of a F-field, followed by the character E or D, followed
by an I-field, without intervening blanks is allowed. The following
are equivalent F-fields:

{
\footnotesize
\begin{verbatim}
           0      .0      0.      0.0      +0.0
\end{verbatim}
}
as are:

{
\footnotesize
\begin{verbatim}
           0.0003      3E-4      3D-4      0.3E-3     +30E-5
\end{verbatim}
}
and:

{
\footnotesize
\begin{verbatim}
           -500.0       -500      -5E2      -5E+2      -5D+2
\end{verbatim}
}
\end{itemize}

Format descriptors are used to describe a data line. The simplest of
these takes the form nA, or nI or nF, where n is a decimal integer.
Thus the format nF describes a line consisting of n successive
F-fields. The omission of the integer (n) before a letter code
implies that n=1. More complicated descriptors are used; for example
(iA, jI, kF) describes a line consisting of i A-fields, followed by
j I-fields, followed by k F-fields. Consider the following line:

{
\footnotesize
\begin{verbatim}
                ATOM 10 1.0 12 HCH
\end{verbatim}
}

read to the variables TEXT, I, X, J, BTEXT in format (A, I, F, I, A).

\begin{itemize}
\item      TEXT    will be set to the character string ATOM.
\item      I       will be set to the decimal integer 10.
\item      X       will be set to the floating point value 1.0.
\item      J       will be set to the decimal integer 12.
\item      BTEXT   will be set to the character string HCH.
\end{itemize}
The same line could be read to variables TEXT, X, Y, ATEXT, BTEXT in format
(A, 2F, 2A), so that:
\begin{itemize}
\item      TEXT    will be set to the character string ATOM.
\item      X       will be set to the floating point value 10.0.
\item      Y       will be set to the floating point value 1.0.
\item      ATEXT   will be set to the character string 12.
\item      BTEXT   will be set to the character string HCH.
\end{itemize}
Consider reading the following line:
{
\footnotesize
\begin{verbatim}
         GENERALISED ATOMIC AND MOLECULAR ELECTRONIC STRUCTURE SYSTEM
\end{verbatim}
}
to (A(I),I=1,7) in format (7A).
\begin{itemize}
\item      A(1)    will be set to the string GENERALI. Note the truncation.
\item      A(2)    will be set to the string ATOMIC.
\item      A(3)    will be set to the string AND.
\item      A(4)    will be set to the string MOLECULA. Note the truncation.
\item      A(5)    will be set to the string ELECTRON. Note the truncation.
\item      A(6)    will be set to the string STRUCTUR. Note the truncation.
\item      A(7)    will be set to the string SYSTEM.
\end{itemize}

\subsection{GAMESS-UK Directives}

Data input is largely structured as a sequence of `directives'.
Directives sometimes extend over many lines, sometimes they comprise a
single line, and usually contain some reasonably self-contained data,
built up out of A,I, and/or F-fields.  The first field of a directive
is invariably an A-field, carrying the name of the directive.  Note
that only the first four characters of such an identifier are in
practice required.  The directives group into two classes, with {\em
Class 1} directives (see Part 3) specified before {\em Class 2}
directives (see Part 4); note that the order within each class may be
important, and the user is recommended to follow the order outlined in
the relevant sections of the manuals, or example inputs where
available.

\subsubsection{INCLUDE - Including external files in an input file}
As of version 7.0 it is possible to include external text files into a
GAMESS-UK input file through the use of the 'include' directive. Thus
the line:

{
\footnotesize
\begin{verbatim}
include /home/wab/inputs/geometry.txt
\end{verbatim}
}

placed in a GAMESS-UK input file would cause the ''geometry.txt'' file to
be inserted at that point in the input file.

\section{Restrictions}

The restrictions which the present program place on the user are
as follows
\begin{itemize}
\item Gaussian functions of s, p, d, f and g may be used.
\item The maximum number of contracted orbitals depends in general
on the functionality requested. Conventional SCF and DFT energy and gradient
calculations are now effectively open-ended (the default being 1024
functions). Note that lifting this restriction to an arbitrary number
is trivially performed by changing the build size when configuring the code.
\item The maximum number of primitive functions within a 
contraction is 25.
\item The maximum number of centres is 1000, allowing for the use
of extensive point-charge arrays.
\end{itemize}

\section{Release Notes for GAMESS-UK}

\subsection{Changes introduced in GAMESS-UK Version 7.0}

\subsubsection{Changes to GAMESS-UK functionality}

\begin{enumerate}

\item Fermi-dirac smearing implemented

Normally orbitals are filled according to a step function, but this
can cause convergence problems if it is uncertain which state to
converge on. Fermi smearing allows orbitals to be partially filled,
which can improve convergence in some problematic cases.

\item Atomic Guess extended

It is now possible to specify the per-atom electron configuration or
charge for the atomic guess. It is also possible to specify alpha and
beta populations for atoms in UHF calculations.

\item Update of the Graphics Module

The graphics modules have been revised to enable both density and
potential codes to drive through full (s,p,d,f,g) basis sets.

\item User-defined convergence schemes:

It is now possible for a user to specify a particular convergence
scheme, rather than the default one within GAMESS-UK. This is
activated by placing the convergence criteria within a block in the
input file delimited by the keywords "newscf" and "end", following the
specification of the runtype.

This functionality is available in serial and within the parallel MPI
ScaLAPACK driver.

\item Changes to the DFT Module

\begin{itemize}

\item New DFT Functionals
Several new functionals have been added:

\begin{itemize}
\item PBE exchange-correlation functional
\item HCTH120, HCTH147, HCTH407, KT1 and KT2 functionals.
\item PW92 local correlation functional
\item PW91 exchange, correlation and exchange-correlation functionals.
\item B95 meta-GGA correlation functional.
\item BB95, B1B95 and BB1K meta-GGA functionals.
NB - for the meta-GGA functionals, only the energies and gradients can
be calculated - the second derivatives of the energy are not
available.
\end{itemize}

\item More flexible DFT grid specifications

The atom size is used in a number of tests within in the DFT
code. Previously the same size was used for all tests, however it is
now possible to specify the atomic size for each test separately, as
shown in the examples below:

\begin{enumerate}
\item The atomic radii for the angular grid pruning schemes: pradii 3.0

\item The atomic size adjustment radii for the weighting scheme: wradii 3.0

\item  The screening radius: sradii 3.0

\item The radial grid scale factor: radii 3.0

\end{enumerate}

It is also now possible to specify the grid size for a row of the
periodic table - previously this had to be done on a per-element
basis. For example, to specify the grid size for the Lebedev-Laikov
angular grid for the first row of the periodic table:

lebedev row 1 194

\end{itemize}


\item Changes to the Post-HF Modules

\begin{itemize}
\item The 255 basis function limit in the MRDCI code has been removed,
  so that the code is effecively open-ended in the number of basis
  functions permitted.

\item Configuration Interaction Transformation between AO and MO
  basis' has been extended to g functions.


\item A facility to allow the punching of both transformed integrals
  and CI coefficients from the full CI code has been introduced. This
  uses (i) the cards directive, and (ii) an extension of the current
  print facility within the Full CI module. Thus the data file:

{
\footnotesize
\begin{verbatim}
    core 10000000
    title
    h2co - 3-21g basis - valence full-ci
    super off nosym
    cards trans fullci
    zmatrix angstrom
    c
    o 1 1.203
    h 1 1.099 2 121.8
    h 1 1.099 2 121.8 3 180.0
    end
    active\5 to 22 end\core\1 to 4\end
    runtype ci\fullci 18 4 4
    punch 1 -8
    enter
\end{verbatim}
}

    Would result in a complete list of transformed 1e- and 2e-integrals to
    the file "moints,ascii" and all ci coefficients (greater than 1 *
    10**-8) to the file civecs.ascii.


\item The filenames for the files use in CI calculations can now be
  set through "file" directives in the input file as well as
  environment variables.

\item MR-ACPF (Multireference Averaged Coupled-Pair-Functional), MR-AQCC
(Multi-Reference Average Quadratic Coupled-Cluster) and the CEPA0
coupled electron-pair approximation methods have been added to the
MRDCI module.

\end{itemize}


\item The use of the ''HARMONIC'' directive has been extended to the Valence
Bond code, allowing spherical harmonic basis sets to be used for d, f
and g functions.

\item A simple solvation model has been implemented in the Valence Bond code.

\item Arbitrary (not spinadapted) wavefunctions are now possible in the
Valence Bond code.

\item The Valence Bond module can be compiled with 8-byte integers.

\item The ''RESTORE'' option has been added to the ZORA module to allow zora
relativistic (atomic) corrections to be stored and restored on the
current and foreign dumpfile. This permits easy restart calculations,
when the ZORA corrections do not need to be recalculated.

%Joops additions
%- improved orbital optimisation (improved pert) and stop criterium
%(still to be tested and to be part of thesis) This could make orbital
%optimisation on bigger molecules cheap,since the same number of
%matrix-elements are required as are needed in a Fock-matrix type
%optimisation.
%- basis set optimisation is to optimise outer functions for e.g. negative ions
%(is not finished yet)
%- Freeze atom is to not calculate hessians and gradiens for frozen atoms
%(in testphase)
%- Spin-orbit ZORA, will be finished in 3 months (I hope)
%To do relativistic calculations inclusing spin-orbit coupling 
%brillouin theorem in parallell? VB!


\item New distributed data MPI HF/DFT driver

A distributed-data HF and DFT module has been developed using
MPI-based tools such as ScaLAPACK.  All data structures, except those
required for the Fock build are fully distributed. The functionality
of this code is currently limited to closed shell and unrestricted
open shell. To build this code BLACS and ScaLAPACK (available from
\http{http://www.netlib.org/}), must be installed on the target machine and
the "mpi" build option selected when configuring the code. 

\item Taskfarming Harness

A taskfarming harness has been developed. This is an MPI program
designed to be run on a large number of processors on a parallel
machine and 'batch processes' numerous small GAMESS-UK jobs. The
taskfarming harness is currently only available as a separate binary
of the "mpi" build of the code and is selected by choosing the
"taskfarm" keyword when configuring the mpi code.

\end{enumerate}

\subsubsection{Changes to the structure of the code}

\begin{enumerate}

\item New Ports

The code has been ported to several new platforms:
\begin{itemize}
\item Macintosh OSX (G3, G4, and G5 processors)
\item Windows XP
\item AMD Opteron and Athlon processors running Linux.
\item Intel Xeon, EM64T and Itanium processors running Linux.
\item HP-UX running on Itanium processors.
\item Sunfire v880 server
\item SGI Altix
\item Cray XD1
\end{itemize}

\item Global Arrays

The version of the Global Arrays supplied with GAMESS-UK has been
updated from 3.3 to GA 3.4b.

\item MOPAC

The MOPAC code within GAMESS-UK has been updated to version 7.0

\item Configuring GAMESS-UK

A new configure process has been developed to ease porting the code
to new platforms and making it easier for users to configure the
build on their own machine. All platform-specific variables for the make
process are stored in a file with an .mk suffix in the
{\footnotesize\textbf{GAMESS-UK/config}} directory and the configuration process is run
by the configure script in the main {\footnotesize\textbf{GAMESS-UK}} directory. There are 
further notes in the file: {\footnotesize\textbf{GAMESS-UK/INSTALL}}.

\end{enumerate}

\subsubsection{Peripheral Changes}

\begin{enumerate}
\item Demo Binaries

Free Windows, Macintosh and Linux demo versions are available for
users who wish to try out the code.


\item Changes to the website

The website has been updated and moved to a new url and a Bugzilla facility for
logging and querying bugs with the code has also been added.

The new URL for the main GAMESS-UK website is:

\http{http://www.cfs.dl.ac.uk}

\item CCP1GUI

The CCP1GUI is a free, extensible Graphical User Interface to various
computational chemistry codes. Although it has interfaces to other
codes such as Dalton and Mopac, the CCP1GUI has been developed around
GAMESS-UK and provides a powerful tool for setting up and viewing the
results of calculations with GAMESS-UK.

The CCP1GUI website can be found at:

\http{http://www.cse.scitech.ac.uk/ccg/software/ccp1gui}


\end{enumerate}

\subsection{Changes introduced in GAMESS-UK Version 6.3}

\begin{enumerate}

\item Section numbers for vectors/enter can now be omitted.

\item DMA analysis now uses Slater radii table.

\item The old option of performing a DFT energy calculation on a HF
density (a posteriori) is no longer in the default build. This can still
be accessed using the old-dft keyword in the configure response at
build time.

\item Changes to the DFT module:
\begin{itemize}
\item More flexible grid input.
\item Additional DFT functionals (FT97, HCTH, B97, B97-1 and B97-2 etc.)
\item Optimisations to coulomb fit DFT code.
\item Built-in DFT Orbital sets (DZVP, DZVP2 and TZVP) and 
Auxiliary Coulomb Fitting Basis sets (DGauss-A1 and -A2, DeMon
and those due to Ahlrichs).
\item Gradient term from derivative of quadrature weights included.
\end{itemize}

\item Reduced functionality build options (SCF+DFT and MP2) for use on
parallel machines with limited node memory, and for benchmark porting
exercises.

\item Major changes to all ECP-related code. with re-writes of
the integrals and derivatives code. This has sped the energy integrals
by between 2-3, and the gradients by a factor of at least 5.
Addition of 4 sets of ECPs to internal libraries
\begin{itemize} 
\item  LANL2 (including all inner-valence TM ECPs etc to library;
the original HW ECP's are now code-named lanl
\item  CRENBS and CRENBL (Christiansen et al Small- and Large-core ECPs)
\item  STRLC and STRSC (Stuttgart RLC and RSC sets)
\end{itemize} 
all ECPs + associated basis sets added for all elements.

\item New Ports 
\begin{itemize}
\item Linux PowerPC 
\item Windows95
\item Parallel version for Alpha/Linux clusters.
\item 64-bit build supported on a range of platforms (Origin, Alpha Linux,
AIX, Solaris)
\end{itemize}

\item Global Array and Peigs source for parallel builds has been updated,
supporting a wider range of platforms.

\item Ability to read in hessian information under runtype optx.

\item Ability to use include directive in input files.

\item Extensions/enhancements to ZORA.

\item Alpha release of Multi-Reference Moller Plesset module.

\item NBO analysis extended to UHF wavefunctions. Interface provided
to Bader's AIMPAC code.

\item Alpha release of newscf module allowing more flexible treatment
of cases showing poor SCF convergence.

\item Reduction in memory usage for systems with many nuclear centres,
e.g. for QM/MM studies.

\item Developments to support VB code (under development, not included
in Release 6.3.1)

\item Work on CHARMM interface code.
\end{enumerate}

\subsection{Changes introduced in GAMESS-UK Version 6.2}

\begin{enumerate}
\item New functionality available includes;

\begin{itemize}
\item Alpha release of Direct Reaction Field module.

\item Gaussian functions of s, p, d, f and g may now be used

\item Alpha release of ZORA Relativistic treatment.

\item Production release of Density Functional Theory (DFT)
     module.

\item Alpha release of semi-direct Table-CI module.
\end{itemize}

\item Change to the default behaviour of the DFT module

   Users of alpha-releases of the DFT module should note that now
\begin{itemize}
\item the default functional is now B-LYP
\item the default grids are now constructed from the logarithmic
      grid of Knowles and Mura (previously the SG1 grid was selected
      by default).
\end{itemize}

\item Point charge calculations using BQ centres - the nuclear energy
   now excludes the term arising from BQ-BQ interactions. This should
   simplify the construction of QM/MM models by allowing the classical
   electrostatic energy to be computed entirely within the MM code. The
   original behaviour may be restored with the BQBQ directive.
\end{enumerate}

\section{Enquiries}

All enquiries regarding the availability of the program
documented herein, should be directed in the first instance to
the following E-Mail address at Daresbury:
{
\footnotesize
\mailto{gamess\_uk\_contact@dl.ac.uk}
}

For any problems arising from use of the program, there are two
sources of help.

\begin{itemize}
\item There is a bug tracker that lists known problems with the code,
which can be found at:

{
\footnotesize
\http{http://ccpforge.cse.rl.ac.uk/tracker/?group\_id=14}
}

\item The gamess-uk-users email list is the place to raise any queries
  about the code or to ask for advice on using the code. The homepage
  for the list can be found at:

{
\footnotesize
\http{http://ccpforge.cse.rl.ac.uk/mailman/listinfo/gamess-uk-users}
}

\end{itemize}
                

Note that an electronic copy of this manual is available
under the CFS web pages:
{
\footnotesize
\http{http://www.cfs.ac.uk/docs/index.shtml}
}

\section{Acknowledgement}

This program resulted from the efforts of many researchers: we
gratefully acknowledge the on-going collaborative efforts of those
colleagues detailed in the preface to this manual.

% \clearpage

\begin{thebibliography}{10}
\bibitem{gamess} 
M.F. Guest, I.J. Bush, H.J.J. van Dam, P. Sherwood, J.M.H. Thomas,
J.H. van Lenthe, R.W.A Havenith, J. Kendrick, 
"The GAMESS-UK electronic structure package: algorithms, developments
 and applications", Molecular Physics, {\bf 103} (2005) 719-747,
\doi{10.1080/00268970512331340592}.
M.F. Guest, J. Kendrick, GAMESS Users Manual, SERC Daresbury Laboratory,
CCP1/86/1, 1986: M. Dupuis, D. Spangler and J. Wendoloski, NRCC
Software Catalog, Vol. 1, Program No. QG01 (GAMESS), 1980:
M.F. Guest, R.J. Harrison, J.H. van Lenthe and L.C.H. van Corler, 
Theor. Chim. Acta {\bf 71} (1987) 117, \doi{10.1007/BF00526413}; 
Note: The CDC version of the GAMESS code was obtained
from M. Dupuis in June 1981. Serious development work on the program
commenced in January 1982, after conversion to VAX and IBM systems.
All sections of the original code have either been extensively modified
or replaced in the present GAMESS--UK code.

\bibitem{dupuis} 
M. Dupuis, J. Rys and H.F. King,
  J. Chem. Phys. {\bf 65} (1976) 111, \doi{10.1063/1.432807};
J.A. Pople and W.J. Hehre,
  J. Comp. Phys. {\bf 27} (1978) 161, \doi{10.1016/0021-9991(78)90001-3}.

\bibitem{pulay} 
P. Pulay,
  Chem. Phys. Lett. {\bf 73} (1980) 393, \doi{10.1016/0009-2614(80)80396-4};
P. Pulay,
  J. Comp. Chem. {\bf 3} (1982) 556, \doi{10.1002/jcc.540030413}.

\bibitem{bobrowicz} 
F.W. Bobrowicz and W.A. Goddard, in `Modern Theoretical Chemistry',
Vol. 3, ed. H.F. Schaefer, Plenum, New York (1977) 79.

\bibitem{roos}
B. Jonsson, B.O. Roos, P.R. Taylor and P.E.M. Siegbahn,
  J. Chem. Phys.  {\bf 74} (1981) 4566, \doi{10.1063/1.441645};
B.O. Roos, P. Linse, P.E.M. Siegbahn and M.R.A. Blomberg,
  Chem. Phys. {\bf 66} (1982) 197, \doi{10.1016/0301-0104(82)88019-1};
P.J. Knowles, G.J. Sexton and N.C. Handy,
  Chem. Phys.  {\bf 72} (1982) 337, \doi{10.1016/0301-0104(82)85131-8}:
  The original CASSCF module, as developed by Dr. P.J. Knowles was incorporated
  into GAMESS in April 1983.

\bibitem{knowles}  
P.J. Knowles and H.J. Werner,
  Chem. Phys. Lett.  {\bf 115} (1985) 259, \doi{10.1016/0009-2614(85)80025-7}.

\bibitem{handy}
N.C. Handy and H.F. Schaefer,
  J. Chem. Phys. {\bf 81} (1984) 5031, \doi{10.1063/1.447489};
J.E. Rice, R.D.Amos, N.C. Handy, T.J. Lee and H.F. Schaefer,
  J. Chem. Phys. {\bf 85} (1986) 963, \doi{10.1063/1.451253}.

\bibitem{ormas}
"Direct configuration interaction and multiconfigurational self-consistent 
field method for multiple active spaces with variable occupations. 
I. Method", J. Ivanic, J. Chem. Phys. {\bf 119}, 9364-9376(2003),
\doi{10.1063/1.1615954}.

\bibitem{schlegel} 
H.B. Schlegel, J. Chem. Phys. {\bf 77} (1982) 3676, \doi{10.1063/1.444270}.

\bibitem{bell} 
S. Bell and J.S. Crighton,
  J. Chem. Phys. {\bf 80} (1984) 2464, \doi{10.1063/1.446996}.

\bibitem{cerjan} 
C.J. Cerjan and W.H. Miller,
  J. Chem. Phys. {\bf 75} (1981) 2800, \doi{10.1063/1.442352}.

\bibitem{simons}  
J. Simons, P. Jorgensen, H. Taylor and J. Ozment,
  J. Phys. Chem. {\bf 87} (1983) 2745, \doi{10.1021/j100238a013}:
A. Banerjee, N. Adams, J. Simons and R. Shepard,
  J. Phys. Chem. {\bf 89} (1985) 52, \doi{10.1021/j100247a015}.

\bibitem{baker}  
J. Baker,
  J. Comp. Chem. {\bf 7} (1986) 385, \doi{10.1002/jcc.540070402}.

\bibitem{pople} 
J.A. Pople, R. Krishnan, H.B. Schlegel and J.S. Binkley,
  Int. J. Quant. Chem. {\bf 14} (1978) 545,
  \doi{10.1002/qua.560140503};
Y. Osamura, Y. Yamaguchi, P. Saxe, M.A. Vincent, J.F. Gaw and H.F. Schaefer,
  Chem. Phys. {\bf 72} (1982) 131,
  \doi{10.1016/0301-0104(82)87073-0}.

\bibitem{stevens} 
R.M. Stevens, R.M. Pitzer and W.N. Lipscomb,
  J. Chem. Phys. {\bf 38} (1963) 550, \doi{10.1063/1.1733693}.

\bibitem{amos} 
R.D. Amos,
  Adv. Chem. Phys. {\bf 67} (1987) 99.
% No doi available.

\bibitem{hay} 
P.J. Hay and W.R. Wadt, J. Chem. Phys. {\bf 82} (1985)
270 \doi{10.1063/1.448799},
284 \doi{10.1063/1.448800},
299 \doi{10.1063/1.448975}.

\bibitem{durand} 
Ph. Durand and J.-C. Berthelat,
  Theoret. Chim. Acta, {\bf 38} (1975) 283, \doi{10.1007/BF00963468}.

\bibitem{buenker} 
R.J. Buenker in `Proc. of the Workshop on Quantum Chemistry and
Molecular Physics', Wollongong, Australia (1980);
R.J. Buenker in `Studies in Physical and Theoretical Chemistry',
{\bf 21} (1982) 17.
% This material is not electronically available.

\bibitem{siegbahn} 
P.E.M. Siegbahn,
  J. Chem. Phys. {\bf 72} (1980) 1647, \doi{10.1063/1.439365};
V.R. Saunders and J.H. van Lenthe,
  Mol. Phys. {\bf 48} (1983) 923, \doi{10.1080/00268978300100661}.

\bibitem{zarrabian}  
S. Zarrabian and R.J. Harrison,
  Chem. Phys. Lett. {\bf 81} (1989) 393--398,
  \doi{10.1016/0009-2614(89)87358-0}.

\bibitem{lee} 
T.J. Lee and J.E. Rice,
  Chem. Phys. Lett. {\bf 150} (1988) 406--415,
  \doi{10.1016/0009-2614(88)80427-5}.

\bibitem{scuseria} 
G.E. Scuseria, A.C. Scheiner, T.J. Lee, J.E. Rice and H.F. Schaefer,
  J. Chem. Phys. {\bf 86} (1987) 2881, \doi{10.1063/1.452039}.

\bibitem{lee2} 
T.J. Lee, A.P. Rendell and P.R. Taylor,
  J. Phys. Chem.{\bf 94} (1990) 5463, \doi{10.1021/j100377a008}.

\bibitem{rendell} 
A.P. Rendell, T.J. Lee and A. Komornicki,
  Chem. Phys. Lett, {\bf 178} (1991) 462--470,
  \doi{10.1016/0009-2614(91)87003-T}.
 
\bibitem{lee3} 
T.J. Lee, J.E. Rice and A.P. Rendell,
The TITAN Set of Electronic Structure Programs, 1991.
% No more information electronically accessible.

\bibitem{cederbaum}
L.S. Cederbaum and W. Domcke,
  Adv. Chem. Phys.  {\bf 36} (1977) 205.
% Not electronically accessible.

\bibitem{schirmer}
J. Schirmer and L.S. Cederbaum,
  J. Phys. {\bf B11} (1978) 1889, \doi{10.1088/0022-3700/11/11/006}.

\bibitem{fuchs} 
C. Fuchs, V. Bona\v{c}i\'{c}-Kouteck\'{y} and J. Kouteck\'{y},
J.\ Chem.\ Phys.\ {\bf 98} (1993) 3121, \doi{10.1063/1.464086}.

\bibitem{weinhold}
Based upon Version 3.0 of the NBO program from the Quantum Chemistry
Program Exchange (No. 408 ,1980);
A. E. Reed, L. A. Curtiss, and F. Weinhold,
  Chem. Rev. {\bf 88} (1988) 899-926, \doi{10.1021/cr00088a005};
F. Weinhold and J. E. Carpenter, in, R. Naaman and Z. Vager (eds.),
"The Structure of Small Molecules and Ions," (Plenum, New York, 1988),
pp. 227-236.
% NBO source code is also available from http://www.ccl.net/

\bibitem{aimpac}
Interface to Version 94, Revision B of AIMPAC, 
F.W. Biegler-K\"onig, R.F.W. Bader and T-H. Tang,
  J. Comp. Chem. {\bf 3} (1982) 317--328, \doi{10.1002/jcc.540030306}.

\bibitem{stone} 
A.J. Stone, 
  Chem. Phys. Lett. {\bf 83} (1983) 233--239,
  \doi{10.1016/0009-2614(81)85452-8}.

\bibitem{vries}
A.H. de Vries, P.Th. van Duijnen, A.H. Juffer, J.A.C. Rullmann,
J.P. Dijkman, H. Merenga, and B.T. Thole,
  J. Comp. Chem. {\bf 16} (1995) 37--55, \doi{10.1002/jcc.540160105} 
  erratum 1445--1446, \doi{10.1002/jcc.540161113}.

\bibitem{duijnen}
P.Th. van Duijnen and A.H. de Vries,
  Int. J. Quant. Chem., {\bf 60} (1996) 1111,
  \doi{10.1002/(SICI)1097-461X(1996)60:6<1111::AID-QUA2>3.0.CO;2-2}.

\bibitem{vanDam}
H.J.J. van Dam, J.H. van Lenthe and P. Pulay,
  Mol. Phys. {\bf 93} (1998) 431, \doi{10.1080/002689798169122}.

\bibitem{zora}
S. Faas, J.G. Snijders, J.H. van Lenthe, E. van Lenthe, E.J.  Baerends,
  Chem. Phys. Lett. {\bf 246} (1995) 632--640,
  \doi{10.1016/0009-2614(95)01156-0}.

\bibitem{atmol}
D. Moncrieff and V.R. Saunders, ATMOL-Introduction Notes, UMRCC, May,
1986; Cyber-205 Note Number 32, UMRCC, September, 1985; V.R. Saunders and
M.F. Guest, ATMOL3 Part 9, RL-76-106 (1976); M.F. Guest and V.R. Saunders,
Mol. Phys. {\bf 28} (1974) 819, \doi{10.1080/00268977400102171}.

\bibitem{mopac}
MOPAC 1993 Version 7.00, J.J.P. Stewart, Fujitsu Limited, Tokyo, Japan.
% See also http://sourceforge.net/projects/mopac7/
\end{thebibliography}
\end{document}
