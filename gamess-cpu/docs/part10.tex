\documentclass[11pt,fleqn]{article}

\usepackage{hyperref}

% package HTML requires Latex2HTML to be installed for html.sty
\usepackage{html}
\newcommand{\doi}[1]{doi:\href{http://dx.doi.org/#1}{#1}}
\begin{htmlonly}
\renewcommand{\href}[2]{\htmladdnormallink{#2}{#1}}
\end{htmlonly}
\hypersetup{colorlinks,
            %citecolor=black,
            %filecolor=black,
            %linkcolor=black,
            %urlcolor=black,
            bookmarksopen=true,
            pdftex}
 
\addtolength{\textwidth}{1.0in}
\addtolength{\oddsidemargin}{-0.5in}
\addtolength{\topmargin}{-0.5in}
\addtolength{\textheight}{1.0in}

\pagestyle{headings}
\pagenumbering{roman}
\begin{document}
\sf
\parindent 0cm
\parskip 1ex
\begin{flushleft}
 
Computing for Science (CFS) Ltd.,\\CCLRC Daresbury Laboratory.\\[0.30in]
{\large Generalised Atomic and Molecular Electronic Structure System }\\[.2in]
\rule{150mm}{3mm}\\
\vspace{.2in}
{\huge G~A~M~E~S~S~-~U~K}\\[.3in]
{\huge USER'S GUIDE~~and}\\[.2in]
{\huge REFERENCE MANUAL}\\[0.2in]
{\huge Version 8.0~~~June 2008}\\ [.2in]
{\large PART 10. THE UTILITY FUNCTIONS}\\
\vspace{.1in}
{\large M.F. Guest, J. Kendrick, J.H. van Lenthe and P. Sherwood}\\[0.2in]
 
Copyright (c) 1993-2008 Computing for Science Ltd.\\[.1in]
This document may be freely reproduced provided that it is reproduced\\
unaltered and in its entirety.\\
\vspace{.2in}
\rule{150mm}{3mm}\\
\end{flushleft}

\tableofcontents
\newpage
% 

\pagenumbering{arabic}

\section[Introduction]{Introduction}

The Utility functions associated with the program provide facilities for
\begin{enumerate}
\item data handling commonly required
when manipulating GAMESS--UK generated datasets, and is oriented towards
manipulation of the 2-electron integral files in particular. 
The program will copy, list, check, summarise, edit etc such files. 
\item  generation of the Library file required 
when performing non-local pseudopotential calculations.
\item generation of plotting output using the grids of 
densities, potentials etc generated under previous control
of the GRAPHICS directive and assumed resident on a Dumpfile.
\item manipulation of orbitals (vectors) with extended flexibility
(SERVEC) e.g., if orthogonalisation of the 1-electron integrals
is needed. This module may be invoked as part of the Utilities module,
when an existing dumpfile is provided (or the 1-electron integrals are
not needed) and later in the input after the basis-set and geometry
information is read.  SERVEC may prove useful for Valence Bond
calculations (although not exclusively) and is only included when VB is
specified when GAMESS is configured.
\end{enumerate}
Note that these
functions are performed independently of the computation 
associated with  a specific molecular calculation, 
with the data input requirements quite separate from that described
in sections 3-6 above. The following points should be noted:

\begin{itemize}

\item  Certain of the File Manipulation  Utilities  
require input and output areas
to be simultaneously operational. When using a GAMESS--UK dataset it is
possible for input and output to be taken from the same dataset.
When an  block is written to a dataset, all blocks except that
explicitly written remain unaltered.
\item  The restrictions applicable to use the same dataset for input and
output within certain of the Utilities (specifically
COPY, EDIT and COPYDUMP) concern the possibility of overlapping
fields. The following rules are applicable:
\begin{itemize}
\item Operations involving non-overlapping data areas 
on the same dataset are always legal.
\item Operations involving overlapping fields are valid if the starting
block number of the output area is less than or equal to the
starting block number of the input area.
\item Operations involving overlapping fields are not valid when the
starting block number of the input area is less than that of the
output area, and yet the output area commences within the input area.
\end{itemize}
\item The module is a collection of routines, which may be called in any
order, as dictated by the user. Each routine will require some input
data, and will perform a task. When the work of a given routine is
complete, control is handed to the next routine nominated by the user.
On successful completion, all routines reply with the printed message:

{
\footnotesize
\begin{verbatim}
          `NAME' COMPLETED
\end{verbatim}
}
where NAME represents the task. The following paragraphs describe the
tasks and their purpose.
\end{itemize}

\section{Utility Directives}

The Utilities module is requested by a single data line
containing the data string UTILITY in the first data field, and
should immediately follow any pre-directives. Subsequent utility
functions are user-driven through the sequence of directives
outlined below.

\section{File Manipulations}

\subsection{The LIST Utility}

Data for the LIST utility consists of one dataline, read to variables
TEXT, DDLIST, IBLK, NBLK, TYPE using format (2A,2I,A).
\begin{itemize}
\item TEXT should be set to the character string LIST.
\item DDLIST should be set to the LFN of the input dataset.
\item IBLK is an integer used to specify the starting block number of
the input dataset. Instead of using an integer to specify
the IBLK parameter, the character * may be used, and in
this case listing commences from the `current position' of
the input file. The syntax * + m or * - m, where m is an
integer may be used to specify IBLK, and will be taken to
mean the current block number + or - m respectively.
\item  NBLK is an integer used to specify the number of blocks to be
listed. If NBLK is set to zero (or omitted), listing
continues until an ENDFILE block is reached.
\item TYPE should be set to the appropriate character string 
characterising the format of the 2-electron file to be LISTed - 
valid strings include 2E (for conventional integral format),
P (for P-supermatrix files) and JK (for JK-supermatrix files).
\end{itemize}
 
 A list of the 2-electron integrals, with associated indices (labelled
I, J, K and L) stored in the dataset assigned by the file DDLIST
will be printed. Listing commences from the block specified by IBLK,
and continues for NBLK blocks (or to the first ENDFILE if NBLK=0).\\

{\bf Example 1}\\

All datasets are positioned at block 1 when the service program is
first entered. If the following dataline is presented:

{
\footnotesize
\begin{verbatim}
          LIST ED6 * + 1 0 2E
\end{verbatim}
}
the file assigned to ED6 will be listed from block 2 (* + 1=2)
until the first ENDFILE block is encountered.

{\bf Example 2}
{
\footnotesize
\begin{verbatim}
          LIST MT5 98 5 P
\end{verbatim}
}
This example list the P-Supermatrix  file assigned to MT5 
from block 98 for 5
blocks. The file will be positioned at block 103 on completion of the
list.


\subsection{The PUNCH Utility}

Data for the PUNCH directive consists of one dataline read to variables
TEXT, DDPUNC, IBLK, NBLK using the format (2A,2I).
\begin{itemize}
\item  TEXT should be set to the character string PUNCH.
\item  DDPUNC specifies the LFN of the input dataset. This file
must be associated to the FILE specified in the JCL by
the pre-directives.
\item  IBLK is an integer used to specify the starting block number
of the input file. Alternatively the `*' or `* + m' or
`* - m' formats may be used, as described in the LIST
directive.
\item  NBLK is an integer used to specify the number of blocks to be
`punched'. Alternatively, if NBLK is zero (or omitted),
the file will be punched until an ENDFILE block is reached
\end{itemize}
 The 2-electron integrals, with associated indices, as stored on the
file specified by the DDPUNC starting at block IBLK for NBLK
blocks (or until an ENDFILE block is detected if NBLK=0) are punched.
Each `punched' line is produced by the Fortran code of the form:

{
\footnotesize
\begin{verbatim}
          WRITE(7,1)ISEQ,N,(I(M),J(M),K(M),L(M),G(M),M=1,N)
     1    FORMAT(I4,I2,2(4I4,F17.10)
\end{verbatim}
}
where
\begin{itemize}
\item  ISEQ is a card sequence number.
\item  N is an integer whose value may be 1 or 2.
\item  I,J,K,L,G are arrays to hold the indices and values respectively
of a maximum of two integrals.
\end{itemize}
 The PUNCH directive should be used sparingly, since the output
generated by this directive is considerable even for small NBLK numbers.\\

{\bf Example 1}
{
\footnotesize
\begin{verbatim}
          PUNCH MT1 75 5
\end{verbatim}
}
 The  file assigned to MT1 will be `punched' starting at block 75
for 5 blocks. The dataset will be positioned at block 80 on termination
of the PUNCH task.\\

{\bf Example 2}
{
\footnotesize
\begin{verbatim}
          PUNCH ED2 * 0
\end{verbatim}
}
The file assigned to ED2 will be `punched' commencing from the
current position (usually block 1 if first access) until the ENDFILE
block has been encountered.


\subsection{The SUMMARY Utility}

Data for the SUMMARY utility consists of one dataline read to variables
TEXT, DDSUM, IBLK, NBLK using format (2A,2I).
\begin{itemize}
\item TEXT should be set to the character string SUMMARY.
\item DDSUM should be set to the file of the dataset to be summarised.
\item IBLK is an integer used to specify the starting block of the
input file. Alternatively the `*' or `* + m' or `* - m'
formats may be used, see LIST routine.
\item NBLK is an integer used to specify the number of blocks to be
summarised. Alternatively, if NBLK is set to zero (or
omitted), the file will be summarised until an ENDFILE
block is reached.
\end{itemize}
 The SUMMARY routine causes one printed line per block, and each
block is assumed to be in MAINFILE format. The printing consists of the
first and last 2-electron integral in the block (with associated
indices) plus the number of integrals stored in the block.\\

{\bf Example}
{
\footnotesize
\begin{verbatim}
          SUMMARY MT4 1 0
\end{verbatim}
}
The file assigned to MT4 is summarised from block 1 until the
first ENDFILE block is encountered.

\subsection{The CHECKSUM Utility}

Data for the CHECKSUM routine consists of one dataline read to variable
TEXT, DDCHEK, IBLK, NBLK using the format (2A,2I).
\begin{itemize}
\item TEXT should be set to the character string CHECKSUM.
\item DDCHEK should be set to the LFN used to assign the dataset
to be `checksummed'.
\item IBLK is an integer used to specify the starting block number of
he input file. Alternatively the `*' or `* + m' or `* - m'
formats may be used, as described by the LIST directive.
\item NBLK is an integer used to specify the number of blocks to be
`checksummed', or if set to zero (or omitted), the dataset
will be `checksummed' until an ENDFILE block is detected.
\end{itemize}
 The CHECKSUM routine scans the nominated blocks of the stated dataset.
Blocks with an invalid Checksum words cause the message:

{
\footnotesize
\begin{verbatim}
          BLOCK x CHECKSUM ERROR
\end{verbatim}
}
to be printed, where x is the block number. ENDFILE blocks give rise to
the message:
{
\footnotesize
\begin{verbatim}
          BLOCK x ENDFILE
\end{verbatim}
}
The CHECKSUM routine may be used on any GAMESS--UK generated dataset.\\

{\bf Example}
{
\footnotesize
\begin{verbatim}
          CHECKSUM ED2 1 999
\end{verbatim}
}
The file assigned to ED2 will be `checksummed' from block 1 for
999 blocks. The dataset will be positioned at block 1000 on completion
of the operation.

\subsection{The CHECK Utility}

The CHECK routine provides a more complete check facility for the
MAINFILE than is available when using the CHECKSUM directive. Data
consists of one line read to variables TEXT, DDCHEK, IBLK, NBLK, FP, FN
using format (2A,2I,2F).
\begin{itemize}
\item  TEXT should be set to the character string CHECK.
\item  DDCHEK specifies the file name of the input dataset.
\item  IBLK is an integer used to specify the starting block of the
input dataset. Alternatively the `*' or `* + m' or
`* - m' formats may be used, as described in LIST routine.
\item  NBLK is an integer used to specify the number of blocks to be
`checked'. If NBLK zero, checking continues until the
first ENDFILE block is detected.
\item  FP is a +ve real variable whose value is used to monitor +ve
valued 2-electron integrals. If such an integral is found
to be greater than FP, its value plus associated indices
is printed.
\item  FN is a +ve real variable. If the absolute value of a -ve
value 2-electron integral is found to be greater than FN,
its value plus its associated indices is printed. If FN is
omitted, then FP may also be omitted, the default value
5.0 being chosen for the latter.
\end{itemize}
 The CHECK routine reads the nominated blocks of a given dataset,
printing messages if unusual conditions are met. The printing is of the
general form:

{
\footnotesize
\begin{verbatim}
          BLOCK x message
\end{verbatim}
}
where x is the block number, and `message' denotes explanatory text.
The text messages may be one of the following:
\vspace{0.15in}

\begin{centering}
\begin {tabular}{ll}
Message    &    Explanation \\  \hline\hline
CHECKSUM ERROR &   A block containing an invalid checksum word\\
             &        has been detected. \\

NOT MAINFILE &     Indicates that whilst the block is in standard \\
             &         format, it is not in MAINFILE format. \\
ENDFILE      &     An ENDFILE block has been detected. \\

INVALID NUMBER OF & The 2-electron integral counter word has been \\
INTEGRALS    &   set -ve, or to a value greater than 340. \\

INVALID FP   &     A 2-electron integral has been detected which \\
             &        is not in a normalised floating point \\
             &        representation. \\

G,I,J,K,L=   &     The characters f,i,j,k,l correspond to the \\
f,i,j,k,l    &     value and associated indices respectively of a \\
             &     2-electron integral. A +ve integral whose value \\
             &     is greater than FP, or a -ve integral whose \\
             &     absolute value is greater than FN has been \\
             &     detected. \\  \hline\hline
\end{tabular}
 
\end{centering}
\vspace{0.15in}
{\bf Example}
{
\footnotesize
\begin{verbatim}
          CHECK MT5 * 0 8 .3
\end{verbatim}
}
The file associated with MT5 will be checked from the current
position to the first ENDFILE block. Positive integrals greater than 8
or -ve integrals whose absolute values are greater than .3 will be
flagged.


\subsection{The SCAN Utility}

Data for the SCAN directive consists of one line, read to variables
TEXT, DDSCAN, IBLK using format (2A,2I).
\begin{itemize}
\item TEXT should be set to the character string SCAN.
\item DDSCAN specifies the file name of the dataset to be
`scanned'.
\item IBLK is an integer used to specify the starting block for the
scan operation. The formats `*' or `* + m' or
`* - m' may also be used, as described in the LIST directive.
\end{itemize}
The SCAN routine will read the nominated dataset from the block
specified by the IBLK parameter, until the first ENDFILE block is
detected. The dataset will be positioned immediately after the ENDFILE
block on termination of the operation. The SCAN routine will read
blocks containing invalid checksum words without diagnosing an error.\\

{\bf Example}\\

A dataset has been assigned to a file MT4, and is to be
positioned immediately after the third ENDFILE block. The following
data should be presented:

{
\footnotesize
\begin{verbatim}
          SCAN MT4 1
          SCAN MT4 *
          SCAN MT4 *
\end{verbatim}
}

\subsection{The COPY Utility}

Data for the COPY routine consists of one line, read to variables
TEXT, DDIN, IBLK, DDOUT, JBLK, NBLK using format (2A,I,A,2I).
\begin{itemize}
\item TEXT should be set to the character string COPY.
\item DDIN specifies the file name of the input dataset.
\item IBLK is an integer used to specify the starting block of the
input dataset. The `*' or `* + m' or
`* - m' formats also may be used, as described in the LIST
directive.
\item DDOUT specifies the file name of the output dataset.
\item JBLK is an integer used to specify the starting block of the
output dataset. The `*' or `* + m' or
`* - m' formats may also be used, as described in the LIST
directive.
\item NBLK is an integer specifying the number of blocks to be copied.
If the value zero is entered, the copy is terminated
when an ENDFILE block is detected on the input file. In
this case the ENDFILE block is copied to the output file.
\end{itemize}
{\bf Example 1}
{
\footnotesize
\begin{verbatim}
          COPY ED2 11 MT2 * 0
\end{verbatim}
}
The dataset assigned to the file ED2 will be copied from block 11
until the first ENDFILE block is detected, the copied blocks being
routed to a dataset assigned to the file MT2 starting at the
current position.\\

{\bf Example 2}
{
\footnotesize
\begin{verbatim}
          COPY ED4 500 ED4 499 10
\end{verbatim}
}
The above example illustrates a valid `intra' dataset copy with
overlapping input and output areas. The net effect is to move the
information one block down the dataset.\\

{\bf Example 3}
{
\footnotesize
\begin{verbatim}
          COPY ED4 500 ED4 501 10
\end{verbatim}
}
The above example illustrates an invalid `intra' dataset copy. An error
will be diagnosed when block 501 is read, since it will have been
overwritten by the copied block 500.


\subsection{The EDIT Utility}

The EDIT utility is used to copy, in a selective manner, MAINFILE
datasets. The selectivity arises because integrals whose absolute value
are less than an input threshold are not copied to the output dataset.
The first line of data is read to variables TEXT, DDOUT, JBLK, IACC using
format (2A,2I).
\begin{itemize}
\item  TEXT should be set to the character string EDIT.
\item  DDOUT specifies the file name of the output dataset.
\item  JBLK is an integer specifying the starting block of the output
dataset. Alternatively the formats `*' or `* + m' or
`* - m' may be used, as described in the LIST directive.
\item  IACC the quantity 10**(-IACC) is calculated, and used as the
threshold.
\end{itemize}
 Subsequent datalines define the input files, each line being read to
variables DDIN, IBLK, LBLK using format (A,2I). An `edited' copy of the
dataset specified by DDIN starting at block IBLK and continuing up to
but not including LBLK will be routed for output. The number of blocks
scanned will be equal to LBLK-IBLK. Note that `*' or `* + m' or `* - m'
formats may be used to specify IBLK, and that if LBLK is set to zero,
editing will continue until the first ENDFILE block is detected on the
input dataset. Note that if an ENDFILE block is detected before LBLK is
reached, input processing will cease, so that block LBLK should be
regarded as the highest block number reachable. Further `input file
definition lines' may be presented, the output generated by each line
being appended to the already generated EDIT output. The data sequence
is terminated by a line whose first data field contains the character
string END. The routine will then append an ENDFILE block to the output
dataset.\\

{\bf Example}
{
\footnotesize
\begin{verbatim}
          EDIT MT4 * 60
          ED2 1 0
          ED3 1 0
          END
\end{verbatim}
}
The MAINFILE held in two sections starting at block 1 of ED2 and ED3 is
edited to an file assigned to MT4, where it will be held in the
form of one section terminated by an ENDFILE block. The editing
threshold of 10**(-60) ensures that all integrals are transcribed.\\

{\bf Example}
{
\footnotesize
\begin{verbatim}
          EDIT ED2 1 7
          ED3 1 0
          END
\end{verbatim}
}
The dataset assigned to ED3 will be edited to the dataset assigned as
ED2, both files commencing at block 1, the threshold being 10$^{-7}$.
Editing will terminate when an ENDFILE block is detected on ED3.
The EDIT routine is here being used to reduce the size of the MAINFILE,
at the cost of the loss of some accuracy, thereby reducing subsequent
SCF iteration times. It is suggested that an editing threshold of
10$^{-7}$ is used, SCF convergence to tolerance of 10$^{-4}$ is all
that is reasonable.

\subsection{The ENDFILE Utility}

Data for the ENDFILE directive consists of a single line read to
variables TEXT, DDOUT, JBLK using format (2A,I).
\begin{itemize}
\item TEXT should be set to the character string ENDFILE.
\item DDOUT specifies the LFN assigned to the output
dataset.
\item JBLK is an integer used to specify the block number where an
ENDFILE block is to be written. Alternatively, the `*' or
`* + m' or `* - m' formats may be used, as described in
LIST directive.
\end{itemize}

{\bf Example}
{
\footnotesize
\begin{verbatim}
          ENDFILE MT0 *
\end{verbatim}
}

An ENDFILE block is written to a dataset assigned to MT0, at the
current position.


\subsection{The COPYDUMP Utility}

Data for the COPYDUMP directive consists of a single dataline read to
variables TEXT, DDIN, IBLK, DDOUT, JBLK using format (2A,I,A,I).
\begin{itemize}
\item  TEXT should be set to the character string COPYDUMP.
\item  DDIN specifies the file name of the input dataset.
\item  IBLK is an integer used to specify the starting block of the
Dumpfile resident on the input dataset. Alternatively,
the formats `*' or `* + m' or `* - m' may be used, as
outlined in the LIST directive.
\item  DDOUT specifies the LFN of the output dataset.
\item  JBLK is an integer used to specify the starting block for the
output dataset. As before formats `*' or `* + m' or
`* - m' may be employed.
\end{itemize}
The Dumpfile resident on the dataset nominated by the DDIN parameter,
and starting at block IBLK is copied to the dataset nominated by DDOUT
starting at block JBLK.\\

{\bf Example}
{
\footnotesize
\begin{verbatim}
          COPYDUMP ED2 1 MT2 *
\end{verbatim}
}
The Dumpfile starting at block 1 of ED2 is copied to MT2 starting at
the current position.


\subsection{The FINDDUMP Utility}

Data for  FINDDUMP  consists of one line, read to variables
TEXT, DDIN, IBLK, NBLK, TEST using format (2A,2I,A).
\begin{itemize}
\item TEXT should be set to the character string FINDDUMP.
\item DDIN specifies the file name for the input dataset.
\item IBLK is an integer used to specify the starting block number of
the input file. Alternatively, `*' or `* + m' or `* - m'
may be used, as described in the LIST directive.
\item NBLK is an integer used to specify the number of blocks to be
read. NBLK may not be set to zero.
\item TEST should be set to one of the character strings HIGH or LOW,
if omitted the default is LOW.
\end{itemize}
The routine scans over the nominated blocks, seeking out evidence of
valid Dumpfiles. Upon detection of a valid Dumpfile
the program will
print a summary of the Dumpfile, the scale of the printing being
controlled by the setting of variable TEST. If TEST=HIGH, a fairly
complete listing of the Dumpfile will be issued, if TEST=LOW, only a
low level of information will be printed. The FINDDUMP facility will
skip blocks containing invalid checksum words.\\

{\bf Example}
{
\footnotesize
\begin{verbatim}
          FINDDUMP ED3 1 600 HIGH
\end{verbatim}
}
will cause the first 600 blocks of the dataset assigned to ED3 to be
scanned for Dumpfiles. If Dumpfiles are detected, a complete listing
will be printed.

\subsection{The STOP Directive}

Data for the STOP directive consists of one dataline with either the
character string STOP or EXIT in the first datafield. All  data
sets will be closed, and execution ended. A `STOP' dataline must be
presented last in the datastream.


\section{SERVEC - Vector Service module}

This module was originally written  to provide vector (orbital) 
manipulation capabilities to the ATMOL package. Sometimes also in GAMESS
more flexible user-controlled orbital manipulations are required,
e.g. for the VB package. Since SERVEC shares many routines with VB, 
it is included in that set of programs. 
The package works exclusively in a non-symmetry adapted basis
and can write vector-sections of arbitrary dimensions. These are
not recognised by the normal GAMESS routines. If however the keyword
'gamess' is used on the write directive or the number of vectors equals 
their dimension a normal gamess vector-section
is written, which may be back symmetry adapted using a 
vectors getq in GAMESS.
The order of directives is very important for the result and 
an auxiliary dumpfile to store intermediate vectors may be required.

It is invoked by specifying the keyword SERVEC (all 6 characters to distinguish
it from SERV), either at the beginning of the input (like SERV), when no 1-electron
integrals are available, or after the basis and geometry specification, in which case
the one-electron integrals are available. The restart-like situation is created, allowing
one to startup from the vectors section generated by servec, without a preceding gamess-job
This package has always been a ad-hoc toolchest and can be adapted easily to suit ones needs.

If one for example would like a gamess calculation to start on a (Lowdin) orthogonalised
set of unit vectors, one would have to use (a default basis will not do)
 
{
\footnotesize
\begin{verbatim}
          title
          h2co - 3-21g - closed shell SCF
          adapt off
          zmatrix angstrom
          c
          o 1 1.203
          h 1 1.099 2 121.8
          h 1 1.099 2 121.8 3 180.0
          end
          basis 3-21g
          servec
          extra 
          1 to 22 end
          s
          lowdin
          1 to 22 end
          write ed3 1 9
          finish servec
          vectors getq ed3 1 9
          enter
\end{verbatim}
}

\subsection{Notation Conventions}

{
\footnotesize
\begin{verbatim}
     .. istring   =  string of integer numbers finished by 'end'
                     'to' convention is allowed i.e. 2 to 5 = 2 3 4 5
     .. tapnam    =  atmol file 
                     next parameters are iblock (start-block
                     for dumpfile) and isect (section-number)
     .. 'a'/'b'   =  choice between textstrings 'a' and 'b'
     .. 'flop'    =  text-string  (directive is always text)
     .. ('aa')    =  optional text-string
     ..  **       =  after ** comment is given
     ..  ex.      =  ex. denotes an example (starting in column 1)
 ** note  tapnam etc. must be on same card                          **
 **       istring may extend over many lines                        **
\end{verbatim}
}

\subsection{The TITLE Directive}

The next line is a title that will be used in all the following writes

{
\footnotesize
\begin{verbatim} 
       ex.  title
       ex.  this is the title that will appear on vector-files
\end{verbatim}
}

\subsection{The READ Directive}

{
\footnotesize
\begin{verbatim} 
 read   tapnam  ('notran')   ** read vectors
\end{verbatim} 
}
If 'notran' is specified the vectors are not transformed to the
original basis. All ctrans (symmetry adaption) information is lost
however (**not retained**)

{
\footnotesize
\begin{verbatim} 
       ex.   read ed3  1 1

 read  'input' ndim nmdim    ** read vectors from input free format
       ex    read input 3 2
 read  'free' ndim nmdim     ** read vectors from input using read *

 read  'format'    ** read vectors and occupations etc. formatted
\end{verbatim}
}
For the exact form of input *see* the print vectors 'format' directive.
this option is used to transport vectors as a text-file.

{
\footnotesize
\begin{verbatim} 
       ex.   read format
       ex.   .....................
       ex.   ............................
\end{verbatim}
}

\subsection{The WRITE Directive}

{
\footnotesize
\begin{verbatim} 
 write tapnam  ('gamess') ** write vectors (see read)
                          ** if 'gamess' is given vectors are gamess-compatible
\end{verbatim}
}

\subsection{The INIT Directive}

{
\footnotesize
\begin{verbatim} 
 init   tapnam         ** produce new (empty) dumpfile
       ex.   init  ed3 1
\end{verbatim}
}

\subsection{The S Directive}

{
\footnotesize
\begin{verbatim} 
 s  'print'             ** read overlap integrals 
   or
 s tapnam (without section) ndims 'print' ** read s from foreign dumpfile
                        **  if print specified the s-matrix is printed
       ex.   s  
       ex.   s ed5 1 122
\end{verbatim}
}

\subsection{The overlap Directive}

{
\footnotesize
\begin{verbatim} 
  overlap 'format'      ** print the overlap matrix (multiplied by 100) between the orbitals

       ex.  overlap
       ex.  overlap  (i5,(t8,20f7.2),1x)

\end{verbatim}
}


\subsection{The SCREEN Directive}

{
\footnotesize
\begin{verbatim} 
  screen 'crit'   ** make all s-matrix elements < crit exactly zero
                  ** if crit is omitted the current criterion is used
\end{verbatim}
}

\subsection{The PRINT Directive}

{
\footnotesize
\begin{verbatim} 
  print  'on'/'off'    ** switches print flag
  print  'vectors'  nv  ('occ'/'eig')  ** print first nv vectors
                               **  default nv = nmdim
                               **  if 'occ' => give occupations
                               **  if 'eig' => give orbital energies
       ex.   print off
      2ex.   print vectors 7

  print  'vectors'  nv  'format' format  ** print vectors formatted
           ** according to format(format). nv=0 means nv = nmdim
           **  default format =  5f16.9
           ** this option is used to transport vectors as text-file
           ** see also *read format* / the output looks as follows
           *** 'vectors' ndim  nv     format
           *** ((vc(i,j),i=1,ndim),j=1,nv)  (format(format))
           *** 'occ'  nn         ** nn = 0 for no occupations
           *** (occ(i),i=1,nn)   (format(format))
           *** 'eig'  nn         ** nn = 0 for no eigen-values
           *** (eig(i),i=1,nn)   (format(format))
           *** 'end of format-print'
           ** the input for read format should look the same

      ex.   print vectors 0 format   5e16.10
\end{verbatim}
}

\subsection{The SCHMIDT Directive}

{
\footnotesize
\begin{verbatim} 
  schmidt  istring     ** schmidt orthogonalize the orbitals specified
       ex.   schmidt   1 5 7 3 end

  schmidt  'set' ('norm'/'nonorm') istring istring 
                     ** schmidt first istring onto second
                     ** (no internal orthogonalisations)
                     ** if 'norm' is specified the set is normalised
                     == The default is nonorm (the string may be omitted)
       ex.   schmidt  set  1 4 end 2 3 end
       ** as accuracy of schmidt crit is used
\end{verbatim}
}

\subsection{The LOWDIN Directive}

{
\footnotesize
\begin{verbatim} 
  lowdin   istring     ** lowdin orthogonalise the orbitals specified
       ex.   lowdin    2 3 4 5  end
       ** as accuracy of lowdin crit*10 is used
\end{verbatim}
}

\subsection{The NORMALISE Directive}

{
\footnotesize
\begin{verbatim} 
  normalise istring   ** normalise the orbitals specified
\end{verbatim}
}

\subsection{The CHECK Directive}

{
\footnotesize
\begin{verbatim} 
  check  ('print')    **  check vectors on orthonormality  within crit.
                      **  if in the overlap-matrix of the vectors an
                      **  element > crit is found the matrix is printed
                      **  if 'print' is selected it is printed anyway
\end{verbatim}
}

\subsection{The SDIAG Directive}

{
\footnotesize
\begin{verbatim} 
  sdiag ('set') ('istring') **  diagonalise s-matrix and leave its eigen-
                            **  vectors as current vector-set, ordered by
                            **  decreasing eigen-value / needs s-matrix
                            **  see  's' and 'crit' directives
                            **  ** if set is specified the s-matrix
                            **  for that set of unit vectors is diago-
                            **  nalised and the eigenvectors are of
                            **  the original dimension
\end{verbatim}
}

\subsection{The LOCMP Directive}

{
\footnotesize
\begin{verbatim} 
  locmp iset nset     ** localise within the set of nset orbitals
                      ** starting at orbital iset.
                      ** v. magnasco, a. perico jcp 47,971 (1967)
  istring             ** a string of numbers indicating
                      ** the ao' to localise an mo on. when all mo's
                      ** to be localised are specified, give 'end'
  'end'               ** # mo's to be localised  .le. nset
                      ** an moperm may be useful before applying locmp
                      **   *** locmp needs s-integrals and vectors so
                      **   *** the 's' and a 'read' directive must
       ex.   locmp 3  7
       ex.   1 4 5  0
       ex.   1 2 3  0
       ex.   end
\end{verbatim}
}

\subsection{The MOPERM Directive}

{
\footnotesize
\begin{verbatim} 
  moperm   istring     ** permute mo's
       ex.   moperm   1 to 10 13 12  end
\end{verbatim}
}

\subsection{The AOPERM Directive}

{
\footnotesize
\begin{verbatim} 
  aoperm   istring     ** permute ao's
       ex.   aoperm  17 18 end
\end{verbatim}
}

\subsection{The COMBINE Directive}

{
\footnotesize
\begin{verbatim} 
  combine  tapnam  tapnam ** identical to atmolscf combine option
                          ** except that no orthogonalisation is done
                          ** combine first and second vector - set
        ** next cards define the origin of the ao's **
           ifr ito 'a'/'b' ii ** ao's ifr to ito (inclusive) in new set
           ... ...   ...   .. ** are ao's ii onwards of vectorset a/b
           ... ...   ...   .. ** go on until all ao's are assigned
           'end'
        ** next cards define the origin of the mo's **
           ifr ito 'a'/'b' ii ** same as for ao's / finish with 'end'
           'end'
       ex.   combine ed6 1 1 ed4 200 3
       ex.   1 5 a 1
       ex.   6 10 b 1
       ex.   11 20 a 6
       ex.   21 26 b 6
       ex.   end
       ex.   1  4  a 1
       ex.   5  7  b 1
       ex.   8 18  a 5
       ex.   19 26  b 4
       ex.   end
        **  Alternatively a simplified form may be used just specifying quantities
        **  and on the same line
       ex.  combine  ed6 1 1 ed4 200 3
       ex.  a 5 b 5  a 10  b 6  end
       ex.  a 4 b 3  a 11  b 8  end
\end{verbatim}
}

\subsection{The EXTRA Directive}

{
\footnotesize
\begin{verbatim} 
  extra    istring     ** add ao's and corresponding unit-vectors
                       ** (in existing vectors zeros are inserted)
                       ** warning  : ndim **must** equal nmdim
           ** e.g. one performed a dz-calculation on h2 using ao's
           ** 1,2 and 3,4 . now one adds polarisation functions (p)
           ** on each h-atom (p's will be 3,4,5 and 8,9,10). the start-
           ** orbitals for this new calculation are produced (starting
           ** from the old 4*4 set) by putting :
       ex.   extra 3 4 5 8 9 10 end
\end{verbatim}
}

\subsection{The SUMMARY Directive}

{
\footnotesize
\begin{verbatim} 
  summary              ** print summary of files and parameters
\end{verbatim}
}

\subsection{The TRANSFORM Directive}

{
\footnotesize
\begin{verbatim} 
  transform tapnam ('notran')   ** transform nmdim*ndim vectors in core
                                ** with a nmdim*nmdim set on tapnam
                                ** notran as in read
\end{verbatim}
}

\subsection{The OCCUP or OCC Directive}

{
\footnotesize
\begin{verbatim} 
  occup   nval (occ(i),i=1,nval) ** read occupation numbers (free form)
      ex.    occup 3  2.0 2.0 1.5
      ex2.   occ 2  2 1 
           ** the occupation-numbers will appear on any tape4 or dump-
           ** file written , following this directive
\end{verbatim}
}

\subsection{The EIGEN Directive}

{
\footnotesize
\begin{verbatim} 
  eigen   nval              ** read nval orbital-energies on following
          (eig(i),i=1,nval) ** according to format(5e15.5)
      ex.    eigen 2
      ex.     -20.5673456     -11.43456
\end{verbatim}
}

\subsection{The CRIT Directive}

{
\footnotesize
\begin{verbatim} 
  crit    acr  npower ** set criterion to acr.10**(-npower)
                      ** (default  1.0e-14)
      ex.    crit    1.0   20
\end{verbatim}
}

\subsection{The NDIM Directive}

{
\footnotesize
\begin{verbatim} 
  ndim    nd   ** reduce ao-dimension to nd (loose last ao's)
      ex.    ndim    22
\end{verbatim}
}

\subsection{The NMDIM Directive}

{
\footnotesize
\begin{verbatim} 
  nmdim   nmd  ** reduce mo-dimension to nmd (loose last mo's)
\end{verbatim}
}

\subsection{The LOG Directive}

{
\footnotesize
\begin{verbatim} 
  log  'on'/'of' ** switch printflag for printing of directive-cards
                 ** in interactive sessions log on is more forgiving
\end{verbatim}
}

\subsection{The ADDVEC Directive}

{
\footnotesize
\begin{verbatim} 
  addvec  tapnam ** add vectors (same dimension as current vectors)
                 ** from specified file ,as last orbitals
      ex.    addvec ed3 1 2
  addvec  'unit'  istring .. .. ** add specified unit vectors
                             ** (i.e. with 1.0 in positions specified by istring
                             **  respectively )
      ex.    addvec unit 11 7 1
\end{verbatim}
}

\subsection{The SYMMETRY Directive}

{
\footnotesize
\begin{verbatim} 
  symmetry  nao nirr n1 n2 n3 .. ** 'symmetry' selection of mo's
           ** nao = number of ao's in each mo to look at (max 8)
           ** nirr= number of representations to select  (max 12)
           ** n1,n2,n3 etc : dimensions of subsets within mo's
            ** next cards define selection (nirr cards) **
               (iao(i),ich(i),i=1,nao) ** iao = number of ao
               .. .. .. .. .. .. .. .. ** ich = 1,0 or -1 = sign of ao
               .. .. .. .. .. .. .. .. **       in mo  (parity is ok)
       ex.   symmetry 3 2  3 5
       ex.   1 1    3 0    12  -1
       ex.   1 0    4 0    10   1

  symmetry 'h'            ** the symmetry is determined from the
                          ** (1-electron) h-matrix 
          ** following directives may be issued / finish with 'end'
      'sets'  is1 is2 ... ** dimensions of subsets in orbital-space
     'rename' ir1 ir2 ... ** new numbers of the representations
      'end'               **
       ex.   symmetry h tape3
       ex.   order 1 4 2 3
       ex.   sets  1 4 8 1
       ex.   end
      ** this sorts the orbitals for water in a dz basis , keeping
      ** the 1s and the 1s* , the occupied and the virtual spaces
      ** apart / the a2 symmetry does not occur in the occupied
      ** space ,is therefore originally assigned # 4 and is reordered
      ** to position 2

 *******************************************************************
 **** a directive : crit 5 5 : is recommended before symmetry   ****
 **** selection, since orbitals are usually only accurate to    ****
 **** 5 figures                                                 ****
 *******************************************************************
\end{verbatim}
}

\subsection{The MAXDIM Directive}

{
\footnotesize
\begin{verbatim} 
  maxdim 'low'/'medium'/'high' / set maximum dimension  (default low)
                              ** low : all actions possible
                              ** medium : no orthogonalisation possible
                              ** high  : and no combine possible
       ex.   maxdim   high
\end{verbatim}
}

\subsection{The ATOMMO Directive}

{
\footnotesize
\begin{verbatim} 
  atommo 'atom' istring 'mo' istring / sorts the mo's listed in the order of the atoms
      ex. atommo atom 1 2 3 end mo 1 2 3 4 5  end  
\end{verbatim}
}

\subsection{The DUMPFILE Directive}

{
\footnotesize
\begin{verbatim} 
  dumpfile tapnam (without section)  / set the standard dumpfile to tapnam
               ** the dumpfile must exist
               ** default the GAMESS dumpfile is used
               ** if not available the first dumpfile vectors were read from

      ex. dumpfile ed4 1
\end{verbatim}
}


\subsection{The STOP / FINISH Directive}

{
\begin{verbatim} 
  stop/finish     **  ends calculation
       ex.   stop
\end{verbatim}
}

\section{OPTBS - Single variable Basis Set  optimisation}

The program optimises a set of even tempered outer exponents, taking the
existing basis as a start. Currently only a single parameter optimisation  of
the exponent ratio is available. Optimisation is possible for mSCF,MP2 and CASSCF
and DFT; Dft can give problematic convergence though.

Data for  OPTBS  consists of one line, read to variables
TEXT, OUTEXP,NEXP using format (A,2I).
In addition to these required parameters, STEP and TOL may be  optionally  specified, using
The TEXT,VARIABLE format (A,F). Also a PRINT (format(A)) may be specified.
\begin{itemize}
\item TEXT should be set to the character string OPTBS.
\item OUTEXP specifies the index of the outer core exponent, i.e. the last one not changed.
\item NEXP  is the number of exponents to be optimised.
\item STEP specified the stepsize taken in changing the exponent ratio (default 0.1)
\item TOL is the termination criterion (default 0.01)
\item PRINT gives output of all SCF-type calculations
\end{itemize}


{\bf Example}
{
\footnotesize
\begin{verbatim}
          SERVER
          OPTBS 15 2 STEP 0.1 TOL 0.01
          ....... GAMESS - INPUT .........
          ENTER
\end{verbatim}
}

Starting from the GAMESS input it will optimise exponents 16 and 17.

\end{document}
