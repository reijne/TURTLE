\documentclass[11pt,fleqn]{article}

\usepackage{hyperref}

% package HTML requires Latex2HTML to be installed for html.sty
\usepackage{html}
\newcommand{\doi}[1]{doi:\href{http://dx.doi.org/#1}{#1}}
\newcommand{\http}[1]{\href{#1}{#1}}
\begin{htmlonly}
\renewcommand{\href}[2]{\htmladdnormallink{#2}{#1}}
\end{htmlonly}
\hypersetup{colorlinks,
            %citecolor=black,
            %filecolor=black,
            %linkcolor=black,
            %urlcolor=black,
            bookmarksopen=true,
            pdftex}
 
\addtolength{\textwidth}{1.0in}
\addtolength{\oddsidemargin}{-0.5in}
\addtolength{\topmargin}{-0.5in}
\addtolength{\textheight}{1.0in}
\newcommand{\degree}[1]{\mbox{$#1^{o}$}}
\newcommand{\cah}{\mbox{CaH$_{2}$}}

\pagestyle{headings}
\pagenumbering{roman}
\begin{document}
\sf
\parindent 0cm
\parskip 1ex
\begin{flushleft}
 
Computing for Science (CFS) Ltd.,\\CCLRC Daresbury Laboratory.\\[0.30in]
{\large Generalised Atomic and Molecular Electronic Structure System }\\[.2in]
\rule{150mm}{3mm}\\
\vspace{.2in}
{\huge G~A~M~E~S~S~-~U~K}\\[.3in]
{\huge USER'S GUIDE~~and}\\[.2in]
{\huge REFERENCE MANUAL}\\[0.2in]
{\huge Version 8.0~~~June 2008}\\ [.2in]
{\large PART 9. TROUBLE SHOOTING}\\
\vspace{.1in}
{\large M.F. Guest, J. Kendrick, J.H. van Lenthe and P. Sherwood}\\[0.2in]
 
Copyright (c) 1993-2008 Computing for Science Ltd.\\[.1in]
This document may be freely reproduced provided that it is reproduced\\
unaltered and in its entirety.\\
\vspace{.2in}
\rule{150mm}{3mm}\\
\end{flushleft}

% 
\tableofcontents
\newpage

\pagenumbering{arabic}

\section[Introduction]{Introduction}

In this chapter we try to anticipate areas where the user may run into
problems, either from lack of appreciation of the workings of the
program, or from the author's failure to present a clear picture of
requirements.  The items below are presented in no real order, but
reflect some of the known aspects of running the program that have
given problems historically. Again users are welcome to use the EMAIL
addresses given in Chapter~1 in trying to resolve matters not discussed
herein. Note that the original use of error numbers has been replaced
with textual messages designed to provide some insight into the problem
causing the abort. If the user meets an error condition characterised
only by a number in the log, please inform the author.

\section[SCF Calculations]{SCF Calculations}
\begin{enumerate}
\item {\em The SCF aborts at the outset of processing}; this
will almost certainly have arisen from supplying a two-electron
file that is not in the format expected. Care must be taken
in understanding the workings of the SUPER directive and the
use of BYPASS.
\item {\em The SCF fails to converge, with violent oscillations
in the total energy}; increase the LEVEL parameters, which
should remove the problem.
\item {\em Large negative values for the energy}: the program
assumes in constructing the Fock matrix that the 
density matrix transforms as the totally symmetric
representation i.e. the wave function belongs to a
non-degenerate representation of the molecular point group.
If this is not the case, the energy may collapse to a
meaningless value. If this occurs, the user should 
modify the TAGs in the z-matrix to lower the effective
molecular point group.
\item {\em The SCF output reveals that the DIIS procedure has
been initiated, but the energy still fluctuates, with DIIS
appearing to push the energy up}; this may be 
indicative of trying to converge on an excited state,
particularly in closed-shell calculations. If the TESTER
is small, and the problem persists, it may be advantageous
to suppress DIIS through the data line DIIS OFF.
\item {\em The two-electron integral file
is lost or corrupted}, say in the middle of an optimisation,
use the REGEN parameter on the RESTART directive.
\item {\em The error message ATTEMPTING TO RETRIEVE UNDEFINED DUMPFILE
SECTION appears during vector generation in a restart job}; this
may have arisen from specifying an incorrect section on the
VECTORS directive, but it is also indicative of
the user having inadvertently omitted the RESTART directive, or
incorrectly allocated the Dumpfile to the Job.
\item {\em Confusion over Multiple Section Specification}: When
performing open shell or GVB calculations, the user should be
aware that the final list of output vectors refers to the
canonicalised set i.e. to the set of MOs residing in the
{\em second} section specified on the ENTER directive. Confusion
may arise when restoring MOs from the first section, and
assigning, for example, orbital symmetries from the SCF output.
\item {\em Symmetry contaminated molecular orbitals}: the SCF
modules are fairly demanding in their attempts to maintain the
symmetry classification of the MOs. If the orbitals appear
contaminated, the cause may lie in the choice of an 
inappropriate starting set used to initialise the SCF.

\end{enumerate}

\section[The role of the VECTORS and ENTER Directives]{The role of the VECTORS and ENTER Directives}

{\em GAMESS is reporting an error when trying to restore vectors
in a restart job after a startup step has completed successfully,
referencing a section number unknown to the User}\\

Consider the following data, that aims to first conduct an initial
geometry optimisation, followed in a subsequent step by a direct-CI
calculation at the optimised geometry.

{
\footnotesize
\begin{verbatim}
       TITLE
       WATER
       ZMAT ANGS
       O
       H 1 R
       H 1 R 2 A
       VARIABLES
       R             0.9429786  HESSIAN    0.749526
       A            105.9806454 HESSIAN    0.178413
       END
       BASIS 6-31G
       RUNTYPE OPTIMIZE
       ENTER 1
\end{verbatim}
}
with the following data specified in the restart job;

{
\footnotesize
\begin{verbatim}
       RESTART
       TITLE
       WATER
       ZMAT ANGS
       O
       H 1 R
       H 1 R 2 A
       VARIABLES
       R             0.9429786  HESSIAN    0.749526
       A            105.9806454 HESSIAN    0.178413
       END
       BASIS 6-31G
       RUNTYPE CI
       DIRECT
       ENTER 1
\end{verbatim}
}
This restart will fail, with the diagnostics shown below taken from the
failing output:

{
\footnotesize
\begin{verbatim}
 MOLECULAR ORBITALS RESTORED FROM DUMPFILE
 -A-   VECTORS RESTORED FROM SECTION 410 OF DUMPFILE ON ED3
 
 *********************
 PROBLEM WITH SECTION
 *********************
 MPOS, MTYPE, BLOCK   410    3*****
 CURRENT INPUT LINE IS:
 ENTER 1
#*#*#*#*#*#*#*#*#*#*#*#*#*#*#*#*#*#*#*#*#*#*#*#*#*#*#*#*#*#*#*#*#*#*#*#
                                                  ERROR DETECTED
                                                  **************
 
                    ATTEMPTING TO RETRIEVE UNDEFINED DUMPFILE SECTION
 
*#*#*#*#*#*#*#*#*#*#*#*#*#*#*#*#*#*#*#*#*#*#*#*#*#*#*#*#*#*#*#*#*#*#*
 
 
GAMESS-UK ERROR: ATTEMPTING TO RETRIEVE UNDEFINED DUMPFILE SECTION
\end{verbatim}
}
{\bf Diagnosis}\\

In the absence of section specification on the VECTORS and ENTER
directive, GAMESS will use a default section (410) for keeping the
current set of vectors. If the user specifies a section on the ENTER
directive of the startup job (section 1) above, then that section MUST
be specified in subsequent jobs, for GAMESS will not redefine the
initial "ENTER" section in the first step as being the section for use
of VECTORS in the second step. If no VECTORS directive is specified,
its default choice is a function of the perceived status of the
processing:
\begin{itemize}
\item in startup job, with no vectors available, it will assume "VECTORS
ATOMS" and proceed from an atomic startup.
 
\item in a restart job, it will assume that vectors are available, and
that such vectors are to be restored from a section on the Dumpfile. If
no such section is specified, it will try to input those from a default
section, 410. This will lead to the diagnostic above.
\end{itemize}
{\bf Solution 1}\\

Do not specify section numbers and let the programme defaults apply
(this is somewhat limiting, since direct, data-driven access to the
default section is not permitted by the user i.e. section 410 cannot be
explicitly specified on a directive). However, in the present example
the following data will suffice:

{\bf Data for Step 1}
{
\footnotesize
\begin{verbatim}
       TITLE
       WATER
       ZMAT ANGS
       O
       H 1 R
       H 1 R 2 A
       VARIABLES
       R             0.9429786  HESSIAN    0.749526
       A            105.9806454 HESSIAN    0.178413
       END
       BASIS 6-31G
       RUNTYPE OPTIMIZE
       ENTER
\end{verbatim}
}

{\bf Data for Step 2}
{
\footnotesize
\begin{verbatim}
       RESTART
       TITLE
       WATER
       ZMAT ANGS
       O
       H 1 R
       H 1 R 2 A
       VARIABLES
       R             0.9429786  HESSIAN    0.749526
       A            105.9806454 HESSIAN    0.178413
       END
       BASIS 6-31G
       RUNTYPE CI
       DIRECT
       ENTER
\end{verbatim}
}

{\bf Solution 2}\\

Use section number for explicit control of vector placing.
This will require use of the VECTORS directive in the second
step to nominate section 1.

{\bf Data for Step 1}\\

This is just the data specified in the startup job above:
{
\footnotesize
\begin{verbatim}
       TITLE
       WATER
       ZMAT ANGS
       O
       H 1 R
       H 1 R 2 A
       VARIABLES
       R             0.9429786  HESSIAN    0.749526
       A            105.9806454 HESSIAN    0.178413
       END
       BASIS 6-31G
       RUNTYPE OPTIMIZE
       ENTER 1
\end{verbatim}
}

{\bf Data for Step 2}\\

This now specifies the explicit section for the VECTORS the absence of
which caused the problem in the restart job above:
{
\footnotesize
\begin{verbatim}
       RESTART
       TITLE
       WATER
       ZMAT ANGS
       O
       H 1 R
       H 1 R 2 A
       VARIABLES
       R             0.9429786  HESSIAN    0.749526
       A            105.9806454 HESSIAN    0.178413
       END
       BASIS 6-31G
       RUNTYPE CI
       DIRECT
       VECTORS 1
       ENTER 1
\end{verbatim}
}


\section[Transition State and Geometry Optimisation]{Transition State and Geometry Optimisation}

\begin{enumerate}
\item {\em ZMATRIX and Point Group Problems}: One of the constraints
in the z-matrix formalism concerns potential problems in dealing with
directly bonded angles of \degree{180}. The use of dummy X-centres
in circumventing these problems has been discussed in Part 3, \S8.3. Consider
below the data file for optimisation of the \cah\ species, where we
are starting the optimisation with a bond angle of \degree{150}.

{
\footnotesize
\begin{verbatim}
          TITLE\CAH2 .. 3-21G
          ZMAT ANGS\CA\H 1 CAH\H 1 CAH 2 HCAH
          VARIABLES\CAH 2.148 HESS 0.7\HCAH 150.0 HESS 0.1\END
          RUNTYPE OPTIMIZE
          VECTORS EXTGUESS\ENTER 1
\end{verbatim}
}
This job will fail, with the diagnostics shown below taken from the
failing output:

{
\footnotesize
\begin{verbatim}
     ==================================================================
                             Z-MATRIX (ANGSTROMS AND DEGREES)
     CD CENT ATOM  N1      LENGTH      N2     ALPHA      N3      BETA 
     ------------------------------------------------------------------
      1   1   CA
      2   2   H     1   2.148195 (  1)
      3   3   H     1   2.148195 (  2)  2  181.008 (  3)
     ==================================================================

     ==============================================
     VARIABLE           VALUE              HESSIAN
     ==============================================
     CAH            2.1481952 ANGS        0.699998
     HCAH         181.0076613 DEGS        0.034888
     
     ERROR ON Z-MATRIX CARD NUMBER    3
     
     ANGLE ALPHA IS OUTSIDE THE VALID RANGE OF 0 TO 180

     *#*#*#*#*#*#*#*#*#*#*#*#*#*#*#*#*#*#*#*#*#*#*#*#*#*#*#*#*#*#*#

                                   ERROR DETECTED
                                   **************

      ERROR DETECTED IN CONVERTING Z-MATRIX TO CARTESIAN COORDINATES 

     *#*#*#*#*#*#*#*#*#*#*#*#*#*#*#*#*#*#*#*#*#*#*#*#*#*#*#*#*#*#*#
\end{verbatim}
}
In most cases this can be avoided by using the half-angle and
dummy centre specification, as shown below:

{
\footnotesize
\begin{verbatim}
          TIME 60 
          TITLE\CAH2 .. 3-21G
          ZMAT ANGS\CA\X 1 1.0\ H 1 CAH 2 XCAH\H 1 CAH 2 XCAH 3 180.0
          VARIABLES\CAH 2.148 HESSIAN  0.7\XCAH 75.0 HESSIAN 0.1\END
          RUNTYPE OPTIMIZE
          VECTORS EXTGUESS\ENTER 1
\end{verbatim}
}
An additional constraint in the default optimisation algorithm
is the inability to handle a change in point group during the
optimisation pathway. This example has been chosen to illustrate
this point: thus while the above data file avoids the angle
problem, it will again fail because of the point group change,
as shown below:

{
\footnotesize
\begin{verbatim}
     ====================================================================
                              Z-MATRIX (ANGSTROMS AND DEGREES)
      CD CENT ATOM  N1      LENGTH      N2     ALPHA      N3      BETA  
     --------------------------------------------------------------------
       1   1   CA
       2       X     1   1.000000 (  1)
       3   2   H     1   2.148105 (  2)  2   89.998 (  4)
       4   3   H     1   2.148105 (  3)  2   89.998 (  5)  3  180.000 (  6) 
     ===================================================================
     
      ==============================================
      VARIABLE           VALUE              HESSIAN
      ==============================================
      CAH            2.1481048 ANGS        0.699995
      XCAH          89.9978901 DEGS        0.156813
     
      **** CHANGE IN POINT GROUP ****
     
               CNV          2     DNH          2
     
     *#*#*#*#*#*#*#*#*#*#*#*#*#*#*#*#*#*#*#*#*#*#*#*#*#*#*#*#*#*#*#*#*#*#*#
               ERROR DETECTED
       POINT GROUP CHANGE DURING OPTIMISATION 
     *#*#*#*#*#*#*#*#*#*#*#*#*#*#*#*#*#*#*#*#*#*#*#*#*#*#*#*#*#*#*#*#*#*#*#
\end{verbatim}
}
In such cases the optimisation may be completed by changing to the
OPTXYZ algorithm, using the RESTART option to retrieve the partially
converged geometry from the Dumpfile as the starting point. Note that
OPTXYZ will not suffer from either ZMATRIX or Point group problems.

{
\footnotesize
\begin{verbatim}
          RESTART
          TITLE\CAH2 .. 3-21G
          ZMAT ANGS\CA\X 1 1.0\ H 1 CAH 2 XCAH\H 1 CAH 2 XCAH 3 180.0
          VARIABLES\CAH 2.148 HESSIAN  0.7\XCAH 75.0 HESSIAN 0.1\END
          RUNTYPE OPTXY
          VECTORS 1\ENTER 1
\end{verbatim}
}
\item Particular care should be taken when dealing with the
ZMATRIX specification for cyclic systems. If this is specified
such that a multiple bond is not explicitly defined, then the
initial step taken in an optimisation with either a poor
starting geometry or ill-defined Hessian can lead to unreasonable
values for such a bond. In the worst case, the bond can become so
compressed (with a disastrous effect on the total energy) that
linear dependence will be (correctly) diagnosed! Certainly 
in such cases the starting hessian should be taken from a smaller
basis calculation (STO-3G will do ..).
\item {\em PARAMETER ERRORS}: Possible errors may stem from the 
user not appreciating
the workings of the optimisation modules (as loaded 
under RUNTYPE OPTIMIZE or RUNTYPE SADDLE). Both algorithms
are based on the notion of a `history' file, with any restart
jobs working through the previous points along the
optimisation pathway, prior to continuing execution 
at the interrupted point. Anything which perturbs this
process, causing the `history' file, as present on the 
Dumpfile, to appear at odds with that generated from the
data input will lead to a `PARAMETER ERROR', and
termination of execution. Basically, once the optimisation
has been initiated, the user should not modify the
input data, other than through the introduction of the
RESTART directive, modification of the VECTORS specification etc.
Specifically, modifying the ZMATRIX data or optimisation controls
(XTOL etc.) will lead to this error condition.
Let us consider one example where this might occur. In the
data files below we are using the results of a 3-21G
optimisation as a basis for a larger TZVP optimisation. The 3-21G
data might appear as follows:

{
\footnotesize
\begin{verbatim}
          TITLE
          H2O -- 3-21G
          ZMAT ANGSTROM
          O 
          H 1 OH 
          H 1 OH 2 HOH 
          VARIABLES
          OH 0.956  HESSIAN 1.1
          HOH 109.05  HESSIAN 0.2
          END
          RUNTYPE OPTIMIZE
          VECTORS EXTGUESS
          ENTER 1
\end{verbatim}
}
with the TZVP startup job below using the vectors and hessian
from the split-valence calculation:

{
\footnotesize
\begin{verbatim}
          DUMPFILE ED3 300
          TITLE
          H2O -- TZVP -- RESTARTING FROM 3-21G
          ZMAT ANGSTROM
          O 
          H 1 OH 
          H 1 OH 2 HOH 
          VARIABLES
          OH 0.956  HESSIAN 1.1
          HOH 109.05  HESSIAN 0.2
          END
          BASIS TZVP
          RUNTYPE OPTIMIZE ED3
          VECTORS GETQ ED3 1 1
          ENTER 1
\end{verbatim}
}
Let us assume the above calculation terminated on time up: typical
restart data is shown below. Note that the RUNTYPE directive
is as before i.e. still restoring the 3-21G Hessian. If the
ED3 specification is removed from this line, the job would fail
immediately since the starting-Hessian will now be
different from that resident on the `history' file.

{
\footnotesize
\begin{verbatim}
          DUMPFILE ED3 300
          RESTART OPTIMIZE
          TITLE
          H2O -- TZVP -- RESTARTING FROM 3-21G
          ZMAT ANGSTROM
          O 
          H 1 OH 
          H 1 OH 2 HOH 
          VARIABLES
          OH 0.956  HESSIAN 1.1
          HOH 109.05  HESSIAN 0.2
          END
          BASIS TZVP
          RUNTYPE OPTIMIZE ED3
          VECTORS 1
          ENTER 1
\end{verbatim}
}
\item  {\em The optimisation appears to have converged but 
iterations continue}; if too small a value of XTOL is specified 
the energy is not sufficiently numerically accurate 
to justify using line searches. This is especially true 
when employing default accuracy integrals.
\item {\em The optimisation will not converge}:
\begin{itemize}
\item Check that the starting guess is reasonable
\item In a SADDLE calculation  the quality of the initial 
hessian is crucial, and the user should if possible generate this
at the outset of the optimisation (ie. specify TYPE 3 for all
the variables) - it should have only one negative eigenvalue
if the starting point is reasonable.
\item If a large step is made it is possible for the wavefunction
to change state. Usually this is manifested by the linear 
search returning
to a previous point on the surface, or by exceptionally large values
in the hessian matrix. To avoid this reduce STEPMAX, and if dealing
with a closed shell system use LOCK.
\end{itemize}
\end{enumerate}
\subsection{Synchronous Transit Algorithm}
\begin{enumerate}
\item {\em Too many z-subspace iterations}: If 
many more than (n-1) iterations are performed in any one
z-subspace minimisation then the parameter TOLMAX should be reduced.
\item {\em The optimisation will not converge}:

\begin{itemize}
\item Check that the minima have been input correctly
\item Check that the starting guess is reasonable
\item Try calculating the full hessian (ie. specify TYPE 3 for all
the variables) - it should have only one negative eigenvalue
if the starting point is reasonable.
\item Sometimes the linear search along the polynomial for a
maximum causes problems which result in very large rises in energy.
Occasionally this may be overcome by increasing the VALUE
parameter so that the linear search is not as rigorous (check
also that STEPMAX is not too large).
\end{itemize}
\end{enumerate}

\section{Direct-CI Calculations}

The internal symmetry manipulations within the Direct-CI module
are largely hidden from the user, although the re-ordering
performed to align both internal and external MOs into
groups of common IRrep is recorded on the output. It is 
possible for incorrect CONF specification to lead to problems,
and one of the best ways to monitor such effects is to
look at the output from the 'pre-sorting' of the transformed
integrals, a typical example of which is shown below. 

{
\footnotesize
\begin{verbatim}
        **** TRANSFORMED INTEGRAL PRE-SORT CALLED AT   380.21 SECS

        TRANSFORMED 1-ELECTRON INTEGRALS RESTORED FROM SECTION 466 
        OF DUMPFILE STARTING AT BLOCK       1 OF ED3 
       
        *** SYMMETRY CHECK / LARGEST FORBIDDEN H-INTEGRAL :  39  2 -0.40457E-12
\end{verbatim}
}
If there are inconsistencies between the ACTIVE and CONF
specification, this will almost certainly lead to a large
value in the symmetry check, in which case the diagonalisation
results are probably meaningless.

\section{Table-CI Calculations}

In previous releases of the code, failing to allocate, or pre-compute 
the Table-CI data base lead to an operand range, or related error, 
at the outset of selection mode processing. This situation has now
been addressed, with the program checking for correct assignment of
the data base in the absence of the table directive.

\section{Known Bugs/Issues}

The list of know bugs, as well as a list of feature requests, is
maintained on the GAMESS-UK CCPForge site at:

\http{http://ccpforge.cse.rl.ac.uk/tracker/?group\_id=14}

If you think you have encountered a bug that is not listed there,
please send an email to the gamess-uk-users list at:

\http{http://ccpforge.cse.rl.ac.uk/mailman/listinfo/gamess-uk-users}

%% \section{Known Problems in Version 6.0}

%% The following comprises a list of the known problems that
%% have been reported by users of Version 6.0 of GAMESS-UK. We
%% are actively working to fix these bugs - should you encounter examples
%% of these, or any other error, please forward the associated data 
%% file and error to gamess\_uk\_support@dl.ac.uk.

%% \begin{enumerate}

%% \item {\em Restarting MCSCF Calculations (94-5)}:
%% Occasionally MCSCF calculations restart incorrectly. This is not, in
%% general, a serious problem, but can sometimes lead to convergence
%% problems (with the energy increasing).

%% \item {\em  Error during integral evaluation in DIRECT MP2 (94-8)}:
%% This sometimes occurs with large DIRECT MP2 calculations, the code
%% producing an error message stating that the indices for processing 
%% the integrals are too large. Note that this error refers to the
%% serial version of the code; it is {\em NOT} evident if the parallel
%% direct-MP2 described in Part 16.

%% \item {\em Invalid TABLE file (94-10)}:
%% The MRDCI Table generator on the SGI R8000 gives incorrect
%% results, producing an invalid TABLE file, This will lead to slight
%% discrepancies on subsequent MRDCI runs. This problem may be
%% circumvented by using the TABLE file generated from an R4000
%% or R4400.

%% \item {\em Incorrect orbital energies with Green's Functions (and TDA)
%% Calculations (95-7)}:
%% This error was first detected in a pseudopotential calculation on 
%% the Nickel complex NiCpNO. This problem remains to be resolved.

%% \item {\em Automatic Generation of z-matrices (95-11)}
%% The routine which automatically generates a zmatrix from
%% cartesian coordinates may fail when "non-bonded" atoms are present. 
%% This error was first reported for a Potassium crown ether, 
%% with GAMESS failing to bond the potassium to the rest of the structure. 
%% This has been partially resolved, although the algorithm may still 
%% fail for sandwich structures e.g., ferrocene.

%% \end{enumerate}
%% \clearpage

\end{document}
