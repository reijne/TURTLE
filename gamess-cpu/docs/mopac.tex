% Addition of GAMESS-UK directives by Jens .M. H. Thomas, Daresbury
% Laboratory, Warrington, England.
% j.m.h.thomas@dl.ac.uk
%
% Typesetting of MOPAC 6.00 manual 
% by John M. Simmie, Chemistry Dept., University College Galway, Ireland
% xra.simmie@bodkin.ucg.ie
% Main route to processing MOPAC manual thru LaTeX
% does table-of-contents and body
% uses Piet van Oostrums FANCYHEADINGS.STY to header pages
% and Leslie Lamport's indexing tool MAKEIDX.STY
% both of these are widely available from TeX/LaTeX archives
% but are not essential
\documentclass[11pt]{book}
\usepackage{fancyhdr}
\usepackage{makeidx}
%%%%%%%%%%%%%%%%%%%%%%%%%%%%%%%%%%%%%%%%%%%%%%%%%%%%%%%%%%%% Pre-amble
% Set up page parameters---basically A4 --- 72pt is 1 inch
 \topmargin 0pt
 \headheight 12pt % Height of page header
 \headsep 25pt    % Distance between header and text start
 \footskip 30pt   % Distance between text bottom and footer bottom
% \footheight 12pt % Height of page footer
 \oddsidemargin  0.25in  % extra left margin for odd pages
 \evensidemargin 0.125in % extra left margin for even pages
 \textwidth 5.875in
 \marginparwidth 0.5in % width of marginal notes
 \marginparsep   11pt  % Distance between text and marginal notes
 \textheight 53\baselineskip  % reduce if using shorter than A4 paper
 \advance\textheight by \topskip
% Very few constructs are used but c'est la vie
% Change fount in math mode for chemical elements; use as $\chem H_2O^+$
\def\chem{\rm\everymath={\rm}}
% Arrange for sections and subsecs only to be numbered
\setcounter{secnumdepth}{2}
% Tricky hyphenations
\hyphenation{occ-up-ied occ-up-a-tion occ-up-ies}
% Indexing requires Lamport's MakeIndex
% The following lessens the labour of indexing
\newcommand{\mi}[1]{#1\index{#1}}
% use as .... the \mi{sphaghetti} tree grows in the warm fertile soil ...
\makeindex
% This stuff for headering/footering pages
\pagestyle{fancyplain}
\renewcommand{\chaptermark}[1]{\markboth{#1}{#1}} % remember chapter title
\renewcommand{\sectionmark}[1]{\markright{\thesection\ #1}}
                                                % section number and title
\lhead[\fancyplain{}{\bf Page \thepage}]{\fancyplain{}{\bf\rightmark}}
\rhead[\fancyplain{}{\bf\leftmark}]{\fancyplain{}{\bf Page \thepage}}
\cfoot{}
%%%%%%%%%%%%%%%%%%%%%%%%%%%%%%%%%%%%%%%%%%%%%%%%%%%%%%%%%%%%%%%%%%%%%%
\begin{document}

\begin{flushleft}
{\huge G~A~M~E~S~S~-~U~K~~USER'S GUIDE}\\[0.2in]
{\huge Version 8.0~~~June 2008}\\ [.2in]
{\huge MOPAC Interface}\\
\vspace{.1in}
{\large J.M.H. Thomas}\\[0.2in]
\vspace{.2in}
\end{flushleft}

\section{GAMESS-UK/MOPAC Interface}
 This document is the original MOPAC 7.0 Manual written by Dr. James
 J. P. Stewart with a section pre-pended to it that serves to document
 the directives that allow users of GAMESS-UK to drive MOPAC from
 within GAMESS-UK. The original MOPAC manual is freely available for
 download on the web at:

{
\footnotesize
\begin{verbatim}
http://ccl.osc.edu/cca/software/SOURCES/FORTRAN/mopac7_sources/ \
mopac-uncompressed-manuals/mopac-man-latex-source/index.shtml
\end{verbatim}
}

No changes have been made to the manual itself, bar the inclusion of
this section.\\

The MOPAC interface within GAMESS-UK allows MOPAC to be run from a
standard GAMESS-UK input file. The MOPAC version that is supplied with
GAMESS-UK is version 7.0

The interface between GAMESS-UK and MOPAC currently permits geometries
from a MOPAC run to be imported into GAMESS-UK for incorporation in a
standard GAMESS-UK calculation. This permits users to run, for
example, a quick optimisation with the AM1 semi-emprical method before
importing the optimised geometry into GAMESS-UK to run a full ab inito
calculation.

There are example input files for running MOPAC from within GAMESS-UK in the
directory:

{
\footnotesize
\begin{verbatim}
         GAMESS-UK/examples/mopac
\end{verbatim}
}

\subsection{Running MOPAC from GAMESS-UK}
The MOPAC directives should be included in a standard GAMESS-UK
input file and must appear before any GAMESS-UK directives. The first
line of the input file should consist of the single keyword
\textbf{MOPAC} (in A format). A standard MOPAC input (as described in
the rest of this manual) should then follow, terminated by a single
blank line.

With an input of this format, all that will happen is that GAMESS-UK
will drive MOPAC, and the output produced will be a standard MOPAC
output prepended with the some minimal output generated by the
GAMESS-UK input processor.

An example of a such a simple MOPAC job is below, which shows a MNDO
calculation on water, run for a single SCF cycle.\\

The input for this example is the file: {\footnotesize \textbf{mopac\_1.in}}
%INPFILE: mopac_1.in
{
\footnotesize
\begin{verbatim}
mopac
mndo 1scf


h
o 1 1
h 1 1 111 1


\end{verbatim}
}

\subsection{Using MOPAC together with GAMESS-UK}
As described above, GAMESS-UK serves as little more than a wrapper for
running MOPAC. Of greater interest is the use of the results generated
by MOPAC in a GAMESS-UK run. Currently, the only data that can be
exported from MOPAC for use by GAMESS-UK are the atomic coordinates,
allowing MOPAC to serve as a quick optimiser for GAMESS-UK.

For this to work, the directive {\footnotesize\textbf{out=gamess}}
must be included on the MOPAC keyword line (the first line of the
MOPAC directives). This instructs MOPAC to create an archive file that
stores the coordinates for retrieval by GAMESS-UK.

Following the MOPAC directives, there should be a blank line followed
the keyword ''GAMESS'' (in A format), indicating the start of the
GAMESS-UK directives. From this line onwards, the directives should
just be standard GAMESS-UK directives.

To use the geometry from a MOPAC run in a GAMESS-UK job, the flag
''MOPAC'' should be appended to the GAMESS-UK GEOMETRY keyword, as
demonstrated in the example below.\\

The input for this example is the file: {\footnotesize \textbf{mopac\_2.in}}
%INPFILE: mopac_2
{
\footnotesize
\begin{verbatim}
	  mopac
	  prec density local vect mullik pi bonds xyz graph pm3 out=gamess
	  acetone.dat
	  " "
	  0008 -1.2166   0001 -0.0214   0001 0.0000    0001 0000 0000 0000
	  0006 0.0028    0001 0.0032    0001 0.0000    0001 0000 0000 0000
	  0006 0.7539    0001 1.3084    0001 0.0000    0001 0000 0000 0000
	  0006 0.7915    0001 -1.2794   0001 0.0000    0001 0000 0000 0000
	  0001 0.5285    0001 -1.8623   0001 -0.8951   0001 0000 0000 0000
	  0001 0.5285    0001 -1.8623   0001 0.8951    0001 0000 0000 0000
	  0001 1.8767    0001 -1.0977   0001 0.0000    0001 0000 0000 0000
	  0001 1.3862    0001 1.3756    0001 -0.8976   0001 0000 0000 0000
	  0001 1.3862    0001 1.3756    0001 0.8976    0001 0000 0000 0000
	  0001 0.0485    0001 2.1532    0001 0.0000    0001 0000 0000 0000

	  gamess
	  title
	  acetone 6-31g geometry optimisation from mopac starup
	  nosym
	  geometry mopac
	  basis 6-31g
	  runtype optxyz
	  xtol 0.003
	  enter
\end{verbatim}
}


\subsubsection{Specifying the archive file to use}
By default MOPAC will create an archive file called
{\footnotesize\textbf{archive}} containing the coordinates, and this
is what GAMESS-UK will expect to find. If however a file named archive
already exists in the directory, MOPAC will create one called
archiveaa, or if this exists, one called archiveab etc. If an archive
file called ''archive'' cannot be found when GAMESS-UK attempts to
import the geometry then it will crash with the following error
message:

{
\footnotesize
\begin{verbatim}
          0: GAMESS-UK Error: requested archive file missing or empty
\end{verbatim}
}


It is possible to tell GAMESS-UK which archive file to look for by
setting the envirnment variable ''archive'' to the name of the file
before the job is run. This shown below for the Bourne/BASH shells.

{
\footnotesize
\begin{verbatim}
          archive=myarchive; export archive
\end{verbatim}
}

This also allows one to use geometries stored in MOPAC archive files
from previous runs, by setting the ''archive'' environment variable to
point to the relevant file and then inserting the ''GEOMETRY MOPAC''
directive in a standard GAMESS-UK input file.

% BEGIN MOPAC MANUAL
\begin{titlepage}
\begin{center}
{\huge\bf MOPAC Manual (Seventh Edition)}\\
\vfill
%{\large\bf  Edited by}\\
\vfill
{\LARGE\bf Dr James J. P. Stewart}\\
\vfill
{\Large\bf PUBLIC DOMAIN COPY (NOT SUITABLE FOR PRODUCTION WORK)\\
\ \\
}
\vfill
{\Large January 1993}
\end{center}
\end{titlepage}

\pagenumbering{roman}

                   


 This document is intended for use by developers of semiempirical
programs and software.  It is {\em not} intended for use as a guide
to MOPAC.

All the new functionalities which have been donated to the MOPAC project
during the period 1989-1993 are included in the program.  Only minimal
checking has been done to ensure conformance with the donors' wishes.
As a result, this program should not be used to judge the quality
of programming of the donors.  This version of MOPAC-7 is not supported, 
and no attempt has been made to ensure reliable performance.

This program and documentation have been placed entirely in the public
domain, and can be used by anyone for any purpose.  
To help developers, the donated code is packaged into files, each
file representing one donation.

In addition, some notes have been added to the Manual.  These may be useful
in understanding the donations.


If you want to use MOPAC-7 for production work, you should get the
copyrighted copy from the Quantum Chemistry Program Exchange.  
That copy has been carefully written, and allows the donors' contributions
to be used in a full, production-quality program.

\lhead[\fancyplain{}{ }]{\fancyplain{}{\bf\rightmark}}
\rhead[\fancyplain{}{\bf\leftmark}]{\fancyplain{}{ }}

\tableofcontents
\newpage
\begin{itemize}
\item New Functionalities:
\begin{itemize}
\item {\large \bf Michael B. Coolidge}, The Frank J. Seiler Research
Laboratory, U.S.  Air Force Academy, CO 80840, and {\bf James J. P. Stewart},
Stewart Computational Chemistry, 15210 Paddington Circle, Colorado Springs,
CO 80921-2512. (The Air Force code was obtained under the Freedom of Information
Act)

 Symmetry is used to speed up FORCE calculations, and to
 facilitate the analysis of molecular vibrations.
\item  {\large \bf David Danovich}, The Fritz Haber Research Center for
Molecular Dynamics, The Hebrew University of Jerusalem, 91904 Jerusalem,
Israel.

 
 Ionization potentials are corrected using Green's Function techniques.  The resulting
I.P.s are generally more accurate than the conventional I.P.s.

 The point--group of the system is identified, and  molecular orbitals are characterized by irreducible representation.

\index{klamt@{\bf Klamt, Andreas}}
\item {\large \bf Andreas Klamt} Bayer AG, Q18, D-5090
Leverkusen-Beyerwerk, Germany.

A new approach to dielectric screening in solvents with explicit expressions for
the screening energy and its gradient has been added.

\end{itemize}
\item Existing Functionalities:
\begin{itemize}
\item {\large \bf Victor I. Danilov},\index{Danilof@{\bf Danilof, V. I.}} Department of Quantum Biophysics, Academy of
Sciences of the Ukraine, Kiev 143, Ukraine.
 
Edited the MOPAC 7 Manual, and provided the basis for
Section~\ref{FC}, on excited states.
 
\item {\large \bf Henry Kurtz} and {\large \bf Prakashan Korambath},
\index{Kurtz@{\bf Kurtz, Henry, A.}}
\index{Korambath@{\bf Korambath, Prakashan}}
Department of Chemistry, Memphis State University,
Memphis TN 38152.
 
The Hyperpolarizability calculation, originally written by Prof Kurtz, has been
improved so that frequency dependent non-linear optical
calculations can be performed. (Prakashan Korambath, dissertation research)
 
\item {\large \bf Frank Jensen}, Department of Chemistry, Odense Universitet, Campusvej 55, DK--5230 Odense M, Denmark.
 
The efficiency of Baker's EF routine has been improved.
 
\item {\large \bf John M. Simmie}, Chemistry Department, University College, Galway, Ireland.

 The MOPAC Manual has been completely re-formatted in the \mi{LaTeX}
document preparation
system.  Equations are now much easier to read and to understand.
\item {\large \bf Jorge A. Medrano}, 5428 Falcon Ln.,  West Chester, OH 45069,
and Roberto Bochicchio (Universidad de Buenos Aires).
 
The BONDS function has been extended to allow free valence and other quantities to be
calculated.
 
\item {\large \bf George Purvis III}, CAChe Scientific, P.O. Box 500, Delivery Station 13-400,
Beaverton, OR 97077.
 
The  STO-6G Gaussian expansion of the Slater orbitals has been expanded to Principal
Quantum Number 6.  These expansions are used in analytical derivative calculations.
\end{itemize}
\item Bug-reports/bug-fixes:
\begin{itemize}
\item {\large \bf Victor I. Danilov}, Department of Quantum Biophysics, Academy of
Sciences of the Ukraine, Kiev 143, Ukraine.
 
 Several faults in the multi-electron configuration interaction were identified,
and recommendations made regarding their correction.
\end{itemize}
\end{itemize}
\chapter{Description of MOPAC}
% First chapter --- reset page numbers to arabic
\pagenumbering{arabic}
%%%%%%%%%%%%%%%%%%%%%%%%%%%%%%%%%%%%%%%%%%%%%%%%%
MOPAC is a general-purpose semi-empirical molecular orbital  package
for  the  study of chemical structures and reactions.  The semi-empirical
Hamiltonians MNDO, MINDO/3, AM1, and PM3 are used in the electronic  part
of  the  calculation  to obtain \mi{molecular orbitals}, 
the heat of formation
and its derivative with  respect  to  molecular  geometry.   Using  these
results   MOPAC   calculates   the   vibrational  spectra,  thermodynamic
quantities, isotopic substitution effects and \mi{force constants} for
molecules, \mi{radicals}, \mi{ions}, and \mi{polymers}. For studying  chemical
reactions, a \mi{transition state} location routine and two  transition  state
optimizing  routines are available.  For users to get the most out of the
program, they must understand how the program works, how to  enter  data,
how to interpret the results, and what to do when things go wrong.

   While  MOPAC  calls  upon  many  concepts  in  quantum  theory   and
   thermodynamics  and  uses some fairly advanced mathematics, the user need
   not be familiar with these specialized topics.  MOPAC is written with the
   non-theoretician  in mind.  The input data are kept as simple as possible
   so users can give their attention  to  the  chemistry  involved  and  not
   concern themselves with quantum and thermodynamic exotica.

        The simplest description of how MOPAC works is that the user creates
   a data-file which describes a molecular system and specifies what kind of
   calculations and output are desired.  The user  then  commands  MOPAC  to
   carry  out  the  calculation  using  that  data-file.   Finally  the user
   extracts the desired output on the system from the output  files  created
   by MOPAC.

\begin{enumerate}
\item This is the ``sixth edition''. MOPAC has undergone a steady
 expansion  since its first release, and users of the earlier editions are
 recommended  to  familiarize  themselves  with  the  changes  which   are
 described  in this manual.  If any errors are found, or if MOPAC does not
 perform  as   described,   please   contact   Dr.\ James J. P. Stewart,
 Frank J. Seiler  Research  Laboratory,  U.S.  Air Force Academy, Colorado
 Springs, CO 80840--6528.
\item MOPAC runs successfully on normal \mi{CDC},  \mi{Data General},  
\mi{Gould},  and  \mi{DEC}  computers,  and  also  on the CDC 205 and
\mi{CRAY--XMP} ``supercomputers''.\index{supercomputers}  
The CRAY version has been partly 
optimized to take  advantage  of  the  CRAY  architecture.
Several versions exist for microcomputers such as the \mi{IBM PC-AT} and 
XT, Zenith, etc.
\end{enumerate}

\section{Summary of MOPAC capabilities}
\begin{enumerate}
\item MNDO, MINDO/3, AM1, and PM3 Hamiltonians.
\item Restricted  Hartree-Fock  (RHF)  and  Unrestricted  Hartree-Fock
      (UHF) methods.
\item Extensive Configuration Interaction
  \begin{enumerate}
     \item 100 configurations
     \item Singlets,  Doublets, Triplets, Quartets, Quintets, and Sextets
     \item Excited states
     \item Geometry optimizations, etc., on specified states
  \end{enumerate}
 \item Single SCF calculation
 \item Geometry optimization
 \item Gradient minimization
 \item Transition state location
 \item Reaction path coordinate calculation
 \item Force constant calculation
 \item Normal coordinate analysis
 \item Transition dipole calculation
 \item Thermodynamic properties calculation
 \item \mi{Localized orbitals}
 \item Covalent bond orders
 \item Bond analysis into sigma and pi contributions
 \item One dimensional polymer calculation
 \item Dynamic Reaction Coordinate calculation
 \item Intrinsic Reaction Coordinate calculation
\end{enumerate}

\section{Copyright status of MOPAC}
At the request of the Air Force Academy Law Department the following
notice has been placed in MOPAC.\index{MOPAC!copyright}

\begin{quote}
          Notice of Public Domain nature of MOPAC.

         ``This computer program is a work of the United States 
          Government and as such is not subject to protection by 
          copyright (17 U.S.C. \# 105.)  Any person who fraudulently 
          places a copyright notice or does any other act contrary 
          to the provisions of 17 U.S. Code 506(c) shall be subject 
          to the penalties provided therein.  This notice shall not 
          be altered or removed from this software and is to be on 
          all reproductions.''
\end{quote}

   I recommend that a user obtain a  copy  by  either  copying  it  from  an
   existing  site  or ordering an `official' copy from the Quantum Chemistry
   Program Exchange, (QCPE), Department of  Chemistry,  Indiana  University,
   Bloomington, Indiana, 47405.\index{QCPE address}  The cost\index{MOPAC!cost} 
   covers handling only.  Contact the
   Editor, Richard Counts, at (812) 855--4784 for further details.
         
\section{Porting MOPAC to other machines}
 MOPAC is written for the DIGITAL \mi{VAX} computer.  However, the program
 has  been written with the idea that it will be ported to other machines.
 After such a port has been done, the new  program  should  be  given  the
 version  number  6.10,  or,  if two or more versions are generated, 6.20,
 6.30,  etc.   To  validate  the  new  copy,  QCPE  has  a  test-suite  of
 calculations.   If  all  tests are passed, within the tolerances given in
 the tests, then the new program can be called a valid version of MOPAC 6.
 Insofar  as is practical, the mode of submission of a MOPAC job should be
 preserved, e.g.,
\begin{verbatim}   
  (prompt) MOPAC <data-set> [<queue-options>...]
\end{verbatim}
 
      Any changes which do not violate  the FORTRAN--77  conventions,  and
 which  users  believe  would  be  generally desirable, can be sent to the
 author.

\section{Relationship of AMPAC and MOPAC}
   In 1985  MOPAC~3.0  and  AMPAC~1.0  were  submitted  to  QCPE  for
   distribution.  At that time, AMPAC differed from MOPAC in that it had the
   AM1 algorithm.  Additionally, changes in some MNDO  parameters  in  AMPAC
   made  AMPAC  results  incompatable  with  MOPAC Versions 1-3.  Subsequent
   versions of MOPAC, in addition to being more highly debugged than Version
   3.0,  also  had the AM1 method.  Such versions were compatible with AMPAC
   and with versions 1--3 of MOPAC.

        In order to avoid confusion, all versions of MOPAC after 3.0 include
   journal  references  so that the user knows unambiguously which parameter
   sets were used in any given job.

        Since 1985 AMPAC and MOPAC have evolved along different  lines.   In
   MOPAC  I  have  endeavoured  to provide a highly robust program, one with
   only a few new features, but which is easily portable and  which  can  be
   relied  upon  to give precise, if not very exciting, answers.  At Austin,
   the functionality of AMPAC has been enhanced  by  the  research  work  of
   Prof. Dewar's  group\index{Dewar research group}.   
   The  new  AMPAC~2.1 thus has functionalities not
   present in MOPAC.  In  publications,  users  should  cite  not  only  the
   program name but also the version number.

        Commercial concerns have optimized both MOPAC and AMPAC for  use  on
   supercomputers.   The quality of optimization and the degree to which the
   parent algorithm has been preserved differs between MOPAC and  AMPAC  and
   also  between some machine specific versions.  Different users may prefer
   one program to the other, based on considerations such  as  speed.   Some
   modifications  of  AMPAC run faster than some modifications of MOPAC, and
   vice versa, but if these are modified versions of MOPAC 3.0 or AMPAC 1.0,
   they  represent  the  programming  prowess  of  the  companies  doing the
   conversion, and not any intrinsic difference between the two programs.

        Testing of these large algorithms is difficult,  and  several  times
   users  have  reported  bugs in MOPAC or AMPAC which were introduced after
   they were supplied by QCPE.

                       
\subsubsection{Cooperative Development of MOPAC}
   MOPAC\index{MOPAC!development} has developed, and 
   hopefully will continue to develop, by  the
   addition  of  contributed  code.  As a policy, any supplied code which is
   incorporated into MOPAC will be described in  the  next  release  of  the
   Manual,  and  the  author  or  supplier  acknowledged.   In the following
   release only journal references will be retained.  The  objective  is  to
   produce  a good program.  This is obviously not a one-person undertaking;
   if it was, then the product would be poor indeed.  Instead, as we are  in
   a  time  of rapid change in computational chemistry, a time characterized
   by a very free exchange of ideas and code, MOPAC  has  been  evolving  by
   accretion.   The  unstinting and generous donation of intellectual effort
   speaks highly of the donors.  However, with the  rapid  commercialization
   of  computational  chemistry  software  in  the  past  few  years,  it is
   unfortunate but it seems unlikely that this idyllic state will continue.
                                                
\section{Programs recommended for use with MOPAC}
   MOPAC is the core program of a series of  programs  for  the  theoretical
   study  of  chemical  phenomena.  This version is the sixth in an on-going
   development,  and  efforts  are  being  made  to  continue  its   further
   evolution.  In order to make using MOPAC easier, five other programs have
   also been written.  Users of  MOPAC  are  recommended  to  use  all  four
   programs.   Efforts  will  be  made  to continue the development of these
   programs.
  
\subsubsection{HELP}
  HELP is a stand-alone program which mimics the  \mi{VAX}  HELP  function.
  It is intended for users on UNIX computers. \mi{HELP} comes with the basic
  MOPAC 6.00, and is recommended for general use.

\subsubsection{DRAW}
   DRAW\index{DRAW program}, written by Maj.\ Donn Storch, USAF, 
   and available through QCPE,
   is  a  powerful  editing  program  specifically written to interface with
   MOPAC.  Among the various facilities it offers are:
\begin{enumerate}
\item The on-line editing and analysis of a data file,  starting  from
scratch  or from an existing data file, an archive file, or from
a results file.
\item The option of continuous graphical representation of the  system
being  studied.   Several  types  of  terminals  are  supported,
including DIGITAL, TEKTRONIX, and TERAK terminals.
\item The drawing  of  electron  density  contour  maps  generated  by
DENSITY\index{Program!DENSITY} on graphical devices.
\item The drawing of solid-state band structures generated by MOSOL.
\item The sketching of molecular vibrations,  generated  by  a  normal
coordinate analysis.
\end{enumerate}

\subsubsection{DENSITY}
   \mi{DENSITY}, written by Dr.\ James J. P. Stewart, and available  through
   QCPE,  is  an  electron-density  plotting program.  It accepts data-files
   directly from MOPAC, and  is  intended  to  be  used  for  the  graphical
   representation  of  electron density distribution, individual M.O.'s, and
   difference maps.
        
\subsubsection{MOHELP}
   \mi{MOHELP}, also available through QCPE, is an  on-line  help  facility,
   written by Maj.\ Donn Storch and Dr.\ James J. P. Stewart, to allow non-VAX
   users access to the VAX HELP libraries for MOPAC, DRAW, 
   and DENSITY.\index{Program!MOHELP}
        
\subsubsection{MOSOL}
   MOSOL\index{Program!MOSOL} (Distributed by QCPE)  is  
   a  full  solid-state  MNDO  program
   written  by  Dr.\ James J. P. Stewart.  In comparison with MOPAC, MOSOL is
   extremely  slow.   As  a  result,  while  geometry  optimization,   force
   constants,  and  other  functions can be carried out by MOSOL, these slow
   calculations are best done using the solid-state facility  within  MOPAC.
   \mi{MOSOL} should be used for two or three dimensional solids only, a task
   that MOPAC cannot perform.

\section{The data-file}
This section is aimed at the complete novice --- someone  who  knows
nothing at all about the structure of a MOPAC data-file.

   First of all, there are at most four possible  types  of  data-files
   for  MOPAC, but the simplest data-file is the most commonly used.  Rather
   than define it, two examples are shown  below.   An  explanation  of  the
   geometry  definitions  shown  in  the  examples  is  given in the chapter
   ``GEOMETRY SPECIFICATION''.

\subsection{Example of data for ethylene}
\index{data!for ethene}
\begin{verbatim}
     Line   1 :     UHF PULAY MINDO3 VECTORS DENSITY LOCAL T=300 
     Line   2 :      EXAMPLE OF DATA FOR MOPAC
     Line   3 :        MINDO/3 UHF CLOSED-SHELL D2D ETHYLENE
     Line   4a:    C 
     Line   4b:    C    1.400118  1 
     Line   4c:    H    1.098326  1  123.572063  1 
     Line   4d:    H    1.098326  1  123.572063  1  180.000000  0   2  1  3
     Line   4e:    H    1.098326  1  123.572063  1   90.000000  0   1  2  3
     Line   4f:    H    1.098326  1  123.572063  1  270.000000  0   1  2  3
     Line   5 : 
\end{verbatim}
        
As can be seen, the first three lines are textual.  The  first  line
consists  of keywords (here seven keywords are shown).  These control the
calculation.  The next two lines are comments or titles.  The user  might
want to put the name of the molecule and why it is being run on these two
lines.

      These three lines are obligatory.  If no name or comment is  wanted,
 leave  blank lines.  If no keywords are specified, leave a blank line.  A
 common error is to have a blank line before the keyword line:  this error
 is  quite tricky to find, so be careful not to have four lines before the
 start of the geometric data (lines 4a-4f in the  example).   Whatever  is
 decided, the three lines, blank or otherwise, are obligatory.
                              
    In the example given, one line of keywords and two of  documentation
 are shown.  By use of keywords, these defaults can be changed.  Modifying
 keywords are +, \&, and SETUP.  These are defined in the KEYWORDS chapter.
 The following table illustrates the allowed combinations:
\begin{verbatim} 
  Line 1      Line 2     Line 3  Line 4   Line 5  Setup used
 
  Keys        Text       Text   Z-matrix Z-matrix  not used 
  Keys +      Keys       Text    Text    Z-matrix  not used
  Keys +      Keys +     Keys    Text    Text      not used
  Keys &      Keys       Text   Z-matrix Z-matrix  not used
  Keys &      Keys &     Keys   Z-matrix Z-matrix  not used
  Keys SETUP  Text       Text   Z-matrix Z-matrix 1 or 2 lines used
  Keys +      Keys SETUP Text    Text    Z-matrix 1 line used
  Keys &      Keys SETUP Text   Z-matrix Z-matrix 1 line used
 \end{verbatim}

 No other combinations are allowed.

  The proposed use of the SETUP option is to allow  a  frequently
  used  set  of  keywords  to  be  defined  by  a single keyword.  For
  example, if the default  criteria  are  not  suitable,  SETUP  might
  contain:
\begin{verbatim}
       " SCFCRT=1.D-8  SHIFT=30 ITRY=600 GNORM=0.02 ANALYT "
       "                                                   "
\end{verbatim}
The order of usage of a keyword is:
\begin{verbatim}
Line 1 > Line 2 > Line 3.  
Line 1 > SETUP.  
Line 2 > SETUP.  
SETUP > built in default values.
\end{verbatim}

           The next set of lines defines the geometry.   In  the  example,
      the numbers are all neatly lined up; this is not necessary, but does
      make it easier when looking for errors in the data.  The geometry is
        defined  in  lines 4a to 4f; line 5 terminates both the geometry and
        the data-file.  Any additional  data,  for  example  symmetry  data,
        would follow line 5.

Summarizing, then, the structure for a MOPAC data-file is:
\begin{description}
\item[Line 1]  Keywords\index{keywords!specification}. 
               (See chapter 2 on definitions of keywords)
\item[Line 2]  Title of the calculation, e.g. the name of the molecule or ion.
\item[Line 3]  Other information describing the calculation.
\item[Lines 4] Internal or cartesian coordinates (See chapter on 
               specification of geometry)
\item[Line 5]  Blank line to terminate the geometry definition.
\end{description}

Other layouts for data-files involve additions  to  the  simple
layout\index{data!layout}. These  additions  occur at the end 
of the data-file, after line 5.  The three most common additions are:
\begin{itemize}
\item Symmetry data:  This follows the  geometric  data,  and  is
        ended by a blank line.
\item Reaction path:  After all geometry and  symmetry  data  (if
        any) are read in, points on the reaction coordinate are defined.
\item Saddle data:  A complete second  geometry  is  input.   The
        second  geometry  follows  the  first geometry and symmetry data (if
        any).
\end{itemize}

\subsection{Example of data for polytetrahydrofuran}
The following example illustrates the data file for a four hour
 polytetrahydrofuran\index{data!for polytetrahydrofuran} calculation.
\index{polymers!data for}
As you can see the layout of the
 data is almost the same as that for a molecule, the main  difference
 is in the presence of the translation vector atom ``Tv''.
\begin{verbatim}         
  Line 1 :T=4H
  Line 2 :      POLY-TETRAHYDROFURAN (C4 H8 O)2
  Line 3 : 
  Line 4a:   C    0.000000  0    0.000000  0    0.000000  0   0  0  0
  Line 4b:   C    1.551261  1    0.000000  0    0.000000  0   1  0  0
  Line 4c:   O    1.401861  1  108.919034  1    0.000000  0   2  1  0
  Line 4d:   C    1.401958  1  119.302489  1 -179.392581  1   3  2  1
  Line 4e:   C    1.551074  1  108.956238  1  179.014664  1   4  3  2
  Line 4f:   C    1.541928  1  113.074843  1  179.724877  1   5  4  3
  Line 4g:   C    1.551502  1  113.039652  1  179.525806  1   6  5  4
  Line 4h:   O    1.402677  1  108.663575  1  179.855864  1   7  6  5
  Line 4i:   C    1.402671  1  119.250433  1 -179.637345  1   8  7  6
  Line 4j:   C    1.552020  1  108.665746  1 -179.161900  1   9  8  7
  Line 4k:  XX    1.552507  1  112.659354  1 -178.914985  1  10  9  8
  Line 4l:  XX    1.547723  1  113.375266  1 -179.924995  1  11 10  9
  Line 4m:   H    1.114250  1   89.824605  1  126.911018  1   1  3  2
  Line 4n:   H    1.114708  1   89.909148  1 -126.650667  1   1  3  2
  Line 4o:   H    1.123297  1   93.602831  1  127.182594  1   2  4  3
  Line 4p:   H    1.123640  1   93.853406  1 -126.320187  1   2  4  3
  Line 4q:   H    1.123549  1   90.682924  1  126.763659  1   4  6  5
  Line 4r:   H    1.123417  1   90.679889  1 -127.033695  1   4  6  5
  Line 4s:   H    1.114352  1   90.239157  1  126.447043  1   5  7  6
  Line 4t:   H    1.114462  1   89.842852  1 -127.140168  1   5  7  6
  Line 4u:   H    1.114340  1   89.831790  1  126.653999  1   6  8  7
  Line 4v:   H    1.114433  1   89.753913  1 -126.926618  1   6  8  7
  Line 4w:   H    1.123126  1   93.644744  1  127.030541  1   7  9  8
  Line 4x:   H    1.123225  1   93.880969  1 -126.380511  1   7  9  8
  Line 4y:   H    1.123328  1   90.261019  1  127.815464  1   9 11 10
  Line 4z:   H    1.123227  1   91.051403  1 -125.914234  1   9 11 10
  Line 4A:   H    1.113970  1   90.374545  1  126.799259  1  10 12 11
  Line 4B:   H    1.114347  1   90.255788  1 -126.709810  1  10 12 11
  Line 4C:  Tv   12.299490  1    0.000000  0    0.000000  0   1 11 10
  Line 5 :   0    0.000000  0    0.000000  0    0.000000  0   0  0  0
\end{verbatim}

 Polytetrahydrofuran has a repeat unit of $\chem (C_4H_8O)_2$; i.e.,
 twice  the  monomer  unit.   This is necessary in order to allow the
 lattice to repeat after a translation through $12.3$~\AA.   See
 the section on Solid State Capability for further details.

      Note the two dummy atoms on lines 4k and 4l.  These are useful,
 but not essential, for defining the geometry.  The atoms on lines 4y
 to 4B use these dummy atoms, as does the translation vector on  line
 4C.    The  translation  vector  has  only  the  length  marked  for
 optimization.   The  reason  for  this  is  also  explained  in  the
 Background chapter.
%%%%%%%%%%%%%%%%%%%%%%%%%%%%%%%%%%%%%%%%%%%%%%%%%%%%%%%%%%%%%%%%%%%%%%%%%%%
% Chapter 2
\chapter{Keywords}
\section{Specification of keywords}
All control data are entered in the form of keywords, which form the
 first  line  of  a data-file.  A description of what each keyword does is
 given in Section~\ref{defkey}. The order in which keywords  appear is not
 important  although they must be separated by a space.  Some keywords can
 be abbreviated, allowed abbreviations are noted in Section~\ref{defkey} 
 (for
 example 1ELECTRON can be entered as 1ELECT).  However the full keyword is
 preferred in order to  more  clearly  document  the  calculation  and  to
 obviate  the  possibility  that  an  abbreviated  keyword  might  not  be
 recognized.  If there is insufficient space in the first line for all the
 keywords  needed,  then consider abbreviating the longer words.  One type
 of keyword, those with an equal sign, such as, \verb/BAR=0.05/,  may  not  be
 abbreviated, and the full word needs to be supplied.

 Most keywords which involve an equal sign, such as \verb/SCFCRT=1.D-12/
 can,  at  the  user's discretion, be written with spaces before and after
 the  equal  sign.   Thus  all  permutations  of  \verb/SCFCRT=1.D-12/,  
 such  as \verb/SCFCRT =1.D-12/, \verb/SCFCRT = 1.D-12/, 
 \verb/SCFCRT= 1.D-12/, \verb/SCFCRT  =  1.D-12/, etc.
 are allowed.   Exceptions  to  this  are  T=,  T-PRIORITY=,  H-PRIORITY=,
 X-PRIORITY=, IRC=, DRC= and TRANS=.  ` T=' cannot be abbreviated to ` T '
 as many keywords start or end with a `T';  for  the  other  keywords  the
 associated abbreviated keywords have specific meanings.

      If two keywords which are  incompatible,  like  UHF  and  C.I.,  are
 supplied,  or  a keyword which is incompatible with the species supplied,
 for instance TRIPLET and a  methyl  radical,  then  error  trapping  will
 normally occur, and an error message will be printed.  This usually takes
 an insignificant time, so data are quickly checked for obvious errors.

\section{Full list of keywords used in MOPAC}
\begin{verbatim}
 &        - TURN NEXT LINE INTO KEYWORDS
 +        - ADD ANOTHER LINE OF KEYWORDS
 0SCF     - READ IN DATA, THEN STOP
 1ELECTRON- PRINT FINAL ONE-ELECTRON MATRIX 
 1SCF     - DO ONE SCF AND THEN STOP 
 AIDER    - READ IN AB INITIO DERIVATIVES
 AIGIN    - GEOMETRY MUST BE IN GAUSSIAN FORMAT
 AIGOUT   - IN ARC FILE, INCLUDE AB-INITIO GEOMETRY
 ANALYT   - USE ANALYTICAL DERIVATIVES OF ENERGY WRT GEOMETRY
 AM1      - USE THE AM1 HAMILTONIAN 
 BAR=n.n  - REDUCE BAR LENGTH BY A MAXIMUM OF n.n
 BIRADICAL- SYSTEM HAS TWO UNPAIRED ELECTRONS 
 BONDS    - PRINT FINAL BOND-ORDER MATRIX 
 C.I.     - A MULTI-ELECTRON CONFIGURATION INTERACTION SPECIFIED 
 CHARGE=n - CHARGE ON SYSTEM = n (e.g. NH4 => CHARGE=1)
 COMPFG   - PRINT HEAT OF FORMATION CALCULATED IN COMPFG
 CONNOLLY - USE CONNOLLY SURFACE
 DEBUG    - DEBUG OPTION TURNED ON
 DENOUT   - DENSITY MATRIX OUTPUT (CHANNEL 10)
 DENSITY  - PRINT FINAL DENSITY MATRIX 
 DEP      - GENERATE FORTRAN CODE FOR PARAMETERS FOR NEW ELEMENTS
 DEPVAR=n - TRANSLATION VECTOR IS A MULTIPLE OF BOND-LENGTH
 DERIV    - PRINT PART OF WORKING IN DERIV
 DFORCE   - FORCE CALCULATION SPECIFIED, ALSO PRINT FORCE MATRIX.
 DFP      - USE DAVIDON-FLETCHER-POWELL METHOD TO OPTIMIZE GEOMETRIES
 DIPOLE   - FIT THE ESP TO THE CALCULATED DIPOLE
 DIPX     - X COMPONENT OF DIPOLE TO BE FITTED
 DIPY     - Y COMPONENT OF DIPOLE TO BE FITTED
 DIPZ     - Z COMPONENT OF DIPOLE TO BE FITTED
 DMAX     - MAXIMUM STEPSIZE IN EIGENVECTOR FOLLOWING
 DOUBLET  - DOUBLET STATE REQUIRED
 DRC      - DYNAMIC REACTION COORDINATE CALCULATION
 DUMP=n   - WRITE RESTART FILES EVERY n SECONDS
 ECHO     - DATA ARE ECHOED BACK BEFORE CALCULATION STARTS
 EF       - USE EF ROUTINE FOR MINIMUM SEARCH
 EIGINV   -
 EIGS     - PRINT ALL EIGENVALUES IN ITER 
 ENPART   - PARTITION ENERGY INTO COMPONENTS
 ESP      - ELECTROSTATIC POTENTIAL CALCULATION
 ESPRST   - RESTART OF ELECTROSTATIC POTENTIAL
 ESR      - CALCULATE RHF UNPAIRED SPIN DENSITY 
 EXCITED  - OPTIMIZE FIRST EXCITED SINGLET STATE 
 EXTERNAL - READ PARAMETERS OFF DISK
 FILL=n   - IN RHF OPEN AND CLOSED SHELL, FORCE M.O. n 
            TO BE FILLED
 FLEPO    - PRINT DETAILS OF GEOMETRY OPTIMIZATION
 FMAT     - PRINT DETAILS OF WORKING IN FMAT
 FOCK     - PRINT LAST FOCK MATRIX 
 FORCE    - FORCE CALCULATION SPECIFIED
 GEO-OK   - OVERRIDE INTERATOMIC DISTANCE CHECK
 GNORM=n.n- EXIT WHEN GRADIENT NORM DROPS BELOW n.n
 GRADIENTS- PRINT ALL GRADIENTS 
 GRAPH    - GENERATE FILE FOR GRAPHICS
 HCORE    - PRINT DETAILS OF WORKING IN HCORE
 HESS=N   - OPTIONS FOR CALCULATING HESSIAN MATRICES IN EF
 H-PRIO   - HEAT OF FORMATION TAKES PRIORITY IN DRC
 HYPERFINE- HYPERFINE COUPLING CONSTANTS TO BE CALCULATED
 IRC      - INTRINSIC REACTION COORDINATE CALCULATION
 ISOTOPE  - FORCE MATRIX WRITTEN TO DISK (CHANNEL 9 )
 ITER     - PRINT DETAILS OF WORKING IN ITER
 ITRY=N   - SET LIMIT OF NUMBER OF SCF ITERATIONS TO N.
 IUPD     - MODE OF HESSIAN UPDATE IN EIGENVECTOR FOLLOWING
 K=(N,N)  - BRILLOUIN ZONE STRUCTURE TO BE CALCULATED
 KINETIC  - EXCESS KINETIC ENERGY ADDED TO DRC CALCULATION
 LINMIN   - PRINT DETAILS OF LINE MINIMIZATION
 LARGE    - PRINT EXPANDED OUTPUT 
 LET      - OVERRIDE CERTAIN SAFETY CHECKS
 LOCALIZE - PRINT LOCALIZED ORBITALS 
 MAX      - PRINTS MAXIMUM GRID SIZE (23*23)
 MECI     - PRINT DETAILS OF MECI CALCULATION
 MICROS   - USE SPECIFIC MICROSTATES IN THE C.I.
 MINDO/3  - USE THE MINDO/3 HAMILTONIAN 
 MMOK     - USE MOLECULAR MECHANICS CORRECTION TO CONH BONDS
 MODE=N   - IN EF, FOLLOW HESSIAN MODE NO. N
 MOLDAT   - PRINT DETAILS OF WORKING IN MOLDAT 
 MS=N     - IN MECI, MAGNETIC COMPONENT OF SPIN
 MULLIK   - PRINT THE MULLIKEN POPULATION ANALYSIS
 NLLSQ    - MINIMIZE GRADIENTS USING NLLSQ
 NOANCI   - DO NOT USE ANALYTICAL C.I. DERIVATIVES
 NODIIS   - DO NOT USE DIIS GEOMETRY OPTIMIZER
 NOINTER  - DO NOT PRINT INTERATOMIC DISTANCES 
 NOLOG    - SUPPRESS LOG FILE TRAIL, WHERE POSSIBLE
 NOMM     - DO NOT USE MOLECULAR MECHANICS CORRECTION TO CONH BONDS
 NONR     -
 NOTHIEL  - DO NOT USE THIEL'S FSTMIN TECHNIQUE
 NSURF=N  - NUMBER OF SURFACES IN AN ESP CALCULATION
 NOXYZ    - DO NOT PRINT CARTESIAN COORDINATES 
 NSURF    - NUMBER OF LAYERS USED IN ELECTROSTATIC POTENTIAL
 OLDENS   - READ INITIAL DENSITY MATRIX OFF DISK
 OLDGEO   - PREVIOUS GEOMETRY TO BE USED
 OPEN     - OPEN-SHELL RHF CALCULATION REQUESTED
 ORIDE    -
 PARASOK  - IN AM1 CALCULATIONS SOME MNDO PARAMETERS ARE TO BE USED
 PI       - RESOLVE DENSITY MATRIX INTO SIGMA AND PI BONDS
 PL       - MONITOR CONVERGENCE OF DENSITY MATRIX IN ITER
 PM3      - USE THE MNDO-PM3 HAMILTONIAN 
 POINT=N  - NUMBER OF POINTS IN REACTION PATH
 POINT1=N - NUMBER OF POINTS IN FIRST DIRECTION IN GRID CALCULATION
 POINT2=N - NUMBER OF POINTS IN SECOND DIRECTION IN GRID CALCULATION
 POLAR    - CALCULATE FIRST, SECOND AND THIRD ORDER POLARIZABILITIES
 POTWRT   - IN ESP, WRITE OUT ELECTROSTATIC POTENTIAL TO UNIT 21
 POWSQ    - PRINT DETAILS OF WORKING IN POWSQ
 PRECISE  - CRITERIA TO BE INCREASED BY 100 TIMES
 PULAY    - USE PULAY'S CONVERGER TO OBTAIN A SCF
 QUARTET  - QUARTET STATE REQUIRED
 QUINTET  - QUINTET STATE REQUIRED
 RECALC=N - IN EF, RECALCULATE HESSIAN EVERY N STEPS
 RESTART  - CALCULATION RESTARTED
 ROOT=n   - ROOT n TO BE OPTIMIZED IN A C.I. CALCULATION
 ROT=n    - THE SYMMETRY NUMBER OF THE SYSTEM IS n.
 SADDLE   - OPTIMIZE TRANSITION STATE 
 SCALE    - SCALING FACTOR FOR VAN DER WAALS DISTANCE IN ESP
 SCFCRT=n - DEFAULT SCF CRITERION REPLACED BY THE VALUE SUPPLIED
 SCINCR   - INCREMENT BETWEEN LAYERS IN ESP
 SETUP    - EXTRA KEYWORDS TO BE READ OF SETUP FILE
 SEXTET   - SEXTET STATE REQUIRED
 SHIFT=n  - A DAMPING FACTOR OF n DEFINED TO START SCF
 SIGMA    - MINIMIZE GRADIENTS USING SIGMA
 SINGLET  - SINGLET STATE REQUIRED
 SLOPE    - MULTIPLIER USED TO SCALE MNDO CHARGES
 SPIN     - PRINT FINAL UHF SPIN MATRIX 
 STEP     - STEP SIZE IN PATH
 STEP1=n  - STEP SIZE n FOR FIRST COORDINATE IN GRID CALCULATION
 STEP2=n  - STEP SIZE n FOR SECOND COORDINATE IN GRID CALCULATION
 STO-3G   - DEORTHOGONALIZE ORBITALS IN STO-3G BASIS
 SYMAVG   - AVERAGE SYMMETRY EQUIVALENT ESP CHARGES
 SYMMETRY - IMPOSE SYMMETRY CONDITIONS 
 T=n      - A TIME OF n SECONDS REQUESTED  
 THERMO   - PERFORM A THERMODYNAMICS CALCULATION 
 TIMES    - PRINT TIMES OF VARIOUS STAGES 
 T-PRIO   - TIME TAKES PRIORITY IN DRC
 TRANS    - THE SYSTEM IS A TRANSITION STATE 
            (USED IN THERMODYNAMICS CALCULATION)
 TRIPLET  - TRIPLET STATE REQUIRED
 TS       - USING EF ROUTINE FOR TS SEARCH
 UHF      - UNRESTRICTED HARTREE-FOCK CALCULATION 
 VECTORS  - PRINT FINAL EIGENVECTORS 
 VELOCITY - SUPPLY THE INITIAL VELOCITY VECTOR IN A DRC CALCULATION
 WILLIAMS - USE WILLIAMS SURFACE
 X-PRIO   - GEOMETRY CHANGES TAKE PRIORITY IN DRC
 XYZ      - DO ALL GEOMETRIC OPERATIONS IN CARTESIAN COORDINATES.
\end{verbatim}
               
\section{Definitions of keywords}
\label{defkey}
   The definitions below are  given  with  some  technical  expressions
   which are not further defined.  Interested users are referred to Appendix
   E of this manual to locate  appropriate  references  which  will  provide
   further clarification.

There are three  classes  of  keywords:
\begin{enumerate}
\item those  which  CONTROL
   substantial  aspects  of  the  calculation,  i.e., those which affect the
   final heat of formation,
\item those which determine which OUTPUT will be calculated and printed, and
\item those which dictate the WORKING of the calculation, 
but which do not affect the heat of formation. The
   assignment to one of these classes is designated by a (C), (O) or (W),
   respectively, following each keyword in the list below.
\end{enumerate}


\subsection*{\& (C)}
\index{\&} 
      An `\verb*/ &/' means `turn the next line into keywords'. Note  the  space
 before the `\&'  sign.   
 Since `\&' is a keyword, it must be preceeded by a
 space. A `\verb*/ &/' on line 1 would mean that a second line of 
 keywords should be  read  in.
 If that second line contained a `\verb*/ &/', 
 then a third line of keywords would be read in.  
 If the first line has a `\verb*/ &/' then the first
 description  line  is  omitted,  
 if the second line has a `\verb*/ &/', then both
 description lines are omitted.
 
 Examples: Use of one `\&'
\begin{verbatim} 
  VECTORS DENSITY RESTART & NLLSQ T=1H SCFCRT=1.D-8 DUMP=30M ITRY=300 
  PM3 FOCK OPEN(2,2) ROOT=3 SINGLET SHIFT=30
\end{verbatim}
  
Test on a totally weird system: Use of two `\&'s
\begin{verbatim} 
  LARGE=-10 & DRC=4.0 T=1H SCFCRT=1.D-8 DUMP=30M ITRY=300 SHIFT=30 
  PM3 OPEN(2,2) ROOT=3 SINGLET NOANCI ANALYT  T-PRIORITY=0.5 &
  LET GEO-OK VELOCITY KINETIC=5.0 
\end{verbatim} 

\subsection*{+ (C)}
\index{+} 
      A `\verb*/ +/' sign means `read another line of keywords'. Note the space
 before  the  `+' sign.  Since `+' is a keyword, it must be preceeded by a
 space. A `\verb*/ +/' on line 1 would mean that a second line of keywords should
 be  read  in.   If that second line contains a `\verb*/ +/', 
 then a third line of
 keywords will be read in.  Regardless of whether a second or a third line
 of keywords is read in, the next two lines would be description lines.
              
 Example of `\verb*/ +/' option
\begin{verbatim}
    RESTART T=4D FORCE OPEN(2,2) SHIFT=20 PM3 +
    SCFCRT=1.D-8 DEBUG + ISOTOPE FMAT ECHO singlet ROOT=3
    THERMO(300,400,1) ROT=3
\end{verbatim}

Example of data set with three lines of keywords.  Note: There 
are two lines of description, this and the previous line.

\subsection*{0SCF (O)}
 The data can be read in and output, but  no  actual  calculation  is
 performed  when  this  keyword is used.  This is useful as a check on the
 input data.   All  obvious  errors  are  trapped,  and  warning  messages
 printed.\index{0SCF}

      A second use is to convert from one format to  another.   The  input
 geometry  is printed in various formats at the end of a 0SCF calculation.
   If   NOINTER   is   absent,   cartesian    coordinates    are    printed.
   Unconditionally,  MOPAC Z-matrix internal coordinates are printed, and if
   AIGOUT is present, Gaussian Z-matrix internal  coordinates  are  printed.
   0SCF should now be used in place of DDUM.

\subsection*{1ELECTRON (O)}
The final one-electron  matrix  is  printed  out. This  matrix  is
composed  of  atomic orbitals; the array element between orbitals $i$ and $j$
on different atoms is given by:
$$ H(i,j) = 0.5 \times (\beta_i + \beta_j) \times {\rm overlap}(i,j)$$

      The matrix elements between orbitals i and j on the  same  atom  are
 calculated from the electron-nuclear attraction energy, and also from the
 $U(i)$ value if $i=j$.

      The one-electron matrix is unaffected by (a) the charge and (b)  the
 electron  density.  It is only a function of the geometry.  Abbreviation:
 1ELEC.\index{1ELECTRON}
                                
\subsection*{1SCF (C)}
       When users want to examine the results of a single  SCF  calculation
 of a geometry, 1SCF should be used.  1SCF can be used in conjunction with
 RESTART, in which case a single SCF calculation will  be  done,  and  the
 results printed.
 
      When 1SCF is used on its own (that is, RESTART  is  not  also  used)
 then derivatives will only be calculated if GRAD is also specified.

      1SCF is helpful in a learning situation.   MOPAC  normally  performs
 many SCF calculations, and in order to minimize output when following the
 working of the SCF calculation, 1SCF is very useful.\index{1SCF}

\subsection*{AIDER (C)}
 \mi{AIDER} allows MOPAC to optimize an ab-initio geometry.   To  use  it,
 calculate  the  ab-initio  gradients using, e.g., Gaussian.  Supply MOPAC
 with these gradients, after converting them into kcal/mol.  The  geometry
 resulting  from  a  MOPAC  run  will be nearer to the optimized ab-initio
 geometry than if the geometry optimizer in Gaussian had been used.
 
 
\subsection*{AIGIN (C)}
      If the geometry (Z-matrix) is specified using the Gaussian-8X,  then
 normally  this  will be read in without difficulty.  In the event that it
 is mistaken for a  normal  MOPAC-type  Z-matrix,  the  keyword  AIGIN  is
 provided. \mi{AIGIN} will force the data-set to be read in assuming Gaussian
 format.  This is necessary if more than one system is  being  studied  in
 one run.

\subsection*{AIGOUT (O)}
     The ARCHIVE file contains a  data-set  suitable  for  submission  to
MOPAC.  If, in addition to this data-set, the Z-matrix for Gaussian input
is wanted, then \mi{AIGOUT} (ab initio geometry output), should be used.

     The Z-matrix is in full Gaussian  form.   Symmetry,  where  present,
will  be correctly defined.  Names of symbolics will be those used if the
original geometry was in Gaussian format, otherwise `logical' names  will
be  used.  Logical names are of form \verb/<t><a><b>[<c>][<d>]/ 
where \verb/<t>/ is `r'
for bond length, `a' for angle, or `d' for  dihedral, \verb/<a>/ is the atom
number, \verb/<b>/ is the atom to which \verb/<a>/ is related, \verb/<c>/, 
if present, is the atom number to which \verb/<a>/ makes an angle, 
and \verb/<d>/, if present, is the atom
number to which \verb/<a>/ makes a dihedral.

\subsection*{ANALYT (W)}
      By default, finite difference derivatives of energy with respect  to
 geometry  are  used.  If \mi{ANALYT} is specified, then analytical derivatives
 are used instead.  Since the analytical  derivatives  are  over  Gaussian
 functions---a  STO-6G  basis set is used---the overlaps are also over
 Gaussian functions.  This will result in a  very  small  (less  than  0.1
 kcal/mole)  change  in heat of formation.  Use analytical derivatives (a)
 when the mantissa used is  less  than  about 51--53  bits,  or  (b)  when
 comparison   with   finite  difference  is  desired.   Finite  difference
 derivatives are still used when non-variationally optimized wavefunctions
 are present.


\subsection*{AM1 (C)}
The \mi{AM1} method is to be used.  By default MNDO is run.

\subsection*{BAR=n.nn (W)}
In the SADDLE calculation the distance between the two geometries is
   steadily  reduced  until  the  transition  state  is located.  Sometimes,
   however, the user may want  to  alter  the  maximum  rate  at  which  the
   distance  between  the  two geometries reduces. \mi{BAR} is a ratio, normally
   0.15, or 15 percent.  This represents a maximum rate of reduction of  the
   bar  of 15 percent per step.  Alternative values that might be considered
   are BAR=0.05 or BAR=0.10, although other values may be  used.   See  also
   SADDLE.

   If CPU time is not a major consideration, use BAR=0.03.


                                 
\subsection*{BIRADICAL (C)}
   Note: \mi{BIRADICAL} is a redundant keyword, and represents a particular
   configuration  interaction calculation.  Experienced users of MECI (q.v.)
   can duplicate the effect of the  keyword  BIRADICAL  by  using  the  MECI
   keywords OPEN(2,2) and SINGLET.

        For molecules which are believed to have biradicaloid character  the
   option  exists  to optimize the lowest singlet energy state which results
   from the mixing of three states.  These states are,  in  order,  (1)  the
   (micro)state  arising from a one electron excitation from the HOMO to the
   LUMO,  which  is  combined  with  the  microstate  resulting   from   the
   time-reversal  operator acting on the parent microstate, the result being
   a full singlet state; (2) the state resulting from de-excitation from the
   formal  LUMO  to  the  HOMO;  and (3) the state resulting from the single
   electron in the formal HOMO being excited into the LUMO.
\begin{verbatim}
                       Microstate 1          Microstate 2      Microstate 3 


                  Alpha Beta   Alpha Beta    Alpha  Beta       Alpha  Beta


      LUMO         *                 *                           *    *
                  ---  ---     ---  ---       ---  ---          ---  ---


                             +


      HOMO              *       *              *    * 
                  ---  ---     ---  ---       ---  ---          ---  ---
\end{verbatim}

   A configuration interaction calculation is involved  here.   A  biradical
   calculation  done  without  C.I. at  the  RHF level would be meaningless.
   Either rotational invariance would  be  lost,  as  in  the  D2d  form  of
   ethylene,  or  very artificial barriers to rotations would be found, such
   as in a methane molecule ``orbiting'' a D2d ethylene.  In  both  cases  the
   inclusion  of  limited  configuration  interaction  corrects  the  error.
   BIRADICAL should not be used if either the HOMO or LUMO is degenerate; in
   this case, the full manifold of HOMO $\times$ LUMO should be included in the
   C.I., using MECI options.  The user should be aware  of  this  situation.
   When  the  biradical  calculation  is  performed correctly, the result is
   normally a net stabilization.  However,  if  the  first  singlet  excited
   state  is  much  higher  in  energy  than  the closed-shell ground state,
   BIRADICAL can lead to a destabilization.  Abbreviation:  BIRAD.  See also
   MECI, C.I., OPEN, SINGLET.

\subsection*{BONDS (O)}
   The rotationally invariant bond order between all pairs of atoms  is
   printed.   In this context a bond is defined as the sum of the squares of
   the density matrix  elements  connecting  any  two  atoms.   For  ethane,
   ethylene,  and  acetylene the carbon-carbon bond orders are roughly 1.00,
   2.00, and 3.00  respectively.   The  diagonal  terms  are  the  valencies
   calculated  from  the atomic terms only and are defined as the sum of the
   bonds the atom makes with other  atoms.   In  UHF  and  non-variationally
   optimized  wavefunctions  the  calculated  valency will be incorrect, the
   degree of error  being  proportional  to  the  non-duodempotency  of  the
   density matrix.  For an RHF wavefunction the square of the density matrix
   is equal to twice the density matrix.\index{BONDS}

        The bonding contributions of all M.O.'s in the  system  are  printed
   immediately  before  the  bonds  matrix.   The  idea of molecular orbital
   valency was developed by Gopinathan, Siddarth, and Ravimohan.  Just as an
   atomic  orbital  has a `valency', so has a molecular orbital.  This leads
   to the following relations:  The sum of the bonding contributions of  all
   occupied M.O.'s is the same as the sum of all valencies which, in turn is
   equal to two times the  sum  of  all  bonds.   The  sum  of  the  bonding
   contributions of all M.O.'s is zero.

\subsection*{C.I.=n (C)}
   Normally configuration interaction is invoked if any of the keywords
   which imply a \mi{C.I.} calculation are used, such as BIRADICAL, TRIPLET or
   QUARTET.  Note that ROOT= does not imply a  C.I. calculation:   ROOT=  is
   only  used  when  a  C.I. calculation is done.  However, as these implied
   C.I.'s involve the minimum number of configurations practical,  the  user
   may  want to define a larger than minimum C.I., in which case the keyword
   C.I.=n can be used.   When  C.I.=n  is  specified,  the  n  M.O.'s  which
   `bracket' the occupied- virtual energy levels will be used.  Thus, C.I.=2
   will include both the HOMO  and  the  LUMO,  while  C.I.=1  (implied  for
   odd-electron  systems)  will  only include the HOMO (This will do nothing
   for a closed-shell system, and leads to Dewar's half-electron  correction
   for  odd-electron  systems).  Users should be aware of the rapid increase
   in the size of the C.I. with increasing numbers  of  M.O.'s  being  used.
   Numbers  of  microstates  implied by the use of the keyword C.I.=n on its
   own are as follows:                    
\begin{verbatim}
    Keyword        Even-electron systems           Odd-electron systems
                No. of electrons, configs       No. of electrons, configs
                Alpha   Beta                    Alpha Beta
    
     C.I.=1       1      1          1            1     0             1
     C.I.=2       1      1          4            1     0             2
     C.I.=3       2      2          9            2     1             9
     C.I.=4       2      2         36            2     1            24
     C.I.=5       3      3        100            3     2           100
     C.I.=6       3      3        400            3     2           300
     C.I.=7       4      4       1225            4     3          1225
     C.I.=8   (Do not use unless other keywords also used, see below)
\end{verbatim}
    
    If a change of spin is defined, then larger numbers of M.O.'s can be
used  up  to a maximum of 10.  The C.I. matrix is of size 100 x 100.  For
calculations involving up to  100  configurations,  the  spin-states  are
exact  eigenstates of the spin operators.  For systems with more than 100
configurations, the 100 configurations of lowest energy  are  used.   See
also MICROS and the keywords defining spin-states.

    Note that for any system, use of C.I.=5 or higher  normally  implies
the  diagonalization  of a 100 by 100 matrix.  As a geometry optimization
using a C.I. requires the derivatives to be calculated using  derivatives
of  the C.I. matrix, geometry optimization with large C.I.'s will require
more time than smaller C.I.'s.

Associated keywords:  MECI, ROOT=, MICROS, SINGLET, DOUBLET, etc.
                                          
\subsection*{C.I.=(n,m)}
    In addition to specifying the number of M.O.'s in the active  space,
the  number  of  electrons  can also be defined.  In C.I.=(n,m), n is the
number of M.O.s in the active space, and m is the number of doubly filled
levels to be used.
Examples:
\begin{verbatim}
  Keywords           Number of M.O.s  No. Electrons

  C.I.=2                   2             2 (1)
  C.I.=(2,1)               2             2 (3)
  C.I.=(3,1)               3             2 (3)
  C.I.=(3,2)               3             4 (5)
  C.I.=(3,0) OPEN(2,3)     3             2 (N/A)
  C.I.=(3,1) OPEN(2,2)     3             4 (N/A)
  C.I.=(3,1) OPEN(1,2)     3           N/A (3)
\end{verbatim}
Odd electron systems given in parentheses.

\subsection*{CHARGE=n (C)}
 When the system being studied is an ion, the charge, $n$, on  the  ion
 must be supplied by \verb/CHARGE=n/.  For cations $n$ can be 1, 2, 3, etc, 
 for anions $-1$ or $-2$ or $-3$, etc.\index{CHARGE}
 Examples:
\begin{verbatim}
      ION               KEYWORD              ION          KEYWORD

      NH4(+)           CHARGE=1             CH3COO(-)      CHARGE=-1
      C2H5(+)          CHARGE=1             (COO)(=)       CHARGE=-2
      SO4(=)           CHARGE=-2            PO4(3-)        CHARGE=-3
      HSO4(-)          CHARGE=-1            H2PO4(-)       CHARGE=-1
\end{verbatim}

\subsection*{DCART (O)}
 The  cartesian  derivatives  which  are  calculated  in \mi{DCART}   for
 variationally  optimized  systems  are  printed  if  the keyword DCART is
 present.  The  derivatives  are  in  units  of  kcals/Angstrom,  and  the
 coordinates are displacements in x, y, and z.


                                   
\subsection*{DEBUG (O)}
        Certain keywords have specific  output  control  meanings,  such  as
   FOCK,  VECTORS  and  DENSITY.  If they are used, only the final arrays of
   the relevant type are printed.  If DEBUG is supplied, then all arrays are
   printed.   This is useful in debugging ITER.  
   \mi{DEBUG} can also increase the
   amount of output produced when certain output  keywords  are  used,  e.g.
   COMPFG.


                                  
\subsection*{DENOUT (O)}
        The density matrix at the end of the calculation is to be output  in
   a  form  suitable  for input in another job.  If an automatic dump due to
   the time being exceeded occurs during the  current  run  then  DENOUT  is
   invoked automatically.  (see RESTART)\index{DENOUT}


                                  
\subsection*{DENSITY (O)}
        At the end of a job, when the results are being printed, the density
   matrix  is  also  printed.  For RHF the normal density matrix is printed.
   For UHF the sum of the alpha and beta density matrices is printed.

        If density is not  requested,  then  the  diagonal  of  the  density
   matrix,  i.e.,  the  electron  density  on  the  atomic orbitals, will be
   printed.\index{DENSITY (0)}


\subsection*{DEP (O)}
      For use only with EXTERNAL=.  When  new  parameters  are  published,
 they  can  be  entered  at  run-time  by  using EXTERNAL=, but as this is
 somewhat clumsy, a permanent change can be made by use of DEP.

      If DEP is  invoked,  a  complete  block  of  FORTRAN  code  will  be
 generated, and this can be inserted directly into the BLOCK DATA file.
 
      Note that the output is designed for use with PM3.  By modifying the
 names, the output can be used with MNDO or AM1.

\subsection*{DEPVAR=n.nn (C)}
      In polymers the translation vector is frequently a multiple of  some
 internal  distance.   For example, in polythene it is the C1--C3 distance.
 If a cluster unit cell of C6H12 is used, then symmetry can be used to tie
 together  all  the  carbon  atom  coordinates  and the translation vector
 distance.  In this example DEPVAR=3.0 would be suitable.

\subsection*{DFP (W)}
By default the Broyden--Fletcher--Goldfarb--Shanno method will be  used
 to  optimize geometries.  The older Davidon--Fletcher--Powell method can be
 invoked by specifying DFP.  This is intended to be used for comparison of
 the two methods.

\subsection*{DIPOLE (C)}
 Used in the ESP calculation, DIPOLE will  constrain  the  calculated
 charges  to  reproduce  the cartesian dipole moment components calculated
 from the density matrix and nuclear charges.
 
\subsection*{DIPX (C)}
 Similar to DIPOLE, except the fit will be for the X-component only.
 
\subsection*{DIPY (C)}
Similar to DIPOLE, except the fit will be for the Y-component only.
 
\subsection*{DIPZ (C)}
Similar to DIPOLE, except the fit will be for the Z-component only.

\subsection*{DMAX=n.nn (W)}
In the EF routine,  the  maximum  step-size  is  $0.2$  (Angstroms  or
 radians),  by  default.   This  can  be  changed by specifying DMAX=n.nn.
 Increasing DMAX can lead to faster convergence  but  can  also  make  the
 optimization  go  bad  very  fast.  Furthermore, the Hessian updating may
 deteriorate when using large stepsizes.  Reducing the stepsize to $0.10$ or
 $0.05$ is recommended when encountering convergence problems.

\subsection*{DOUBLET (C)}
   When a configuration  interaction  calculation  is  done,  all  spin
   states are calculated simultaneously, either for component of spin=0 or
   1/2.  When only doublet states are  of  interest,  then  DOUBLET  can  be
   specified,  and  all  other spin states, while calculated, are ignored in
   the choice of root to be used.

        Note that while almost every odd-electron system will have a doublet
   ground state, DOUBLET should still be specified if the desired state must
   be a doublet.

        DOUBLET has no meaning in a UHF calculation.

\subsection*{DRC (C)}
        A Dynamic Reaction Coordinate calculation is to be run.  By default,
   total  energy  is  conserved, so that as the `reaction' proceeds in time,
   energy is transferred between kinetic and potential forms.

\subsection*{DRC=n.nnn (C)}
        In a DRC calculation, the `half-life' for loss of kinetic energy  is
   defined as n.nnn femtoseconds.  If n.nnn is set to zero, infinite damping
   simulating a very condensed phase is obtained.

        This keyword cannot be written with spaces around the `=' sign.

\subsection*{DUMP (W)}
 Restart files  are  written  automatically  at  one  hour  cpu  time
 intervals  to  allow  a long job to be restarted if the job is terminated
 catastrophically.  To change  the  frequency  of  dump,  set  DUMP=nn  to
 request  a dump every nn seconds.  Alternative forms, DUMP=nnM, DUMP=nnH,
 DUMP=nnD for a dump every nn minutes, hours, or days, respectively.  DUMP
 only  works  with geometry optimization, gradient minimization, path, and
 FORCE calculations.  It does not (yet) work with a SADDLE calculation.
                                       

\subsection*{ECHO (O)}
Data are echoed back if ECHO is specified.  Only useful if data  are
suspected to be corrupt.

\subsection*{EF (C)}
      The Eigenvector Following routine is an alternative to the BFGS, and
 appears  to be much faster.  To invoke the Eigenvector Following routine,
 specify EF.  EF is particularly good in the end-game, when  the  gradient
 is small.  See also HESS, DMAX, EIGINV.
 
\subsection*{EIGINV (W)} 
Not recommended for normal use.  Used  with  the  EF  routine.   See
source code for more details.

\subsection*{ENPART (O)}
 This is a very useful tool for analyzing the energy terms  within  a
 system.   The  total  energy,  in  eV,  obtained  by  the addition of the
 electronic and nuclear terms, is partitioned into  mono-  and  bi-centric
 contributions,  and  these contributions in turn are divided into nuclear
 and one- and two-electron terms.


                                  
\subsection*{ESP (C)} 
 This is the ElectroStatic Potential calculation  of  K. M. Merz  and
 B. H. Besler.  ESP calculates the expectation values of the electrostatic
 potential of a  molecule  on  a  uniform  distribution  of  points.   The
 resultant  ESP  surface is then fitted to atom centered charges that best
 reproduce the distribution, in a least squares sense.
 
 
                                
\subsection*{ESPRST (W)}
 ESPRST restarts a stopped ESP calculation.  Do not use with RESTART.

\subsection*{ESR (O)}
   The unpaired spin density arising from an odd-electron system can be
   calculated  both  RHF  and  UHF.  In a UHF calculation the alpha and beta
   M.O.'s have  different  spatial  forms,  so  unpaired  spin  density  can
   naturally  be  present  on in-plane hydrogen atoms such as in the phenoxy
   radical.

        In the RHF formalism  a  MECI  calculation  is  performed.   If  the
   keywords  OPEN  and  C.I.=  are  both  absent then only a single state is
   calculated.  The unpaired spin density is then calculated from the  state
   function.   In  order  to have unpaired spin density on the hydrogens in,
   for example, the phenoxy radical, several states should be mixed.

\subsection*{EXCITED (C)}
        The state to be calculated is the first excited  open-shell  singlet
   state.   If the ground state is a singlet, then the state calculated will
   be S(1); if the ground state is a triplet, then S(2).  This  state  would
   normally  be  the state resulting from a one-electron excitation from the
   HOMO to the LUMO.  Exceptions would be if the lowest singlet state were a
   biradical, in which case the EXCITED state could be a closed shell.

        The EXCITED state will be calculated from a BIRADICAL calculation in
   which  the  second  root  of  the C.I. matrix is selected.  Note that the
   eigenvector of the C.I. matrix is not  used  in  the  current  formalism.
   Abbreviation:  EXCI.

        Note:  EXCITED is a redundant keyword, and represents  a  particular
   configuration  interaction  calculation.   Experienced  users of MECI can
   duplicate the effect of the keyword EXCITED by using  the  MECI  keywords
   OPEN(2,2), SINGLET, and ROOT=2.
                               
\subsection*{EXTERNAL=name (C)}
       Normally, PM3, AM1 and MNDO parameters are taken from the BLOCK DATA
  files within MOPAC.  When the supplied parameters are not suitable, as in
   an element recently  parameterized,  and  the  parameters  have  not  yet
   installed  in  the  user's  copy of MOPAC, then the new parameters can be
   inserted at run time by use of EXTERNAL=\verb/<filename>/, where  
   \verb/<filename>/ is the name of the file which contains the new parameters.

   \verb/<filename>/ consists of a series  of  parameter  definitions  in  the
   format:
\begin{verbatim}
<Parameter> <Element> <Value of parameter>
\end{verbatim}
   where the possible parameters are USS, UPP, UDD, ZS, ZP, ZD,  BETAS,
   BETAP, BETAD, GSS, GSP, GPP, GP2, HSP, ALP, FNnm, n=1,2, or 3, and m=1 to
   10, and the elements are defined by their chemical symbols, such as Si or
   SI.                                   

        When new parameters for elements are published, they can be typed in
   as  shown.   This file is ended by a blank line, the word END or nothing,
   i.e., no end-of-file delimiter.  An example  of  a  parameter  data  file
   would be (put at least 2 spaces before and after parameter name):
\begin{verbatim}
  Line  1:     USS      Si      -34.08201495
  Line  2:     UPP      Si      -28.03211675
  Line  3:     BETAS    Si       -5.01104521
  Line  4:     BETAP    Si       -2.23153969
  Line  5:     ZS       Si        1.28184511
  Line  6:     ZP       Si        1.84073175
  Line  7:     ALP      Si        2.18688712
  Line  8:     GSS      Si        9.82
  Line  9:     GPP      Si        7.31
  Line 10:     GSP      Si        8.36
  Line 11:     GP2      Si        6.54
  Line 12:     HSP      Si        1.32
\end{verbatim}

        Derived parameters do no need to be entered; they will be calculated
   from  the optimized parameters.  All "constants" such as the experimental
   heat of atomization are already inserted for all elements.

        NOTE:  EXTERNAL can only be used to input parameters for MNDO,  AM1,
   or  PM3.   It is unlikely, however, that any more MINDO/3 parameters will
   be published.

        See also DEP to make a permanent change.


                                  
\subsection*{FILL=n (C)}
        The n'th M.O.  in an RHF calculation is constrained  to  be  filled.
   It  has no effect on a UHF calculation.  After the first iteration (NOTE:
   not after the first SCF calculation, but after the first iteration within
   the first SCF calculation) the n'th M.O.  is stored, and, if occupied, no
   further action is taken at that time.  If unoccupied, then the  HOMO  and
   the n'th M.O.'s are swapped around, so that the n'th M.O.  is now filled.
   On all subsequent iterations the M.O.  nearest in character to the stored
   M.O.   is forced to be filled, and the stored M.O.  replaced by that M.O.
   This is necessitated by the fact that in a  reaction  a  particular  M.O.
   may change its character considerably.  A useful procedure is to run 1SCF
   and DENOUT first, in order to identify the M.O.'s; the  complete  job  is
   then  run  with OLDENS and FILL=nn, so that the eigenvectors at the first
   iteration are fully known.  As FILL is known to give difficulty at times,
   consider also using C.I.=n and ROOT=m.


                                   
\subsection*{FLEPO (O)}
        The predicted and actual changes in the geometry,  the  derivatives,
   and  search  direction  for each geometry optimization cycle are printed.
   This is useful if there is any question regarding the efficiency  of  the
   geometry optimizer.

                                     
\subsection*{FMAT}
        Details of the construction of the  Hessian  matrix  for  the  force
   calculation are to be printed.


                                   
\subsection*{FORCE (C)}
        A force-calculation is to be run.  The Hessian, that is  the  matrix
   (in  millidynes  per  Angstrom)  of second derivatives of the energy with
   respect to displacements of all pairs of atoms in x, y, and z directions,
   is calculated.  On diagonalization this gives the force constants for the
   molecule.  The force matrix, weighted for isotopic masses, is  then  used
   for   calculating   the  vibrational  frequencies.   The  system  can  be
   characterized as a ground state or a transition state by the presence  of
   five  (for a linear system) or six eigenvalues which are very small (less
   than about 30 reciprocal centimeters).  A  transition  state  is  further
   characterized by one, and exactly one, negative force constant.

        A FORCE calculation is a prerequisite for a THERMO calculation.

        Before a FORCE calculation is started, a check  is  made  to  ensure
   that  a  stationary point is being used.  This check involves calculating
   the gradient norm (GNORM) and if it is significant,  the  GNORM  will  be
   reduced  using  BFGS.   All  internal  coordinates are optimized, and any
   symmetry constraints are ignored at this point.  An implication  of  this
   is  that  if the specification of the geometry relies on any angles being
   exactly 180 or zero degrees, the calculation may fail.

        The geometric definition supplied to FORCE should not rely on angles
   or  dihedrals  assuming  exact  values.   (The test of exact linearity is
   sufficiently slack that most molecules that are linear, such as acetylene
   and  but-2-yne,  should  not  be  stopped.)  See also THERMO, LET, TRANS,
   ISOTOPE.
   
        In a FORCE calculation, PRECISE will eliminate quartic contamination
   (part  of  the anharmonicity).  This is normally not important, therefore
   PRECISE should not routinely be used.  In a FORCE  calculation,  the  SCF
   criterion is automatically made more stringent; this is the main cause of
   the SCF failing in a FORCE calculation.


                                  
\subsection*{GEO-OK (W)}
        Normally the program will stop with a warning message if  two  atoms
   are within $0.8$ Angstroms of each other, or, more rarely, the BFGS routine
   has difficulty optimizing the geometry.  GEO-OK will  over-ride  the  job
   termination sequence, and allow the calculation to proceed.  In practice,
   most jobs that terminate due to these checks contain errors in  data,  so
   caution should be exercised if GEO-OK is used.  An important exception to
   this warning is when the system contains, or may give rise to, a Hydrogen
   molecule.  GEO-OK will override other geometric safety checks such as the
   unstable  gradient  in  a  geometry  optimization   preventing   reliable
   optimization.

   See also the message \verb/"GRADIENTS OF OLD GEOMETRY, GNORM= nn.nnnn"/.


                                
\subsection*{GNORM=n.nn (W)}
        The geometry optimization  termination  criteria  in  both  gradient
   minimization  and  energy minimization can be over-ridden by specifying a
   gradient  norm  requirement.   For  example,  GNORM=20  would  allow  the
   geometry  optimization to exit as soon as the gradient norm dropped below
   20.0, the default being 1.0.
   
        For high-precision work, GNORM=0.0 is recommended.   Unless  LET  is
   also  used, the GNORM will be set to the larger of 0.01 and the specified
   GNORM.   Results  from  GNORM=0.01  are  easily  good  enough   for   all
   high-precision work.


                                 
\subsection*{GRADIENTS (O)}
        In a 1SCF calculation gradients are not calculated by  default:   in
   non-variationally  optimized  systems  this would take an excessive time.
   GRADIENTS allows the gradients to  be  calculated.   Normally,  gradients
   will  not  be printed if the gradient norm is less than 2.0.  However, if
   GRADIENTS is present, then the  gradient  norm  and  the  gradients  will
   unconditionally be printed.  Abbreviation:  GRAD.


                                   
\subsection*{GRAPH (O)}
        Information needed to generate electron density contour maps can  be
   written to a file by calling GRAPH.  GRAPH first calls MULLIK in order to
   generate the inverse-square-root of the overlap matrix, which is required
   for the re-normalization of the eigenvectors.  All data essential for the
   graphics package DENSITY are then output.


                                  
\subsection*{HESS=n (W)}   
        When  the  Eigenvector  Following  routine  is  used  for   geometry
   optimization,  it  frequently  works faster if the Hessian is constructed
   first.  If HESS=1 is specified, the Hessian matrix  will  be  constructed
   before the geometry is optimized.  There are other, less common, options,
   e.g.  HESS=2.  See comments in subroutine EF for details.


                                
\subsection*{H-PRIORITY (O)}
        In  a  DRC  calculation,  results  will  be  printed  whenever   the
   calculated  heat  of  formation  changes by 0.1 kcal/mole.  Abbreviation:
   H-PRIO.

   

                             
\subsection*{H-PRIORITY=n.nn (O)}
        In  a  DRC  calculation,  results  will  be  printed  whenever   the
   calculated heat of formation changes by n.nn kcal/mole.


                                    
\subsection*{IRC (C)}
        An Intrinsic Reaction Coordinate calculation  is  to  be  run.   All
   kinetic   energy  is  shed  at  every  point  in  the  calculation.   See
   Background.


                                   
\subsection*{IRC=n (C)}
        An Intrinsic Reaction Coordinate calculation to be run;  an  initial
   perturbation in the direction of normal coordinate n to be applied.  If n
   is negative, then perturbation is reversed, i.e., initial  motion  is  in
   the  opposite direction to the normal coordinate.  This keyword cannot be
   written with spaces around the `=' sign.


                                  
\subsection*{ISOTOPE (O)}
        Generation of the  FORCE  matrix  is  very  time-consuming,  and  in
   isotopic  substitution  studies  several  vibrational calculations may be
   needed.  To allow the frequencies to be calculated  from  the  (constant)
   force  matrix,  ISOTOPE  is used.  When a FORCE calculation is completed,
   ISOTOPE will cause the force matrix to be stored, regardless  of  whether
   or  not  any  intervening  restarts  have been made.  To re-calculate the
   frequencies, etc.  starting at the end of the force  matrix  calculation,
   specify RESTART.

        The two keywords RESTART and ISOTOPE  can  be  used  together.   For
   example, if a normal FORCE calculation runs for a long time, the user may
   want to divide it up into stages and save the final force  matrix.   Once
   ISOTOPE  has been used, it does not need to be used on subsequent RESTART
   runs.
   
        ISOTOPE can also be used with FORCE to set up a RESTART file for  an
   IRC=n calculation.


                                  
\subsection*{ITRY=NN (W)}
        The default maximum number of SCF  iterations  is  200.   When  this
   limit  presents  difficulty,  ITRY=nn  can  be used to re-define it.  For
   example, if ITRY=400 is used, the maximum number of  iterations  will  be
   set to 400.  ITRY should normally not be changed until all other means of
   obtaining a SCF have been exhausted, e.g.  PULAY CAMP-KING etc.
   

                                  
\subsection*{IUPD=n (W)}
           IUPD is used only in the EF routine.  IUPD  should  very  rarely  be
   touched.   IUPD=1  can  be  used  in  minimum searches if the message
\begin{verbatim}
   "HEREDITARY POSITIVE DEFINITENESS ENDANGERED.  UPDATE SKIPPED THIS CYCLE"
\end{verbatim}
   occurs  every  cycle  for  10--20  iterations.   Never use IUPD=2 for a TS
   search!  For more information, read the comments in subroutine EF.
   
   
                                
\subsection*{K=(n.nn,n) (C)}
           Used  in  band-structure  calculations,  K=(n.nn,n)  specifies   the
   step-size in the Brillouin zone, and the number of atoms in the monomeric
   unit.  Two band-structure calculations  are  supported:   electronic  and
   phonon.   Both  require a polymer to be used.  If FORCE is used, a phonon
   spectrum is assumed, otherwise an electronic band structure  is  assumed.
   For  both  calculations,  a  density  of  states  is also done.  The band
   structure calculation is very fast, so a small  step-size  will  not  use
   much time.
   
        The output is designed to be fed into a graphics package, and is not
   `elegant'.  For polyethylene, a suitable keyword would be K=(0.01,6).


                               
\subsection*{KINETIC=n.nnn (C)}
        In a DRC calculation n.nnn kcals/mole of excess  kinetic  energy  is
   added  to  the  system  as  soon  as  the kinetic energy builds up to 0.2
   kcal/mole.  The excess energy is added to the  velocity  vector,  without
   change of direction.


                                 
\subsection*{LARGE (O)}
      Most of the time the  output  invoked  by  keywords  is  sufficient.
 LARGE  will  cause  less-commonly  wanted, but still useful, output to be
 printed.
 
 1.  To save space, DRC and IRC outputs will, by default, only  print  the
 line  with  the percent sign.  Other output can be obtained by use of the
 keyword LARGE, according to the following rules:
\begin{description}
\item[LARGE] Print all internal and cartesian coordinates 
                and cartesian velocities.
\item[LARGE=1]  Print all internal coordinates.
\item[LARGE=-1] Print all internal and cartesian coordinates 
                and cartesian velocities.
\item[LARGE=n] Print every n'th set of internal coordinates.
\item[LARGE=-n] Print every n'th set of internal and cartesian 
                 coordinates and cartesian velocities.
\end{description}
   
        If LARGE=1 is used, the output will be the same as that  of  Version
   5.0,  when  LARGE was not used.  If LARGE is used, the output will be the
   same as that of Version 5.0, when LARGE was used.  To save disk space, do
   not use LARGE.
   
   
                                  
\subsection*{LINMIN (O)}
           There  are  two  line-minimization  routines  in  MOPAC,  an  energy
   minimization  and  a  gradient  norm  minimization.   LINMIN  will output
   details of the line minimization used in a given job.


                                    
\subsection*{LET (W)}
        As MOPAC evolves, the meaning of LET is changing.

        Now LET means essentially ``I know what I'm  doing,  override  safety
   checks''.

        Currently, LET has the following meanings:
\begin{enumerate}
\item In a FORCE calculation, it means that the supplied  geometry  is
            to be used, even if the gradients are large.

\item In a geometry optimization, the specified GNORM is to  be  used,
            even if it is less than 0.01.

\item In a POLAR calculation, the molecule is to be  orientated  along
            its  principal moments of inertia before the calculation starts.
            LET will prevent this step being done.
\end{enumerate}


                                 
\subsection*{LOCALIZE (O)}
        The occupied eigenvectors are transformed into a  localized  set  of
M.O.'s by a series of 2 by 2 rotations which maximize $\langle\psi^4\rangle$.  The
value of $1/\langle\psi^4\rangle$ is a direct measure of the number of centers involved
in  the MO. Thus  the value of $1/\langle\psi^4\rangle$ is 2.0 for H2, 3.0 for a
three-center bond and 1.0 for a  lone  pair.   Higher  degeneracies  than
allowed by point group theory are readily obtained.  For example, benzene
would give rise to a 6-fold degenerate C--H bond, a 6-fold degenerate  C--C
sigma  bond and a three-fold degenerate C--C pi bond.  In principle, there
is no single step method to unambiguously obtain the most  localized  set
of  M.O.'s  in  systems  where several canonical structures are possible,
just as no simple method exists for finding the most stable conformer  of
some  large  compound.   However,  the  localized  bonds  generated  will
   normally be quite acceptable  for  routine  applications.   Abbreviation:
   LOCAL.
                
\subsection*{MAX}
  In a grid  calculation,  the  maximum  number  of  points  (23)  in  each
  direction  is  to  be  used.  The default is 11.  The number of points in
  each direction can be set with POINTS1 and POINTS2.

                                   
\subsection*{MECI (O)}
        At the  end  of  the  calculation  details  of  the  Multi  Electron
   Configuration  Interaction  calculation are printed if MECI is specified.
   The state vectors  can  be  printed  by  specifying  VECTORS.   The  MECI
   calculation is either invoked automatically, or explicitly invoked by the
   use of the C.I.=n keyword.

\subsection*{MICROS=n (C)}
        The microstates used by MECI are normally  generated  by  use  of  a
   permutation operator.  When individually defined microstates are desired,
   then MICROS=n can be used, where n defines the number of  microstates  to
   be read in.

                            
\subsubsection{Format for Microstates}
        After the geometry data plus any symmetry data  are  read  in,  data
   defining  each  microstate  is read in, using format 20I1, one microstate
   per line.  The microstate data is preceded by the word ``MICROS'' on a line
   by  itself.   There  is  at  present no mechanism for using MICROS with a
   reaction path.

        For a system with n M.O.'s in the C.I. (use OPEN=(n1,n) or C.I.=n to
   do  this), the populations of the n alpha M.O.'s are defined, followed by
   the n beta M.O.'s.  Allowed occupancies are zero and one.   For  n=6  the
   closed-shell  ground  state would be defined as 111000111000, meaning one
   electron in each of the first three alpha M.O.'s,  and  one  electron  in
   each of the first three beta M.O.'s.

        Users are warned that they are responsible for completing  any  spin
   manifolds.   Thus  while  the  state 111100110000 is a triplet state with
   component of spin = 1, the state 111000110100, while having  a  component
   of spin = 0 is neither a singlet nor a triplet.  In order to complete the
   spin manifold the microstate 110100111000 must also be included.

        If a manifold of spin states is not complete, then  the  eigenstates
   of  the  spin  operator will not be quantized.  When and only when 100 or
   fewer microstates are supplied, can spin quantization be conserved.

        There are two other limitations on possible microstates.  First, the
   number  of  electrons  in  every  microstate should be the same.  If they
   differ, a warning message will be printed, and the calculation  continued
   (but  the  results  will  almost  certainly  be  nonsense).   Second, the
   component of spin for every microstate  must  be  the  same,  except  for
   teaching  purposes.  Two microstates of different components of spin will
   have a zero matrix element connecting them.  No warning will be given  as
   this  is a reasonable operation in a teaching situation.  For example, if
   all states arising from two electrons in two levels are to be calculated,
   say for teaching Russel-Saunders coupling, then the following microstates
   would be used:                     
\begin{verbatim}
         Microstate       No. of alpha, beta electrons  Ms  State

           1100                    2     0              1   Triplet
           1010                    1     1              0   Singlet
           1001                    1     1              0   Mixed
           0110                    1     1              0   Mixed
           0101                    1     1              0   Singlet
           0011                    0     2             -1   Triplet
\end{verbatim}

        Constraints on the space manifold are just  as  rigorous,  but  much
   easier  to  satisfy.   If  the  energy  levels  are  degenerate, then all
   components of a manifold of degenerate M.O.'s should be  either  included
   or  excluded.   If  only  some,  but  not  all,  components are used, the
   required degeneracy of the states will be missing.

        As an example, for the  tetrahedral  methane  cation,  if  the  user
   supplies  the  microstates  corresponding  to  a component of spin = 3/2,
   neglecting Jahn-Teller distortion, the minimum number of states that  can
   be supplied is $90 = (6!/(1!5!))(6!/(4!2!))$.

        While the total number of electrons  should  be  the  same  for  all
   microstates,  this  number  does not need to be the same as the number of
   electrons supplied to the C.I.; thus in the  example  above,  a  cationic
   state could be 110000111000.

        The format is defined as 20I1 so that spaces can be used  for  empty
   M.O.'s.

\subsection*{MINDO/3 (C)}
        The default Hamiltonian within MOPAC is MNDO, with the  alternatives
   of  AM1  and MINDO/3.  To use the MINDO/3 Hamiltonian the keyword MINDO/3
   should be used.  Acceptable alternatives to the keyword MINDO/3 are MINDO
   and MINDO3.

\subsection*{MMOK (C)}
        If the system contains a peptide linkage, then  MMOK  will  allow  a
   molecular  mechanics  correction  to  be  applied  so that the barrier to
   rotation is increased (to 14.00 kcal/mole in N-methyl acetamide).

\subsection*{MODE (C)}
        MODE is used in the EF routine.  Normally the default MODE=1 is used
   to locate a transition state, but if this is incorrect, explicitly define
   the vector to be followed by using MODE=n.  (MODE is  not  a  recommended
   keyword).   If  you  use  the  FORCE  option  when deciding which mode to
   follow, set all isotopic masses to 1.0.  The normal modes from FORCE  are
   normally  mass-weighted; this can mislead.  Alternatively, use LARGE with
   FORCE:  this gives the force constants and vectors  in  addition  to  the
   mass-weighted  normal  modes.   Only the mass-weighted modes can be drawn
   with DRAW.

\subsection*{MS=n}
        Useful for checking the MECI calculation  and  for  teaching.   MS=n
   overrides  the normal choice of magnetic component of spin.  Normally, if
   a triplet is requested, an MS of  1  will  be  used;  this  excludes  all
   singlets.   If MS=0 is also given, then singlets will also be calculated.
   The use of MS should not affect the values of the results at all.


\subsection*{MULLIK (O)}
        A full Mulliken Population analysis is to be done on the  final  RHF
   wavefunction.  This involves the following steps:
\begin{enumerate}
 \item The eigenvector matrix is divided by the square root 
           of the overlap matrix, $S$.
 \item The Coulson-type density matrix, $P$, is formed.
 \item The overlap population is formed from $P(i,j) S(i,j)$.
 \item Half the off-diagonals are added onto the diagonals.
\end{enumerate}
                                   
\subsection*{NLLSQ (C)}
        The gradient norm is to be minimized by Bartel's method.  This is  a
   Non-Linear   Least   Squares  gradient  minimization  routine.   Gradient
   minimization will locate one of three possible points:

        (a) A minimum in the energy surface.  The gradient norm will  go  to
   zero,  and  the  lowest  five  or  six eigenvalues resulting from a FORCE
   calculation will be approximately zero.

        (b) A transition state.  The gradient norm will vanish, as  in  (a),
   but  in  this  case  the  system  is  characterized by one, and only one,
   negative force constant.

        (c) A local minimum in the gradient norm space.  In  this  (normally
   unwanted)  case  the gradient norm is minimized, but does not go to zero.
   A FORCE calculation will not  give  the  five  or  six  zero  eigenvalues
   characteristic  of  a stationary point.  While normally undesirable, this
   is sometimes the only way to obtain  a  geometry.   For  instance,  if  a
   system is formed which cannot be characterized as an intermediate, and at
   the same time is  not  a  transition  state,  but  nonetheless  has  some
   chemical significance, then that state can be refined using NLLSQ.
                                
\subsection*{NOANCI (W)}
           RHF open-shell derivatives are normally calculated  using  Liotard's
   analytical C.I. method.  If this method is NOT to be used, specify NOANCI
   (NO ANalytical Configuration Interaction derivatives).

                                  
\subsection*{NODIIS (W)}
        In the event that the G-DIIS option is not  wanted,  NODIIS  can  be
   used.   The  G-DIIS  normally  accelerates the geometry optimization, but
   there is no guarantee that it will do so.  If the heat of formation rises
   unexpectedly  (i.e., rises during a geometry optimization while the GNORM
   is larger than about 0.3), then try NODIIS.
                                  
\subsection*{NOINTER (O)}
        The interatomic distances are printed by default.   If  you  do  not
   want  them to be printed, specify NOINTER.  For big jobs this reduces the
   output file considerably.


\subsection*{NOLOG (O)}
           Normally a copy of the archive file will  be  directed  to  the  LOG
   file, along with a synopsis of the job.  If this is not wanted, it can be
   suppressed completely by NOLOG.

\subsection*{NOMM (C)}
      All  four  semi-empirical  methods  underestimate  the  barrier   to
   rotation  of  a  peptide bond.  A Molecular Mechanics correction has been
   added which increases the barrier in N-methyl acetamide to 14  kcal/mole.
   If   you  do  not  want  this  correction,  specify  NOMM  (NO  Molecular
   Mechanics).


                                   
\subsection*{NONR (W)}
Not recommended for normal use.  Used  with  the  EF  routine.   See
source code for more details.

\subsection*{NOTHIEL (W)}
           In a normal geometry optimization using the  BFGS  routine,  Thiel's
   FSTMIN  technique  is  used.  If normal line-searches are wanted, specify
   NOTHIEL.


                                   
\subsection*{NOXYZ (O)}
        The cartesian coordinates are printed by default.   If  you  do  not
   want  them  to  be printed, specify NOXYZ.  For big jobs this reduces the
   output file considerably.
                       
\subsection*{NSURF (C)}
        In an ESP calculation,  NSURF=n  specifies  the  number  of  surface
   layers for the Connolly surface.

\subsection*{OLDENS (W)}
        A density matrix produced by an earlier run of MOPAC is to  be  used
   to start the current calculation.  This can be used in attempts to obtain
   an SCF when a previous calculation ended successfully  but  a  subsequent
   run failed to go SCF.

\subsection*{OLDGEO (C)}
        If multiple geometries are to be run, and the  final  geometry  from
   one  calculation  is  to  be  used  to start the next calculation, OLDGEO
   should be specified.  Example:  If a MNDO, AM1, and PM3 calculation  were
   to  be done on one system, for which only a rough geometry was available,
   then after the MNDO calculation, the AM1 calculation could be done  using
   the  optimized  MNDO  geometry  as  the  starting geometry, by specifying
   OLDGEO.

\subsection*{OPEN(n1,n2) (C)}
        The M.O.  occupancy during the SCF calculation  can  be  defined  in
   terms  of  doubly occupied, empty, and fractionally occupied M.O.'s.  The
   fractionally occupied M.O.'s are  defined  by  OPEN(n1,n2),  where  n1  =
   number  of  electrons  in  the  open-shell  manifold,  and n2 = number of
   open-shell M.O.'s; n1/n2 must be in the range 0 to 2.  OPEN(1,1) will  be
   assumed  for odd-electron systems unless an OPEN keyword is used.  Errors
   introduced by use of fractional occupancy are automatically corrected  in
   a MECI calculation when OPEN(n1,n2) is used.
                                  
\subsection*{ORIDE (W)}  
Do not use this keyword until you have read Simons' article.   ORIDE
is  part  of  the EF routine, and means ``Use whatever $\Lambda$'s are produced
even if they would normally be `unacceptable'.''\\
J. Simons, P. Jorgensen, H. Taylor, J. Ozment, 
{\em J. Phys. Chem.\/} 87:2745 (1983).

\subsection*{PARASOK (W)}
   {\em Use this keyword with extreme caution!\/} The  AM1  method  has  been
   parameterized  for only a few elements, less than the number available to
   MNDO or PM3.  If any elements which are  not  parameterized  at  the  AM1
   level  are  specified,  the  MNDO parameters, if available, will be used.
   The resulting mixture of methods, AM1 with MNDO, has not been studied  to
   see  how good the results are, and users are strictly on their own as far
   as accuracy and  compatibility  with  other  methods  is  concerned.   In
   particular,  while all parameter sets are referenced in the output, other
   programs may not cite the parameter sets used and thus compatibility with
   other MNDO programs is not guaranteed.

\subsection*{PI (O)}
        The normal density matrix is composed of atomic orbitals, that is s,
   px,  py and pz.  PI allows the user to see how each atom-atom interaction
   is split into $\sigma$ and $\pi$ bonds. The  resulting ``density  matrix'' is
   composed  of  the  following  basis-functions:-  
   s-$\sigma$,  p-$\sigma$, p-$\pi$, d-$\sigma$, d-$\pi$, d-$\delta$.  
   The on-diagonal  terms  give  the  hybridization
   state,  so  that an sp$^2$ hybridized system would be represented as 
   s-$\sigma$: 1.0, p-$\sigma$: 2.0, p-$\pi$: 1{.}0.
                                    
\subsection*{PM3 (C)}
The PM3 method is to be used.

\subsection*{POINT=n (C)}
        The number of  points  to  be  calculated  on  a  reaction  path  is
   specified by POINT=n.  Used only with STEP in a path calculation.
                                 
\subsection*{POINT1=n (C)}
        In a grid calculation, the number of points to be calculated in  the
   first  direction  is  given  by  POINT1=n.   `n'  should be less than 24;
   default:  11.
                                    
\subsection*{POINT2=n (C)}
        In a grid calculation, the number of points to be calculated in  the
   second  direction  is  given  by  POINT2=n.  `n' should be less than 24,
   default:  11;
                                  
\subsection*{POTWRT (W)}
   In an  ESP  calculation,  write  out  surface  points  and  electrostatic
   potential values to UNIT 21.
                                   
\subsection*{ POLAR (C)}\index{POLAR}
     
          The polarizability and first and second hyperpolarizabilities are to
     be  calculated.   At present this calculation does not work for polymers,
     but should work for all other systems.  Two different options are
     implemented: the older finite field method and a new time-dependent
     Hartree-Fock method.
\subsubsection{     Time-Dependent Hartree-Fock}

          This procedure is based on the detailed description given by
     M. Dupuis and S. Karna (J. Comp. Chem. 12, 487 (1991)).  The
     program is capable of calculating:

\begin{tabular}{ll}
 
        Frequency Dependent Polarizability       & alpha(-w;w)    \\
        Second Harmonic Generation               & beta(-2w;w,w)    \\
        Electrooptic Pockels Effect              & beta(-w;0,w)    \\
        Optical Rectification                    & beta(0;-w,w)    \\
        Third Harmonic Generation                & gamma(-3w;w,w,w)    \\
        DC-EFISH                                 & gamma(-2w;0,w,w)    \\
        Optical Kerr Effect                      & gamma(-w;0,0,w)    \\
        Intensity Dependent Index of Refraction  & gamma(-w;w,-w,w)    \\
\end{tabular}

          The input is given at the end of the MOPAC deck and 
     consists of two lines of free-field input followed by a list
     energies.  
          The variables on the first line are:

\begin{tabular}{ll}
             Nfreq  = &How many energies will be used to calculate    \\
                      &the desired quantities.   \\
             Iwflb  = &Type of beta calculation to be performed.   \\
                      &This valiable is only important if iterative   \\
                      &beta calculations are chosen.   \\
                      & 0 - static   \\
                      & 1 - SHG   \\
                      & 2 - EOPE   \\
                      & 3 - OR    \\
             Ibet   = &Type of beta calculation:   \\
                      & 0 - beta(0;0)  static   \\
                      & 1 - iterative calculation with type of   \\
                      &     beta chosen by Iwflb.   \\
                      & 1 - Noniterative calculation of SHG   \\
                      &-2 - Noniterative calculation of EOPE   \\
                      &-3 - Noniterative calculation of OR   \\
             Igam   = &Type of gamma calculation:   \\
                      & 0 - No gamma calculation   \\
                      & 1 - THG   \\
                      & 2 - DC-EFISH   \\
                      & 3 - IDRI   \\
                      & 4 - OKE   \\
\end{tabular}

          The vaiables on the second line are:

\begin{tabular}{ll}
             Atol   = & Cutoff tolerance for alpha calculations  \\
                      & (1.0e-4 seems reasonable)  \\
             Maxitu = & Maximum number of iteractions for beta  \\
                      & calculations  \\
             Maxita = & Maximum number of iterations for alpha  \\
                      & calculations  \\
             Btol   = & Cutoff tolerance for beta calculations  \\
\end{tabular}

          Nfreq lines follow, each with an energy value in eV's at
     which the hyperpolarizabilites are to be calculated.



\subsection*{POWSQ (C)}
        Details of the working of POWSQ  are  printed  out.   This  is  only
   useful in debugging.
                                  
\subsection*{PRECISE (W)}
        The criteria  for  terminating  all  optimizations,  electronic  and
   geometric,  are  to be increased by a factor, normally, 100.  This can be
   used where more precise results are wanted.  If the results are going  to
   be  used  in  a  FORCE  calculation, where the geometry needs to be known
   quite precisely, then PRECISE is recommended; for small systems the extra
   cost  in CPU time is minimal.  PRECISE is not recommended for experienced
   users, instead GNORM=n.nn and SCFCRT=n.nn are suggested.  PRECISE  should
   only  very  rarely  be  necessary in a FORCE calculation:  all it does is
   remove quartic  contamination,  which  only  affects  the  trivial  modes
   significantly, and is very expensive in CPU time.

\subsection*{PULAY (W)}
        The default converger in the SCF calculation is to  be  replaced  by
   Pulay's  procedure  as soon as the density matrix is sufficiently stable.
   A considerable improvement in speed can be achieved by the use of  PULAY.
   If a large number of SCF calculations are envisaged, a sample calculation
   using 1SCF and PULAY should be compared with using 1SCF on its  own,  and
   if  a  saving  in  time  results,  then  PULAY should be used in the full
   calculation.  PULAY should be used with care in that its use will prevent
   the  combined  package  of  convergers  (SHIFT,  PULAY  and the CAMP-KING
   convergers) from automatically being used in the event  that  the  system
   fails to go SCF in (ITRY-10) iterations.

        The combined set of convergers very seldom fails.

                                  
\subsection*{QUARTET (C)}
        RHF interpretation:  The desired spin-state is a quartet, i.e.,  the
   state  with component of spin = 1/2 and spin = 3/2.  When a configuration
   interaction calculation is done, all spin states of  spin  equal  to,  or
   greater  than  1/2 are calculated simultaneously, for component of spin =
   1/2.  From these states the quartet states are selected when  QUARTET  is
   specified,  and  all  other spin states, while calculated, are ignored in
   the choice of root to be used.  If QUARTET is used on  its  own,  then  a
   single  state, corresponding to an alpha electron in each of three M.O.'s
   is calculated.
   
        UHF interpretation:  The system will have three more alpha electrons
   than beta electrons.


                                  
\subsection*{QUINTET (C)}
        RHF interpretation:  The desired spin-state is a quintet,  that  is,
   the  state with component of spin = 0 and spin = 2.  When a configuration
   interaction calculation is done, all spin states of  spin  equal  to,  or
   greater  than 0 are calculated simultaneously, for component of spin = 0.
   From these states  the  quintet  states  are  selected  when  QUINTET  is
   specified,  and  the  septet states, while calculated, will be ignored in
   the choice of root to be used.  If QUINTET is used on  its  own,  then  a
   single  state,  corresponding to an alpha electron in each of four M.O.'s
   is calculated.
   
        UHF interpretation:  The system will have three more alpha electrons
   than beta electrons.
                                   
\subsection*{RECALC=n}
   RECALC=n  calculates  the  Hessian  every  n   steps   in   the   EF
   optimization.   For  small n this is costly but is also very effective in
   terms  of  convergence.   RECALC=10  and  DMAX=0.10  can  be  useful  for
   difficult  cases.   In  extreme  cases RECALC=1 and DMAX=0.05 will always
   find a stationary point, if it exists.


                                  
\subsection*{RESTART (W)}
        When a job has been stopped, for whatever reason,  and  intermediate
   results  have  been  stored, then the calculation can be restarted at the
   point where it stopped by specifying RESTART.  The most common cause of a
   job  stopping  before  completion is its exceeding the time allocated.  A
   saddle-point calculation has no restart, but  the  output  file  contains
   information  which  can  easily  be  used to start the calculation from a
   point near to where it stopped.

        It is not necessary to change the geometric data to reflect the  new
   geometry.   As a result, the geometry printed at the start of a restarted
   job will be that of the original data, not that of the restarted file.
        A convenient way to monitor a  long  run  is  to  specify  1SCF  and
   RESTART; this will give a normal output file at very little cost.
\begin{description}
\item[Note 1:]  In the FORCE calculation two restarts are possible.   These
   are (a) a restart in FLEPO if the geometry was not optimized fully before
   FORCE was called, and (b) the normal restart in the construction  of  the
   force  matrix.   If the restart is in FLEPO within FORCE then the keyword
   FORCE should be deleted,  and  the  keyword  RESTART  used  on  its  own.
   Forgetting this point is a frequent cause of failed jobs.

\item[Note 2:]  Two restarts also exist in the IRC calculation.  If an  IRC
   calculation  stops  while in the FORCE calculation, then a normal restart
   can be done.  If the job stops while doing  the  IRC  calculation  itself
   then  the keyword IRC=n should be changed to IRC, or it can be omitted if
   DRC is also specified.  The absence of  the  string  "IRC="  is  used  to
   indicate  that  the  FORCE  calculation  was completed before the restart
   files were written.
\end{description}

\subsection*{ROOT=n (C)}
        The  n'th  root  of  a  C.I. calculation  is  to  be  used  in   the
   calculation.   If  a  keyword  specifying the spin-state is also present,
   e.g.  SINGLET or TRIPLET, then the  n'th  root  of  that  state  will  be
   selected.   Thus  ROOT=3  and SINGLET will select the third singlet root.
   If ROOT=3 is used on its own, then the third root will be used, which may
   be  a  triplet, the third singlet, or the second singlet (the second root
   might be a triplet).  In normal use, this keyword would not be used.   It
   is  retained  for educational and research purposes.  Unusual care should
   be exercised when ROOT= is specified.

\subsection*{ROT=n (C)}
        In  the  calculation  of  the  rotational   contributions   to   the
   thermodynamic  quantities  the  symmetry  number  of the molecule must be
   supplied.  The symmetry  number  of  a  point  group  is  the  number  of
   equivalent  positions  attainable  by  pure rotations.  No reflections or
   improper rotations  are  allowed.   This  number  cannot  be  assumed  by
   default,  and  may  be  affected by subtle modifications to the molecule,
   such as isotopic substitution.  A list of  the  most  important  symmetry
   numbers follows:
\begin{verbatim}
            ----    TABLE OF SYMMETRY NUMBERS    ----


       C1 CI CS     1      D2 D2D D2H  4       C(INF)V   1
       C2 C2V C2H   2      D3 D3D D3H  6       D(INF)H   2
       C3 C3V C3H   3      D4 D4D D4H  8       T TD     12
       C4 C4V C4H   4      D6 D6D D6H  12      OH       24
       C6 C6V C6H   6      S6          3
\end{verbatim}
                                                
\subsection*{SADDLE (C)}
        The transition  state  in  a  simple  chemical  reaction  is  to  be
   optimized.    Extra   data  are  required.   After  the  first  geometry,
   specifying the reactants, and any symmetry functions have  been  defined,
   the  second geometry, specifying the products, is defined, using the same
   format as that of the first geometry.

        SADDLE often fails to work successfully.  Frequently this is due  to
   equivalent dihedral angles in the reactant and product differing by about
   360 degrees rather than zero degrees.  As the choice of dihedral  can  be
   difficult,  users  should  consider  running  this  calculation  with the
   keyword XYZ.  There  is  normally  no  ambiguity  in  the  definition  of
   cartesian coordinates.  See also BAR=.

        Many of the bugs in SADDLE have been removed in this  version.   Use
   of the XYZ option is strongly recommended.
                                   
\subsection*{SCALE (C)}
        SCALE=n.n specifies the scaling factor for Van der Waals' radii  for
   the initial layer of the Connolly surface in the ESP calculation.
                                
\subsection*{SCFCRT=n.nn (W)}
        The default SCF criterion is to  be  replaced  by  that  defined  by
   SCFCRT=.
   
        The SCF criterion is  the  change  in  energy  in  kcal/mol  on  two
   successive  iterations.   Other  minor criteria may make the requirements
   for an SCF slightly more stringent.  The SCF criterion can be varied from
   about 0.001 to 1.D-25, although numbers in the range 0.0001 to 1.D-9 will
   suffice for most applications.

        An overly tight criterion can lead to failure to achieve a SCF,  and
   consequent failure of the run.
                                  
\subsection*{SCINCR=n.nn}
        In an ESP calculation, SCINCR=n.nn specifies the  increment  between
   layers of the surface in the Connolly surface.  (default:  0.20)
                                   
\subsection*{SETUP (C)}
        If, on the keyword line, the word `SETUP' is specified, then one  or
   two  lines  of  keywords  will  be read from a file with the logical name
   SETUP.  The logical file SETUP must exist, and must contain at least  one
   line.  If the second line is defined by the first line as a keyword line,
   and the second line contains the word SETUP, then one  line  of  keywords
   will be read from a file with the logical name SETUP.
              
                                
\subsection*{SETUP=name (C)}
           Same as SETUP, only the logical or actual name of the SETUP file  is
   `name'.


                                  
\subsection*{SEXTET (C)}
        RHF interpretation:  The desired spin-state is a sextet:  the  state
   with component of spin = 1/2 and spin = 5/2.

        The sextet states are the highest spin  states  normally  calculable
   using MOPAC in its unmodified form.  If SEXTET is used on its own, then a
   single state, corresponding to one alpha electron in each of five M.O.'s,
   is  calculated.   If several sextets are to be calculated, say the second
   or third, then OPEN(n1,n2) should be used.
   
        UHF interpretation:  The system will have five more alpha  electrons
   than beta electrons.

\subsection*{SHIFT=n.nn (W)}
        In an attempt to obtain an SCF by damping  oscillations  which  slow
   down  the  convergence or prevent an SCF being achieved, the virtual M.O.
   energy levels are shifted up or down in energy by a shift technique.  The
   principle is that if the virtual M.O.'s are changed in energy relative to
   the occupied set, then the polarizability of  the  occupied  M.O.'s  will
   change  pro  rata.   Normally,  oscillations  are due to autoregenerative
   charge fluctuations.

        The SHIFT method has been re-written so  that  the  value  of  SHIFT
   changes  automatically  to  give  a  critically-damped  system.  This can
   result in a positive or  negative  shift  of  the  virtual  M.O.   energy
   levels.   If  a non-zero SHIFT is specified, it will be used to start the
   SHIFT technique, rather than the default 15eV.  If SHIFT=0 is  specified,
   the SHIFT technique will not be used unless normal convergence techniques
   fail and the automatic ``ALL CONVERGERS \ldots'' message is produced.

\subsection*{SIGMA (C)}
        The McIver-Komornicki gradient norm minimization routines, POWSQ and
   SEARCH  are  to  be used.  These are very rapid routines, but do not work
   for all species.  If the gradient norm is low, i.e., less  than  about  5
   units,   then   SIGMA  will  probably  work;  in  most  cases,  NLLSQ  is
   recommended.  SIGMA first calculates a quite accurate Hessian  matrix,  a
   slow  step,  then works out the direction of fastest decent, and searches
   along that direction until the gradient norm is minimized.   The  Hessian
   is  then  partially  updated  in  light of the new gradients, and a fresh
   search direction found.  Clearly, if the Hessian changes  markedly  as  a
   result  of  the  line-search, the update done will be inaccurate, and the
   new search direction will be faulty.
              
        SIGMA should be avoided if at all  possible  when  non-variationally
   optimized calculations are being done.

        If the Hessian is suspected to be corrupt within SIGMA  it  will  be
   automatically  recalculated.  This frequently speeds up the rate at which
   the transition state is located.  If you do not want the  Hessian  to  be
   reinitialized  ---  it is costly in CPU time --- specify LET on the keyword
   line.

\subsection*{SINGLET (C)}
        When a configuration  interaction  calculation  is  done,  all  spin
   states are calculated simultaneously, either for component of spin = 0 or
   1/2.  When only singlet states are  of  interest,  then  SINGLET  can  be
   specified,  and  all  other spin states, while calculated, are ignored in
   the choice of root to be used.

        Note that while  almost  every  even-electron  system  will  have  a
   singlet  ground  state,  SINGLET should still be specified if the desired
   state must be a singlet.

        SINGLET has no meaning in a UHF calculation, but see also TRIPLET.


                                   
\subsection*{SLOPE (C)}
        In an ESP calculation, SLOPE=n.nn specifies  the  scale  factor  for
   MNDO charges.  (default=1.422)

\subsection*{SPIN (O)}
        The spin matrix, defined as the difference  between  the  alpha  and
   beta   density  matrices,  is  to  be  printed.   If  the  system  has  a
   closed-shell ground state, e.g.  methane run UHF, the spin matrix will be
   null.

        If SPIN is not requested in a UHF calculation, then the diagonal  of
   the spin matrix, that is the spin density on the atomic orbitals, will be
   printed.

\subsection*{STEP (C)}
        In a reaction path, if the path step is constant, STEP can  be  used
   instead  of  explicitly  specifying  each  point.  The number of steps is
   given by POINT.  If the reaction coordinate is an  interatomic  distance,
   only positive STEPs are allowed.
              
                                
\subsection*{STEP1=n.nnn (C)}
        In a grid calculation the step size in degrees or Angstroms for  the
   first  of  the two parameters is given by n.nnn.  By default, an 11 by 11
   grid is generated.  See POINT1 and POINT2 on how to adjust  this  number.
   The  first point calculated is the supplied geometry, and is in the upper
   left hand corner.  This is a change from Version 5.00, where the supplied
   geometry was the central point.

\subsection*{STEP2=n.nnn (C)}
        In a grid calculation the step size in degrees or Angstroms for  the
   second of the two parameters is given by n.nnn.

\subsection*{STO3G (W)}
        In an ESP calculation STO3G means ``Use  the  STO-3G  basis  set  to
   de-orthogonalize the semiempirical orbitals''.

\subsection*{SYMAVG (W)}   
        Used by the ESP, SYMAVG will average charges which should  have  the
   same value by symmetry.

\subsection*{SYMMETRY (C)}
        Symmetry data defining related bond lengths,  angles  and  dihedrals
   can  be included by supplying additional data after the geometry has been
   entered.  If there are any other data, such as values  for  the  reaction
   coordinates,  or  a second geometry, as required by SADDLE, then it would
   follow the symmetry data.  Symmetry data  are  terminated  by  one  blank
   line.   For  non-variationally optimized systems symmetry constraints can
   save a lot of time because many derivatives do not need to be calculated.
   At  the  same  time,  there  is  a  risk that the geometry may be wrongly
   specified,  e.g.   if  methane  radical  cation  is  defined   as   being
   tetrahedral,  no  indication  that  this  is faulty will be given until a
   FORCE calculation is run.  (This system undergoes spontaneous Jahn-Teller
   distortion.)

        Usually a lower heat of formation can be obtained when  SYMMETRY  is
   specified.   To  see  why,  consider  the  geometry  of  benzene.   If no
   assumptions are made regarding  the  geometry,  then  all  the  C--C bond
   lengths  will  be very slightly different, and the angles will be almost,
   but not quite 120 degrees.  Fixing all angles at 120  degrees,  dihedrals
   at  180 or 0 degrees, and only optimizing one C--C and one C--H bond-length
   will  result  in  a  2--D optimization, and exact D$_{6h}$ symmetry. Any
   deformation  from  this  symmetry  must  involve  error,  so  by imposing
   symmetry some error is removed.
              

   The layout of the symmetry data is:
\begin{verbatim}
   <defining atom> <symmetry relation> <defined atom> <defined atom>,...
\end{verbatim}
where the numerical code for \verb/<symmetry relation>/ is given 
in the table of symmetry functions below.

   For example,  ethane,  with  three  independent  variables,  can  be
   defined as:
\begin{verbatim}
     SYMMETRY 
     ETHANE, D3D                                   NA NB NC
                                                    
      C    
      C    1.528853 1                               1  
      H    1.105161 1  110.240079 1                 2  1  
      H    1.105161 0  110.240079 0  120.000000 0   2  1  3
      H    1.105161 0  110.240079 0  240.000000 0   2  1  3
      H    1.105161 0  110.240079 0   60.000000 0   1  2  3
      H    1.105161 0  110.240079 0  180.000000 0   1  2  3
      H    1.105161 0  110.240079 0  300.000000 0   1  2  3
      0    0.000000 0    0.000000 0    0.000000 0   0  0  0
       3,    1,    4,    5,     6,     7,     8,
       3,    2,    4,    5,     6,     7,     8,
\end{verbatim}

        Here atom 3,  a  hydrogen,  is  used  to  define  the  bond  lengths
   (symmetry  relation  1)  of  atoms  4,5,6,7 and 8 with the atoms they are
   specified to bond with in the NA column of the data file; similarly,  its
   angle  (symmetry  relation  2)  is used to define the bond-angle of atoms
   4,5,6,7 and 8 with the two atoms specified in the NA and  NB  columns  of
   the  data  file.   The other angles are point-group symmetry defined as a
   multiple of 60 degrees.

        Spaces, tabs or commas can be used to separate data.  Note that only
   three  parameters  are  marked to be optimized.  The symmetry data can be
   the last line of the data file unless more data follows, in which case  a
   blank line must be inserted after the symmetry data.

The full list of available symmetry relations is as 
follows:\index{symmetry functions!defined}
\begin{verbatim}              
<Symmetry                  SYMMETRY FUNCTIONS
 relation>

   1     BOND LENGTH    IS SET EQUAL TO THE REFERENCE BOND LENGTH 
   2     BOND ANGLE     IS SET EQUAL TO THE REFERENCE BOND ANGLE 
   3     DIHEDRAL ANGLE IS SET EQUAL TO THE REFERENCE DIHEDRAL ANGLE
   4     DIHEDRAL ANGLE VARIES AS  90 DEGREES - REFERENCE DIHEDRAL 
   5     DIHEDRAL ANGLE VARIES AS  90 DEGREES + REFERENCE DIHEDRAL 
   6     DIHEDRAL ANGLE VARIES AS 120 DEGREES - REFERENCE DIHEDRAL 
   7     DIHEDRAL ANGLE VARIES AS 120 DEGREES + REFERENCE DIHEDRAL 
   8     DIHEDRAL ANGLE VARIES AS 180 DEGREES - REFERENCE DIHEDRAL 
   9     DIHEDRAL ANGLE VARIES AS 180 DEGREES + REFERENCE DIHEDRAL 
  10     DIHEDRAL ANGLE VARIES AS 240 DEGREES - REFERENCE DIHEDRAL 
  11     DIHEDRAL ANGLE VARIES AS 240 DEGREES + REFERENCE DIHEDRAL 
  12     DIHEDRAL ANGLE VARIES AS 270 DEGREES - REFERENCE DIHEDRAL 
  13     DIHEDRAL ANGLE VARIES AS 270 DEGREES + REFERENCE DIHEDRAL 
  14     DIHEDRAL ANGLE VARIES AS THE NEGATIVE OF THE REFERENCE 
         DIHEDRAL 
  15     BOND LENGTH VARIES AS HALF THE REFERENCE BOND LENGTH 
  16     BOND ANGLE VARIES AS HALF THE REFERENCE BOND ANGLE 
  17     BOND ANGLE VARIES AS 180 DEGREES - REFERENCE BOND ANGLE 
  18     BOND LENGTH IS A MULTIPLE OF REFERENCE BOND-LENGTH 
\end{verbatim}

        Function  18  is  intended  for  use  in  polymers,  in  which   the
   translation  vector  may be a multiple of some bond-length.  1,2,3 and 14
   are most commonly used.  Abbreviation:  SYM.

        SYMMETRY is not available for use with cartesian coordinates.

\subsection*{T= (W)}
        This is a facility to allow the program to shut down in  an  orderly
   manner on computers with execution time cpu  limits.

        The total cpu  time allowed for the current  job  is  limited  to
   nn.nn  seconds;  by default this is one hour, i.e., 3600 seconds.  If the
   next cycle of the calculation cannot be completed without running a  risk
   of  exceeding the assigned time the calculation will write a restart file
   and then stop.  The safety margin is 100 percent; that is, to do  another
   cycle, enough time to do at least two full cycles must remain.

        Alternative specifications of the time are  T=nn.nnM,  this  defines
   the  time in minutes, T=nn.nnH, in hours, and T=nn.nnD, in days, for very
   long jobs.  This keyword cannot be written with  spaces  around the `='
   sign.
                                  
\subsection*{THERMO (O)}
        The  thermodynamic  quantities,  internal  energy,  heat   capacity,
   partition  function,  and  entropy  can  be  calculated  for translation,
   rotation and vibrational degrees of freedom for a single temperature,  or
   a  range  of temperatures.  Special situations such as linear systems and
   transition states are  accommodated.   The  approximations  used  in  the
   THERMO  calculation  are  invalid  below  100K, and checking of the lower
   bound of the temperature range is done to prevent  temperatures  of  less
   than 100K being used.

        Another limitation, for which no checking is  done,  is  that  there
   should  be  no  internal  rotations.   If  any  exist,  they  will not be
   recognized as such, and the calculated quantities will be too  low  as  a
   result.

        In order to use THERMO the keyword FORCE must also be specified,  as
   well as the value for the symmetry number; this is given by ROT=n.

        If THERMO is specified on its own, then the default  values  of  the
   temperature  range  are  assumed.   This  starts at 200K and increases in
   steps of 10 degrees to 400K.  Three  options  exist  for  overriding  the
   default temperature range.  These are:

\subsection*{THERMO(nnn)  (O)}
        The thermodynamic quantities for a 200 degree range of temperatures,
   starting at nnnK and with an interval of 10 degrees are to be calculated.

\subsection*{THERMO(nnn,mmm)  (O)}
        The thermodynamic quantities for the temperature range limited by  a
   lower bound of nnn Kelvin and an upper bound of mmm Kelvin, the step size
   being calculated  in  order  to  give  approximately  20  points,  and  a
   reasonable  value  for  the step.  The size of the step in Kelvin degrees
   will be 1, 2, or 5, or a power of 10 times these numbers.

\subsection*{THERMO(nnn,mmm,lll)  (O)}
        Same as for THERMO(nnn,mmm), only now the user can explicitly define
   the step size.  The step size cannot be less than 1K.

\subsection*{T-PRIORITY (O)}
In  a  DRC  calculation,  results  will  be  printed  whenever   the
calculated time changes by 0.1 femtoseconds.  Abbreviation, T-PRIO.

                              
\subsection*{T-PRIORITY=n.nn (O)}
In  a  DRC  calculation,  results  will  be  printed  whenever   the
calculated time changes by n.nn femtoseconds.


                                   
\subsection*{TRANS (C)}
        The imaginary frequency due to the reaction vector in  a  transition
   state calculation must not be included in the thermochemical calculation.
   The number of genuine vibrations considered can be:
        $3N-5$ for a linear ground state system,
        $3N-6$ for a non-linear ground state system, or
        $3N-6$ for a linear transition-state complex,
        $3N-7$ for a non-linear transition-state complex.

        This keyword must be used in conjunction with THERMO if a transition
   state is being calculated.


                                 
\subsection*{TRANS=n  (C)}
        The facility exists  to  allow  the  THERMO  calculation  to  handle
   systems  with  internal  rotations.   TRANS=n  will  remove  the n lowest
   vibrations.  Note that TRANS=1 is equivalent to TRANS on  its  own.   For
   xylene, for example, TRANS=2 would be suitable.

        This keyword cannot be written with spaces around the `=' sign.

\subsection*{TRIPLET (C)}
        The triplet state is defined.  If the system has an  odd  number  of
   electrons, an error message will be printed.

\subsubsection{UHF interpretation}
        The number of alpha electrons exceeds that of the beta electrons  by
   2.   If  TRIPLET  is  not  specified,  then the numbers of alpha and beta
   electrons are set equal.  This  does  not  necessarily  correspond  to  a
   singlet.

\subsubsection{RHF interpretation}
        An RHF MECI calculation is performed to calculate the triplet state.
   If  no other C.I. keywords are used, then only one state is calculated by
   default.  The occupancy of the M.O.'s in the SCF calculation  is  defined
   as (\ldots 2,1,1,0,\ldots), that is, one electron is put in each of the two
   highest occupied M.O.'s.
         
   See keywords C.I.=n and OPEN(n1,n2).

\subsection*{TS (C)}
        Within the Eigenvector  Following  routine,  the  option  exists  to
   optimize   a   transition  state.   To  do  this,  use  TS.   Preliminary
   indications are that the TS method is much faster and more reliable  than
   either SIGMA or NLLSQ.
   
        TS appears to work well with cartesian coordinates.
   
        In the event that TS does not converge on a  stationary  point,  try
   adding RECALC=5 to the keyword line.


                                    
\subsection*{UHF (C)}
        The unrestricted Hartree-Fock Hamiltonian is to be used.


                                  
\subsection*{VECTORS (O)}
        The eigenvectors are to be printed.  In UHF calculations both  alpha
   and  beta  eigenvectors  are printed; in all cases the full set, occupied
   and virtual, are output.  The eigenvectors are normalized to unity,  that
   is  the  sum of the squares of the coefficients is exactly one.  If DEBUG
   is specified, then ALL eigenvectors  on  every  iteration  of  every  SCF
   calculation  will  be printed.  This is useful in a learning context, but
   would normally be very undesirable.


                                 
\subsection*{VELOCITY (C)}
        The user can supply the initial  velocity  vector  to  start  a  DRC
   calculation.  Limitations have to be imposed on the geometry in order for
   this keyword to work.  These are  (a)  the  input  geometry  must  be  in
   cartesian coordinates, (b) the first three atoms must not be coaxial, (c)
   triatomic systems are not allowed (See geometry specification - triatomic
   systems are in internal coordinates, by definition.)

        Put the velocity vector after the geometry as three data  per  line,
   representing  the  x, y, and z components of velocity for each atom.  The
   units of velocity are centimeters per second.

        The velocity vector  will  be  rotated  so  as  to  suit  the  final
   cartesian coordinate orientation of the molecule.
   
        If KINETIC=n.n is also specified, the velocity vector will be scaled
   to  equal  the  velocity corresponding to n.n kcal/mole.  This allows the
   user to define the direction of the velocity  vector;  the  magnitude  is
   given by KINETIC=n.n.
         
                                 
\subsection*{WILLIAMS (C)}   
        Within the ESP calculation, the Connolly  surface  is  used  as  the
   default.   If  the  surface  generation  procedure  of Donald Williams is
   wanted, the keyword WILLIAMS should be used.


                                
\subsection*{X-PRIORITY (O)}
        In  a  DRC  calculation,  results  will  be  printed  whenever   the
   calculated  geometry  changes  by 0.05~\AA.  The geometry change is
   defined as the linear sum of the translation vectors of  motion  for  all
   atoms in the system.  Abbreviation, X-PRIO.


                              
\subsection*{X-PRIORITY=n.nn (O)}
        In  a  DRC  calculation,  results  will  be  printed  whenever   the
   calculated geometry changes by n.nn~\AA.


                                    
\subsection*{XYZ (W)}
        The SADDLE calculation quite often fails due to faulty definition of
   the  second  geometry because the dihedrals give a lot of difficulty.  To
   make this option easier to use, XYZ was developed.  A  calculation  using
   XYZ runs entirely in cartesian coordinates, thus eliminating the problems
   associated with dihedrals.  The connectivity of the two  systems  can  be
   different,  but the numbering must be the same.  Dummy atoms can be used;
   these will be removed at the start of the run.  A  new  numbering  system
   will be generated by the program, when necessary.
   
        XYZ is also  useful  for  removing  dummy  atoms  from  an  internal
   coordinate file; use XYZ and 0SCF.
   
        If a large ring system is being optimized, sometimes the closure  is
   difficult, in which case XYZ will normally work.
   
        Except for SADDLE, do not use XYZ by  default:   use  it  only  when
   something goes wrong!
   
        In order for XYZ to be used, the supplied geometry must either be in
   cartesian coordinates or, if internal coordinates are used, symmetry must
   not be used, and all coordinates must be flagged  for  optimization.   If
   dummy  atoms  are  present,  only 3N-6 coordinates need to be flagged for
   optimization.
   
        If at all possible, the first 3 atoms should  be  real.   Except  in
   SADDLE,  XYZ  will still work if one or more dummy atoms occur before the
   fourth real atom, in which  case  more  than  3N-6  coordinates  will  be
   flagged  for  optimization.   This  could  cause difficulties with the EF
   method,  which  is  why  dummy  atoms  at  the  start  of  the   geometry
   specification  should be avoided.  The coordinates to be optimized depend
   on the internal coordinate definition of real atoms 1, 2, and 3.  If  the
   position  of  any  of  these  atoms  depends  on  dummy  atoms,  then the
   optimization flags will be different from the case where the first  three
   atoms defined are all real.  The geometry is first converted to cartesian
   coordinates and dummy atoms excluded.  The cartesian  coordinates  to  be
   optimized are:
\begin{verbatim}   
    Atoms  R R R  R R X  R X R  X R R  R X X  X R X  X X R  X X X  
   
           X Y Z  X Y Z  X Y Z  X Y Z  X Y Z  X Y Z  X Y Z  X Y Z  
   Atom 1         
        2  +      +      + +    + +    + + +  + +    + + +  + + +  
        3  + +    + + +  + + +  + + +  + + +  + + +  + + +  + + +  
     4 on  + + +  + + +  + + +  + + +  + + +  + + +  + + +  + + +  
\end{verbatim}   

    Where R and X apply to real and dummy atoms in the internal coordinate
    Z-matrix, and atoms 1, 2, 3, and 4 are the real atoms in cartesian
    coordinates.  A `+' means that the relevant coordinate is flagged for
    optimization.  Note that the number of flagged coordinates varies from
    $3N-6$ to $3N-3$, atom 1 is never optimized.
   
\section{Keywords that go together}
 Normally only a subset of keywords are used in any  given  piece  of
 research.  Keywords which are related to each other in this way are:
\begin{enumerate}   
\item In getting an SCF:  SHIFT, PULAY, ITRY, CAMP, SCFCRT, 1SCF, PL.
\item In C.I. work:  SINGLET, DOUBLET,  etc.,  OPEN(n,m),  C.I.=(n,m),
            LARGE, MECI, MS=n, VECTORS, ESR, ROOT=n, MICROS.
\item In excited states: UHF with (TRIPLET, QUARTET, etc.), C.I.=n, C.I.=(n,m).
\item In geometry optimization:
\begin{enumerate}   
            \item  Using BFGS:  GNORM=n.n, XYZ, PRECISE.
            \item Using EF:  GNORM=n.n, XYZ, PRECISE
            \item Using NLLSQ:  GNORM=n.n, XYZ, PRECISE
            \item Using SIGMA:  GNORM=n.n, XYZ, PRECISE
\end{enumerate}   
\item In Gaussian work:  AIGIN, AIGOUT, AIDER.   
\item In SADDLE:  XYZ, BAR=n.n
\end{enumerate}
%%%%%%%%%%%%%%%%%%%%%%%%%%%%%%%%%%%%%%%%%%%%%%%%%%%%%%%%%%%%%%%%%%%%%

\chapter{Geometry specification}
   FORMAT:  The geometry is read in using essentially ``Free-Format''  of
   FORTRAN-77.   In  fact, a character input is used in order to accommodate
   the  chemical  symbols,  but  the  numeric  data  can  be   regarded   as
   ``free-format''.index{data!free-format}   
   This  means  that  integers  and  real  numbers  can  be
   interspersed, numbers can be separated by one or more spaces, a
   tab\index{data!tabs in}
   and/or  by  one comma.  If a number is not specified, its value is set to
   zero.\index{data!commas in}

        The geometry can be defined in terms of either internal or cartesian
   coordinates.   

\section{Internal coordinate definition}
        For any one atom (i) this consists of  an  interatomic  distance  in
   Angstroms  from  an  already-defined  atom  (j),  an interatomic angle in
   degrees between atoms i and j and an already defined k, (k and j must  be
   different  atoms), and finally a torsional angle in degrees between atoms
   i, j, k, and an already defined atom l (l cannot be the same as k or  j).
   See also dihedral angle coherency.\index{internal coordinate definition}

Exceptions:
\begin{enumerate}
\item Atom 1 has no coordinates at all:  this is the origin.
\item Atom 2 must be connected to atom 1 by  an  interatomic  distance
     only.
\item Atom 3 can be connected to atom 1 or 2, and must make  an  angle
     with  atom  2  or  1  (thus  3--2--1  or 3--1--2); no dihedral is
     possible for atom 3.  By default, atom 3 is connected to atom 2.
\end{enumerate}
               

\subsection{Constraints}
\begin{enumerate}
\item  Interatomic distances must be greater than zero.  Zero Angstroms
    is  acceptable  only  if  the  parameter  is symmetry-related to
    another atom, and is the dependent function.

\item Angles must be in the  range  0.0  to  180.0,  inclusive.   This
    constraint  is for the benefit of the user only; negative angles
    are the result of errors in the construction  of  the  geometry,
    and  angles  greater  than  180  degrees are fruitful sources of
    errors in the dihedrals.

\item Dihedrals angles must be definable.  If atom i makes a  dihedral
    with atoms j, k, and l, and the three atoms j, k, and l are in a
    straight line, then the dihedral has no definable angle.  During
    the  calculation this constraint is checked continuously, and if
    atoms j, k, and l lie within 0.02 Angstroms of a straight  line,
    the calculation will output an error message and then stop.  Two
    exceptions to this constraint are:
\begin{enumerate}
\item if the angle is zero or 180 degrees, in which case the dihedral 
is not used.

\item if atoms j, k, and l lie in an  exactly  straight  line
    (usually  the result of a symmetry constraint), as in acetylene,
    acetonitrile, but-2-yne, etc.
\end{enumerate}
\end{enumerate}

If the exceptions are used, care must be taken to  ensure  that  the
program  does  not  violate these constraints during any optimizations or
during any calculations of derivatives - see also FORCE.

                      
\subsubsection{Conversion to Cartesian Coordinates}
\index{coordinates!internal to Cartesian}
By definition, atom 1 is at the origin of cartesian coordinate space---be 
careful, however, if atom 1 is a dummy atom.  Atom 2 is defined as
   lying on the positive X axis --- for atom 2, Y=0 and Z=0.  Atom  3  is  in
   the  X-Y  plane unless the angle 3--2--1 is exactly 0 or 180 degrees.  Atom
   4, 5, 6, etc.  can lie anywhere in 3-D space.\index{coordinates!Cartesian}
 
\section{Gaussian Z-matrices}
 With certain limitations, geometries can  now  be  specified  within
 MOPAC using the Gaussian Z-matrix format.\index{Gaussian coordinates}

\subsubsection{Exceptions to the full Gaussian standard}
\begin{enumerate} 
\item  The option of defining an atom's position by  one  distance  and
       two  angles  is  not  allowed.   In other words, the N4 variable
       described in the Gaussian manual must  either  be  zero  or  not
       specified.   MOPAC  requires the geometry of atoms to be defined
       in terms of, at most, one distance, one angle, and one dihedral.
 
\item  Gaussian cartesian coordinates are not supported.
 
\item Chemical symbols must not be followed by an integer  identifying
 the  atom.  Numbers after a symbol are used by MOPAC to indicate
 isotopic mass.  If labels are desired, they should  be  enclosed
 in parentheses, thus \verb/"Cl(on C5)34.96885"/.

\item The connectivity (N1, N2, N3) must be integers.  Labels are  not
          allowed.
\end{enumerate} 
 
\subsubsection{Specification of Gaussian Z-matrices}
 The information contained in the Gaussian Z-matrix is  identical  to
 that  in a MOPAC Z-matrix.  The order of presentation is different.  Atom
 N, (real or dummy) is specified in the format:
 \begin{verbatim}
 Element   N1   Length  N2  Alpha   N3  Beta
\end{verbatim}
 where Element is the same as for the MOPAC Z-matrix.  N1, N2, and N3  are
 the  connectivity,  the  same as the MOPAC Z-matrix NA, NB, and NC:  bond
 lengths are between N and N1, angles  are  between  N,  N1  and  N2,  and
 dihedrals are between N, N1, N2, and N3.  The same rules apply to N1, N2,
 and N3 as to NA, NB, and NC.
 
      Length, Alpha, and  Beta  are  the  bond  lengths,  the  angle,  and
 dihedral.   They  can be `real', e.g.  1.45, 109.4, 180.0, or `symbolic'.
 A symbolic is an alphanumeric string of up to 8  characters,  e.g.   R51,
 A512,  D5213, CH, CHO, CHOC, etc.  Two or more symbolics can be the same.
 Dihedral symbolics can optionally be preceeded by a minus sign, in  which
 case  the  value  of  the  dihedral  is  the negative of the value of the
 symbolic.  This is the equivalent of the normal MOPAC SYMMETRY operations
 1, 2, 3, and 14.
 
      If an internal coordinate is real, it will not be  optimized.   This
 is  the  equivalent  of  the MOPAC optimization flag ``0''.  If an internal
 coordinate is symbolic, it can be optimized.
 
      The Z-matrix is terminated by a blank line, after  which  comes  the
 starting values of the symbolics, one per line.  If there is a blank line
 in this set, then all symbolics  after  the  blank  line  are  considered
 fixed;  that  is,  they  will not be optimized.  The set before the blank
 line will be optimized.

Example of Gaussian Z-matrix geometry specification
\begin{verbatim}
 Line 1    AM1
 Line 2  Ethane                                   
 Line 3  
 Line 4    C  
 Line 5    C     1     r21    
 Line 6    H     2     r32       1     a321   
 Line 7    H     2     r32       1     a321      3  d4213
 Line 8    H     2     r32       1     a321      3 -d4213
 Line 9    H     1     r32       2     a321      3   60.
 Line 10   H     1     r32       2     a321      3  180.
 Line 11   H     1     r32       2     a321      3  d300
 Line 12 
 Line 13      r21        1.5
 Line 14      r32        1.1
 Line 15      a321     109.5
 Line 16      d4313    120.0
 Line 17
 Line 18      d300     300.0
 Line 19 
\end{verbatim}


\section{Cartesian coordinate definition}
 A definition of geometry in cartesian coordinates  consists  of  the
   chemical  symbol  or atomic number, followed by the cartesian coordinates
   and optimization flags but no connectivity.\index{geometry!flags for}

   MOPAC uses the lack  of  connectivity  to  indicate  that  cartesian
   coordinates  are  to  be used.  A unique case is the triatomics for which
   only internal coordinates are allowed.  This  is  to  avoid  conflict  of
   definitions:   the  user does not need to define the connectivity of atom
   2, and can elect to use the  default  connectivity  for  atom  3.   As  a
   result,  a  triatomic may have no explicit connectivity defined, the user
   thus taking  advantage  of  the  default  connectivity.   Since  internal
   coordinates  are  more commonly used than cartesian, the above choice was
   made.

        If the keyword XYZ is absent every coordinate  must  be  marked  for
   optimization.   If  any  coordinates are not to be optimized, the keyword
   XYZ must be present.  The coordinates of all atoms, including atoms 1,  2
   and  3  can  be  optimized.   Dummy atoms should not be used, for obvious
   reasons.
 
\section{Conversion between various formats}
 MOPAC can accept any of the  following  formats:   cartesian,  MOPAC
 internal  coordinates, and Gaussian internal coordinates.  Both MOPAC and
 Gaussian Z-matrices can also  contain  dummy  atoms.   Internally,  MOPAC
 works  with  either  a  cartesian coordinate set (if XYZ is specified) or
 internal coordinates (the default).  If the 0SCF option is requested, the
 geometry defined on input will be printed in MOPAC Z-matrix format, along
 with other optional formats.
 
      The type(s) of geometry printed at the end  of  a  0SCF  calculation
 depend only on the keywords XYZ, AIGOUT, and NOXYZ.  The geometry printed
 is independent of the type of  input  geometry,  and  therefore  makes  a
 convenient conversion mechanism.
 
      If XYZ is present, all dummy atoms  are  removed  and  the  internal
 coordinate  definition remade.  All symmetry relations are lost if XYZ is
 used.
 
      If NOXYZ is present, cartesian coordinates will not be printed.
 
      If AIGOUT is present, a data set using Gaussian Z-matrix  format  is
 printed.
 
      Note:  (1) Only if  the  keyword  XYZ  is  absent  and  the  keyword
 SYMMETRY  present in a MOPAC internal coordinate geometry, or two or more
 internal coordinates in a Gaussian Z-matrix have the same  symbolic  will
 symmetry be present in the MOPAC or Gaussian geometries output.  (2) This
 expanded use of 0SCF replaces the program  DDUM,  supplied  with  earlier
 copies of MOPAC.
 
\section{Definition of elements and isotopes}
 Elements are defined in terms  of  their  atomic  numbers  or  their
 chemical symbols, case insensitive.\index{isotopes!specification of}  
 Thus, chlorine could be specified as
 17, or Cl.  In Version 6, only main-group elements and transition  metals
 for which the `d' shell is full are 
 available.\index{elements!specification of}
 
 Acceptable symbols for MNDO are:\index{MNDO!elements in}
\begin{verbatim}
         Elements                      Dummy atom, sparkles and
                                         Translation Vector
   H
  Li  *          B  C  N  O  F         
  Na' *         Al Si  P  S Cl           +                       o
   K' * ...  Zn  * Ge  *  * Br         XX  Cb  ++   +  --   -  Tv
  Rb' * ...   *  * Sn  *  *  I         99 102 103 104 105 106 107
  *   * ...  Hg  * Pb  *  

'  These symbols refer to elements which lack a basis set.
+  This is the dummy atom for assisting with geometry specification.
*  Element not parameterized.
o  This is the translation vector for use with polymers.
\end{verbatim}

      Old parameters for some elements are available.  These are  provided
 to  allow compatibility with earlier copies of MOPAC.  To use these older
 parameters, use a keyword composed of the chemical symbol followed by the
   year  of  publication  of  the parameters.  Keywords currently available:
   Si1978, S1978.

For AM1, acceptable symbols are:\index{AM1!elements in}
\begin{verbatim}
       Elements                      Dummy atom, sparkles and
                                       Translation Vector
 H
 *  *          B  C  N  O  F         
Na' *         Al Si  P  S Cl           +                       o
 K' * ...  Zn  * Ge  *  * Br         XX  Cb  ++   +  --   -  Tv
Rb' * ...   *  * Sn  *  *  I         99 102 103 104 105 106 107
*   * ...  Hg  *  *  *  
\end{verbatim}

     If users need to use other elements, such as beryllium or lead, they
can  be  specified,  in  which case MNDO-type atoms will be used.  As the
behavior of such systems is not well investigated, users are cautioned to
exercise  unusual  care.   To  alert users to this situation, the keyword
PARASOK is defined.

For PM3, acceptable symbols are:
\begin{verbatim}
        Elements                      Dummy atom, sparkles and
                                        Translation Vector

  H
  *  Be          *  C  N  O  F         
 Na' Mg         Al Si  P  S Cl           +                       o
  K' * ...  Zn  Ga Ge As Se Br         XX  Cb  ++   +  --   -  Tv
 Rb' * ...  Cd  In Sn Sb Te  I         99 102 103 104 105 106 107
 *   * ...  Hg  Tl Pb Bi  

      Diatomics Parameterized within the MINDO/3 Formalism
     H   B   C   N   O   F  Si   P   S  Cl     A star (*) indicates
   -----------------------------------------   that the atom-pair is 
  H  *   *   *   *   *   *   *   *   *   *     parameterized within 
  B  *   *   *   *   *   *                     MINDO/3.
  C  *   *   *   *   *   *   *   *   *   *
  N  *   *   *   *   *   *           *   *
  O  *   *   *   *   *   *       *   *
  F  *   *   *   *   *   *       *
 Si  *       *               *
  P  *       *       *   *       *       *
  S  *       *   *   *               *   *
 Cl  *       *   *               *   *   *
\end{verbatim}

Note:  MINDO/3  should  now  be  regarded  as  being  of  historical
interest  only.   MOPAC  contains  the original parameters.  These do not
reproduce the original reported results in the case of P, Si, or S.   The
original  work  was  faulty, see G. Frenking, H. Goetz, and F. Marschner,
{\em J. Am. Chem. Soc.}, 100:5295 (1978).  
Re-optimized parameters  for  P--C  and  P--Cl
were derived later which gave better results.  These are:
\begin{itemize}
\item Alpha(P--C):  0.8700  G. Frenking, H. Goetz, F. Marschner, 
\item Beta(P--C):   0.5000  {\em J. Am. Chem. Soc.}, 100:5295-5296 (1978).
\item Alpha(P--Cl): 1.5400  G. Frenking, F. Marschner, H. Goetz, 
\item Beta(P--Cl):  0.2800  {\em Phosphorus and Sulfur}, 8:337-342 (1980).
\end{itemize}
  
Although better than the original parameters, these  have  not  been
adopted  within MOPAC because to do so at this time would prevent earlier
calculations from being duplicated.  Parameters for P--O and P--F have been
added:    these   were   abstracted   from  Frenking's  1980  paper.   No
inconsistency is involved as MINDO/3 historically did not have P--O or P--F
parameters.

Extra entities available to MNDO, MINDO/3, AM1 and PM3:
\begin{verbatim}
      +     A 100% ionic alkali metal.
     ++     A 100% ionic alkaline earth metal.
      -     A 100% ionic halogen-like atom
     --     A 100% ionic group VI-like atom.
     Cb     A special type of monovalent atom
\end{verbatim}

     Elements 103, 104, 105, and 106 are the sparkles; elements 11 and 19
are  sparkles  tailored  to  look like the alkaline metal ions; Tv is the
   translation vector for polymer calculations.  See ``Full description  of
   sparkles'' in Chapter~6.\index{sparkles}

Element 102, symbol Cb, is designed to satisfy valency  requirements
   of atoms for which some bonds are not completed.  Thus in ``solid'' diamond
   the usual way to complete the normal valency in a cluster model is to use
   hydrogen  atoms.  This approach has the defect that the electronegativity
   of hydrogen is different from that of carbon.  The ``capped bond''  atom,
   Cb, is designed to satisfy these valency requirements without acquiring a
   net charge.\index{capped bonds}

        Cb behaves like a monovalent atom, with the exception  that  it  can
   alter its electronegativity to achieve an exactly zero charge in whatever
   environment it finds itself.  It is thus all things  to  all  atoms.   On
   bonding to hydrogen it behaves similar to a hydrogen atom.  On bonding to
   fluorine it behaves like a very electronegative atom.  If several  capped
   bond  atoms  are  used,  each will behave independently.  Thus if the two
   hydrogen atoms in formic acid were replaced by Cb's then  each  Cb  would
   independently become electroneutral.

        Capped bonds internal coordinates should not be optimized.  A  fixed
   bond-length of $1.7$~\AA\ is recommended, if two Cb are on one atom, a
   contained angle of $109.471221$ degrees is suggested, and if three  Cb  are
   on  one  atom, a contained dihedral of $-120$ degrees (note sign) should be
   used.

        Element 99, X, or XX is known as a dummy atom, and is  used  in  the
   definition  of  the  geometry;  it  is  deleted  automatically  from  any
   cartesian coordinate geometry files.  Dummy  atoms  are  pure  mathematic
   points,  and  are  useful in defining geometries; for example, in ammonia
   the definition of C3v symmetry is facilitated by using one dummy atom and
   symmetry relating the three hydrogens to it.

        Output normally only gives chemical symbols.

        Isotopes are used in  conjunction  with  chemical  symbols.   If  no
   isotope is specified, the average isotopic mass is used, thus chlorine is
   35.453.  This is different from some earlier versions of MOPAC, in  which
   the most abundant isotope was used by default.  This change was justified
   by the removal of any ambiguity in the  choice  of  isotope.   Also,  the
   experimental  vibrational  spectra  involve  a mixture of isotopes.  If a
   user wishes to specify any specific isotope it should immediately  follow
   the  chemical  symbol  (no  space),  e.g.,  H2, H2.0140, C(meta)13,  or
   C13.00335.

   The sparkles ++, +, --, and $-$ have no mass; if they are to  be  used
   in a force calculation, then appropriate masses should be used.

        Each internal coordinate is followed by an integer, to indicate  the
   action to be taken.\index{grid map}
\begin{verbatim}
      Integer                        Action
        1                Optimize the internal coordinate.
        0                Do not optimize the internal coordinate.
       -1                Reaction coordinate, or grid index.
\end{verbatim}

   Remarks: Only one reaction coordinate is allowed, but this can be made more
   versatile  by the use of SYMMETRY.  If a reaction coordinate is used, the
   values of the reaction coordinate should  follow  immediately  after  the
   geometry   and  any  symmetry  data.   No  terminator  is  required,  and
   free-format-type input is acceptable.
   \index{reaction coordinate!specification}

   If two ``reaction coordinates'' are used, then MOPAC assumes that  the
   two-dimensional  space  in  the  region of the supplied geometry is to be
   mapped.  The two dimensions to be mapped are in the plane defined by  the
   ``$-1$'' labels. Step  sizes  in the two directions must be supplied using
   STEP1 and STEP2 on the keyword line.

        Using internal coordinates, the first atom has  three  unoptimizable
   coordinates,  the second atom two, (the bond-length can be optimized) and
   the third atom has one  unoptimizable  coordinate.   None  of  these  six
   unoptimizable  coordinates  at the start of the geometry should be marked
   for optimization.  If any are so marked, a  warning  is  given,  but  the
   calculation will continue.

In cartesian coordinates all parameters can be optimized.

\section{Examples of coordinate definitions}
   Two examples will be given.  The first is formic acid, HCOOH, and is
   presented  in  the  normal  style  with  internal  coordinates.   This is
   followed by formaldehyde, presented in such a manner as to demonstrate as
   many different features of the geometry definition as
possible.\index{coordinates!examples}

\begin{verbatim} 
 MINDO/3
 Formic acid  
 Example of normal geometry definition
   O                                        Atom 1 needs no coordinates.
   C    1.20 1                              Atom 2 bonds to atom 1.
   O    1.32 1  116.8 1            2  1     Atom 3 bonds to atom 2 and 
                                            makes an angle with atom 1. 
   H    0.98 1  123.9 1    0.0 0   3  2  1  Atom 4 has a dihedral of 0.0
                                            with atoms 3, 2 and 1.
   H    1.11 1  127.3 1  180.0 0   2  1  3 
   0    0.00 0    0.0 0    0.0 0   0  0  0  
\end{verbatim}

        Atom 2, a carbon, is bonded to  oxygen  by  a  bond-length  of  1.20
   Angstroms,  and to atom 3, an oxygen, by a bond-length of 1.32 Angstroms.
   The O-C-O angle is 116.8 degrees.  The first hydrogen is  bonded  to  the
   hydroxyl  oxygen  and  the  second hydrogen is bonded to the carbon atom.
   The H-C-O-O dihedral angle is 180 degrees.

        MOPAC can generate data-files, both in the Archive  files,  and  at
   the  end  of  the normal output file, when a job ends prematurely due to
   time restrictions.  Note that the data are all neatly  lined  up.   This
   is,  of  course, characteristic of machine-generated data, but is useful
   when checking for errors.

\subsubsection{Format of internal coordinates in ARCHIVE file}
\begin{verbatim}
 O    0.000000 0    0.000000 0    0.000000 0    0   0   0
 C    1.209615 1    0.000000 0    0.000000 0    1   0   0
 O    1.313679 1  116.886168 1    0.000000 0    2   1   0
 H    0.964468 1  115.553316 1    0.000000 0    3   2   1
 H    1.108040 1  128.726078 1  180.000000 0    2   1   3
 0    0.000000 0    0.000000 0    0.000000 0    0   0   0
\end{verbatim}

   Polymers are defined by the presence of a translation  vector.   In
   the  following example, polyethylene, the translation vector spans three
   monomeric units, and is 7.7 Angstroms long.  Note in  this  example  the
   presence  of  two  dummy  atoms.   These  not  only  make  the  geometry
   definition easier but also allow the translation vector to be  specified
   in terms of distance only, rather than both distance and angles.
   \index{data!for polythene}

Example of polymer coordinates from ARCHIVE file:
\begin{verbatim}
     T=20000 
        POLYETHYLENE, CLUSTER UNIT :  C6H12 
    
      C    0.000000  0    0.000000  0    0.000000  0    0   0   0
      C    1.540714  1    0.000000  0    0.000000  0    1   0   0
      C    1.542585  1  113.532306  1    0.000000  0    2   1   0
      C    1.542988  1  113.373490  1  179.823613  1    3   2   1
      C    1.545151  1  113.447508  1  179.811764  1    4   3   2
      C    1.541777  1  113.859804  1 -179.862648  1    5   4   3
     XX    1.542344  1  108.897076  1 -179.732346  1    6   5   4
     XX    1.540749  1  108.360151  1 -178.950271  1    7   6   5
      H    1.114786  1   90.070026  1  126.747447  1    1   3   2
      H    1.114512  1   90.053136  1 -127.134856  1    1   3   2
      H    1.114687  1   90.032722  1  126.717889  1    2   4   3
      H    1.114748  1   89.975504  1 -127.034513  1    2   4   3
      H    1.114474  1   90.063308  1  126.681098  1    3   5   4
      H    1.114433  1   89.915262  1 -126.931090  1    3   5   4
      H    1.114308  1   90.028131  1  127.007845  1    4   6   5
      H    1.114434  1   90.189506  1 -126.759550  1    4   6   5
      H    1.114534  1   88.522263  1  127.041363  1    5   7   6
      H    1.114557  1   88.707407  1 -126.716355  1    5   7   6
      H    1.114734  1   90.638631  1  127.793055  1    6   8   7
      H    1.115150  1   91.747016  1 -126.187496  1    6   8   7
     Tv    7.746928  1    0.000000  0    0.000000  0    1   7   8
      0    0.000000  0    0.000000  0    0.000000  0    0   0   0
\end{verbatim}

%%%%%%%%%%%%%%%%%%%%%%%%%%%%%%%%%%%%%%%%%%%%%%%%%%%%%%%%%%%%%%%%%%%%%
\chapter{Examples}
   In this chapter various examples of data-files are described.  With
   MOPAC  comes two sets of data for running calculations.  One of these is
   called MNRSD1.DAT, and this will now be described.\index{data!MNRSD1!input}

\section{MNRSD1 test data file for formaldehyde}
 The following file is suitable for generating the results described
 in the next section, and would be suitable for debugging data.
\begin{verbatim}
Line  1:         SYMMETRY 
Line  2:  Formaldehyde, for Demonstration Purposes 
Line  3: 
Line  4:   O 
Line  5:   C 1.2 1
Line  6:   H 1.1 1 120 1
Line  7:   H 1.1 0 120 0 180 0 2 1 3 
Line  8: 
Line  9:   3 1 4
Line 10:   3 2 4
Line 11:
\end{verbatim}                 

These data could be more neatly written as:
\begin{verbatim}
Line  1:         SYMMETRY 
Line  2:  Formaldehyde, for Demonstration Purposes 
Line  3: 
Line  4:   O    
Line  5:   C    1.20  1                         1  
Line  6:   H    1.10  1  120.00  1              2  1  
Line  7:   H    1.10  0  120.00  0  180.00  0   2  1  3 
Line  8:   
Line  9:   3,   1,   4,
Line 10:   3,   2,   4,
Line 11:
\end{verbatim}
      
These two data-files will produce identical results files.

   In all geometric specifications, care must be taken in defining the
   internal  coordinates to ensure that no three atoms being used to define
   a fourth atom's dihedral angle ever fall into a straight line.  This can
   happen in the course of a geometry optimization, in a SADDLE calculation
   or in following a reaction  coordinate.   If  such  a  condition  should
   develop,   then   the  position  of  the  dependent  atom  would  become
   ill-defined.

\section{MOPAC output for test-data file MNRSD1}\index{data!MNRSD1!output}
\begin{verbatim}
 ****************************************************************************
 ** FRANK J. SEILER RES. LAB., U.S. AIR FORCE ACADEMY, COLO. SPGS., CO. 80840
 ****************************************************************************
                                MNDO CALCULATION RESULTS              Note 1
 ****************************************************************************
 *          MOPAC:  VERSION  6.00               CALC'D.  4-OCT-90     Note 2
 *  SYMMETRY - SYMMETRY CONDITIONS TO BE IMPOSED
 *   T=      - A TIME OF  3600.0 SECONDS REQUESTED
 *  DUMP=N   - RESTART FILE WRITTEN EVERY  3600.0 SECONDS
 ********************************************************************043BY043
     PARAMETER DEPENDENCE DATA
        REFERENCE ATOM      FUNCTION NO.    DEPENDENT ATOM(S)
            3                  1             4
            3                  2             4
             DESCRIPTIONS OF THE FUNCTIONS USED
   1      BOND LENGTH    IS SET EQUAL TO THE REFERENCE BOND LENGTH   
   2      BOND ANGLE     IS SET EQUAL TO THE REFERENCE BOND ANGLE    
         SYMMETRY                                                     Note 3
  Formaldehyde, for Demonstration Purposes                                      
                                                                                
    ATOM   CHEMICAL  BOND LENGTH    BOND ANGLE     TWIST ANGLE
   NUMBER  SYMBOL    (ANGSTROMS)     (DEGREES)      (DEGREES)
    (I)                  NA:I          NB:NA:I      NC:NB:NA:I    NA NB NC
      1      O                                                        Note 4
      2      C         1.20000 *                                  1
      3      H         1.10000 *      120.00000 *                 2   1
      4      H         1.10000        120.00000     180.00000     2   1   3
          CARTESIAN COORDINATES 
    NO.       ATOM         X         Y         Z
     1         O        0.0000    0.0000    0.0000
     2         C        1.2000    0.0000    0.0000                    Note 5
     3         H        1.7500    0.9526    0.0000
     4         H        1.7500   -0.9526    0.0000
  H: (MNDO):  M.J.S. DEWAR, W. THIEL, J. AM. CHEM. SOC., 99, 4899, (1977) 
  C: (MNDO):  M.J.S. DEWAR, W. THIEL, J. AM. CHEM. SOC., 99, 4899, (1977) 
  O: (MNDO):  M.J.S. DEWAR, W. THIEL, J. AM. CHEM. SOC., 99, 4899, (1977) 
          RHF CALCULATION, NO. OF DOUBLY OCCUPIED LEVELS =  6
            INTERATOMIC DISTANCES
                  O  1       C  2       H  3       H  4
 ------------------------------------------------------
     O    1   0.000000
     C    2   1.200000   0.000000
     H    3   1.992486   1.100000   0.000000                          Note 6
     H    4   1.992486   1.100000   1.905256   0.000000
 CYCLE:   1 TIME:   0.75 TIME LEFT:   3598.2 GRAD.:     6.349 HEAT:-32.840147
 CYCLE:   2 TIME:   0.37 TIME LEFT:   3597.8 GRAD.:     2.541 HEAT:-32.880103   
 HEAT OF FORMATION TEST SATISFIED                                     Note 7
 PETERS TEST SATISFIED                                                Note 8
 ---------------------------------------------------------------------------
         SYMMETRY                                                     Note 9
  Formaldehyde, for Demonstration Purposes                            Note 10
                                                                                
     PETERS TEST WAS SATISFIED IN BFGS            OPTIMIZATION        Note 11
     SCF FIELD WAS ACHIEVED                                           Note 12

                               MNDO    CALCULATION                    Note 13
                                                       VERSION  6.00  
                                                        4-OCT-90     
          FINAL HEAT OF FORMATION =        -32.88176 KCAL             Note 14
          TOTAL ENERGY            =       -478.11917 EV
          ELECTRONIC ENERGY       =       -870.69649 EV
          CORE-CORE REPULSION     =        392.57733 EV
          IONIZATION POTENTIAL    =         11.04198
          NO. OF FILLED LEVELS    =          6
          MOLECULAR WEIGHT        =     30.026
          SCF CALCULATIONS  =               15
          COMPUTATION TIME =   2.740 SECONDS                          Note 15
    ATOM   CHEMICAL  BOND LENGTH    BOND ANGLE     TWIST ANGLE
   NUMBER  SYMBOL    (ANGSTROMS)     (DEGREES)      (DEGREES)
    (I)                  NA:I          NB:NA:I      NC:NB:NA:I     NA   NB   NC
      1      O
      2      C         1.21678 *                                   1  Note 16
      3      H         1.10590 *      123.50259 *                  2    1
      4      H         1.10590        123.50259     180.00000      2    1    3
            INTERATOMIC DISTANCES
                  O  1       C  2       H  3       H  4
 ------------------------------------------------------
     O    1   0.000000
     C    2   1.216777   0.000000
     H    3   2.046722   1.105900   0.000000
     H    4   2.046722   1.105900   1.844333   0.000000

                  EIGENVALUES
-42.98352 -25.12201 -16.95327 -16.29819 -14.17549 -11.04198  0.85804  3.6768
   3.84990   7.12408                                                  Note 17
              NET ATOMIC CHARGES AND DIPOLE CONTRIBUTIONS
         ATOM NO.   TYPE          CHARGE        ATOM  ELECTRON DENSITY
           1          O          -0.2903          6.2903
           2          C           0.2921          3.7079              Note 18
           3          H          -0.0009          1.0009
           4          H          -0.0009          1.0009
 DIPOLE           X         Y         Z       TOTAL
 POINT-CHG.     1.692     0.000     0.000     1.692
 HYBRID         0.475     0.000     0.000     0.475                   Note 19
 SUM            2.166     0.000     0.000     2.166
          CARTESIAN COORDINATES 
    NO.       ATOM               X         Y         Z
     1         O                  0.0000    0.0000    0.0000
     2         C                  1.2168    0.0000    0.0000
     3         H                  1.8272    0.9222    0.0000
     4         H                  1.8272   -0.9222    0.0000
          ATOMIC ORBITAL ELECTRON POPULATIONS
1.88270   1.21586   1.89126   1.30050   1.25532   0.86217   0.89095  0.69950
1.00087   1.00087                                                    Note 20
 TOTAL CPU TIME:             3.11 SECONDS
 == MOPAC DONE ==
\end{verbatim}

\begin{description}
\item[Note 1:] The banner indicates whether the calculation uses a  MNDO,
   MINDO/3,  AM1  or PM3 Hamiltonian; here, the default MNDO Hamiltonian is
   used.

\item[Note 2:] The Version number is a constant for any release of MOPAC,
   and  refers  to  the program, not to the Hamiltonians used.  The version
   number should be cited in any correspondence  regarding  MOPAC.   Users'
   own  in-house  modified  versions  of  MOPAC  will  have  a  final digit
   different from zero, e.g.  6.01.\index{version number!of MOPAC}
\index{MOPAC!version number}

        All the keywords used, along with a brief  explanation,  should  be
   printed  at  this  time.   If  a keyword is not printed, it has not been
   recognized by the program.  Keywords can  be  in  upper  or  lower  case
   letters,  or  any  mixture.  The cryptic message at the right end of the
   lower line of asterisks indicates the number of heavy  and  light  atoms
   this version of MOPAC is configured for.

\item[Note 3:]  Symmetry information is output to allow the user to verify
   that  the  requested symmetry functions have in fact been recognized and
   used.

\item[Note 4:]  The data for this example used a mixture of atomic numbers
   and chemical symbols, but the internal coordinate output is consistently
   in chemical symbols.

  The atoms in the system are, in order:
\begin{itemize}
\item Atom 1, an oxygen atom; this is  defined as being at the origin.
\item Atom 2, the carbon atom.  Defined  as  being  1.2  Angstroms
    from  the  oxygen  atom, it is located in the +x direction.  This
    distance is marked for optimization.
\item Atom 3, a  hydrogen  atom.   It  is  defined  as  being  1.1
    Angstroms  from  the  carbon  atom,  and  making  an angle of 120
    degrees with the oxygen atom.  The asterisks  indicate  that  the
    bond length and angle are both to be optimized.
\item Atom 4, a hydrogen atom.  The bond length supplied has  been
    overwritten with the symmetry-defined C-H bond length.  Atom 4 is
    defined as being 1.1 \AA\ from atom 2, making  a  bond-angle
    of  120  degrees with atom 1, and a dihedral angle of 180 degrees
    with atom 3.

         None  of  the  coordinates  of  atom  4   are   marked   for
    optimization.   The bond-length and angle are symmetry-defined by
    atom 3, and the  dihedral  is  group-theory  symmetry-defined  as
    being 180 degrees.  (The molecule is flat.)
\end{itemize}

\item[Note 5:]  The cartesian coordinates are calculated as follows:

               Stage 1:  The coordinate of the first  atom  is  defined  as
          being  at  the origin of cartesian space, while the coordinate of
          the second atom is defined as being displaced by its defined bond
          length  along  the  positive x-axis.  The coordinate of the third
          atom is defined as being displaced by its bond length in the  x-y
          plane,  from  either  atom 1 or 2 as defined in the data, or from
          atom 2 if no numbering is given.  The angle it makes with atoms 1
          and 2 is that given by its bond angle.

               The dihedral, which first appears in  the  fourth  atom,  is
          defined  according to the IUPAC  convention. Note:  This is
          different from previous versions of MNDO and MINDO/3,  where  the
          dihedral  had  the  opposite  chirality  to  that  defined by the
          IUPAC  convention.

               Stage 2:  Any dummy atoms are removed.  As  this  particular
          system contains no dummy atoms, nothing is done.

\item[Note 6:]  The  interatomic  distances  are  output  for  the  user's
   advice,  and a simple check made to insure that the smallest interatomic
   distance is greater than $0.8$~\AA.

  \item[Note 7:]  The geometry is optimized in  a  series  of  cycles,  each
   cycle consisting of a line search and calculation of the gradients.  The
   time given is the cpu  time for the cycle; time  left  is  the  total
   time  requested (here 100 seconds) less the cpu time since the start
   of the calculation (which  is  earlier  than  the  start  of  the  first
   cycle!).   These  times  can  vary  slightly  from cycle to cycle due to
   different options being used, for example whether or not two or more SCF
   calculations  need  to  be  done to ensure that the heat of formation is
   lowered.  The gradient is the scalar length in kcal/mole/Angstrom of the
   gradient vector.

  \item[Note 8:]  At the end of the BFGS geometry optimization a message  is
   given   which  indicates  how  the  optimization ended. All ``normal''
   termination messages contain the word ``satisfied''; other  terminations
   may give acceptable results, but more care should be taken, particularly
   regarding the gradient vector.

  \item[Notes 9, 10:]   The  keywords  used,  titles  and  comments  are
   reproduced here to remind the user of the name of the calculation.

  \item[Notes 11, 12:]  Two messages are given  here.   The  first  is  a
   reminder   of   how   the   geometry  was  obtained,  whether  from  the
   Broyden--Fletcher--Goldfarb--Shanno, Eigenvector Following, Bartel's or the
   McIver--Komornicki  methods.   For  any further results to be printed the
   second message must be as shown; when no SCF is obtained no results will
   be printed.

  \item[Note 13:]  Again, the results are headed with either MNDO or MINDO/3
   banners,  and  the version number.  The date has been moved to below the
   version number for convenience.

  \item[Note 14:]  The total energy of the system is  the  addition  of  the
   electronic  and nuclear terms.  The heat of formation is relative to the
   elements in their standard state.  The I.P.   is  the  negative  of  the
   energy  level  of  the  highest  occupied, or highest partially occupied
   molecular orbital (in accordance with Koopmans' theorem).

  \item[Note 15:]  Advice on time required for  the  calculation.   This  is
   obviously useful in estimating the times required for other systems.

  \item[Note 16:]  The fully optimized  geometry  is  printed  here.   If  a
   parameter  is not marked for optimization, it will not be changed unless
   it is a symmetry-related parameter.

  \item[Note 17:]  The  roots  are  the  eigenvalues  or  energy  levels  in
   electron  volts of the molecular orbitals.  There are six filled levels,
   therefore  the  HOMO  has  an  energy  of  -11.041eV;  analysis  of  the
   corresponding  eigenvector  (not  given  here)  shows  that it is mainly
   lone-pair on oxygen.  The eigenvectors form an orthonormal set.

  \item[Note 18:]  The charge on an atom is the sum  of  the  positive  core
   charge; for hydrogen, carbon, and oxygen these numbers are 1.0, 4.0, and
   6.0, respectively, and the negative of the number of valence  electrons,
   or  atom  electron  density on the atom, here 1.0010, 3.7079, and 6.2902
   respectively.

  \item[Note 19:]  The  dipole  is  the  scalar  of  the  dipole  vector  in
   cartesian  coordinates.   The  components of the vector coefficients are
   the point-charge dipole and the hybridization dipole.   In  formaldehyde
   there is no z-dipole since the molecule is flat.

  \item[Note 20:]  MNDO AM1, PM3, and MINDO/3 all use  the  Coulson  density
   matrix.   Only  the  diagonal  elements  of the matrix, representing the
   valence orbital  electron  populations,  will  be  printed,  unless  the
   keyword DENSITY is specified.
\end{description}

Extra lines are added as a result of user requests:
\begin{enumerate}
\item The total CPU time  for  the  job,  excluding  loading  of  the
   executable, is printed.
\item In order to know that MOPAC has ended, the message
  \verb/== MOPAC DONE ==/ is printed.
\end{enumerate}
%%%%%%%%%%%%%%%%%%%%%%%%%%%%%%%%%%%%%%%%%%%%%%%%%%%%%%%%%%%%%%%%%%%%%

\chapter{Testdata}
 TESTDATA.DAT, supplied with MOPAC~6.00,  is  a  single  large  job
 consisting  of several small systems, which are run one after the other.
 In order, the calculations run are:
\begin{enumerate}
 
    \item A FORCE calculation on formaldehyde.  The extra keywords at the
          start  are  to  be used later when TESTDATA.DAT acts as a SETUP
          file.  This unusual usage of a data set was made  necessary  by
          the need to ensure that a SETUP file existed.  If the first two
          lines are removed, the data set used in the example given below
          is generated.
 
    \item The vibrational frequencies of a  highly  excited  dication  of
          methane are calculated.  A non-degenerate state was selected in
          order  to  preserve  tetrahedral   symmetry   (to   avoid   the
          Jahn-Teller effects).
 
    \item Illustration of the use of the \& in the keyword line, and  of
          the new optional definition of atoms 2 and 3
 
    \item Illustration of Gaussian Z-matrix input.
 
    \item An example of Eigenvector Following,  to  locate  a  transition
          state.
 
    \item Use of SETUP.  Normally, SETUP would point to  a  special  file
          which  would contain keywords only.  Here, the only file we can
          guarantee exists, is the file being run, so  that  is  the  one
          used.
 
    \item Example of labelling atoms.
 
    \item This part of the test writes the density matrix  to  disk,  for
          later use.
 
    \item A simple calculation on water.
         
    \item The previous, optimized, geometry is to be used to  start  this
          calculation.
 
    \item The density matrix written out earlier is now used as input  to
          start an SCF.
\end{enumerate} 
 
 This example is taken from the first data-file in TESTDATA.DAT, and
 illustrates the working of a FORCE calculation.\index{data!TESTDATA!input}
 \index{THERMO!example of}

\section{Data file for a force calculation}
\begin{verbatim} 
  Line  1   nointer  noxyz + mndo dump=8
  Line  2    t=2000 + thermo(298,298) force isotope
  Line  3  ROT=2 
  Line  4    DEMONSTRATION OF MOPAC - FORCE AND THERMODYNAMICS CALCULATION
  Line  5    FORMALDEHYDE, MNDO ENERGY = -32.8819  See Manual.
  Line  6    O    
  Line  7    C    1.216487  1                                 1  0  0 
  Line  8    H    1.106109  1  123.513310  1                  2  1  0 
  Line  9    H    1.106109  1  123.513310  1  180.000000  1   2  1  3 
  Line 10    0    0.000000  0    0.000000  0    0.000000  0   0  0  0
\end{verbatim}

\section{Results file for the force calculation}\index{data!TESTDATA!output}
\begin{verbatim}
 ****************************************************************************
 ** FRANK J. SEILER RES. LAB., U.S. AIR FORCE ACADEMY, COLO. SPGS., CO. 80840
 ****************************************************************************
 
                                 MNDO CALCULATION RESULTS
 
 
  ***************************************************************************
  *          MOPAC:  VERSION  6.00               CALC'D. 12-OCT-90               
  *   T=      - A TIME OF  2000.0 SECONDS REQUESTED
  *  DUMP=N   - RESTART FILE WRITTEN EVERY     8.0 SECONDS
  *  FORCE    - FORCE CALCULATION SPECIFIED
  *  PRECISE  - CRITERIA TO BE INCREASED BY 100 TIMES
  *  NOINTER  - INTERATOMIC DISTANCES NOT TO BE PRINTED              Note 1
  *  ISOTOPE  - FORCE MATRIX WRITTEN TO DISK (CHAN. 9 )
  *  NOXYZ    - CARTESIAN COORDINATES NOT TO BE PRINTED
  *  THERMO   - THERMODYNAMIC QUANTITIES TO BE CALCULATED
  *  ROT      - SYMMETRY NUMBER OF  2 SPECIFIED
  *******************************************************************040BY040
                                           
   NOINTER  NOXYZ + MNDO DUMP=8
    T=2000 + THERMO(298,298) FORCE ISOTOPE
  ROT=2  PRECISE
    DEMONSTRATION OF MOPAC - FORCE AND THERMODYNAMICS CALCULATION
    FORMALDEHYDE, MNDO ENERGY = -32.8819  See Manual.
 
   ATOM   CHEMICAL  BOND LENGTH    BOND ANGLE     TWIST ANGLE
  NUMBER  SYMBOL    (ANGSTROMS)     (DEGREES)      (DEGREES)
   (I)                  NA:I          NB:NA:I      NC:NB:NA:I    NA  NB  NC
   ATOM   CHEMICAL  BOND LENGTH    BOND ANGLE     TWIST ANGLE
 
     1      O       
     2      C         1.21649  *                                 1
     3      H         1.10611  *     123.51331  *                2   1
     4      H         1.10611  *     123.51331  *  180.00000  *  2   1   3
   H: (MNDO):  M.J.S. DEWAR, W. THIEL, J. AM. CHEM. SOC., 99, 4899, (1977)
   C: (MNDO):  M.J.S. DEWAR, W. THIEL, J. AM. CHEM. SOC., 99, 4899, (1977)
   O: (MNDO):  M.J.S. DEWAR, W. THIEL, J. AM. CHEM. SOC., 99, 4899, (1977)
 
 
           RHF CALCULATION, NO. OF DOUBLY OCCUPIED LEVELS =  6
 
 
           HEAT OF FORMATION =  -32.881900 KCALS/MOLE
 
 
           INTERNAL COORDINATE DERIVATIVES
 
    ATOM  AT. NO.  BOND           ANGLE          DIHEDRAL
 
      1     O
      2     C     0.000604
      3     H     0.000110    -0.000054
      4     H     0.000110    -0.000054     0.000000
 
 
           GRADIENT NORM =   0.00063                             Note 2
 
 
           TIME FOR SCF CALCULATION =    0.45
 
 
           TIME FOR DERIVATIVES     =    0.32                    Note 3
 
           MOLECULAR WEIGHT =   30.03
                                           
            PRINCIPAL MOMENTS OF INERTIA IN CM(-1)
 
           A =    9.832732   B =    1.261998   C =    1.118449
 
 
 
            PRINCIPAL MOMENTS OF INERTIA IN UNITS OF 10**(-40)*GRAM-CM**2
 
           A =    2.846883   B =   22.181200   C =   25.028083
 
 
          ORIENTATION OF MOLECULE IN FORCE CALCULATION
 
     NO.       ATOM         X         Y         Z
 
      1         8       -0.6093    0.0000    0.0000
      2         6        0.6072    0.0000    0.0000
      3         1        1.2179    0.9222    0.0000
      4         1        1.2179   -0.9222    0.0000
 
 
     FIRST DERIVATIVES WILL BE USED IN THE CALCULATION OF SECOND DERIVATIVES
 
           ESTIMATED TIME TO COMPLETE CALCULATION =    36.96 SECONDS
  STEP:   1 TIME =     2.15 SECS, INTEGRAL =      2.15 TIME LEFT:   1997.08
  STEP:   2 TIME =     2.49 SECS, INTEGRAL =      4.64 TIME LEFT:   1994.59
  STEP:   3 TIME =     2.53 SECS, INTEGRAL =      7.17 TIME LEFT:   1992.06
  STEP:   4 TIME =     2.31 SECS, INTEGRAL =      9.48 TIME LEFT:   1989.75
  STEP:   5 RESTART FILE WRITTEN, INTEGRAL =     11.97 TIME LEFT:   1987.26
  STEP:   6 TIME =     2.43 SECS, INTEGRAL =     14.40 TIME LEFT:   1984.83
  STEP:   7 TIME =     2.32 SECS, INTEGRAL =     16.72 TIME LEFT:   1982.51
  STEP:   8 TIME =     2.30 SECS, INTEGRAL =     19.02 TIME LEFT:   1980.21
  STEP:   9 RESTART FILE WRITTEN, INTEGRAL =     22.17 TIME LEFT:   1977.06
  STEP:  10 TIME =     2.52 SECS, INTEGRAL =     24.69 TIME LEFT:   1974.54
  STEP:  11 TIME =     2.25 SECS, INTEGRAL =     26.94 TIME LEFT:   1972.29
  STEP:  12 TIME =     3.15 SECS, INTEGRAL =     30.09 TIME LEFT:   1969.14
 
 
            FORCE MATRIX IN MILLIDYNES/ANGSTROM
 0
                   O  1       C  2       H  3       H  4
  ------------------------------------------------------
      O    1   9.557495
      C    2   8.682982  11.426823
      H    3   0.598857   2.553336   3.034881
      H    4   0.598862   2.553344   0.304463   3.034886
 
 
           HEAT OF FORMATION =  -32.881900 KCALS/MOLE
 
 
            ZERO POINT ENERGY      18.002 KILOCALORIES PER MOLE      Note 4
                                           
     THE LAST 6 VIBRATIONS ARE THE TRANSLATION AND ROTATION MODES
     THE FIRST THREE OF THESE BEING TRANSLATIONS IN X, Y, AND Z, RESPECTIVELY
 
 
            NORMAL COORDINATE ANALYSIS
 
 
 
                                                                     Note 5
 ROOT NO.    1           2           3           4           5           6
 
       1209.90331  1214.67040  1490.52685  2114.53841  3255.93651  3302.12319
   
      1   0.00000     0.00000    -0.04158    -0.25182     0.00000     0.00067
      2   0.06810     0.00001     0.00000     0.00000     0.00409     0.00000
      3   0.00000    -0.03807     0.00000     0.00000     0.00000     0.00000
      4   0.00000     0.00000    -0.03819     0.32052     0.00000    -0.06298
      5  -0.13631    -0.00002     0.00000     0.00000     0.08457     0.00000
      6  -0.00002     0.15172     0.00000     0.00000     0.00000     0.00000
      7  -0.53308    -0.00005     0.55756     0.08893    -0.39806     0.36994
      8   0.27166     0.00003    -0.38524     0.15510    -0.53641     0.57206
      9   0.00007    -0.60187     0.00001     0.00000     0.00000     0.00000
     10   0.53307     0.00006     0.55757     0.08893     0.39803     0.36997
     11   0.27165     0.00003     0.38524    -0.15509    -0.53637    -0.57209
     12   0.00007    -0.60187     0.00001     0.00000     0.00000     0.00000
 
 
 
 
 ROOT NO.    7           8           9          10          11          12
 
         -0.00019    -0.00044    -0.00016     3.38368     2.03661    -0.76725
   
      1   0.25401     0.00000     0.00000     0.00000     0.00000     0.00000
      2   0.00000    -0.25401     0.00000     0.00000     0.00000    -0.17792
      3   0.00000     0.00000    -0.25401     0.00000    -0.19832     0.00000
      4   0.25401     0.00000     0.00000     0.00000     0.00000     0.00000
      5   0.00000    -0.25401     0.00000     0.00000     0.00000     0.17731
      6   0.00000     0.00000    -0.25401     0.00000     0.19764     0.00000
      7   0.25401     0.00000     0.00000     0.00000     0.00000    -0.26930
      8   0.00000    -0.25401     0.00000     0.00000     0.00000     0.35565
      9   0.00000     0.00000    -0.25401     0.70572     0.39642     0.00000
     10   0.25401     0.00000     0.00000     0.00000     0.00000     0.26930
     11   0.00000    -0.25401     0.00000     0.00000     0.00000     0.35565
     12   0.00000     0.00000    -0.25401    -0.70572     0.39642     0.00000
                                           
            MASS-WEIGHTED COORDINATE ANALYSIS
 
 
                                                                     Note 6
 
 ROOT NO.    1           2           3           4           5           6
 
       1209.90331  1214.67040  1490.52685  2114.53841  3255.93651  3302.12319
   
      1   0.00000     0.00000    -0.16877    -0.66231     0.00000     0.00271
      2   0.26985     0.00003     0.00000     0.00000     0.01649     0.00000
      3   0.00002    -0.15005     0.00000     0.00000     0.00000     0.00000
      4   0.00000     0.00000    -0.13432     0.73040     0.00001    -0.22013
      5  -0.46798    -0.00005     0.00000     0.00000     0.29524     0.00001
      6  -0.00006     0.51814     0.00000     0.00000     0.00000     0.00000
      7  -0.53018    -0.00005     0.56805     0.05871    -0.40255     0.37455
      8   0.27018     0.00003    -0.39249     0.10238    -0.54246     0.57918
      9   0.00007    -0.59541     0.00001     0.00000     0.00000     0.00000
     10   0.53018     0.00006     0.56806     0.05871     0.40252     0.37457
     11   0.27018     0.00003     0.39249    -0.10238    -0.54242    -0.57922
     12   0.00007    -0.59541     0.00001     0.00000     0.00000     0.00000
 
 
 
                                                                     Note 7
 ROOT NO.    7           8           9          10          11          12
 
         -0.00025    -0.00022    -0.00047     3.38368     2.03661    -0.76725
   
      1   0.72996     0.00000     0.00000     0.00000     0.00000     0.00000
      2   0.00000    -0.72996     0.00000     0.00000     0.00000    -0.62774
      3   0.00000     0.00000    -0.72996     0.00000    -0.66681     0.00000
      4   0.63247     0.00000     0.00000     0.00000     0.00000     0.00000
      5   0.00000    -0.63247     0.00000     0.00000     0.00000     0.54204
      6   0.00000     0.00000    -0.63247     0.00000     0.57578     0.00000
      7   0.18321     0.00000     0.00000     0.00000     0.00000    -0.23848
      8   0.00000    -0.18321     0.00000     0.00000     0.00000     0.31495
      9   0.00000     0.00000    -0.18321     0.70711     0.33455     0.00000
     10   0.18321     0.00000     0.00000     0.00000     0.00000     0.23848
     11   0.00000    -0.18321     0.00000     0.00000     0.00000     0.31495
     12   0.00000     0.00000    -0.18321    -0.70711     0.33455     0.00000
                                           
           DESCRIPTION OF VIBRATIONS
 
 
  VIBRATION   1            ATOM PAIR     ENERGY CONTRIBUTION          RADIAL
  FREQ.     1209.90       C 2 --  H 3           42.7% ( 79.4%)          12.6%
  T-DIPOLE   0.8545       C 2 --  H 4           42.7%                   12.6%
  TRAVEL     0.1199       O 1 --  C 2           14.6%                    0.0%
  RED. MASS  1.9377
 
  VIBRATION   2            ATOM PAIR     ENERGY CONTRIBUTION          RADIAL
  FREQ.     1214.67       C 2 --  H 3           45.1% ( 62.3%)           0.0%
  T-DIPOLE   0.1275       C 2 --  H 4           45.1%                    0.0%
  TRAVEL     0.1360       O 1 --  C 2            9.8%                    0.0%
  RED. MASS  1.5004
 
  VIBRATION   3            ATOM PAIR     ENERGY CONTRIBUTION          RADIAL
  FREQ.     1490.53       C 2 --  H 4           49.6% ( 61.5%)           0.6%
  T-DIPOLE   0.3445       C 2 --  H 3           49.6%                    0.6%
  TRAVEL     0.1846       O 1 --  C 2            0.9%                  100.0%
  RED. MASS  0.6639
 
  VIBRATION   4            ATOM PAIR     ENERGY CONTRIBUTION          RADIAL
  FREQ.     2114.54       O 1 --  C 2           60.1% (100.5%)         100.0%
  T-DIPOLE   3.3662       C 2 --  H 4           20.0%                   17.7%
  TRAVEL     0.0484       C 2 --  H 3           20.0%                   17.7%
  RED. MASS  6.7922
 
  VIBRATION   5            ATOM PAIR     ENERGY CONTRIBUTION          RADIAL
  FREQ.     3255.94       C 2 --  H 3           49.5% ( 72.2%)          98.1%
  T-DIPOLE   0.7829       C 2 --  H 4           49.5%                   98.1%
  TRAVEL     0.1174       O 1 --  C 2            1.0%                    0.0%
  RED. MASS  0.7508
 
  VIBRATION   6            ATOM PAIR     ENERGY CONTRIBUTION          RADIAL
  FREQ.     3302.12       C 2 --  H 4           49.3% ( 69.8%)          95.5%
  T-DIPOLE   0.3478       C 2 --  H 3           49.3%                   95.5%
  TRAVEL     0.1240       O 1 --  C 2            1.4%                  100.0%
  RED. MASS  0.6644
 
 
           SYSTEM IS A GROUND STATE
 
 
    FORMALDEHYDE, MNDO ENERGY = -32.8819  See Manual.
    DEMONSTRATION OF MOPAC - FORCE AND THERMODYNAMICS CALCULATION
 
 
           MOLECULE IS NOT LINEAR
 
           THERE ARE  6 GENUINE VIBRATIONS IN THIS SYSTEM
           THIS THERMODYNAMICS CALCULATION IS LIMITED TO
           MOLECULES WHICH HAVE NO INTERNAL ROTATIONS
                                           
                                                                     Note 8
                     CALCULATED THERMODYNAMIC PROPERTIES
                                           *
 TEMP. (K)  PARTITION FUNCTION   H.O.F.    ENTHALPY   HEAT CAPACITY  ENTROPY
                                KCAL/MOL   CAL/MOLE    CAL/K/MOL   CAL/K/MOL
 
 
 298  VIB.         1.007                    23.39484    0.47839    0.09151
      ROT.     709.                        888.305      2.981     16.026
      INT.     714.                        911.700      3.459     16.117
      TRA.    0.159E+27                   1480.509      4.968     36.113
      TOT.                       -32.882  2392.2088     8.4274    52.2300
 
     * NOTE: HEATS OF FORMATION ARE RELATIVE TO THE
              ELEMENTS IN THEIR STANDARD STATE AT 298K
 
 
 
  TOTAL CPU TIME:            32.26 SECONDS
 
  == MOPAC DONE ==
\end{verbatim}


\begin{description}
\item[Note 1:]  All three words, ROT, FORCE, and THERMO are  necessary  in
 order  to  obtain  thermodynamic properties.  In order to obtain results
 for only one temperature, THERMO has  the  first  and  second  arguments
 identical.  The symmetry number for the C$_{2v}$ point-group is 2.

\item[Note 2:]  Internal coordinate derivatives are  in  kcal/Angstrom  or
 kcal/radian.  Values of less than about 0.2 are quite acceptable.

\item[Note 3:]  In larger calculations, the time estimates are useful.  In
 practice  they are pessimistic, and only about 70\% of the time estimated
 will be used, usually.  The principal moments of inertia can be directly
 related  to  the  microwave  spectrum  of the molecule.  They are simple
 functions of the geometry of the system, and are usually predicted  with
 very high accuracy.

\item[Note 4:]  Zero point  energy  is  already  factored  into  the  MNDO
 parameterization.   Force  constant data are not printed by default.  If
 you want this output, specify LARGE in the keywords.\index{force constants}
 \index{zero point energy}

\item[Note 5:]  Normal coordinate analysis has been  extensively  changed.
 The  first set of eigenvectors represent the `normalized' motions of the
 atoms.  The sum of the speeds (not the velocities) of the atoms adds  to
 unity.   This  is verified by looking at the motion in the `z' direction
 of the atoms in vibration 2.  Simple addition of these terms,  unsigned,
 adds to 1.0, whereas to get the same result for mode 1 the scalar of the
 motion of each atom needs to be calculated first.

      Users might be concerned about reproducibility.   As  can  be  seen
 from  the vibrational frequencies from Version 3.00 to 6.00 given below,
 the main difference over earlier FORCE calculations is  in  the  trivial
 frequencies.

\begin{verbatim}
                        Real Frequencies of Formaldehyde 

 Version 3.00  1209.96   1214.96   1490.60   2114.57   3255.36   3301.57
 Version 3.10  1209.99   1215.04   1490.59   2114.57   3255.36   3301.58
 Version 4.00  1209.88   1214.67   1490.52   2114.52   3255.92   3302.10
 Version 5.00  1209.89   1214.69   1490.53   2114.53   3255.93   3302.10
 Version 6.00  1209.90   1214.67   1490.53   2114.54   3255.94   3302.12

                       Trivial Frequencies of Formaldehyde 
                 T(x)      T(y)      T(z)       R(x)      R(y)      R(z)
 Version 3.00  -0.00517  -0.00054  -0.00285   57.31498  11.59518   9.01619
 Version 3.10  -0.00557   0.00049  -0.00194   87.02506  11.18157  10.65295
 Version 4.00  -0.00044  -0.00052  -0.00041   12.99014  -3.08110  -3.15427
 Version 5.00   0.00040  -0.00044  -0.00062   21.05654   2.80744   3.83712
 Version 6.00  -0.00025  -0.00022  -0.00047    3.38368   2.03661  -0.76725
\end{verbatim}

\item[Note 6:]  Normal modes are not of much  use  in  assigning  relative
 importance  to  atoms  in a mode.  Thus in iodomethane it is not obvious
 from an examination of the normal modes which mode  represents the C--I
 stretch.   A  more  useful  description  is  provided  by  the energy or
 mass-weighted coordinate analysis.  Each set of three  coefficients  now
 represents  the  relative  energy  carried  by  an  atom.   (This is not
 strictly accurate as a definition, but is believed (by JJPS) to be  more
 useful than the stricter definition.)\index{mass-weighted coordinates}

\item[Note 7:]  The following description of the  coordinate  analysis  is
 given  without  rigorous  justification.   Again, the analysis, although
 difficult to understand, has been found to be more useful than  previous
 descriptions.\index{normal coordinate analysis}\index{vibrational analysis}

      On the left-hand side are printed the  frequencies  and  transition
 dipoles.    Underneath  these  are  the  reduced  masses  and  idealized
 distances traveled which represent the simple  harmonic  motion  of  the
 vibration.   The  mass  is  assumed  to  be  attached  by a spring to an
 infinite mass.  Its displacement is the travel.

      The next column is a list of all pairs  of  atoms  that  contribute
 significantly  to  the  energy of the mode.  Across from each pair (next
 column) is the percentage energy contribution of the pair to  the  mode,
 calculated according to the formula described below.

                     
\begin{center}{\bf Formula for energy contribution}\end{center}

 The total vibrational energy, $T$, carried by  all  pairs  of  bonded
 atoms in a molecule is first calculated.  For any given pair of atoms, A
 and B, the relative contribution, ${\cal R}(A,B)$, as a percentage, is given
 by the energy of the pair, $P(A,B)$, times 100 divided by $T$, i.e.,
$$            {\cal R}(A,B)   =    \frac{100\times P(A,B)}{T}$$


      As an example, for formaldehyde the energy carried by the  pair  of
 atoms  (C,O)  is  added  to  the energy of the two (C,H) pairs to give a
 total, $T$.  Note that this total cannot be related to anything  which  is
 physically  meaningful  (there  is obvious double-counting), but it is a
 convenient  artifice.   For  mode  4,  the  C=O  stretch,  the  relative
 contribution  of  the carbon-oxygen pair is 60.1\%.  It might be expected
 to be about 100\% (after all, we envision the C=O bond as  absorbing  the
 photon);  however,  the  fact  that the carbon atom is vibrating implies
 that it is changing its position relative to the two hydrogen atoms.  If
 the  total  vibrational  energy, $E_v$  (the actual energy of the absorbed
   photon, as distinct from $T$), were carried  equally  by  the  carbon  and
   oxygen  atoms, then the relative contributions to the mode would be C=O,
   50\% ; C--H, 25\% ; C--H, 25\%, respectively.  This leads to the next  entry,
   which is given in parentheses.

        For the pair with the highest relative contribution (in mode 4, the
   C=O stretch), the energy of that pair divided by the total energy of the
   mode, $E_v$,  is  calculated  as  a  percentage.   This  is  the  absolute
   contribution, $\cal A$ as a percentage, to the total energy of the mode.
   $$        {\cal A}(A,B) = \frac{100 \times P(A,B)}{E_v} $$


   Now the C=O is seen to contribute 100.5 percent of the energy.  For this
   sort  of  partitioning only the sum of all $\cal A$'s must add to 100\%, each
   pair can contribute more or less than 100\%.   In  the  case  of  a  free
   rotator,  e.g.   ethane,  the  $\cal A$ of any specific bonded pair to the
   total energy can be very high (several hundred percent).

 It may be easier to view $P/E_v$ as a contribution to the total energy
 of the  mode, $E_v$.  In this case the fact that $P/E_v$ can be greater than
 unity can be explained by the fact that there are other relative motions
   within the molecule which make a negative contribution to $E_v$.

   From the $\cal R$'s an idea can be obtained of where the energy of  the
   mode  is  going;  from  the  $\cal A$  value the significance of the highest
   contribution can be inferred.  Thus, in  mode  4  all  three  bonds  are
   excited,  but  because the C=O bond carries about 100\% of the energy, it
   is clear that this is really a C=O  bond  stretch  mode,  and  that  the
   hydrogens are only going along for the ride.

   In the last column the percentage radial motion is  printed.   This
   is  useful  in  assigning  the  mode  as  stretching  or  bending.   Any
   non-radial motion is de-facto tangential or bending.

   To summarize:  The new analysis is more  difficult  to  understand,
   but  is  considered  by  the  author  (JJPS)  to  be  the easiest way of
   describing what are often complicated vibrations.

\item[Note 8:]  In order, the thermodynamic quantities calculated are:
\begin{enumerate}
\item The vibrational contribution,
\item The rotational contribution,
\item The sum of (1) and (2), this gives the internal contribution,
\item The translational contribution.

        For partition functions the various  contributions  are  multiplied
   together.
 
      A new quantity is the heat of formation at the defined temperature.
 This is intended for use in calculating heats of reaction.  Because of a
 limitation in the data available, the H.o.F. at T Kelvin is  defined  as
 ``The heat of formation of the compound at T Kelvin from it's elements in
 their standard state at 298 Kelvin''.  Obviously, this definition of heat
 of  formation is incorrect, but should be useful in calculating heats of
 reaction, where the elements in their standard state at 298 Kelvin  drop
 out.
\end{enumerate}
\end{description}

\section{Example of reaction path with symmetry}
\index{PATH calculation}
   In this example, one methyl group in ethane is rotated relative  to
   the  other and the geometry is optimized at each point.  As the reaction
   coordinate involves three hydrogen atoms moving, symmetry is imposed  to
   ensure equivalence of all hydrogens.
\begin{verbatim}
     Line  1:          SYMMETRY   T=600
     Line  2:    ROTATION OF METHYL GROUP IN ETHANE
     Line  3:    EXAMPLE OF A REACTION PATH CALCULATION
     Line  4:    C 
     Line  5:    C    1.479146 1 
     Line  6:    H    1.109475 1  111.328433 1 
     Line  7:    H    1.109470 0  111.753160 0  120.000000 0   2  1  3
     Line  8:    H    1.109843 0  110.103163 0  240.000000 0   2  1  3
     Line  9:    H    1.082055 0  121.214083 0   60.000000 -1  1  2  3
     Line 10:    H    1.081797 0  121.521232 0  180.000000 0   1  2  3
     Line 11:    H    1.081797 0  121.521232 0  -60.000000 0   1  2  3
     Line 12:    0    0.000000 0    0.000000 0    0.000000 0   0  0  0
     Line 13:    3 1 4 5 6 7 8
     Line 14:    3 2 4 5 6 7 8
     Line 15:    6 7 7
     Line 16:    6 11 8
     Line 17: 
     Line 18:     70 80 90 100 110 120 130 140 150
\end{verbatim}

        Points to note:
\begin{enumerate}
\item The dihedrals of the second and third hydrogens are not  marked
   for optimization:  the dihedrals follow from point-group symmetry.

\item All six C--H bond  lengths  and  H--C--C  angles  are  related  by
   symmetry:  see lines 13 and 14.

\item The dihedral on line 9 is the reaction  coordinate,  while  the
   dihedrals  on lines 10 and 11 are related to it by symmetry functions on
   lines 15 and 16.  The symmetry  functions  are  defined  by  the  second
   number  on  lines 13 to 16 (see SYMMETRY for definitions of functions 1,
   2, 7, and 11).

\item Symmetry data are ended by a blank line.

\item The reaction coordinate data are ended  by  the  end  of  file.
   Several lines of data are allowed.

\item Whenever symmetry is used in addition to other data  below  the
   geometry  definition  it will always follow the ``blank line'' immediately
   following the geometry definition.  The other data  will  always  follow
   the symmetry data.
\end{enumerate}

%%%%%%%%%%%%%%%%%%%%%%%%%%%%%%%%%%%%%%%%%%%%%%%%%%%%%%%%%%%%%%%%%%%%%

\chapter{Background}
\section{Introduction}
  While all the theory used in MOPAC is in the literature, so that in
 principle one could read and understand the algorithm, many parts of the
 code involve programming concepts or constructions which, while  not  of
 sufficient  importance  to  warrant  publication,  are described here in
 order to facilitate understanding.
 
\section{AIDER}
 AIDER will allow gradients to be defined for a system.  MOPAC  will
 calculate  gradients, as usual, and will then use the supplied gradients
 to form an error function.  This error function is:  (supplied gradients
 $-$  initial  calculated  gradients),  which is then added to the computed
 gradients, so that for the initial  SCF,  the  apparent  gradients  will
 equal the supplied gradients.
 
 A typical data-set using AIDER would look like this:
\begin{verbatim}
    PM3 AIDER AIGOUT GNORM=0.01 EF
  Cyclohexane
 
   X  
   X     1    1.0
   C     1    CX    2  CXX
   C     1    CX    2  CXX    3  120.000000   
   C     1    CX    2  CXX    3 -120.000000   
   X     1    1.0   2  90.0   3    0.000000
   X     1    1.0   6  90.0   2  180.000000
   C     1    CX    7  CXX    3  180.000000   
   C     1    CX    7  CXX    3   60.000000   
   C     1    CX    7  CXX    3  -60.000000   
   H     3    H1C   1  H1CX   2    0.000000   
   H     4    H1C   1  H1CX   2    0.000000   
   H     5    H1C   1  H1CX   2    0.000000   
   H     8    H1C   1  H1CX   2  180.000000   
   H     9    H1C   1  H1CX   2  180.000000   
   H    10    H1C   1  H1CX   2  180.000000   
   H     3    H2C   1  H2CX   2  180.000000   
   H     4    H2C   1  H2CX   2  180.000000   
   H     5    H2C   1  H2CX   2  180.000000   
   H     8    H2C   1  H2CX   2    0.000000   
   H     9    H2C   1  H2CX   2    0.000000   
   H    10    H2C   1  H2CX   2    0.000000   
   CX      1.46613   
   H1C     1.10826   
   H2C     1.10684   
   CXX    80.83255 
   H1CX  103.17316
   H2CX  150.96100
 
   AIDER
     0.0000
    13.7589   -1.7383
    13.7589   -1.7383    0.0000
    13.7589   -1.7383    0.0000
     0.0000    0.0000    0.0000
     0.0000    0.0000    0.0000
    13.7589   -1.7383    0.0000
    13.7589   -1.7383    0.0000
    13.7589   -1.7383    0.0000
   -17.8599   -2.1083    0.0000
   -17.8599   -2.1083    0.0000
   -17.8599   -2.1083    0.0000
   -17.8599   -2.1083    0.0000
   -17.8599   -2.1083    0.0000
   -17.8599   -2.1083    0.0000
   -17.5612   -0.6001    0.0000
   -17.5612   -0.6001    0.0000
   -17.5612   -0.6001    0.0000
   -17.5612   -0.6001    0.0000
   -17.5612   -0.6001    0.0000
   -17.5612   -0.6001    0.0000
\end{verbatim}
 
      Each  supplied  gradient  goes  with  the  corresponding   internal
 coordinate.   In  the  example  given,  the  gradients came from a 3-21G
 calculation on the geometry shown.  Symmetry will be taken into  account
 automatically. Gaussian  prints  out  gradients in atomic units; these
 need to be converted into kcal/mol/Angstrom or kcal/mol/radian for MOPAC
 to use.  The resulting geometry from the MOPAC run will be nearer to the
 optimized 3-21G geometry than  if  the  normal  geometry  optimizers  in
 Gaussian had been used.\index{coordinates!Gaussian!example}

\section{Correction to the peptide linkage}

        The residues in peptides are joined together by  peptide  linkages,
   --HNCO--.   These  linkages  are  almost  flat, and normally adopt a trans
   configuration; the hydrogen and oxygen atoms being on opposite sides  of
   the  C--N  bond.   Experimentally,  the  barrier  to  interconversion  in
   N-methyl acetamide is about 14 kcal/mole, but all  four  methods  within
   MOPAC  predict  a  significantly  lower  barrier,  PM3 giving the lowest
   value.

        The low barrier can be traced  to  the  tendency  of  semiempirical
   methods   to   give   pyramidal   nitrogens.    The   degree   to  which
   pyramidalization of the nitrogen atom is preferred can be  seen  in  the
   following series of compounds.
\begin{verbatim}
        Compound      MINDO/3    MNDO    AM1    PM3     Exp

  Ammonia             Py          Py     Py     Py      Py
  Aniline             Py          Py     Py     Py      Py
  Formamide           Py          Py     Flat   Py      Py
  Acetamide           Flat        Py     Flat   Py      Flat
  N-methyl formamide  Flat        Py     Flat   Py      Flat
  N-methyl acetamide  Flat        Flat   Flat   Py      Flat
\end{verbatim}

        To correct this, a molecular-mechanics correction has been applied.
   This  consists  of  identifying  the -R-HNCO- unit, and adding a torsion
   potential of form:
   $$ k\times \sin\theta^2 $$   
   where $\theta$ is the X--N--C--O angle, X=R or H, and $k$ varies from method
   to  method.   This  has  two effects:  there is a force constraining the
   nitrogen to be planar, and HNCO barrier in N--methyl acetamide is  raised
   to  14.00  kcal/mole.   When the MM correction is in place, the nitrogen
   atom for all methods for the last three compounds shown above is planar.
   The correction should be user-transparent.

\begin{center}{\bf Cautions}\end{center}

\begin{enumerate}
\item This correction will lead to errors of 0.5--1.5  kcal/mole  if
the   peptide   linkage   is  made  or  broken  in  a  reaction
calculation.
\item If the correction is applied to formamide the nitrogen will  be
            flat, contrary to experiment.
\item When calculating rotation barriers, take into account the rapid
            rehybridization  which  occurs.   When the dihedral is 0 or 180
            degrees the nitrogen will be planar (sp2), but  at  90  degrees
            the nitrogen should be pyramidal, as the partial double bond is
            broken.  At that geometry the true  transition  state  involves
            motion  of the nitrogen substituent so that the nitrogen in the
            transition state is more nearly sp2.  In other words, a  simple
            rotation  of  the  HNCO  dihedral will not yield the activation
            barrier, however it will be within 2 kcal/mole of  the  correct
            answer.   The  14  kcal barrier mentioned earlier refers to the
            true transition state.
\item Any job involving a CONH group will require either the  keyword
            NOMM or MMOK.  If you do not want the correction to be applied,
use the keyword ``NOMM'' (NO Molecular Mechanics).\index{CONH linkage}
\end{enumerate}     

\section{Level of precision within MOPAC}

   Several users have criticised the  tolerances  within  MOPAC.   The
   point  made  is  that significantly different results have been obtained
   when different starting conditions have been used, even  when  the  same
   conformer  should  have  resulted.  Of course, different results must be
   expected --- there will always be small differences --- nonetheless  any
   differences   should   be  small,  e.g.   heats  of  formation  (HoF)
   differences should be less than about 0.1  kcal/mole.   MOPAC  has  been
   modified  to  allow  users  to  specify a much higher precision than the
   default when circumstances warrant it.

 \subsubsection{Reasons for low precision}
        There are several reasons for obtaining low quality  results.   The
   most  obvious  cause of such errors is that for general work the default
   criteria will result in  a  difference  in  HoF   of  less  than  0.1
   kcal/mole.    This   is   only  true  for  fairly  rigid  systems,  e.g.
   formaldehyde and benzene.  For systems with low barriers to rotation  or
   flat  potential  surfaces,  e.g.   aniline  or  water dimer, quite large
   HoF  errors can result.

 \subsubsection{Various precision levels}
   In normal (non-publication quality) work the default  precision  of
   MOPAC  is recommended.  This will allow reasonably precise results to be
   obtained  in  a  reasonable  time.    Unless   this   precision   proves
   unsatisfactory, use this default for all routine work.

        The  best  way  of  controlling  the  precision  of  the   geometry
   optimization  and gradient minimization is by specifying a gradient norm
   which must be satisfied.  This is done via the keyword GNORM=.  Altering
   the  GNORM  automatically disables the other termination tests resulting
   in the gradient norm dominating the calculation.  This works both  ways:
   a  GNORM of 20 will give a very crude optimization while a GNORM of 0.01
   will give a very precise optimization.  The default GNORM is 1.0.

        When the highest precision is needed, such as in exacting  geometry
   work,  or  when  you want results which cannot be improved, then use the
   combination keywords GNORM=0.0 and \verb/SCFCRT=1.D-NN/; 
   NN should  be  in  the range  2--15.  Increasing the SCF criterion 
   (the default is \verb/SCFCRT=1.D-4/)
   helps the line search routines by increasing the precision of  the  heat
   of  formation  calculation; however, it can lead to excessive run times,
   so take care.  Also, there is an increased chance of  not  achieving  an
   SCF when the SCF criterion is excessively increased.

        Superficially, requesting a GNORM of zero  might  seem  excessively
   stringent,  but  as soon as the run starts, it will be cut back to 0.01.
   Even that might seem too  stringent.   The  geometry  optimization  will
   continue to lower the energy, and hopefully the GNORM, but frequently it
   will not prove possible to lower  the  GNORM  to  0.01.   If,  after  10
   cycles,  the energy does not drop then the job will be stopped.  At this
   point you have the best geometry that MOPAC, in its  current  form,  can
   give.
 
        If a slightly less than highest precision is needed,  such  as  for
   normal publication quality work, set the GNORM to the limit wanted.  For
   example, for a flexible system, a GNORM of 0.1 to 0.5 will  normally  be
   good enough for all but the most demanding work.

        If higher than the default, but still not very  high  precision  is
   wanted,  then  use  the  keyword  PRECISE.  This will tighten up various
   criteria so that higher than routine precision will be given.

        If high precision is used, so that the printed GNORM is 0.000,  and
   the   resulting   geometry   resubmitted   for  one  SCF  and  gradients
   calculation, then normally a GNORM higher than 0.000 will result.   This
   is  NOT  an error in MOPAC:  the geometry printed is only precise to six
   figures after the decimal point.  Geometries need  to  be  specified  to
   more than six decimals in order to drive the GNORM to less than 0.000.

        If you want to test MOPAC, or use it  for  teaching  purposes,  the
   GNORM  lower limit of 0.01 can be overridden by specifying LET, in which
   case you can specify any limit for GNORM.  However, if it is too low the
   job  may  finish  due to an irreducible minimum in the heat of formation
   being encountered.  If this happens, the ``STATIONARY POINT'' message will
   be printed.

        Finally there is  a  full  analytical  derivative  function  within
   MOPAC.   These use STO-6G Gaussian wavefunctions because the derivatives
   of the overlap integral are easier to calculate  in  Gaussians  than  in
   STO's.  Consequently, there will be a small difference in the calculated
   HoFs when \mi{analytical derivatives} are used.  If  there  is  any  doubt
   about  the  accuracy of the finite derivatives, try using the analytical
   derivatives.  They are a bit slower than finite derivatives but are more
   precise  (a  rough  estimate is 12 figures for finite difference, 14 for
   analytical).

        Some calculations, mainly open shell RHF or closed shell  RHF  with
   C.I. have untracked errors which prevent very high precision.  For these
   systems GNORM should be in the range 1.0 to 0.1.

\subsubsection{How large can a gradient be and still be acceptable?}
        A common source of confusion is the limit to which the GNORM should
   be  reduced  in  order  to  obtain acceptable results.  There is no easy
   answer, however a few guidelines can be given.
 
      First of all reducing the GNORM to an arbitarily  small  number  is
 not  sensible.   If the keywords GNORM=0.000001, LET, and EF are used, a
 geometry con be obtained which is precise to about  0.000001 \AA.
 If ANALYT is also used, the results obtained will be slightly different.
 Chemically, this change is meaningless, and no  significance  should  be
 attached  to  such  numbers.   In  addition,  any  minor  change  to the
 algorithm, such as porting it to a new machine, will give rise to  small
 changes  in  the optimized geometry.  Even the small changes involved in
 going from MOPAC  5.00  to  MOPAC  6.00  caused  small  changes  in  the
 optimized geometry of test molecules.

   As a guide, a GNORM of 0.1 is sufficient for all  heat-of-formation
   work,  and  a  GNORM  of  0.01 for most geometry work.  If the system is
   large, you may need to settle for a GNORM of 1.0--0.5.

   This whole topic was raised by Dr.\ Donald B. Boyd of Lilly
   Research  Laboratories,  who provided unequivocal evidence for a failure
   of MOPAC and convinced me of the importance of increasing  precision  in
   certain circumstances.\index{boyd@{\bf Boyd, Donald B.}}

\section{Convergence tests in subroutine ITER}

\subsubsection{Self--consistency test}
   The SCF iterations are stopped when two tests are satisfied.  These
   are (1) when the difference in electronic energy, in eV, between any two
   consecutive iterations drops below the adjustable parameter, SELCON, and
   the  difference between any three consecutive iterations drops below ten
   times SELCON, and (2) the difference in density matrix elements  on  two
   successive iterations falls below a preset limit, which is a multiple of
   SELCON.

        SELCON is set initially to 0.0001 kcal/mole; this can be  made  100
   times  smaller by specifying PRECISE or FORCE.  It can be over-ridden by
   explicitly defining the SCF criterion via \verb/SCFCRT=1.D-12/.

        SELCON is further modified by the value of the  gradient  norm,  if
   known.   If GNORM is large, then a more lax SCF criterion is acceptable,
   and SCFCRT can be relaxed up to 50 times it's  default  value.   As  the
   gradient norm drops, the SCF criterion returns to its default value.

        The SCF test is performed using the energy calculated from the Fock
   matrix  which  arises  from  a  density matrix, and not from the density
   matrix which arises from a Fock.  In the limit, the two  energies  would
   be  identical,  but  the first converges faster than the second, without
   loss of precision.

\section{Convergence in SCF calculation}
        A  brief  description  of  the  convergence  techniques   used   in
   subroutine ITER follows.

        ITER, the  SCF  calculation,  employs  six  methods  to  achieve  a
   self-consistent field.  In order of usage, these are:
\begin{enumerate}
   \item Intrinsic convergence by virtue of the way the  calculation  is
   carried  out.   Thus  a trial Fock gives rise to a trial density matrix,
   which in turn is used to generate a better Fock matrix.

        This is normally convergent, but many exceptions  are  known.   The
   main situations when the intrinsic convergence does not work are:
\begin{enumerate}
          \item A bad starting density  matrix.   This  normally  occurs
          when the default starting density matrix is used.  This is a very
          crude approximation, and is only  used  to  get  the  calculation
          started.   A  large  charge  is generated on an atom in the first
          iteration,  the  second   iteration   overcompensates,   and   an
          oscillation is generated.
          \item The equations are only very slowly convergent.  This can
          be  due  to  a  long-lived  oscillation  or to a slow transfer of
          charge.
\end{enumerate}

   \item Oscillation damping.  If, on any two consecutive iterations,  a
   density  matrix  element  changes  by  more  than 0.05, then the density
   matrix element is set equal to the old element shifted by  0.05  in  the
   direction  of  the calculated element.  Thus, if on iterations 3 and 4 a
   certain density matrix element was 0.55 and 0.78, respectively, then the
   element  would  be set to 0.60 (=0.55+0.05) on iteration 4.  The density
   matrix from iteration 4 would then be used in the  construction  of  the
   next  Fock  matrix.   The arrays which hold the old density matrices are
   not filled until after iteration 2.  For this reason they are  not  used
   in the damping before iteration 3.

   \item Three-point interpolation of the  density  matrix.   Subroutine
   CNVG monitors the number of iterations, and if this is exactly divisible
   by three, and certain other conditions relating to the density  matrices
   are  satisfied,  a  three-point interpolation is performed.  This is the
   default converger,  and  is  very  effective  with  normally  convergent
   calculations.    It  fails  in  certain  systems,  usually  those  where
   significant charge build-up is present.

   \item Energy-level shift technique.  The virtual M.O.  energy  levels
   are  normally  shifted  to more positive energy.  This has the effect of
   damping oscillations, and intrinsically divergent equations can often be
   changed   to  intrinsically  convergent  form.   With  slowly-convergent
   systems the virtual M.O.  energy levels can be moved to a more  negative
   value.

        The precise value of the shift used depends on the behavior of  the
   iteration energy.  If it is dropping, then the HOMO-LUMO gap is reduced,
   if the iteration energy rises, the gap is increased rapidly.

   \item Pulay's method.  If  requested,  when  the  largest  change  in
   density  matrix elements on two consecutive iterations has dropped below
   0.1, then routine CNVG is abandoned in  favor  of  a  multi-Fock  matrix
   interpolation.   This  relies  on  the fact that the eigenvectors of the
   density and Fock matrices are identical at self-consistency, so  [P.F]=0
   at  SCF.  The extent to which this condition does not occur is a measure
   of  the  deviance  from  self-consistency.   Pulay's  method  uses  this
   relationship to calculate that linear combination of Fock matrices which
   minimize  [P.F].   This  new  Fock  matrix  is  then  used  in  the  SCF
   calculation.

        Under certain circumstances, Pulay's method  can  cause  very  slow
   convergence,   but   sometimes   it   is  the  only  way  to  achieve  a
   self-consistent field.  At other times the procedure  gives  a  ten-fold
   increase  in  speed,  so care must be exercised in its use.  (invoked by
   the keyword PULAY)

   \item The Camp-King converger.  If  all  else  fails,  the  Camp-King
   converger  is  just about guaranteed to work every time.  However, it is
   time-consuming, and therefore should only be invoked as a last resort.

        It  evaluates  that  linear  combination   of   old   and   current
   eigenvectors  which  minimize the total energy.  One of its strengths is
   that systems which  otherwise  oscillate  due  to  charge  surges,  e.g.
   CHO--H,  the C--H distance being very large, will converge using this very
   sophisticated converger.
\end{enumerate}

\section{Causes of failure to achieve an SCF}

        In a system where a biradical can form, such as ethane  decomposing
   into   two   CH3  units,  the  normal  RHF  procedure  can  fail  to  go
   self-consistent.  If the system has marked biradicaloid character,  then
   BIRADICAL  or UHF and TRIPLET can often prove successful.  These options
   rely on the assumption that two unpaired  electrons  can  represent  the
   open shell part of the wave-function.

   Consider H--Cl,  with  the  interatomic  distance  being   steadily
   increased.    At   first  the  covalent  bond  will  be  strong,  and  a
   self-consistent field is readily  obtained.   Gradually  the  bond  will
   become  more  ionic,  and  eventually the charge on chlorine will become
   very large.  The hydrogen, meanwhile, will become very  electropositive,
   and  there  will be an increased energy advantage to any one electron to
   transfer from chlorine to hydrogen.   If  this  in  fact  occurred,  the
   hydrogen would suddenly become very electron-rich and would, on the next
   iteration, lose  its  extra  electron  to  the  chlorine.   A  sustained
   oscillation  would  then be initiated.  To prevent this, if BIRADICAL is
   specified, exactly one electron will end  up  on  hydrogen.   A  similar
   result can be obtained by specifying TRIPLET in a UHF calculation.

\section{Torsion or dihedral angle coherency}
\index{dihedral angle coherency}
        MOPAC  calculations  do  not   distinguish   between   enantiomers,
   consequently  the  sign of the dihedrals can be multiplied by $-1$ and the
   calculations will be unaffected.  However, if chirality is important,  a
   user should be aware of the sign convention used.

        The dihedral angle convention used in  MOPAC  is  that  defined  by
   Klyne and Prelog in {\it Experientia\/} 16, 521 (1960).
   \index{klyne@{\bf Klyne and Prelog}}  In this convention,
   four atoms, AXYB, with a dihedral angle of 90 degrees, will have atom  B
   rotated  by 90 degrees clockwise relative to A when X and Y are lined up
   in the direction of sight, X being nearer to the eye.  In  their  words,
   ``To distinguish between enantiomeric types the angle `tau' is considered
   as positive when it is measured clockwise from the front  substituent  A
   to   the   rear   substituent  B,  and  negative  when  it  is  measured
   anticlockwise.'' The alternative  convention  was  used  in  all  earlier
   programs, including QCPE 353.

\section{Vibrational analysis}
        Analyzing normal coordinates is very tedious.  Users  are  normally
   familiar  with the internal coordinates of the system they are studying,
   but not familiar with the cartesian coordinates.  To  help  characterize
   the  normal  coordinates,  a very simple analysis is done automatically,
   and users are strongly encouraged to use this analysis first,  and  then
   to look at the normal coordinate eigenvectors.

        In the analysis, each pair of bonded atoms is examined  to  see  if
   there  is  a  large  relative  motion  between them.  By bonded is meant
   within the Van der Waals' distance.  If there  is  such  a  motion,  the
   indices  of  the  atoms,  the  relative  distance  in Angstroms, and the
   percentage radial motion are printed.   Radial  plus  tangential  motion
   adds  to  100\%,  but  as there are two orthogonal tangential motions and
   only one radial, the radial component is printed.

\section{A note on thermochemistry}
\begin{center}
By\\
\ \\
Tsuneo Hirano\index{hirano@{\bf Hirano, Tsuneo}}\\
Department of Synthetic Chemistry\\
Faculty of Engineering\\
University of Tokyo\\
Hongo, Bunkyo-ku, Tokyo, Japan\\
\end{center}

\subsection{Basic Physical Constants}
Taken from: ``Quantities, Units and Symbols in Physical Chemistry,''
Blackwell Scientific Publications Ltd, Oxford OX2 0EL, UK, 1987
(IUPAC, based on CODATA of ICSU, 1986).  pp 81--82.
\index{Boltzmann constant}
\index{constants!physical}
\index{definitions!Boltzmann constant}
\index{definitions!velocity of light}
\index{gas constant, R}

\begin{center}
\begin{tabular}{|l|}\hline
Speed of light, $c = 2.997 92458 \times 10^{10}$ cm/s (Definition)\\
Boltzmann constant, $k = R/Na = 1.380 658 \times 10^{-23}$ J/K 
$ = 1.380 658 \times 10^{-16}$ erg/K\\
Planck constant, $h = 6.626 0755 \times 10^{-34}$ J s
$ = 6.626 0755 \times 10^{-27}$  erg s\\
Gas constant, $ R = 8.314 510$  J/mol/K $ = 1.987 216$  cal/mol/K\\
Avogadro number, $ N_a = 6.022 1367 \times 10^{23}$  /mol\\
Volume of 1 mol of gas, $V_0 = 22.414 10$ l/mol (at 1 atm, 25 C) \\
1 J = $1. \times 10^7$ erg\\
1 kcal = $4.184$ kJ  (Definition)\\
1 eV = $23.060 6$ kcal/mol\\
1 a.u. = $27.211 35$ eV/mol = 627.509 6 kcal/mol\\
1 cm$^{-1}$ = $2.859 144$ cal/mol $ = N_a h c / 4.184^7$\\
1 atm = $1.013 25 \times 10^5$ Pa = $1.013 25 \times 10^6$ 
dyn/cm$^2$ (Definition)\\ \hline
\end{tabular} 
\end{center}
                          
Moment of inertia: $I$ 1 amu angstrom$^2$ = 
$1.660 540 \times 10^{-40}$ g cm$^2$.

Rotational constants: $A$, $B$, and $C$ (e.g. $A = h/(8\pi^2I)$)\\
With $I$ in amu angstroms$^2$ then: 
$A$ (in MHz)  = $5.053 791 \times 10^5 / I$ \\
$A$ (in cm$^{-1}$) = $5.053 791 \times 10^5/ cI = 16.857 63 / I$ 


\subsection{Thermochemistry from ab initio MO methods}
Ab initio MO methods provide total energies, $E_{\rm eq}$,  as  the  sum  of
electronic   and   nuclear-nuclear  repulsion  energies  for  molecules,
isolated in vacuum, without vibration at 0 K.\index{Ab initio total energies}
\begin{equation}
E_{\rm eq} = E_{\rm el} + E_{\rm nuclear-nuclear}
\end{equation}
>From the 0 K potential surface and using  the  harmonic  oscillator
approximation,  we can calculate the vibrational frequencies, $\nu_i$, of the
normal modes of vibration.  Using these, we can  calculate  vibrational,
rotational   and   translational   contributions  to  the  thermodynamic
quantities such as the partition function and heat capacity which  arise
from heating the system from 0 to T K.

$Q$: \mi{partition function}, $E$: \mi{energy}, $S$: \mi{entropy}, 
and $C$: \mi{heat capacity}.

\subsection*{[Vibration]}
\begin{equation}
Q_{\rm vib} = \sum_i{\frac{1}{[1 - \exp(-h\nu_i/kT)]}} \label{eq2}
\end{equation}
$E_{\rm vib}$, for a molecule at the temperature $T$ as:
\begin{equation}
E_{\rm vib} = \sum_i\left\{\frac{h\nu_i}{2} + 
\frac{h\nu_i\exp(-h\nu_i/kT)}{[1 - \exp(-h\nu_i/kT)]}\right\}  \label{eq3}
\end{equation}
where $h$ is the Planck constant, $\nu_i$ the $i$--th normal 
vibration  frequency,
and $k$  the Boltzmann constant.  For 1 mole of molecules, $E_{\rm vib}$ 
should be multiplied by the Avogadro number $N_a = R/k$.  Thus:
\begin{equation}
E_{\rm vib} = N_a  \sum_i\left\{ \frac{h\nu_i}{2}
 + \frac{h\nu_i\exp(-h\nu_i/kT)}{[1-\exp(-h\nu_i/kT)]}\right\}  \label{eq4}
\end{equation}

Note that the first term  in equation~(\ref{eq4})  is  
the  Zero-point  vibration energy.   Hence,  the second term 
in eq.~(\ref{eq4}) is the additional vibrational
   contribution due to the temperature increase from 0 K to T K.  Namely,
\begin{eqnarray}
E_{\rm vib}                & = & E_{\rm zero} + E_{\rm vib}(0\rightarrow T)
\label{eq5}\\
E_{\rm zero}               & = & N_a \sum_i {\frac{h\nu_i}{2}} \label{eq6}\\
E_{\rm vib}(0\rightarrow T) & = & N_a  \sum_i 
{\frac{h\nu_i\exp(-h\nu_i/kT)}{[1 - \exp(-h\nu_i/kT)]}} \label{eq7}
\end{eqnarray}                                         
The value of $E_{\rm vib}$ from GAUSSIAN 82 and 86 includes 
$E_{\rm zero}$ as defined by Eqs.~(\ref{eq4},\ref{eq7}).
\begin{eqnarray}
S_{\rm vib} & = & R \sum_i\left\{
\frac{(h\nu_i/kT)\exp(-h\nu_i/kT)}{[1 - \exp(-h\nu_ii/kT)]}
- \ln[1 - \exp(-h\nu_i/kT)]\right\} \label{eq8}\\
C_{\rm vib} & = & R \sum_i\left\{
\frac{(h\nu_i/kT)^2 \exp(-h\nu_i/kT)}{[1 - \exp(-h\nu_i/kT)]^2}\right\} 
\label{eq9}
\end{eqnarray}

   At temperature $T>0$ K, a molecule rotates about  the  x,  y,  and
   z-axes  and  translates  in  x,  y,  and  z-directions.  By assuming the
   equipartition of energy, energies for rotation and translation, 
   $E_{\rm rot}$ and $E_{\rm tr}$, are calculated.

\subsection*{[Rotation]}
$\sigma$ is symmetry number. $I$ is moment of inertia. 
$I_A$, $I_B$, and $I_C$ are moments of inertia about A, B, and C axes.

\subsubsection{Linear molecule}
\begin{eqnarray}
Q_{\rm rot} & = & \frac{8\pi^2 IkT}{\sigma h^2} \label{eq10} \\
E_{\rm rot} & = & (2/2)RT  \label{eq11} \\
S_{\rm rot} & = &  R \ln \left[ \frac{8\pi^2 IkT}{\sigma h^2} \right] + R 
\label{eq12} \\
            & = & R \ln I + R \ln T - R \ln\sigma - 4.349 203  \nonumber
\end{eqnarray}
where $ -4.349 203 = R \ln[8\times 10^{-16} \pi^2 k/(N_a h^2)] + R$.
\begin{equation}
C_{\rm rot} = (2/2)R    \label{eq14}
\end{equation}

\subsubsection{Non-linear molecule}
\begin{eqnarray}
 Q_{\rm rot} & = & \left( \frac{\sqrt{\pi}}{\sigma}\right)
 \left[ \frac{8\pi^2kT}{h^2}\right]^{3/2} \sqrt{I_AI_BI_C} \nonumber\\
& = & \left( \frac{\sqrt{\pi}}{\sigma}\right)
\left[ \left( \frac{8\pi^2cI_A}{h} \right) 
       \left( \frac{8\pi^2cI_B}{h} \right)
       \left( \frac{8\pi^2cI_C}{h} \right) \right]^{1/2} 
\left(\frac{kT}{hc}\right)^{3/2} \label{eq15} \\
 E_{\rm rot} & = & (3/2)RT \label{eq16} \\
 S_{\rm rot} & = & \frac{R}{2} \ln\left\{
 \left(\frac{\pi}{\sqrt{\sigma}}\right) \left(\frac{8\pi^2cI_A}{h}\right)
 \left(\frac{8\pi^2cI_B}{h}\right)      \left(\frac{8\pi^2cI_C}{h}\right)
 \left(\frac{kT}{hc}\right)^3\right\} + (3/2)R  \label{eq17} \\
& = & (R/2) \ln{(I_A I_B I_C)} + (3/2) R\ln{T}  - R \ln{\sigma} - 5.3863921
\nonumber 
\end{eqnarray}

Here, $-5.386 3921$ is calculated as:
$$ R \ln\left\{\frac{1}{h^3} \left(\frac{10^{-16}}{N_a}\right)^{3/2} 
\sqrt{(3\times 2^9 \times \pi^7 \times k)}\right\} + (3/2)R.
$$

\begin{equation}
C_{\rm rot} = (3/2)R \label{eq18}
\end{equation}

\subsection*{[Translation]}
$M$ is Molecular weight.
\begin{eqnarray}
Q_{\rm tra} & = & \left( \frac{\sqrt{2\pi MkT/N_a}}{h}\right)^3
\label{eq19} \\
E_{\rm tra} & = & (3/2)RT                            \label{eq20} \\
S_{\rm tra} & = & R\left\{ \frac{5}{2} + 
\frac{3}{2}\ln\left(\frac{2\pi k}{h^2}\right) + \ln k + 
\frac{3}{2}\ln\left(\frac{M}{N_a}\right) + \frac{5}{2}\ln T - \ln p \right\}
\label{eq21} \\
            & = & (5/2)R\ln T + (3/2)R\ln M - R\ln p - 2.31482 \label{eq22}\\
C_{\rm tra} & = & (5/2)R                              \label{eq23}
\end{eqnarray}
or  $H_{\rm tra} = (5/2)RT$  due to the $pV$ term (cf.  $H = U + pV$).
The internal energy $U$ at $T$ is:
\begin{equation}
U = E_{\rm eq} + [E_{\rm vib} + E_{\rm rot} + E_{\rm tra}] \label{eq24}
\end{equation}
or
\begin{equation}
U = E_{\rm eq} + [(E_{\rm zero} + E_{\rm vib}(0\rightarrow T)) 
  + E_{\rm rot} + E_{\rm tra}]  \label{eq25}
\end{equation}
Enthalpy $H$ for one mole of gas is defined as
\begin{equation}
H = U + pV     \label{eq26}
\end{equation}
Assumption of an ideal gas (i.e.,  $pV = RT$) leads to
\begin{equation}
H = U + pV = U + RT  \label{eq27} 
\end{equation}
Thus, \mi{Gibbs free energy} $G$ can be calculated as:
\begin{equation}
           G = H - T S(0\rightarrow T) \label{eq28}
\end{equation}

\subsection*{Thermochemistry in MOPAC}
It should be noted that MO parameters for MINDO/3,  MNDO,  AM1  and
PM3  are optimized so as to reproduce the experimental heat of formation
(i.e., standard enthalpy of formation or the enthalpy change to  form  a
mole  of  compound  at  25 degrees C from its elements in their standard
state) as well as observed geometries (mostly at 25 degrees C), and  not
to reproduce the $E_{\rm eq}$ and equilibrium geometry at 0 K.\index{heat of
formation}

In  this  sense, $E_{\rm scf}$  (defined  as  Heat  of  formation), force
constants,  normal  vibration  frequencies  etc  are  all related to the
values at 25 degree C, not to 0 K!!!!!  Therefore, the $E_{\rm zero}$ calculated
in FORCE is not the true $E_{\rm zero}$. Its use as $E_{\rm zero}$ 
should be made at your own risk, bearing in mind the situation discussed above.

Since $E_{\rm scf}$ is standard enthalpy of formation (at 25 degree C):
\begin{equation}
E_{\rm scf} = E_{\rm eq} + E_{\rm zero} + 
E_{\rm vib}(0\rightarrow 298.15) + E_{\rm rot} + E_{\rm tra} + pV
 + \sum\left[ - E_{\rm elec}({\rm atom}) + \Delta H_f({\rm atom})\right]
\label{eq29}
\end{equation}
To avoid the complication arising from the definition of $E_{\rm scf}$, within
the  thermodynamics  calculation  the  Standard  Enthalpy  of Formation,
$\Delta H$, is calculated by
\begin{equation}
\Delta H = E_{\rm scf} + (H_T - H_{298}) \label{eq30}
\end{equation}

Here, $E_{\rm scf}$ is the heat of formation (at 25 degree C) 
given in the output
list, and $H_T$ and $H_{298}$ are the enthalpy contributions for the increase of
the temperature from 0 K to $T$ and 298.15, respectively.  In other words,
the enthalpy of formation is corrected for the difference in temperature
from 298.15 K to $T$.  The method of calculation for $T$ and $H_{298}$ will  be
given below.

In MOPAC, the variables defined below are used:
\begin{equation}
C_1 = \frac{hc}{kT}   \label{eq31}
\end{equation}
The wavenumber, $\omega_i$, in cm$^{-1}$:
\begin{equation}
\nu_i = \omega_i c    \label{eq32}
\end{equation}
\begin{equation}
E_{\rm WJ} = \exp( -h\nu_i/kT) = \exp(-\omega_i hc/kT) = \exp(-\omega_i C_1) \label{eq33}
\end{equation}
The rotational constants $A$, $B$, and $C$ in cm$^{-1}$:
\begin{equation}
A = \frac{h}{(8\pi^2 cI_A)}   \label{eq34}
\end{equation}

Energy and Enthalpy in cal/mol, and Entropy in cal/mol/K.
Thus, eqs. (\ref{eq2}--\ref{eq28}) can be written as follows.

\subsubsection{[Vibration]}
\begin{eqnarray}
Q_{\rm vib} & = & \pi \sum_i \frac{1 }{ (1 - E_{\rm WJ})}  \label{eq35}\\
E_0         & = & \frac{0.5 N_a hc}{4.184 \times 10^7}\sum_i {\omega_i} 
\label{eq36}\\
            & = & 1.429 572 \sum_i {\omega_i}    \label{eq37}\\
E_{\rm vib}(0\rightarrow T) & = & 
N_a h c\sum_i{\frac{\omega_i E_{\rm WJ}}{1 - E_{\rm WJ}}}
= (R/k) h c\sum_i{\frac{W_ iE_{\rm WJ}}{1 - E_{\rm WJ}}}     \label{eq38}\\
S_{\rm vib} & = & R (hc/kT)
\sum_i\left\{\frac{\omega_i E_{\rm WJ}}{(1 - E_{\rm WJ})}\right\}
                - R\sum_i {\ln (1 - E_{\rm WJ})} \nonumber \\
& = & R C_1\sum_i\left\{\frac{\omega_i E_{\rm WJ}}{(1 - E_{\rm WJ})}\right\}
- R\sum_i {\ln(1 - E_{\rm WJ})}            \label{eq39}\\
C_{\rm vib} & = & R (hc/kT)^2
\sum_i\left\{\frac{\omega_i^2 E_{\rm WJ}}{(1- E_{\rm WJ})^2} \right\} \nonumber \\
& = & R C_1^2
\sum_i\left\{\frac{\omega_i^2 E_{\rm WJ}}{(1- E_{\rm WJ})^2}\right\} 
\label{eq40}
\end{eqnarray}

\subsection*{[Rotation]}

\subsubsection{Linear molecule}
\begin{eqnarray}
Q_{\rm rot} & = & (1/\sigma) (1/A) (kT/hc) = \frac{1}{\sigma A C_1} 
\label{eq41}\\
E_{\rm rot} & = & (2/2)RT   \label{eq42} \\
S_{\rm rot} & = & R\ln\left(\frac{kT}{\sigma hcA}\right) + R
= R\ln \left(\frac{1}{\sigma A C_1}\right) + R 
= R\ln\left(\frac{kT}{\sigma hcA}\right) + R \label{eq43}\\
C_{\rm rot} & = & (2/2)R       \label{eq44}
\end{eqnarray}
        
\subsubsection{Non-linear molecule}
\begin{eqnarray}
Q_{\rm rot} & = & \frac{1}{\sigma}
\left[\frac{\pi}{(A B C C_1^3)}\right]^{1/2} \label{eq45}\\
E_{\rm rot} & = & (3/2)RT        \label{eq46} \\
S_{\rm rot} & = & \frac{R}{2}\ln\left\{\frac{\pi}{\sigma^2ABC} 
\left(\frac{kT}{hc}\right)^3 \right\} + (3/2)R \nonumber \\
& = & 0.5R { 3\ln(kT/hc) - 2\ln\sigma +\ln\left(\frac{\pi}{A B C}\right) + 3}
\label{eq47} \\
& = & 0.5R { -3\ln C_1 - 2\ln\sigma + \ln\left(\frac{\pi}{A B C}\right) + 3}
\nonumber \\
C_{\rm rot} & = & (3/2)R     \label{eq48}
\end{eqnarray}

\subsection*{[Translation]}
\begin{eqnarray}
Q_{\rm tra} & = & 
   \left( \frac{\sqrt{2\pi MkT/N_a}}{h} \right)^3
=  \left( \frac{\sqrt{1.660540\times^{-24}\times 2\pi MkT}}{h} \right)^3
\label{eq49} \\
E_{\rm tra} & = &  (3/2)RT                    \label{eq50} \\
H_{\rm tra} & = &  (3/2)RT + pV = (5/2)RT \; {\rm cf.}
\;pV = RT  \label{eq51}\\
S_{\rm tra} & = & (R/2) [ 5\ln T + 3\ln M ] - 2.31482  
\;{\rm cf.}\; p = 1 {\rm atm} \nonumber \\
& = & 0.993608 [ 5\ln T + 3\ln M] - 2.31482    \label{eq52}
\end{eqnarray}

In MOPAC:
\begin{equation}
H _{\rm vib} = E_{\rm vib}(0\rightarrow T)     \label{eq53}
\end{equation}

(Note: $E_{\rm zero}$ is {\em not\/} included in $H_{\rm vib}$
$\omega_i$ is not derived from force-constants at 0 K)
and for $T$:
\begin{equation}
H_T   = [H_{\rm vib} + H_{\rm rot} + H_{\rm tra}]   \label{eq54}
\end{equation}
while for $T=298.15$~K:
\begin{equation}
H_{298} = [H_{\rm vib} + H_{\rm rot} + H_{\rm tra}]  \label{eq55}
\end{equation}

Note that $H_T$ (and $H_{298}$) is equivalent to:
\begin{equation}
(E_{\rm vib} - E_{\rm zero}) + E_{\rm rot} + (E_{\rm tra} + pV) \label{eq56}
\end{equation}
except that  the  normal  frequencies  are  those  obtained  from  force
constants at 25 degree C, or at least not at 0 K.

Thus, Standard Enthalpy of Formation, $\Delta H$,  can  be  calculated
according to Eqs.~(\ref{eq25},\ref{eq26}) and (\ref{eq29}), 
as shown in Eq.~(\ref{eq30});
\begin{equation}
\Delta H = E_{\rm scf} + (H_T - H_{298})   \label{eq57}
\end{equation}
Note that $E_{\rm zero}$ is already counted in $E_{\rm scf}$,
see Eq.~(\ref{eq29}).

By using Eq.~(\ref{eq27}), Standard Internal Energy of Formation, $\Delta U$,
can be calculated as:
\begin{equation}
\Delta U = \Delta H - R(T - 298.15) \label{eq58}
\end{equation}


Standard Gibbs Free-Energy of Formation, $\Delta G$, can be calculated
by  taking  the difference from that for the isomer or that at different
temperature:
\begin{equation}
\Delta G = [\Delta H - TS] \;(\mbox{for the state under consideration})
- [\Delta H - TS]\; (\mbox{for reference state})  \label{eq59}
\end{equation}

Taking the difference is necessary  to  cancel  the  unknown  values  of
standard entropy of formation for the constituent elements.

\section{Reaction coordinates}\index{coordinates!reaction}
The Intrinsic Reaction Coordinate method pioneered and developed by
Mark Gordon\index{gordon@{\bf Gordon, Mark}}  has been 
incorporated in a modified form into MOPAC.  As
this facility is quite complicated all the keywords associated with  the
IRC have been grouped together in this section.

\section*{DRC}
\index{DRC!background}
\index{DRC!definition}
The Dynamic Reaction Coordinate is the path  followed  by  all  the
atoms  in  a  system  assuming  conservation  of  energy,  i.e.,  as the
potential energy changes the kinetic energy of  the  system  changes  in
exactly  the  opposite  way  so  that  the  total  energy  (kinetic plus
potential) is a constant.  If started at a  ground  state  geometry,  no
significant  motion should be seen.  Similarly, starting at a transition
state geometry should not produce  any  motion  -  after  all  it  is  a
stationary point and during the lifetime of a calculation it is unlikely
   to accumulate enough momentum to travel far from the starting position.

        In order to calculate the DRC path from a transition state,  either
   an  initial  deflection  is  necessary  or some initial momentum must be
   supplied.

        Because of the time-dependent nature of the DRC  the  time  elapsed
   since the start of the reaction is meaningful, and is printed.

\subsection*{Description}
  The course of a molecular vibration can be followed by  calculating
   the  potential  and  kinetic  energy  at  various  times.   Two  extreme
   conditions can be identified:  (a) gas phase, in which the total  energy
   is a constant through time, there being no damping of the kinetic energy
   allowed, and (b) liquid phase, in which kinetic energy is always set  to
   zero, the motion of the atoms being infinitely damped.\index{liquids}

        All possible degrees of damping  are  allowed.   In  addition,  the
   facility  exists  to  dump  energy into the system, appearing as kinetic
   energy.  As kinetic energy is a function of velocity, a vector quantity,
   the  energy  appears  as  energy of motion in the direction in which the
   molecule would naturally move.  If the system  is  a  transition  state,
   then  the  excess  kinetic  energy  is added after the intrinsic kinetic
   energy has built up to at least 0.2 kcal/mole.\index{kinetic energy!damping}

        For ground-state systems, the excess energy sometimes  may  not  be
   added;  if  the  intrinsic kinetic energy never rises above 0.2kcal/mole
   then the excess energy will not be added.


\subsection*{Equations used}
Force acting on any atom:
$$ g(i) + g'(i)t + g''(i)t^2 = \frac{dE}{dx(i)} + 
        \frac{d^2E}{dx(i)^2} + \frac{d^3E}{dx(i)^3} $$
Acceleration due to force acting on each atom:
$$ a(i) = \frac{1}{M(i)} (g(i) + g'(i)t + g''(i)t^2) $$
New velocity:
$$ V(o) + \frac{1}{M(i)} \left(Dt g(i) + (1/2) Dt^2 g'(i) + 
(1/3) Dt^3g''(i)\right) $$
or:
$$ V(i) = V(i) + V'(i)t + V''(i)t^2 + V'''(i)t^3 $$
That is, the change in velocity is equal to the integral  over  the
time interval of the acceleration.

New position of atoms:
$$ X(i) = X(o) + V(o)t + (1/2) V't^2 + (1/3) V''t^3 + (1/4) V'''t^4 $$
That is, the change in position is equal to the integral  over  the
time interval of the velocity.

        The velocity vector is accurate to the extent that  it  takes  into
   account  the  previous velocity, the current acceleration, the predicted
   acceleration, and the change in predicted  acceleration  over  the  time
   interval.    Very  little  error  is  introduced  due  to  higher  order
   contributions to the velocity; those that do occur  are  absorbed  in  a
   re-normalization of the magnitude of the velocity vector after each time
   interval.

        The magnitude of $Dt$, the time interval, is determined mainly by the
   factor   needed   to   re-normalize  the  velocity  vector.   If  it  is
   significantly different from unity, Dt will be reduced; if  it  is  very
   close to unity, Dt will be increased.

        Even with all this, errors creep in and a system,  started  at  the
   transition  state,  is  unlikely  to  return precisely to the transition
   state  unless  an  excess  kinetic  energy  is  supplied,  for   example
   0.2kcal/mole.

        The calculation  is  carried  out  in  cartesian  coordinates,  and
   converted   into   internal  coordinates  for  display.   All  cartesian
   coordinates must be allowed to vary, in order to  conserve  angular  and
   translational momentum.\index{DRC!conservation of momentum}

\subsection*{IRC}
\index{IRC!definition}
The Intrinsic Reaction Coordinate is the path followed by  all  the
   atoms  in  a  system  assuming  all kinetic energy is completely lost at
   every point, i.e., as the potential energy changes  the  kinetic  energy
   generated  is  annihilated  so  that  the  total  energy  (kinetic  plus
   potential) is always equal to the potential energy only.

        The IRC is intended for use  starting  with  the  transition  state
   geometry.    A   normal  coordinate  is  chosen,  usually  the  reaction
   coordinate, and the system  is  displaced  in  either  the  positive  or
   negative  direction  along  this  coordinate.   The  internal  modes are
   obtained by calculating the mass-weighted  Hessian  matrix  in  a  force
   calculation   and   translating  the  resulting  cartesian  normal  mode
   eigenvectors to conserve  momentum.   That  is,  the  initial  cartesian
   coordinates  are  displaced  by  a  small  amount  proportional  to  the
   eigenvector coefficients plus a translational constant; the constant  is
   required  to  ensure that the total translational momentum of the system
   is conserved as zero.  At the present time there may be  small  residual
   rotational  components  which  are not annihilated; these are considered
   unimportant.

\subsection*{General description of the DRC and IRC}
As the IRC usually requires a normal coordinate, a  force  constant
   calculation  normally  has to be done first.  If IRC is specified on its
   own a normal coordinate is not used and the IRC calculation is performed
   on the supplied geometry.

A recommended sequence of operations to start an IRC calculation is
as follows:
\begin{enumerate}
\item Calculate the transition state geometry.  If  the  T/S  is  not
            first  optimized,  then  the  IRC  calculation  may  give  very
            misleading results.  For example, if NH3 inversion  is  defined
            as  the  planar  system  but  without the N--H bond length being
            optimized the first normal coordinate might be for N--H  stretch
            rather  than  inversion.   In  that case the IRC will relax the
            geometry to the optimized planar structure.

\item Do a normal FORCE calculation, specifying ISOTOPE in  order  to
            save  the  FORCE  matrices.   Do  not  attempt  to  run the IRC
            directly unless you have confidence that the FORCE  calculation
            will work as expected.  If the IRC calculation is run directly,
            specify ISOTOPE anyway:  that will save the FORCE matrix and if
            the  calculation  has  to  be  re-done  then  RESTART will work
            correctly.

\item Using IRC=n and RESTART run the IRC calculation.  If RESTART is
            specified with IRC=n then the restart is assumed to be from the
            FORCE calculation.  If RESTART is specified without IRC=n,  say
            with  IRC on its own, then the restart is assumed to be from an
            earlier IRC calculation that was  shut  down  before  going  to
            completion.
\end{enumerate}



        A DRC calculation is simpler in that a force calculation is  not  a
   prerequisite;  however,  most  calculations of interest normally involve
   use of an internal coordinate.  For this reason IRC=n  can  be  combined
   with  DRC  to  give  a  calculation in which the initial motion (0.3kcal
   worth of kinetic energy) is supplied by  the  IRC,  and  all  subsequent
   motion  obeys conservation of energy.  The DRC motion can be modified in
   three ways:
\begin{enumerate}
\item It is possible to calculate the reaction  path  followed  by  a
            system  in  which  the  generated  kinetic energy decays with a
            finite half-life.  This can  be  defined  by  DRC=n.nnn,  where
            n.nnn  is  the  half-life in femtoseconds.  If n.nn is 0.0 this
            corresponds  to  infinite  damping  simulating  the   IRC.    A
            limitation  of  the  program is that time only has meaning when
            DRC is specified without a half-life.

\item Excess kinetic energy can be added to the calculation by use of
            KINETIC=n.nn.   After  the  kinetic  energy  has  built  up  to
            0.2kcal/mole or if IRC=n is used then n.nn kcal/mole of kinetic
            energy  is  added  to  the  system.   The excess kinetic energy
            appears as a velocity vector  in  the  same  direction  as  the
            initial motion.

\item The RESTART file \verb/<filename>.RES/ can be edited to allow the user
            to  modify the velocity vector or starting geometry.  This file
            is formatted.
\end{enumerate}

        Frequently DRC leads to a periodic, repeating orbit.   One  special
   type --- the  orbit in which the direction of motion is reversed so that
   the system retraces its own path --- is sensed for  and  if  detected  the
   calculation  is  stopped after exactly one cycle.  If the calculation is
   to be continued,  the  keyword  GEO-OK  will  allow  this  check  to  be
   by-passed.

        Due to the potentially very large output files  that  the  DRC  can
   generate  extra  keywords  are  provided  to allow selected points to be
   printed.  After the system has changed by a preset amount the  following
   keywords can be used to invoke a print of the geometry.

\begin{verbatim}
          KeyWord         Default             User Specification

          X-PRIO      0.05 Angstroms             X-PRIORITY=n.nn
          T-PRIO      0.10 Femtoseconds          T-PRIORITY=n.nn
          H-PRIO      0.10 kcal/mole             H-PRIORITY=n.nn
\end{verbatim}

\subsection*{Option to allow only extrema to be output}
  In the geometry specification, if an internal coordinate is  marked
   for  optimization  then  when that internal coordinate passes through an
   extremum a message will be printed and the geometry output.

        Difficulties can  arise  from  the  way  internal  coordinates  are
   processed.   The  internal  coordinates are generated from the cartesian
   coordinates, so an internal coordinate supplied  may  have  an  entirely
   different  meaning  on  output.  In particular the connectivity may have
   changed.  For obvious reasons dummy atoms should  not  be  used  in  the
   supplied  geometry  specification.   If  there  is  any  doubt about the
   internal coordinates or if the starting geometry  contains  dummy  atoms
   then  run  a  1SCF calculation specifying XYZ.  This will produce an ARC
   file with the ``ideal'' numbering --- the internal numbering system used  by
   MOPAC.   Use this ARC file to construct a data file suitable for the DRC
   or IRC.\index{DRC!dummy atoms in}

Notes:
\begin{enumerate}
\item Any coordinates marked for optimization  will  result  in  only
            extrema being printed.
\item If extrema are being printed then kinetic energy  extrema  will
            also be printed.
\end{enumerate}

\subsection*{Keywords for use with the IRC and DRC}
\index{IRC!keywords for}
\begin{enumerate}
\item Setting up the transition state:  NLLSQ SIGMA TS.
\item Constructing the FORCE matrix:  FORCE or IRC=n, ISOTOPE, LET.
\item Starting an IRC:  RESTART and IRC=n, T-PRIO, X-PRIO, H-PRIO.
\item Starting a DRC:  DRC or DRC=n.nn, KINETIC=n.nn.
\item Starting a DRC from a transition state:   (DRC  or  DRC=n)  and
            IRC=n, KINETIC=n.
\item Restarting an IRC:  RESTART and IRC.
\item Restarting a DRC:  RESTART and (DRC or DRC=n.nn).
\item Restarting a DRC starting from a transition state:  RESTART and
            (DRC or DRC=n.nn).
\end{enumerate}
Other keywords, such as T=nnn or GEO-OK can be used anytime.


\subsection*{Examples of DRC/IRC data}
Use of the IRC/DRC facility is quite complicated.  In the following
   examples  various `reasonable' options are illustrated for a calculation
   on water. It is  assumed  that  an  optimized  transition-state  geometry  is
   available.

        Example  1:   A  Dynamic  Reaction  Coordinate,  starting  at   the
   transition  state  for  water  inverting, initial motion opposite to the
   transition normal mode, with 6kcal of excess kinetic  energy  added  in.
   Every point calculated is to be printed (Note all coordinates are marked
   with a zero, and T-PRIO, H-PRIO and X-PRIO are all absent).  The results
   of  an  earlier calculation using the same keywords is assumed to exist.
   The earlier calculation would have constructed the force matrix.   While
   the  total  cpu  time  is specified, it is in fact redundant in that the
   calculation will run to completion in less than 600 seconds.
\index{IRC!example of}

\begin{verbatim}
    KINETIC=6 RESTART  IRC=-1 DRC T=600 
        WATER 
     
      H    0.000000  0    0.000000  0    0.000000  0   0  0  0 
      O    0.911574  0    0.000000  0    0.000000  0   1  0  0 
      H    0.911574  0  180.000000  0    0.000000  0   2  1  0 
      0    0.000000  0    0.000000  0    0.000000  0   0  0  0
\end{verbatim}

        Example 2:  An Intrinsic Reaction Coordinate calculation.  Here the
   restart  is from a previous IRC calculation which was stopped before the
   minimum was reached.  Recall that RESTART with IRC=n implies  a  restart
   from  the FORCE calculation.  Since this is a restart from within an IRC
   calculation the keyword IRC=n has been replaced by IRC.  IRC on its  own
   (without the ``=n'') implies an IRC calculation from the starting position
   --- here the RESTART position --- without initial
displacement.\index{IRC!example of restart}
\begin{verbatim}
       RESTART  IRC  T=600 
        WATER 
     
      H    0.000000  0    0.000000  0    0.000000  0   0  0  0 
      O    0.911574  0    0.000000  0    0.000000  0   1  0  0 
      H    0.911574  0  180.000000  0    0.000000  0   2  1  0 
      0    0.000000  0    0.000000  0    0.000000  0   0  0  0
\end{verbatim}

\subsection*{Output format for IRC and DRC}
The IRC and DRC can produce  several  different  forms  of  output.
   Because of the large size of these outputs, users are recommended to use
   search functions to extract information.  To facilitate  this,  specific
   lines  have specific characters.  Thus, a search for the ``\%'' symbol will
   summarize the energy profile while a search  for ``AA'' will  yield  the
   coordinates of atom 1, whenever it is printed.  The main flags to use in
   searches are:
\begin{verbatim}    
           SEARCH FOR                    YIELDS

            '% '         Energies for all points calculated, 
                         excluding extrema
            '%M'         Energies for all turning points
            '%MAX'       Energies for all maxima
            '%MIN'       Energies for all minima
            '%'          Energies for all points calculated
            'AA*'        Internal coordinates for atom 1 for every point
            'AE*'        Internal coordinates for atom 5 for every point
            '123AB*'     Internal coordinates for atom 5 for point 123
\end{verbatim}            

        As the keywords for the IRC/DRC are interdependent,  the  following
   list of keywords illustrates various options.\index{DRC!keyword options}
\index{KINETIC}

\begin{verbatim}
    KEYWORD                 RESULTING ACTION
    DRC                     The Dynamic Reaction Coordinate is calculated.
                            Energy is conserved, and no initial impetus.
    DRC=0.5                 In the DRC kinetic energy is lost with a half-
                            life of 0.5 femtoseconds.
    DRC=-1.0                Energy is put into a DRC with an half-life of 
                            -1.0 femtoseconds, i.e., the system gains 
                            energy.
    IRC                     The Intrinsic Reaction Coordinate is 
                            calculated.  No initial impetus is given. 
                            Energy not conserved.
    IRC=-4                  The IRC is run starting with an impetus in the
                            negative of the 4th normal mode direction. The
                            impetus is one quantum of vibrational energy.
    IRC=1 KINETIC=1         The first normal mode is used in an IRC, with
                            the initial impetus being 1.0kcal/mole.
    DRC KINETIC=5           In a DRC, after the velocity is defined, 5 kcal
                            of kinetic energy is added in the direction of
                            the initial velocity.
    IRC=1 DRC KINETIC=4     After starting with a 4 kcal impetus in the 
                            direction of the first normal mode, energy is
                            conserved.
    DRC VELOCITY KINETIC=10 Follow a DRC trajectory which starts with an
                            initial velocity read in, normalized to a 
                            kinetic energy of 10 kcal/mol.  
\end{verbatim}

        Instead of every point being printed, the option  exists  to  print
   specific  points  determined  by the keywords T-PRIORITY, X-PRIORITY and
   H-PRIORITY.  If any one of these words is specified, then the calculated
   points  are used to define quadratics in time for all variables normally
   printed.  In addition, if the flag for the first atom is set to  T  then
   all  kinetic  energy  turning  points  are printed.  If the flag for any
   other internal coordinate is set to T then, when that coordinate  passes
   through an extremum, that point will be printed.  As with the PRIORITYs,
   the point will be calculated via  a  quadratic  to  minimize  non-linear
   errors.

   N.B.:  Quadratics are unstable in the regions of inflection points,
   in  these  circumstances linear interpolation will be used.  A result of
   this is that points printed in the  region  of  an  inflection  may  not
   correspond  exactly to those requested.  This is not an error and should
   not affect the quality of the results.

\subsection*{Test of DRC---verification of trajectory path}
Introduction:  Unlike  a  single-geometry  calculation  or  even  a
   geometry  optimization, verification of a DRC trajectory is not a simple
   task.  In this section  a  rigorous  proof  of  the  DRC  trajectory  is
   presented;  it  can be used both as a test of the DRC algorithm and as a
   teaching exercise.  Users of the DRC are asked to  follow  through  this
   proof in order to convince themselves that the DRC works as it should.

\subsection*{Part 1: The nitrogen molecule}
For the nitrogen molecule and using MNDO, the equilibrium  distance
is  $1.103802$ \AA, the heat of formation is 8.276655 kcal/mole and
the vibrational frequency is $2739.6$ cm$^{-1}$.   For  small  displacements,
the  energy curve versus distance is parabolic and the gradient curve is
approximately linear, as is shown in the following  table.   A  nitrogen
molecule is thus a good approximation to a harmonic oscillator.
\begin{verbatim}
             STRETCHING CURVE FOR NITROGEN MOLECULE
             
             N--N DIST     HoF            GRADIENT
           (Angstroms)   (kcal/mole) (kcal/mole/Angstrom)
             
             1.1180       8.714564        60.909301          
             1.1170       8.655723        56.770564          
             1.1160       8.601031        52.609237          
             1.1150       8.550512        48.425249          
             1.1140       8.504188        44.218525          
             1.1130       8.462082        39.988986          
             1.1120       8.424218        35.736557          
             1.1110       8.390617        31.461161          
             1.1100       8.361303        27.162720          
             1.1090       8.336299        22.841156          
             1.1080       8.315628        18.496393          
             1.1070       8.299314        14.128353          
             1.1060       8.287379         9.736959          
             1.1050       8.279848         5.322132          
             1.1040       8.276743         0.883795          
             1.1030       8.278088        -3.578130          
             1.1020       8.283907        -8.063720          
             1.1010       8.294224       -12.573055          
             1.1000       8.309061       -17.106213          
             1.0990       8.328444       -21.663271       
             1.0980       8.352396       -26.244309         
             1.0970       8.380941       -30.849404          
             1.0960       8.414103       -35.478636          
             1.0950       8.451906       -40.132083          
             1.0940       8.494375       -44.809824          
             1.0930       8.541534       -49.511939          
             1.0920       8.593407       -54.238505          
             1.0910       8.650019       -58.989621          
             1.0900       8.711394       -63.765330          
\end{verbatim}

\subsubsection{Period of vibration}
The period of vibration (time taken for the oscillator to undertake
one complete vibration, returning to its original position and velocity)
can be calculated in three ways.  Most direct is  the  calculation  from
the  energy  curve; using the gradient constitutes a faster, albeit less
direct, method, while calculating it from the vibrational  frequency  is
very  fast  but  assumes  that the vibrational spectrum has already been
calculated.

\begin{enumerate}
\item From the energy curve.
For a simple harmonic oscillator the period $r$ is given by:
$$ r = 2 \pi \sqrt{\frac{m}{k}}  $$
where $m$ is the \mi{reduced mass} and $k$ is the \mi{force constant}. 
The  reduced  mass  (in amu)   of   a   nitrogen  molecule  is  
$14.0067/2  =  7.00335$,  and  the force-constant can be calculated from:
$$                 E-c = (1/2) k(R-R_o)^2 $$
Given $R_o = 1.1038$, $R = 1.092$, $c = 8.276655$ and $E = 8.593407$~kcal/mol
then:
\begin{eqnarray*}                      
& = & 4548.2 \; \mbox{kcal/mole/A}^2 \\
& = & 4545 \times 4.184 \times 10^3 \times 10^7 \times 10^{16}\; 
\mbox{ergs/cm}^2 \\
& = & 1.9029 \times 10^{30} \; \mbox{ergs/cm}^2
\end{eqnarray*}
Therefore:      
$$ r = 2 \times 3.14159 \times \sqrt{\frac{7.0035}{1.9029\times 10^{30}}}
\;{\rm seconds} = 12.054 \times 10^{-15}\;{\rm s} = 12.054\;{\rm fs} $$

\item From the gradient curve.
The force  constant  is  the  derivative  of  the  gradient  wrt
distance:
$$ k = \frac{dG}{dx} $$
Since we are using discrete points,  the  force  constant  is  best
obtained from finite differences:
$$ k = \frac{(G_2-G_1)}{(x_2-x_1)} $$
For $x_2 = 1.1100$, $G_2 = 27.163$ and for $x_1 =  1.0980$,  
$G_1  =  -26.244$,
giving rise to $k = 4450$ kcal/mole/A$^2$ and a period of $12.186$~fs.

\item From the vibrational frequency.
Given a ``frequency'' (wavenumber) of vibration of N$_2$ of $\bar{\nu}=2739.6$  
cm$^{-1}$,  the period of oscillation, in seconds, is given directly by:
$$ r = \frac{1}{c\bar{\nu}} = \frac{1}{2739.6 \times 2.998 \times 10^{10}} 
$$
or as $12.175$ femtoseconds.
\end{enumerate}

Summarizing, by three different methods the period  of  oscillation
of N$_2$  is calculated to be $12.054$, $12.186$ and $12.175$~fs, average
$12.138$~fs.

\subsubsection{Initial dynamics of N$_2$ with N--N distance = 1.094 \AA}

A useful check on the dynamics of N$_2$ is to  calculate  the  initial
acceleration  of  the  two  nitrogen  atoms  after releasing them from a
starting interatomic separation of 1.094 \AA.

At R(N-N) = 1.094 \AA,
$G = -44.810$~kcal/mole/\AA\ or $-18.749 \times 10^{19}$~erg/cm.
Therefore acceleration, $f = -18.749 \times 10^{19} /14.0067$~cm/sec/sec 
or $-13.386 \times 10^{18}$~cm/s$^2$ which is
$ -13.386 \times 10^{15} \times$ Earth surface gravity!

Distance from equilibrium  $= 0.00980$ \AA.
After $0.1$ fs, velocity is $0.1  10^{-15} (-13.386  10^{18})$ cm/sec
or $1338.6$ cm/s.

In the  DRC  the  time-interval  between  points  calculated  is  a
complicated function of the curvature of the local surface.  By default,
the first time-interval is 0.105fs, so the calculated velocity  at  this
time should be $0.105 \times 1338.6 = 1405.6$ cm/s, in the DRC calculation the
predicted velocity is $1405.6$ cm/s.

The option is provided to allow sampling of the system at  constant
time-intervals,  the  default being $0.1$~fs.  For the first few points the
calculated velocities are as follows.
\begin{verbatim}
      TIME   CALCULATED    LINEAR       DIFF.
              VELOCITY    VELOCITY    VELOCITY   

      0.000        0.0       0.0        0.0
      0.100     1338.6    1338.6        0.0
      0.200     2673.9    2677.2       -3.3
      0.300     4001.0    4015.8      -14.8
      0.400     5317.3    5354.4      -37.1
      0.500     6618.5    6693.0      -74.5
      0.600     7900.8    8031.6     -130.8
\end{verbatim}

As the calculated velocity is  a  fourth-order  polynomial  of  the
acceleration,   and  the  acceleration,  its  first,  second  and  third
derivatives, are all changing, the predicted velocity rapidly becomes  a
poor guide to future velocities.
  
For simple harmonic motion the velocity at any time is given by:
$$ v = v_0  \sin(2\pi t/r) $$
By fitting the computed velocities to simple harmonic motion, a much
better fit is obtained:
\begin{verbatim}  
             Calculated   Simple Harmonic       Diff      
      Time    Velocity    25316.Sin(0.529t)
  
     0.000        0.0           0.0              0.0
     0.100     1338.6        1338.6              0.0
     0.200     2673.9        2673.4             +0.5
     0.300     4001.0        4000.8             +0.2
     0.400     5317.3        5317.0             +0.3
     0.500     6618.5        6618.3             +0.2
     0.600     7900.8        7901.0             -0.2
\end{verbatim}
  
  The repeat-time required for this  motion  is  $11.88$~fs,  in  good
  agreement  with  the  three  values calculated using static models.  The
  repeat time should not be calculated from the time required to go from a
  minimum  to  a  maximum and then back to a minimum -- only half a cycle.
  For all real systems the potential energy is a skewed parabola, so  that
  the  potential energy slopes are different for both sides; a compression
  (as in this case) normally leads to a higher force-constant, and shorter
  apparent  repeat  time  (as in this case).  Only the addition of the two
  half-cycles is meaningful.

                      
\subsubsection{Conservation of normal coordinate}
So far this analysis has only considered a homonuclear diatomic.  A
detailed  analysis  of  a  large  polyatomic  is  impractical,  and  for
simplicity a molecule of formaldehyde will be studied.

In polyatomics, energy can  transfer  between  modes.   This  is  a
result  of the non-parabolic nature of the potential surface.  For small
displacements the surface can be considered as  parabolic.   This  means
that  for small displacements interconversion between modes should occur
only very slowly.  Of the six normal modes, mode 1, at 1204.5~cm$^{-1}$,
the in-plane C--H asymmetric bend, is the most unsymmetric vibration, and
is chosen to demonstrate conservation of vibrational purity.

Mode 1 has a  frequency  corresponding  to  3.44  kcal/mole  and  a
predicted vibrational time of $27.69$~fs.  By direct calculation, using the
DRC, the cycle time is $27.55$~fs.  The rate of decay of this mode  has  an
estimated half-life of a few thousands femtoseconds.
                        
\subsubsection{Rate of decay of starting mode}
For trajectories initiated by an IRC=n  calculation,  whenever  the
potential  energy is a minimum the current velocity is compared with the
supplied velocity.  The square of the cosine of the  angle  between  the
two  velocity vectors is a measure of the intensity of the original mode
in the current vibration.

\subsubsection{Half-Life for decay of initial mode}
Vibrational purity is assumed to decay according to  zero'th  order
kinetics. The  half-life is thus $-0.6931472t/\log(\psi^2)$~fs, where $\psi^2$
is the square of the overlap integral of the original vibration with the
current  vibration.   Due to the very slow rate of decay of the starting
mode, several half-life calculations  should  be  examined.   Only  when
successive  half-lives  are  similar  should any confidence be placed in
their value.
                              
\subsubsection{DRC print options}
The amount of output in the DRC is  controlled  by  three  sets  of
options.  These sets are:\index{H--PRIORITY}
\begin{itemize}
\item Equivalent Keywords H-PRIORITY, T-PRIORITY, and X-PRIORITY
\item Potential Energy Turning Point option.
\item Geometry Maxima Turning Point options.
\end{itemize}
If T-PRIORITY is used then  turning  points  cannot  be  monitored.
Currently  H-PRIORITY and X-PRIORITY are not implemented, but will be as
soon as practical.

To monitor geometry turning points, put  a ``T'' in  place  of  the
geometry optimization flag for the relevant geometric variable.

To monitor the potential energy turning points, put a ``T'' for  the
flag for atom 1 bond length (Do not forget to put in a bond-length (zero
will do)!).

The effect of these flags together is as follows.
\begin{enumerate}
\item No options:  All calculated points will be printed.  No turning
points will be calculated.

\item Atom 1 bond length flagged with a ``T'': If  T-PRIO,  etc.   are
            NOT  specified,  then  potential  energy turning points will be
            printed.

\item Internal coordinate flags set to ``T'':  If T-PRIO, etc.  are NOT
            specified,  then geometry extrema will be printed.  If only one
            coordinate is flagged, then the turning point will be displayed
            in  chronologic  order; if several are flagged then all turning
            points occuring in a given time-interval  will  be  printed  as
            they  are  detected.   In  other  words,  some  may  be  out of
            chronologic order.  Note that each coordinate flagged will give
            rise  to a different geometry:  minimize flagged coordinates to
            minimize output.

\item Potential and geometric flags set:  The effect is equivalent to
            the sum of the first two options.
\item T-PRIO set:  No turning points will be  printed,  but  constant
            time-slices  (by  default  $0.1$~fs)  will  be used to control the
            print.
\end{enumerate}

\section{Sparkles}
Four extra `elements'' have been put into  MOPAC.   These  represent
pure  ionic  charges,  roughly  equivalent  to  the  following  chemical
entities:
\begin{verbatim}
    Chemical Symbol          Equivalent to

          +                 Tetramethyl ammonium radical, Potassium 
                            atom or Cesium atom.
          ++                Barium atom.
          -                 Borohydride radical, Halogen, or 
                            Nitrate radical
          --                Sulfate, oxalate.
\end{verbatim}


For  the  purposes  of  discussion  these   entities   are   called
`sparkles':  the name arises from consideration of their behavior.

\subsection*{Behavior of sparkles in MOPAC}
Sparkles have the following properties:
\begin{enumerate}
\item Their nuclear charge is integer, and is $+1$, $+2$, $-1$,  or  $-2$;
there  are  an  equivalent  number  of  electrons  to  maintain
electroneutrality, $+1$, $+2$, $-1$, and $-2$ respectively.  For example,
a  `+'  sparkle  consists  of  a  unipositive  nucleus  and  an
electron.  The electron is donated  to  the  quantum  mechanics
calculation.

\item  They all have an  ionic  radius  of  $0.7$~\AA.   Any  two
            sparkles  of  opposite  sign  will  form  an  ion-pair  with  a
            interatomic separation of $1.4$~\AA.

\item They have a zero heat  of  atomization,  no  orbitals,  and  no
            ionization potential.
\end{enumerate}

        They can be regarded as unpolarizable ions of diameter $1.4$\AA.   They
   do  not  contribute  to  the  orbital count, and cannot accept or donate
   electrons.

        Since they appear as uncharged species  which  immediately  ionize,
   attention  should  be  given  to  the  charge  on the whole system.  For
   example, if the alkaline metal salt of formic acid was run, the  formula
   would be: \verb/HCOO+/ where `+' is the unipositive sparkle.  
   The charge on the system would then be zero.

        A water molecule polarized by a positive  sparkle  would  have  the
   formula H$_2$O$^+$, and the charge on the system would be +1.

        At first sight, a sparkle would appear to be  too  ionic  to  be  a
   point charge and would combine with the first charge of opposite sign it
   encountered.

This representation is faulty, and a better description would be of
an  ion,  of diameter $1.4$\AA, and the charge delocalized over its surface.
Computationally, a sparkle is an integer  charge  at  the  center  of  a
repulsion  sphere  of form $\exp(-\alpha r)$.  The hardness of the sphere is
such that other atoms or sparkles can approach within about $2$\AA\
   quite easily, but only with great difficulty come closer than $1.4$\AA.

\subsection*{Uses of Sparkles}
\begin{enumerate}
\item They can be used as counterions, e.g.  for acid anions  or  for
      cations.   Thus,  if  the ionic form of an acid is wanted, then
    the moieties H$\cdot$X, H$\cdot -$, and $+\cdot$X could be examined.

\item Two sparkles of equal and opposite sign can form a  dipole  for
       mimicking solvation effects.  Thus water could be surrounded by
 six dipoles to simulate the solvent cage.  A dipole of value  D
 can  be made by using the two sparkles + and $-$, or using ++ and
\verb/ --/.  If + and $-$ are used, the inter-sparkle separation would be
  $D/4.803$\AA.  If \verb/++/ and \verb/--/ are used, the separation would
  be $D/9.606$\AA.  If the inter-sparkle separation is  less
  than $1.0$\AA\  (a situation that cannot occur naturally)
  then the energy due to the dipole on its own is subtracted from
  the total energy.

\item They can operate  as  polarization  functions.   A  controlled,
            shaped  electric  field  can  easily  be  made from two or more
            sparkles.  The polarizability in cubic Angstroms of a  molecule
            in any particular orientation can then easily be calculated.
\end{enumerate}

\section{Mechanism of the frame in FORCE calculation}
\index{frame!description of}
   The FORCE calculation uses cartesian coordinates, and all 3N  modes
   are  calculated, where N is the number of atoms in the system.  Clearly,
   there will be 5 or 6 ``trivial'' vibrations,  which  represent  the  three
   translations  and two or three rotations.  If the molecule is exactly at
   a stationary point, then these ``vibrations'' will have a  force  constant
   and  frequency  of  precisely  zero.   If the force calculation was done
   correctly, and the molecule was not exactly at a stationary point,  then
   the  three  translations should be exactly zero, but the rotations would
   be non-zero.  The extent to  which  the  rotations  are  non-zero  is  a
   measure of the error in the geometry.

        If  the  distortions  are  non-zero,  the  trivial  vibrations  can
   interact  with  the  low-lying genuine vibrations or rotations, and with
   the transition vibration if present.

        To prevent this the analytic form of the rotations  and  vibrations
   is  calculated,  and arbitrary eigenvalues assigned; these are 500, 600,
   700, 800, 900, and 1000 millidynes/angstrom for Tx, Ty, Tz, Rx,  Ry  and
   Rz  (if  present),  respectively.  The rotations are about the principal
   axes of inertia for the system, taking  into  account  isotopic  masses.
   The ``force matrix'' for these trivial vibrations is determined, and added
   on to the calculated force matrix.  After diagonalization the  arbitrary
   eigenvalues are subtracted off the trivial vibrations, and the resulting
   numbers are the ``true'' values.  Interference with genuine vibrations  is
   thus avoided.



\section{Configuration interaction}
\index{MECI!description of}
   MOPAC  contains   a   very   large   Multi-Electron   Configuration
   Interaction  calculation,  MECI,  which  allows almost any configuration
   interaction calculation to be performed.  Because of its complexity, two
   distinct  levels  of  input are supported; the default values will be of
   use to the novice while an expert has available  an  exhaustive  set  of
   keywords from which a specific C.I. can be tailored.

        A  MECI  calculation  involves  the  interaction   of   microstates
   representing  specific  permutations  of  electrons  in a set of M.O.'s.
   Starting with a set electronic configuration,  either  closed  shell  or
   open  shell, but unconditionally restricted Hartree-Fock, the first step
   in a MECI calculation is the removal from the M.O.'s of the electrons to
   be used in the C.I.

        Each microstate is then constructed  from  these  empty  M.O.'s  by
   adding  in  electrons  according  to  a prescription.  The energy of the
   configuration is evaluated, as is the energy  of  interaction  with  all
   previously-defined  configurations.   Diagonalization  then  results  in
   state functions.  From the eigenvectors the expectation value of 
   $s^2$ is
   calculated, and the spin-states of the state functions calculated.

\subsection*{General overview of keywords}
Keywords associated with the operations of MECI are:
\begin{verbatim}
     SINGLET                DOUBLET               EXCITED
     TRIPLET                QUARTET               BIRADICAL
     QUINTET                SEXTET                ESR
     OPEN(n1,n2)            C.I.=n                MECI
     ROOT=n
\end{verbatim}


        Each keyword may imply others; thus TRIPLET implies  an  open-shell
   system,  therefore  OPEN(2,2),  and  C.I.=2  are  implied,  if  not user
   specified.



\subsection*{Starting electronic configuration}
   MECI is restricted  to  RHF  calculations,  but  with  that  single
   restriction  any  starting configuration will be supported.  Examples of
   starting configurations would be
\begin{verbatim}
    System               KeyWords used       Starting Configuration

   Methane                 <none>             2.00 2.00 2.00 2.00 2.00
   Methyl Radical          <none>             2.00 2.00 2.00 2.00 1.00
   Twisted Ethylene        TRIPLET            2.00 2.00 2.00 1.00 1.00
   Twisted Ethylene        OPEN(2,2)          2.00 2.00 2.00 1.00 1.00
   Twisted Ethylene Cation OPEN(1,2)          2.00 2.00 2.00 0.50 0.50
   Methane Cation          CHARGE=1 OPEN(5,3) 2.00 2.00 1.67 1.67 1.67
\end{verbatim}

        Choice of starting configuration is  important.   For  example,  if
   twisted  ethylene,  a ground-state triplet, is not defined using TRIPLET
   or OPEN(2,2), then  the  closed-shell  ground-state  structure  will  be
   calculated.   Obviously,  this configuration is a legitimate microstate,
   but from the symmetry of the system a better choice would be  to  define
   one electron in each of the two formally degenerate pi-type M.O.'s.  The
   initial SCF calculation  does  not  distinguish  between  OPEN(2,2)  and
   TRIPLET  since  both  keywords  define  the same starting configuration.
   This can be verified by monitoring the convergence using PL,  for  which
   both keywords give the same SCF energy.



\subsection*{Removal of electrons from starting configuration}
For a starting configuration of alpha M.O. occupancies $O(i)$, $O(i)$
being  in  the  range 0.0 to 1.0, the energies of the M.O.'s involved in
the MECI can be calculated from:
$$ E(i) = \sum_j\left\{ [2J(i,j)-K(i,j)]O(j)\right\} $$
where $J(i,j)$ and $K(i,j)$ are the coulomb and exchange  integrals  between
M.O.'s  $i$  and  $j$.  The M.O.  index $j$ runs over those M.O.'s involved in
the MECI only.  Most MECI calculations will  involve  between  1  and  5
M.O.'s,  so  a system with about 30 filled or partly filled M.O.'s could
have M.O.'s 25--30 involved.  The  resulting  eigenvalues  correspond  to
those  of  the  cationic  system  resulting from removal of n electrons,
where n is twice the sum of the  orbital  occupancies  of  those  M.O.'s
involved in the C.I.

The arbitrary zero of energy in a MECI calculation is the  starting
   ground state, without any correction for errors introduced by the use of
   fractional occupancies.  In order to calculate the energy of the various
   configurations,  the  energy  of  the  vacuum  state  (i.e.,  the  state
   resulting from removal of the electrons used in the C.I.)  needs  to  be
   evaluated.  This energy is defined by:
$$ GSE = \sum_i\left[ E(i)O(i) + J(i,i) \times O(i) \times O(i)
+ \sum_{j<i}\{ 2[2J(i,j) - K(i,j)] \times O(i) \times O(j)\} \right] $$

\subsection*{Formation of microstate configuration}
Microstates are particular electron configurations.  Thus if  there
are  5  electrons  in  5  levels,  then  various microstates could be as
follows:

\begin{verbatim}
               Microstates for 5 electrons in 5 M.O.'s
      Electron Configuration               Electron Configuration

        Alpha       Beta       M(s)          Alpha       Beta        M(s)
      1 2 3 4 5  1 2 3 4 5                 1 2 3 4 5  1 2 3 4 5

1     1,1,1,0,0  1,1,0,0,0    1/2      4    1,1,1,1,1  0,0,0,0,0     5/2
2     1,1,0,0,0  1,1,1,0,0   -1/2      5    1,1,0,1,0  1,1,0,0,0     1/2
3     1,1,1,0,0  0,0,0,1,1    1/2      6    1,1,0,1,0  1,0,1,0,0     1/2
\end{verbatim}
     
For  5  electrons  in  5   M.O.'s   there   are   252   microstates
($10!/(5!5!)$),  but as states of different spin do not mix, we can use a
   smaller  number.   If  doublet  states  are  needed  then   100   states
($5!/(2!3!)(5!/3!2!$)  are  needed.   If  only  quartet  states  are of
interest then 25 states ($5!/(1!4!)(5!/4!1!$) are  needed  and  if  the
   sextet state is required, then only one state is calculated.

   In  the  microstates  listed,   state   1   is   the   ground-state
   configuration.   This can be written as (2,2,1,0,0), meaning that M.O.'s
   1 and 2 are doubly occupied, M.O.  3 is  singly  occupied  by  an  alpha
   electron, and M.O.'s 4 and 5 are empty.  Microstate 1 has a component of
   spin of 1/2, and is a pure doublet.  By Kramer's degeneracy --- sometimes
   called time-inversion symmetry --- microstate 2 is also a doublet, and has
   a spin of 1/2 and a component of spin of $-1/2$.

   Microstate 3, while it has a component of spin of  1/2,  is  not  a
   doublet,  but  is  in  fact  a  component  of a doublet, a quartet and a
   sextet.  The coefficients of these states can  be  calculated  from  the
   Clebsch-Gordon  3-J  symbol.  For example, the coefficient in the sextet
   is $1/\sqrt(5)$.

   Microstate 4 is a pure sextet.  If all 100 microstates of component
   of  spin  =  1/2  were used in a C.I., one of the resulting states would
   have the same energy as the state resulting from microstate 4.

   Microstate 5 is an excited doublet, and microstate 6 is an  excited
   state of the system, but not a pure spin-state.

   By default, if n M.O.'s are included in the MECI, then all possible
   microstates which give rise to a component of spin = 0 for even electron
   systems, or 1/2 for odd electron systems, will be used.
\begin{verbatim}
              Permutations of Electrons among Molecular Orbitals

     (0,1) =   0      (2,4) = 1100   (3,5) = 11100   (2,5) = 11000
                              1010           11010           10100
     (1,1) =   1              1001           11001           10010
                              0110           10110           10001
     (0,2) =   0              0101           10101           01100
                              0011           10011           01010
     (1,2) =  10                             01110           01001
              01      (1,4) = 1000           01101           00110
                              0100           01011           00101
     (1,3) = 100              0010           00111           00011
             010              0001
             001

     (2,3) = 110
             101
             011

              Sets of Microstates for Various MECI Calculations
           Odd Electron Systems        Even Electron Systems
             Alpha   Beta   No. of        Alpha   Beta   No. of 
                            Configs.                     Configs.
      C.I.=1 (1,1) * (0,1)  =   1          (1,1) * (1,1) =    1
           2 (1,2) * (0,2)  =   2          (1,2) * (1,2) =    4
           3 (2,3) * (1,3)  =   9          (2,3) * (2,3) =    9
           4 (2,4) * (1,4)  =  24          (2,4) * (2,4) =   36
           5 (3,5) * (2,5)  = 100          (3,5) * (3,5) =  100
\end{verbatim}


\subsection*{Multi electron configuration interaction}
The numbering of the M.O.'s used  in  the  MECI  is  standard,  and
   follows  the  Aufbau  principle.   The  order  of filling is in order of
   energy, and alpha before beta.  This point is  critically  important  in
   deciding  the  sign of matrix elements.  For a 5 M.O.  system, then, the
   order of filling is:
 $$      (1)(\bar{1})(2)(\bar{2})(3)(\bar{3})(4)(\bar{4})(5)(\bar{5}) $$

A triplet state arising from two microstates, each with a component
of spin = 0, will thus be the positive combination.
$$        (\bar{1})(2)   +    (1)(\bar{2}) $$

This is in variance  with  the  sign  convention  used  in  earlier
   programs  for running MNDO.  This standard sign convention was chosen in
   order to allow the signs of the microstate coefficients  to  conform  to
   those resulting from the spin step-down operator.

   Matrix elements between all pairs of microstates are calculated  in
   order   to   form  the  secular  determinant.   Many  elements  will  be
   identically zero, due to the interacting determinants differing by  more
   than two M.O.'s.  For the remaining interactions the following types can
   be identified.
\begin{enumerate}
\item The two determinants are identical:
No permutations are necessary in order  to  calculate  the
sign of the matrix element.  $E(p,p)$ is given simply by:
% Equation uncertain
%\begin{eqnarray*}
%E(p,p) & = & \sum_i\left[ O_{\alpha}(i,p) 
%[Eig(i) + \sum\left\{(1/2)(\langle ii|jj \rangle - \langle ij|ij \rangle )  
%O_{\alpha}(j,p) + (\langle ii|jj \rangle) O_{\beta}(j,p)]\right\} \\
%       & + & \sum_i \left(O_{\beta}(i,p) 
%[Eig(i) + \sum\left\{(1/2)(\langle ii|jj \rangle) - (\langle ij|ij \rangle) 
%O_{\beta}(j,p)]\right\}\right)\right]
%\end{eqnarray*}

where: $O_{\alpha}(i,p)$ is the occupancy of $\alpha$ M.O. $i$ 
in microstate $p$ and $O_{\beta}(i,p)$ is the occupancy of 
$\beta$ M.O. $i$ in microstate $p$.

\item  Determinants differing by exactly one M.O.:
The differing M.O.  can be of type $\alpha$ or $\beta$.   It  is
sufficient  to  evaluate  the  case in which both M.O.'s are of
alpha type, the beta form is obtained in like manner.
$$
E(p,q) = \sum_k \left\{ (\langle ij|kk \rangle - \langle ik|jk \rangle ) 
 [Occa(k) - Occg(k)] +  (\langle ij|kk \rangle)  [Occb(k) - Occg(k)]
\right\} 
$$
$E(p,q)$ may need to be multiplied by $-1$, if the number of 
two electron permutations required to bring M.O.'s $i$ and $j$
into coincidence is odd.

Where $Occa(k)$ is the alpha molecular orbital occupancy  in
the configuration interaction.

\item  Determinants differing by exactly two M.O.'s:
The two M.O.'s can have the same or opposite spins.  Three
cases can be identified:
\begin{enumerate}
\item Both M.O.'s have alpha spin:
For the first microstate having M.O.'s $i$ and $j$, and
the  second  microstate  having  M.O.'s $k$ and $l$, the matrix
element connecting the two microstates is given by:
$$ Q(p,q) =  \langle ik|jl \rangle - \langle il|jk \rangle $$
$E(p,q)$ may need to be multiplied by $-1$, if the number of 
two electron permutations required to bring M.O. $i$ into
coincidence with M.O. $k$ and M.O. $j$ into coincidence with
M.O. $l$ is odd.

\item Both M.O.'s have beta spin:
The matrix element is calculated in the same manner as
in the previous case.

\item  One M.O.  has alpha spin, and one beta spin:
For the first microstate having  M.O.'s  alpha(i)  and
beta(j),  and  the second microstate having M.O.'s alpha(k)
and  beta(l),  the  matrix  element  connecting   the   two
microstates is given by:
$$ Q(p,q) =  \langle ik|jl \rangle       $$
$E(p,q)$ may need to be multiplied by $-1$, if the number of 
two electron permutations required to bring M.O. $i$ into
coincidence with M.O. $k$ and M.O. $j$ into coincidence with
M.O. $l$ is odd.
\end{enumerate}
\end{enumerate}

\subsection*{States arising from various calculations}
Each MECI calculation invoked by use of the keyword C.I.=n normally
   gives  rise to states of quantized spins.  When C.I. is used without any
   other modifying keywords, the following states will be obtained.
\begin{verbatim}
No. of M.O.'s      States Arising            States Arising From
              From Odd Electron Systems    Even Electron Systems
  in MECI     Doublets                     Singlets Triplets 

    1            1                            1
    2            2                            3        1
    3            8         1                  6        3
    4           20         4                 20       15        1
    5           75        24       1         50       45        5
\end{verbatim}
These numbers of spin states will be obtained irrespective  of  the
chemical nature of the system.
          
\subsection*{Calculation of spin-states}
In order to calculate the spin-state, the expectation value  of  $S2$
is calculated.
%\begin{eqnarray*}
%S2 & = & S(S+1) = S_z^2 + 2 S(+)S(-) \\
%   & = & Ne - \sum_i\left\{
%C(i,k) C(i,k) \left((1/4)(N_{\alpha}(i)-N_{\beta}(i))^2 
%+ \sum_l O_{\alpha}(l,i) O_{\beta}(l,i)\right)
%+ \sum_j2\left[C(i,k) C(j,k) [\delta C(i,k)\{S(+)S(-)\} C(j,k)]\right]
%\right\}
%\end{eqnarray*}

where  $Ne$  is the no. of electrons in C.I.,
$C(i,k)$  is the coefficient of microstate $i$ in State $k$,
$N_{\alpha}(i)$ is the number of alpha electrons in microstate $i$,
$N_{\beta}(i)$ is the number of beta electrons in microstate $i$,
$O_{\alpha}(l,k)$ is the occupancy of alpha M.O. $l$ in microstate $k$,
$O_{\beta}(l,k)$ is the occupancy of beta M.O. l in microstate $k$,
$S(+)$ is the spin shift up or step up operator,
$S(-)$ is the spin shift down or step down operator,
the Kroneker delta is 1 if the two terms in brackets following it 
           are identical.

The spin state is calculated from: 
$$ S = (1/2) [\sqrt(1+4 S2) - 1 ]$$
In practice, $S$  is  calculated  to  be  exactly  integer,  or  half
integer.   That  is,  there is insignificant error due to approximations
   used.  This does not mean, however, that the method  is  accurate.   The
   spin  calculation  is  completely precise, in the group theoretic sense,
   but the accuracy of the calculation is limited by the Hamiltonian  used,
   a space-dependent function.

\subsection*{Choice of state to be optimized}
   MECI can calculate a large number of states of various total  spin.
   Two  schemes are provided to allow a given state to be selected.  First,
   ROOT=n will, when used on its own, select the n'th  state,  irrespective
   of  its  total  spin.  By default n=1.  If ROOT=n is used in conjunction
   with a keyword from the set SINGLET, DOUBLET, TRIPLET, QUARTET, QUINTET,
   or  SEXTET,  then  the  n'th  root of that spin-state will be used.  For
   example, ROOT=4 and SINGLET will select the 4th singlet state.  If there
   are  two  triplet  states  below the fourth singlet state then this will
   mean that the sixth state will be selected.

\subsubsection{Calculation of unpaired spin density}

        Starting  with  the  state  functions  as  linear  combinations  of
   configurations,  the  unpaired  spin density, corresponding to the alpha
   spin density minus the beta spin density, will  be  calculated  for  the
   first  few  states.   This  calculation  is straightforward for diagonal
   terms, and only those terms are used.

\section{Reduced masses in a force calculation}
Reduced masses for a diatomic are given by:
$$ \frac{m_1 \times m_2}{m_1 + m_2} $$

For a Hydrogen molecule the reduced mass is thus 0.5;  for  heavily
hydrogenated  systems,  e.g.  methane, the reduced mass can be very low.
A vibration involving only heavy atoms , e.g.  a C--N in cyanide,  should
give a large reduced mass.

For the `trivial' vibrations the reduced mass is  ill-defined,  and
where this happens the reduced mass is set to zero.

\section{Use of SADDLE calculation}
A SADDLE calculation uses two complete geometries, as shown on  the
following  data  file  for the ethyl radical hydrogen migration from one
methyl group to the other.
\begin{verbatim}
    Line  1:            UHF  SADDLE
    Line  2:         ETHYL RADICAL HYDROGEN MIGRATION
    Line  3: 
    Line  4:     C    0.000000 0    0.000000 0    0.000000 0   0  0  0
    Line  5:     C    1.479146 1    0.000000 0    0.000000 0   1  0  0
    Line  6:     H    1.109475 1  111.328433 1    0.000000 0   2  1  0
    Line  7:     H    1.109470 1  111.753160 1  120.288410 1   2  1  3
    Line  8:     H    1.109843 1  110.103163 1  240.205278 1   2  1  3
    Line  9:     H    1.082055 1  121.214083 1   38.110989 1   1  2  3
    Line 10:     H    1.081797 1  121.521232 1  217.450268 1   1  2  3
    Line 11:     0    0.000000 0    0.000000 0    0.000000 0   0  0  0
    Line 12:     C    0.000000 0    0.000000 0    0.000000 0   0  0  0
    Line 13:     C    1.479146 1    0.000000 0    0.000000 0   1  0  0
    Line 14:     H    1.109475 1  111.328433 1    0.000000 0   2  1  0
    Line 15:     H    1.109470 1  111.753160 1  120.288410 1   2  1  3
    Line 16:     H    2.109843 1   30.103163 1  240.205278 1   2  1  3
    Line 17:     H    1.082055 1  121.214083 1   38.110989 1   1  2  3
    Line 18:     H    1.081797 1  121.521232 1  217.450268 1   1  2  3
    Line 19:     0    0.000000 0    0.000000 0    0.000000 0   0  0  0
    Line 20: 
\end{verbatim}

Details of the mathematics of SADDLE appeared  in  print  in  1984,
   (M. J. S. Dewar,  E. F. Healy,  J. J. P. Stewart, 
   {\em J. Chem. Soc.  Faraday Trans.  II\/}, 3, 227, (1984)) so only a 
   superficial  description  will  be given here.

The main steps in the saddle calculation are as follows:
\begin{enumerate}
\item The heats of formation of both systems are calculated.

\item A vector $R$ of length $3N-6$ defining the difference  between  the
            two geometries is calculated.

\item  The scalar $P$ of  the  difference  vector  is  reduced  by  some
            fraction, normally about 5 to 15 percent.

\item Identify the geometry of lower energy; call this G.

\item Optimize G, subject to  the  constraint  that  it  maintains  a
            constant distance P from the other geometry.
\item If the newly-optimized geometry is higher in  energy  then  the
            other  geometry,  then  go to 1.  If it is higher, and the last
            two steps involved the same geometry  moving,  make  the  other
            geometry G without modifying $P$, and go to 5.

\item Otherwise go back to 2.
\end{enumerate}

        The mechanism of 5 involves the coordinates of the moving  geometry
   being  perturbed  by  an  amount equal to the product of the discrepancy
   between the calculated and required $P$ and the vector $R$.

        As the specification of the geometries is quite difficult, in  that
   the  difference  vector  depends  on  angles  (which  are,  of necessity
   ill-defined by 360 degrees) SADDLE can  be  made  to  run  in  cartesian
   coordinates  using  the  keyword XYZ.  If this option is chosen then the
   initial steps of the calculation are as follows:
\begin{enumerate}
\item Both geometries are converted into cartesian coordinates.

\item Both geometries are centered about the origin of cartesian space.

\item One geometry is  rotated  until  the  difference  vector  is  a
minimum  ---  this  minimum  is  within  1 degree of the absolute bottom.

\item The SADDLE calculation then proceeds as described above.
\end{enumerate}

\subsection*{Limitations:}
The two geometries must be related by a continuous  deformation  of
   the   coordinates.    By  default,  internal  coordinates  are  used  in
   specifying geometries, and  while  bond  lengths  and  bond  angles  are
   unambiguously  defined (being both positive), the dihedral angles can be
   positive or  negative.   Clearly  300  degrees  could  equally  well  be
   specified  as  $-60$ degrees.  A wrong choice of dihedral would mean that
   instead  of  the  desired  reaction  vector  being  used,  a  completely
   incorrect vector was used, with disastrous results.

        To correct this, ensure that one geometry can be obtained from  the
   other by a continuous deformation, or use the XYZ option.

\section{How to escape from a hilltop}
   A  particularly  irritating  phenomenon  sometimes  occurs  when  a
   transition  state is being refined.  A rough estimate of the geometry of
   the transition state has been obtained by either a  SADDLE  or  reaction
   path or by good guesswork.  This geometry is then refined by SIGMA or by
   NLLSQ, and the system characterized by a force calculation.   It  is  at
   this  point  that  things  often go wrong.  Instead of only one negative
   force constant, two or more are found.  In the past, the  recommendation
   has been to abandon the work and to go on to something less masochistic.
   It is possible, however, to  systematically  progress  from  a  multiple
   maximum to the desired transition state.  The technique used will now be
   described.

        If a multiple maximum is identified, most likely one negative force
   constant  corresponds  to  the  reaction  coordinate,  in which case the
   objective  is  to  render  the  other  force  constants  positive.   The
   associated  normal  mode  eigenvalues are complex, but in the output are
   printed as negative frequencies, and for the sake of simplicity will  be
   described  as  negative  vibrations.  Use DRAW-2 to display the negative
   vibrations,  and  identify  which  mode  corresponds  to  the   reaction
   coordinate.  This is the one we need to retain.

        Hitherto, simple motion in the direction of  the  other  modes  has
   proved  difficult.   However the DRC provides a convenient mechanism for
   automatically following a normal coordinate.  Pick the  largest  of  the
   negative  modes to be annihilated, and run the DRC along that mode until
   a minimum is reached.  At that point,  refine  the  geometry  once  more
   using  SIGMA  and  repeat  the  procedure  until  only one negative mode
   exists.

        To be on the safe  side,  after  each  DRC+SIGMA  sequence  do  the
   DRC+SIGMA  operation  again,  but use the negative of the initial normal
   coordinate to start the trajectory.  After both  stationary  points  are
   reached,  choose  the  lower  point  as  the starting point for the next
   elimination.  The lower point is chosen  because  the  transition  state
   wanted  is  the  highest  point  on  the  lowest  energy path connecting
   reactants to products.  Sometimes the two points will have equal energy:
   this  is normally a consequence of both trajectories leading to the same
   point or symmetry equivalent points.

        After  all  spurious  negative  modes  have  been  eliminated,  the
   remaining  normal  mode  corresponds to the reaction coordinate, and the
   transition state has been located.

        This technique is relatively rapid, and relies on starting  from  a
   stationary  point to begin each trajectory.  If any other point is used,
   the trajectory will  not  be  even  roughly  simple  harmonic.   If,  by
   mistake,  the reaction coordinate is selected, then the potential energy
   will  drop  to  that  of  either  the  reactants  or  products,   which,
   incidentally,  forms a handy criterion for selecting the spurious modes:
   if the potential energy only drops by  a  small  amount,  and  the  time
   evolution  is  roughly  simple  harmonic,  then  the  mode is one of the
   spurious modes.  If there is any doubt as to whether a minimum is in the
   vicinity  of  a stationary point, allow the trajectory to continue until
   one complete cycle is executed.  At that point the  geometry  should  be
   near to the initial geometry.

        Superficially, a line-search might appear more attractive than  the
   relatively  expensive  DRC.   However,  a line-search in cartesian space
   will normally not locate the minimum in a mode.  An obvious  example  is
   the mode corresponding to a methyl rotation.

\subsection*{Keyword Sequences to be Used}
\begin{enumerate}
\item To locate the starting stationary point  given  an  approximate
            transition state:-
\verb/SIGMA/

\item To define the normal modes:-
\verb/FORCE ISOTOPE/

At this point, copy all the files to a second filename, for use later.

\item Given vibrational frequencies of $-654$, $-123$,  234,  and  456,
identify  via DRAW--2 the normal coordinate mode, let's say that
is the $-654$ mode.  Eliminate the second mode by:
\verb/IRC=2 DRC T=30M RESTART LARGE/
  
            Use is made of the FORCE restart file.

\item Identify  the  minimum  in  the  potential  energy  surface  by
            inspection or using the VAX SEARCH command, of form:
\verb/SEARCH <Filename>.OUT %/


\item Edit out of the output file the data file corresponding to  the
            lowest point, and refine the geometry using:
\verb/SIGMA/

\item Repeat the last three steps but for the negative of the  normal
            mode,  using  the  copied files.  The keywords for the first of
            the two jobs are:
\verb/IRC=-2 DRC T=30M RESTART LARGE/


\item Repeat the last four steps  as  often  as  there  are  spurious
            modes.

\item Finally, carry out a DRC to confirm that the  transition  state
            does, in fact, connect the reactants and products.  The drop in
            potential energy  should  be  monotonic.   If  you  are  unsure
            whether  this  last  operation will work successfully, do it at
            any time you have a stationary point.  If it fails at the  very
            start,  then we are back where we were last year -- give up and
            go home!!
\end{enumerate}


\subsection{ EigenFollowing  }\index{Jensen@{\bf Jensen, Frank}}
\begin{center}
Description of the EF and TS function \\
by\\
Dr Frank Jensen  \\
Department of Chemistry  \\
Odense University  \\
5230 Odense  \\
Denmark
\end{center}

The current version of the EF optimization routine is a combination of the 
original EF algorithm of Simons et al. (J. Phys. Chem. 89, 52) as implemented by 
Baker (J. Comp. Chem. 7, 385) and the QA algorithm of Culot et al. 
(Theo. Chim. Acta 82, 189), with some added features for improving stability.

\index{Taylor expansion}\index{Hessian}
The geometry optimization is based on a second order Taylor expansion of 
the energy around the current point. At this point the energy, the gradient 
and some estimate of the Hessian are available. There are three fundamental 
steps in determining the next geometry based on this information:
\begin{itemize}
\item  finding the ``best'' step within or on the hypersphere with 
    the current trust radius.
\item possibly reject this step based on various criteria.
\item  update the trust radius.
\end{itemize}

\begin{enumerate}\index{Eigenvector!following}\index{OMIN}\index{NONR}\index{RSCAL}
\item For a minimum search the correct Hessian has only positive eigenvalues. 
For a Transition State (TS) search the correct Hessian should 
have exactly one negative 
eigenvalue, and the corresponding eigenvector should be in the direction 
of the desired reaction coordinate.  The geometry step is parameterized 
as $g/(s-H)$, where $s$ is a shift factor which ensure that the step-length is 
within or on the hypersphere. If the Hessian has the correct structure, a pure 
Newton-Raphson step is attempted. This corresponds to setting the shift factor 
to zero. If this step is longer than the trust radius, a P-RFO step is attempted. 
If this is also too long, then the best step on the hypersphere is made via 
the QA formula.  This three step procedure is the default. The pure NR step can 
be skipped by giving the keyword {\bf NONR}. An alternative to the QA step is to 
simply scale the P-RFO step down to the trust radius by a multiplicative 
constant, this can be accomplished by specifying {\bf RSCAL}.

\index{RMIN}\index{RMAX}\index{CYCLE}\index{MODE}
\item  
 Using the step determined from 1), the new energy and gradient are 
evaluated. If it is a TS search, two criteria are used in determining whether 
the step is ``appropriate''. The ratio between the actual and predicted energy 
change should ideally be 1. If it deviates substantially from this value, the 
second order Taylor expansion is no longer accurate. {\bf RMIN }and {\bf RMAX }(default 
values 0 and 4) determine the limits on how far from 1 the ratio can be before 
the step is rejected. If the ratio is outside the {\bf RMIN }and {\bf RMAX }limits, the 
step is rejected, the trust   radius reduced by a factor of two and a new step 
is determined. The second criteria is that the eigenvector along which the 
energy is being maximized should not change substantially between iterations. 
The minimum overlap of the TS eigenvector with that of the previous iteration 
should be larger than {\bf OMIN}, otherwise the step is rejected. Such a step 
rejection can be recognized in the output by the presence of (possibly more) 
lines with the   same CYCLE number. The default {\bf OMIN} value is 0.8, which allows 
fairly large changes to occur, and should be suitable for most uncomplicated 
systems. See below for a discussion of how to use {\bf RMIN}, {\bf RMAX }and {\bf OMIN} for 
difficult cases. The selection of which eigenvector to follow towards the TS is 
given by {\bf MODE=$n$}, where $n$ is the number of the Hessian eigenvector to follow. 
The default is {\bf MODE=1}. These features can be turned off by giving suitable 
values as keywords, e.g. {\bf RMIN=-100 RMAX=100} effectively inhibits step 
rejection. Similarly setting {\bf OMIN=0} disables step rejection based on large 
changes in the structure of the TS mode. The default is to use mode following 
even if the TS mode is the lowest eigenvector. This means that the TS mode may 
change to   some higher mode during the optimization. To turn of mode 
following, and thus always follow the mode with lowest eigenvalue, set {\bf MODE=0}. If 
it is a minimum search the new energy should be lower than the previous. 

The 
acceptance criteria used is that the actual/predicted ratio should be larger 
than {\bf RMIN}, which for the default   value of {\bf RMIN=0} is equivalent to a lower 
energy. If the ratio is below {\bf RMIN}, the step is rejected, the trust radius 
reduced by a factor of two and a new step is predicted. The {\bf RMIN}, {\bf RMAX }and {\bf OMIN} 
features has been introduced in the current version of EF to improve the 
stability of TS optimizations. Setting {\bf RMIN }and {\bf RMAX }close to one will give a 
very stable, but also very slow, optimization. Wide limits on {\bf RMIN }and {\bf RMAX }may 
in some cases give a   faster convergence, but there is always the risk that 
very poor steps are accepted, causing the optimization to diverge. The default 
values of 0 and 4 rarely rejects steps which would lead to faster convergence, 
but may occasionally accept poor steps. If TS searches are found to cause
problems, the first try should be to lower the limits to 0.5 and 2. Tighter 
limits like 0.8 and 1.2, or   even 0.9 and 1.1, will almost always slow the 
optimization down significantly but may be necessary in some cases. 

In minimum 
searches it is usually desirable that the energy decreases in each iteration. 
In certain very rigid systems, however, the initial diagonal Hessian may be so 
poor that the   algorithm cannot find an acceptable step larger than DDMIN, and 
the optimization terminates after only a few cycles with the ``TRUST RADIUS 
BELOW DDMIN" warning long before the stationary point is reached. In such cases 
the user can specify {\bf RMIN }to some negative value, say -10, thereby allowing 
steps which increases the energy. 

The algorithm has the capability of following 
Hessian eigenvectors other than the one with the lowest eigenvalue toward a TS. 
Such higher mode following are always much more difficult to make converge. 
Ideally, as the optimization progresses, the TS mode should at some point 
become the lowest eigenvector. Care must be taken during the optimization, 
however, that the   nature of the mode does not change all of a sudden, leading to 
optimization to a different TS than the one desired. {\bf OMIN} has been designed for 
ensuring that the nature of the TS mode only changes gradually, specifically 
the overlap between to successive TS modes should be higher than OMIN. While 
this concept at first appears very promising, it is not without problems when 
the Hessian is updated. 

As the updated Hessian in each step is only 
approximately correct,   there is a upper limit on how large the TS mode 
overlap between steps can be. To understand this, consider a series of steps 
made from the same geometry (e.g. at some point in the optimization), but with 
steadily smaller step-sizes. The update adds corrections to the Hessian to 
make it a better approximation to the exact Hessian. As the step-size become 
small, the updated Hessian converges toward the exact Hessian, at least in the 
direction of the step.   The old Hessian is constant, thus the overlap between 
TS modes thus does not converge toward 1, but rather to a constant value which 
indicate how well the old approximate Hessian resembles the exact Hessian. Test 
calculations suggest a typical upper limit around 0.9, although cases have been 
seen where the limit is more like 0.7. It appears that an updated Hessian in 
general is not of sufficient accuracy for reliably rejecting steps with TS 
overlaps much   greater than 0.80. The default {\bf OMIN} of 0.80 reflects the 
typical use of an updated Hessian. If the Hessian is recalculated in each step, 
however, the TS mode overlap does converge toward 1 as the step-size goes toward 
zero, and in this cases there is no problems following high lying modes. 


Unfortunately setting {\bf RECALC=1} is very expensive in terms of computer time, but 
used in conjecture with {\bf OMIN=0.90} (or possibly higher), and maybe also tighter 
limits on {\bf RMIN }  and RMAX, it represents an option of locating transitions 
structures that otherwise might not be possible. If problems are encountered with 
many step rejections due to small TS mode overlaps, try reducing {\bf OMIN}, maybe 
all the way down to 0. This most likely will work if the TS mode is the lowest 
Hessian eigenvector, but it is doubtful that it will produce any useful results 
if a high lying   mode is followed. Finally, following modes other than the 
lowest toward a TS indicates that the starting geometry is not ``close" to the 
desired TS. In most cases it is thus much better to further refined the 
starting geometry, than to try following high lying modes. There are cases, 
however, where it is very   difficult to locate a starting geometry which has 
the correct Hessian, and mode following may be of some use here
\end{enumerate}

\subsection{Franck-Condon considerations}\index{Franck-Condon}\label{FC}
This section was written based on discussions with
\begin{center} Victor I. Danilov\\
 Department of Quantum Biophysics\\ Academy of
Sciences of the Ukraine\\Kiev 143\\Ukraine\end{center}
The Frank-Condon principle states that electronic transitions take
place in times that are very short compared to the time required for
the nuclei to move significantly.  Because of this, care must be taken
to ensure that the calculations actually do reflect what is wanted.

Examples of various phenomena which can be studied are:
\begin{description}
\item[Photoexcitation]\index{Photoexcitation energy}

  If the purpose of
a calculation is to predict the energy of photoexcitation, then the
ground-state should first be optimized.  Once this is done,
then a C.I. calculation can be carried out using
{\bf 1SCF}.  With the appropriate keywords ({\bf MECI C.I.=$n$ } etc.),
the energy of photoexcitation to the various states can be predicted.

A more expensive, but more rigorous, calculation, would be to optimize
the geometry using all the C.I. keywords.  This is unlikely to change
the results significantly, however.
\item[Fluorescence]\index{Fluorescence}\index{Red-shift}\index{Photoemission}

If the excited state has a sufficiently long lifetime, so that the geometry
can relax, then if the system returns to the ground state by emission of
a photon, the energy of the emitted photon will be less (it will be red-shifted) than
that of the exciting photon.  To do such a calculation, proceed as follows:
\begin{itemize}
\item Optimize the ground-state geometry using all the keywords for the
later steps, but specify the ground state, e.g. {\bf C.I.=3 EF GNORM=0.01 MECI}.
\item Optimize the excited state, e.g. {\bf C.I.=3 ROOT=2 EF GNORM=0.01 MECI}.
\item Calculate the Franck-Condon excitation energy, using the results of the
ground-state calculation only.
\item Calculate the Franck-Condon emission energy, using the results of the
excited state calculation only.
\item If indirect emission energies are wanted, these can be obtained from
the $\Delta H_f$ of the optimized excited and optimized ground-state calculations.
\end{itemize}
In order for fluorescence to occur, the photoemission probability must be quite
large, therefore only transitions of the same spin are allowed.  For example,
if the ground state is S$_0$, then the fluorescing state would be      S$_1$.
\item[Phosphorescence]\index{Phosphorescence}
If the photoemission probability is very low, then the lifetime
of the excited state can be very long (sometimes minutes).  Such states can
become populated by S$_1 \rightarrow $ T$_1$ intersystem crossing. 
Of course, the geometry of the system will relax before the photoemission occurs.
\item[Indirect emission]
If the system relaxes from the excited electronic, ground vibrational state
to the ground electronic, ground vibrational state, then a more
complicated calculation is called for.  The steps of such a calculation are:
\begin{itemize}
\item Optimize the geometry of the excited state.
\item Using the same keywords, except that the ground state is specified, 
optimize the geometry of the ground state.
\item Take the difference in $\Delta H_f$ of the optimized excited and optimized ground-state calculations.
\item Convert this difference into the appropriate units.
\end{itemize}
\item[Excimers]\index{Excimers}
An excimer is a pair of molecules, one of which is in an electronic
excited state.  Such systems are usually stabilized relative to the
isolated systems.  Optimization of the geometries of such systems
is difficult.  Suggestions on how to improve this type of calculation
would be appreciated.
\end{description}
\section{Outer Valence Green's Function}\label{gf}
This section is based on materials supplied by
\begin{center}Dr David Danovich\\The Fritz Haber Research Center for 
Molecular Dynamics\\ The Hebrew University of Jerusalem\\ 91904 Jerusalem\\
Israel \end{center}

The OVGF technique was used with the self-energy part extended to include
third order perturbation corrections,~\cite{gf1}.  The higher order contributions
were estimated by the renormalization procedure.  The actual expression used to
calculate the self-energy part, $\sum_{pp}(w)$, chosen in the diagonal form,
is given in equation~(\ref{gfeq1}), where $\sum_{pp}^{(2)}(w)$ and $\sum_{pp}^{(3)}(w)$ are
the second- and third-order corrections, and $A$ is the screening factor accounting
for all the contributions of higher orders.
\begin{equation}\label{gfeq1}
\sum_{pp}(w) = \sum_{pp}^{(2)}(w)+(1-A)^{-1}\sum_{pp}^{3}(w)
\end{equation}
The particular expression which was used for the second-order corrections is given in
equation~(\ref{gfeq2}).
\begin{equation}\label{gfeq2}
\sum_{pp}^{(2)}(w) = \sum_a\sum_{i,j}\frac{(2V_{paij}-V_{paji})V_{paij}}{w+e_a-e_i-e_j}
+\sum_{a,b}\sum_i\frac{(2V_{piab}-V_{piba})V_{piab}}{w+e_i-e_a-e_b}
\end{equation}
where
$$
V_{pqrs} = \int\int\phi_p^*(1)\phi_q^*(2)(1/r_{12})\phi_r^*(1)\phi_s^*(2){\rm d}\tau_1{\rm d}\tau_2
$$

In equation~(\ref{gfeq2}), $i$ and $j$ denote occupied orbitals, $a$ and $b$ denote
virtual orbitals, $p$ denotes orbitals of unspecified occupancy, and $e$ 
denotes an orbital energy.
The equations are solved by an iterative procedure, represented in 
equation~(\ref{gfeq3}).
\begin{equation}\label{gfeq3}
w_p^{i+1}=e_p+\sum_{pp}(w^i)
\end{equation}

The SCF energies and the corresponding integrals, which were calculated by one of the semiempirical methods (MNDO, AM1, or PM3), were taken as the zero'th approximation
and all M.O.s may be included in the active space for the OVGF calculations.

The expressions used for $\sum_{pp}^{(3)}$ and $A$ are given in \cite{gf2}.

The OVGF method itself, is described in detail in \cite{gf1}.

\subsection{Example of OVGF calculation}
Because Danovich's OVGF method is new to MOPAC, users will want to see how well
it works.  The data-set {\bf test\_ green.dat} will calculate the first 8 I.P.s
for dimethoxy-$s$-tetrazine.  This calculation is discussed in detail in \cite{gf6}.
The experimental and calculated I.P.s are shown in Table~\ref{gftab}.
\begin{table}[hbt]
\caption{\label{gftab}OVGF Calculation, Comparison with Experiment}
\begin{center}
\begin{tabular}{lccccc}\\
M.O.     &  Expt*  &   PM3   & Error   &  OVGF(PM3)  &  Error \\
$n_1$    &  9.05   &   10.15 &  1.10   &   9.46      &   0.41 \\
$\pi_1$  &  9.6    &   10.01 &  0.41   &   9.65      &   0.05 \\
$n_2$    &  11.2   &   11.96 &  0.76   &  11.13      &  -0.07 \\
$\pi_2$  &  11.8   &   12.27 &  0.47   &  11.43      &  -0.37 \\
\end{tabular}

*: R. Gleiter,  V. Schehlmann, J. Spanget-Larsen, H. Fischer and F. A. Neugebauer,
{\em J. Org. Chem.}, {\bf 53}, 5756 (1988).
\end{center}
\end{table}


From this, we see that for PM3 the average error is 0.69eV, but after OVGF 
correction, the error drops to 0.22eV.  This is typical of nitrogen heterocycle 
calculations.

\section{COSMO (Conductor-like Screening Model)}\index{COSMO}
This section was written based on material provided by:
\begin{center} Andreas Klamt\\
Bayer AG\\
Q18, D-5090 Leverkusen-Bayerwerk\\
Germany
\end{center}

Unlike the Self-Consistent Reaction Field model~\cite{scrf}, the
 {\bf Co}nductor-like {\bf S}creening {\bf Mo}del (COSMO) is a new
continuum approach which, while more complicated, is computationally
quite efficient.  The expression for the total screening energy is simple
enough to allow the first derivatives of the energy with respect to atomic 
coordinates to be easily evaluated.

Details of the procedure have been submitted for publication: A. Klamt and
G. Schuurmann, {\em COSMO: A New Approach to Dielectric Screening in Solvents 
with Explicit Expressions for the Screening Energy and its Gradient},
{\bf J. Chem. Soc., Perkin Trans.}  2, 1993. (in press).

The COSMO procedure generates a conducting polygonal surface around the 
system (ion or molecule), at the van der Waals' distance.  By introducing
a $\varepsilon$-dependent correction factor,
$$
f(\varepsilon)=\frac{(\varepsilon-1)}{(\varepsilon+\frac{1}{2})}, 
$$
into the expressions for the screening energy and its gradient, the theory can
be extended to finite dielectric constants with only a small error.

The accuracy of the method can be judged by how well it reproduces known
quantities, such as the heat of solution in water (water has a dielectric
constant of 78.4 at 25$^{\circ}$C), Table~\ref{cosmo_tab}.  
Here, the keywords used were

{\bf      NSPA=60 GRADIENTS 1SCF  EPS=78.4 AM1 CHARGE=1}

From the Table we see that the glycine zwitterion becomes the stable form in
water, while the neutral species is the stable gas-phase form.

The COSMO method is easy to use, and the derivative calculation is of
sufficient precision to allow gradients of 0.1 to be readily achieved.
\begin{table}[hbt]
\begin{center}
\caption{\label{cosmo_tab} Calculated and Observed Hydration Energies}
\begin{tabular}{llrrrr}
\hline
Compound   &   Method  &  \multispan 2 \mbox{$\Delta H_f$ (kcal/mol)} & \multispan 2 Hydration \\
           &           & gas phase & solution phase & $\Delta H$(calc.)  & Enthalpy(exp.)~\dag  \\
\hline
NH$_4^+$ & AM1 & 150.6\hspace*{0.1in} & 59.5\hspace*{0.2in} & 91.1\hspace*{0.1in} & 88.0\hspace*{0.4in} \\
N(Me)$_4^+$  & AM1 & 157.1\hspace*{0.1in} & 101.1\hspace*{0.2in} & 56.0\hspace*{0.1in} & 59.9\hspace*{0.4in} \\
N(Et)$_4^+$  & AM1 & 132.1\hspace*{0.1in} &  84.2\hspace*{0.2in} & 47.9\hspace*{0.1in} & 57.0\hspace*{0.4in} \\
Glycine \\
neutral & AM1 &-101.6\hspace*{0.1in} & -117.3\hspace*{0.2in} & 15.7\hspace*{0.1in} & $--$\hspace*{0.4in} \\
zwitterion & AM1 & -59.2\hspace*{0.1in} & -125.6\hspace*{0.2in} & 66.4\hspace*{0.1in} & $--$\hspace*{0.4in} \\
\hline
\end{tabular}

\dag : Y. Nagano, M. Sakiyama, T. Fujiwara, Y. Kondo, J. Phys. Chem., {\bf 92},
5823 (1988).
\end{center}
\end{table}

\section{Solid state capability}
   Currently MOPAC can only  handle  up  to  one-dimensional  extended
   systems.   As  the solid-state method used is unusual, details are given
   at this point.

        If a polymer unit cell is large enough,  then  a  single  point  in
   k-space,  the Gamma point, is sufficient to specify the entire Brillouin
   zone.  The secular determinant for this  point  can  be  constructed  by
   adding together the Fock matrix for the central unit cell plus those for
   the adjacent unit cells.  The \mi{Born--von Karman} cyclic boundary conditions
   are satisfied, and diagonalization yields the correct density matrix for
   the Gamma point.

        At this point  in  the  calculation,  conventionally,  the  density
   matrix  for  each  unit  cell  is constructed.  Instead, the Gamma-point
   density  and  one-electron  density  matrices  are   combined   with   a
   ``Gamma-point-like'' Coulomb  and  exchange integral strings to produce a
   new Fock matrix.  The  calculation  can  be  visualized  as  being  done
   entirely in reciprocal space, at the Gamma point.

        Most  solid-state  calculations  take  a  very  long  time.   These
   calculations,   called ``Cluster'' calculations   after  the  original
   publication, require between 1.3 and 2 times  the  equivalent  molecular
   calculation.\index{cluster model}

        A minor `fudge'  is  necessary  to  make  this  method  work.   The
   contribution  to  the  Fock  matrix  element  arising  from the exchange
   integral between an atomic orbital and its equivalent  in  the  adjacent
   unit  cells  is  ignored.   This  is  necessitated  by the fact that the
   density matrix element involved is invariably large.

        The unit cell must be large enough that an atomic  orbital  in  the
   center  of  the  unit  cell has an insignificant overlap with the atomic
   orbitals at the ends of the  unit  cell.   In  practice,  a  translation
   vector  of more that about 7 or 8\AA\ is sufficient.  For one rare
   group of compounds a larger translation vector is needed.  Polymers with
   delocalized  $\pi$--systems,  and  polymers  with  very small band-gaps will
   require a larger translation  vector,  in  order  to  accurately  sample
   k-space.   For these systems, a translation vector in the order of 
   15-20 Angstroms is needed.

%%%%%%%%%%%%%%%%%%%%%%%%%%%%%%%%%%%%%%%%%%%%%%%%%%%%%%%%%%%%%%%%%%%%%

\chapter{Program}
The logic within MOPAC is best understood by use of flow-diagrams.

There are two main sequences, geometric and electronic.  These join
only  at  one  common  subroutine COMPFG.  It is possible, therefore, to
understand the geometric or electronic sections  in  isolation,  without
having studied the other section.
   
\section{Main geometric sequence}
\index{MOPAC! geometric structure}

\begin{verbatim}
                          ______ 
                         |      |
                         | MAIN |
                         |      |
                         |______|
        _____________________|______________________________
       |      |         ___|___       ____|_____    |   ___|___ 
       |   ___|___     |       |     |          |   |  |       |
       |  |IRC/DRC|    | FORCE |     |  REACT1  |   |  | PATHS |
       |  |  or   |    |       |     |          |   |  |       |
       |  |  EF   |    |_______|___  |__________|   |  |_______|
       |  |  or   |     |    |     |      |         |      |
       |  | POLAR |     |    |     |      |         |      |
       |  |_______|     |    |__   |      |_________|______|
      _|_____   |      _|____   |  |             ___|___ 
     | NLLSQ |  |     |      |  |  |____________|       |
     |  and  |  |     | FMAT |  |               | FLEPO |
     | POWSQ |  |     |      |  |               |       |
     |_______|  |     |______|  |               |_______|
  ____|___  |   |         |     |          _____|__   |
 | SEARCH | |   |         |     |         |        |  |
 |   or   | |   |         |     |         | LINMIN |  |
 | LOCMIN | |   |         |     |         |        |  |
 |________| |   |         |     |         |________|  | 
     |______|___|_________|_____|______________|______|
                         ____|___ 
                        |        |
                        | COMPFG |  (See ELECTRONIC SEQUENCE)
                        |        |
                        |________|
\end{verbatim}

\newpage
              
\section{Main electronic flow}
\index{MOPAC!electronic structure}
\begin{verbatim}
                           ________ 
                          |        |
                          | COMPFG |  (See GEOMETRIC SEQUENCE)
                          |        |
                          |________|
               _______________|____________________ 
           ___|___         ___|___      ___|____   |
          |       |       |       |    |        |  |
          | HCORE |_______| DERIV |____| GMETRY |  |
          |       |       |       |    |        |  |
          |_______|       |_______|    | SYMTRY |  |
              |            |     |     |        |  |
              |        ____|__   |     |________|  |
              |       |       |  |                 |
              |       | DCART |  |                 |
              |       |       |  |                 |
              |       |_______|  |______    _______|
              |         __|__          _|__|_      _________ 
              |        |     |        |      |    |         |
              |        | DHC |    ____| ITER |____|   RSP   |
              |        |     |   |    |      |    |         |
              |        |_____|   |    |______|    |_________|
              |         |   |    |        | |
              |___    __|   |    |        | |       ________ 
                 |    |     |    |        | |      |        |
                _|____|_    |____|_       | |______| DENSIT | 
               |        |  |       |      |        |        |
               | ROTATE |  | FOCK1 |      |        |  CNVG  |
               |        |  |       |      |        | PULAY  |
               | H1ELEC |  | FOCK2 |      |        |________|
               |        |  |       |      |        
               |________|  |_______|      |_____   
                   |                     |      |
                ___|__                   | MECI |
               |      |                  |      |
               | DIAT |                  |______|
               |      |                 
               |______|                 
                   |                     
                 __|_                    
                |    |                   
                | SS |                   
                |____|
\end{verbatim}
              
\section{Control within MOPAC}
\index{MOPAC!programming policy}
 Almost all the control information is passed via the  single  datum
 ``KEYWRD'',  a  string  of 80 characters, which is read in at the start of
 the job.

      Each subroutine is made independent, as far as  possible,  even  at
 the  expense  of  extra code or calculation.  Thus, for example, the SCF
 criterion is set in  subroutine  ITER,  and  nowhere  else.   Similarly,
 subroutine  DERIV  has  exclusive  control  of  the  step  size  in  the
 finite-difference calculation of the energy derivatives.  If the default
 values  are  to  be reset, then the new value is supplied in KEYWRD, and
 extracted via INDEX and READA.  The flow of control is  decided  by  the
 presence of various keywords in KEYWRD.

      When a subroutine is called, it assumes that all data required  for
 its  operation  are  available  in  either  common  blocks or arguments.
 Normally no check is made as to the validity of the data received.   All
 data  are ``owned'' by one, and only one, subroutine.  Ownership means the
 implied permission and ability to change the data.  Thus  MOLDAT ``owns''
 the  number  of  atomic orbitals, in that it calculates this number, and
 stores it in the variable NORBS.  Many subroutines use NORBS,  but  none
 of  them  is  allowed  to  change it.  For obvious reasons no exceptions
 should be made to this rule.   To  illustrate  the  usefulness  of  this
 convention,  consider the eigenvectors, C and CBETA.  These are owned by
 ITER.  Before ITER is called, C and CBETA are not calculated, after ITER
 has  been called C and CBETA are known, so any subroutine which needs to
 use the eigenvectors can do so in the certain knowledge that they exist.

      Any variables which are only  used  within  a  subroutine  are  not
 passed  outside the subroutine unless an overriding reason exists.  This
 is found in PULAY and CNVG, among  others  where  arrays  used  to  hold
 spin-dependent  data  are used, and these cannot conveniently be defined
 within the subroutines.  In these  examples,  the  relevant  arrays  are
 ``owned'' by ITER.

      A general subroutine, of which ITER  is  a  good  example,  handles
 three  kinds of data:  First, data which the subroutine is going to work
 on, for  example  the  one  and  two  electron  matrices;  second,  data
 necessary  to  manipulate  the  first set of data, such as the number of
 atomic orbitals; third, the calculated quantities, here  the  electronic
 energy, and the density and Fock matrices.

      Reference data are entered into a subroutine by way of  the  common
 blocks.  This is to emphasize their peripheral role.  Thus the number of
 orbitals, while essential to ITER, is not central to the task it has  to
 perform, and is passed through a common block.

      Data the subroutine is going to work on are passed via the argument
 list.  Thus the one and two electron matrices, which are the main reason
 for ITER's existence, are entered as two of the four arguments.  As ITER
 does  not  own  these  matrices it can use them but may not change their
 contents.  The other argument is EE, the electronic energy.  EE is owned
 by ITER even though it first appears before ITER is called.

      Sometimes common block data should  more  correctly  appear  in  an
 argument  list.   This is usually not done in order to prevent obscuring
 the main role the subroutine has to perform.  Thus ITER  calculates  the
 density and Fock matrices, but these are not represented in the argument
 list as the calling subroutine never needs to know them;  instead,  they
 are stored in common.

\subsection{Subroutine GMETRY}
\index{GMETRY! description}
\subsubsection{Description for programmers}
      GMETRY has two arguments, GEO and COORD.   On  input  GEO  contains
 either  (a)  internal coordinates or (b) cartesian coordinates.  On exit
 COORD contains the cartesian coordinates.

      The normal mode of usage is to supply the internal coordinates,  in
 which case the connectivity relations are found in common block GEOKST.

      If the contents of NA(1)  is  zero,  as  required  for  any  normal
 system, then the normal internal to cartesian conversion is carried out.

      If the contents of NA(1) is 99, then the coordinates found  in  GEO
 are  assumed  to  be  cartesian, and no conversion is made.  This is the
 situation in a FORCE calculation.

      A  further  option  exists  within  the   internal   to   cartesian
 conversion.  If STEP, stored in common block REACTN, is non-zero, then a
 reaction path is assumed, and  the  internal  coordinates  are  adjusted
 radially in order that the ``distance'' in internal coordinate space from
 the geometry specified in GEO is STEPP away from the geometry stored  in
 GEOA, stored in REACTN.

      During the internal to cartesian conversion, the angle between  the
 three  atoms used in defining a fourth atom is checked to ensure that it
 is not near to 0 or 180 degrees.  If it is near to  these  angles,  then
 there is a high probability that a faulty geometry will be generated and
 to prevent this the calculation is stopped and an error message printed.

 Note:
 \begin{enumerate}
 \item If the angle  is  exactly  0  or  180  degrees,  then  the
 calculation  is  not  terminated:   This  is  the  normal situation in a
 high-symmetry molecule such as propyne.

 \item The check is only made if the fourth atom has a bond angle
 which is not zero or 180 degrees.
 \end{enumerate}
                                
%%%%%%%%%%%%%%%%%%%%%%%%%%%%%%%%%%%%%%%%%%%%%%%%%%%%%%%%%%%%%%%%%%%%%

\chapter{Error messages produced by MOPAC}
\index{MOPAC!error messages}
\index{error messages}
\index{messages}
   MOPAC produces several hundred messages, all of which are  intended
   to  be  self-explanatory.  However, when an error occurs it is useful to
   have more information than is given in the standard messages.

        The following alphabetical list gives more complete definitions  of
   the messages printed.

\subsubsection{\tt AN UNOPTIMIZABLE GEOMETRIC PARAMETER \ldots}
        When internal coordinates are supplied, six coordinates  cannot  be
   optimized.   These  are  the  three coordinates of atom 1, the angle and
   dihedral on atom 2 and the dihedral on atom 3.  An attempt has been made
   to  optimize  one of these.  This is usually indicative of a typographic
   error, but might simply be an oversight.  Either way, the error will  be
   corrected and the calculation will not be stopped here.

\subsubsection{\tt ATOM NUMBER nn IS ILLDEFINED}
 The rules for definition of atom connectivity are:
 \begin{enumerate}
 \item Atom 2 must be connected to atom 1 (default - no override)

 \item Atom 3 must be connected to atom 1 or 2, and make an angle with
     2 or 1.

 \item All other atoms must be defined  in  terms  of  already-defined
     atoms:   these  atoms must all be different.  Thus atom 9 might
     be connected to atom 5, make an angle with atom 6, and  have  a
     dihedral  with  atom  7.  If the dihedral was with atom 5, then
     the geometry definition would be faulty.
 \end{enumerate}

 If any of these rules is broken, a fatal error message is  printed,
   and the calculation stopped.

\subsubsection{\tt ATOMIC NUMBER nn IS NOT AVAILABLE \ldots}
        An element has been used for which parameters  are  not  available.
   Only  if  a typographic error has been made can this be rectified.  This
   check is not exhaustive, in that even if  the  elements  are  acceptable
   there  are  some  combinations  of  elements within MINDO/3 that are not
   allowed.  This is a fatal error message.

                            
\subsubsection{\tt ATOMIC NUMBER OF nn ?}
        An atom has been specified with a negative or zero  atomic  number.
   This  is  normally  caused  by forgetting to specify an atomic number or
   symbol.  This is a fatal error message.

             
\subsubsection{\tt ATOMS  nn AND nn ARE SEPARATED BY nn.nnnn ANGSTROMS}
   Two genuine atoms (not dummies)  are  separated  by  a  very  small
   distance.    This  can  occur  when  a  complicated  geometry  is  being
   optimized, in which case the user may wish to  continue.   This  can  be
   done  by  using  the  keyword GEO-OK.  More often, however, this message
   indicates a mistake, and the calculation is, by default, stopped.

                  
\subsubsection{\tt ATTEMPT TO GO DOWNHILL IS UNSUCCESSFUL \ldots}
   A  quite  rare  message,  produced  by   Bartel's   gradient   norm
   minimization.  Bartel's method attempts to minimize the gradient norm by
   searching the gradient space for a minimum.  Apparently  a  minimum  has
   been found, but not recognized as such.  The program has searched in all
   $(3N-6)$ directions, and found no way down, but the criteria for a minimum
   have  not been satisfied.  No advice is available for getting round this
   error.

\subsubsection{\tt BOTH SYSTEMS ARE ON THE SAME SIDE \ldots}
   A non-fatal message, but still cause for concern.  During a  SADDLE
   calculation  the  two  geometries  involved are on opposite sides of the
   transition  state.   This  situation  is  verified  at  every  point  by
   calculating  the  cosine  of the angle between the two gradient vectors.
   For as long as it is negative, then the two geometries are  on  opposite
   sides  of  the  T/S.  If, however, the cosine becomes positive, then the
   assumption is made that one moiety has fallen over the T/S  and  is  now
   below  the other geometry.  That is, it is now further from the T/S than
   the other, temporarily  fixed,  geometry.   To  correct  this,  identify
   geometries  corresponding  to  points  on  each  side  of the T/S.  (Two
   geometries on the output separated by  the  message  
   \verb/"SWAPPING..."/)  and
   make  up  a  new  data-file using these geometries.  This corresponds to
   points on the reaction path near to the T/S.  Run a new job using  these
   two geometries, but with BAR set to a third or a quarter of its original
   value, e.g.  BAR=0.05.  This normally allows the T/S to be located.

\subsubsection{\tt C.I. NOT ALLOWED WITH UHF}
   There is no UHF configuration  interaction  calculation  in  MOPAC.
   Either remove the keyword that implies C.I. or the word UHF.

\subsubsection{\tt CALCULATION ABANDONED AT THIS POINT}
   A particularly annoying message!  In  order  to  define  an  atom's
   position,  the  three  atoms  used  in  the  connectivity table must not
   accidentally fall into a  straight  line.   This  can  happen  during  a
   geometry  optimization or gradient minimization.  If they do, and if the
   angle made by the atom being defined is not zero or  180  degrees,  then
   its  position  becomes  ill-defined.   This  is  not  desirable, and the
   calculation will stop in order to allow corrective action to  be  taken.
   Note  that  if  the  three  atoms  are in an exactly straight line, this
   message will not be triggered.  The good news is that the criterion used
   to  trigger  this  message was set too coarsely.  The criterion has been
   tightened so that this message now does  not  often  appear.   Geometric
   integrity does not appear to be compromized.

              
\subsubsection{\tt CARTESIAN COORDINATES READ IN, AND CALCULATION \ldots}
   If cartesian coordinates are read in, but the calculation is to  be
   carried  out  using  internal  coordinates,  then  either  all  possible
   geometric variables must be optimized, or none  can  be  optimized.   If
   only  some  are  marked  for  optimization  then  ambiguity exists.  For
   example, if the ``X'' coordinate of atom 6 is marked for optimization, but
   the ``Y'' is not, then when the conversion to internal coordinates takes
   place, the first coordinate becomes a bond-length,  and  the  second  an
   angle.   These bear no relationship to the ``X'' or ``Y'' coordinates.  This
   is a fatal error.

\subsubsection{\tt CARTESIAN COORDINATES READ IN, AND SYMMETRY \ldots}
        If cartesian coordinates are read in, but the calculation is to  be
   carried  out using internal coordinates, then any symmetry relationships
   between the cartesian coordinates will not be reflected in the  internal
   coordinates.   For  example, if the ``Y'' coordinates of atoms 5 and 6 are
   equal, it does not follow that  the  internal  coordinate  angles  these
   atoms make are equal.  This is a fatal error.

                              
\subsubsection{\tt ELEMENT NOT FOUND}    
   When an external file  is  used  to  redefine  MNDO,  AM1,  or  PM3
   parameters, the chemical symbols used must correspond to known elements.
   Any that do not will trigger this fatal message.
                           
                    
\subsubsection{\tt ERROR DURING READ AT ATOM NUMBER \ldots}
   Something is wrong with the geometry data.  In order to  help  find
   the  error,  the  geometry  already  read in is printed.  The error lies
   either on the last  line  of  the  geometry  printed,  or  on  the  next
   (unprinted) line.  This is a fatal error.

\subsubsection{\tt FAILED IN SEARCH, SEARCH CONTINUING}
        Not a fatal error.   The  McIver-Komornicki  gradient  minimization
   involves use of a line-search to find the lowest gradient.  This message
   is merely advice.  However, if SIGMA takes a long time,  consider  doing
   something  else,  such  as  using  NLLSQ, or refining the geometry a bit
   before resubmitting it to SIGMA.

 \subsubsection{\tt <<<<----**** FAILED TO ACHIEVE SCF. ****---->>>>}
        The SCF calculation failed to go to  completion;  an  unwanted  and
   depressing message that unfortunately appears every so often.

        To  date  three  unconditional  convergers  have  appeared  in  the
   literature:   the  SHIFT  technique,  Pulay's  method, and the Camp-King
   converger.  It would not  be  fair  to  the  authors  to  condemn  their
   methods.   In  MOPAC  all  sorts  of  weird  and  wonderful  systems are
   calculated, systems the authors of  the  convergers  never  dreamed  of.
   MOPAC  uses  a  combination  of all three convergers at times.  Normally
   only a quadratic damper is used.

        If this message appears, suspect first that the  calculation  might
   be  faulty, then, if you feel confident, use PL to monitor a single SCF.
   Based  on  the  SCF  results  either  increase  the  number  of  allowed
   iterations, default:  200, or use PULAY, or Camp-King, or a mixture.

        If nothing works, then consider slackening the SCF criterion.  This
   will   allow  heats  of  formation  to  be  calculated  with  reasonable
   precision, but the gradients are likely to be imprecise.

                  
\subsubsection{\tt GEOMETRY TOO UNSTABLE FOR EXTRAPOLATION \ldots}
   In a reaction path calculation the initial geometry for a point  is
   calculated by quadratic extrapolation using the previous three points.

        If a quadratic fit is likely to lead to an inferior geometry,  then
   the  geometry  of  the  last  point  calculated will be used.  The total
   effect  is  to  slow  down  the  calculation,  but  no  user  action  is
   recommended.

\subsubsection{\tt ** GRADIENT IS TOO LARGE TO ALLOW \ldots}
   Before a FORCE calculation can be performed the gradient norm  must
   be  so small that the third and higher order components of energy in the
   force field are negligible.  If, in the system  under  examination,  the
   gradient  norm  is  too  large,  the gradient norm will first be reduced
   using FLEPO, unless LET has been specified.  In  some  cases  the  FORCE
   calculation  may be run only to decide if a state is a ground state or a
   transition  state,  in  which   case   the   results   have   only   two
   interpretations.  Under these circumstances, LET may be warranted.

\subsubsection{\tt GRADIENT IS VERY LARGE \ldots}
        In a calculation of the thermodynamic properties of the system,  if
   the  rotation  and  translation vibrations are non-zero, as would be the
   case if the gradient norm was significant, then these `vibrations' would
   interfere  with  the  low-lying  genuine  vibrations.   The criteria for
   THERMO  are  much  more  stringent  than  for  a  vibrational  frequency
   calculation,  as  it is the lowest few genuine vibrations that determine
   the internal vibrational energy, entropy, etc.

\subsubsection{\tt ILLEGAL ATOMIC NUMBER}

        An element has been specified by an atomic number which is  not  in
   the  range  1  to  107.   Check the data:  the first datum on one of the
   lines is faulty.  Most likely line 4 is faulty.

                  
\subsubsection{\tt IMPOSSIBLE NUMBER OF OPEN SHELL ELECTRONS}
        The keyword OPEN(n1,n2) has been used,  but  for  an  even-electron
   system  n1  was  specified  as  odd or for an odd-electron system n1 was
   specified as even.  Either way, there is a conflict which the user  must
   resolve.

                         
\subsubsection{\tt IMPOSSIBLE OPTION REQUESTED}
        A  general  catch-all.   This  message  will  be  printed  if   two
   incompatible  options  are  used,  such  as  both  MINDO/3 and AM1 being
   specified.  Check the keywords, and resolve the conflict.

\subsubsection{\tt INTERNAL COORDINATES READ IN, AND CALCULATION \ldots}
        If internal coordinates are read in, but the calculation is  to  be
   carried  out  using  cartesian  coordinates,  then  either  all possible
   geometric variables must be optimized, or none  can  be  optimized.   If
   only  some  are  marked  for  optimization,  then ambiguity exists.  For
   example, if the bond-length of atom 6 is marked  for  optimization,  but
   the  angle  is  not,  then  when the conversion to cartesian coordinates
   takes place, the first coordinate becomes the `X'  coordinate  and  the
   second  the  `Y'  coordinate.   These  bear  no relationship to the bond
   length or angle.  This is a fatal error.
                                                              
\subsubsection{\tt INTERNAL COORDINATES READ IN, AND SYMMETRY \ldots}
        If internal coordinates are read in, but the calculation is  to  be
   carried out using cartesian coordinates, then any symmetry relationships
   between the internal coordinates will not be reflected in the  cartesian
   coordinates.   For  example,  if  the  bond-lengths of atoms 5 and 6 are
   equal, it does not follow that these atoms have equal values  for  their
   `X' coordinates.  This is a fatal error.

\subsubsection{\tt JOB STOPPED BY OPERATOR}
Any MOPAC calculation, for which the SHUTDOWN command works, can be
stopped  by  a  user  who issues the command \verb/"$SHUT <filename>/, 
from the directory which contains \verb/<filename>.DAT/.

        MOPAC will then stop the calculation at the first convenient point,
   usually  after  the  current cycle has finished.  A restart file will be
   written and the job ended.  The message will be printed as soon as it is
   detected, which would be the next time the timer routine is accessed.

                    
\subsubsection{\tt **** MAX. NUMBER OF ATOMS ALLOWED: \ldots}
        At compile time the maximum sizes of the arrays in MOPAC are fixed.
   The  system  being  run exceeds the maximum number of atoms allowed.  To
   rectify this, modify the file DIMSIZES.DAT to  increase  the  number  of
   heavy  and  light  atoms  allowed.  If DIMSIZES.DAT is altered, then the
   whole of MOPAC should be re-compiled and re-linked.

                      
\subsubsection{\tt **** MAX. NUMBER OF ORBITALS: \ldots}
        At compile time the maximum sizes of the arrays in MOPAC are fixed.
   The system being run exceeds the maximum number of orbitals allowed.  To
   rectify this, modify the file DIMSIZES.DAT to change the number of heavy
   and  light atoms allowed.  If DIMSIZES.DAT is altered, then the whole of
   MOPAC should be re-compiled and re-linked.

\subsubsection{\tt **** MAX. NUMBER OF TWO ELECTRON INTEGRALS \ldots}
        At compile time the maximum sizes of the arrays in MOPAC are fixed.
   The  system  being  run  exceeds  the  maximum  number  of  two-electron
   integrals allowed.  To rectify this, modify  the  file  DIMSIZES.DAT  to
   modify  the number of heavy and light atoms allowed.  If DIMSIZES.DAT is
   altered, then the whole of MOPAC should be re-compiled and re-linked.

\subsubsection{\tt NAME NOT FOUND}
   Various atomic parameters can  be  modified  in  MOPAC  by  use  of
   EXTERNAL=.  These comprise:
\begin{verbatim}
          Uss         Betas         Gp2          GSD 
          Upp         Betap         Hsp          GPD 
          Udd         Betad         AM1          GDD 
          Zs          Gss           Expc         FN1 
          Zp          Gsp           Gaus         FN2 
          Zd          Gpp           Alp          FN3 
\end{verbatim}
         
Thus to change the Uss of hydrogen to $-13.6$ the line \verb/USS    H    -13.6/
could be used.  If an attempt is made to modify  any  other  parameters,
then an error message is printed, and the calculation terminated.

\subsubsection{\tt NUMBER OF PARTICLES, nn GREATER THAN \ldots}
        When user-defined microstates are not used, the MECI will calculate
   all  possible  microstates  that  satisfy the space and spin constraints
   imposed.  This is done in PERM, which permutes N electrons in M  levels.
   If  N is greater than M, then no possible permutation is valid.  This is
   not a fatal error - the program will continue to run, but  no  C.I. will
   be done.

                  
\subsubsection{\tt NUMBER OF PERMUTATIONS TOO GREAT, LIMIT 60}
        The number of permutations of alpha or beta microstates is  limited
   to 60.  Thus if 3 alpha electrons are permuted among 5 M.O.'s, that will
   generate $10 = 5!/(3!2!)$ alpha microstates, which is an allowed  number.
   However  if 4 alpha electrons are permuted among 8 M.O.'s, then 70 alpha
   microstates result and the arrays defined will  be  insufficient.   Note
   that  60  alpha  and 60 beta microstates will permit 3600 microstates in
   all, which should be  more  than  sufficient  for  most  purposes.   (An
   exception would be for excited radical icosohedral systems.)

\subsubsection{\tt SYMMETRY SPECIFIED, BUT CANNOT BE USED IN DRC}
        This  is  self  explanatory.   The  DRC  requires   all   geometric
   constraints  to  be  lifted.   Any  symmetry  constraints  will first be
   applied, to symmetrize the geometry,  and  then  removed  to  allow  the
   calculation to proceed.

\subsubsection{\tt SYSTEM DOES NOT APPEAR TO BE OPTIMIZABLE}
        This is a gradient norm minimization message.  These routines  will
   only   work   if  the  nearest  minimum  to  the  supplied  geometry  in
   gradient-norm space is a transition state or a ground  state.   Gradient
   norm  space  can  be  visualized  as  the  space  of  the  scalar of the
   derivative of the energy space with respect to  geometry.   To  a  first
   approximation,  there are twice as many minima in gradient norm space as
   there are in energy space.

        It is unlikely that  there  exists  any  simple  way  to  refine  a
   geometry  that  results in this message.  While it is appreciated that a
   large amount of effort has probably already been expended in getting  to
   this  point,  users  should  steel  themselves  to writing off the whole
   geometry.  It is not recommended that a minor  change  be  made  to  the
   geometry and the job re-submitted.

        Try using SIGMA instead of POWSQ.
                                                  
\subsubsection{\tt TEMPERATURE RANGE STARTS TOO LOW, \ldots}
        The  thermodynamics  calculation  assumes  that   the   statistical
   summations  can be replaced by integrals.  This assumption is only valid
   above 100K, so the lower temperature  bound  is  set  to  100,  and  the
   calculation continued.

                    
\subsubsection{\tt THERE IS A RISK OF INFINITE LOOPING \ldots}
        The SCF criterion has been reset by the user, and the new value  is
   so  small  that  the SCF test may never be satisfied.  This is a case of
   user beware!

\subsubsection{\tt THIS MESSAGE SHOULD NEVER APPEAR, CONSULT A PROGRAMMER!}
        This message should never appear; a fault has been introduced  into
   MOPAC,  most  probably  as  a  result  of  a programming error.  If this
   message appears in the vanilla version of MOPAC  (a  version  ending  in
   00),  please  contact JJPS as I would be most interested in how this was
   achieved.

\subsubsection{\tt THREE ATOMS BEING USED TO DEFINE \ldots}
        If the cartesian coordinates of an  atom  depend  on  the  dihedral
   angle  it makes with three other atoms, and those three atoms fall in an
   almost straight line, then a small change in the  cartesian  coordinates
   of  one  of  those three atoms can cause a large change in its position.
   This is a potential source of trouble, and the data should be changed to
   make the geometric specification of the atom in question less ambiguous.

        This message can appear at any time, particularly in reaction  path
   and saddle-point calculations.

        An exception to this rule is  if  the  three  atoms  fall  into  an
   exactly  straight  line.  For example, if, in propyne, the hydrogens are
   defined in terms of the three  carbon  atoms,  then  no  error  will  be
   flagged.  In such a system the three atoms in the straight line must not
   have the angle between  them  optimized,  as  the  finite  step  in  the
   derivative calculation would displace one atom off the straight line and
   the error-trap would take effect.

        Correction involves re-defining the connectivity.  LET  and  GEO-OK
   will not allow the calculation to proceed.

\subsubsection{\tt - - - - - - - TIME UP - - - - - - -}
        The time defined on the keywords line or 3,600 seconds, if no  time
   was  specified, is likely to be exceeded if another cycle of calculation
   were to be performed.  A controlled termination of the run would  follow
   this  message.   The  job  may terminate earlier than expected:  this is
   ordinarily due to one of the recently completed cycles taking  unusually
   long,  and  the  safety  margin  has  been  increased  to  allow for the
   possibility that the next cycle might also  run  for  much  longer  than
   expected.
                                             
\subsubsection{\tt TRIPLET SPECIFIED WITH ODD NUMBER OF ELECTRONS}
        If TRIPLET has been specified the number of electrons must be even.
   Check  the  charge  on  the  system,  the empirical formula, and whether
   TRIPLET was intended.
               
\subsubsection{\tt """"""""""""""UNABLE TO ACHIEVE SELF-CONSISTENCY}
See the error-message: \verb/<<<<----**** FAILED TO ACHIEVE SCF.  ****---->>>>/.

\subsubsection{\tt UNDEFINED SYMMETRY FUNCTION USED}
        Symmetry operations are restricted to those defined, i.e.,  in  the
   range 1--18.  Any other symmetry operations will trip this fatal message.

\subsubsection{\tt UNRECOGNIZED ELEMENT NAME}
        In the geometric specification a chemical  symbol  which  does  not
   correspond  to  any  known element has been used.  The error lies in the
   first datum on a line of geometric data.

\subsubsection{\tt **** WARNING ****}
        Don't pay too  much  attention  to  this  message.   Thermodynamics
   calculations  require  a  higher  precision  than  vibrational frequency
   calculations.  In particular, the gradient norm should  be  very  small.
   However,  it  is  frequently  not  practical to reduce the gradient norm
   further, and to date no-one has determined just how slack  the  gradient
   criterion  can be before unacceptable errors appear in the thermodynamic
   quantities.  The 0.4 gradient norm is only a suggestion.

\subsubsection{\tt WARNING: INTERNAL COORDINATES \ldots}
        Triatomics  are,  by  definition,  defined  in  terms  of  internal
   coordinates.  This warning is only a reminder.  For diatomics, cartesian
   and internal coordinates are the same.  For  tetra-atomics  and  higher,
   the  presence  or absence of a connectivity table distinguishes internal
   and cartesian coordinates, but for triatomics there is an ambiguity.  To
   resolve  this,  cartesian coordinates are not allowed for the data input
   for triatomics.
                                  
%%%%%%%%%%%%%%%%%%%%%%%%%%%%%%%%%%%%%%%%%%%%%%%%%%%%%%%%%%%%%%%%%%%%%

\chapter{Criteria}
\index{MOPAC!criteria}
        MOPAC uses various criteria which  control  the  precision  of  its
   stages.   These criteria are chosen as the best compromise between speed
   and acceptable errors in the results.  The user can override the default
   settings  by  use  of  keywords;  however,  care  should be exercised as
   increasing a criterion can introduce the potential for  infinite  loops,
   and decreasing a criterion can result in unacceptably imprecise results.
   These are usually characterized by `noise' in a reaction path, or  large
   values for the trivial vibrations in a force calculation.



\section{SCF criterion}
\index{SCFCRT}
\begin{verbatim}
 Name:          SCFCRT. 
 Defined in     ITER. 
 Default value  0.0001 kcal/mole
 Basic Test     Change in energy in kcal/mole on successive
                iterations is less than SCFCRT.

 Exceptions:    If PRECISE is specified,        SCFCRT=0.000001
                If a polarization calculation   SCFCRT=1.D-11
                If a FORCE calculation          SCFCRT=0.0000001
                If SCFCRT=n.nnn is specified    SCFCRT=n.nnn
                If a BFGS optimization, SCFCRT becomes a function
                of the difference between the current energy and
                the lowest energy of previous SCFs.
 Secondary tests: (1) Change in density matrix elements on two 
                      successive iterations must be less than 0.001
                  (2) Change in energy in eV on three successive 
                      iterations must be less than 10 x SCFCRT.
\end{verbatim}

\section{Geometric optimization criteria}
\index{TOLERX}
\index{DELHOF}
\index{TOLERG}
\index{TOLERX}
\index{TOL2}
\index{TOLS1}
\index{heat of formation!criteria}
\begin{verbatim}
 Name:           TOLERX   "Test on X Satisfied"
 Defined in      FLEPO
 Default value   0.0001 Angstroms
 Basic Test      The projected change in geometry is less than 
                 TOLERX Angstroms.

 Exceptions      If GNORM is specified, the TOLERX test is not used.

 Name:           DELHOF    "Herbert's Test Satisfied"
 Defined in      FLEPO
 Default value   0.001
 Basic Test      The projected decrease in energy is less than
                 DELHOF kcals/mole.

 Exceptions      If GNORM is specified, the DELHOF test is not used.

 Name:           TOLERG    "Test on Gradient Satisfied"
 Defined in      FLEPO
 Default value   1.0
 Basic Test      The gradient norm in kcals/mole/Angstrom is less 
                 than TOLERG multiplied by the square root of the
                 number of coordinates to be optimized.

 Exceptions      If GNORM=n.nnn is specified, TOLERG=n.nnn divided 
                 by the square root of the number of coordinates 
                 to be optimized, and the secondary tests are not
                 done.  If LET is not specified, n.nnn is reset to
                 0.01, if it was smaller than 0.01.
                 If PRECISE is specified, TOLERG=0.2

                 If a SADDLE calculation, TOLERG is made a function
                 of the last gradient norm.
 Name:           TOLERF    "Heat of Formation Test Satisfied"
 Defined in      FLEPO
 Default value   0.002 kcal/mole
 Basic Test      The calculated heats of formation on two successive
                 cycles differ by less than TOLERF.

 Exceptions      If GNORM is specified, the TOLERF test is not used.

 Secondary Tests For the TOLERG, TOLERF, and TOLERX tests, a 
                 second test in which no individual component of the 
                 gradient should be larger than TOLERG must be
                 satisfied.

 Other Tests     If, after the TOLERG, TOLERF, or TOLERX test has been
                 satisfied three consecutive times the heat of
                 formation has dropped by less than 0.3kcal/mole, then
                 the optimization is stopped.

 Exceptions      If GNORM is specified, then this test is not performed.

 Name:           TOL2
 Defined in      POWSQ
 Default value   0.4
 Basic Test      The absolute value of the largest component of the 
                 gradient is less than TOL2

 Exceptions      If PRECISE is specified, TOL2=0.01
                 If GNORM=n.nn is specified, TOL2=n.nn
                 If LET is not specified, TOL2 is reset to
                 0.01, if n.nn was smaller than 0.01.

 Name:           TOLS1
 Defined in      NLLSQ
 Default Value   0.000 000 000 001
 Basic Test      The square of the ratio of the projected change in the
                 geometry to the actual geometry is less than TOLS1.

 Name:           <none>
 Defined in      NLLSQ
 Default Value   0.2
 Basic Test      Every component of the gradient is less than 0.2.
\end{verbatim}

%%%%%%%%%%%%%%%%%%%%%%%%%%%%%%%%%%%%%%%%%%%%%%%%%%%%%%%%%%%%%%%%%%%%%

\chapter{Debugging}
        There are three potential sources of  difficulty  in  using  MOPAC,
   each  of  which  requires special attention.  There can be problems with
   data, due to errors in the data, or MOPAC  may  be  called  upon  to  do
   calculations  for which it was not designed.  There are intrinsic errors
   in MOPAC which extensive testing has  not  yet  revealed,  but  which  a
   user's novel calculation uncovers.  Finally there can be bugs introduced
   by the user modifying MOPAC, either to make it compatible with the  host
   computer, or to implement local features.

        For whatever reason, the user may  need  to  have  access  to  more
   information  than  the  normal  keywords  can provide, and a second set,
   specifically  for  debugging,  is   provided.    These   keywords   give
   information  about  the  working  of  individual subroutines, and do not
   affect the course of the calculation.

\section{Debugging keywords}
\index{keywords!debugging}
A full list of keywords for debugging subroutines:
\begin{verbatim}
1ELEC          the one-electron matrix.                          Note 1
COMPFG         Heat of Formation.
DCART          Cartesian derivatives.
DEBUG                                                            Note 2
DEBUGPULAY     Pulay matrix, vector, and error-function.         Note 3
DENSITY        Every density matrix.                             Note 1
DERI1          Details of DERI1 calculation
DERI2          Details of DERI2 calculation
DERITR         Details of DERITR calculation
DERIV          All gradients, and other data in DERIV.
DERNVO         Details of DERNVO calculation
DFORCE         Print Force Matrix. 
DIIS           Details of DIIS calculation
EIGS           All eigenvalues.
FLEPO          Details of BFGS minimization.
FMAT 
FOCK           Every Fock matrix                                 Note 1
HCORE          The one electron matrix, and two electron integrals.
ITER           Values of variables and constants in ITER.
LARGE          Increases amount of output generated by other keywords.
LINMIN         Details of line minimization (LINMIN, LOCMIN, SEARCH)
MOLDAT         Molecular data, number of orbitals, "U" values, etc.
MECI           C.I. matrices, M.O. indices, etc.
PL             Differences between density matrix elements       Note 4
               in ITER. 
LINMIN         Function values, step sizes at all points in the
               line minimization (LINMIN or SEARCH).
TIMES          Times of stages within ITER.
VECTORS        All eigenvectors on every iteration.              Note 1
\end{verbatim}

\subsection*{Notes}
\begin{enumerate}
\item These keywords are activated by the  keyword  DEBUG.   Thus  if
     DEBUG  and  FOCK are both specified, every Fock matrix on every
     iteration will be printed.

\item DEBUG is not intended to increase the output,  but  does  allow
     other keywords to have a special meaning.

\item PULAY is already  a  keyword,  so  DEBUGPULAY  was  an  obvious
     alternative.

\item PL initiates the output of the value of the largest  difference
     between  any  two  density  matrix  elements on two consecutive
     iterations.  This is very useful when investigating options for
     increasing the rate of convergence of the SCF calculation.
\end{enumerate}

                    
\subsection*{Suggested procedure for locating bugs}
\index{bugs!locating}
        Users are supplied with the source code for MOPAC, and,  while  the
   original  code is fairly bug-free, after it has been modified there is a
   possibility that bugs may have been introduced.  In these  circumstances
   the  author  of  the  changes  is obviously responsible for removing the
   offending bug, and the  following  ideas  might  prove  useful  in  this
   context.

        First of all, and most important, before any modifications are done
   a  back-up  copy  of the standard MOPAC should be made.  This will prove
   invaluable in pinpointing deviations from the  standard  working.   This
   point  cannot  be  over-emphasized --- {\em make  a  back-up before modifying
   MOPAC!}.

        Clearly, a bug can occur almost  anywhere,  and  a  logical  search
   sequence is necessary in order to minimize the time taken to locate it.

        If possible, perform the debugging with a small molecule, in  order
   to  save  time  (debugging  is,  of  necessity,  time  consuming) and to
   minimize output.

        The two sets of subroutines  in  MOPAC,  those  involved  with  the
   electronics  and  those  involved  in  the geometrics, are kept strictly
   separate, so the first question to be answered is which set contains the
   bug.   If the heats of formation, derivatives, I.P.s, and charges, etc.,
   are correct,  the  bug  lies  in  the  geometrics;  if  faulty,  in  the
   electronics.

        
\subsubsection{Bug in the Electronics Subroutines}
   Use formaldehyde for this test.  The supplied data-file  MNRSD1.DAT
   could  be  used  as  a  template for this operation.  Use keywords 1SCF,
   DEBUG, and any others necessary.

   The main steps are:
\begin{enumerate}
   \item Check  the  starting  one-electron  matrix  and   two-electron
   integral  string, using the keyword HCORE.  It is normally sufficient to
   verify that the two hydrogen atoms  are  equivalent,  and  that  the  pi
   system  involves  only  pz  on  oxygen  and carbon.  Note that numerical
   values are not checked, but only relative values.

   If an error is found, use MOLDAT to verify the  orbital  character,
   etc.

   If faulty the error lies in READ, GETGEO or MOLDAT.

   Otherwise the error lies in HCORE, H1ELEC or ROTATE.

   If the starting matrices are correct, go on to step (2).

\item Check the density or Fock matrix on every iteration,  with  the
 words FOCK or DENSITY.  Check the equivalence of the two hydrogen atoms,
 and the pi system, as in (1).

 If an error is found, check the first Fock matrix.  If faulty,  the
 bug  lies  in ITER, probably in the Fock subroutines FOCK1 or FOCK2.  or
 in the (guessed) density matrix (MOLDAT).  An exception is  in  the  UHF
 closed-shell  calculation,  where  a  small  asymmetry  is introduced to
 initiate the separation of the alpha and beta UHF wavefunctions.

 If no error is found, check the second Fock matrix.  If faulty, the
 error lies in the density matrix DENSIT, or the diagonalization RSP.

 If the Fock matrix is acceptable, check all the Fock matrices.   If
 the  error starts in iterations 2 to 4, the error probably lies in CNVG,
 if after that, in PULAY, if used.

 If SCF is achieved, and the heat  of  formation  is  faulty,  check
 HELECT.  If C.I. was used check MECI.

 If the derivatives are faulty, use DCART to  verify  the  cartesian
 derivatives.   If  these  are  faulty, check DCART and DHC.  If they are
 correct,  or  not  calculated,  check  the   DERIV   finite   difference
 calculation.   If the wavefunction is non-variationally optimized, check
 DERNVO.

      If the geometric calculation is faulty, use FLEPO  to  monitor  the
 optimization, DERIV may also be useful here.

      For  the  FORCE  calculation,  DCART  or  DERIV  are   useful   for
 variationally   optimized   functions,   COMPFG   for  non-variationally
 optimized functions.

      For reaction paths, verify that FLEPO is working correctly; if  so,
 then PATHS is faulty.

      For  saddle-point  calculations,  verify  that  FLEPO  is   working
 correctly; if so, then REACT1 is faulty.
\end{enumerate}

 Keep in mind the fact that MOPAC is a large calculation, and  while
 intended  to  be  versatile,  many combinations of options have not been
 tested.  If a bug is found in  the  original  code,  please  communicate
 details  to  the  Academy,  to  Dr.\ James J. P. Stewart, Frank J. Seiler
 Research Laboratory, U.S.   Air  Force  Academy,  Colorado  Springs,  CO
 80840--6528.

%%%%%%%%%%%%%%%%%%%%%%%%%%%%%%%%%%%%%%%%%%%%%%%%%%%%%%%%%%%%%%%%%%%%%

\chapter{Installing MOPAC}
\index{MOPAC!installing}
   MOPAC is distributed on a magnetic tape  as  a  set  of  FORTRAN-77
   files,  along  with  ancillary documents such as command, help, data and
   results files.  The  format  of  the  tape  is  that  of  DIGITAL'S  VAX
   computers.   The  following  instructions  apply  only to users with VAX
   computers:   users  with  other  machines  should  use   the   following
   instructions as a guide to getting MOPAC up and running.
\begin{enumerate}
\item Put the magnetic tape on the tape drive, write protected.
\item Allocate the tape drive with a command such as \verb/$ALLOCATE MTA0:/
\item Go into an empty directory which is to hold MOPAC
\item Mount the magnetic tape with the command \verb/$MOUNT MTA0:  MOPAC/
\item Copy all the files from the tape with the command \verb/$COPY MTA0:*.* */
\end{enumerate}

A useful operation after this would be to make a hard copy  of  the
directory.   You  should  now  have  the  following sets of files in the
directory:
\begin{enumerate}
\item A file, AAAINVOICE.TXT, summarizing this list.
\item A set of FORTRAN--77 files, see Appendix A.
\item The command files COMPILE, MOPACCOM, MOPAC, RMOPAC, and SHUT.
\item A file, MOPAC.OPT, which lists all the object modules used by MOPAC.
\item Help files MOPAC.HLP and HELP.FOR
\item A text file MOPAC.MAN.
\item A manual summarizing the updates, called UPDATE.MAN.
\item Two  test-data  files:   TESTDATA.DAT   and   MNRSD1.DAT,   and
            corresponding results files, TESTDATA.OUT and MNRSD1.OUT.
\end{enumerate}

                          
\section*{Structure of command files: COMPILE}
\index{command file!COMPILE}
The parameter file DIMSIZES.DAT should be read and,  if  necessary,
modified before COMPILE is run.
\begin{verbatim}
DO NOT RUN COMPILE AT THIS TIME!!
\end{verbatim}

        COMPILE should be run once  only.   It  assigns  DIMSIZES.DAT,  the
   block  of  FORTRAN which contains the PARAMETERS for the dimension sizes
   to the logical name ``SIZES''.  This is a temporary  assignment,  but  the
   user  is strongly recommended to make it permanent by suitably modifying
   LOGIN file(s).  COMPILE is a  modified  version  of  Maj  Donn  Storch's
   COMPILE for DRAW-2.

        All the FORTRAN files are then  compiled,  using  the  array  sizes
   given  in DIMSIZES.DAT:  these should be modified before COMPILE is run.
   If, for whatever reason, DIMSIZES.DAT needs to be changed, then  COMPILE
   should  be  re-run, as modules compiled with different DIMSIZES.DAT will
   be incompatible.

        The parameters within DIMSIZES.DAT that the  user  can  modify  are
   MAXLIT,  MAXHEV, MAXTIM and MAXDMP.  MAXLIT is assigned a value equal to
   the largest number of hydrogen atoms that a MOPAC  job  is  expected  to
   run, MAXHEV is assigned the corresponding number of heavy (non-hydrogen)
   atoms.  The ratio of light to heavy atoms should not be less  than  1/2.
   Do  not  set  MAXHEV or MAXLIT less than 7.  If you do, some subroutines
   will not compile correctly.  Some molecular orbital  eigenvector  arrays
   are  overlapped  with  Hessian  arrays,  and to prevent compilation time
   error messages, the number of allowed A.O.'s must be  greater  than,  or
   equal  to  three  times the number of allowed real atoms.  MAXTIM is the
   default maximum time in seconds a job is allowed to  run  before  either
   completion  or a restart file being written.  MAXDMP is the default time
   in seconds for the automatic writing of  the  restart  files.   If  your
   computer  is  very  reliable,  and disk space is at a premium, you might
   want to set MAXDMP as MAXDMP=999999.
 
      If SYBYL output is wanted, set ISYBYL to 1,  otherwise  set  it  to
 zero.
 
      If you want, NMECI can be changed.  Setting it to 1 will save  some
 space, but will prevent all C.I. calculations except simple radicals.
 
      If you want, NPULAY can be set to 1.  This saves memory,  but  also
 disables the PULAY converger.
 
      If you want, MESP can be varied.  This is only meaningful if ESP is
 installed.

      Compile MOPAC.  This operation takes about 7 minutes, and should be
 run ``on-line'', as a question and answer session is involved.

        When everything is successfully compiled,  the  object  files  will
   then  be  assembled into an executable image called MOPAC.EXE.  Once the
   image exists, there is no reason to keep the object files, and if  space
   is at a premium these can be deleted at this time.

        If you need to make any  changes  to  any  of  the  files,  COMPILE
   followed  by  the  names  of  the  changed files will reconstruct MOPAC,
   provided all the other OBJ files exist.  For example, if you change  the
   version number in DIMSIZES.DAT, then READ.FOR and WRITE.FOR are affected
   and will need to be recompiled.  This can be done using the command
   \verb/@COMPILE WRITE,READ/

   In the unlikely event that you want to link only, use the command
   \verb/@COMPILE LINK/
         
Sometimes the link stage will fail, and give the message
\begin{verbatim}  
 "%LINK-E-INSVIRMEM, insufficient virtual memory for 2614711. pages
  -LINK-E-NOIMGFIL, image file not created",
\end{verbatim}
  
 or your MOPAC will not run due to the size of the image.  In these cases
 you  should  ask  the  system manager to alter your PGFLQUO and WSEXTENT
 limits.  Possibly the system limits, VIRTUALPAGECNT CURRENT and MAX will
 need  to  be  changed.   As  an example, on a Microvax 3600 with 16Mb of
 memory:
\begin{verbatim}
 PGFLQUO=50000,  WSEXTENT=16000,  VIRTUALPAGECNT  CURRENT=40768,
 VIRTUALPAGECNT MAX=600000
\end{verbatim}
 are sufficient for the default MOPAC values of 43 heavy and 43 light atoms.
  
In order for users to have access to  MOPAC  they  must  insert  in
their individual LOGIN.COM files the line:
\begin{verbatim}
$@ <Mopac-directory>MOPACCOM
\end{verbatim}
where \verb/<Mopac-directory>/ is the name of the disk and directory which
holds all the MOPAC files.  For example:
\begin{verbatim}
DRA0:[MOPAC]
\end{verbatim}
thus: \verb/$@ DRA0:[MOPAC]MOPAC/

MOPACCOM.COM  should  be  modified  once   to   accommodate   local
definitions  of  the directory which is to hold MOPAC.  This change must
also be made to RMOPAC.COM and to MOPAC.COM.

                                    
\section*{MOPAC}
\index{command file!MOPAC}
        This command file submits a MOPAC job  to  a  queue.   Before  use,
   MOPAC.COM  should  be modified to suit local conditions.  The user's VAX
   is assumed to run three queues, called QUEUE3, QUEUE2, and QUEUE1.   The
   user  should  substitute  the  actual  names of the VAX queues for these
   symbolic names.  Thus, for example, if the local names of the queues are
   ``TWELVEHOUR'',  for jobs of length up to 12 hours, ``ONEHOUR'', for jobs of
   less than one hour, and ``30MINS''  for  quick  jobs,  then  in  place  of
   ``QUEUE3'',  ``QUEUE2'', and ``QUEUE1'' the words 
``TWELVEHOUR'', ``ONEHOUR'', and
``30MINS'' should be inserted.


                                    
\section*{RMOPAC}
\index{command file!RMOPAC}
        RMOPAC is the command file for running MOPAC.  It assigns  all  the
   data  files  that  MOPAC uses to the channels.  If the user wants to use
   other file-name endings than those supplied, the modifications should be
   made to RMOPAC.

        When a long job ends, RMOPAC will also send a mail message  to  the
   user  giving a brief description of the job.  You may want to change the
   default definition of ``a long job''; currently  it  is  12  hours.   This
   feature was written by Dr.\ James Petts of Kodak Ltd Research Labs.
\index{petts@{\bf Petts, Dr. J.}}

        A recommended sequence of operations to get MOPAC  up  and  running
   would be:
\begin{enumerate}
\item Modify the file DIMSIZES.DAT.  The default sizes are  40  heavy
atoms  and 40 light atoms.  Do not make the size less than 7 by 7.
\item Read through the COMMAND files  to  familiarize  yourself  with
            what is being done.
\item Edit the file MOPAC.COM to use the local queue names.

\item Edit the file RMOPAC.COM if  the  default  file-names  are  not
            acceptable.

\item Edit MOPACCOM.COM to assign  MOPACDIRECTORY  to  the  disk  and
            directory which will hold MOPAC.

\item Edit the individual LOGIN.COM files  to  insert  the  following
      line:
\begin{verbatim}
$@ <Mopac-directory>MOPACCOM
\end{verbatim}

Note that MOPACDIRECTORY cannot be used, as the definition
of MOPACDIRECTORY is made in MOPACCOM.COM

\item Execute the modified LOGIN command so that the new commands are
            effective.

\item Run COMPILE.COM.  This takes about 8 minutes to execute.

\item Enter the command \verb/$MOPAC /
                  
You will receive the message \verb/What file? :/
to which the reply should be the actual data-file name. For
example, ``MNRSD1'', the file is assumed to end in .DAT, e.g. MNRSD1.DAT.
You will then be prompted for the queue:
\begin{verbatim}
What queue? :
\end{verbatim}
Any queue defined in MOPAC.COM will suffice: \verb/"SYS$BATCH"/
\end{enumerate}

Finally, the priority will be requested:
\verb/What priority? [5]:/
To which any value between 1 and 5 will suffice. Note that the
maximum priority is limited by the system (manager).

\section{ESP calculation}
\index{ESP!installing}
 As supplied, MOPAC will not do the ESP calculation because  of  the
 large  memory  requirement  of  the  ESP.   To install the ESP, make the
 following changes:
\begin{enumerate} 
\item Rename ESP.ROF to ESP.FOR
\item Add to the first line of MOPAC.OPT the string ``\verb/ ESP, /'' (without
          the quotation marks).
\item Edit MNDO.FOR to uncomment the line \verb/C# CALL ESP/.
\item Compile ESP and MNDO, and relink MOPAC using,  e.g.
\verb/@COMPILE ESP,MNDO/.
\item If the resulting executable is too large,  modify  DIMSIZES.DAT
          to  reduce  MAXHEV  and  MAXLIT,  then recompile everything and
          relink MOPAC with \verb/@COMPILE/.
\end{enumerate} 



To familiarize yourself with the system, the  following  operations
might be useful.
\begin{enumerate}
\item Run the (supplied) test molecules, and  verify  that  MOPAC  is
            producing ``acceptable'' results.

\item Make some simple modifications to  the  datafiles  supplied  in
            order to test your understanding of the data format

\item  When satisfied that MOPAC is working, and that data  files  can
            be made, begin production runs.
\end{enumerate}

\subsection*{Working of SHUTDOWN command}
\index{SHUTDOWN}
        If, for whatever reason, a run needs to be stopped prematurely, the
   command \verb/$SHUT  <jobname>/ can be issued. This  will execute a small
   command-language file, which copies the data-file to  form  a  new  file
   called \verb/<filename>.END/.

        The next time MOPAC  calls  function  SECOND,  the  presence  of  a
   readable file called SHUTDOWN, logically identified with 
   \verb/<filename>.END/,
   is checked for, and if it exists,  the  apparent  elapsed  CPU  time  is
   increased  by  1,000,000  seconds,  and  a  warning  message issued.  No
   further action is taken until the elapsed time  is  checked  to  see  if
   enough  time remains to do another cycle.  Since an apparently very long
   time has been used, there is not enough time left to do  another  cycle,
   and the restart files are generated and the run stopped.

   SHUTDOWN is completely machine--independent.

        Specific instructions for mounting MOPAC on  other  computers  have
 been  left  out  due to limitations of space in the Manual; however, the
 following points may prove useful:
\begin{enumerate} 
\item Function    SECOND    is    machine-specific.     SECOND     is
          double-precision,  and  should  return the CPU time in seconds,
          from an arbitary zero of time.  If the SHUT  command  has  been
          issued,  the  value  returned  by SECOND should be increased by
          1,000,000.
 
\item On UNIX-based and other machines, on-line help can be  provided
      by using \verb/help.f/.  Documentation on \verb/help.f/ is in 
      \verb/help.f/.\index{UNIX!on-line help}
 
\item OPEN and CLOSE statements are a fruitful  source  of  problems.
          If  MOPAC  does not work, most likely the trouble lies in these
          statements.
 
\item RMOPAC.COM should be read to see what  files  are  attached  to
          what logical channel.
\end{enumerate} 
 
\subsection*{How to use MOPAC}
\index{MOPAC!using}
        The COM file to run the MOPAC can be  accessed  using  the  command
   ``MOPAC'' followed by none, one, two or three arguments.  Possible options
   are:
\begin{verbatim}
   MOPAC   MYDATAFILE 120  4
   MOPAC   MYDATAFILE 120 
   MOPAC   MYDATAFILE 
\end{verbatim}

   In the latter case it  is  assumed  that  the  shortest  queue  will  be
   adequate.   The  COM  file  to  run  the MOPAC can be accessed using the
   command "MOPAC" followed  by  none,  one  or  two  arguments.   Possible
   options are:
\begin{verbatim}
   MOPAC   MYDATAFILE 120 
   MOPAC   MYDATAFILE 
\end{verbatim}
   In the latter case it is assumed that the default time (15 seconds) will
   be adequate.

\subsection*{MOPAC}
        In this case you will be prompted for the datafile,  and  then  for
   the queue.  Restarts should be user transparent.  If MOPAC does make any
   restart files, do not change them (It would be hard  to  do  anyhow,  as
   they're  in  machine  code), as they will be used when you run a RESTART
   job.  The files used by MOPAC are:
\begin{verbatim}
     File         Description                     Logical name

<filename>.DAT    Data                                FOR005
<filename>.OUT    Results                             FOR006
<filename>.RES    Restart                             FOR009
<filename>.DEN    Density matrix (in binary)          FOR010
SYS$OUTPUT        LOG file                            FOR011
<filename>.ARC    Archive or summary                  FOR012
<filename>.GPT    Data for program DENSITY            FOR013
<filename>.SYB    SYBYL data                          FOR016
SETUP.DAT         SETUP data                          SETUP
\end{verbatim}

                                
\subsection*{Short version}
        For various reasons it might  not  be  practical  to  assemble  the
   entire  MOPAC  program.   For  example,  your  computer  may have memory
   limitations, or you may have very large  systems  to  be  run,  or  some
   options  may  never be wanted.  For whatever reason, if using the entire
   program is undesirable, an abbreviated version,  which  lacks  the  full
   range of options of the whole program, can be specified at compile time.

        At the bottom of the DIMSIZES.DAT file the programmer is asked  for
   various  options to be used in compiling.  These options allow arrays of
   MECI, PULAY, and ESP to assume their correct size.

        As long as no attempt is made to use the reduced  subroutines,  the
   program  will function normally.  If an attempt is made to use an option
   which has been excluded then the program will error.
                                
\subsection*{Size of MOPAC}
      The amount of storage required  by  MOPAC  depends  mainly  on  the
 number  of  heavy  and  light atoms.  As it is useful for programmers to
 have an idea of how large various MOPACs are,  the  following  data  are
 presented as a guide.


  Sizes of various MOPAC Version 6.00 executables in which the number 
 of heavy atoms is equal to the number of light atoms, assembled on 
 a VAX computer, are:
\begin{verbatim}  
  No. of heavy atoms     Size of Executable (Kbytes)
                        MOPAC 5.00   MOPAC 6.00     (AMPAC 2.00)
      10                    1,653      2,054           N/A
      20                    3,442      4,689          4,590
      30                    6,356      8,990          9,150
      40                   10,400     14,955         15,588
      50                   15,572     22,586         23,944
      60                   21,872     31,880         34,145
     100                   58,361     87,519
     200                  228,602    336.867
     300                  511,723    754,540
\end{verbatim}
  
The size, $S$, of any given MOPAC executable, in Kbytes, may be estimated for
MOPAC~5.00 as:\index{MOPAC!size of}
$$S = 9939 + N* 9.57 + N*N*5.64$$
and for MOPAC~6.00 as:
$$S = 1091 + N*13.40 + N*N*8.33$$
 
  The large increase in size of MOPAC was caused mainly by the inclusion 
 of the analytical C.I. derivatives.  Because they are so much more
 efficient and accurate than finite differences, and because computer
 memory is becoming more available, this increase was accepted as the
 lesser of two evils.

   The size of MOPAC executables will vary from machine to machine,
 due to the different sizes of the code.  For a VAX, this amounts
 to approximately 0.1Mb.  Most machines use a 64 bit or 8 byte 
 double precision real number, so the multipliers of N and N*N
 should apply to them.  For large jobs, 0.1Mb is negligible, therefore 
 the above expression should be applicable to most computers.

  No. of lines in program in 
     Version~5.00 = 22,084 = 17,718 code + 4,366 comment.
     Version 6.00 = 31,857 = 22,526 code + 9,331 comment.

%%%%%%%%%%%%%%%%%%%%%%%%%%%%%%%%%%%%%%%%%%%%%%%%%%%%%%%%%%%%%%%%%%%%%

\appendix
\chapter{Names of FORTRAN-77 files}
\index{subroutines!full list of}
\begin{verbatim}
 AABABC   ANALYT   ANAVIB   AXIS     BLOCK    BONDS    BRLZON   
 CALPAR   CAPCOR   CDIAG    CHRGE    CNVG     COMPFG   DATIN    
 DCART    DELMOL   DELRI    DENROT   DENSIT   DEPVAR   DERI0    
 DERI1    DERI2    DERI21   DERI22   DERI23   DERITR   DERIV    
 DERNVO   DERS     DFOCK2   DFPSAV   DIAG     DIAT     DIAT2    
 DIIS     DIJKL1   DIJKL2   DIPIND   DIPOLE   DOFS     DOT      
 DRC      DRCOUT   EF       ENPART   EXCHNG   FFHPOL   FLEPO    
 FMAT     FOCK1    FOCK2    FORCE    FORMXY   FORSAV   FRAME    
 FREQCY   GEOUT    GEOUTG   GETGEG   GETGEO   GETSYM   GETTXT   
 GMETRY   GOVER    GRID     H1ELEC   HADDON   HCORE    HELECT   
 HQRII    IJKL     INTERP   ITER     JCARIN   LINMIN   LOCAL    
 LOCMIN   MAMULT   MATOUT   MATPAK   MECI     MECID    MECIH    
 MECIP    MNDO     MOLDAT   MOLVAL   MULLIK   MULT     NLLSQ    
 NUCHAR   PARSAV   PARTXY   PATHK    PATHS    PERM     POLAR    
 POWSAV   POWSQ    PRTDRC   QUADR    REACT1   READ     READA    
 REFER    REPP     ROTAT    ROTATE   RSP      SEARCH   SECOND   
 SETUPG   SOLROT   SWAP     SYMTRY   THERMO   TIMER    UPDATE   
 VECPRT   WRITE    WRTKEY   WRTTXT   XYZINT
\end{verbatim}
              

\chapter{Subroutine calls in MOPAC}
\index{subroutines!calls in MOPAC}
A list of the program segments which call various subroutines.
\begin{verbatim}
SUBROUTINE            CALLS

AABABC
AABACD         
AABBCD         
AINTGS          
ANALYT       DERS   DELRI  DELMOL
ANAVIB       
AXIS         RSP
BABBBC       
BABBCD       
BANGLE       
BFN         
BINTGS          
BKRSAV       GEOUT
BONDS        VECPRT MPCBDS
BRLZON       CDIAG  DOFS
CALPAR         
CAPCOR         
CDIAG        ME08A  EC08C   SORT
CHRGE         
CNVG         
COE         
COMPFG       SETUPG SYMTRY  GMETRY  TIMER  HCORE  ITER  
             DIHED  DERIV   MECIP
DANG         
DATIN        UPDATE  MOLDAT  CALPAR
DCART        ANALYT  DHC     DIHED  
DELMOL       ROTAT
DELRI         
DENROT       GMETRY  COE
DENSIT         
DEPVAR          
DERI0          
DERI1        TIMER   DHCORE  SCOPY  DFOCK2 SUPDOT MTXM   MXM   
             DIJKL1  MECID   MECIH  SUPDOT TIMER
DERI2        DERI21  DERI22  MXM    OSINV  MTXM   SCOPY  DERI23   
             DIJKL2  MECID   MECIH  SUPDOT 
DERI21       MTXMC   HQRII   MXM 
DERI22       MXM     MXMT    FOCK2  FOCK1  SUPDOT 
DERI23       SCOPY
DERITR       SYMTRY  GMETRY  HCORE  ITER   DERIV  DERNVO  DCART  
             JCARIN  MXM     GEOUT  DERITR
DERNVO       DERI0   DERI1   DERI2 
DERS         
DFOCK2       JAB     KAB
DFPSAV       XYZINT  GEOUT  
DHC          H1ELEC  ROTATE  SOLROT  FOCK2  
DHCORE       H1ELEC  ROTATE 
DIAG         EPSETA
DIAGI         
DIAT         COE     GOVER   DIAT2
DIAT2        SET   
DIHED        DANG
DIIS         SPACE   VECPRT  MINV
DIJKL1       FORMXY
DIJKL2          
DIPIND       CHRGE   GMETRY 
DIPOLE          
DOFS         
DRC          GMETRY  COMPFG  PRTDRC
DRCOUT         
EA08C        EA09C
EA09C         
EC08C        EA08C
EF           BKRSAV  COMPFG  BKRSAV  UPDHES  HQRII  FORMD  SYMTRY  
ENPART         
EPSETA         
EXCHNG          
FFHPOL       COMPFG  DIPIND   VECPRT  RSP    MATOUT   
FLEPO        DFPSAV  COMPFG   SCOPY   GEOUT  SUPDOT  LINMIN  DIIS  
FMAT         FORSAV  COMPFG   CHRGE  
FOCK2        JAB     KAB
FOCK2D         
FORCE        GMETRY  COMPFG   NLLSQ   FLEPO  WRITE   XYZINT  AXIS  
             FMAT    VECPRT   FRAME   RSP    MATOUT  FREQCY  MATOUT  
             DRC     ANAVIB   THERMO
FORMD        OVERLP
FORMXY
FORSAV         
FRAME        AXIS
FREQCY       BRLZON  FRAME    RSP
GEOUT        XYZINT  WRTTXT   CHRGE
GEOUTG       XXX
GETDAT         
GETGEG       GETVAL  GETVAL   GETVAL
GETGEO       GEOUT   NUCHAR   XYZINT
GETSYM         
GETTXT       UPCASE  
GMETRY       GEOUT
GOVER         
GRID         DFPSAV  FLEPO    GEOUT  WRTTXT
H1ELEC       DIAT    
HADDON       DEPVAR 
HCORE        H1ELEC  ROTATE   SOLROT VECPRT
HELECT         
HQRII         
IJKL         PARTXY 
INTERP       HQRII   SCHMIT   SCHMIB  SPLINE
ITER         EPSETA  VECPRT   FOCK2   FOCK1  WRITE  INTERP  PULAY  
             HQRII   DIAG     MATOUT  SWAP   DENSIT CNVG  
JAB         
JCARIN       SYMTRY  GMETRY   
KAB         
LINMIN       COMPFG  EXCHNG   
LOCAL        MATOUT
LOCMIN       COMPFG  EXCHNG   
MNDO         GETDAT  READ  MOLDAT  DATIN   REACT1  GRID    PATHS    
             PATHK   FORCE DRC     NLLSQ   COMPFG  POWSQ   EF   
             FLEPO   WRITE POLAR
MAMULT         
MATOUT          
ME08A        ME08B 
ME08B          
MECI         IJKL    PERM  MECIH   VECPRT  HQRII   MATOUT  
MECIH         
MECIP        MXM 
MINV         
MOLDAT       REFER  GMETRY VECPRT
MOLVAL         
MPCBDS         
MPCPOP         
MPCSYB         
MTXM          
MTXMC        MXM 
MULLIK       RSP    GMETRY  MULT  DENSIT  VECPRT
MULT         
MXM         
MXMT          
NLLSQ        PARSAV COMPFG  GEOUT  LOCMIN  PARSAV  
NUCHAR         
OSINV          
OVERLP         
PARSAV       XYZINT  GEOUT
PARTXY       FORMXY
PATHK        DFPSAV  FLEPO  GEOUT  WRTTXT
PATHS        DFPSAV  FLEPO  WRITE  
PERM         
POLAR        GMETRY  AXIS   COMPFG  FFHPOL 
POWSAV       XYZINT  GEOUT
POWSQ        POWSAV  COMPFG  VECPRT  RSP    SEARCH  
PRTDRC       CHRGE   XYZINT  QUADR   
PULAY        MAMULT  OSINV
QUADR         
REACT1       GETGEO  SYMTRY  GEOUT  GMETRY  FLEPO  COMPFG   WRITE  
READ         GETTXT  GETGEG  GETGEO DATE    GEOUT  WRTKEY  GETSYM  
             SYMTRY  NUCHAR  WRTTXT GMETRY  
REFER         
REPP         
ROTAT         
ROTATE        REPP
RSP           EPSETA  TRED3  TQLRAT  TQL2  TRBAK3
SAXPY         
SCHMIB         
SCHMIT         
SCOPY          
SEARCH        COMPFG 
SECOND         
SET           AINTGS   BINTGS 
SETUPG         
SOLROT        ROTATE 
SORT         
SPACE         
SPLINE        BFN
SUPDOT         
SWAP         
SYMTRY        HADDON 
THERMO         
TIMCLK         
TIMER         
TIMOUT         
TQL2         
TQLRAT         
TRBAK3         
TRED3         
UPCASE         
UPDATE         
UPDHES         
VECPRT          
WRITE         DATE   WRTTXT  GEOUT  DERIV   TIMOUT SYMTRY  GMETRY GEOUT
              VECPRT MATOUT  CHRGE  BRLZON  MPCSYB DENROT  MOLVAL BONDS
              LOCAL  ENPART  MULLIK MPCPOP  GEOUTG
WRTKEY         
WRTTXT         
XXX         
XYZGEO        BANGLE  DIHED
XYZINT        DIHED  BANGLE  XYZGEO
\end{verbatim}

   A list of subroutines called by various segments (the inverse of the
   first list)
\begin{verbatim}
Subroutine          Called by
AABABC     MECIH
AABACD     MECIH
AABBCD     MECIH
AINTGS     SET      
ANALYT     DCART    
ANAVIB     FORCE    
AXIS       FORCE      FRAME      POLAR    
BABBBC     MECIH
BABBCD     MECIH
BANGLE     XYZGEO     XYZINT   
BFN        SPLINE   
BINTGS     SET      
BKRSAV     EF       
BONDS      WRITE    
BRLZON     FREQCY     WRITE    
CALPAR     DATIN    
CAPCOR     ITER
CDIAG      BRLZON   
CHRGE      DIPIND     FMAT       GEOUT      PRTDRC     WRITE    
CNVG       ITER     
COE        DENROT     DIAT     
COMPFG     DRC        EF         FFHPOL     FLEPO      FMAT       
           FORCE      LINMIN     LOCMIN     MNDO       NLLSQ      
           POLAR      POWSQ      REACT1     SEARCH   
DANG       DIHED    
DATIN      MNDO     
DCART      DERITR   
DELMOL     ANALYT   
DELRI      ANALYT   
DENROT     WRITE    
DENSIT     ITER       MULLIK   
DEPVAR     HADDON   
DERI0      DERNVO   
DERI1      DERNVO   
DERI2      DERI2      DERNVO   
DERI21     DERI2    
DERI22     DERI2    
DERI23     DERI2    
DERITR     DERITR   
DERNVO     DERITR   
DERS       ANALYT   
DFOCK2     DERI1    
DFPSAV     FLEPO      GRID       PATHK      PATHS    
DHC        DCART      DERI1    
DHCORE     DERI1    
DIAG       DERI21     ITER     
DIAGI      DERI21   
DIAT       DIAT       H1ELEC   
DIAT2      DIAT     
DIHED      COMPFG     DCART      XYZGEO     XYZINT   
DIIS       FLEPO    
DIJKL1     DERI1    
DIJKL2     DERI2    
DIPIND     FFHPOL   
DIPOLE     FMAT       WRITE
DOFS       BRLZON   
DRC        FORCE      MNDO     
DRCOUT     PRTDRC
EA08C      EC08C    
EA09C      EA08C    
EC08C      CDIAG    
EF         MNDO     
ENPART     WRITE    
EPSETA     DIAG       ITER       RSP      
EXCHNG     LINMIN     LOCMIN   
FFHPOL     POLAR    
FLEPO      FORCE      GRID       MNDO       PATHK      PATHS      
           REACT1   
FMAT       FORCE    
FOCK2      DERI22     DHC        ITER     
FORCE      MNDO     
FORMD      EF       
FORMXY     DIJKL1     PARTXY   
FORSAV     FMAT     
FRAME      FORCE      FREQCY   
FREQCY     FORCE    
GEOUT      BKRSAV     DERITR     DFPSAV     FLEPO      GETGEO     
           GMETRY     GRID       NLLSQ      PARSAV     PATHK      
           POWSAV     REACT1     READ     
WRITE      WRITE    
GEOUTG     WRITE    
GETDAT     MNDO     
GETGEG     READ     
GETGEO     REACT1     READ     
GETSYM     READ     
GETTXT     READ     
GMETRY     COMPFG     DENROT     DERITR     DIPIND     DRC        
           FORCE      JCARIN     MOLDAT     MULLIK     POLAR      
           REACT1     READ       WRITE    
GOVER      DIAT     
GRID       MNDO     
H1ELEC     DHC        DHCORE     HCORE    
HADDON     SYMTRY   
HCORE      COMPFG     DERITR   
HELECT     DCART      DERI2      ITER
HQRII      EF         INTERP     ITER       MECI     
IJKL       MECI     
INTERP     ITER     
ITER       COMPFG     DERITR   
JAB        DFOCK2     FOCK2    
JCARIN     DERITR   
KAB        DFOCK2     FOCK2    
LINMIN     FLEPO    
LOCAL      WRITE    
LOCMIN     NLLSQ    
MNDO       (main segment)
MAMULT     PULAY    
MATOUT     FFHPOL     FORCE      ITER       LOCAL      MECI       
           WRITE    
ME08A      CDIAG    
ME08B      ME08A    
MECI       COMPFG     DERI1      DERI2      MECI     
MECIH      DERI1      DERI2      MECI     
MECIP      COMPFG   
MINV       DIIS     
MOLDAT     DATIN      MNDO     
MOLVAL     WRITE    
MPCBDS     BONDS    
MPCPOP     WRITE    
MPCSYB     WRITE    
MTXM       DERI1      DERI2      DERI21   
MTXMC      DERI21   
MULLIK     WRITE    
MULT       MULLIK   
MXM        DERI1      DERI2      DERI21     DERI22     DERITR     
           MECIP      MTXMC    
MXMT       DERI22   
NLLSQ      FORCE      MNDO     
NUCHAR     GETGEO     READ     
OSINV      DERI2      PULAY    
OVERLP     FORMD    
PARSAV     NLLSQ    
PARTXY     IJKL     
PATHK      MNDO     
PATHS      MNDO     
PERM       MECI     
POLAR      MNDO     
POWSAV     POWSQ    
POWSQ      MNDO     
PRTDRC     DRC      
PULAY      ITER     
QUADR      PRTDRC   
REACT1     MNDO     
READ       MNDO     
REFER      MOLDAT   
REPP       ROTATE   
ROTAT      DELMOL     DHC        DHCORE     HCORE      SOLROT   
ROTATE     DHC        DHCORE     HCORE      SOLROT   
RSP        AXIS       FFHPOL     FORCE      FREQCY     MULLIK     
           POWSQ    
SCHMIB     INTERP   
SCHMIT     INTERP   
SCOPY      DERI1      DERI2      DERI23     FLEPO    
SEARCH     POWSQ    
SECOND     DERI2      DRC        EF         ESP        FLEPO
           FMAT       FORCE      GRID       ITER       MNDO
           NLLSQ      PATHK      PATHS      POWSQ      REACT1
           TIMER      WRITE
SET        COMPFG     DIAT2    
SETUPG     COMPFG   
SOLROT     DHC        HCORE    
SORT       CDIAG    
SPACE      DIIS     
SPLINE     INTERP   
SUPDOT     DERI1      DERI1      DERI2      DERI22     FLEPO    
SWAP       ITER     
SYMTRY     COMPFG     DERITR     EF         JCARIN     REACT1     
           READ       WRITE    
THERMO     FORCE    
TIMCLK     SECOND
TIMER      COMPFG     DERI1      DERI1    
TIMOUT     WRITE    
TQL2       RSP      
TQLRAT     RSP      
TRBAK3     RSP      
TRED3      RSP      
UPCASE     GETTXT   
UPDATE     DATIN    
UPDHES     EF       
VECPRT     BONDS      DIIS       FFHPOL     FORCE      HCORE    
           ITER       MECI       MOLDAT     MULLIK     POWSQ      
           WRITE    
WRITE      FORCE      ITER       MNDO       PATHS      REACT1   
WRTKEY     READ     
WRTTXT     GEOUT      GRID       PATHK      READ       WRITE    
XXX        GEOUTG   
XYZGEO     XYZINT   
XYZINT     DFPSAV     FORCE      GEOUT      GETGEO     PARSAV     
           POWSAV     PRTDRC   
\end{verbatim}


\chapter{Description of subroutines}
\index{subroutines!description of}
\begin{itemize}
\item AABABC    Utility: Calculates the configuration interaction matrix
           element between two configurations differing by exactly
           one alpha M.O. Called by MECI only.

\item AABACD    Utility: Calculates the configuration interaction matrix
           element between two configurations differing by exactly
           two alpha M.O.'s. Called by MECI only.

\item AABBCD    Utility: Calculates the configuration interaction matrix
           element between two configurations differing by exactly
           two M.O.'s; one configuration has alpha M.O. ``A'' and beta
           M.O. ``C'' while the other configuration has alpha M.O. ``B''
           and beta M.O. ``D''.  Called by MECI only.

\item AINTGS    Utility: Within the overlap integrals, calculates the 
           A-integrals.  Dedicated to function SS within DIAT.

\item ANALYT    Main Sequence:  Calculates the analytical derivatives
           of the energy with respect to cartesian coordinates for all
           atoms. Use only if the mantissa is short (less than 52 bits)
           or out of interest.  Should not be used for routine work
           on a VAX.

\item ANAVIB    Utility:  Gives a brief interpretation of the modes of
           vibration of the molecule. The principal pairs of atoms 
           involved in each vibration are identified, and the mode
           of motion (tangential or radial) is output.

\item AXIS      Utility:  Works out the three principal moments of inertia
           of a molecule. If the system is linear, one moment of inertia
           is zero. Prints moments in units of cm$^{-1}$ and
           $10^{-40}$ g cm$^2$.

\item BABBBC    Utility: Calculates the configuration interaction matrix
           element between two configurations differing by exactly
           one beta M.O. Called by MECI only.

\item BABBCD    Utility: Calculates the configuration interaction matrix
           element between two configurations differing by exactly
           two beta M.O.'s. Called by MECI only.

\item BANGLE    Utility:  Given a set of coordinates, BANGLE will calculate the
           angle between any three atoms.

\item BFN       Utility:  Calculates the B-functions in the Slater overlap.

\item BINTGS    Utility:  Calculates the B-functions in the Slater overlap.

\item BKRSAV    Utility: Saves and restores data used by the eigenvector 
           following subroutine.  Called by EF only.

\item BONDS     Utility:  Evaluates and prints the valencies of atoms and 
           bond-orders between atoms. Main argument: density matrix.
           No results are passed to the calculation, and no data 
           are changed. Called by WRITE only.

\item BRLZON    Main Sequence: BRLZON generates a band structure, or phonon
           structure, for high polymers.  Called by WRITE and FREQCY.

\item CALPAR    Utility: When external parameters are read in via EXTERNAL=,
           the derived parameters are worked out using CALPAR.  Note that
           all derived parameters are calculated for all parameterized
           elements at the same time.

\item CAPCOR    Utility: Capping atoms, of type Cb, should not contribute to 
           the energy of a system.  CAPCOR calculates the energy 
           contribution due to the Cb and subtracts it from the 
           electronic energy.

\item CDIAG     Utility: Complex diagonalization.  Used in generating eigenvalues
           of complex Hermitian secular determinant for band structures.
           Called by BRLZON only.

\item CHRGE     Utility: Calculates the total number of valence electrons
           on each atom. Main arguments: density matrix, array of
           atom charges (empty on input). Called by ITER only.

\item CNVG      Utility: Used in SCF cycle. CNVG does a three-point
           interpolation of the last three density matrices. 
           Arguments: Last three density matrices, Number of iterations,
           measure of self-consistency (empty on input). Called by ITER
           only.

\item COE       Utility:  Within the general overlap routine COE calculates 
           the angular coefficients for the s, p and d real atomic
           orbitals given the axis and returns the rotation matrix.

\item COMPFG    Main Sequence: Evaluates the total heat of formation of the
           supplied geometry, and the derivatives, if requested. This
           is the nodal point connecting the electronic and geometric
           parts of the program. Main arguments: on input: geometry,
           on output: heat of formation, gradients.
 
\item DANG      Utility: Called by XYZINT, DANG computes the angle between a
           point, the origin, and a second point.
 
\item DATIN     Utility:  Reads in external parameters for use within
           MOPAC.  Originally used for the testing of new parameters,
           DATIN is now a general purpose reader for parameters.
           Invoked by the keyword EXTERNAL.

\item DCART     Utility:  Called by DERIV, DCART sets up a list of cartesian
           derivatives of the energy wrt coordinates which DERIV can
           then use to construct the internal coordinate derivatives.

\item DELMOL    Utility:  Part of analytical derivates. Two-electron.

\item DELRI     Utility:  Part of analytical derivates. Two-electron.

\item DENROT    Utility: Converts the ordinary density matrix into 
           a condensed density matrix over basis functions s (sigma), 
           p (sigma) and p (pi), i.e., three basis functions. Useful 
           in hybridization studies. Has capability to handling ``d'' 
           functions, if present.

\item DENSIT    Utility: Constructs the Coulson electron density matrix from
           the eigenvectors. Main arguments: Eigenvectors, No. of singly
           and doubly occupied levels, density matrix (empty on input)
           Called by ITER.

\item DEPVAR    Utility: A symmetry-defined ``bond length'' is related to
           another bond length by a multiple.  This special symmetry
           function is intended for use in Cluster calculations.
           Called by HADDON.

\item DERI0     Utility: Part of the analytical C.I. derivative package. 
           Calculates the diagonal dominant part of the super-matrix.
 
\item DERI1     Utility: Part of the analytical C.I. derivative package. 
           Calculates the frozen density contribution to the derivative of 
           the energy wrt cartesian coordinates, and the derivatives of 
           the frozen Fock matrix in M.O. basis. It's partner is DERI2.
 
\item DERI2     Utility: Part of the analytical C.I. derivative package. 
           Calculates the relaxing density contribution to the derivative
           of the energy wrt cartesian coordinates. Uses the results of
           DERI1.
 
\item DERI21    Utility: Part of the analytical C.I. derivative package. 
           Called by DERI2 only.
 
\item DERI22    Utility: Part of the analytical C.I. derivative package. 
           Called by DERI2 only.
 
\item DERI23    Utility: Part of the analytical C.I. derivative package. 
           Called by DERI2 only.
 
\item DERITR    Utility: Calculates derivatives of the energy wrt internal
           coordinates using full SCF's.  Used as a foolproof way of
           calculating derivatives.  Not recommended for normal use.

\item DERIV     Main Sequence: Calculates the derivatives of the energy with
           respect to the geometric variables.  This is done either by 
           using initially cartesian derivatives (normal mode), by 
           analytical C.I. RHF derivatives, or by full SCF calculations 
           (NOANCI in half-electron and C.I. mode). Arguments: on 
           input: geometry, on output: derivatives. Called by COMPFG.

\item DERNVO    Analytical C.I. Derivative main subroutine.  Calculates the
           derivative of the energy wrt geometry for a non-variationally
           optimized wavefunction (a SCF-CI wavefunction).

\item DERS      Utility:  Called by ANALYT, DERS calculates the analytical
           derivatives of the overlap matrix within the molecular frame.

\item DEX2      Utility: A function called by ESP.

\item DFOCK2    Utility: Part of the analytical C.I. derivative package. Called
           by DERI1, DFOCK2 calculates the frozen density contribution to 
           the derivative of the energy wrt cartesian coordinates.

\item DFPSAV    Utility: Saves and restores data used by the 
           BFGS geometry optimization. Main arguments:
           parameters being optimized, gradients of parameters, last heat 
           of formation, integer and real control data. Called by FLEPO.

\item DHC       Utility:  Called by DCART and calculates the energy of a pair
           of atoms using the SCF density matrix.  Used in the finite
           difference derivatve calculation.

\item DHCORE    Utility: Part of the analytical C.I. derivative package. Called
           by DERI1, DHCORE calculates the derivatives of the 1 and 2 
           electron integrals wrt cartesian coordinates.

\item DIAG      Utility: Rapid pseudo-diagonalization. Given a set of vectors
           which almost block-diagonalize a secular determinant, DIAG
           modifies the vectors so that the block-diagonalization is more
           exact. Main arguments: Old vectors, Secular Determinant, 
           New vectors (on output).  Called by ITER.

\item DIAGI     Utility: Calculates the electronic energy arising from
           a given configuration. Called by MECI.

\item DIAT      Utility: Calculates overlap integrals between two atoms in 
           general cartesian space. Principal quantum numbers up to 6, and
           angular quantum numbers up to 2 are allowed. Main arguments:
           Atomic numbers and cartesian coordinates in Angstroms of the 
           two atoms, Diatomic overlaps (on exit). Called by H1ELEC.

\item DIAT2     Utility: Calculates reduced overlap integrals between atoms 
           of principal quantum numbers 1, 2, and 3, for s and p orbitals.
           Faster than the SS in DIAT. This is a dedicated subroutine, and
           is unable to stand alone without considerable backup. Called
           by DIAT.

\item DIGIT     Utility: Part of READA.  DIGIT assembles numbers given a 
           character string.

\item DIHED     Utility:  Calculates the dihedral angle between four atoms.
           Used in converting from cartesian to internal coordinates.

\item DIIS      Utility: Pulay's Geometric Direct Inversion of the Iterative
           Subspace (G-DIIS) accelerates the rate at which the BFGS
           locates an energy minimum.  (In MOPAC 6.00, the DIIS is only
           partially installed --- several capabilities of the DIIS are not
           used)
 
\item DIJKL1    Utility: Part of the analytical C.I. derivative package. Called
           by DERI1, DIJKL1 calculates the two-electron integrals over
           M.O. bases, e.g. \verb/<i,j (1/r) k,l>/.
 
\item DIJKL2    Utility: Part of the analytical C.I. derivative package. Called
           by DERI2, DIJKL2 calculates the derivatives of the two-electron 
           integrals over M.O. bases, e.g. \verb/<i,j (1/r) k,l>/, wrt
           cartesian coordinates.
 
\item DIPIND    Utility: Similar to DIPOLE, but used by the POLAR calculation 
           only.

\item DIPOLE    Utility: Evaluates and, if requested, prints dipole components
           and dipole for the molecule or ion. Arguments: Density matrix, 
           Charges on every atom, coordinates, dipoles (on exit). 
           Called by WRITE and FMAT.

\item DIST2     Utility: Called by ESP only, DIST2 works out the distance
           between two points in 3D space.
 
\item DOFS      Main Sequence: Calculates the density of states within a 
           Brillouin zone.  Used in polymer work only.

\item DOT       Utility: Given two vectors, X and Y, of length N, function DOT
           returns with the dot product X.Y. I.e., if X=Y, then DOT = the
           square of X. Called by FLEPO.

\item DRC       Main Sequence:  The dynamic and intrinsic reaction coordinates 
           are calculated by following the mass-weighted trajectories.

\item DRCOUT    Utility:  Sets up DRC and IRC data in quadratic form 
           preparatory to being printed.

\item EA08C     Part of the diagonalizer RSP.

\item EA09C     Part of the diagonalizer RSP.

\item EC08C     Part of the diagonalizer RSP.

\item EF        Main Sequence: EF is the Eigenvector Following routine.  
           EF implements the keywords EF and TS.
 
\item ELESP     Utility: Within the ESP, ELESP calculates the electronic 
           contribution to the electrostatic potential.

\item ENPART    Utility: Partitions the energy of a molecule into its monatomic
           and diatomic components. Called by WRITE when the keyword 
           ENPART is specified. No data are changed by this call.

\item EPSETA    Utility:  Calculates the machine precision and dynamic range
           for use by the two diagonalizers.

\item ESP       Main Sequence: ESP is not present in the default copy of MOPAC.
           ESP calculates the atomic charges which would reproduce the
           electrostatic potential of the nuclii and electronic wavefunction.
 
\item ESPBLO    Block Data: Used by the ESP calculation, ESPBLO fills two small
           arrays!
 
\item ESPFIT    Utility: Part of the ESP.  ESP fits the quantum mechanical
           potential to a classical point charge model

\item EXCHNG    Utility: Dedicated procedure for storing 3 parameters and one
           array in a store. Used by SEARCH.

\item FFHPOL    Utility: Part of the POLAR calculation.  Evaluates the 
           effect of an electric field on a molecule.

\item FLEPO     Main Sequence: Optimizes a geometry by minimizing the energy.
           Makes use of the first and estimated second derivatives to
           achieve this end.  Arguments: Parameters to be optimized, 
           (overwritten on exit with the optimized parameters), Number of 
           parameters, final optimized heat of formation. Called by MAIN, 
           REACT1, and FORCE.

\item FM06AS    Utility: Part of CDIAG.
 
\item FM06BS    Utility: part of CDIAG.

\item FMAT      Main sequence: Calculates the exact Hessian matrix for a system
           This is done by either using differences of first derivatives 
           (normal mode) or by four full SCF calculations (half electron 
           or C.I. mode). Called by FORCE.

\item FOCK1     Utility: Adds on to Fock matrix the one-center two electron 
           terms. Called by ITER only.

\item FOCK2     Utility: Adds on to Fock matrix the two-center two electron 
           terms. Called by ITER and DERIV. In ITER the entire Fock matrix
           is filled; in DERIV, only diatomic Fock matrices are 
           constructed.

\item FOCK2D    Written out of MOPAC 6.00.

\item FORCE     Main sequence: Performs a force-constant and vibrational 
           frequency calculation on a given system. If the starting 
           gradients are large, the geometry is optimized to reduce the 
           gradient norm, unless LET is specified in the keywords. 
           Isotopic substitution is allowed. Thermochemical quantities 
           are calculated. Called by MAIN.

\item FORMD     Main Sequence: Called by EF. FORMD constructs the next step
           in the geometry optimization or transition state location.
 
\item FORMXY    Utility: Part of DIJKL1. FORMXY constructs part of the two-
           electron integral over M.O.'s.

\item FORSAV    Utility: Saves and restores data used in FMAT in FORCE 
           calculation. Called by FMAT.

\item FRAME     Utility: Applies a very rigid constraint on the translations
           and rotations of the system. Used to separate the trivial
           vibrations in a FORCE calculation.

\item FREQCY    Main sequence: Final stage of a FORCE calculation. Evaluates
           and prints the vibrational frequencies and modes.

\item FSUB      Utility: Part of ESP.
 
\item GENUN     Utility: Part of ESP. Generates unit vectors over a sphere.
           called by SURFAC only.

\item GEOUT     Utility: Prints out the current geometry. Can be called at 
           any time. Does not change any data.

\item GEOUTG    Utility: Prints out the current geometry in Gaussian Z-matrix
           format.

\item GETDAT    Utility: Reads in all the data, and puts it in a scratch file
           on channel 5.

\item GETGEG    Utility: Reads in Gaussian Z-matrix geometry. Equivalent to
           GETGEO and GETSYM combined.

\item GETGEO    Utility: Reads in geometry in character mode from specified
           channel, and stores parameters in arrays. Some error-checking 
           is done. Called by READ and REACT1.

\item GETSYM    Utility: Reads in symmetry data. Used by READ.

\item GETTXT    Utility: Reads in KEYWRD, KOMENT and TITLE.
 
\item GETVAL    Utility: Called by GETGEG, GETVAL either gets an internal
           coordinate or a logical name for that coordinate.

\item GMETRY    Utility: Fills the cartesian coordinates array. Data are 
           supplied from the array GEO, GEO can be (a) in internal 
           coordinates, or (b) in cartesian coordinates. If STEP is 
           non-zero, then the coordinates are modified in light of the 
           other geometry and STEP. Called by HCORE, DERIV, READ, WRITE, 
           MOLDAT, etc.

\item GOVER     Utility: Calculates the overlap of two Slater orbitals which
           have been expanded into six gaussians.  Calculates the
           STP-6G overlap integrals.

\item GRID      Main Sequence:  Calculates a grid of points for a 2-D search
           in coordinate space.  Useful when more information is needed
           about a reaction surface.

\item H1ELEC    Utility: Given any two atoms in cartesian space, H1ELEC 
           calculates the one-electron energies of the off-diagonal 
           elements of the atomic orbital matrix. 
           $$ H(i,j) = -S(i,j) [\beta(i)+\beta(j)]/2 $$
           Called by HCORE and DERIV.

\item HADDON    Utility: The symmetry operation subroutine, HADDON relates two
           geometric variables by making one a dependent function of the 
           other. Called by SYMTRY only.

\item HCORE     Main sequence: Sets up the energy terms used in calculating the
           SCF heat of formation. Calculates the one and two electron
           matrices, and the nuclear energy.
           Called by COMPFG.

\item HELECT    Utility: Given the density matrix, and the one electron and
           Fock matrices, calculates the electronic energy. No data are 
           changed by a call of HELECT. Called by ITER and DERIV.

\item HQRII     Utility: Rapid diagonalization routine. Accepts a secular
           determinant, and produces a set of eigenvectors and
           eigenvalues. The secular determinant is destroyed.

\item IJKL      Utility: Fills the large two-electron array over a M.O.
           basis set. Called by MECI.

\item INTERP    Utility: Runs the Camp-King converger. q.v.

\item ITER      Main sequence: Given the one and two electron matrices, ITER
           calculates the Fock and density matrices, and the electronic
           energy. Called by COMPFG.

\item JAB       Utility: Calculates the coulomb contribution to the Fock matrix
           in NDDO formalism. Called by FOCK2.
 
\item JCARIN    Utility: Calculates the difference vector in cartesian coordinates
           corresponding to a small change in internal coordinates.
 
\item KAB       Utility: Calculates the exchange contribution to the Fock matrix
           in NDDO formalism. Called by FOCK2.

\item LINMIN    Main sequence: Called by the BFGS geometry optimized FLEPO,
           LINMIN takes a step in the search-direction and if the energy
           drops, returns.  Otherwise it takes more steps until if finds
           one which causes the energy to drop.

\item LOCAL     Utility: Given a set of occupied eigenvectors, produces a 
           canonical set of localized bonding orbitals, by a series of 
           $2\times 2$ rotations which maximize $\langle \psi^4 \rangle$. 
           Called by WRITE.

\item LOCMIN    Main sequence: In a gradient minimization, LOCMIN does a line-
           search to find the gradient norm minimum. Main arguments:
           current geometry, search direction, step, current gradient 
           norm; on exit: optimized geometry, gradient norm.

\item MAMULT    Utility: Matrix multiplication. Two matrices, stored as lower
           half triangular packed arrays, are multiplied together, and the
           result stored in a third array as the lower half triangular
           array. Called from PULAY.

\item MATOUT    Utility: Matrix printer. Prints a square matrix, and a 
           row-vector, usually eigenvectors and eigenvalues. The indices 
           printed depend on the size of the matrix: they can be either 
           over orbitals, atoms, or simply numbers, thus M.O.'s are over 
           orbitals, vibrational modes are over numbers. Called by WRITE, 
           FORCE.

\item ME08A, ME08B  Utilities: Part of the complex diagonalizer, and called by 
     CDIAG.

\item MECI      Main sequence: Main function for Configuration Interaction,
           MECI constructs the appropriate C.I. matrix, and evaluates the
           roots, which correspond to the electronic energy of the states
           of the system. The appropriate root is then returned.
           Called by ITER only.

\item MECID     Utility: Constructs the differential C.I. secular determinant.
 
\item MECIH     Utility: Constructs the normal C.I. secular determinant.
 
\item MECIP     Utility: Reforms the density matrix after a MECI calculation.
 
\item MINV      Utility: Called by DIIS. MINV inverts the Hessian matrix.

\item MNDO      Main sequence: MAIN program. MNDO first reads in data using 
           READ, then calls either FLEPO to do geometry optimization, 
           FORCE to do a FORCE calculation, PATHS for a reaction with a 
           supplied coordinate, NLLSQ for a gradient minimization or 
           REACT1 for locating the transition state. Starts the timer.

\item MOLDAT    Main Sequence: Sets up all the invariant parameters used during
           the calculation, e.g. number of electrons, initial atomic 
           orbital populations, number of open shells, etc. Called once by 
           MNDO only.

\item MOLVAL    Utility: Calculates the contribution from each M.O. to the
           total valency in the molecule.  Empty M.O.'s normally
           have a negative molecular valency.

\item MTXM      Utility: Part of the matrix package. Multiplies together two
           rectangular packed arrays, i.e., C = A.B.

\item MTXMC     Utility: Part of the matrix package.  Similar to MTXM.

\item MULLIK    Utility: Constructs and prints the Mulliken Population 
           Analysis. Available only for RHF calculations. Called by
           WRITE.

\item MULT      Utility: Used by MULLIK only, MULT multiplies two square 
           matrices together. 

\item MXM       Utility: Part of the matrix package. Similar to MTXM.
 
\item MXMT      Utility: Part of the matrix package. Similar to MTXM.

\item MYWORD    Utility: Called in WRTKEY, MYWORD checks for the existance of
           a specific string. If it is found, MYWORD is set true, and
           the all occurances of string are deleted. Any words
           not recognised will be flagged and the job stopped.

\item NAICAP    Utility: Called by ESP.
 
\item NAICAP    Utility: Called by ESP.

\item NLLSQ     Main sequence: Used in the gradient norm minimization.

\item NUCHAR    Takes a character string and reads all the numbers in it
           and stores these in an array.

\item OSINV     Utility: Inverts a square matrix. Called by PULAY only.

\item OVERLP    Utility: Part of EF. OVERLP decides which normal mode to 
           follow.
 
\item OVLP      Utility: Called by ESP only.  OVLP calculates the overlap
           over Gaussian STO's.

\item PARSAV    Utility: Stores and restores data used in the gradient-norm
           minimization calculation.

\item PARTXY    Utility: Called by IJKL only, PARTXY calculates the partial
           product \verb/<i,j (1/r) in <i,j (1/r) k,l>/.
 
\item PATHK     Main sequence: Calculates a reaction coordinate which uses
           a constant step-size.  Invoked by keywords STEP and POINTS. 

\item PATHS     Main sequence: Given a reaction coordinate as a row-vector, 
           PATHS performs a FLEPO geometry optimization for each point,
           the later geometries being initially guessed from a knowledge
           of the already optimized geometries, and the current step.
           Called by MNDO only.

\item PDGRID    Utility: Part of ESP. Calculates the Williams surface.

\item PERM      Utility: Permutes n1 electrons of alpha or beta spin among
              n2 M.O.'s. 

\item POLAR     Utility: Calculates the polarizability volumes for a molecule
           or ion. Uses 19 SCF calculations, so appears after WRITE has
           finished. Cannot be used with FORCE, but can be used anywhere
           else. Called by WRITE.

\item POWSAV    Utility: Calculation store and restart for SIGMA 
           calculation. Called by POWSQ.

\item POWSQ     Main sequence: The McIver - Komornicki gradient 
           minimization routine. Constructs a full Hessian matrix 
           and proceeds by line-searches Called from MAIN when 
           SIGMA is specified.

\item PRTDRC    Utility:  Prints DRC and IRC results according to instructions.
           Output can be (a) every point calculated (default), (b) in
           constant steps in time, space or energy.

\item PULAY     Utility: A new converger. Uses a powerful 
           mathematical non-iterative method for obtaining the SCF Fock 
           matrix. Principle is that at SCF the eigenvectors of the Fock 
           and density matrices are identical, so [F.P] is a measure of
           the non-self consistency. While very powerful, PULAY is not 
           universally applicable. Used by ITER.

\item QUADR:    Utility:  Used in printing the IRC - DRC results.  Sets up
           a quadratic in time of calculated quantities so that PRTDRC 
           can select specific reaction times for printing.

\item REACT1    Main sequence: Uses reactants and products to find the 
           transition state. A hypersphere of N dimensions is centered on 
           each moiety, and the radius steadily reduced. The entity of 
           lower energy is moved, and when the radius vanishes, the 
           transition state is reached. Called by MNDO only.

\item READ      Main sequence: Almost all the data are read in through READ. 
           There is a lot of data-checking in READ, but very little 
           calculation. Called by MNDO.

\item READA     Utility: General purpose character number reader. Used to enter
           numerical data in the control line as \verb/<variable>=n.nnn/ where
           \verb/<variable>/ is a mnemonic such as SCFCRT or CHARGE. 
           Called by READ, FLEPO, ITER, FORCE, and many other subroutines.

\item REFER     Utility: Prints the original references for atomic data.
           If an atom does not have a reference, i.e., it has not been
           parameterized, then a warning message will be printed and
           the calculation stopped.

\item REPP      Utility: Calculates the 22 two-electron reduced repulsion 
           integrals, and the 8 electron-nuclear attraction integrals.
           These are in a local coordinate system. Arguments: atomic
           numbers of the two atoms, interatomic distance, and arrays to 
           hold the calculated integrals. Called by ROTATE only.

\item ROTAT     Utility: Rotates analytical two-electron derivatives from
           atomic to molecular frame.

\item ROTATE    Utility: All the two-electron repulsion integrals, the electron-
           nuclear attraction integrals, and the nuclear-nuclear repulsion
           term between two atoms are calculated here. Typically 100 two-
           electron integrals are evaluated.

\item RSP       Utility: Rapid diagonalization routine. Accepts a secular
           determinant, and produces a set of eigenvectors and 
           eigenvalues. The secular determinant is destroyed. 

\item SAXPY     Utility: Called by the utility SUPDOT only!

\item SCHMIB    Utility: Part of Camp-King converger.

\item SCHMIT    Utility: Part of Camp-King converger.

\item SCOPY     Utility: Copies an array into another array.
 
\item SDOT      Utility: Forms the scalar of the product of two vectors.

\item SEARCH    Utility: Part of the SIGMA and NLLSQ gradient minimizations. 
           The line-search subroutine, SEARCH locates the gradient 
           minimum and calculates the second derivative of the energy 
           in the search direction. Called by POWSQ and NLLSQ.

\item SECOND    Utility: Contains VAX specific code. Function SECOND 
           returns the number of CPU seconds elapsed since an arbitrary 
           starting time. If the SHUTDOWN command has been issued, 
           the CPU time is in error by exactly 1,000,000 seconds, and 
           the job usually terminates with the message ``time exceeded''.

\item SET       Utility: Called by DIAT2, evaluates some terms used in overlap
           calculation.

\item SETUP3    Utility: Sets up the Gaussian expansion of Slater orbitals
           using a STO-3G basis set.
 
\item SETUPG    Utility: Sets up the Gaussian expansion of Slater orbitals
           using a STO-6G basis set.

\item SOLROT    Utility: For Cluster systems, adds all the two-electron
           integrals of the same type, between different unit cells, and
           stores them in a single array. Has no effect on molecules.

\item SORT      Utility: Part of CDIAG, the complex diagonalizer.
 
\item SPACE     Utility: Called by DIIS only.
 
\item SPCG      Written out of Version 6.00.

\item SPLINE    Utility: Part of Camp-King converger.

\item SS        Utility: An almost general Slater orbital overlap calculation.
           Called by DIAT.

\item SUPDOT    Utility: Matrix mutiplication A=B.C
 
\item SURFAC    Utility: Part of the ESP.

\item SWAP      Utility: Used with FILL=, SWAP ensures that a specified 
           M.O. is filled. Called by ITER only.

\item SYMTRY    Utility: Calculates values for geometric parameters from known
           geometric parameters and symmetry data. Called whenever GMETRY
           is called.

\item THERMO    Main sequence: After the vibrational frequencies have been 
           calculated, THERMO calculates thermodynamic quantities such as
           internal energy, heat capacity, entropy, etc, for translational,
           vibrational, and rotational, degrees of freedom.

\item TIMCLK    Utility: Vax-specific code for determining CPU time.

\item TIMER     Utility: Prints times of various steps.

\item TIMOUT    Utility: Prints total CPU time in elegant format.

\item TQL2      Utility:  Part of the RSP.

\item TQLRAT    Utility:  Part of the RSP.

\item TRBAK3    Utility:  Part of the RSP.

\item TRED3     Utility:  Part of the RSP.

\item UPDATE    Utility:  Given a set of new parameters, stores these 
           in their appropriate arrays.  Invoked by EXTERNAL.

\item UPDHES    Utility: Called by EF, UPDHES updates the Hessian matrix.

\item VECPRT    Utility: Prints out a packed, lower-half triangular matrix. 
           The labeling of the sides of the matrix depend on the matrix's
           size: if it is equal to the number of orbitals, atoms, or other.
           Arguments: The matrix to be printed, size of matrix. No data
           are changed by a call of VECPRT.

\item WRITE     Main sequence: Most of the results are printed here. All 
           relevant arrays are assumed to be filled. A call of WRITE only 
           changes the number of SCF calls made, this is reset to zero. 
           No other data are changed. Called by MAIN, FLEPO, FORCE.

\item WRTKEY    Main Sequence:  Prints all keywords and checks for
           compatability and to see if any are not recognised.  
           WRTKEY can stop the job if any errors are found.

\item WRTTXT    Main Sequence:  Writes out KEYWRD, KOMENT and TITLE.  The
           inverse of GETTXT.
 
\item XXX       Utility: Forms a unique logical name for a Gaussian Z-matrix
           logical.  Called  by GEOUTG only.

\item XYZINT    Utility: Converts from cartesian coordinates into internal.

\item XYZGEO    XYZINT sets up its own numbering system, so no connectivity
           is needed. 
\end{itemize}

         

\chapter{Heats of formation}
\section*{Test MNDO, PM3 and AM1 compounds}
   In order to verify that MOPAC is working correctly, a large  number
   of  tests need to be done.  These take about 45 minutes on a VAX 11--780,
   and even then many potential bugs remain undetected.   It  is  obviously
   impractical  to ask users to test MOPAC.  However, users must be able to
   verify the basic working of MOPAC, and to do this  the  following  tests
   for the elements have been provided.

   Each element can be tested by making up a data-file using estimated
   geometries  and running that file using MOPAC.  The optimized geometries
   should give rise to heats of formation as shown.  Any difference greater
   than 0.1 kcal/mole indicates a serious error 
   in the program.\index{heat of formation!molecular standards}

\subsection*{Caveats}

\begin{enumerate}
\item Geometry definitions must be correct.
\item Heats of formation  may  be  too  high  for  certain
  compounds.   This is due to a poor starting geometry
  trapping the system in an excited  state.   (Affects ICl at times)
\end{enumerate}
            
\begin{verbatim}
     Element     Test Compound          Heat of Formation
                                  MINDO/3     MNDO    AM1   PM3
     Hydrogen         CH4           -6.3    -11.9   -8.8  -13.0
     Lithium          LiH                   +23.2        
     Beryllium        BeO                   +38.6         +53.0
     Boron            BF3         -270.2   -261.0 -272.1*
     Carbon           CH4           -6.3    -11.9   -8.8  -13.0
     Nitrogen         NH3           -9.1     -6.4   -7.3   -3.1
     Oxygen           CO2          -95.7    -75.1  -79.8  -85.0
     Fluorine         CF4         -223.9   -214.2 -225.7 -225.1
     Magnesium        MgF2                               -160.7
     Aluminium        AlF                   -83.6  -77.9  -50.1
     Silicon          SiH          +82.9    +90.2  +84.5  +94.6
     Phosphorus       PH3           +2.5     +3.9  +10.2   +0.2
     Sulfur           H2S           -2.6     +3.8   +1.2   -0.9
     Chlorine         HCl          -21.1    -15.3  -24.6  -20.5
     Zinc             ZnMe2                 +19.9  +19.8    8.2
     Gallium          GaCl3                               -79.7
     Germanium        GeF                   -16.4  -19.7   -3.3
     Arsenic          AsH3                                +12.7
     Selenium         SeCl2                               -38.0
     Bromine          HBr                    +3.6  -10.5   +5.3
     Cadmium          CdCl2                               -48.6
     Indium           InCl3                               -72.8
     Tin              SnF                   -20.4         -17.5
     Antimony         SbCl3                               -72.4
     Tellurium        TeH2                                +23.8
     Iodine           ICl                    -6.7  -4.6   +10.8
     Mercury          HgCl2                 -36.9 -44.8   -32.7
     Thallium         TlCl                                -13.4
     Lead             PbF                   -22.6         -21.0
     Bismuth          BiCl3                               -42.6
   * Not an exhaustive test of AM1 boron.
\end{verbatim}


\chapter{References}
\subsubsection{On G-DIIS}
 ``Computational Strategies for the Optimization of 
 Equilibrium Geometry and Transition-State Structures at 
 the Semiempirical Level'', Peter L. Cummings, Jill E. Gready, 
 {\em J. Comp. Chem.}, 10:939-950 (1989).\index{references!on G--DIIS}
 
\subsubsection{On Analytical C.I. Derivatives}
  ``An Efficient Procedure for Calculating the Molecular Gradient, 
using SCF-CI Semiempirical Wavefunctions with a Limited Number 
of Configurations'', M. J. S. Dewar, D. A. Liotard, {\em J. Mol. Struct. 
(Theochem)}, 206:123-133 (1990).\index{references!CI derivatives}

\subsubsection{On Eigenvector Following}
 J. Baker, {\em J. Comp. Chem.}, 7:385 (1986).
\index{references!eigenvector following}

\subsubsection{On ElectroStatic Potentials (ESP)}
 ``Atomic Charges Derived from Semiempirical Methods'', 
  B. H. Besler, K. M. Merz, Jr., P. A. Kollman, 
  {\em J. Comp. Chem.}, 11:431-439 (1990).\index{references!ESP}

\subsubsection{On MNDO }
``Ground States of Molecules. 38. The MNDO Method.
Approximations and Parameters.'', M.J.S. Dewar, W.Thiel,
{\em J. Am. Chem. Soc.}, 99:4899, (1977).\index{references!MNDO}

Original References for Elements Parameterized in MNDO:
\begin{description}
\item[H]    M.J.S. Dewar, W. Thiel, {\em J. Am. Chem. Soc.}, 99, 4907, (1977).
\item[Li]   Parameters taken from the MNDOC program, written by Walter Thiel,
    {\em Quant. Chem. Prog. Exch. No. 438};  2:63, (1982).
\item[Be]   M.J.S. Dewar, H.S. Rzepa, {\em J. Am. Chem. Soc.}, 100:777, (1978).
\item[B]    M.J.S. Dewar, M.L. McKee, {\em J. Am. Chem. Soc.}, 99:5231, (1977).
\item[C]    M.J.S. Dewar, W. Thiel, {\em J. Am. Chem. Soc.}, 99:4907, (1977).
\item[N]    M.J.S. Dewar, W. Thiel, {\em J. Am. Chem. Soc.}, 99:4907, (1977).
\item[O]    M.J.S. Dewar, W. Thiel, {\em J. Am. Chem. Soc.}, 99:4907, (1977).
\item[F]    M.J.S. Dewar, H.S. Rzepa, {\em J. Am. Chem. Soc.}, 100:58, (1978).
\item[Al]   L.P. Davis, R.M. Guidry, J.R. Williams, M.J.S. Dewar, H.S. Rzepa
       {\em J. Comp. Chem.}, 2:433, (1981).
\item[Si]   (a) M.J.S. Dewar, M.L. McKee, H.S. Rzepa, {\em J. Am. Chem. Soc.}, 
       100:3607 (1978).  \ddag\\ 
       (c) M.J.S. Dewar, J. Friedheim, G. Grady, E.F. Healy, 
       J.J.P. Stewart, {\em Organometallics}, 5:375 (1986).
\item[P]    M.J.S. Dewar, M.L. McKee, H.S. Rzepa, {\em J. Am. Chem. Soc.}, 
       100: 3607 (1978).
\item[S]    (a) M.J.S. Dewar, M.L. McKee, H.S. Rzepa, {\em J. Am. Chem. Soc.}, 
       100:3607 (1978).  \ddag \\
       (b) M.J.S. Dewar, C. H. Reynolds, {\em J. Comp. Chem.}, 7:140 (1986).\\
\item[Cl]   (a) M.J.S. Dewar, M.L. McKee, H.S. Rzepa, {\em J. Am. Chem. Soc.}, 
       100, 3607 (1978).  \ddag \\
       (b) M.J.S. Dewar, H.S. Rzepa, J. Comp. Chem., 4, 158, (1983) 
\item[Zn]   M.J.S. Dewar, K. M. Merz, {\em Organometallics}, 5:1494 (1986).%\\
\item[Ge]   M.J.S. Dewar, G.L. Grady, E.F. Healy, {\em Organometallics},
       6:186 (1987).
\item[Br]   M.J.S. Dewar, E.F. Healy,  {\em J. Comp. Chem.}, 4:542, (1983).
\item[I]    M.J.S. Dewar, E.F. Healy, J.J.P. Stewart, {\em J. Comp. Chem.},
       5:358, (1984).
\item[Sn]   M.J.S. Dewar, G.L. Grady, J.J.P. Stewart, {\em J. Am. Chem. Soc.}, 
       106:6771 (1984).
\item[Hg]   M.J.S. Dewar, G.L. Grady, K. Merz, J.J.P. Stewart, 
       {\em Organometallics}, 4:1964, (1985).
\item[Pb]   M.J.S. Dewar, M. Holloway, G.L. Grady, J.J.P. Stewart, 
       {\em Organometallics}, 4:1973, (1985).
\end{description}

N.B.: \ddag --- Parameters defined here are obsolete.

\subsubsection{On MINDO/3}
 Part XXVI, Bingham, R.C., Dewar, M.J.S., Lo, D.H,
 {\em J. Am. Chem. Soc.},  97, (1975).\index{references!MINDO/3}

\subsubsection{On AM1}\index{AM1!references}
``AM1: A New General Purpose Quantum Mechanical Molecular 
Model'', M.J.S. Dewar, E.G. Zoebisch, E.F. Healy, J.J.P. Stewart, 
{\em J. Am. Chem. Soc.}, 107:3902--3909 (1985).\index{references!AM1}

\subsubsection{On PM3}
  ``Optimization of Parameters for Semi-Empirical Methods 
  I--Method'', J.J.P. Stewart, {\em J. Comp. Chem.}, 10:221 (1989).\\
  ``Optimization of Parameters for Semi-Empirical Methods 
  II--Applications, J.J.P. Stewart, {\em J. Comp. Chem.}, 10:221 (1989).
  (These two references refer to H, C, N, O, F, Al, Si, P, 
  S, Cl, Br, and I).\index{references!PM3}

  ``Optimization of Parameters for Semi-Empirical Methods 
  III--Extension of PM3 to Be, Mg, Zn, Ga, Ge, As, Se, Cd, 
  In, Sn, Sb, Te, Hg, Tl, Pb, and Bi'',  J.J.P. Stewart, 
  {\em J. Comp. Chem.\/} (In press, expected date of publication, 
  Feb. 1991).

Original References for Elements Parameterized in AM1:
\begin{description}
\item[H]    M.J.S. Dewar, E.G. Zoebisch, E.F. Healy, J.J.P. Stewart, 
            {\em J. Am. Chem. Soc.}, 107:3902-3909 (1985).
\item[B]    M.J.S. Dewar, C Jie, E. G. Zoebisch, {\em Organometallics}, 
            7:513-521 (1988).
\item[C]    M.J.S. Dewar, E.G. Zoebisch, E.F. Healy, J.J.P. Stewart, 
            {\em J. Am. Chem. Soc.}, 107:3902-3909 (1985).
\item[N]    M.J.S. Dewar, E.G. Zoebisch, E.F. Healy, J.J.P. Stewart, 
            {\em J. Am. Chem. Soc.}, 107:3902-3909 (1985).
\item[O]    M.J.S. Dewar, E.G. Zoebisch, E.F. Healy, J.J.P. Stewart, 
            {\em J. Am. Chem. Soc.}, 107:3902-3909 (1985).
\item[F]    M.J.S. Dewar, E.G. Zoebisch, {\em Theochem.}, 180:1 (1988).
\item[Al]   M.J.S. Dewar, A.J. Holder, {\em Organometallics}, 9:508 (1990).
\item[Si]   M.J.S. Dewar, C. Jie, {\em Organometallics}, 6:1486-1490 (1987). 
\item[P]    M.J.S. Dewar, C.Jie, {\em Theochem.}, 187:1 (1989)
\item[S]    (No reference)
\item[Cl]   M.J.S. Dewar, E.G. Zoebisch, {\em Theochem.}, 180:1 (1988).
\item[Zn]   M.J.S. Dewar, K.M. Merz, Jr., {\em Organometallics}, 7:522 (1988).
\item[Ga]   M.J.S. Dewar, C. Jie, {\em Organometallics}, 8:1544 (1989).
\item[Br]   M.J.S. Dewar, E.G. Zoebisch, {\em Theochem.}, 180:1 (1988).
\item[I]    M.J.S. Dewar, E.G. Zoebisch, {\em Theochem.}, 180:1 (1988).
\item[Hg]   M. J. S. Dewar, C. Jie, {\em Organometallics}, 8:1547 (1989).
   (see also PARASOK for the use of MNDO parameters for other elements)
\end{description}

 \subsubsection{On Shift}
             ``The Dynamic `Level Shift' Method for Improving the 
             Convergence of the SCF Procedure'', A. V. Mitin, 
{\em J. Comp. Chem.}, 9:107-110 (1988).\index{references!SHIFT}

\subsubsection{On Half-Electron}
  ``Ground States of Conjugated Molecules.
  IX. Hydrocarbon Radicals and Radical Ions'', M.J.S. Dewar,
  J.A. Hashmall, C.G. Venier, {\em J.A.C.S.}, 90:1953 (1968).\\
  ``Triplet States of Aromatic Hydrocarbons'', M.J.S. Dewar,
  N.  Trinajstic, {\em Chem. Comm.}, 646, (1970).\\
  ``Semiempirical SCF-MO Treatment of Excited States of 
  Aromatic Compounds'', M.J.S. Dewar, N.  Trinajstic, 
  {\em J. Chem. Soc., (A)}, 1220, (1971).\index{references!half-electron}

\subsubsection{On Pulay's Converger}
``Convergence Acceleration of Iterative Sequences. The Case 
of SCF Iteration'', Pulay, P., 
{\em Chem. Phys. Lett.}, 73:393, (1980).\index{references!Pulay's converger}

\subsubsection{On Pseudodiagonalization}
 ``Fast Semiempirical Calculations'',
  Stewart. J.J.P., Csaszar, P., Pulay, P., 
{\em J. Comp. Chem.}, 3:227, (1982).\index{references!pseudo-diagonalization}

\subsubsection{On Localization}
 ``A New Rapid Method for Orbital Localization.'',
  P.G. Perkins and J.J.P. Stewart,{\em J.C.S. Faraday 
  (II)}, 77:000, (1981).\index{references!localization}

    
\subsubsection{On Diagonalization}
 Beppu, Y., {\em Computers and Chemistry}, Vol.6
(1982).\index{references!diagonalization}

\subsubsection{On MECI}
 ``Molecular Orbital Theory for the Excited States of
  Transition Metal Complexes'', D.R. Armstrong, R. Fortune,
  P.G. Perkins, and J.J.P. Stewart, 
{\em J. Chem. Soc., Faraday II}, 68:1839-1846 (1972).\index{references!MECI}

\subsubsection{On Broyden-Fletcher-Goldfarb-Shanno Method}
  Broyden, C. G., {\em Journal of the Institute for Mathematics 
  and Applications}, Vol. 6, pp 222--231, 1970.\\
  Fletcher, R., {\em Computer Journal}, Vol. 13, pp 317--322, 1970.
  Goldfarb, D., {\em Mathematics of Computation}, Vol. 24, 
  pp 23-26, 1970.\\
  Shanno, D. F.,  {\em Mathematics of Computation}, Vol. 24, 
  pp 647-656 1970. See also summary in:
  Shanno, D. F., {\em J. of Optimization Theory and Applications},
  Vol.46, No 1 pp 87-94 1985.\index{references!BFGS method}

\subsubsection{On Polarizability}
  ``Calculation of Nonlinear Optical Properties of
  Molecules'', H. A. Kurtz, J. J. P. Stewart, K. M. Dieter,
  {\em J. Comp. Chem.}, 11:82 (1990). See also
  ``Semiempirical Calculation of the Hyperpolarizability of Polyenes'',
   H. A. Kurtz, {\em I. J. Quant. Chem. Symp.}, 24, xxx (1990).
\index{references!polarizability}

\subsubsection{On Thermodynamics}
 ``Ground States of Molecules. 44 MINDO/3 Calculations of
  Absolute Heat Capacities and Entropies of Molecules
  without Internal Rotations. Dewar, M.J.S., Ford, G.P.,
  {\em J. Am. Chem. Soc.}, 99:7822 (1977).\index{references!thermodynamics}

\subsubsection{On SIGMA Method}
Komornicki, A., McIver, J. W., {\em Chem. Phys. Lett.}, 10:303, (1971).\\  
Komornicki, A., McIver, J. W., {\em J. Am. Chem. Soc.}, 94:2625 (1971).
\index{references!SIGMA method}

\subsubsection{On Molecular Orbital Valency}
 ``Valency and Molecular Structure'', Gopinathan, M. S., 
 Siddarth, P., Ravimohan, C., {\em Theor. Chim. Acta },
 70:303 (1986).\index{references!MO valency}

\subsubsection{On Bonds }
 ``Bond Indices and Valency'', Armstrong, D.R., 
  Perkins, P.G., Stewart, J.J.P., 
{\em J. Chem. Soc., Dalton}, 838 (1973).\index{references!bonds}
 For a second, equivalent, description, see also: 
Gopinathan, M. S., and Jug, K., {\em Theor. Chim. Acta}, 63:497 (1983).

    
\subsubsection{On Locating Transition States}
 ``Location of Transition States in Reaction Mechanisms'',
 M.J.S. Dewar, E.F. Healy, J.J.P. Stewart, 
{\em J. Chem. Soc., Faraday Trans. II}, 3:227,
(1984).\index{references!transition state location}

\subsubsection{On Dipole Moments for Ions}
 ``Molecular Quadrupole Moments'', A.D. Buckingham, {\em Quarterly
 Reviews}, 182 (1958 or 1959).\index{dipole moments!of ions}
\index{references!dipole moments of ions}

\subsubsection{On Polymers}
``MNDO Cluster Model Calculations on Organic Polymers'', 
 J.J.P. Stewart, {\em New Polymeric Materials}, 1:53-61 (1987).\\
 ``Calculation of Elastic Moduli using Semiempirical 
 Methods'', H. E. Klei, J.J.P. Stewart, 
{\em Int. J. Quant. Chem.}, 20:529-540 (1986).\index{references!polymers}

%%%%%%%%%%%%%%%%%%%%%%%%%%%%%%%%%%%%%%%%%%%%%%%%%%%%%%%%%%%%%%%%%%%%%

\printindex

\end{document}

