\documentclass[11pt,fleqn]{article}

\usepackage{hyperref}

% package HTML requires Latex2HTML to be installed for html.sty
\usepackage{html}
\newcommand{\doi}[1]{doi:\href{http://dx.doi.org/#1}{#1}}
\begin{htmlonly}
\renewcommand{\href}[2]{\htmladdnormallink{#2}{#1}}
\end{htmlonly}
\hypersetup{colorlinks,
            %citecolor=black,
            %filecolor=black,
            %linkcolor=black,
            %urlcolor=black,
            bookmarksopen=true,
            pdftex}
 
\addtolength{\textwidth}{1.0in}
\addtolength{\oddsidemargin}{-0.5in}
\addtolength{\topmargin}{-0.5in}
\addtolength{\textheight}{1.0in}

\pagestyle{headings}
\pagenumbering{roman}
\begin{document}
\sf
\parindent 0cm
\parskip 1ex
\begin{flushleft}
 
Computing for Science (CFS) Ltd.,\\CCLRC Daresbury Laboratory.\\[0.30in]
{\large Generalised Atomic and Molecular Electronic Structure System }\\[.2in]
\rule{150mm}{3mm}\\
\vspace{.2in}
{\huge G~A~M~E~S~S~-~U~K}\\[.3in]
{\huge USER'S GUIDE~~and}\\[.2in]
{\huge REFERENCE MANUAL}\\[0.2in]
{\huge Version 8.0~~~June 2008}\\ [.2in]
{\large PART 11. FORMATTED OUTPUT from GAMESS-UK}\\
\vspace{.1in}
{\large M.F. Guest, J. Kendrick, J.H. van Lenthe and P. Sherwood}\\[0.2in]
 
Copyright (c) 1993-2008 Computing for Science Ltd.\\[.1in]
This document may be freely reproduced provided that it is reproduced\\
unaltered and in its entirety.\\
\vspace{.2in}
\rule{150mm}{3mm}\\
\end{flushleft}

% 
\tableofcontents
\pagenumbering{arabic}
\newpage


\section[The Punch Directive]{The Punch Directive}

This chapter provides a description of the way formatted output may be
obtained from GAMESS--UK for use by other programs.  Output is
controlled by the Class 1 PUNCH directive followed by associated
keywords, which determine what is to be written out.  The output is
written on FORTRAN unit 58. Under control of the rungamess script, the
punchfile is saved in the job submission directory with the filename 
extension .pun.  Otherwise, it may directed to a specific file in the manner
used for the other fortran streams (See the Machine Specific Sections
later in the manual).

Specification of the PUNCH directive without any keywords is
equivalent to the directive 
{
\footnotesize
\begin{verbatim}
          PUNC GEOM CONN OCCU VECT BASI TYPE LEVE
\end{verbatim} 
}
(where we use the shortest valid abbreviations for the keywords)

\begin{table}
 \caption{\label{table:1}\  Keywords of the PUNCH Directive}
 
 \begin{centering}
 \begin{tabular}{llr}
\\ \hline\hline
  keyword         &     Data &   \\ \cline{1-3}
    TITLE       & title   & \\
    COORDINATES & coordinates (single-point or at the end of optimisation)  & \\
    INITIAL     & coordinates before optimisation  & \\
    OPTIMISE    & coordinates during optimisation   & \\
    CONNECTIVITY & atom connectivity   & \\
    BASIS       & basis set  & \\
    OVERLAP     & overlap matrix  & \\
    VECTORS     & MO vectors   & \\
    OCCUPATIONS & MO occupations   & \\
    EIGENVALUES & scf orbital energies   & \\
    SCFENERGY   & scf energy   & \\
    GVB         & gvb pair coefficients   & \\
    VIBRATIONAL & vibrational frequencies   & \\
    NORMAL      & vibrational normal modes   & \\
    INFRARED    & infra-red intensities & \\
    RAMAN       & raman intensities & \\
    GRID        & grid data from graphics directive   & \\
    GSYM        & symmetry data for grid points & \\ 
    MULLIKEN    & mulliken population analysis   & \\
    LOWDIN      & lowdin population analysis   & \\
    SPIN        & spin densities at the nuclei & \\
    PDC         & potential derived charges & \\
    GRADIENT    & gradients of the energy w.r.t. nuclear positions   & \\
    TRANSFORM   & the rototranslation matrix used to reorient the molecule   & \\
    SECDER      & cartesian second derivatives & \\
    DMA         & distributed multipole analysis & \\
    TYPE        & class of calculation   & \\
    LEVEL       & level of calculation   & \\
    HIGH        & request higher precision format & \\
\hline\hline
 \end{tabular}
 
 \end{centering}
\end{table}

First, we provide an overview of the structure of
the punchfile, and then describe each the arrangement and format
used for each type of data. For jobs with multiple RUNTYPE directives,
the placement of the PUNCH directive is important, and
multiple PUNCH directives may be used, see section \ref{multrun}.

\section[Punch File Structure]{Punch File Structure}

The GAMESS--UK punchfile consists of records which are grouped into blocks, each block
being preceded by a block header. The header serves to identify the block, specify its
length (in records) and
to provide information to be used when the block is read. Within a block,
all records are written with the same FORTRAN format. The formats
used by GAMESS--UK when writing each type of block are described below.
All information in the header takes the form of variable assignments,
each of which has the syntax VAR = DATA where VAR is the name of the 
header variable, and DATA is the associated value. 
Two such variables appear in the header of every block, regardless of the
contents of the block.
These are ``block'' (which is set to a character 
string defining the type of data) and ``records'' (which is set to an integer 
specifying the number of data records in the block). The value of the ``block'' variable
will be used in this chapter to refer to blocks of that particular type, and will
be denoted by a {\bf bold} font. If the format used to write out the data within
the block differs from the default for the given block type ( i.e. that described
in this chapter) the variable ``f\_format'' is set to a character string containing
the FORTRAN format used.
The number of additional variables set will depend on the type of data block,
and where appropriate these are described below.

The block header is usually contained on a single record. In cases where
continuation lines are needed
the continuation character backslash 
($\backslash$) will be  placed at the end of all incomplete lines.

It should be noted that the case of the punchfile will depend on 
the machine specific version of GAMESS--UK in use. Currently, it will
be lower case for all UNIX machines, and uppercase for the
IBM 3090 and Cray implementations.

As a simple example, the following punchfile results from running a default (STO3-21G) basis
set calculation on formaldehyde (see Part 2 of this manual) with the PUNCH 
directive added. The output is in lower case as it was generated on a UNIX machine.

{
\footnotesize
\begin{verbatim}
          block = calculation_type records = 1
          single_point             
          block = calculation_level records = 1
          rhf     
          block=fragment records=0
          block = coordinates records =     4 unit = au
            c             0.0000000      0.0000000      0.9998722
            o             0.0000000      0.0000000     -1.2734689
            h             0.0000000      1.7650653      2.0942591
            h             0.0000000     -1.7650653      2.0942591
          block = connectivity records =    3
              1    2
              1    3
              1    4
          block = basis records =  21
          c          1  s   1   1      172.256000       0.061767
          c          1  s   1   2       25.910900       0.358794
          c          1  s   1   3        5.533350       0.700713
          c          1  s   2   1        3.664980      -0.395897
          c          1  s   2   2        0.770545       1.215840
          c          1  p   3   1        3.664980       0.236460
          c          1  p   3   2        0.770545       0.860619
          c          1  s   4   1        0.195857       1.000000
          c          1  p   5   1        0.195857       1.000000
          o          2  s   1   1      322.037000       0.059239
          o          2  s   1   2       48.430800       0.351500
          o          2  s   1   3       10.420600       0.707658
          o          2  s   2   1        7.402940      -0.404453
          o          2  s   2   2        1.576200       1.221560
          o          2  p   3   1        7.402940       0.244586
          o          2  p   3   2        1.576200       0.853955
          o          2  s   4   1        0.373684       1.000000
          o          2  p   5   1        0.373684       1.000000
          h          3  s   1   1        5.447178       0.156285
          h          3  s   1   2        0.824547       0.904691
          h          3  s   2   1        0.183192       1.000000
          block = occupations records =    22 index =   1
            2.000000
            2.000000
            2.000000
            2.000000
            2.000000
            2.000000
            2.000000
            2.000000
            0.000000
            0.000000
            0.000000
            0.000000
            0.000000
            0.000000
            0.000000
            0.000000
            0.000000
            0.000000
            0.000000
            0.000000
            0.000000
            0.000000
          block = vectors records =    88 index =   1 elements =  22
            -0.000199  -0.000556   0.000000   0.000000   0.002516  -0.018231
             0.000000   0.000000   0.013900  -0.983533  -0.098056   0.000000
             0.000000  -0.004511   0.047908   0.000000   0.000000   0.013258
             0.000801  -0.000450   0.000801  -0.000450
          
          ... etc ...
          
             0.001786  -0.103327   0.000000   0.000000   0.276317  -1.150242
             0.000000   0.000000   0.780772   0.062516  -1.711541   0.000000
             0.000000  -0.253675   2.543902   0.000000   0.000000   0.952243
             0.074248  -0.016513   0.074248  -0.016513
\end{verbatim}
}


\section[Punch Keywords and Output Data Formats]{Punch Keywords and Output Data Formats}

\subsection[TITLE]{TITLE}

This keyword requests a single block of type {\bf title}, containing
a the job title, written as a single record with format (a80).

\subsection[COORDINATES, INITIAL and OPTIMISE]{COORDINATES, INITIAL and OPTIMISE}

These keywords cause cartesian coordinates to be written to the 
punchfile. For a single-point calculation the COORDINATES keyword 
should be used to punch a single data block of type {\bf coordinates}. 

The progress of geometry optimisation or saddle-point calculations
may be monitored using the INITIAL and OPTIMISE keywords. The INITIAL
keyword requests that the starting geometry be punched in a block of
type {\bf initial\_coordinates}. The OPTIMISE keyword requests that a block
of type {\bf update\_coordinates} will be punched at every point of the 
optimisation pathway, including the initial and final points. The header variable
``index''  will be set to the position of the structure in the sequence.
If the COORDINATES keyword is also supplied,
a {\bf coordinates} block will also be punched out, containing the 
geometry of the last point. 

All three types of {\bf coordinates} blocks have the same format:

 \begin{centering}
 \begin{tabular}{ll}
\\ \hline
  item         & format\\ \cline{1-2}
atom label &   2x,a8 \\
x,y,z      &   3(3x,f12.7)\\
\hline
 \end{tabular}
 
 \end{centering}

\subsection[CONNECTIVITY]{CONNECTIVITY}

This keyword requests a single block containing a list
of connected atom pairs. The contact list is calculated using an 
internal table of approximate covalent radii.

Each atom pair is written as a single record with format (2i5)
The coordinates and atom ordering are those used in the 
{\bf coordinates} block. Each pair is listed once only, with the
lower atom number first. The number of pairs is given by the header
variable ``records''.

\subsection[BASIS]{BASIS}
This keyword results in a single data block being punched, which 
contains a definition of the basis set in terms of the contracted
Gaussian primitives. Each line describes a single primitive function, 
with contents and format as shown in the following table.

 \begin{centering}
 \begin{tabular}{ll}
\\ \hline
  item         & format\\ \cline{1-2}
atom tag   &  a8,1x \\
atom index   & i3,1x \\
type (s,p,d or f)  &      a2,1x  \\
shell index    &  i3,1x \\
primitive index  & i3,1x \\
exponent      &  f15.6 \\
coefficient   &  f15.6 \\
\hline
\\
 \end{tabular}

 \end{centering}
The atom index is incremented each time a new atom type is started. 
The shell index, which is set to 1 at the start of each atom type,
is incremented for each shell (The s and p components of
GAUSSIAN-type sp shells are treated as independent shells). Likewise, 
the primitive index is set to 1 at the start of each shell. The exponents
are given in atomic units. The coefficients are used to expand the contraction
in terms of primitives which are themselves normalised - the normalisation factors
are {\em not} included in the coefficients. 

\subsection[OVERLAP]{OVERLAP}

This keyword request the overlap (or one-electron integral) matrix
in the AO basis. A single block of type {\bf overlap} is written.
The AO ordering follows from the order of atoms in
the {\bf coordinates} block, and the ordering of contracted basis functions
on a given atom type follows that written to the {\bf basis} block.
The header variable ``elements'' is set to the number of AO functions (nf), and
the matrix can then be read with a FORTRAN loop structure of the form

{
\footnotesize
\begin{verbatim}
          do 100 j = 1,nf
               read(npun,1000)(q(i,j),i=1,num)
       100  continue
       1000 format(6f11.6)
\end{verbatim}
}

\subsection[VECTORS, EIGENVALUES, OCCUPATIONS and SCFENERGY]{VECTORS, EIGENVALUES, OCCUPATIONS and SCFENERGY}

These keywords request the output of one or more sets of MO vectors and associated 
information about the wavefunction. The four keywords each give rise to one block for each
wavefunction requested, the types being {\bf vectors}, {\bf eigenvalues}, {\bf occupations}
and {\bf scf\_energy} respectively. 
The keywords may optionally be followed by one or more integers, specifying
which sets of vectors are required by the section numbers of the
dumpfile sections on which they are stored. The section selections on each keyword are combined, and
the same list of sections is used for output of all four block types. It is thus not 
possible to specify the sections independently for any of the four blocks, although if any the keywords is
omitted no blocks of that type will be written. The block header variable ``index'' is set to a different
integer value (counting from 1) for each set of vectors punched. 
If the section specification is omitted,
the program will attempt to determine a suitable set (or sets) from the data given on
the ENTER directive, but at present there are a number of run types for which this has not been implemented.

The vectors are output in the AO basis, and should be read from a FORTRAN program using
a loop structure similar to that given above for the {\bf overlap} block. The value of nf 
(the number of AO function and the number of vectors punched) should again be taken 
from the value of the header variable ``elements''. The fastest varying index (i) is the 
MO (vector) index, and j is the AO (basis) index.

The SCFENERGY keyword results in a {\bf scf\_energy} energy block for each vectors section 
requested, containing a single record with the SCF energy written with format (f15.8). 

The EIGENVALUES and OCCUPATIONS keywords both give rise to blocks ({\bf eigenvalues} and 
{\bf occupations}) respectively, containing one record per MO, in format (f10.6). The eigenvalues
are in Hartrees, and the occupations in electrons.

\subsection[GVB]{GVB}

The GVB pair information associated with a set of canonicalised orbitals
may be punched using the GVB keyword. Only one set of GVB pair
information is stored on the dumpfile, that associated with the
second entry on the ENTER directive. This section must be one
of those for which a {\bf vectors}, {\bf scf\_energy}, {\bf eigenvalues} or 
{\bf occupations} block has been requested.

There is one record per gvb pair, arranged as follows

 \begin{centering}
 \begin{tabular}{lc}
\\ \hline
  item         & format\\ \cline{1-2}
orbital indices   &  2i5 \\
pair coefficients &  2f12.6 \\
\hline
\\
 \end{tabular}

 \end{centering}

\subsection[GRID and GSYM]{GRID and GSYM}
The GRID keyword requests output of numerical data arrays constructed using the directives described in 
Part 8 (under control of RUNTYPE ANALYSE, GRAPHICS) and written to the dumpfile.
Both arrays of point coordinates (ie grids without associated data values, generated
by the GDEF directive)  and
arrays of data calculated at specified points (from the CALC directive) may be output.
The GRID keyword must be 
followed by at least one integer value, specifying the dumpfile section 
from which the data is to be taken. Up to 10 grids may be requested.
The GSYM keyword requests that symmetry equivalences between the points also
be computed and written out.

The punchfile will contain a number of data blocks for each grid requested.

The {\bf data} block is empty, and is used signify the beginning of set of data
block.

The {\bf grid\_title} block simply contains one record - the title, as specified on the GRAPHICS
directive.

The {\bf grid\_axes} block contains one record for each axis of a regular 
grid specification. The contents of each of the records are as follows.

 \begin{centering}
 \begin{tabular}{ll}
\\ \hline
  item         & format\\ \cline{1-2}
number of data points along the axis  &  1x,i3 \\
axis ranges & 2f11.6 \\
axis type  &  i2 \\
axis unit  &  a3  \\
axis name  &  1x,a5 \\
\hline
\\
 \end{tabular}

 \end{centering}

In general, the axis range fields hold the minimum and maximum values of the 
parameter corresponding to the axis. In the grids written by GAMESS--UK this is always 
a distance, and the minimum and maximum values are set to 0.0 and the length of the 
plot edge (in aus) respectively.
The axis type is currently always 0 (denoting a regularly spaced axis), 
and axis unit is the string ' au'.

The {\bf grid\_mapping} block serves to define the 3D coordinates of the grid.
There is one record for each axis, containing two coordinate triples. Like {\bf grid\_axes} This block only
appears for regular grids.

 \begin{centering}
 \begin{tabular}{ll}
\\ \hline
  item         & format\\ \cline{1-2}
x,y,z coords of start of axis & 3f10.6\\
x,y,z coords of end of axis & 3f10.6\\
\hline
\\
 \end{tabular}

 \end{centering}


The {\bf grid\_points} block contains the coordinates in 3D space
of each point on an irregular grid. Each record contains the
x, y, and z coordinates of a point on the grid, in  format 3f11.6.
Irregular grids are generated by the GAMESS--UK 3D contouring routines.

Certain irregular grids, such as those generated by the GAMESS--UK contouring routines 
have the property that although the points do not lie regularly in 
3D space, the space they occupy may be divided into 
cells (in general parallelepiped shaped) with either 1 or 0 
points in each. This property arises simply because the 
contouring routines start by dividing space in this way, and
performing an interpolation calculation, where appropriate, in each cell.
The {\bf grid\_indices} block contains one record per grid point,
specifying the cell concerned by 3 integer indices (format 3i5)

The data itself is written out in a {\bf grid\_data} block using format (f11.6). 
It may be read using
FORTRAN code of the following type. np1 and np2 are the numbers of points along the
axes, obtained from the {\bf grid\_axes} block, and nel is
the number of elements associated with each data point (1 = scalar data, 3 = vector etc) obtained
from the value of the header variable ``elements''.
Note that the fastest varying index (apart from the loop over elements) 
is that corresponding to the {\em first} axis.

{
\footnotesize
\begin{verbatim}
          do 100 j = 1,np2
             do 100 i = 1,np1
                read(npun,1000)(q(k,i,j),k=1,nel)
     100 continue
    1000 format(3f11.6)
\end{verbatim}
}

The {\bf grid\_data} block will be empty (``elements''= 0) if the dumpfile
section requested on the punch directive contained a grid definition (from
a GDEF directive) rather than an array of values (from CALC).

The {\bf grid\_symmetry} block will be included if the GSYM keyword is
given. For each point there will be a record containing a single integer 
(format (1x,i8)) corresponding to the index in the list of points of the
first point which is symmetry equivalent to the current point.

\subsection[MULLIKEN and LOWDIN]{MULLIKEN and LOWDIN}
These keywords request that the respective population analyses
(by atomic orbital and  reduced to atoms) and calculated charges
be written out. For UHF wavefunctions, the alpha and beta components
are written out separately. The blocks generated by the MULLIKEN keyword
are as follows, the block type naming should be self-explanatory.

 \begin{centering}
 \begin{tabular}{ll}
\\ \hline\hline
 block type      &     Notes\\ \cline{1-2}
mulliken\_ao\_populations           & \\
mulliken\_atom\_populations        & \\
alpha\_mulliken\_ao\_populations    & UHF only \\
alpha\_mulliken\_atom\_populations  & UHF only \\
beta\_mulliken\_ao\_populations     & UHF only \\
beta\_mulliken\_atom\_populations   & UHF only \\
mulliken\_atomic\_charges         & \\
\hline\hline
\\
 \end{tabular}

 \end{centering}


The data in each block is arranged one number per record, with
format (f10.6). The atom ordering follows that of the {\bf coordinates} block,
while the AO ordering on a given centre follows the arrangement of 
contracted functions given in the {\bf basis} block.

The blocks generated by the LOWDIN directive are precisely analogous with
corresponding names.

\subsection[SPIN]{SPIN}

The spin densities at the nuclei are output in response the SPIN keyword.
A single block {\bf spin\_densities} is written, with one record per atom in 
format (f10.6).

\subsection[PDC]{PDC}
This keyword requests that the results of a potential derived charges
calculation be written to the punchfile. A single data block
({\bf potential\_derived\_charges}) is generated, with one record per atom in 
format f10.6.

\subsection[VIBRATIONAL and NORMAL]{VIBRATIONAL and NORMAL}

These keywords control the output of blocks of types {\bf vibrational\_frequencies}
and {\bf normal\_coordinates} respectively, which contain the results
of the normal coordinate analysis which is performed under
RUNTYPE FORCE or RUNTYPE HESSIAN.  The {\bf vibrational\_frequencies} block
contains the frequencies (in wavenumbers) for each mode with a real frequency.
There is one value per record, format (f16.6). One {\bf normal\_coordinates} block
is written out for each mode with a real frequency, the block variable ``index'' identifies
the corresponding value in the {\bf vibrational\_frequencies} block. Each record contains an
atom symbol, followed by the normal coordinate for that atom, the format (2x,a8,3(f12.7)) 
is the same as that for the spatial coordinates.

\subsection[INFRARED and RAMAN]{INFRARED and RAMAN}

These keywords control the output of block of the types
{\bf infrared\_intensity} and {\bf raman\_intensity} respectively, which 
contain the intensities computed under RUNTYPE HESSIAN, RUNTYPE INFRARED or
RUNTYPE RAMAN. A separate block is written for the intensity of each vibrational
mode with one value per record, format (f16.6).

\subsection[GRADIENT and TRANSFORM]{GRADIENT and TRANSFORM}

These keywords are used when the gradient of the 
energy with respect to the nuclear coordinates is required. GRADIENT requests the 
output of a {\bf gradients} block, each record of which contains the derivative of the
energy with respect to the cartesian coordinates of one atom, format (2x,3f15.7).

Since the molecule may well have been reoriented by GAMESS--UK as part of the point group 
symmetry analysis, these gradients will have to be transformed if they are required to
correspond to the orientation of the molecule as it was input to GAMESS--UK, and account must
also be taken of the reordering of atoms.
The TRANSFORM keyword requests a block of type {\bf tr\_matrix}, containing a rotation and translation
matrices (denoted here as R and T respectively) which may be used to construct the gradient in the
original, input frame.
There are 3 records, record number i containing R(i,1), R(i,2), R(i,3), T(i) in format (2x,4f15.7).
Taking the coordinates {\bf c} in the GAMESS--UK coordinate system (i.e. after symmetry adaption, as found in the
{\bf coordinates} block)
the transformation R{\bf c}-T yields the coordinates of the atom as it was input to GAMESS--UK, and R{\bf g}, 
(where {\bf g} is the gradient from the {\bf gradients} block) is the gradient in the initial coordinate
system. 
Note that the record structure written out in the {\bf tr\_matrix } block is suitable
for inclusion (without the block header) after the ORIENT directive.

The TRAN keyword also results in the punching of a {\bf point\_group} block, which contains
a single record containing a point group symbol and the order of the principle rotation
axis (format (1x,a8,2x,i2) ). Note that the point group symbol matches that printed
out by GAMESS--UK in the listing
(one of c1, cs, ci, cn, s2n, cnh, cnv, dn, nh, dnd, cinfv, dinfh, t, th, td, o, oh, i or ih )
The symbol and axis order are also required as input to the ORIENT directive.

\subsection[SECDER]{SECDER}

  The cartesian second derivative matrix is punched out in a block
named {\bf cartesian\_hessian}. One element is written per line, with
format f15.7. Only the {\em analytic} second derivatives can be punched
in this way.

\subsection[DMA]{DMA}

  The punch DMA option will result in the output of the sites and
magnitudes of the poles from a DMA analysis. It results in a single block 
{\bf dma\_sites}, containing one record per site, of the following structure:

 \begin{centering}
 \begin{tabular}{ll}
\\ \hline
  item         & format\\ \cline{1-2}
site label   &  a8,2x \\
x,y,z coordinates  &  3f10.6  \\
highest pole (lmax) &  i4  \\
relative radius     &  f10.6 \\
\hline 
\\
 \end{tabular}

 \end{centering}

% 1021 format(a8,2x,3f22.14,i4,f10.6)

The poles are then punched out, using one block of type {\bf dma\_pole}
for each pole. The block header defines a variable ``sites'' indicating
the location of the pole in the list of DMA sites. The components of the pole are 
printed, one per line, in the sequence q00, q10, q11c, q11s, q20 etc,
using a format e16.16.


\subsection[TYPE and LEVEL]{TYPE and LEVEL}

{\em The punch facilities controlled by these keywords have not yet been fully
implemented. A brief description is included here for completeness only. We
do not recommend using the output from these directives at the present time} \\

These keywords cause information concerning the mode of calculation to be
provided, based on the RUNTYPE and SCFTYPE specifications.
The TYPE keyword causes a {\bf calculation\_type} block
to be written containing one of the following strings 

{
\footnotesize
\begin{verbatim}
          single_point
          geometry_optimisation
          saddle_point_search
          force_constants
\end{verbatim}
}
The string is left justified in a field of 25 characters.

The LEVEL keyword causes a {\bf calculation\_level} block
to be punched containing a single line. The first field 
will normally be the scf type, with the exception that direct scf is 
classified as rhf and casscf is classified as mcscf.
This string is left justified in a field of 8 characters.
This field may be followed by one of the following, depending
on the run type.

{
\footnotesize
\begin{verbatim}
           plus ci
           plus gf
           plus tda
\end{verbatim}
}

\subsection[HIGH]{HIGH}
 
The default format statements described above have been chosen 
to provide sufficient precision for most applications that need
to analyse the GAMESS--UK wavefunction or geometry, without resulting in
punchfiles of unreasonable length.  However, sometimes
data in the punchfile is needed to higher accuracy. For some data
items (currently the {\bf coordinates, grid\_mapping, grid\_points, 
grid\_data gradients normal\_coordinates, vibrational\_frequency,
tr\_matrix, potential\_derived\_charges, mulliken.. lowdin.. 
and spin\_densities} blocks) higher precision output is 
available. The HIGH keyword should then appear on the PUNCH directive,
before the keyword requesting the data that is required to higher.
The effect of the HIGH keyword does not carry over to subsequent
PUNCH directives. The following example file might be used if the
coordinates were required to higher precision.

{
\footnotesize
\begin{verbatim}
          TITLE
          SCF, PUNCH COORDINATES
          PUNCH HIGH COOR
          ZMAT ANGS
          O
          H 1 1.0
          H 1 1.1 2 109.0
          END             
          RUNTYPE SCF
          ENTER
\end{verbatim}
}
The resultant punchfile is given below.
{
\footnotesize
\begin{verbatim}
    block = fragment records = 0
    block = coordinates records =     3 unit = au \
     f_format = "(2x,a8,3(3x,f22.14))"
      o                0.00000000000000         0.00000000000000        -0.21947397179724
      h                0.00000000000001         1.53845580221546         0.87789588718897
      h                0.00000000000000        -1.53845580221546         0.87789588718897
\end{verbatim}
}
Note that if the punchfile is to be read by a FORTRAN program, with a fixed
format read, the value of the ``f\_format'' variable on the header must
be extracted and passed to the FORTRAN read. FORTRAN- and C-callable routines
to simplify this operation are available from the authors.

The line length of high format punched output will, in general, exceed 80 characters.


\section[Punchfiles from Multiple Runtype Jobs]{Punchfiles from Multiple Runtype Jobs}
\label{multrun}

When an input file contains multiple RUNTYPE directives, care should
be taken with the placing of the PUNCH directive. An error will occur
if you request data that is not generated during current job step.  It
is sometimes possible to punch data from an earlier job step, but only
if the data is resident on the dumpfile (e.g. the vectors).

A simple example is shown below, where an SCF calculation is followed
by two wavefunction analysis steps: a graphical analysis calculation
and a distributed multiple analysis. Here, none of the SCF results are
needed, so it is possible to omit the PUNCH directive from the first
job step


\begin{verbatim}
title
h2o 6-31g**
adapt off
accuracy 19 8
zmat angs
o
x 1 1.0
h 1 ho 2 hox
h 1 ho 2 hox 3 180.0
variables
ho 0.967
hox 53.8
end
basis 6-31g**
enter 1
runtype analyse
punch coordinates grid 100
graphics
gdef 
type 2d
points 50
calc
type density
section 100
vectors 1
enter 1
runtype analyse
punch dma
dma
vectors 1
enter 1
\end{verbatim}

\clearpage

\end{document}
