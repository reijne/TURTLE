\documentclass[11pt,fleqn]{article}

\usepackage{hyperref}

% package HTML requires Latex2HTML to be installed for html.sty
\usepackage{html}
\newcommand{\doi}[1]{doi:\href{http://dx.doi.org/#1}{#1}}
\begin{htmlonly}
\renewcommand{\href}[2]{\htmladdnormallink{#2}{#1}}
\end{htmlonly}
\hypersetup{colorlinks,
            %citecolor=black,
            %filecolor=black,
            %linkcolor=black,
            %urlcolor=black,
            bookmarksopen=true,
            pdftex}

\addtolength{\textwidth}{1.0in}
\addtolength{\oddsidemargin}{-0.5in}
\addtolength{\topmargin}{-0.5in}
\addtolength{\textheight}{1.0in}
\newcommand{\water}{\mbox{H$_{2}$O}}
\newcommand{\waterp}{\mbox{H$_{2}$O$^{+}$}}
\newcommand{\formaldehyde}{\mbox{H$_{2}$CO}}

\pagestyle{headings}
\pagenumbering{roman}
\begin{document}
\sf
\parindent 0cm
\parskip 1ex
%jmht - number paragraphs too
\setcounter{secnumdepth}{4}
\begin{flushleft}
Computing for Science (CFS) Ltd.,\\CCLRC Daresbury Laboratory.\\[0.30in]
{\large Generalised Atomic and Molecular Electronic Structure System }\\[.2in]
\rule{150mm}{3mm}\\
\vspace{.2in}
{\huge G~A~M~E~S~S~-~U~K}\\[.3in]
{\huge USER'S GUIDE~~and}\\[.2in]
{\huge REFERENCE MANUAL}\\[0.2in]
{\huge Version 8.0~~~June 2008}\\ [.2in]
{\large PART 4. DATA INPUT - CLASS 2 Directives}\\
\vspace{.1in}
{\large M.F. Guest, H.J.J. van Dam, J. Kendrick, J.H. van Lenthe,
  P. Sherwood and G. D. Fletcher}\\[0.2in]

Copyright (c) 1993-2008 Computing for Science Ltd.\\[.1in]
This document may be freely reproduced provided that it is reproduced\\
unaltered and in its entirety.\\
\vspace{.2in}
\rule{150mm}{3mm}\\
\end{flushleft}

\tableofcontents
 
\newpage

\pagenumbering{arabic}

\section[Introduction]{Introduction}

The {\em Class 2} directives define the nature of the computation
in hand, and are used to input the details necessary to enable
this computation to proceed. In this chapter we concentrate on those
directives required to characterise;
\begin{enumerate}
\item SCF,  M\o ller Plesset, MCSCF, MASSCF and CASSCF calculations;
\item Density Functional Theory (DFT) calculations;
\item Geometry optimisation, transition state calculations and
force constant calculations. 
\end{enumerate}
Data input required in performing Direct-CI and Table-CI calculations,
and in using the various wavefunction analysis tools are considered
below in Parts 5,6 and 7 respectively.

\section[RUNTYPE]{RUNTYPE}

The RUNTYPE directive is used to define the type of computation to be 
performed in the present run of the program, and consists of a single
a line with the first two data fields read to variables 
TEXT, TYPE using format (2A).
\begin{itemize}
\item TEXT should be set to the character string RUNTYPE
\item TYPE should be set to a character string defining the 
type of computation. Valid strings are shown in the table below.
\end{itemize}
Subsequent data fields are a function of the TYPE setting, and may
be used to further characterise the nature of the computation. We 
consider the format for each TYPE specification in the notes below.

\begin{itemize}
\item  In the absence of a RUNTYPE directive, the default of a single point
SCF calculation is assumed.
\item  When restoring a Hessian Matrix for use in OPTIMISE and
SADDLE computations, the source of this Hessian must
be specified on the RUNTYPE directive. This may take two
forms;
\begin{enumerate}
\item When the initial Hessian has been computed under control
of the RUNTYPE HESSIAN directive, the additional keyword FCM
should be specified on the OPTIMISE or SADDLE data line, thus

{
\footnotesize
\begin{verbatim}
          RUNTYPE SADDLE FCM
\end{verbatim}
}
\item When the Hessian is to be restored from some
smaller basis set calculation, i.e. from a Dumpfile separate to
that of the present calculation, this `foreign Dumpfile'  must be
specified on the RUNTYPE directive, so the data line

{
\footnotesize
\begin{verbatim}
          RUNTYPE SADDLE ED4 150
\end{verbatim}
}

requests the initial Hessian to be restored from the Dumpfile
commencing at block 150 on the data set assigned with the LFN ED4.
\end{enumerate}
\end{itemize}

\begin{tabular}{ll}
\\ \hline
RUNTYPE INTEGRAL & single point 1e- and 2e-Integral evaluation \\
RUNTYPE SCF      & single point Integrals + SCF calculation \\
RUNTYPE OPTIMISE & search for a local minimum on the potential energy surface \\
                 & using an internal-coordinate quasi-Newton rank-2 update \\
                 & method.\\
RUNTYPE OPTXYZ & search for a local minimum on the potential energy surface \\
                 & using a cartesian-based quasi-Newton update method.\\
RUNTYPE SADDLE   & search for a saddle point on the potential energy surface, \\
                 & using in default the `trust region' method \cite{cerjan}.\\
RUNTYPE FORCE    & numerical force constant calculation at an \\
                 & equilibrium geometry.\\
RUNTYPE HESSIAN  & analytical force constant calculation at an \\
                 & equilibrium geometry.\\
RUNTYPE POLARISABILITY & Polarisability calculation \\
RUNTYPE HYPER    & Hyperpolarisability calculation \\
RUNTYPE MAGNET   & Magnetisability calculation \\
RUNTYPE RAMAN    & Calculation of Raman Intensities\\
RUNTYPE INFRARED & Calculation of IR intensities \\
RUNTYPE TRANSFORM  & single point Integrals, SCF and 4-index transformation  \\
RUNTYPE CI       & single point Integrals, SCF, transformation and CI \\
                 & calculation \\
RUNTYPE GF       & single point Integrals, SCF, transformation and Green's  \\
                 & function OVGF  calculation \\
RUNTYPE TDA      & single point Integrals, SCF, transformation and Green's  \\
                 & function 2ph-TDA  calculation \\
RUNTYPE RESPONSE & single point Integrals, SCF, transformation and Response  \\
                 & function (RPA, TDA or MCLR)  calculation \\
RUNTYPE ANALYSE  & analyse a nominated set of eigenvectors, either by\\
                 & computing 1-electron properties, localised orbitals, or by \\
                 & performing DMA, Mulliken or graphical analysis.\\ \hline
\hline
\end{tabular}

\subsection[Notes on RUNTYPE Specification]{Notes on RUNTYPE Specification}

The simplest RUNTYPE is either that requesting just an SCF
calculation (TYPE=SCF) or some analysis of a pre-computed 
wavefunction (TYPE=ANALYSE). All other TYPEs comprise multiple tasks,
each of which could itself be controlled by its own TYPE
specification. Consider, for example, Direct-CI calculations 
performed under TYPE=CI specification; this of course involves
three separate tasks, namely
\begin{itemize}
\item the SCF computation
\item the integral transformation
\item the Direct-CI calculation
\end{itemize}
The TYPE=CI specification implies the execution of all {\em three}
steps. Performing the calculation under such control would
involve the straightforward specification

{
\footnotesize
\begin{verbatim}
          RESTART CI
          ..
          ..
          RUNTYPE CI
          ..
\end{verbatim}
}
in any restart jobs. Consider now performing the same calculation
in a sequence of steps. The SCF computation might, say, be completed
as the first step, under control of RUNTYPE SCF specification. The
second step, the integral transformation, might involve the
directive sequence

{
\footnotesize
\begin{verbatim}
          RESTART NEW
          ..
          BYPASS SCF
          ..
          RUNTYPE TRANSFORM
          ..
\end{verbatim}
}
assuming of course that the two-electron integral file in the
SCF step had been generated in the appropriate format. Note that
since the TRANSFORM specification implies both SCF and integral
transformation, we need BYPASS the SCF step. Finally, having
generated the transformed integrals. the third step would
involve the data specification

{
\footnotesize
\begin{verbatim}
          RESTART NEW
          ..
          BYPASS TRANSFORM
          ..
          RUNTYPE CI
          ..
\end{verbatim}
}
This type of breakdown is typical of that employed when
\begin{itemize}
\item performing many different CI calculations
based on the same set of transformed integrals
\item having to analyse the SCF calculation to provide
the necessary data specification for the subsequent CI
\end{itemize}

Strictly speaking, the TYPE=SCF specification itself implies of course
the execution of two separate steps, but it is unusual to invoke
this functionality. The SCF job above could be split into two
jobs, the first involving just integral evaluation, through the
specification

{
\footnotesize
\begin{verbatim}
          RUNTYPE INTEGRAL
\end{verbatim}
}
and the second by the directive sequence
{
\footnotesize
\begin{verbatim}
          RESTART NEW
          ..
          BYPASS 
          ..
          RUNTYPE SCF
          ..
\end{verbatim}
}
where the BYPASS directive avoids regeneration of the integral list.


A similar sequence to the CI example above 
is often required when carrying out both
OVGF and TDA Green's function calculations of ionization energies.
This would normally involve performing an initial SCF 
calculation, under control of RUNTYPE SCF, followed by the
Green's function calculation, thus

{
\footnotesize
\begin{verbatim}
          RESTART NEW
          ..
          BYPASS SCF
          ..
          RUNTYPE GF
          ..
\end{verbatim}
}
where the integral transformation and OVGF computation of the
ionization energies are performed in the same job.

Finally consider RUNTYPE specification when performing geometry or
transition state calculations. It is of course possible in the startup
job to present a data sequence of the form

{
\footnotesize
\begin{verbatim}
          ..
          ..
          RUNTYPE OPTIMIZE
          ENTER
\end{verbatim}
}
If however the SCF converges to, say, an excited state
on the first point, any subsequent computation will be wasted. It
is normally better practice to perform the initial SCF under
RUNTYPE SCF control, then initiate the optimization with a sequence
such as

{
\footnotesize
\begin{verbatim}
          RESTART NEW
          ..
          BYPASS SCF
          ..
          RUNTYPE OPTIMIZE
          ENTER
          ..
\end{verbatim}
}
Subsequent restarts would be carried out with the sequence
{
\footnotesize
\begin{verbatim}
          RESTART OPTIMIZE
          ..
          ..
          RUNTYPE OPTIMIZE
          ENTER
          ..
\end{verbatim}
}
Although perhaps an obvious point, note the removal of the BYPASS
directive -- failure to remove this will almost certainly lead to an
erroneous  energy and/or gradients in the optimisation, with predictable
consequences on the convergence of the geometry optimization.

\section[SCFTYPE]{SCFTYPE}

The SCFTYPE directive specifies the category of self-consistent field
wavefunction to be used in conducting the task nominated by the RUNTYPE
directive.  Valid data lines include the following


\begin{tabular}{ll}
\\ \hline
SCFTYPE RHF         &  perform a restricted Hartree-Fock calculation \\
SCFTYPE DIRECT RHF  &  perform a restricted Hartree-Fock Direct-SCF calculation \\
SCFTYPE UHF         &  perform an unrestricted Hartree-Fock calculation \\
SCFTYPE DIRECT UHF  &  perform an unrestricted Direct-SCF calculation \\
SCFTYPE GVB {\em n} &   perform a GVB-n/PP calculation i.e. involving {\em n}  GVB pairs. \\
SCFTYPE DIRECT GVB {\em n} & perform a direct-GVB-n/PP calculation \\
SCFTYPE MP2         & perform a second-order M\ oller Plesset MP2 calculation \\
SCFTYPE DIRECT MP2  &  perform a M\ oller Plesset Direct-MP2 calculation \\
SCFTYPE MP3         & perform a third-order M\ oller Plesset MP3 calculation \\
SCFTYPE CASSCF      & perform a complete active space SCF calculation \\
SCFTYPE MCSCF       & perform a second-order  MCSCF calculation \\
SCFTYPE MASSCF      & perform an ORMAS-MCSCF calculation \\ \hline
\end{tabular}


The following points should be noted;
\begin{itemize}

\item  in the absence of a SCFTYPE directive, a restricted Hartree-Fock
calculation is assumed for both closed- and open-shell systems.
\item  performing RHF calculations on open-shell systems will assume
the appropriate high-spin configuration in default. If this is not the
required configuration, then the OPEN directive must also be specified
to define the required orbital occupancies.
\item  when performing CASSCF calculations, the CONFIG directive
must be specified.
\item  when performing MCSCF calculations, the MCSCF and ORBITAL
directives must be specified.
\item  when performing MASSCF calculations, a MASSCF section must be
  present in the input that specifies values for at least the following
  directives: NCORE, NACT and NELS.
\item  the significance of the sections nominated under control of
the VECTORS and ENTER directives is a function of SCFTYPE.
\item  when restarting a task involving GVB wavefunctions, the RESTORE
parameter must be specified on the SCFTYPE directive. Thus the data
line

{
\footnotesize
\begin{verbatim}
          SCFTYPE GVB 1 RESTORE
\end{verbatim}
}
signifies a GVB-1/PP calculation, with pair coefficients restored from
the Dumpfile.
\item  the format of the LEVEL directive is a function of SCFTYPE.
\end{itemize}

\section[Directives defining the wavefunction]{Directives defining the wavefunction}

\subsection[SCF Wavefunctions: The OPEN Directive]{SCF Wavefunctions: The OPEN Directive}

The default electronic configuration in open-shell RHF calculations
corresponds to the high-spin configuration; in such cases the
OPEN directive is not required. If this default does not apply
then the OPEN directive must be used to define the electronic configuration
and hence the energy expression in both open-shell RHF and open-shell
GVB calculations. In performing such calculations, the User must define
the shell structure, where orbitals which can have the same Fock
operator are said to belong to the same shell \cite{bobrow}.
The OPEN directive is used to define
\begin{itemize}
\item  the number of orbitals in each open shell
\item  the number of electrons in each open shell
\end{itemize}
 The general syntax of the directive is
{
\footnotesize
\begin{verbatim}
           OPEN  NO1 NE1  NO2 NE2  .......  STATE
\end{verbatim}
}
where (NO1,NE1) correspond to the number of orbitals and electrons
respectively in the first open shell, (NO2,NE2) to the number of
orbitals and electrons in the second open shell, and so on until
all open shells have been specified.
 As outlined in Section 3.II, certain assumptions are made
concerning the ordering of the input orbitals within RHF and GVB
modules, and this ordering ties in with the OPEN definition.
Specifically, it is assumed that the orbitals are ordered thus:

\begin{itemize}
\item  all doubly occupied orbitals occur first in the list, the
complete set defining Shell 1, the closed-shell.
\item  following the closed-shell manifold comes the open-shells, with
the number of orbitals and electrons in each open-shell defined
by the OPEN directive.
\item  the GVB pair orbitals.
\end{itemize}

In many cases the specification of the number of component orbitals and
electrons associated with each open shell provides, together with the
MULT directive, a unique definition of the associated electronic
state.  When this is not the case, and ambiguity remains in the
definition, the character string STATE may be used to nominate the
required state.

A full list of the possible STATEs available is shown 
in Table~\ref{table:1}.
below. Note they include all those possible under the SERHF
and GRHF options of ATMOL3 \cite{moncrieff}.
The STATE parameter is not required when no ambiguity
in the required energy expression occurs.\\

{\bf Example}\\

When performing an open shell SERHF calculation on
the $^{4}\Sigma^{-}$ state of NH$^{+}$, 
characterised by the configuration
\begin{equation}
  1\sigma^2 2\sigma^2 3\sigma^1 1\pi_x 1\pi_y
\end{equation}
The data line
{
\footnotesize
\begin{verbatim}
          OPEN 1 1 2 2
\end{verbatim}
}
would be required, together with a MULT 4 specification, 
with the input orbitals ordered thus:

\begin{centering}
\begin {tabular}{llllll}
 M.O.     &  1 & 2 & 3 & 4 & 5 \\
 Symmetry  & 1$\sigma$ & 2$\sigma$ & 3$\sigma$ & 1$\pi$x & 1$\pi$y\\
\end{tabular}

\end{centering}
% \vspace{0.15in}

where shell 1 comprises the closed-shell orbitals (1$\sigma$ , 2$\sigma$), shell 2
the singly occupied 3$\sigma$ orbital, and shell 3 the degenerate 1$\pi$ orbital.
Calculations on the corresponding $^{2}\Delta$  state would require the
directives

{
\footnotesize
\begin{verbatim}
          MULT 2
\end{verbatim}
}
and
{
\footnotesize
\begin{verbatim}
          OPEN 1 1 2 2 DELTA
\end{verbatim}
}

 
\begin{table}
 \caption{\label{table:1}\ Energy Expressions Available and
   the OPEN Directive}
 
 \begin{centering}
 \begin{tabular}{lllllllr}
 \\ \hline\hline
Configuration &  State   & \multicolumn{6}{c}{Data Specification}\\
         \cline{3-8}
   & & \multicolumn{4}{c}{OPEN} & STATE  & MULT   \\ \cline{3-8}

 $\delta^{1}$   &   $^{2}\Delta$   &    2 & 1  & &  & &   2\\
 $\pi^{1}$      &  $^{2}\Pi$  \\
 $e^{1}$        &  $^{2}E$    \\
\\ 
 $\delta^{2}$   & $^{3}\Sigma^{-}$  &  2 & 2   & & & & 3 \\
 $\pi^{2}$   & $^{3}\Sigma^{-}$  \\
 $e^{2}$   & $^{3}A$  \\
\\
 $\delta^{2}$   & $^{1}\Gamma$   &   2 & 2   & & &   GAMMA &  1 \\
 $\pi^{2}$   & $^{1}\Delta$   &   2 & 2   & & &   DELTA &  1 \\
\\
 $\delta^{2}$   & $^{1}\Sigma^{+}$   &   2 & 2   & & &   SIGP &  1 \\
 $\pi^{2}$   & $^{1}\Sigma^{+}$      &     &     & & &   SIGP  \\
 $e^{2}$   & $^{1}A$                 &     &     & & &  A  \\
\\
 $\delta^{3}$   & $^{2}\Delta$       &   2 & 3   & & &   &  2 \\
 $\pi^{3}$   & $^{2}\Pi$   \\
 $e^{3}$   & $^{2}E$   \\
\\
 $p^{1}$   & $^{2}P$                 &   3 & 1   & & &   &  2 \\
 $t^{1}$   & $^{2}T$   \\
\\
 $p^{2}$   & $^{3}P$                 &   3 & 2   & & &   &  3 \\
 $t^{2}$   & $^{3}T$   \\
\\
 $p^{2}$   & $^{1}D$                 &   3 & 2   & & & D &  1 \\
\\
 $p^{2}$   & $^{1}S$                 &   3 & 2   & & & S &  1 \\
 $t^{2}$   & $^{1}A$                 &   3 & 2   & & & A &  1 \\
\\
 $p^{3}$   & $^{4}S$                 &   3 & 3   & & &  &  4 \\
 $t^{3}$   & $^{4}A$                 &   3 & 3   & & &  &  4\\
\\
 $p^{3}$   & $^{2}D$                 &   3 & 3   & & & D &  2 \\
 $t^{3}$   & $^{2}P$                 &   3 & 3   & & & P &  2\\
\\
 $p^{4}$   & $^{3}P$                 &   3 & 4   & & &   &  3 \\
 $t^{4}$   & $^{3}T$                 &   3 & 4   & & &   &  3\\
\\ \hline \hline
\end{tabular}

\end{centering}
\end{table}
\clearpage
\begin{table}
 \begin{centering}
 \begin{tabular}{lllllllr}
 \\ \hline \hline
Configuration &  State   & \multicolumn{6}{c}{Data Specification}\\
         \cline{3-8}
   & & \multicolumn{4}{c}{OPEN} & STATE  & MULT   \\ \cline{3-8}
 $p^{4}$   & $^{1}D$   &   3 & 4 & &   & D &  1 \\
\\
 $p^{4}$   & $^{1}S$   &   3 & 4 & &   & S &  1 \\
 $t^{4}$   & $^{1}A$   &   3 & 4 & &   & A &  1 \\
\\
 $p^{5}$   & $^{2}P$   &   3 & 5 & &   &  &  2 \\
 $t^{5}$   & $^{2}T$   &   3 & 5 & &   &  &  2 \\
\\
 $\sigma^{1}\pi^{1}$   & $^{1}\Pi$   &  1 &  1 &  2 &  1  &  &  1 \\
\\
 $\sigma^{1}\pi^{1}$   & $^{3}\Pi$   &  1 &  1 &  2 &  1  &  &  3 \\
\\
 $\sigma^{1}\pi^{2}$   & $^{4}\Sigma^{-}$   & 1 &  1 &  2 &  2  &  &  4 \\
\\
 $\sigma^{1}\pi^{2}$   & $^{2}\Sigma^{+}$   & 1  & 1  & 2  & 2  & SIGP  &  2 \\
 $\sigma^{1}\pi^{2}$   & $^{2}\Sigma^{-}$   &  1  & 1 & 2  & 2  & SIGM  &  2 \\
 $\sigma^{1}\pi^{2}$   & $^{2}\Delta$   &  1 & 1 & 2 & 2  & DELTA  &  2 \\
\\
 $\sigma^{1}\pi^{3}$   & $^{1}\Pi$   &  1 & 1 & 2 & 3  &   &  1 \\
 $\sigma^{1}\pi^{3}$   & $^{3}\Pi$   &  1 & 1 & 2 & 3  &   &  3 \\
\\
 $\pi^{1}\pi^{1}$   & $^{3}\Sigma^{+}$   &  2 & 1 & 2 & 1 & SIGP  &  3 \\
 $\pi^{1}\pi^{1}$   & $^{3}\Sigma^{-}$   &  2 & 1 & 2 & 1 & SIGM  &  3 \\
 $\pi^{1}\pi^{1}$   & $^{3}\Delta$   &  2 & 1 & 2 & 1 & DELTA  &  3 \\
\\
 $\pi^{1}\pi^{1}$   & $^{1}\Sigma^{+}$   &  2 & 1 & 2 & 1 & SIGP  &  1 \\
 $\pi^{1}\pi^{1}$   & $^{1}\Sigma^{-}$   &  2 & 1 & 2 & 1 & SIGM  &  1 \\
 $\pi^{1}\pi^{1}$   & $^{1}\Delta$   &  2 & 1 & 2 & 1 & DELTA  &  1 \\
\\
 $\pi^{3}\pi^{1}$   & $^{3}\Sigma^{+}$   &  2 & 3 & 2 & 1 & SIGP  &  3 \\
 $\pi^{3}\pi^{1}$   & $^{3}\Sigma^{-}$   &  2 & 3 & 2 & 1 & SIGM  &  3 \\
 $\pi^{3}\pi^{1}$   & $^{3}\Delta$   &  2 & 3 & 2 & 1 & DELTA  &  3 \\
\\
 $\pi^{3}\pi^{1}$   & $^{1}\Sigma^{+}$   &  2 & 3 & 2 & 1 & SIGP  &  1 \\
 $\pi^{3}\pi^{1}$   & $^{1}\Sigma^{-}$   &  2 & 3 & 2 & 1 & SIGM  &  1 \\
 $\pi^{3}\pi^{1}$   & $^{1}\Delta$   &  2 & 3 & 2 & 1 & DELTA  &  1 \\
\\
 $\pi^{3}\pi^{3}$   & $^{3}\Sigma^{+}$   &  2 & 3 & 2 & 3 & SIGP  &  3 \\
 $\pi^{3}\pi^{3}$   & $^{3}\Sigma^{-}$   &  2 & 3 & 2 & 3 & SIGM  &  3 \\
 $\pi^{3}\pi^{3}$   & $^{3}\Delta$   &  2 & 3 & 2 & 3 & DELTA  &  3 \\
\\
 $\pi^{3}\pi^{3}$   & $^{1}\Sigma^{+}$   &  2 & 3 & 2 & 3 & SIGP  &  1 \\
 $\pi^{3}\pi^{3}$   & $^{1}\Sigma^{-}$   &  2 & 3 & 2 & 3 & SIGM  &  1 \\
 $\pi^{3}\pi^{3}$   & $^{1}\Delta$   &  2 & 3 & 2 & 3 & DELTA  &  1 \\ \hline\hline
\end{tabular}
 
\end{centering}
 
\end{table}
 
\clearpage

\subsection[GVB Wavefunctions]{GVB Wavefunctions}

While there are no additional directives required in characterising
a GVB wavefunction, we elaborate below on certain features
of such calculations, presenting a somewhat more complex example
than that given previously (section 2.6).
In particular, we consider the impact of using a set of localised
orbitals (see Part 8) to initiate the GVB calculation.

The following points on performing GVB pair calculations
should again be noted.
\begin{enumerate}
\item Remember that in the general case of a GVB \cite{bobrow}
calculation on an open-shell system,
comprising {\bf m} doubly-occupied orbitals, {\bf n} open shell orbitals
and {\bf 2p} GVB orbitals ( that is p GVB-pairs)
the program expects the trial vectors to be organised thus:

{
\footnotesize
\begin{verbatim}
     orbitals   1 ->  m          doubly occupied

              m+1 ->  m+n        open shell orbitals

            m+n+1 ->  m+n+2      the first GVB pair, with the strongly
                             occupied MO preceding the weakly
                             occupied MO
                  .
                  .
                  .

        m+n+2p-1 -> m+n+2p      the component orbitals of the
                            p-th GVB pair
\end{verbatim}
}

It is the users responsibly to ensure, through use of the
SWAP directive, that the input orbitals are so arranged.

\item  In default GAMESS--UK will
automatically, based on the z-matrix geometry specification, deduce
the molecular point group and hence generate and retain only the
unique integrals required in the process of constructing a
`skeletonised' Fock matrix  \cite{dupuis}.
Such a symmetry-truncated integral list
is, however, {\em NOT}  usable in pair-GVB calculations, and
{\em considerable caution}  should be exercised when considering use
of an integral file generated in an earlier SCF run  directly in
a subsequent pair-GVB calculation under control of the BYPASS directive.
The way to proceed in such cases has been outlined in section 2.6.

\end{enumerate}

Assume we wish to perform a 4-pair GVB/PP calculation on \formaldehyde,
treating the C-H bonds and two C-O orbitals within the
perfect pairing approximation. In such cases little progress is
possible using the set of SCF MOs directly in the GVB; it will be
necessary to generate a trial set of GVB vectors based on
a  localised orbital set, using the VECTORS option NOGEN
to generate the set of secondary GVB pair orbitals based on the
set of LMO. Thus the sequence of calculations required is
\begin{itemize}
\item perform the closed shell SCF calculation;
\item localise the set of valence SCF orbitals;
\item perform the GVB calculation using the set of LMO
input under control of the VECTORS option NOGEN.
\end{itemize}
First, we show the data sequences for performing the initial
closed shell calculation, and the subsequent Boys localisation
in which the LOCAL directive is used to exclude the inner
shell and virtual orbitals from the unitary transformation.
Note the SUPER~OFF~NOSYM specification in the closed-shell
calculation, enabling subsequent use of the BYPASS directive
in the GVB calculation itself.\\

{\bf Closed-shell SCF}
{
\footnotesize
\begin{verbatim}
          TITLE
          H2CO - 3-21G  CLOSED SHELL SCF  -  
          SUPER OFF NOSYM
          ZMATRIX ANGSTROM
          C
          O 1 1.203
          H 1 1.099 2 121.8
          H 1 1.099 2 121.8 3 180.0
          END
          ENTER
\end{verbatim}
}

{\bf Localised Orbital Analysis}
{
\footnotesize
\begin{verbatim}
          RESTART NEW
          TITLE
          H2CO - 3-21G DEFAULT BASIS - VALENCE LMOS
          ZMATRIX ANGSTROM
          C
          O 1 1.203
          H 1 1.099 2 121.8
          H 1 1.099 2 121.8 3 180.0
          END
          RUNTYPE ANALYSE
          LOCAL
          3 TO 8 END
          ENTER 2
\end{verbatim}
}

An examination of the closed-shell SCF-MOs and localised MOs
reveals the  ordering shown below:
 
 \begin{centering}
 \begin{tabular}{llll}
\\ \hline\hline
  MO Sequence  &   Symmetry &   LMO Sequence & Type \\ 
  Number &    & Number  &   \\ \hline
      1   &      1a$_{1}$    &     1   &      1a$_{1}$ \\
      2   &      2a$_{1}$    &     2   &      2a$_{1}$\\
      3   &      3a$_{1}$    &     3   &     C-O$_{1}$\\
      4   &      4a$_{1}$    &     4   &     C-H$_{1}$\\
      5   &      1b$_{2}$    &     5   &     C-H$_{2}$\\
      6   &      5a$_{1}$    &     6   &      lp$_{1}$O \\
      7   &      1b$_{1}$    &     7   &     C-O$_{1}$\\
      8   &      2b$_{2}$    &     8   &      lp$_{2}$O \\ 
\hline\hline
 \end{tabular}
 
 \end{centering}
\vspace{0.15in}
The following points should be noted regarding the
data sequence shown below for performing  the GVB/PP calculation:
\begin{itemize}
\item Note the form of the SCFTYPE
directive - the integer specified after the GVB
keyword indicates the number of GVB pairs - in the present
case, just 4.  
\item Taking the localised orbitals as the starting point,
the trial vectors for the GVB calculation
are obtained through use of the NOGEN directive, where the
specified integer identifies the Dumpfile section containing the
input set of localised orbitals. 
\item NOGEN assumes that the orbitals to be correlated are
positioned at the {\em top} of the doubly occupied orbital input set
(see section 4.8.1).
As only a subset of four of the LMOs are to be correlated, we must
ensure that the corresponding orbitals occupy the top 
four positions in the occupied set (with sequence
numbers 5-8), using the SWAP directive
to reposition the two oxygen lone pair LMOs below the
two C-O and C-H bonds (i.e., with sequence numbers 3 and 4).
\item Note the presence of the ADAPT~OFF data line. This is
now required when using the set of orbitals from the LMO
analysis, where the process of symmetry adaptation is
automatically suppressed.
\end{itemize}

{\bf GVB/PP Data}
{
\footnotesize
\begin{verbatim}
          RESTART NEW
          TITLE
          H2CO - 3-21G  4PAIR GVB
          BYPASS
          SUPER OFF NOSYM
          ADAPT OFF
          ZMATRIX ANGSTROM
          C
          O 1 1.203
          H 1 1.099 2 121.8
          H 1 1.099 2 121.8 3 180.0
          END
          SCFTYPE GVB 4
          VECTORS NOGEN 2
          SWAP
          3 6\4 8\END
          ENTER 3 4
\end{verbatim}
}


\subsection[CASSCF wavefunctions: The CONFIG Directive]{CASSCF wavefunctions: The CONFIG Directive}

{\bf CONFIG}\\

The CONFIG directive must be presented in a CASSCF calculation, and acts

\begin{enumerate}
\item  to define the active space in the  calculation, by partitioning
the orbitals into a primary and secondary set  \cite{roos}.
\item  to specify an initial reference configuration (for example
the associated SCF configuration) that will be employed in
generating the complete CI space and the associated loop-formulae
tape.
\end{enumerate}

CONFIG data comprises a sequence of orbital classification lines
in which each orbital in the primary space is classified
by type, with the following orbital TAGs used in this
classification:


\begin{itemize}
\item  FZC - frozen core orbital i.e. an orbital which will remain
doubly occupied in all configurations.

\item DOC - doubly occupied i.e. an orbital which is doubly occupied in
the reference configuration, and which  will be permitted
variable occupancy in the CASSCF treatment.

\item  ALP - an unpaired orbital i.e. an orbital which is singly occupied
in the reference configuration, and which will be permitted
variable occupancy in the MCSCF treatment.

\item AOS, BOS - those singly occupied orbitals in the reference
configuration belonging to non-identical singlet-coupled pairs.
Again such orbitals will be permitted variable occupancy.

\item UOC - formally unoccupied orbitals, corresponding to SCF virtual
MOs, which will be permitted variable occupancy in the MCSCF.
\end{itemize}


Each orbital definition line comprises an orbital TAG in the
first data field, followed by the sequence numbers of the
input orbitals (as restored under control of the VECTORS directive)
of the nominated type.\\

{\bf Example 1}\\

Let us consider initially various calculations on the water
molecule to illustrate CONFIG specification, and assume the
following set of input MOs, derived from a closed-shell SCF
calculation.\\


\begin{tabular}{llllll}
MO Sequence &  Symmetry & SCF      &    MO Sequence &  Symmetry & SCF\\
     number &      &   Occupation  &     number   &    &   Occupation\\  \hline
   1 &    1a$_{1}$  &  2.0         &       6   &   4a$_{1}$ &    0.0\\
   2 &   2a$_{1}$   & 2.0          &      7    &  2b$_{2}$ &    0.0\\
   3 &   1b$_{2}$   & 2.0          &      8    &  5a$_{1}$ &    0.0\\
   4 &   3a$_{1}$   & 2.0          &      9  &    2b$_{1}$ &    0.0\\
   5 &   1b$_{1}$   & 2.0 \\ \hline \\
\end{tabular}


To perform a full-valence space calculation, comprising 7 primary
orbitals, with the 4a$_{1}$ and 2b$_{2}$ virtual
MOs included in the active space, would require the
following data lines:

{
\footnotesize
\begin{verbatim}
          CONFIG
          DOC 1 TO 5
          UOC 6 7
          END
\end{verbatim}
}
To freeze the O1s orbital, we would present the sequence
{
\footnotesize
\begin{verbatim}
          CONFIG
          FZC 1
          DOC 2 TO 5
          UOC 6 7
          END
\end{verbatim}
}
The following sequence would be used to extend the space to include
the 5a$_{1}$ and 2b$_{1}$ orbitals:
{
\footnotesize
\begin{verbatim}
          CONFIG
          FZC 1
          DOC 2 TO 5
          UOC 6 TO 9
          END
\end{verbatim}
}
Note that all orbitals in the active space must appear first in the
input orbital set. Thus if we wished to include the
2b$_{1}$ MO, but not the 5a$_{1}$, we
would present the sequence:

{
\footnotesize
\begin{verbatim}
          CONFIG
          FZC 1
          DOC 2 TO 5
          UOC 6 TO 8
          END
\end{verbatim}
}
in conjunction with the SWAP data lines
{
\footnotesize
\begin{verbatim}
          SWAP
          8 9
          END
\end{verbatim}
}
Assuming the same orbital ordering had been derived from an RHF
calculation on the X$^{2}B_{1}$ state of the
\waterp\ ion, then a full-valence calculation, with frozen O1s, would
be performed under control of the following CONFIG data:

{
\footnotesize
\begin{verbatim}
          CONFIG
          FZC 1
          DOC 2 TO 4
          ALP 5
          UOC 6 7
          END
\end{verbatim}
}
{\bf Example 2}\\

 Assume the following set of input MOs from an RHF calculation on the
X$^{3}B_{1}$ state of methylene:\\



\begin{tabular}{llllll}
MO Sequence &  Symmetry & SCF      &    MO Sequence &  Symmetry & SCF\\
     number &      &   Occupation  &     number   &    &   Occupation\\  \hline
   1  &  1a$_{1}$   &  2.0         &        6   &  4a$_{1}$   &  0.0 \\
   2  &  2a$_{1}$   &  2.0         &        7   &  2b$_{2}$   &  0.0 \\
   3  &  1b$_{2}$   &  2.0 \\
   4  &  3a$_{1}$   &  1.0 \\
   5  &  1b$_{1}$   &  1.0 \\ \hline \\
\end{tabular}



A full-valence space calculation on the X$^{3}B_{1}$ state, with
frozen C1s, would be controlled thus:
{
\footnotesize
\begin{verbatim}
          CONFIG
          FZC 1
          DOC 2 3
          ALP 4 5
          UOC 6 7
          END
\end{verbatim}
}
The corresponding calculation on the $^{1}B_{1}$ state,
featuring singlet coupling of the 3a$_{1}$ and 1b$_{1}$ MOs,
would require use of the AOS and BOS tags, thus

{
\footnotesize
\begin{verbatim}
          CONFIG
          FZC 1
          DOC 2 3
          AOS 4
          BOS 5
          UOC 6 7
          END
\end{verbatim}
}
remembering to change (or remove) the MULT 3 specification!\\

{\bf Note}\\

 Additional keywords may be specified on the directive
initiator line, as follows:
\begin{itemize}

\item {\bf PRINT}: to obtain a complete list of the CASSCF configurations,
characterised by occupation pattern.

\item {\bf BYPASS}: at the outset of a CASSCF calculation the loop-formulae tape
(written in default to ED9) must be generated. The BYPASS keyword
allows this step to be bypassed in a subsequent restart job, assuming
of course that the data set had been retained between jobs (note that
the PRINT option is not effective in BYPASS mode).

\item {\bf  NOSORT}: if the 1-step Newton Raphson optimisation technique
is to be used during CASSCF iteration (under control of
the SIMUL directive, see 5.4.2), then the loop formulae
tape is reordered in the interests of efficiency. This reordered
file is written in default to ED10, with the reordering
process carried out automatically unless suppressed by
presenting the NOSORT keyword on the CONFIG data line.
\end{itemize}

{\bf Example}\\

Assuming that simultaneous optimisation is not to be performed, and
that a list of configurations is required, then the
following data line

{
\footnotesize
\begin{verbatim}
          CONFIG PRINT NOSORT
\end{verbatim}
}
should be presented in the startup job, and the line
{
\footnotesize
\begin{verbatim}
          CONFIG BYPASS NOSORT
\end{verbatim}
}
in a subsequent restart, assuming that ED9 had been `kept'.
The user is recommended to take advantage of the BYPASS and NOSORT
options where applicable, since the space requirements and generation
time of the loop formulae tape may often prove costly.

\section[Directives Controlling Wavefunction Convergence]{Directives Controlling Wavefunction Convergence}


\subsection[MAXCYC]{MAXCYC}

This directive consists of a single data line, 
read to variables TEXT,MAXC using format (A,I).
\begin{itemize}
\item TEXT should be set to the character string MAXCYC.
\item MAXC is an integer used to specify the maximum number of
iteration cycles required.
\end{itemize}
The directive may be omitted when MAXC will be set to the default
value of 50.
The following conditions cause termination of iteration:
\begin{itemize}
\item When the desired accuracy is reached, iteration of the 
SCF or MCSCF process stops.

\item If the job time remaining is insufficient to complete another
iterative cycle, or the maximum number of cycles as set by the
MAXCYC directive (or default) has completed,  iteration will cease.
\end{itemize}

{\bf Example}
{
\footnotesize
\begin{verbatim}
          MAXCYC 100
\end{verbatim}
}

Note that the maximum number of allowed cycles in CASSCF calculations
is limited to 20.

\subsection[THRESHOLD]{THRESHOLD}

This directive may be used to define a convergence threshold for
SCF and MCSCF iterations, and comprises a single data line read
to the variables TEXT, ISET using format (A,I);
\begin{itemize}
\item TEXT should be set to the character string THRESH
\item ISET is an integer parameter used in defining the 
threshold.
\end{itemize}
At SCF convergence, the elements of the density matrix will be
converged to within an absolute error 10$^{-ISET}$. The directive
may be omitted, when the default value 10$^{-5}$ will be used.
For CASSCF iterations values of the maximum Brillouin element
(super-CI) or maximum first derivative (Newton Raphson) control
the monitoring of convergence. Note that in most CASSCF and MCSCF
calculations setting ISET=4 will prove quite adequate.

\section[SCF Convergence - Default]{SCF Convergence - Default Driver}

In default the RHF, UHF and GVB modules of GAMESS--UK iterate under
control of a hybrid scheme of level shifting \cite{saunders} and DIIS
(Direct inversion in the Iterated Subspace \cite{pulay}). A set of
built-in level shifters will, when used in conjunction with DIIS, lead
in most cases to adequate convergence, and the user need only consider
providing data in troublesome cases.

A typical SCF calculation will, when far from convergence, proceed
under control of level shifting alone, and it is at this
point of the calculation that overriding the default shifters (under
control of the LEVEL directive) may be necessary. Once convergence
has `set-in', the DIIS procedure is initiated (corresponding to
a TESTER of ca. 0.1); experience to date suggests that rapid
convergence proceeds once this point has been reached.

\subsection[LEVEL]{LEVEL}

In its most general form the LEVEL directive may be used to nominate
two sets of level shifters to apply during an SCF calculation, the
first set to be used up to and including some user-nominated iteration,
the second to apply after this point. Note that the number and role of
the level shifters within each set is a function of SCFTYPE.  Note also
that the primary role of level shifting is to ensure the calculation
`arrives' in the quadratic region of convergence, from which point DIIS
will effectively control the SCF iterations.  This on occasions
requires setting higher values than those used, say, in a
level-shifting only environment, where larger values would act to slow
down convergence in the latter stages of the computation.  We describe
below the format of the directive for each SCFTYPE.

\subsubsection{Closed-shell RHF Calculations} 

In the case of a closed-shell RHF calculation 
the LEVEL directive consists of a single  data line 
read to variables TEXT, E1, IBRK, E2 using format (A,F,I,F).
\begin{itemize}
\item TEXT should be set to the character string LEVEL.
\item  E1 is the level shifter up to iterative cycle specified by IBRK.
\item  IBRK is an integer used to a specify the cycle number.
\item  E2 is the level shifter after the iterative cycle given by IBRK.
\end{itemize}
An alternative form of LEVEL is permitted, consisting of only two
parameters, read to variables TEXT, E1 using format (A,F). If used,
this sets the IBRK parameter to the default 999,  E2 will be given
the  value of 0.0, while  E1 has its usual meaning.
The LEVEL directive may be omitted, when the program will assign the
following default settings for most molecular systems:

{
\footnotesize
\begin{verbatim}
             E1=1.0 , E2=0.3 and IBRK=5
\end{verbatim}
}
SCF calculations on transition metal complexes comprising first
row metals are typically found to require higher values of
level shifter for satisfactory convergence. In such cases the
above defaults are now modified by the code to the following:

{
\footnotesize
\begin{verbatim}
             E1=2.0 , E2=2.0 and IBRK=999
\end{verbatim}
}
The following points should be noted on the specification
of level shifters:
\begin{itemize}
\item For the first three cycles, a value of at least unity should be
chosen, to stabilise what are usually the most erratic cycles.
\item If divergence is experienced, with the DIIS procedure
never instigated, increasing the level shifter will
force convergence to the onset of DIIS.
Increasing the level shifters through a data line of the form
data line

{
\footnotesize
\begin{verbatim}
             LEVEL 2.0
\end{verbatim}
}
is usually sufficient to force convergence to this onset.
\end{itemize}

\subsubsection{Open-shell RHF  and GVB Calculations}

In the case of open-shell RHF and GVB calculations the LEVEL directive
consists of a single  data line read to 
variables TEXT, OCC1, V1, IBRK, OCC2, V2
using format (A,2F,I,2F).
\begin{itemize}
\item TEXT should be set to the character string LEVEL.
\item  OCC1, V1 are the  doubly-occupied--partially-occupied 
and occupied-virtual
level shifters up to the iterative cycle specified by IBRK.
\item  IBRK is an integer used to a specify the cycle number.
\item  OCC2, V2 are the  doubly-occupied--partially-occupied 
and occupied-virtual
level shifters after the iterative cycle specified by IBRK.
\end{itemize}
An alternative form of LEVEL is permitted, consisting of only three
parameters, read to variables TEXT, OCC1, V1 using 
format (A,2F). In this form
TEXT, OCC1 and V1 have their usual meanings, whilst the program will set
IBRK=999 and OCC2, V2 to zero.
The LEVEL directive may be omitted, when the program will assign the
following default settings:

{
\footnotesize
\begin{verbatim}
             OCC1=0.05 , V1=1.0 ,  IBRK=5
             OCC2=0.01 , V2=0.5 
\end{verbatim}
}
The following points should be noted on using level shifters
to proceed to the onset of DIIS:
\begin{itemize}
\item Note that the value of the doubly-occupied--partially-occupied
level shifter is typically far smaller than that involving the
virtual interaction;
\item It can be shown that convergence to a stationary point on the
energy surface can be guaranteed if 'sufficiently' large and
positive level shifters are used. Thus if divergence is experienced
the user may repeat the job but with increased level shifters. The
data line
\item As with closed shell systems, the default shifters above
are doubled in value for systems containing first row transition
metal atom(s).

{
\footnotesize
\begin{verbatim}
             LEVEL 0.3 1.5
\end{verbatim}
}
is usually sufficient to force convergence to the onset of DIIS.
\end{itemize}

\subsubsection{Open-shell UHF Calculations} 

In the case of an open-shell UHF  calculation the LEVEL directive
consists of a single  data line read to 
variables TEXT, EA1, EB1, IBRK, EA2, EB2 using format (A,2F,I,2F).
\begin{itemize}
\item TEXT should be set to the character string LEVEL.
\item  EA1, EB1 are the  $\alpha$-spin orbital 
and $\beta$-spin  orbital
level shifters up to the iterative cycle specified by IBRK.
\item  IBRK is an integer used to  specify the cycle number.
\item  EA2, EB2 are the  $\alpha$-spin and $\beta$-spin
level shifters after the iterative cycle specified by IBRK.
\end{itemize}
An alternative form of LEVEL is permitted, consisting of only three
parameters, read to variables TEXT, EA1, EB1 using 
format (A,2F). In this form
TEXT, EA1 and EB1 have their usual meanings, whilst the program will set
IBRK=999 and EA2, EB2 to zero.
The LEVEL directive may be omitted, when the program will assign the
following default settings:

{
\footnotesize
\begin{verbatim}
             EA1=1.0, EB1=1.0 ,  IBRK=5
             EA2=0.3, EB2=0.3
\end{verbatim}
}
These defaults are doubled in value for systems containing first-row
transition metals atoms, thus:

{
\footnotesize
\begin{verbatim}
             EA1=2.0, EB1=2.0 ,  IBRK=999
             EA2=2.0, EB2=2.0
\end{verbatim}
}

\subsection[Core-Hole States]{Core-Hole States}

The present implementation of LEVEL within the open-shell RHF program
is such that core-hole states may be converged with an appropriate
setting of the parameters of the LEVEL directive. Specifically, such
states may be studied by presenting the data line

{
\footnotesize
\begin{verbatim}
          LEVEL 0.0 1.0
\end{verbatim}
}
where the doubly-occupied--partially occupied level shifter
is set to zero. It is assumed in such studies that the
open-shell orbital comprises the core orbital involved
in the ionization process.\\

{\bf Example}\\

The following two data files may be used to optimise the
geometry of the 1s core-hole state of the water molecule.
The first file generates the closed-shell MOs, the second
performs the hole-state calculation, with the SWAP
directive placing the oxygen 1s MO as the singly occupied
orbital.\\

{\bf The Closed-shell SCF}
{
\footnotesize
\begin{verbatim}
          TITLE
          H2O .. 3/21G
          ZMAT ANGSTROM
          O
          H 1 ROH
          H 1 ROH 2 THETA
          VARIABLES
          ROH 0.956   HESSIAN 0.7
          THETA 104.5  HESSIAN 0.2
          END
          ENTER
\end{verbatim}
}
{\bf The Core-Hole state open-shell SCF}
{
\footnotesize
\begin{verbatim}
          RESTORE NEW
          TITLE
          H2O +  1S-CORE HOLE STATE
          CHARGE 1
          MULT 2
          ZMAT ANGSTROM
          O
          H 1 ROH
          H 1 ROH 2 THETA
          VARIABLES
          ROH 0.956   HESSIAN 0.7
          THETA 104.5  HESSIAN 0.2
          END
          SWAP
          1 5
          END
          RUNTYPE OPTIMIZE
          LEVEL 0.0 1.0
          XTOL 0.003
          ENTER
\end{verbatim}
}

\subsection[DIIS]{DIIS}

The DIIS directive consists of a single  data line 
containing the character string DIIS in the first data
field. Subsequent data fields may be used
\begin{itemize}
\item to suppress DIIS, by presenting a data line of the
form
{
\footnotesize
\begin{verbatim}
             DIIS OFF
\end{verbatim}
}
when in default only level shifting will be in effect;
\item to request printing of the DIIS equations by
specifying the character string PRINT;
\item to modify the onset of DIIS by specifying the 
value of TESTER: this is achieved by typing the character
string ONSET, immediately followed by a floating point variable
defining the required value of TESTER. Thus the data line

{
\footnotesize
\begin{verbatim}
             DIIS ONSET 0.2
\end{verbatim}
}
would override the default onset of 0.1.
\item  to route information necessary for SCF cycling to proceed in
uninterrupted fashion across restart jobs. Such a process is important,
for example, in large direct-SCF calculations when perhaps only a single
SCF cycle may be possible in a given step.  The successful functioning of
DIIS relies on the process controlling the SCF over multiple iterations,
hence the need arises to save DIIS information. This is achieved by
presenting the LFNAME of a direct-access file, and starting block number
to which DIIS information will be written. Thus presenting the data line

{
\footnotesize
\begin{verbatim}
             DIIS ED4 1
\end{verbatim}
}
will result in the DIIS information being written
to ED4 commencing at block 1.
\end{itemize}
{\bf Example}\\

The following two data files illustrate this saving of DIIS information
between separate jobs. Assuming the file allocated to ED4 is saved in
the Startup job below, which terminated during SCF processing,  and
is subsequently allocated to the Restart job, the SCF cycling will be
identical to that observed if the first job had been run to completion.\\

{\bf Startup Job}
{
\footnotesize
\begin{verbatim}
          TITLE        
          C6H5.NO2  TZVP DIIS INFORMATION TO ED4
          NOPRINT
          ZMAT ANGSTROM
          C
          N 1 RCN
          X 2 1.0 1 90.0
          C 1 RCC1 2 T1 3 P1
          C 1 RCC1 2 T1 3 -P1
          C 4 RCC2 1 T2 2 P2
          C 5 RCC2 1 T2 2 P2
          C 7 RCC3 5 T3 1 P3
          O 2 RNO1 1 T5 3 -90.0
          O 2 RNO1 1 T5 3  90.0
          H 4 RCH1 1 T6 2 P5
          H 5 RCH1 1 T6 2 P5
          H 6 RCH2 4 T7 11 P6
          H 7 RCH2 5 T7 12 P6
          H 8 RCH3 7 T8 14 P7
          VARIABLES
          RCN 1.49\RCC1 1.37\RCC2 1.43\RCC3 1.37
          RNO1 1.21\RCH1 1.084\RCH2 1.084\RCH3 1.084
          T1 120.0\T2 120.0\ T3 120.0\T5 120.0\T6 120.0
          T7 120.0\T8 120.0\P1 90.0\P2 180.0\P3 0.0
          P5 0.0\P6 0.0\P7 0.0\END
          BASIS TZVP
          SCFTYPE DIRECT RHF
          DIIS ED4 1
          ENTER
\end{verbatim}
}
{\bf Restart Job}
{
\footnotesize
\begin{verbatim}
          RESTART SCF
          TITLE        
          C6H5.NO2  TZVP DIIS INFORMATION TO ED4
          NOPRINT
          ZMAT ANGSTROM
          C
          N 1 RCN
          X 2 1.0 1 90.0
          C 1 RCC1 2 T1 3 P1
          C 1 RCC1 2 T1 3 -P1
          C 4 RCC2 1 T2 2 P2
          C 5 RCC2 1 T2 2 P2
          C 7 RCC3 5 T3 1 P3
          O 2 RNO1 1 T5 3 -90.0
          O 2 RNO1 1 T5 3  90.0
          H 4 RCH1 1 T6 2 P5
          H 5 RCH1 1 T6 2 P5
          H 6 RCH2 4 T7 11 P6
          H 7 RCH2 5 T7 12 P6
          H 8 RCH3 7 T8 14 P7
          VARIABLES
          RCN 1.49\RCC1 1.37\RCC2 1.43\RCC3 1.37
          RNO1 1.21\RCH1 1.084\RCH2 1.084\RCH3 1.084
          T1 120.0\T2 120.0\ T3 120.0\T5 120.0\T6 120.0
          T7 120.0\T8 120.0\P1 90.0\P2 180.0\P3 0.0
          P5 0.0\P6 0.0\P7 0.0\END
          BASIS TZVP
          SCFTYPE DIRECT RHF
          DIIS ED4 1
          ENTER
\end{verbatim}
}

\subsection[CONV]{CONV}

This directive may be used to override the default convergence techniques
of level shifting and DIIS, and consists of a single data line
read to the variables TEXT,INDEX using format(A,I).
\begin{itemize}
\item TEXT should be set to the character string CONV
\item INDEX is an integer  used to specify the particular technique(s)
required. The following options are available:

{
\footnotesize
\begin{verbatim}
    0........use Pople's extrapolation
    1........do NOT use damping, extrapolation or level shifting
    2........use damping and Pople's extrapolation
    3........use Davidson's damping
    4........use level shifting and extrapolation
    5........use level shifting (default)
    6........use damping, extrapolation and level shifting
    7........use level shifting and damping
   10........use damping, extrapolation and restrict orbital
             mixing in GVB or open shell calculations.
   11........use damping and restrict orbital mixing
   14........use damping, extrapolation, level shifting and
              restrict orbital mixing
   15........use level shifting, damping and restrict orbital
             mixing
\end{verbatim}
}
\end{itemize}

\subsection[AVERAGE]{AVERAGE}

    AVERAGE [ ON $|$ OFF $|$ $<$tolerance$>$ ]

In SCF calculations convergence problems may arise if the molecule has a 
partly occupied set of degenerate orbitals. A way of dealing with this situation
is to occupy all degenerate orbitals equally, using fractional occupations if
neseccary. This equals building the density from an average of a number of 
states. The AVERAGE directive controls this procedure which is "on" by default.
The options "on" and "off" do the obvious, alternatively the tolerance for 
detecting degenerate orbitals can be specified. All orbitals in the energy
range of the HOMO energy plus/minus the tolerance will be included in the 
degenerate set. By default the tolerance is $10^{-4}$.


\subsection[SMEAR]{SMEAR}
The SMEAR directive implements Fermi smearing~\cite{warren} for filling
up the molecular orbitals. Normally, orbitals are either fully occupied or
empty. Fermi smearing allows orbitals to be fractionally filled,
according to a step function, that depends on the ''Fermi
temperature'' employed.

Fermi smearing can be useful in certain problematic convergence cases
where degenerecies in the orbital energies mean that it is uncertain
which state the SCF should converge on. Smearing may alleviate the
problem by using an average state thereby removing the need to make
a discrete choice.

The format of the directive is:

{
\footnotesize
\begin{verbatim}
          SMEAR <START_TEMP> [<FINAL_TEMP>] [<UNITS>] [SCALE <SCALE_VALUE>]
\end{verbatim}
}

where:

\begin{itemize}
\item START\_TEMP (F) is the starting Fermi Temperature
\item FINAL\_TEMP (F) is the final Fermi Temperature
\item UNITS (A) specifies the units to be used for the final and
  starting tempratures. By default the units are Hartrees, but setting
  UNITS to EV changes the units to Electron Volts.
\item SCALE (A) is the keyword SCALE followed by the scale factor 
  SCALE\_VALUE (F). 
\end{itemize}
The scale factor is used in the temperature updates
going from the start to the final temperature. To ensure that the final
temperature has been reached at convergence the current temperature
is updated as a linear function of the difference between the SCF tester and
the SCF convergence criterion
\begin{eqnarray}
   T_0     &=& T_{start} \\
   T_{i+1} &=& \max(T_{final},\min(T_i, SCALE\_VALUE * (tester-convergence))
\end{eqnarray}

A couple of points should be noted about the use of Fermi-smearing:

\begin{enumerate}
\item The Fermi-Dirac smearing enforces strict Aufbau ordering of the
  orbitals through the occupations. Thus it cannot be used together
  with options that may break the Aufbau ordering such as locking.
\item The Fermi-Dirac smearing breaks the strict distinction between
  occupied and virtual orbitals. In practice there will be three categories
  occupied, partially occupied and virtual orbitals. This requires a 
  modified definition of the tester. The tester now becomes the absolute
  maximum off-diagonal value of the Fock matrix excluding the 
  occupied-occupied and virtual-virtual blocks.
\end{enumerate}

Finally with the ''IPRINT SMEAR'' directive more detailed information about
the Fermi smearing can be obtained, such as the actual orbital occupations
used in every iteration, and the chemical potential in Hartree.

\subsection[CASSCF Convergence]{CASSCF Convergence}

Four directives are available for specifying the optimisation
techniques to be used in the course of a CASSCF calculation. The
user nominates which techniques are to be employed on
each iterative cycle of the computation through use of
the SUPERCI, NEWTON, HESSIAN and SIMUL directives.
Thus the data sequence

{
\footnotesize
\begin{verbatim}
          SUPERCI 1 TO 4
          NEWTON  5 TO 20
          HESSIAN 5 9 13 17
\end{verbatim}
}
would specify super-CI optimisation for the first four
cycles, followed by subsequent 2-step Newton-Raphson (NR)
for the remainder of the computation, with explicit
construction of the orbital Hessian on cycles 5, 9, 13 and 17. The
above sequence corresponds in fact to the default
specification, and will be used in the absence of controlling
directives.
The following points should be noted:
\begin{enumerate}

\item  The optimum technique on a given cycle is very much dependent on the
current degree of convergence. At the outset of the
calculation starting from, say, an SCF orbital set, the user is strongly
recommended to use the first-order super-CI option, and to
continue with this until the degree of convergence (monitored by
the maximum Brillouin element) suggests switching to the
`pseudo second-order' NR technique. Instigating NR too rapidly however
will lead to divergence, with the maximum first derivative increasing,
leading finally to the error message `Hessian diagonalisation
has failed to converge'. Experience suggests that a maximum Brillouin
element of 0.05-0.01 represents the optimum point at which to
instigate NR control, with between 5 - 10 cycles of super-CI
normally required to reach this point.
\item  The sequence of directives specified is dynamic, and is not
remembered between separate runs of the program. Thus while the data
sequence

{
\footnotesize
\begin{verbatim}
          SUPERCI 1 TO 7
          NEWTON  8 TO 20
          HESSIAN 8 to 20
\end{verbatim}
}
may be presented in a startup job, the user should modify this sequence
in any restart job to reflect the current degree of convergence. Thus if
the startup job dumped, say, on cycle 11 with the maximum first
derivative suggesting satisfactory convergence of the NR process, the
data sequence

{
\footnotesize
\begin{verbatim}
          NEWTON 1 TO 20
          HESSIAN 1 TO 20
\end{verbatim}
}
should be presented in the restart job.

\item  Within a given run of the program the sequence is remembered, and
will be applied, for example, in each separate CASSCF calculation of
a geometry optimisation. Some caution should be exercised at the
outset of such an optimisation, with the onset of the NR method specified
to allow for possible large changes in wavefunction associated
with large steps in the geometry optimisation.
\item  Hessian construction must be performed on the first NR cycle.
\item  The program currently requires `in-core' treatment of the orbital
hessian, and as such NR usage is limited to cases with a rather
modest number of orbital rotation parameters. If memory requirements
preclude such a treatment, the user should present the data
sequence

{
\footnotesize
\begin{verbatim}
          SUPERCI 1 TO 20
          THRESH 4
\end{verbatim}
}
which will lead to satisfactory convergence in many cases.
\item  In many instances the most rapid convergence is realised using
the 1-step NR method, featuring simultaneous optimisation of both CI
coefficients and orbital rotation parameters. This technique may be
requested under control of the SIMUL directive, with a typical
data sequence shown below:

{
\footnotesize
\begin{verbatim}
          SUPERCI 1 TO 5
          NEWTON 6 TO 20
          HESSIAN 6 TO 20
          SIMUL 8 TO 20
\end{verbatim}
}
The memory requirements are, of course, aggravated by the inclusion of the
CI terms in the Hessian. When applicable this technique is highly
effective in stable geometry-optimisations.

\end{enumerate}

% BEGIN NEWSCF
\section[SCF Convergence - Alternate Driver]{SCF Convergence -  Alternate Driver} 

A new SCF driver has has been developed within GAMESS-UK that
provides an alternative way of controlling SCF calculations. The
driver is orthogonal to the default driver and is currently still in
the ''development'' stage, and is therefore not included in the default
build of GAMESS-UK. Users who have access to the source code and
wish to include the driver in their build, should add the:

{
\footnotesize
\begin{verbatim}
          newscf_f90
\end{verbatim}
}

keyword as an option when configuring the code.

There were a number of motives for developing the new driver. These
included:

\begin{itemize}

\item to allow more flexible control of convergence schemes for cases
  that were proving difficult to converge with the default driver.
\item to reduce IO activity to the dumpfile/scratchfile by holding
  more structures in memory, thereby reducing a bottleneck for
  parallel calculations.
\end{itemize}

The consequence of more structures being held in memory has enabled
the development of a parallel version of the driver in
which these data structures can be distributed across all the nodes of
a parallel machine, allowing the code to take advantage of the large
aggregate memory on these machines (see the parallel ScaLAPACK version
described in chapter 14). However, a consequence of this is that the
code is rather memory-hungry when run in serial or on small processor
counts.

As the driver is still relatively new, it does not have support for
all of the features within GAMESS-UK (such as ZORA, DRF, etc.). As of
the writing of the manual, the driver supports:

\begin{itemize}
\item RHF and UHF (but not ROHF) SCF Calculations (both direct and conventional)
\item RHF and UHF (but not ROHF) DFT Calculations (both direct and conventional)
\end{itemize}

\subsection[Input Control]{Input Control}
Use of the module is requested by including a block of directives of the
following form:

{
\footnotesize
\begin{verbatim}
          newscf
          <control directives>
          end
\end{verbatim}
}

The $<$control directives$>$ block can be empty (in which case a default
convergence scheme will be used), or it can contain the control
directives described in the following sections.

An input file with a set of example control directives is provided at
the end of this section for readers who wish to gain an overview of
how things are structured before before becoming bogged down in the
details of each directive.

\subsection[Overall Control Flags]{Overall Control Flags}

\subsubsection{Controlling Printed Output - PRINT}
This directive controls which data from the calculation will be
included in the output. The format of the directive is a single line
containing the keyword PRINT (A), followed by any combination of the
following parameters to the PRINT command (all in A format).

\begin{tabular}{ll}
\\ \hline
FOCK &  requests printing of the Fock matrix at SCF convergence \\
GUESS & Print out the trail vectors\\
DENSITY & Print out the density matrix\\
VECTORS & Print out the eigenvectors\\
FULL & Print everything (WARNING: this may produce a lot of output!)\\
FRONTIER & Print the frontier orbitals\\
DIIS & Monitoring of the solution of the DIIS equations\\
TIME & Print the cycle and total wallclock and cpu times each cycle\\ \hline
\end{tabular}

So, for example, the line:

{
\footnotesize
\begin{verbatim}
          PRINT FOCK
\end{verbatim}
}

would cause the entire Fock matrix to be printed on SCF convergence.

\subsubsection{SCF exit status - SOFTFAIL}
This directive consists of the the single character string SOFTFAIL
(A), and will prevent the code from generating an error if MAXCYC SCF
cycles are reached without convergence.

\subsection[Controlling the convergence of a calculation]{Controlling the convergence of a calculation}
Convergence is controlled by providing parameters for a series of
''phases'', each becoming active depending on particular criteria and
employing various convergence controls.

If convergence control directives are omitted, a default scheme is
adopted, otherwise the user will need to provide the control
information for a number of phases, including the criteria that are
used to switch from one phase to another.

The control information is provided in a ''phase block'', which is
started by a line with two data fields PHASE (A) and IPHASE (I).

\begin{itemize}
\item PHASE is set to the character string PHASE
\item IPHASE is an integer (greater than zero) identifying the phase
\end{itemize}

A phase block is terminated by a subsequent ''PHASE'' directive, or an
''END'' directive, indicating the end of the newscf control
directives.

Within a phase block, the criteria for when to jump to a subsequent
phase is determined by the ''NEXT'' directive.

An individual phase block is arranged as shown below:

{
\footnotesize
\begin{verbatim}
          PHASE <N>
          <convergence controls>
          NEXT <M>
          <convergence criteria>
\end{verbatim}
}

where $<$N$>$ is an integer denoting the phase in questions and $<$M$>$ an
integer denoting the next phase to jump to.

Within a phase block, the user can determine which convergence
controls are to be used, under what conditions to switch to another
phase block and when the SCF will be deemed to have converged.

\subsubsection[CONVERGENCE CONTROLS]{CONVERGENCE CONTROLS}
The convergence controls available (together with the default values) are:

%WHAT HALFWIT DECIDED YOU WOULD NEVER WANT MORE THAN 2 SECTIONS IN LATEX!?!
\subsubsection[Level Shifters - LEVEL]{Level Shifters - LEVEL}
The format of the LEVEL directive is the same as that for the Default
SCF driver described in section 6.1.

The default value for the NEWSCF driver is 0.5

\subsubsection[DIIS]{DIIS}
The single keyword DIIS in format (A) indicates that DIIS should be
active for this phase. 

\subsubsection[NEWDIIS]{NEWDIIS}
This directive consists of the single keyword NEWDIIS in format (A),
and resets the DIIS space for the phase, as explained below.

DIIS works by taking a linear combination of the previous Fock
matricies to determine the subsequent set of vectors. DIIS therefore
has a ''memory'' of the previous Fock matrices that it uses to
generate the next solution. The NEWDIIS keyword causes the memory from
subsequent phases to be wiped and for DIIS to start afresh.

\textbf{NB:} DIIS requires a ''memory'' of at least 3 cycles to be able to
function, so resetting DIIS means that it will only become active
again on the 4th cycle of the phase.

\subsubsection[EXTRAPOLATION]{EXTRAPOLATION}
The format of this directive is a line with three data fields, the
first being the character EXTRAP (A) followed by the two values TEST (F)
and COEF (F). The TEST and COEF variables are used as explained below.

When two successive SCF steps satisfy the colinearity criteria
\begin{equation}
          \frac {d1.d2}{ |d1| \times |d2| } > TEST
\end{equation}

where

\begin{equation}
          d1 = f_{1} - f_{-2}
\end{equation}

and

\begin{equation}
          d2 = f - f_{-1}
\end{equation}

and f, f$_{-1}$ and f$_{-2}$ are the current, previous and next previous
fock matrices, the extrapolation.

\begin{equation}
f +\!= COEF \times d2
\end{equation}

will be applied.

Typically TEST should be around 0.95 and COEF around
1. In addition, there is also a test on $|d1| / |d2|$, so that if
successive steps are in the same direction but of very different
lengths extrapolation is suppressed.

The line below demonstrates the use of the directive with the
recomended values:

{
\footnotesize
\begin{verbatim}
          EXTRAP 0.95  1
\end{verbatim}
}

\subsubsection[NEWEXT]{NEWEXT}
This directive consists of the single keyword NEWEXT in format (A),
and resets the extrapolation counter to zero so that extrapolation
will be inactive for the first cycle.

\subsubsection[RESTORE]{RESTORE}
This directive consists of the single keyword RESTORE in format (A).

The restore directive recovers the best set of vectors attained during
the calculation thus far at the start of the phase.

\subsubsection[LOCK]{LOCK}
This directive consists of the single keyword LOCK in format (A) and
applies configurational locking to the phase.

Normally, the electrons populate the orbitals according to Hund's
rules, so that they are filled up sequentially starting from the
lowest energy orbital and leaving no gaps until all the electrons are
placed in an orbital.

When configurational locking is applied, the overlap between the
occupied orbitals in the previous SCF cycle and the current is
calculated and the electrons remain in the orbital that overlaps most
closely with the one they were in previously - regardless of whether
there is an unoccupied orbital lower in energy.

Locking can therefore be used to maintain an excited state
configuration during a calculation.

\subsubsection[SMEAR]{SMEAR}
The SMEAR directive implements Fermi smearing~\cite{warren} for filling up the
molecular orbitals. Normally, orbitals are either fully occupied or
empty. Fermi smearing allows orbitals to be fractionally filled,
according to a step function, that depends on the ''Fermi
temperature'' employed.

Fermi smearing can be useful in certain problematic convergence cases
where degenerecies in the orbital energies mean that it is uncertain
which state the SCF should converge on. Smearing may alleviate the
problem by using an average state thereby removing the need to make
a discrete choice.

The format of the directive is:

{
\footnotesize
\begin{verbatim}
          SMEAR <START_TEMP> [<FINAL_TEMP>] [<UNITS>] [SCALE <SCALE_VALUE>]
\end{verbatim}
}

where:

\begin{itemize}
\item START\_TEMP (F) is the starting Fermi Temperature
\item FINAL\_TEMP (F) is the final Fermi Temperature
\item UNITS (A) specifies the units to be used for the final and
  starting tempratures. By default the units are Hartrees, but setting
  UNITS to EV changes the units to Electron Volts.
\item SCALE (A) is the keyword SCALE followed by the scale factor 
  SCALE\_VALUE (F). 
\end{itemize}
The scale factor is used in the temperature updates
going from the start to the final temperature. To ensure that the final
temperature has been reached at convergence the current temperature
is updated as a linear function of the difference between the SCF tester and
the SCF convergence criterion
\begin{eqnarray}
   T_0     &=& T_{start} \\
   T_{i+1} &=& \max(T_{final},\min(T_i, SCALE\_VALUE * (tester-convergence))
\end{eqnarray}

A couple of points should be noted about the use of Fermi-smearing:

\begin{enumerate}

\item The Fermi-Dirac smearing enforces strict Aufbau ordering of the
  orbitals through the occupations. Thus it cannot be used together
  with options that may break the Aufbau ordering such as locking.
\item The Fermi-Dirac smearing breaks the strict distinction between
  occupied and virtual orbitals. In practice there will be three categories
  occupied, partially occupied and virtual orbitals. This requires a 
  modified definition of the tester. The tester now becomes the absolute
  maximum off-diagonal value of the Fock matrix excluding the 
  occupied-occupied and virtual-virtual blocks.
\end{enumerate}

\subsection[Changing Phase]{Changing Phase}
The change from one phase to another is controlled by a number blocks of
commands, each starting with a line with two data fields NEXT (A) and PHASE
(I).

\begin{itemize}
\item NEXT is set to the character string NEXT
\item PHASE is an integer identifying the phase to jump to once the
  relevant criteria (described below) have been met.
\end{itemize}

The block is terminated either by another NEXT directive, a PHASE
directive, or an END directive.

Within a NEXT block, the criteria for when to change to phase 0
(i.e. NEXT 0) is particularly important
as this specifies when the calculation is determined to have
converged. If no ''NEXT 0'' directive is specified for a phase then
the calculation cannot converge from this phase and can only jump to
other phases when it meets their jump criteria; converging from them if it
meets their ''NEXT 0'' criteria. If no ''NEXT 0'' is specified in any
phase, the calculation will run until it runs out of cycles (as
specified by MAXCYC).

Multiple criteria can be specified within a ''NEXT'' block, and the
jump will only then occur when \emph{all} of the criteria have been
met (i.e. a logical \emph{and} test is used).

Users wishing to use an \emph{or} test, can specify multiple next
blocks for the same phase.

The criteria for changing phases are as follows:

\subsubsection[TESTER]{TESTER}
This directive consists of the keyword TESTER (A) followed
by the parameters DIRECTION (A) and CRITERIA (F).

\begin{itemize}
\item DIRECTION should be either set to the character string ''ABOVE'' or
  ''BELOW'' to indicate the the change should happen when the tester is
  greater than or less than the specified value.
\item CRITERIA should be set to the desired value of the TESTER.
\end{itemize}

The TESTER is defined as the maximum Fock matrix element in the MO
basis for the occupied-virtual block. This value will tend to zero
when the calculation has converged.

\subsubsection[Change in TESTER - DTESTER]{Change in TESTER - DTESTER}
This directive consists of the keyword DTESTER (A) followed
by the parameters DIRECTION (A) and CRITERIA (F).

\begin{itemize}
\item DIRECTION should be either set to the character string ''ABOVE'' or
  ''BELOW'' to indicate the the change should happen when the change
  in the tester is greater than or less than the specified value.
\item CRITERIA should be set to the change in the tester within this
  phase.
\end{itemize}


\subsubsection[Change in energy - DE]{Change in energy - DE}
This directive consists of the keyword DE (A) followed
by the parameters DIRECTION (A) and CRITERIA (F).

\begin{itemize}
\item DIRECTION should be either set to the character string ''ABOVE'' or
  ''BELOW'' to indicate the the change should happen when the change in energy is
  greater than or less than the specified value.
\item CRITERIA should be set to the change in energy in Hartrees
  within this phase.
\end{itemize}

\subsubsection[Absolute change in energy - DEABS]{Absolute change in energy - DEABS}
This directive consists of the keyword DEABS (A) followed by the
parameters DIRECTION (A) and CRITERIA (F).

\begin{itemize}
\item DIRECTION should be either set to the character string ''ABOVE'' or
  ''BELOW'' to indicate the the change should happen when the change in energy is
  greater than or less than the specified value.
\item CRITERIA should be set to the absolute change in energy
  (i.e. from the starting ''Guess'' energy) in Hartrees.
\end{itemize}

\subsubsection[Number of cycles in this phase - NCYC]{Number of cycles in this phase - NCYC}
This directive consists of the keyword NCYC (A) followed
by the parameters DIRECTION (A) and CRITERIA (I).

\begin{itemize}
\item DIRECTION should be either set to the character string ''ABOVE'' or
  ''BELOW'' to indicate the the change should happen when the number
  of cycles is greater than or less than the specified value.
\item CRITERIA should be set to the number of SCF cycles within this phase.
\end{itemize}

\subsubsection[Total number of SCF cycles - TOTCYC]{Total number of SCF cycles - TOTCYC}
This directive consists of the keyword TOTCYC (A) followed
by the parameters DIRECTION (A) and CRITERIA (I).

\begin{itemize}
\item DIRECTION should be either set to the character string ''ABOVE'' or
  ''BELOW'' to indicate the the change should happen when the number
  of cycles is greater than or less than the specified value.
\item CRITERIA should be set to the total number of SCF cycles within
  this energy calculation.
\end{itemize}

This directive is most useful when used to trigger a phase change /
indicate convergence when used in conjunction with a slacker TESTER
than the usual.

Often, when performing geometry optimisations, the wavefunction will
not converge to the desired value of the tester at a paticular
geometry, regardless of the number of SCF cycles undertaken. However,
the quality of the wavefunction may still be good enough to generate
an energy profile that allows the optimiser to take a step towards a
more favourable geometry where the wavefunction will converge with no
problems.

The TOTCYC directive therefore allows a user to make this step after
the SCF has been grinding away unsuccessfully for a large number of
cycles without compromising on the quality of the wavefunction for
less problematic steps.


An example of its usage is below:

{
\footnotesize
\begin{verbatim}
          next 0 
          tester below 1.E-06
          info Converged normally as tester < 1.E-06
          next 0
          totcyc above 50
          tester below  1.E-05
          info converged with slacker tester of 1.E-05 as totcyc > 50
\end{verbatim}
}


\subsubsection[Information on a phase change - INFO]{Information on a phase change - INFO}
This directive consists of the keyword INFO (format A) followed
by a string of text, that may extend up to the end of the logical line
(80 characters by default)

This directive can be used to print a string to the output file
indicating why a particular phase change has been carried out, e.g.

{
\footnotesize
\begin{verbatim}
          info Jumping to phase 3 as tester  < 0.001
\end{verbatim}
}

\subsection[Example newscf input file]{Example newscf input file}
The example below demonstrates a newscf convergence scheme that uses
the following phases and criteria for shifting between them:

\textbf{Phase 1}
\begin{itemize}
\item Level shifters of 2.0 and 2.0 (alpha and beta)
\item Switch to phase 2 when tester $<$ 0.01
\end{itemize}

\textbf{Phase 2}
\begin{itemize}
\item Level shifters of 0.5 and 0.5, DIIS and configurational locking
\item Switch to phase 3 when tester $<$ 0.002
\end{itemize}

\textbf{Phase 3}
\begin{itemize}
\item DIIS and configurational locking
\item convergence when tester $<$ 5.0d-6
\end{itemize}

The input is as follows:

{
\footnotesize
\begin{verbatim}
time 1000
core 3000000
#restart new
punch coor conn
punch basi vect 1 occu scfe eige
mult 2
symt 6
title
Cu-NO, DZ Cation  - new geom, new code
geometry
      1.931427      -0.080801      -0.022645  29  Cu
     -0.924018      -1.179405       2.034581  8    O
     -3.444799       0.046586       0.020112  13  Al
      7.478471       0.371579       0.192578  8    O
     -5.043143       2.660967       1.071269  8    O
     -0.902242       1.249695      -1.985263  8    O
     -5.027333      -2.570596      -1.047347  8    O
     -6.519617      -3.214017      -1.846467  1    H
     -0.954846      -2.572273       3.214984  1    H
      5.298164      -0.243155      -0.154613  7    N
     -0.874905       2.707837      -3.083889  1    H
     -6.599492       3.284123       1.755577  1    H
end
basis dz
runt scf
# Start newscf directives
newscf
print full diis frontier
#
phase 1
level 2.0 2.0
next 2
info Changing to phase 2 as tester < 0.01
tester below  0.01
#
phase 2
level 0.5 0.5
diis
lock
next 3
info Changing to phase 3 as tester < 0.002
tester below 0.002
#
phase 3
lock
diis
next 0
info Converged form phase 3 as tester < 5.0d-6
tester below 5.0d-6
end
# End newscf directives
runtype optx
scftype uhf
cdft quad high
cdft b3lyp screen
#vectors 3 4
enter 1 2
\end{verbatim}
}

% END NEWSCF

% START VanDerWaals
\section{Specification of Dispersion Corrections}
\label{vanderwaals-corr}

A problem frequently encountered with effective one-electron models is that
long range
correlation effects such as dispersion or Van der Waals interactions are not
properly described. This leads to problems where these relatively weak 
dispersion forces are important such as in DNA base pair stacking or the 
interaction between aromatic molecules. So far attempts to rigorously address
these problems have had limited success. In response to this empirical 
approaches to correct for the lack of dispersion have been suggested. GAMESS-UK
supports two versions of one of these
approaches~\cite{grimme04,grimme06,antony06}.
Both these approaches were originally designed to be used in the context of 
DFT calculations but they may be useful in a wider context.

Both of the supported approaches are based on a simple
model for the dispersion energy of two interacting atoms
\begin{equation}
D_{ij}(R_{ij}) = C_6^{ij}\frac{f_{\mathrm{damp}}(R_{ij})}{R^{6}_{ij}}.
\label{dispersion-dij}
\end{equation}
The function $f_{\mathrm{damp}}$ is simply chosen such that it goes to zero
when $R_{ij}$ goes to zero to prevent that atoms collapse onto eachother.
This damping function is of the form
\begin{equation}
f_{\mathrm{damp}}(R_{ij}) = \frac{1}{1+e^{-\alpha\left(R_{ij}/R_0-1\right)}}
\label{dispersion-damping}
\end{equation}
where $\alpha$ is a universal exponent and $R_0$ is the sum of Van der Waals
radii of atoms $i$ and $j$. 

The total dispersion energy may now be expressed as
\begin{equation}
E_{\mathrm{disp}} = -s_6 \sum_{i=1}^{N_{\mathrm{atom}}-1}
                         \sum_{j=i+1}^{N_{\mathrm{atom}}} D_{ij}
\end{equation}
where $s_6$ is a scale factor that depends on {\it ab initio} energy expression
that is used in conjunction to the dispersion correction.

The dispersion energy expression~\ref{dispersion-dij} requires a $C_6$
coefficient for every pair of chemical elements. To reduce the number of these
required parameters the pair $C_6$ coefficients are approximated and expressed
in terms of $C_6$ coefficients of the elements. The current implementation
supports two models for this:
\begin{description}
\item[The {\em Average} $C_6$ pair model]
     In this model the pair $C_6$ coefficient is approximated simply as the
     average of the elemental $C_6$ coefficients~\cite{grimme04}:
     \begin{equation}
        C_6^{ij} = 2\frac{C_6^i~C_6^j}{C_6^i + C_6^j}
        \label{average-c6-pair-model}
     \end{equation}
     This is presently the default.
\item[The {\em Geometric Mean} $C_6$ pair model]
     In this model the pair $C_6$ coefficient is approximated as the
     geometric mean of the elemental $C_6$
     coefficients~\cite{grimme06,antony06}:
     \begin{equation}
        C_6^{ij} = \sqrt{C_6^i~C_6^j}
        \label{geometric-c6-pair-model}
     \end{equation}
     This is currently considered as the most accurate model. It also supports
     the most chemical elements through published atomic $C_6$
     coefficients~\cite{grimme06}.
\end{description}

\subsubsection{VDWAALS -- Directives controlling the dispersion corrections}

Detailed control of the dispersion corrections is available through a special
input block (for use with DFT calculations a simplified input is available,
see~\ref{DFT-dispersion-corr}). This input block starts with the VDWAALS
directive and terminates with the END directive. In between these two
directives any number of directives to fine tune the dispersion corrections
may be presented. So the input block is of the form:

{\bf Example}

{
\footnotesize
\begin{verbatim}
          VDWAALS
             ...
          END
\end{verbatim}
}

Within the above structure the directives ON, OFF, C6MODEL, SCALE, ALPHA, 
RADIUS, and C6 may be used as described below. Note that the directives
SCALE, ALPHA, RADIUS and C6 set values independently for the active $C_6$ 
pair model. I.e. 
{
\footnotesize
\begin{verbatim}
          VDWAALS
             C6MODEL AVERAGE
             SCALE 1.3
             EXPONENT 30.0
             C6MODEL GEOMETRIC
          END
\end{verbatim}
} 
results in using the geometric mean $C_6$ pair model with the default settings
and NOT with the scale factor and exponent from the input. The reason is that
changing the parameters for the average $C_6$ pair model does not affect those
of the geometric mean model and {\it vice versa}.

\paragraph{The ON directive}

The ON directive may be used to explicitly turn the dispersion corrections on.
The directive is read to the variable TEXT using the format (A) where TEXT
should be set to the character string ON.

\paragraph{The OFF directive}

The OFF directive may be used to explicitly turn the dispersion corrections off.
The directive is read to the variable TEXT using the format (A) where TEXT
should be set to the character string OFF.

\paragraph{The C6MODEL directive}

The C6MODEL directive may be used to choose the $C_6$ coefficient pair model
to be used. The directive consists of two data fields read to the variables
TEXT and TEXTA using the format (A,A)
\begin{itemize}
\item TEXT should be set to the character string C6MODEL
\item TEXTA may be set to either
      \begin{itemize}
      \item AVERAGE to select the {\em average} $C_6$ coefficient pair model
            according to equation~\ref{average-c6-pair-model}
      \item GEOMETRIC to select the {\em geometric mean} $C_6$ coefficient pair
            model according to equation~\ref{geometric-c6-pair-model}
      \end{itemize}
\end{itemize}

\paragraph{The SCALE directive}

The SCALE directive may be used to change the overall scale factor $s_6$ for
the dispersion correction. The directive consists of two data fields read
to the variables TEXT and SCALE using the format (A,F)
\begin{itemize}
\item TEXT should be set to the character string SCALE.
\item SCALE should be set to the scale factor for the current $C_6$ coefficient
      pair model which has to be a non-negative floating point value.
\end{itemize}

\paragraph{The ALPHA directive}

The ALPHA directive may be used to change the exponent $\alpha$ for
the damping function of equation~\ref{dispersion-damping}. The directive
consists of two data fields read to the variables TEXT and EXPONENT using the
format (A,F)
\begin{itemize}
\item TEXT should be set to the character string ALPHA.
\item EXPONENT should be set to the exponent of the damping function for the
      current $C_6$ coefficient pair model. It has to be a non-negative
      floating point value.
\end{itemize}

\paragraph{The RADIUS directive}

The RADIUS directive may be used to change the Van der Waals radius of a
chemical element.
The directive consists of three data fields read
to the variables TEXT, ELEMENT and RAD0 using the format (A,A,F)
\begin{itemize}
\item TEXT should be set to the character string RADIUS.
\item ELEMENT should be set to the chemical symbol of the element.
\item RAD0 should be set to the Van der Waals radius of the element for the
      current $C_6$ coefficient pair model. The value should be specified in
      Bohr (atomic units).
\end{itemize}

\paragraph{The C6 directive}

The C6 directive may be used to change the $C_6$ coefficient of a
chemical element.
The directive consists of three data fields read
to the variables TEXT, ELEMENT and COEFF using the format (A,A,F)
\begin{itemize}
\item TEXT should be set to the character string C6.
\item ELEMENT should be set to the chemical symbol of the element.
\item COEFF should be set to the $C_6$ coefficient of the element for the
      current $C_6$ coefficient pair model. The value should be specified in
      Hartree*Bohr$^6$ (atomic units).
\end{itemize}

{\bf Example}

This example sets the parameters as recommended by Antony et al.~\cite{antony06}
for a B3LYP calculation on formaldehyde.
{
\footnotesize
\begin{verbatim}
          VDWAALS
              C6MODEL GEOMETRIC
              SCALE     1.05
              ALPHA    20.0
              RADIUS H  1.89
              RADIUS C  2.74
              RADIUS O  2.54
              C6     H  2.43
              C6     C 30.35
              C6     O 12.14
          END
\end{verbatim}
}
% END VanDerWaals

% START MCSCF
\section[Directives Controlling MCSCF Calculations]{Directives Controlling MCSCF Calculations}

Data input characterising the MCSCF calculation commences with the
MCSCF data line and is typically followed by a sequence of directives,
terminated by presenting a valid {\em Class 2} directive, such as VECTORS
or ENTER.

\subsection[MCSCF]{MCSCF}
The MCSCF data initiator consists of a single data line with the character
string MCSCF in the first data field.  It acts to transfer control
to those routines responsible for inputing all data relevant to the
MCSCF calculation. Termination of this data is achieved by presenting a
valid {\em Class 2} directive that is not recognised by the MCSCF input
routines, for example ENTER.

\subsection[ORBITAL]{ORBITAL}

The ORBITAL directive must be presented in a MCSCF calculation, 
and acts to,
\begin{enumerate}
\item  define the active space in the  calculation, by partitioning
the orbitals into a primary and secondary set  \cite{roos};
\item  specify an initial reference configuration 
that will be employed, for example, in
generating the complete CI space in a CASSCF calculation.
\end{enumerate}
The directive comprises a number of data lines, with the first line
containing the character string ORBITAL in the first data field.
Subsequent lines comprise
a sequence of orbital TAGS whereby each orbital
in the primary space is classified both by type  and by symmetry, 
with this sequence terminated by the character string END.
The following orbital types are used in this
classification:
\begin{itemize}
\item  FZC - frozen orbital i.e. an orbital which will remain
frozen as input throughout the MCSCF iterations.
\item  COR - core orbital i.e. an orbital which will remain
doubly occupied in all configurations.
\item DOC - doubly occupied i.e. an orbital which is doubly occupied in
the reference configuration, and which  will be permitted
variable occupancy in the MCSCF treatment.
\item  ALP - an unpaired orbital i.e. an orbital which 
is singly occupied with $\alpha$--spin
in the principle reference configuration, and which will be permitted
variable occupancy in the MCSCF treatment.
\item  BET - an unpaired orbital i.e. an orbital which 
is singly occupied with $\beta$--spin
in the principle reference configuration, and which will be permitted
variable occupancy in the MCSCF treatment.
\item UOC - formally unoccupied orbitals, corresponding to SCF virtual
MOs, which will be permitted variable occupancy in the MCSCF.
\end{itemize}
Each of the core and active orbitals must be classified according
to its type above, and in addition, its symmetry 
(IRrep) under the point group symmetry in use. The integer flag
characterising the symmetry (as produced for example in the
SCF output) is appended to the appropriate 3-character string
above, so that for a C$_{2v}$ molecule, a doubly occupied
orbital of a$_{1}$ symmetry would be tagged DOC1, an unoccupied
orbital of b$_{2}$ symmetry, UOC3.

\subsubsection{ORBITAL - Example 1}

Let us consider initially various calculations on the water
molecule to illustrate ORBITAL specification. The example
is based on a TZVP basis, with the following 
set of input MOs, derived from a closed-shell SCF
calculation:

{
\footnotesize
\begin{verbatim}
                       TOTAL ENERGY       -76.0553958459
          ===============================================
          M.O.  IRREP  ORBITAL ENERGY   ORBITAL OCCUPANCY
          ===============================================
            1      1    -20.55996277           2.0000000
            2      1     -1.35317135           2.0000000
            3      3     -0.71806974           2.0000000
            4      1     -0.58054044           2.0000000
            5      2     -0.50734492           2.0000000
            6      1      0.13628400           0.0000000
            7      3      0.19133694           0.0000000
            8      3      0.52235466           0.0000000
            9      1      0.52876578           0.0000000
           10      2      0.55149622           0.0000000
           11      1      0.62397742           0.0000000
           12      3      0.73116091           0.0000000
           13      1      1.04616882           0.0000000
           14      1      1.88402305           0.0000000
           15      4      1.92286966           0.0000000
           16      2      2.12532394           0.0000000
           17      3      2.18404917           0.0000000
           18      1      2.28261839           0.0000000
           19      3      2.37902858           0.0000000
           20      3      2.69431254           0.0000000
           21      1      2.70971740           0.0000000
           22      2      2.71654746           0.0000000
           23      1      3.05834268           0.0000000
           24      3      3.25415102           0.0000000
           25      2      3.54107441           0.0000000
           26      1      3.55600364           0.0000000
           27      4      3.59173828           0.0000000
           28      1      3.83258378           0.0000000
           29      1      4.79289351           0.0000000
           30      3      5.12382812           0.0000000
           31      1      7.71922625           0.0000000
           32      1     47.55358026           0.0000000
          ===============================================
\end{verbatim}
}
To perform a full-valence space calculation, comprising 7 primary
orbitals, with the 4a$_{1}$  and 2b$_{2}$ virtual
MOs (with sequence numbers 6 and 7  respectively) 
included in the active space, would require the
following ORBITAL data:

{
\footnotesize
\begin{verbatim}
           ORBITAL
           DOC1 DOC1 DOC3 DOC1 DOC2 UOC1 UOC3
           END
\end{verbatim}
}
Note that it is possible to abbreviate the data specification when
successive orbitals of identical symmetry and type are involved.
This is achieved by preceding the orbital tag with an integer
depicting the number of repeated orbitals. Thus the data above
may be presented thus:

{
\footnotesize
\begin{verbatim}
           ORBITAL
           2DOC1 DOC3 DOC1 DOC2 UOC1 UOC3
           END
\end{verbatim}
}
To freeze the O1s orbital as the SCF orbital, 
we would present the sequence
{
\footnotesize
\begin{verbatim}
           ORBITAL
           FZC1 DOC1 DOC3 DOC1 DOC2 UOC1 UOC3
           END
\end{verbatim}
}
and to maintain the double-occupancy of the orbital, while
enabling it to relax, would require the sequence:

{
\footnotesize
\begin{verbatim}
           ORBITAL
           COR1 DOC1 DOC3 DOC1 DOC2 UOC1 UOC3
           END
\end{verbatim}
}
The following sequence would be used to extend the space to include
the 5a$_{1}$ and 2b$_{1}$ orbitals:
{
\footnotesize
\begin{verbatim}
           ORBITAL
           COR1 DOC1 DOC3 DOC1 DOC2 UOC1 UOC3 UOC1 UOC2
           END
\end{verbatim}
}
Note that it is not required that all orbitals in the active space 
appear first in the
input orbital set, the specified sequence being derived automatically
from the input MOs.
Based on an RHF calculation on the X$^{2}B_{1}$ state of the
\waterp\ ion, then a full-valence calculation, with frozen O1s, would
be performed under control of the following ORBITAL data:

{
\footnotesize
\begin{verbatim}
           ORBITAL
           2DOC1 DOC3 DOC1 ALP2 UOC1 UOC3
           END
\end{verbatim}
}

\subsubsection{ORBITAL - Example 2}
Assume the following set of input MOs from a TZVP--RHF
calculation on the X$^{3}B_{1}$ state of methylene:

{
\footnotesize
\begin{verbatim}
                      TOTAL  ENERGY         -38.9321296186
          ===============================================
          M.O.  IRREP  ORBITAL ENERGY   ORBITAL OCCUPANCY
          ===============================================
            1      1    -11.24483142           2.0000000
            2      1     -0.85646872           2.0000000
            3      3     -0.59962511           2.0000000
            4      1     -0.47810023           1.0000000
            5      2     -0.40348167           1.0000000
            6      1      0.16128671           0.0000000
            7      3      0.24034249           0.0000000
            8      3      0.28796496           0.0000000
            9      1      0.30384049           0.0000000
           10      2      0.32139063           0.0000000
           11      1      0.45957080           0.0000000
           12      1      0.63630157           0.0000000
           13      3      0.70724008           0.0000000
           14      3      1.40064194           0.0000000
           15      1      1.44828705           0.0000000
           16      2      1.47979149           0.0000000
           17      4      1.54845894           0.0000000
           18      1      1.69050079           0.0000000
           19      2      1.73938256           0.0000000
           20      1      1.78688113           0.0000000
           21      1      1.94167219           0.0000000
           22      3      1.94783841           0.0000000
           23      3      2.30755067           0.0000000
           24      2      2.48250909           0.0000000
           25      1      2.48720842           0.0000000
           26      4      2.73866260           0.0000000
           27      3      2.93545830           0.0000000
           28      1      2.99931959           0.0000000
           29      3      3.66924826           0.0000000
           30      1      3.71124278           0.0000000
           31      1      4.85256025           0.0000000
           32      1     26.27807526           0.0000000
          ===============================================
\end{verbatim}
}
A full-valence space calculation on the X$^{3}B_{1}$ state, with
frozen C1s, would be controlled thus:
{
\footnotesize
\begin{verbatim}
           ORBITAL
           2DOC1 DOC3 ALP1 ALP2 UOC1 UOC3
           END
\end{verbatim}
}
The corresponding calculation on the $^{1}B_{1}$ state,
featuring singlet coupling of the 3a$_{1}$ and 1b$_{1}$ MOs,
would require use of the ALP and BET tags, thus

{
\footnotesize
\begin{verbatim}
           ORBITAL
           2DOC1 DOC3 ALP1 BET2 UOC1 UOC3
           END
\end{verbatim}
}
remembering to change (or remove) the MULT 3 specification~!\\

\subsection[CANONICAL]{CANONICAL}
The CANONICAL directive may be used to control the routing
of MCSCF natural orbitals to the Dumpfile, and
to specify the canonicalisations in effect for the three
categories of orbital --  core, active and secondary. The directive
consists of a single data line read to the variables
TEXT, ISECNO, TEXTC, TEXTA, TEXTS using format (A,I,3A).
\begin{itemize}
\item TEXT should be set to the character string CANONICAL.
\item ISECNO should be set to a section number on the Dumpfile
for output of the MCSCF natural orbitals.
\item TEXTC is a character string for controlling
the canonicalisation of the core orbitals. In generating
the optimum MOs for subsequent use in CI calculations, 
TEXTC should be set to the string FOCK.
\item TEXTA is a character string for controlling
the canonicalisation of the active orbitals. In generating
the MCSCF natural orbitals,
TEXTA should be set to the string DENSITY.
\item TEXTS is a character string for controlling
the canonicalisation of the secondary orbitals. In generating
the optimum MOs for subsequent use in CI calculations, 
TEXTS should be set to the string FOCK.
\end{itemize}
{\bf Example}
{
\footnotesize
\begin{verbatim}
          CANONICAL 10 FOCK DENSITY FOCK
\end{verbatim}
}
would be used to route the natural orbitals to section 10 on the
Dumpfile. Note that the above settings are now applied in default
(Version 6.3 onwards), so the user need only present the CANONICAL
directive to override these defaults e.g. a different section for the
MCSCF natural orbitals.

\subsection[PRINT]{PRINT}
The PRINT directive may be used to increase
the default MCSCF output, and consists of a single data line with
the character string PRINT in the first data field. Subsequent
data fields may comprise any combination of the following parameters:

\begin{tabular}{ll}
\\ \hline \hline
NATORB     & 1-particle density matrix and natural orbitals \\
ORBITALS   & Molecular orbital coefficient array \\
CIVECTOR   & The CI vector \\
VIRTUALS   & Print virtual MOs in addition to core and active  \\ \hline\hline
\end{tabular}

Note that the default PRINT settings (Version 6.3 onwards) correspond
to presenting the data line

{
\footnotesize
\begin{verbatim}
          PRINT ORBITALS VIRTUALS NATORB
\end{verbatim}
}

\clearpage
% End MCSCF Section

% MASSCF Section
\section[Directives Controlling MASSCF Calculations]{Directives Controlling MASSCF Calculations}

This section describes the parallel Multiple Active Space SCF (MASSCF) code 
in GAMESS-UK. The MASSCF method draws upon a mature body of literature 
covering techniques for constructing and optimizing MCSCF 
wavefunctions \cite{mcscf_background}. MASSCF is shorthand for ORMAS-MCSCF 
using the full Newton-Raphson (full-NR) orbital optimization 
technique \cite{fullnr}.  MASSCF proceeds via a two-step decoupled approach 
in which the CI and orbital update problems are solved separately.

ORMAS (Occupation Restricted Multiple Active Space) allows multiple 
active spaces to be defined governed by certain occupation rules or 
restrictions \cite{ormas}. The machinary of ORMAS then generates all possible 
determinants satisfying the occupation restrictions. The usefulness 
of this approach is twofold. Firstly, a single compact notation generalizes 
to traditional CI wave function types such as full CI as well as more 
novel situations involving more than two orbital spaces. Secondly, 
the use of multiple small active spaces has been shown to accurately 
recover the quality of much larger full CI calculations at a fraction 
of the cost \cite{ormas2}. In this way ORMAS may be viewed as a powerful 
tool for generating N-particle wave functions.

A MASSCF calculation is invoked by the directive,

{\footnotesize
\begin{verbatim}
         SCFTYPE MASSCF
\end{verbatim}
}

The above keywords by themselves are insufficient to specify a MASSCF 
calculation. MASSCF requires minimal information regarding the numbers 
of electrons and orbitals involved. In most cases, a keyword is to be 
followed by one or more integers, or a floating-point number in any 
standard representation.\\

\subsection[Basic MASSCF input keywords and their parameters]{Basic MASSCF input keywords and their parameters}

The minimum requirements for defining a MASSCF calculation are as 
follows,

\begin{itemize}
\item  MASSCF -- Keyword identifying the MASSCF input clause 
(this is a flag, no numerical parameter needed). 
\item  NCORE  -- Total number of orbitals doubly occupied in all
determinants.  
\item  NACT   -- Total number of active orbitals.  
\item  NELS   -- Total number of active electrons.  
\end{itemize}

Additional input parameters controlling convergence of the full 
Newton-Raphson solver are as follows,

\begin{itemize}
\item  TOLENG -- Convergence tolerance on the total energy 
(Default value is 1.0d-10).
\item  ACURCY -- Newton-Raphson and Davidson convergence
(Default value is 1.0d-5).
\item  DAMP   -- Newton-Raphson damping factor (Default value is 0.0).
\item  MAXMAS -- Maximum allowed MASSCF iterations (Default is 30).
\item  NFRZ   -- Number of frozen orbitals (Default is 0, maximum is 20).
\item  MOFRZ  -- A list of NFRZ integers specifying the orbitals to be 
kept frozen throughout the MASSCF calculation.
\item  NOROT  -- Number of frozen rotations (Default is 0, maximum is 20).
\item  NROTAB -- A list of NOROT integer-pairs specifying the orbital 
rotations to be kept frozen throughout the MASSCF calculation.
\item  MXPN   -- Maximum number of Davidson expansions (Default is 10).
\item  MXIT   -- Maximum number of Davidson iterations (Default is 30).
\item  FCORE  -- Option to freeze all core orbitals 
(this is a flag, no numerical parameter needed). 
\item  SRSO   -- Perform single-reference second order (SRSO) calculation. 
SRSO is useful in cases where an RHF or ROHF wavefunction is difficult to 
converge and the more powerful Newton-Raphson solver in MASSCF could be 
successful. A single SCF iteration is executed to provide orbitals and the 
CI step executes trivially only to provide the 1- and 2-particle densities 
necessary to form the orbital hessian matrix
(this is a flag, no numerical parameter needed). 
\end{itemize}

By themselves, the above input specifications define the equivalent of 
a CASSCF calculation using the full-NR method on the ground state. 
In order to make full use of the capabilities of the ORMAS code, 
the following input is required.\\

\subsection[ORMAS input keywords and their parameters]{ORMAS input keywords and their parameters}

There are no sensible defaults for the next four inputs but in their 
absense the ORMAS code will perform a full CI calculation, as described above. 
That is, the defaults are NSPACE 1; NORB [NACT]; MINE [NELS]; MAXE [NELS]; 
meaning all active orbitals are in one partition.

\begin{itemize}
\item  NSPACE  -- Number of orbital groups you wish to define.
\item  NORB    -- A list of NSPACE integers. These specify the 
number of orbitals in each active space or group.  
\item  MINE    -- A list of NSPACE integers. These specify the
minimum numbers of electrons that must always occupy the orbital groups.  
In other words, MINE(I) is the minimum number of electrons that can 
occupy space I in any of the determinants.  
\item  MAXE    -- A list of NSPACE integers. These specify the
maximum numbers of electrons that must always occupy the orbital groups.  
In other words, MAXE(I) is the maximum number of electrons that can 
occupy space I in any of the determinants.
\item  ROOT    -- Selects the root of the CI problem to solve for. 
This enables the selection of ground and (singly) excited states, 
the default is ROOT=0 for the ground state, ROOT=1 will select the 
first excited state and so on (Default is 0).
\item  NSTATE  -- Number of states to average over (Default is 1).
\item  WSTATE  -- A list of NSTATE integers specifying the weights of 
the states to average over (default entry is 1.0 for the first state).
\item  SYMSTATE -- Symmetry of the target state (Default is the input 
symmetry).
\end{itemize}

Additional input parameters controlling convergence of the ORMAS 
CI solver are as follows,

\begin{itemize}
\item  KEEPVEC -- Number of CI vectors to keep for next CI step
(Default is 1 so the current solution provides the guess for the next 
CI step).
\item  CIGUESS -- Dimension of CI guess Hamiltonian (Default is 300).
\item  ITERCI  -- Maximum number of Davidson iterations to solve CI
(Default is 100).
\item  CRIT    -- Energy convergence criterion (Default value is 1.0d-5).
\item  MAXP    -- Maximum number of Davidson expansions in CI
(Default is 10).
\end{itemize}

The following examples serve to illustrate the construction of CI 
wave functions using ORMAS.\\

\subsubsection{MASSCF - Example 1}
{
\footnotesize
\begin{verbatim}
          TITLE
          H2O  - MASSCF   - 3-21G BASIS
          ZMATRIX ANGSTROM
          O
          H 1 OH
          H 1 OH 2 HOH
          VARIABLES
          OH  0.956
          HOH 104.5 
          END
          SCFTYPE MASSCF
          MASSCF
            NCORE  1
            NACT   8
            NELS   8
          END
          ENTER
\end{verbatim}
}

This is entirely equivalent to-

\subsubsection{MASSCF - Example 2}
{
\footnotesize
\begin{verbatim}
          TITLE
          H2O  - MASSCF   - 3-21G BASIS
          ZMATRIX ANGSTROM
          O
          H 1 OH
          H 1 OH 2 HOH
          VARIABLES
          OH  0.956
          HOH 104.5 
          END
          SCFTYPE MASSCF
          MASSCF
            NCORE  1
            NACT   8
            NELS   8
            NSPACE 1
            NORB   8
            MINE   8
            MAXE   8
          END
          ENTER
\end{verbatim}
}

\subsubsection{MASSCF - Example 3}
In this example, up to double excitations (CISD) are allowed into a 
space containing eight virtual orbitals.

This case is difficult to converge, so the maximum number of MASSCF iterations
needs to be increased.

{
\footnotesize
\begin{verbatim}
          TITLE
          HCN  6-31G** MASSCF ...RHF geometry 
          GEOMETRY ANGSTROM
            0.0 0.0 -1.0589956  1.0  H
            0.0 0.0  0.0000000  6.0  C
            0.0 0.0  1.1327718  7.0  N
          END
          BASIS 6-31G**
          SCFTYPE MASSCF
          MASSCF
            NCORE  2
            NELS  10
            NACT  13
            NSPACE 2
            NORB   5  8
            MINE   8  0
            MAXE  10  2
	    MAXMAS 100
          END
          ENTER
\end{verbatim}
} 

\subsubsection{MASSCF - Example 4}
One need not be limited to conventional CI wave function types as the 
following example serves to illustrate.

{
\footnotesize
\begin{verbatim}
          TITLE
          HCN  6-31G** MASSCF ...RHF geometry 
          GEOMETRY ANGSTROM
            0.0 0.0 -1.0589956  1.0  H
            0.0 0.0  0.0000000  6.0  C
            0.0 0.0  1.1327718  7.0  N
          END
          BASIS 6-31G**
          SCFTYPE MASSCF
          MASSCF
            NCORE  2
            NACT  11
            NELS  10
            NSPACE 3
            NORB   5  4  2
            MINE   4  0  0
            MAXE  10  4  2
	    MAXMAS 1000
          END
          ENTER
\end{verbatim}
}

In the above example, up to quadruple excitations are allowed into the 
first virtual space and up to double excitations in the second virtual 
space.

Again, this calculation is extremely slow to converge so the maximum number of
MASSCF iterations must be increased to nearly 1000 to ensure convergence.

\clearpage
% End MASSCF Section

\section[Directives Controlling DFT Calculations]{Directives Controlling DFT Calculations}

\subsection{Introductory remarks}

Before describing the DFT-specific input, we briefly outline some 
background material that users should be aware of before attempting to
use the Density Functional Theory module within GAMESS-UK.
\begin{enumerate}
\item {\bf The functionals}:
An essential result from the paper by Hohenberg and Kohn \cite{hohen64}
was that DFT would yield the exact ground state energy and electron
density if the exchange-correlation functional was known. In practice
the exact functional is unknown but one may try some approximate form.
This has led to a extensive search for functionals with
new variations being published on a regular basis.  Because the quality of
the results depends critically on the functional selecting a suitable 
form will be a vital factor in using the module.

\item {\bf The integration grids}:
Another issue related to the functionals stems from their form; most
functionals are such that they can not be integrated analytically over
all space. Therefore, the exchange-correlation energy can be evaluated
only through numerical integration.

It was found that this numerical integration could only be successful if
the integration grid is adapted to the particular features of the
molecular density. These features are that the density is high and
nearly spherically symmetric near the nuclei. Between the nuclei the
density has less symmetry and is smaller. However because most of the
chemistry depends on the density between the nuclei accurate
integration in that region is essential.

To devise integration grids adapted to these features the atoms of a
molecule were taken as the central points. Each nucleus would be the
center of a set of spherical grids with ever larger radii. The simplest
way to obtain such a grid would be to take the Cartesian product of a
radial grid with a spherical grid.  But more advanced schemes can be
engineered.

Once the atomic grids have been constructed they have to be merged into
a molecular grid. To avoid artifacts from the finite size of the atomic
grids it is essential that grid points be faded out if they get to
close to another atom.  So called weighting functions were designed for
this purpose.

So for proper integration the selection of angular grids, radial grids and
weighting functions have to be addressed.

\item {\bf Improving integration efficiency}:
Although the above approach properly defines the molecular integration grid
the efficiency of applying this grid can be improved through 2 strategies:
\begin{enumerate}
\item Screening -- This is based on removing grid points or functions
at grid points that contribute little to the exchange-correlation
energy from the calculation at the earliest opportunity;
\item Pruning -- This is based on replacing 2 close grid points by 1
new grid point.  
\end{enumerate}
\end{enumerate}


\subsection{The DFT Directive and Default Settings}

In common with most Density Functional programs, the DFT module within
GAMESS--UK is implemented as a modified Hartree-Fock program with the
exchange term in the Hartree-Fock equations replaced by the
exchange-correlation term \cite{hohen64}.  Thus input for a DFT calculation is
essentially that for the closed-shell RHF or UHF module, with
additional keywords that control the DFT specific features.

In the simplest case, the user need just introduce a single data
with the character string CDFT or DFT in the first data field to
request a DFT rather than HF calculation, thus input for a closed-shell
DFT calculation would appear as follows:

{
\footnotesize
\begin{verbatim}
          TITLE
          H2CO - 3-21G  CLOSED SHELL DFT (B-LYP DEFAULT QUADRATURE)
          SUPER OFF NOSYM
          ZMATRIX ANGSTROM
          C
          O 1 1.203
          H 1 1.099 2 121.8
          H 1 1.099 2 121.8 3 180.0
          END
          DFT
          ENTER
\end{verbatim}
}
while the corresponding UHF data for performing an open-shell
unrestricted UKS calculation would appear thus,

{
\footnotesize
\begin{verbatim}
          TITLE
          H2CO+ - 2B2 - DEFAULT 3-21G BASIS - UKS CALCULATION
          CHARGE 1
          MULT 2
          ZMATRIX ANGSTROM
          C
          O 1 1.203
          H 1 1.099 2 121.8
          H 1 1.099 2 121.8 3 180.0
          END
          SCFTYPE UHF
          DFT
          ENTER
\end{verbatim}
}
The directive DFT thus "switches on" the DFT specific modifications to
the Hartree-Fock scheme. Leaving the directive out would yield the
corresponding Hartree-Fock input.

If, as in the above, the DFT module is switched on without specifying
any options then the following functional and quadrature settings will
apply;
\begin{itemize}
\item the Becke (1988) exchange functional \cite{becke88}
\item the Lee, Yang and Parr (LYP) correlation functional \cite{lyp}
\item quadrature grids designed to obtain a relative error of less than
1.0e-6 in the number of electrons per atom. These grids are constructed
from the logarithmic radial grid \cite{mura96,murray93} and
Lebedev angular grid \cite{lebe99}, using the MHL8SSF weighting scheme with
screening and MHL angular grid pruning \cite{murray93}.
Note that this choice corresponds to the "QUADRATURE MEDIUM" setting
described below.
\item the gradient of the energy will be evaluated without considering the
gradient of the quadrature weights and grid points, this corresponds to
"GRADQUAD OFF".
\end{itemize}

\subsection{DFT Directive Options}

The role of the DFT directive is twofold, (i) to trigger a DFT rather
than HF calculation, and (ii) to provide a mechanism for overriding the
default DFT functional and quadrature settings.  The latter is achieved
by specifying the DFT options described below on one or more data
lines, each containing the character string DFT in the first data
field; the user may present as many data lines as desired in specifying
these options, providing the mechanism for presenting long option lists
over several lines.  Note that the DFT data lines should be presented
after both RUNTYPE and SCFTYPE directives (if present), and before the
VECTORS directive (if present).

Thus the default DFT specifications invoked by the data input above may
also be invoked by explicit specification, thus

{
\footnotesize
\begin{verbatim}
          TITLE
          H2CO - 3-21G  CLOSED SHELL DFT (B-LYP DEFAULT QUADRATURE)
          ZMATRIX ANGSTROM
          C
          O 1 1.203
          H 1 1.099 2 121.8
          H 1 1.099 2 121.8 3 180.0
          END
          DFT B-LYP QUADRATURE MEDIUM
          ENTER
\end{verbatim}
}
or by specifying the functional and quadrature settings on separate
DFT data lines, thus

{
\footnotesize
\begin{verbatim}
          TITLE
          H2CO - 3-21G  CLOSED SHELL DFT (B-LYP DEFAULT QUADRATURE)
          ZMATRIX ANGSTROM
          C
          O 1 1.203
          H 1 1.099 2 121.8
          H 1 1.099 2 121.8 3 180.0
          END
          DFT BLYP 
          DFT QUADRATURE MEDIUM
          ENTER
\end{verbatim}
}
or even,

{
\footnotesize
\begin{verbatim}
          TITLE
          H2CO - 3-21G  CLOSED SHELL DFT (B-LYP DEFAULT QUADRATURE)
          ZMATRIX ANGSTROM
          C
          O 1 1.203
          H 1 1.099 2 121.8
          H 1 1.099 2 121.8 3 180.0
          END
          DFT BECKE88
          DFT LYP
          DFT QUADRATURE MEDIUM
          ENTER
\end{verbatim}
}

\subsection{Specification of Functionals}

As described above, The default functional used in the current DFT
implementation is the so-called B-LYP functional, employing the Becke88
exchange functional \cite{becke88} and Lee, Yang and Parr correlation
(LYP) correlation energy functional \cite{lyp}. Over-riding this
default may be achieved through the following DFT keywords:
\begin{itemize}

\item NULL\_X;
The keyword NULL\_X selects the null exchange functional.
Obviously this is useful only in very special cases.

\item HF\_X; 
The keyword HF\_X selects the Hartree-Fock exchange term as the
exchange functional.

\item LDA\_X;
The keyword LDA selects the LDA exchange energy functional.

\item B88\_X; 
The keyword B88\_X selects the default Becke'88 exchange functional
This is a gradient-corrected exchange energy functional with correct
{\em 1/r} asymptotic behaviour of the exchange-energy density \cite{becke88}.

\item B3\_X; 
The keyword B3\_X selects the Becke3 exchange functional
This is the three parameter gradient-corrected hybrid exchange energy functional
with the Becke'88 functional as one of its components \cite{becke93}.

\item B97\_X; 
The keyword B97\_X selects the Becke97 exchange functional \cite{becke97}.

\item EDF1\_X; 
The keyword EDF1\_X selects the exchange functional of Empirical Density 
Functional one \cite{adamson98}.

\item FT97A\_X; 
The keyword FT97A\_X selects the Filatov-Thiel exchange functional variant A
\cite{filatov97x}.

\item FT97B\_X or FT97\_X; 
The keywords FT97B\_X or FT97\_X select the recommended Filatov-Thiel exchange
functional variant B \cite{filatov97x}.

\item PBE\_X;
The keyword PBE\_X selects the Perdew-Burke-Ernzerhof exchange functional
\cite{perdew96}.

\item PW91\_X;
The keyword PW91\_X selects the Perdew-Wang'91 exchange functional
\cite{perdew92}.

\item NULL\_C;
The keyword NULL\_C selects the NULL correlation functional.

\item B95 or B95\_C;
The keyword B95 or B95\_C select the Becke'95 meta correlation functional
\cite{becke96}.

\item EDF1\_C;
The keyword EDF1\_C selects the correlation part of the Empirical Density
Functional one \cite{adamson98}.

\item FT97\_C;
The keyword FT97\_C selects the Filatov-Thiel correlation functional
\cite{filatov97c}

\item LYP or LYP\_C;
The keywords LYP or LYP\_C select the default Lee-Yang-Parr correlation 
energy functional \cite{lyp}.

\item P86 or P86\_C;
The keywords P86 or P86\_C select the Perdew'86 gradient corrected
correlation functional \cite{perdew86}.

\item PBE\_C;
The keyword PBE\_C selects the Perdew-Burke-Ernzerhof correlation functional
\cite{perdew96}.

\item PZ81 or PZ81\_C;
The keywords PZ81 or PZ81\_C select the Perdew-Zunger local density 
correlation functional \cite{perdew81}.

\item PW91\_C;
The keyword PW91\_C selects the Perdew-Wang'91 correlation functional
\cite{perdew92}.

\item PW92 or PW92\_C;
The keywords PW92 or PW92\_C selects the Perdew-Wang'92 local density
correlation functional \cite{perdew92a}.

\item VWN, VWN5 or VWN\_C;
The keywords VWN, VWN5 or VWN\_C select the recommended Vosko-Wilk-Nusair local
density correlation functional \cite{vosko80}.

\item VWNRPA, VWN5RPA or VWNRPA\_C;
The keywords VWNRPA, VWN5RPA or VWNRPA\_C select the Vosko-Wilk-Nusair local 
density correlation functional with the RPA parametrisation \cite{vosko80}.

\item NULL;
The keyword NULL selects the null exchange-correlation energy. 
Obviously this is should be used only in very special cases.

\item B3LYP;
The keyword B3LYP selects the infamous hybrid exchange-correlation functional
proposed by Stephens et al. \cite{stephens94}. As Stephens et al. did not use the
recommended VWN functional as one of the components this functional has 
attracted much controversy. The current understanding is that the functional 
employs VWN3 \cite{hertwig97}.

\item B1B95;
The keyword B1B95 selects a meta hybrid exchange-correlation functional build
from Hartree-Fock exchange (28\%), the Becke'88 exchange functional (72\%) and
the Becke'95 meta correlation functional \cite{becke96}.

\item B97;
The keyword B97 selects the Becke'97 hybrid exchange-correlation functional
\cite{becke97}.

\item B97-1;
The keyword B97-1 selects the hybrid exchange-correlation functional of the same
form as B97 but reoptimised by Hamprecht et al. \cite{hamprecht98,becke97}.

\item B97-2;
The keyword B97-2 selects the hybrid exchange-correlation functional of the same
form as B97 but reoptimised by Wilson et al. \cite{wilson01,becke97}.

\item B97-3;
The keyword B97-3 selects the hybrid exchange-correlation functional of the same
form as B97 but reoptimised by Keal et al. \cite{keal05,becke97}.

\item B97-D;
The keyword B97-D selects the hybrid exchange-correlation functional of the same
form as B97 but reoptimised by Grimme et al. \cite{grimme06,becke97}.

\item BB1K;
The keyword BB1K selects a meta hybrid exchange-correlation functional build from
Hartree-Fock exchange (42\%), the Becke'88 exchange functional (58\%) and the
Becke'95 meta correlation functional \cite{zhao04}.

\item BB95;
The keyword BB95 selects the Becke'88 exchange functional and the Becke'95
meta correlation functional \cite{becke96}.

\item BLYP;
The keyword BLYP selects the Becke88 exchange energy functional
\cite{becke88} and the Lee, Yang and Parr correlation energy functional
\cite{lyp}.

\item BP86;
The keyword BP86 selects the Becke88 exchange energy functional
\cite{becke88} and the Perdew 1986 gradient corrected correlation functional
\cite{perdew86}.

\item EDF1;
The keyword EDF1 selects Empirical Density Functional One as proposed by 
Adamson et al. \cite{adamson98}.

\item FT97;
The keyword FT97 selects the Filatov, Thiel gradient corrected 
exchange-correlation energy functional \cite{filatov97c,filatov97x}. 
This functional comprises 
the Filatov, Thiel correlation energy functional and the exchange energy 
functional variant B.

\item HCTH or HCTH93;
The keywords HCTH or HTCH93 select the Hamprecht-Cohen-Tozer-Handy
exchange-correlation energy functional fitted against a training set of
93 molecules \cite{hamprecht98}.

\item HCTH120;
The keyword HCTH120 selects the Hamprecht-Cohen-Tozer-Handy
exchange-correlation energy functional fitted against a training set of
120 molecules \cite{boese00}.

\item HCTH147;
The keyword HCTH147 selects the Hamprecht-Cohen-Tozer-Handy
exchange-correlation energy functional fitted against a training set of
147 molecules \cite{boese00}.

\item HCTH407;
The keyword HCTH407 selects the Hamprecht-Cohen-Tozer-Handy
exchange-correlation energy functional fitted against a training set of
407 molecules \cite{boese01}.

\item PBE;
The keyword PBE selects the Perdew-Burke-Enrzerhof gradient corrected 
exchange-correlation functional \cite{perdew96}.

\item REVPBE;
The keyword REVPBE selects the Zhang-Yang revised PBE exchange-correlation
function \cite{zhang98}.

\item RPBE;
The keyword RPBE selects the Hammer-Hansen-N{\/o}rskov alternative revised
PBE functional that locally satisfies the Lieb-Oxford criterion
\cite{hammer99}.

\item PW91;
The keyword PW91 selects the Perdew-Wang'91 gradient corrected
exchange-correlation functional \cite{perdew92}.

\item SVWN;
The keyword SVWN selects the LDA exchange functional and the 
Vosko-Wilk-Nusair local density correlation functional \cite{vosko80}.

\end{itemize}

\subsection{Specification of Dispersion Corrections}
\label{DFT-dispersion-corr}

A problem frequently encountered with DFT calculations is that long range
correlation effects such as dispersion or Van der Waals interactions are not
properly described. This leads to problems where these relatively weak 
dispersion forces are important such as in DNA base pair stacking or the 
interaction between aromatic molecules. So far attempts to rigorously address
these problems have had limited success. In response to this empirical 
approaches to correct for the lack of dispersion have been suggested. GAMESS-UK
supports two versions of one of these approaches named
DFT-D~\cite{grimme04,grimme06,antony06}. 
Both these approaches are based on a simple
model for the dispersion energy of two interacting atoms
\begin{equation}
D_{ij}(R_{ij}) = C_6^{ij}\frac{f_{\mathrm{damp}}(R_{ij})}{R^{6}_{ij}}.
\end{equation}
The function $f_{\mathrm{damp}}$ is simply chosen such that it goes to zero
when $R_{ij}$ goes to zero to prevent that atoms collapse onto eachother.
The above expression requires a $C_6$ coefficient for every pair of chemical
elements. To reduce the number of these required parameters the pair $C_6$
coefficients are approximated and expressed in terms of $C_6$ coefficients of
the elements. The current implementation supports two models for this:
\begin{description}
\item[The {\em Average} $C_6$ pair model]
     In this model the pair $C_6$ coefficient is approximated simply as the
     average of the elemental $C_6$ coefficients~\cite{grimme04}:
     \begin{equation}
        C_6^{ij} = 2\frac{C_6^i~C_6^j}{C_6^i + C_6^j}
     \end{equation}
     This is presently the default.
\item[The {\em Geometric Mean} $C_6$ pair model]
     In this model the pair $C_6$ coefficient is approximated as the
     geometric mean of the elemental $C_6$
     coefficients~\cite{grimme06,antony06}:
     \begin{equation}
        C_6^{ij} = \sqrt{C_6^i~C_6^j}
     \end{equation}
     This is currently considered as the most accurate model. It also supports
     the most chemical elements through published atomic $C_6$
     coefficients~\cite{grimme06}.
\end{description}

\subsubsection{The DISPERSION directive}

In the DFT input section the basic properties of the dispersion correction may
be controlled throught the DISPERSION directive
(for a much more detailed control over these correction see 
section~\ref{vanderwaals-corr}). This directive may consist of
one or two data fields read to the variables TEXT and TEXTOPT using the format
(A,A)
\begin{itemize}
\item TEXT should be set to the character string DISPERSION;
\item TEXTOPT is an optional field which may be set to either of
      \begin{itemize} 
      \item ON to switch the dispersion corrections on (although this is 
            currently implied in specifying the DISPERSION correction);
      \item OFF to explicitly swith the dispersion correction off;
      \item AVERAGE to select the {\em average} $C_6$ coefficient pair model
            (presently the default);
      \item GEOMETRIC to select the {\em geometric mean} $C_6$ coefficient
            pair model
      \end{itemize}
\end{itemize}
The following three examples all result in using the dispersion corrections
in accordance with the default settings.

{\bf Examples}

{
\footnotesize
\begin{verbatim}
          DFT DISPERSION

          DFT DISPERSION ON

          DFT DISPERSION AVERAGE
\end{verbatim}
}


\subsection{Specification of Integration Grids}

While a large number of options are available in specifying possible
integration grids (see below), the inexperienced user is strongly
advised to use just the QUADRATURE directive for this purpose.

\subsubsection{The QUADRATURE Directive}

This directive may be used to select a quadrature grid that is designed
to achieve a specified accuracy. The resulting grids are constructed
from the logarithmic radial grid \cite{mura96} and
Lebedev angular grids \cite{lebe99}, using the SSF weighting 
scheme with screening \cite{strat96} and MHL angular grid pruning 
\cite{murray93}.
The directive consists of two data fields, read to the variables TEXT,
ACCU  using format 2A;
\begin{itemize}
\item TEXT should be set to the character string QUADRATURE;
\item ACCU is a keyword used to define the required grid accuracy.
Valid keywords include;
\begin{itemize}
\item LOW - The LOW accuracy grid should only be used for preliminary
studies; it is designed to obtain the total number of electrons from
the density integration with a relative error of 1.0e-4 per atom.
\item MEDIUM - The MEDIUM accuracy grid is designed to obtain a
relative error of less than 1.0e-6 in the number of electrons per
atom.
\item HIGH - The HIGH accuracy grid is designed to obtain a relative
error of less than 1.0e-8 in the number of electrons per atom.
\item VERYHIGH - The VERYHIGH accuracy grid is meant only for benchmark 
calculations.  It is designed to be significantly more accurate than
the high accuracy grid.
\end{itemize}
\end{itemize}
The directive may be omitted when ACCU will be set to the default
MEDIUM quadrature setting.\\

{\bf Example}

{
\footnotesize
\begin{verbatim}
          TITLE
          H2CO - 6-31G  CLOSED SHELL DFT (B3LYP HIGH QUADRATURE)
          ZMATRIX ANGSTROM
          C
          O 1 1.203
          H 1 1.099 2 121.8
          H 1 1.099 2 121.8 3 180.0
          END
          BASIS 6-31G
          DFT B3LYP
          DFT QUADRATURE HIGH
          ENTER
\end{verbatim}
}

Note that the LABEL and ELEMENT keywords discussed in the next subsection
may also be used with the QUADRATURE sub-directive.


\subsection{Detailed Grid Specification}

A number of sub-directives of DFT are available to control the grids to
be used, although it is not expected that these would be routinely
invoked when running the DFT module. These sub-directives include those
for (i) specifying both the angular integration grid (LEBEDEV or
GAUSS-LEGENDRE) and radial grid (EULER-MACLAURIN or LOGARITHMIC), 
(ii) the screening of grid points (SCREEN), 
(iii) an appropriate weighting scheme (WEIGHT) and
(iv) activating angular grid pruning (ANGPRUNE).


\subsubsection{Grid Specification on a Per Atom Basis -- ELEMENT and LABEL}

For optimal control over the integration it is required that the grid
may be modified for each atom separately. For this purpose the 
keywords ELEMENT and LABEL were introduced. The keyword ELEMENT is 
followed by a list of one or more elements and the requested setting
is applied to all atoms of that element. The keyword LABEL works 
similarly to ELEMENT but uses the atom labels as specified in the
geometry. The various specifications are executed in order of
appearance.

{\bf Example}

{
\footnotesize
\begin{verbatim}
          DFT LEBEDEV 302 ELEMENT C H 194 LABEL C1 H2 H4 266
\end{verbatim}
}
This directive sets the angular grid for all atoms to the 302 point 
Lebedev grid, then sets the grid for all carbon and hydrogen atoms to 
the 194 point grid, and finally overrides the grid for all atoms with 
names C1, H2 and H4 giving them the 266 point grid.


\subsubsection{Angular Integration Grid -- LEBEDEV}

The LEBEDEV directive requests the grids of Lebedev for angular
integration \cite{lebe99}.  These grids have been
designed to integrate polynomials on a sphere exactly up to a specific
order.  Grids with 6, 14, 26, 38, 50, 74, 86, 110, 146, 170, 194, 234, 
266, 302, 350, 434, 590, 770, 974 and 1202 points are supported. 
In its simplest form, the directive consists of
two data fields read to the variables TEXT, NPT using format (A,I)
\begin{itemize}
\item TEXT should be set to the character string LEBEDEV
\item NPT is an integer specifying the required number of points
\end{itemize}
It has been noted that close to the nucleus the density is more spherically
symmetric than at larger distances, so that a smaller angular grid can be used
for smaller radii. This capability is provided by an extension to the directive
whereby different angular grid may be specified for different radii. In this
case the LEBEDEV directive comprises the following data fields;
\begin{itemize}
\item TEXT should be set to the character string LEBEDEV
\item pairs of data fields are then presented, each pair characterising a
specific radial zone and read to the variables (NPT$_i$, RZ$_i$) using format 
(I,F), where
\begin{itemize}
\item NPT$_i$ specifies the grid size in the $i$-th radial zone.
\item The floating point values of RZ$_i$ subdivide the radial coordinate
into different zones. The values RZ$_i$ are fractions of the
Bragg-Slater radius \cite{slater64} of the atom. Each zone runs from
RZ$_{i-1}$ to RZ$_i$. The first zone starts at 0, while the last zone runs up 
to infinity.
\end{itemize}
\item NPT is again an integer specifying the required number of points in 
      the outer most zone running up to infinity.
\end{itemize}
Lebedev published the grids with 38, 50, 86, 110, 146, 194, 266, 302, 434,
590, 770, 974, and 1202
points to be exact for polynomials up to orders 9, 11, 15, 17, 19, 23, 27,
29, 35, 41, 47, 53, and 59 respectively.

{\bf Examples}

{
\footnotesize
\begin{verbatim}
          LEBEDEV 302

          LEBEDEV 194 0.1 302 0.5 434

          LEBEDEV LABEL C1 C2 194 0.1 302 0.5 434 ELEMENT CL 590
\end{verbatim}
}


\subsubsection{Angular integration Grid -- GAUSS-LEGENDRE}

The GAUSS-LEGENDRE directive requests a Gauss-Legendre grid for the
angular integration. This grid is based on separating functions on a
sphere into 2 functions of angles, $\theta$, (0 $\leq$ $\theta$ $\leq$
$\pi$) and $\phi$,  (0 $\leq$ $\phi$ $\leq$ 2$\pi$) respectively.  The
total grid size is specified through the number of grid points, NTHETA
in the $\theta$ coordinate.  The number of points in the $\phi$
coordinate will be simply 2 $\times$ NTHETA, so that the total angular
grid size will be 2 $\times$ NTHETA$^2$.  In its simplest form, the
directive consists of two data fields read to the variables TEXT, NTHETA
using format (A,I)
\begin{itemize}
\item TEXT should be set to the character string GAUSS-LEGENDRE, or
more simply, GAUSS;
\item NTHETA is an integer specifying the required number of points
\end{itemize}
In the same way as described above for LEBEDEV grids, it is possible
to specify different angular grids for different radii. In the 
present case the GAUSS-LEGENDRE directive comprises the following data fields;
\begin{itemize}
\item TEXT should be set to the character string GAUSS-LEGENDRE;
\item pairs of data fields are then presented, each pair characterising a
specific radial zone and read to the variables (NTHETA$_i$, RZ$_i$) using format (I,F), where
\begin{itemize}
\item NTHETA$_i$ specifies the grid size in the $i$-th radial zone.
\item The floating point values of RZ$_i$ subdivide the radial coordinate
into different zones. The values RZ$_i$ are fractions of the
Bragg-Slater radius \cite{slater64} of the atom. Each zone runs from
RZ$_{i-1}$ to RZ$_i$. The first zone starts at 0, while the last zone runs up 
to infinity.
\end{itemize}
\item NPT is again an integer specifying the required number of points 
      in the outer most zone running up to infinity.
\end{itemize}

{\bf Examples}

{
\footnotesize
\begin{verbatim}
          GAUSSLEGENDRE 15

          GAUSSLEGENDRE 11 0.1 15 0.5 17

          GAUSSLEGENDRE LABEL C1 C2 11 0.1 15 0.5 17 ELEMENT CL 13
\end{verbatim}
}


\subsubsection{Radial Integration Grid - EULER-MACLAURIN}

The EULER-MACLAURIN directive or shorter EULER selects the
Euler-MacLaurin radial integration grid \cite{murray93}.  The grid size
is specified through the number of grid points, NPT. The directive thus
consists of two data field read to the variables TEXT, NPT using format
(A,I)

\begin{itemize}
\item TEXT should be set to the character string EULER-MACLAURIN or
more simply, EULER;
\item NPT is an integer specifying the required number of points
\end{itemize}

The grid points will be located at
\begin{eqnarray*}
   r_i &=& a \frac{x_i^2}{ 1 - x_i^2} \\
   x_i &=& \frac{i}{\mathrm{NPT} + 1} 
\end{eqnarray*}
where $ 1\leq i \leq \mathrm{NPT}$.
In this expression a is an element dependent scale factor. Note that all points 
with $i > (\mathrm{NPT} + 1)/2$ will have $r_i > a$.  Moreover the most distant
point will be at $r_i = a \times \mathrm{NPT}^2$.  In practice this means that
relatively many points will be far from the nucleus.

{\bf Examples}

{
\footnotesize
\begin{verbatim}
          EULER 45

          EULER LABEL H1 H1 45

          EULER ELEMENT C H 45 LABEL O1 20 
\end{verbatim}
}

\subsubsection{Radial Integration Grid - LOG}

The LOG directive selects the logarithmic radial integration grid \cite{mura96}.
The grid size is specified through the number of grid points, NPT. In addition 
a power M must be specified which is defined below. The directive thus consists 
of three data fields read to the variables TEXT, NPT, M using format (A,I,F)

\begin{itemize}
\item TEXT should be set to the character string LOG;
\item NPT is an integer specifying the required number of points;
\item M is a floating point number defined below
\end{itemize}

The grid points will be located at
\begin{eqnarray*}
   r_i &=& -a \log(1-x_i^{\mathrm{M}}) \\
   x_i &=& \frac{2i-1}{2\times\mathrm{NPT}} 
\end{eqnarray*}
where $ 1\leq i \leq \mathrm{NPT}$,
and $a$ is an element dependent scale factor. The recommended value for M is
3.0 \cite{mura96}.
Note that all points 
with $i > \mathrm{NPT}[1-\exp(-1)]^1/\mathrm{M}+1/2$ will have $r_i > a$.  
This means that with M $= 3.0$ about 85\% of all grid point have $r_i < a$.
In practice this means that this radial grid has a tendency to focus on the 
area close to the atom.

{\bf Examples}

{
\footnotesize
\begin{verbatim}
          LOG 45 3.0

          LOG LABEL H1 H1 45 1.0

          LOG ELEMENT C H 45 3.0 LABEL O1 20 3.0
\end{verbatim}
}


\subsubsection{Scaling Radial Grids -- SCALE}

The SCALE directive provides the ability to scale the radial
grids of all atoms by a uniform factor, FACTOR. This may 
prove helpful in moving points into a sensible range, especially with the 
Euler-MacLaurin radial integration grid \cite{murray93}. The 
directive consists of two data field read to the variables TEXT, FACTOR
using format (A,F)
\begin{itemize}
\item TEXT should be set to the character string SCALE;
\item FACTOR specifies the required scaling factor.
\end{itemize}

{\bf Examples}

{
\footnotesize
\begin{verbatim}
          SCALE 3.0

          SCALE LABEL C1 H1 3.0 LABEL C2 O1 4.0

          SCALE 4.0 LABEL C1 H1 3.0

          SCALE 4.0 ELEMENT C1 H1 3.0
\end{verbatim}
}


\subsubsection{Weighting scheme -- WEIGHT}

The WEIGHT directive allows the user to select a weighting scheme to
combine the atomic integration grids to a molecular integration grid.
The directive consists of two data field read to the variables TEXT,
SCHEME using format 2A.
\begin{itemize}
\item TEXT should be set to the character string WEIGHT
\item SCHEME specifies the required weighting scheme, and should
be set to one the following character strings;
\begin{itemize}
\item BECKE -- The original Becke weighting scheme  \cite{becke88a}
\item BECKESCR --
The Becke weighting scheme \cite{becke88a} with additional screening.
\item HML -- The Murray, Handy and Laming weighting scheme \cite{murray93}.
This scheme differs from the Becke scheme in that it used a different
cell function. It leads to more accurate integrals than the Becke scheme.
\item SSF -- The Stratmann, Scuseria and Frisch weighting scheme \cite{strat96}.
For sufficiently large quadrature grids this scheme seems to be the most
accurate.
\item SSFSCR -- The Stratmann, Scuseria and Frisch weighting scheme 
      \cite{strat96} with screening.
\item MHL4SSF -- The Stratmann, Scuseria and Frisch weighting scheme 
      \cite{strat96} with screening, but employing the cell function by Murray,
      Handy and Laming weighting scheme \cite{murray93} with $m_\mu$ equals 4.
\item MHL8SSF -- The Stratmann, Scuseria and Frisch weighting scheme 
      \cite{strat96} with screening, but employing the cell function by Murray,
      Handy and Laming weighting scheme \cite{murray93} with $m_\mu$ equals 8.
\end{itemize}
\end{itemize}
The screening referred to reduces the cost of the normalisation of the molecular
grid weights. This reduction becomes larger with increasing molecule size.

The weighting scheme is a global option and can not be set on a per atom basis,
i.e. the ELEMENT and LABEL keywords can not be used with the WEIGHT directive.


\subsubsection{Grid Point Screening -- SCREEN}

    SCREEN [PSI PSITOL] [P DENTOL] [RHO RHOTOL] [CONV]

The SCREEN Directive allows the user to activate the screening of grid points.
This may involve the optional specification of a number of tolerances and/or
a request to change dynamically the quadrature grid size according to the
degree of convergence of the calculation.  Following the directive initiator, 
SCREEN, the following data fields may be presented;
\begin{itemize}
\item PSI PSITOL (format A,F) -- 
This criterion is used in generating the radial grids for the atoms. 
Based on this criterion a radius is computed for every atom beyond which
the most diffuse basis function is assumed to be zero. When building the
radial grid all grid points that would end up outside this radius will be
discarded. 

\item P DENTOL (format A,F) -- 
The tolerance for the (spin) density matrix elements. If an element in the
(spin) density matrix has a value smaller than DENTOL the matrix element will
be discarded in the electron density evaluation.

\item RHO RHOTOL (format A,F) -- 
The tolerance for the (spin) density in a batch of grid points. If the
maximal (spin) density in a batch of grid points is less than RHOTOL, the
whole batch will excluded from the functional integration.

\item CONV (format A) -- 
This option switches on the dynamic adaption of the quadrature precision 
with the convergence of the calculation. The idea is that if the Kohn-Sham
orbitals are not very precise than there is no reason to integrate the
exchange-correlation energy very accurately. Through choosing a smaller
quadrature computation can be saved in the early iterations. Along with 
the calculation converging the quadrature is improved. Near the convergence
criterion the full quadrature grid as input will be applied.
\end{itemize}

Note that DENTOL and RHOTOL are global parameters for which LABEL and ELEMENT
can not be used. However PSITOL can be set on a per atom basis.


{\bf Examples}

{
\footnotesize
\begin{verbatim}
          SCREEN 

          SCREEN OFF

          SCREEN P 1.0D-7 PSI 1.0D-5

          SCREEN P 1.0D-7 PSI 1.0D-5 ELEMENT C 1.0D-6 LABEL H1 1.0D-5
\end{verbatim}
}

The use of screening may significantly improve efficiency. 

\subsubsection{Angular Grid Pruning -- ANGPRUNE}

This directive activates angular grid pruning as function of radius,
uses the scheme proposed by Murray, Handy and Laming \cite{murray93}.
This scheme chooses the number of angular grid points according to the
equation
\begin{center}
n$_{theta}$  = min (K$_{theta}$ N$_{theta}$ r/r$_{Bragg}$,N$_{theta}$)
\end{center}
where
\begin{itemize}
\item n$_{theta}$ is the current number of grid points in the theta coordinate,
\item N$_{theta}$ is the maximum number of grid points in the theta coordinate,
\item K$_{theta}$ is some scaling factor which is set to 5 in as suggested by 
      Murray et al.,
\item r$_{Bragg}$ is the Bragg-Slater radius of the atom \cite{slater64}
\item r is the radius of the current angular shell.
\end{itemize}
From the above equation it is clear that this pruning scheme is
designed to be used with the Gauss-Legendre angular grid. When it is
applied to the Lebedev grids the total number of grid points is set to
2n$_{theta}$$^2$ and then truncated to the first smaller sized Lebedev
grid.

{\bf Examples}

{
\footnotesize
\begin{verbatim}
          ANGPRUNE

          ANGPRUNE OFF

          ANGPRUNE AUTO

          ANGPRUNE LABEL C1 C2 ON

          ANGPRUNE LABEL C1 O1 AUTO LABEL C2 H2 OFF
\end{verbatim}
}

\subsubsection{Atom Radii -- RADII}

This directive sets the atomic radii to be used in the grid generation. 
The radii are to be specified in Bohr's.
This is not the same as the SCALE directive which affects only the spacing
between the radial grid points. Changing the atomic radii does the same thing
as SCALE and affects the pruning of the angular grids and affects the atomic
size adjustments in the weighting schemes. The envisaged use of this directive
is mainly to change the default grid on a BQ center which matches that of a 
Carbon atom to one that matches the grid of some other element.

{\bf Examples}

{
\footnotesize
\begin{verbatim}
          RADII 2.0

          RADII LABEL BQ1 BQ2 3.0

          RADII ELEMENT C O 1.5 LABEL C2 BQ2 2.0
\end{verbatim}
}


\subsubsection{Integration grids and BQ centers}

Using the Becke approach for the numerical integration will result
in each atom having an associated grid. However in various calculations
BQ centers will be included in the geometry to specify point charges, 
additional basis functions or both. Whether or not a BQ center should have a 
grid and what the grid parameters should be is not entirely clear yet. 
The approach currently assumed is:
\begin{itemize}
\item A BQ center will be assigned a grid only if it has basis functions
      associated with it. 
\item The default grid parameters are chosen to equate those of Carbon atoms
      according to the current quadrature accuracy setting (low, medium, high
      veryhigh, or sg1).
\item The grid parameters for BQ centers can be changed using the same
      directives as for atoms. In this context the name BQ can be thought of
      as a chemical element like C or H.
\end{itemize}
Although the ELEMENT and LABEL constructs work for BQ centers similar as they 
do for all normal atoms some care is required. In particular it is not allowed
to try and change the grid of a BQ center that does not have one. Therefore
the ELEMENT construct will not work if there is at least one BQ center without
any basis functions.

\subsection[Energy Gradient Evaluation -- GRADQUAD]{Energy Gradient Evaluation -- GRADQUAD}

The GRADQUAD directive controls the form of the energy gradient expression in
a DFT calculation. This directive consists of 2 data fields read to variables 
TEXT, TEXTOPT using format (A,A).
\begin{itemize}
\item TEXT should be set to the character string GRADQUAD 
\item TEXTOPT should be set to 
   \begin{itemize}
   \item ON or YES to include the gradients of the quadrature weights and 
         grid points in the energy gradient evaluation,
   \item OFF or NO to ignore the contributions from the gradient of the 
         quadrature
   \end{itemize}
\end{itemize}
For details see for example Johnson et al. \cite{johnson93}.

\subsection[Coulomb fitting]{Coulomb fitting}

The cost of DFT calculations of medium sized molecules can be reduced
significantly by avoiding the calculation of 4-center 2-electron
integrals.  This can be achieved by choosing a functional without
Hartree-Fock exchange and evaluating the Coulomb energy with an
auxiliary basis set. The basic idea behind this technology is
described by Dunlap {\it et al.}~\cite{dunlap79} and is referred to as
"Coulomb fitting".  Currently Coulomb fitting can be used in energy and
gradient evaluations. The four directives that control this functionality,
JFIT, JFITG, JBAS and SCHWARZ, are described below.

\subsubsection[JFIT and JFITG]{JFIT and JFITG}

The JFIT directive is used to switch on Coulomb fitting; it
is read to the variables TEXT, TEXTA, and IMEM using format (A,A,I).
\begin{itemize}
\item TEXT should be set to the character string JFIT
\item TEXTA may be set to the character string MEMORY. In this case the program
      will try to store as many of the required 3-center 2-electron integrals
      as possible in memory. These integrals will only be calculated once
      during a KS calculation, while those 3-center integrals that do not fit into 
      memory will be recomputed whenever needed.
      The program will estimate how much memory is needed
      for the normal KS operation and set all other memory aside for the 
      3-center integral storage. Although the implementation aims to avoid
      user intervention, it may happen that the memory needed for the normal KS
      activities is underestimated. This results in the calculation aborting
      with an "out of memory" error. In such cases the User may either, (i)
      increase the memory available to the calculation (through use of the
      CORE pre-directive), or (ii) use the IMEM parameter to control memory
      allocation in more detailed fashion.
\item IMEM may be set to a number of words. This should be used only with the
      MEMORY subdirective and only if the program runs out of memory. With the
      MEMORY subdirective the program estimates the amount needed for the normal
      KS operations and reserves all other memory for the 3-center integral 
      storage. If the memory needed for the normal KS operations is
      underestimated IMEM may be set. IMEM words of memory will then
      be added to the amount estimated for the normal KS operations. i.e., the
      3-center integral storage will be reduced by IMEM words.
\end{itemize}

To switch on Coulomb fitting for the energy gradient evaluation, use the 
JFITG directive. This comprises the single character string JFITG.

\subsubsection[JBAS]{JBAS}
To use Coulomb Fitting requires an auxiliary basis set to be specified
for each atom of the molecule. The JBAS directive should be used for
this purpose.  It consists of 2 data fields, read to variables TEXT,
TEXTOPT using format (A,A).

\begin{itemize}
\item TEXT should be set to the character string JBAS
\item TEXTOPT should be set to one of the following;
  \begin{itemize}
  \item GAMESS to initiate explicit basis set specification from
standard input in GAMESS-UK format (see Part 3 and Example 2 below).
  \item NWCHEM to initiate explicit basis set specification from
standard input in NWChem format,
  \item Setting TEXTOPT to one of the strings A1-DGAUSS, A2-DGAUSS,
DEMON or AHLRICHS signals that the fitting basis set is not to be
defined in the input stream, but is to be loaded from the appropriate
library of internal basis sets. A1-DGAUSS or A2-DGAUSS result in the A1
or A2 DGauss fitting sets~\cite{godbout}, DEMON the DeMon fitting
basis~\cite{godbout}, and AHLRICHS the fitted basis sets tabulated by
Ahlrichs and co-workers~\cite{ahlrichs} (see Example 1 below).
\end{itemize}
TEXTOPT may be omitted in which case it is assumed that the basis set is
to be specified in GAMESS-UK format.
\end{itemize}

\subsubsection[SCHWARZ]{SCHWARZ}
Finally, to reduce the number of 3-center integrals, small terms may be 
eliminated using the Schwarz inequality. The SCHWARZ directive which is read
to the variables TEXT, ISCHWARZ using the format (A,I) can be used to set
the tolerance.
\begin{itemize}
\item TEXT should be set to the character string SCHWARZ.
\item ISCHWARZ should be set to an integer value. The Schwarz tolerance will 
      then be set to $10^{-ISCHWARZ}$.
\end{itemize}

{\bf Example 1: Coulomb Fitting Using the A1-DGAUSS Fitting Basis}\\
{
\footnotesize
\begin{verbatim}
          TITLE
          H2CO - 6-31G/BLYP-DFT WITH A1-DGAUSS COULOMB FITTING
          ZMATRIX ANGSTROM
          C
          O 1 CO
          H 1 CH 2 121.8
          H 1 CH 2 121.8 3 180.0
          VARIABLES
          CO 1.203
          CH 1.099
          END
          BASIS 6-31G
          RUNTYPE OPTIMISE
          SCFTYPE DIRECT RHF
          DFT BLYP
          DFT JFIT MEMORY
          DFT SCHWARZ 6
          DFT JBAS A1-DGAUSS
          ENTER
\end{verbatim}
}
{\bf Example 2: Coulomb Fitting with Explicit Specification of the A1-DGAUSS Basis}\\
{
\footnotesize
\begin{verbatim}
          TITLE
          H2CO - 6-31G/BLYP-DFT WITH A1-DGAUSS COULOMB FITTING
          ZMATRIX ANGSTROM
          C
          O 1 CO
          H 1 CH 2 121.8
          H 1 CH 2 121.8 3 180.0
          VARIABLES
          CO 1.203
          CH 1.099
          END
          BASIS 6-31G
          RUNTYPE OPTIMISE
          SCFTYPE DIRECT RHF
          DFT BLYP
          DFT JFIT MEMORY
          DFT SCHWARZ 6
          DFT JBAS 
          #
          # DGauss A1 Coulomb fitting basis (gamess basis set format)
          #
          S H
          1.000000   45.000000000
          S H
          1.000000    7.500000000
          S H
          1.000000    1.500000000
          S H
          1.000000    0.300000000
          S C
          1.000000 1114.000000000
          S C
          1.000000  223.000000000
          S C
          1.000000   55.720000000
          S C
          1.000000   13.900000000
          SP C
          1.000000    4.400000000   1.00000000
          SP C
          1.000000    0.870000000   1.00000000
          SP C
          1.000000    0.220000000   1.00000000
          D C
          1.000000    4.400000000
          D C
          1.000000    0.870000000
          D C
          1.000000    0.220000000
          S O
          1.000000 2000.000000000
          S O
          1.000000  400.000000000
          S O
          1.000000  100.000000000
          S O
          1.000000   25.000000000
          SP O
          1.000000    7.800000000   1.00000000
          SP O
          1.000000    1.560000000   1.00000000
          SP O
          1.000000    0.390000000   1.00000000
          D O
          1.000000    7.800000000
          D O
          1.000000    1.560000000
          D O
          1.000000    0.390000000
          END
          ENTER
\end{verbatim}
}

\section[Controlling the input Orbitals: The VECTORS Directive]{Controlling the input Orbitals: \protect \\ The VECTORS Directive}

Each of the SCF modules requires one or more sets of trial molecular
orbitals or eigenvectors to initiate the iterative process. The analysis
routines also require definition of the input set of orbitals to be
analysed. In both cases the origin of such a set is defined under control
of the VECTORS directive. The syntax and usage of the directive is very
much a function of the status of the computation in hand. We may identify
two differing situations, where the User must either,

\begin{itemize}
\item  define a mechanism for generating the orbitals, or rely on the
default mechanism (ATOMS, see below);
\item nominate a {\em Section}  on either the parent Dumpfile, or some
`foreign' Dumpfile, wherein a suitable set of orbitals may be found.
In contrast to previous versions of GAMESS-UK, which required explicit
specification of these section numbers, the current release provides a
set of default values so that the user may avoid the task of nominating
sections. These defaults, which are a function of SCFTYPE, are summarised
in Table~\ref{table:2}.
\end{itemize}
 

\subsection[Mechanism Specification]{Mechanism Specification}

At the outset of a calculation, with no associated SCF
computation available, the User must either define a mechanism
for generating the trial MOs though keyword specification
on the VECTORS line, or rely on the default mechanism (note
that the mechanism chosen is often a function of basis set).
If presented for such usage, the VECTORS directive comprises a single data line
read to the variables TEXT, ATEXT, BTEXT using format (3A).
\begin{itemize}
\item TEXT should be set to the character string VECTORS
\item ATEXT should be set to the appropriate string (see below)
defining the generate mechanism to be employed.
\item BTEXT is an optional string that may be set to the character
string PRINT when the trial vectors will be printed. If BTEXT is
omitted, no vectors will be sent to the printer
\end{itemize}
Valid ATEXT strings include the following:

\begin{itemize}

\item  {\bf VECTORS ATOMS} : Construct an initial starting guess
based on concatenating the 1-particle density matrices for each
of the component atoms of the molecular system.
The present implementation of ATOMS represents a significant 
improvement over that available in previous releases
of the code, and should normally be the option of choice.
It is now the default option in the absence of the VECTORS directive.

The following extra optims are recognised in addition to a {\bf print}
directive :
\begin{itemize}
\item ALWAYS : The atomic startup is used again in every point in a geometry 
optimisation to determine the start-orbitals instead of the orbitals of the previous 
point.
\item GROUND : The atomscf routines try to use the groundstate of the
atoms; normally an average of their lowest states is used, which is
usually quite adequate.
\item UHF : The atomic startup produces alpha and beta density matrices, 
which are used straightaway in the UHF scf. Therefore :
\begin{itemize}\item This option only makes sense if the two density matrices differ, which can be 
effected, using the CONFIGURATION or SPECIFY subdirective.
\item As the density matrices are not idempotent, the integrated density will not
be correct; Therefore an ACCURACY IGNORE should be given in DFT calculations.
\end{itemize}
\item CONFIGURATION or SPECIFY: Allows the user to specify explicitly the
configuration used for certain atoms; this is useful for
pseudopotentials, when the user supplies the pseudopotentials under
control of the CARDS option or if the user does not like the configuration
chosen by the atomscf program. The configuration(s) are specified for each atom on
subsequent lines following the VECTORS directive, with the data
terminated by END : {\bf ATOM subdirectives} :
\begin{itemize}
\item CONF : specify the required configuration to be used in the atomic scf 
for the ATOM as (e.g.) d5s1. It is worthwhile checking that the required effect is
obtained.
\item DCONF or DENS : specify the configuration to be used in calculating the 
final atomic density matrix.
\item CHARGE (A,F) : specify  the charge for the ATOM. The extra charge is preferably
added to the highest open shell. Otherwise it is spread over the closed shells.

{\bf Example:} 
{
\footnotesize
\begin{verbatim}
         Na charge +1.0
         CL charge -1.0
\end{verbatim}
}
\item SPIN (A,A,I,I) Specify the division of the electrons over the alpha and beta
density matrices for the specified (s,p,d,f) shell.  This directive is only allowed 
(and then really required) if UHF is specified on the VECTORS card; More 
spin directives may given to specify the division for different shells. This directive
overrides specification by the DENS or CHARGE sub-directives.
{\bf Example:} 
{
\footnotesize
\begin{verbatim}
         Fe1  CONF d5s1 DENS d5s1 SPIN d 5 0
         Fe2  CONF d5s1 DENS d5s1 SPIN d 0 5
\end{verbatim}
}
\item NORE or NONREL  Specifies that this atom is to be treated non-relativistic in
an ATOMIC ZORA calculation.
\item FORCE   Normally the changes specified in these subdirectives are only used to generate 
a start and are not included when calculating the atomic ZORA corrections. Specifying
FORCE will cause the program to use them always. So one might have ZORA corrections for
a charged atom.
\end{itemize}

{\bf Example:} The following specifies explicitly the configuration for the iron atoms
in Fe$_2$O$_3$, where the two iron d-shells are either completely alpha or beta and the 
s open shell is divided. For fun the oxygens are made slightly negative.

{
\footnotesize
\begin{verbatim}
          VECTORS ATOMS CONF UHF
          Fe1  CONF d5s1 DENS d5s1 SPIN d 5 0
          Fe2  CONF d5s1 DENS d5s1 SPIN d 0 5
          O      CHARGE -0.6
          END
\end{verbatim}
}
\end{itemize}

\item {\bf VECTORS ATORBS} : Construct starting orbitals as a
concatenation of the atomic orbitals of the atoms (not generally
recommended). This is useful however for subsequent VB calculations and
for calculations on atoms. The SECTION keyword is recognised as input
on the same line followed by the section number.  If given, the atomic
orbitals are written to the section specified and the density matrices
are used as in the ATOMS option.

\item  {\bf VECTORS HCORE} : Diagonalise the 1-electron (core) 
Hamiltonian. This is the most
general mechanism available, but is also the least reliable
in that the resulting MOs may often not exhibit the required
ordering. We return to this point below.

\item  {\bf VECTORS MINGUESS} : Construct and diagonalise a 
Huckel type matrix. This option is
limited to minimal basis sets (e.g., BASIS STO3G), but in such cases
often leads to a reliable set of MOs.
Note that the original
MINGUESS implementation has been extended to handle
nuclei up to and including Xenon.

\item  {\bf VECTORS EXTGUESS} : Limited to split-valence 
basis sets (e.g., 3-21G, 4-31G etc.), leading
in general to a reliable set of orbitals. Note that the original
EXTGUESS implementation has been extended to handle
polarisation basis sets such as 4-31G*, 4-31G** etc. and
can now handle nuclei up to and including Xenon.

\item  {\bf VECTORS ALPHAS} : Trial vectors are generated from a 
Fock matrix based on a
Mulliken-type approximation together with a set of input diagonal
Fock elements. Following the VECTORS line, a sequence of NBASIS
real numbers (where NBASIS is the number of basis functions) must be
input, with the I'th such number set to the negative of the expected
value of the I'th diagonal Fock matrix (in the basis representation)


{\bf Example} the following ALPHAS data 
refers to a triple-zeta (TZV) calculation on formaldehyde :

{
\footnotesize
\begin{verbatim}
          ALPHAS
          8.9 11.6 5.5 2.7 1.2
          -0.2 0.0 0.2  0.1 0.4 0.8  0.0 0.1 0.4
          15.5 15.1 6.6 3.6 1.7
          -0.3 -0.1 -0.1  0.0 0.1 0.5  0.0 0.0 0.5
          -0.5 0.5 0.5   -0.5 0.5 0.5
\end{verbatim}
}
Having generated the trial MOs (and possibly manipulated  them under
control of the SWAP directive), the resulting set of vectors
is written to the Dumpfile Section(s) nominated on the ENTER
directive. Thus a typical data sequence in a closed-shell SCF
calculation would be

{
\footnotesize
\begin{verbatim}
          VECTORS ATOMS
          ENTER 10
\end{verbatim}
}
or just

{
\footnotesize
\begin{verbatim}
          ENTER 10
\end{verbatim}
}
where the vectors generated by concatenating the atomic SCF densities
are written to Section 10 of the Dumpfile. Any subsequent
SCF undertaken during the job will refresh this set of vectors.

\item  {\bf VECTORS NOGEN section} : The NOGEN option is specific to
generating a trial set of GVB orbitals, with {\bf section}  an integer
referencing a section on the Dumpfile where a suitable starting set
of either SCF- or localised-MOs may be found.  For each GVB pair, two
trial orbitals are required, the strongly occupied MO (corresponding to
an SCF-occupied MO) and a weakly occupied orbital. The NOGEN facility
will generate such weakly occupied orbitals from the strongly occupied
counterparts. The user must ensure that, given an n-pair GVB treatment,
the top n orbitals from {\bf section}  correspond to the strongly occupied
MOs of each pair. This might typically be achieved under control of the
SWAP directive.  \end{itemize}

\subsection[Section specification from the `Parent' Dumpfile]{Section specification from the `Parent' Dumpfile}

The above discussion leads in obvious fashion to the second method of
specifying the trial orbitals, namely by nominating a Section on the
Dumpfile wherein such orbitals may be found, the orbitals having been
written to that Section by some preceding run of the program under control
of the ENTER directive.  In contrast to previous versions of GAMESS-UK,
which required explicit specification of these section numbers, the
current release provides a set of default values so that the user may
avoid the task of nominating sections. These defaults, which are a
function of SCFTYPE, are summarised in Table~\ref{table:2}.

\begin{table}
 \caption{\label{table:2}\  Default Vector Sections as a function of SCFTYPE}
 
 \begin{centering}
 \begin{tabular}{lccc}
\\ \hline\hline
  SCFTYPE               & Number of &   \multicolumn{2}{c}{Default}  \\
           \cline{3-4}
                        & Sections  &   Section & Numbers \\ \cline{1-4}
 
  Closed-shell SCF      &      1   &    1 &   \\
  UHF                   &      2   &    2 & 3 \\
  Open-shell RHF        &      2   &    4 & 5 \\
  GVB                   &      2   &    4 & 5 \\
  CASSCF                &      2   &    6 & 7 \\
  MCSCF                 &      2   &    8 & 9 \\
\hline\hline
 \end{tabular}
 
 \end{centering}
\end{table}

Clearly a discussion of the VECTORS directive, where we define Sections
containing input MOs, should now be linked to that of the ENTER directive,
where Sections for orbital output are nominated. We also consider
Section specification as a function of SCFTYPE in the notes below,
focusing attention initially on the use of orbitals resident on the
`parent' Dumpfile.

\begin{enumerate}
\item {\em Closed Shell SCF} :  Here we are only involved in nominating
a single Section on the Dumpfile to contain the closed-shell SCF
eigenvectors. Thus given the data sequence

{
\footnotesize
\begin{verbatim}
          VECTORS ATOMS
          ENTER 1
\end{verbatim}
}
or just ENTER~1 in a startup-job, the sequence

{
\footnotesize
\begin{verbatim}
          VECTORS 1
          ENTER 1
\end{verbatim}
}
would be specified in a subsequent SCF restart, in which case
the MOs in Section 1 would be updated, or perhaps the sequence

{
\footnotesize
\begin{verbatim}
          VECTORS 1
          ENTER 10
\end{verbatim}
}
if the User wished to keep copies of both initial and final MOs.

\item {\em UHF SCF} : Now two Sections are required, the first referring
to the $\alpha$-spin orbitals, the second to the $\beta$-spin
orbitals. When initiating a UHF calculation, the data sequence

{
\footnotesize
\begin{verbatim}
          VECTORS 1
          ENTER 2 3
\end{verbatim}
}
is permitted, indicating that both $\alpha$- and $\beta$-spin trial MOs are to
be taken from the same Section. Restarting the above computation
would typically involve the sequence

{
\footnotesize
\begin{verbatim}
          VECTORS 2 3
          ENTER 2 3
\end{verbatim}
}

\item {\em Open-shell RHF and GVB Calculations}  : Again two Sections are
involved. The first is used to hold the `internal' non-canonicalised
MOs, the orbital set used during the RHF or GVB iterations. The second
Section is used for output of the `external' canonicalised orbitals,
with energy weighting in the virtual manifold.
Again, given a set of trial MOs in Section 1, we typically instigate
the RHF/GVB calculation with the data sequence

{
\footnotesize
\begin{verbatim}
          VECTORS 1
          ENTER 4 5
\end{verbatim}
}
and continue the processing with the sequence
{
\footnotesize
\begin{verbatim}
          VECTORS 4 5
          ENTER 4 5
\end{verbatim}
}

\item {\em CASSCF Calculations} : Again two Sections are required, with
both Sections containing information crucial to the internal running
of the CASSCF module. The first Section is assumed to contain just the
orbital set, while the second contains the canonicalised MOs ( relevant to
subsequent CI studies), {\em plus}  a set of CI coefficients which may be
used in assisting CASSCF restarts.  Given a trial set of MOs in Section 1,
we would typically initiate the CASSCF calculation with the sequence

{
\footnotesize
\begin{verbatim}
          VECTORS 1
          ENTER 6 7
\end{verbatim}
}

On completion of the job, Section 6 will contain the CASSCF MOs, while
Section 7 will hold canonicalised orbitals {\em plus}  the current set of
CI coefficients. In a subsequent restart we would specify

{
\footnotesize
\begin{verbatim}
          VECTORS 6 7
          ENTER 6 7
\end{verbatim}
}
when the orbitals will be read from Section 6, and the CI coefficients
from Section 7. Note that the program assumes, given two Sections
on the VECTORS line, that the second may be used as a source
of CI coefficients - if that Sections contains no such data, or
coefficients from some different CASSCF calculation with a
different CI space), an error condition will result.
\end{enumerate}

There are times when it is useful to see the starting vectors, in which
case the character string PRINT may be specified on the VECTORS
directive line. eg. VECTORS ATOMS PRINT
If the vectors are read from a section, one may forego the orthogonalisation
of these vectors by specifying the keyword NOORTH e.g., VECTORS 19 NOORTH.
This may be used for example in calculating coulomb energies, where the
(non)-orthogonal vectors are combined by the SERVEC subprogram.

\subsection{Using Default Sections under VECTORS and ENTER}

In all the examples above, we have assumed that the User is explicitly
defining the sections of the Dumpfile to be used for vector retrieval
(under control of the VECTORS directive) and vector storage (under
control of the ENTER directive). While this mechanism provides an
additional degree of user control, it is now possible to use a set
of default sections that avoids the need for explicit specification
on both VECTORS and ENTER directives, and in most cases removes
the need for presenting the VECTORS directive altogether. The
following points should be noted regarding use of these defaults;
\begin{itemize}
\item This default usage is not designed to completely remove the need
for section specification, and is intended primarily to cover simple
operations e.g. a simple SCF or geometry optimisation.
\item While an expanded summary of section usage is now routinely
printed on job termination, the user should be aware of the attributes
of the various vector sections before mixing default and input-driven
section specification.
\item the contents of each of the sections specified in
Table~\ref{table:2} have been described in some detail in Part 2 of
the manual. With the exception of "RUNTYPE ANALYSE", the choice of
default section(s) is determined solely by the SCFTYPE that has been
requested. Given this, the decision on the choice of input eigenvectors is
made in three stages.  Thus when deciding on the section(s) to be used
for these orbitals (i.e the VECTORS section(s)), the default sections
appropriate to the nominated SCFTYPE will be examined first. If found
to exist, and in the absence of explicit section specification on the
VECTORS directive, the input eigenvectors will be taken from these default
section(s). If these default sections have not been written to in some
previous job or job step, then the default closed-shell eigenvector
section (section 1) will be examined, and the input eigenvectors taken
from this section assuming again that this section has been written
to previously. If the closed-shell vectors section does not exist,
then the eigenvectors will be generated from an atomic-guess. The
choice of section number(s) for the output of eigenvectors (i.e. those
sections typically nominated on the ENTER directive) is taken directly
from Table~\ref{table:2}; clearly this can lead to the final vectors
over-writing the input eigenvectors.  We illustrate these choices by
considering a number of cases below as a function of SCFTYPE.

\begin{enumerate}
\item {\em Closed Shell SCF} :  Here we are only involved in considering
a single Section on the Dumpfile that contains the closed-shell SCF
eigenvectors. Thus presenting the single data line

{
\footnotesize
\begin{verbatim}
          ENTER
\end{verbatim}
}
in a startup-job would act to request an atomic GUESS for generating
the initial SCF eigen vectors, that would be stored, and subsequently
updated and written to section 1 of the Dumpfile (see Table~\ref{table:2})
during the SCF process. Presenting the same data line in a RESTART job
would result in the eigenvectors of section 1 being used as the trial
vector set, and subsequently updated and overwritten during the SCF
process. In practice this default section would be examined for content
at job outset, and the contents used for the SCF process assuming the
section had been written to by a previous job. If the section is not
present on the Dumpfile, then a trial set will be generated using the
"VECTORS ATOMS" mechanism.

If the User wished to keep copies of both initial and final MOs, then
the single data line of the form

{
\footnotesize
\begin{verbatim}
          ENTER 10
\end{verbatim}
}
would be required, with the final set of MOS being written to section 10.
Note that usage of this set in some subsequent job would require explicit
introduction of the data line "VECTORS 10" to avoid use of the default
section.

\item {\em UHF SCF} : Two sets of eigenvectors are generated in an open-shell
unrestricted Hartree Fock (UHF) calculation, the $\alpha$-spin SCF MOs
and  $\beta$-spin orbitals.  In default the $\alpha$-spin MOs will be
written to section 2 of the Dumpfile, and the $\beta$-spin MOs to
section 3.  (see Table~\ref{table:2}).  Thus presenting the single data line

{
\footnotesize
\begin{verbatim}
          ENTER
\end{verbatim}
}
in a startup-job would act to request an atomic GUESS for generating
a set of SCF eigen vectors, that would be be used to initiate the UHF
process, with the $\alpha$- and $\beta$-spin MOs subsequently updated
and written to sections 2 and 3 of the Dumpfile (see Table~\ref{table:2})
during the SCF process.

Presenting the same data line in a subsequent RESTART job would result
in the eigenvectors of section 2 and 3  being used as the trial vector
set, and subsequently updated during the UHF process. In
practice these default sections would be examined for content at job
outset, and the contents used for the SCF process assuming the sections
had been written to by a previous job. If the sections are not present
on the Dumpfile, then either (i) the closed shell MOS (if present)
in section 1 will be used as the trial orbitals for both $\alpha$- and
$\beta$-spin, or (ii) with no closed shell section, a trial set will be
generated using the "VECTORS ATOMS" mechanism.

If the User wished to keep copies of both initial and final UHF MOs, then
the single data line of the form

{
\footnotesize
\begin{verbatim}
          ENTER 10 11
\end{verbatim}
}
would be required, with the final set of $\alpha$-spin MOs MOS being
written to section 10 and $\beta$-spin orbitals to section 11.
Note that usage of this set in some subsequent job would require
explicit introduction of the data line "VECTORS 10 11" to avoid use of
the default section.

\item {\em Open-shell RHF and GVB Calculations}  : Again two Sections are
involved, the first holding the `internal' non-canonicalised MOs, the
orbital set used during the RHF or GVB iterations, while the second is
used for output of the `external' canonicalised orbitals on termination
of the SCF process. In default the `internal' MO set will be written to
section 4 and the canonicalised orbital set to section 5 of the Dumpfile.
(see Table~\ref{table:2}). Thus presenting the single data line,

{
\footnotesize
\begin{verbatim}
          ENTER
\end{verbatim}
}
in a startup-job would act to request an atomic GUESS for generating
a set of SCF eigen vectors, that would be be used to initiate the open-shell
SCF, with the `internal' non-canonicalised MOs subsequently updated and
written to section 4 of the Dumpfile and the canonicalised orbital set
written to section 5.

Presenting the same data line in a subsequent RESTART job would result in
the eigenvectors of section 4 and 5  being used as the trial vector set,
and subsequently updated and overwritten during the open-shell SCF. In
practice these default sections would be examined for content at job
outset, and the contents used for the SCF process assuming the sections
had been written to by a previous job. If the sections are not present
on the Dumpfile, then either (i) the closed shell MOS (if present)
in section 1 will be used as the trial orbitals, (ii) with no closed
shell section, a trial set will be generated using the "VECTORS ATOMS"
mechanism.  If the User wished to keep copies of both sets of initial
and final open-shell MOS, then a single data line of the form

{
\footnotesize
\begin{verbatim}
          ENTER 10 11
\end{verbatim}
}
would be required, with the final set of `internal' MOS being written to
section 10 and the canonicalised orbitals to section 11.  Note that usage
of this set in some subsequent job would require explicit introduction
of the data line "VECTORS 10 11" to avoid use of the default section.

\item {\em CASSCF Calculations} : Again, two Sections are required;
the first housing the non-canonicalised CASSCF MOs that are used during
the CASSCF process, and the second set, the canonicalised vectors that
are generated on termination of the CASSCF process.  In default the
non-canonicalised vectors will be written to section 6 of the Dumpfile,
while the canonicalised vectors will be written to section 7 (see
Table~\ref{table:2}). Note that the latter section also contains the
current CI coefficients. Presenting the single data line,

{
\footnotesize
\begin{verbatim}
          ENTER
\end{verbatim}
}
in a startup-job would act to request an atomic GUESS for generating a set
of eigen vectors, that would be be used to initiate the CASSCF process,
with the non-canonicalised CASSCF MOs subsequently written to section 6 of
the Dumpfile and the final canonicalised orbital set written to section 7.

Presenting the same data line in a subsequent RESTART job would result
in the eigenvectors of section 6 and CI coefficients of section 7 being
used to initiate the CASSCF process, each being subsequently updated
and overwritten. In practice these default sections would be examined
for content at job outset, and the contents used for the CASSCF process
assuming the sections had been written to by a previous job. If the
sections are not present on the Dumpfile, then either (i) the closed
shell MOS (if present) in section 1 will be used as the trial orbitals,
(ii) with no closed shell section, a trial set will be generated using
the "VECTORS ATOMS" mechanism.

\item {\em Wavefunction Analysis} : The variety of wavefunction
analysis and property modules of GAMESS--UK (see Part9) also rely on
the VECTORS directive to define the section number of the eigenvectors
to be analysed. Note that in contrast to SCF processing, {\em no
default specification} is available under "RUNTYPE ANALYSE". Explicit
specification is required even if this section had been written to using
one of the default options during, for example, previous SCF processing.

\end{enumerate}

\end{itemize}

\subsection{Specification from a `Foreign' Dumpfile : GETQ}

The GETQ option causes restoration of eigenvectors from a `foreign'
Dumpfile. Characterisation of such a Dumpfile is achieved by
specification, through data input, of

\begin{itemize}
\item  the LFN of the data set on which this Dumpfile resides. The
foreign Dumpfile may reside on a direct data set assigned
using a DDNAME in the range ED0 to ED19, or on a sequential data set
assigned using a DDNAME in the range MT0 to MT19;
\item  the starting block of the Dumpfile;
\item  the Section number wherein the required vectors are to be found,
this integer having been that associated with the ENTER directive
(whether by default or by explicit specification) of the run which
created the trial vectors.
\end{itemize}

A given set of eigenvectors is deemed to represent valid input to GETQ
if it has the following attributes;
\begin{enumerate}
\item  It is derived from a previous calculation performed
at either the same or at a different geometry.
\item  It is derived from a previous calculation where the symmetry adapted
option is different
\item  It is derived from a previous calculation conducted in a {\em smaller}
basis set. The program maps the `old' basis onto the `new' using
standard projection techniques, and it is now routine practice to use,
for example, the vectors generated in a SV 3-21G basis to initiate
an SCF calculation in a triple zeta + polarisation basis set.
Again the symmetry adapted option may be different in the `old' and
`new' calculation.
\item  The role of the EXTRA option (see ATMOL3 9) is largely
redundant given the availability of the revised GETQ, and it is
suggested that GETQ be used in such cases.
\item  The only major restriction in the use of GETQ is that the
ordering of the nuclei presented in the z-matrix definition lines
be the same in both `old' and `new' calculation.
\end{enumerate}

The syntax of the GETQ option is again dependent on the number of
sets of vectors to be retrieved from `foreign' Dumpfiles.
\begin{enumerate}
 \item When restoring a single set of vectors, as in closed and open-shell
restricted Hartree Fock calculations, the VECTORS directive
consists of a single data line, as follows\\

 VECTORS GETQ {\em lfnvec}  {\em iblkv} {\em isectv}

where,

\begin{itemize}
\item {\em lfnvec} : should be set to a valid character string (ED0-ED19,
MT0-MT19) specifying the LFN of the data set on which the `foreign'
Dumpfile resides;
\item {\em iblkv} : should be set to the starting block of the `foreign'
Dumpfile;
\item {\em isectv} is an integer used to specify the section where the
required vectors are to be found on the `foreign' Dumpfile. Note that
{\em isectv} must be specified even if referring to the appropriate
default section.
\end{itemize}


\item When restoring two sets of eigenvectors from a `foreign' Dumpfile,
for example the $\alpha$-spin and $\beta$-spin vectors in a UHF
calculation, then the VECTORS directive is a straightforward extension
of the single-vector case, with the location of each set specified. Thus
the data line is as follows\\

VECTORS GETQ {\em lfnveca}  {\em iblka} {\em isecta} {\em lfnvecb} {\em iblkb} {\em isectb}

where,\\
\begin{itemize}
\item {\em lfnveca}, {\em iblka} and {\em isecta} : The trial
$\alpha$-spin vectors are to be restored
from the Section specified by {\em isecta}  on a `foreign' Dumpfile commencing
at the block specified by {\em iblka}, with the Dumpfile residing on the
data set assigned using the LFN {\em lfnveca} .

\item {\em lfnvecb}, {\em iblkb} and {\em isectb} : The trial
$\beta$-spin vectors are to be restored
from {\em isectb}  on a foreign Dumpfile residing on a data set
assigned using the LFN {\em lfnvecb}, commencing at block {\em iblkb}.
\end{itemize}

 Note that
\begin{itemize}
\item  the trial $\beta$-spin orbitals will be set equal to the
trial $\alpha$-spin orbitals if {\em lfnvecb}, {\em iblkb} and {\em isectb} are
omitted from the data line.
\item  there is currently no provision for restoring the CI Vector
Section in a CASSCF calculation under control of the GETQ option.
In such cases it is only possible to restore a single set of vectors,
the CASSCF orbitals.
\end{itemize}
\end{enumerate}
{\bf Example 1}\\

A geometry optimisation has been performed with a
closed shell SCF calculation in a 3-21G basis set. A single point
calculation in a TZVP basis set is to be performed at the
optimised geometry. Assuming that the Dumpfile from the optimisation
run is located on a data set assigned to the present calculation as
ED4, starting at block 1, and that the converged 3-21G vectors
are stored in Section 2 of this Dumpfile, then these vectors
will provide a satisfactory starting point for the TZVP
study, and may be restored using a directive of the form

{
\footnotesize
\begin{verbatim}
          VECTORS GETQ ED4 1 2
\end{verbatim}
}
where the data set used to hold the `foreign' Dumpfile is
assigned to the program using the LFN ED4.\\

{\bf Example 2}\\

The Dumpfile for the TZVP calculation above
is to be located on the {\em same}  data set used for performing the
the 3-21G calculation. Assume that on completion of the original
optimisation the total length of the Dumpfile was found to be
smaller than 250 blocks. Siting the TZVP calculation at block 250
on this data set, by means of the directive

{
\footnotesize
\begin{verbatim}
          DUMPFILE ED3 250
\end{verbatim}
}
permits use of the following GETQ directive
{
\footnotesize
\begin{verbatim}
          VECTORS GETQ ED3 1 2
\end{verbatim}
}
in making the 3-21G vectors available to the present calculation.\\

{\bf Example 3}\\

A UHF calculation has been performed in a 3-21G
basis, with the converged $\alpha$- and $\beta$-spin orbitals stored
in Sections 8 and 9 of the Dumpfile which had started
at block 50. We now wish to perform the same
calculation using a TZV basis set. Assigning the data set
containing this Dumpfile from the previous calculation
as ED4, we may commence the TZV UHF calculation by means of a
directive of the form

{
\footnotesize
\begin{verbatim}
          VECTORS GETQ ED4 50 8 ED4 50 9
\end{verbatim}
}

\subsection[LOCK]{LOCK}

This directive consists of one data line, with the character string
LOCK in the first data field. In the presence of the LOCK directive,
the program will seek a stationary value for the functional, and
proceed in the direction of `minimum change'. The use of the LOCK
directive is automatic in open shell, GVB and CASSCF/MCSCF
calculations.

\subsection[SWAP]{SWAP}

The first data line should contain the character string SWAP in the
first data field. Subsequent data lines are read to variables I,J using
format (2I).
The effect is that the I'th and J'th molecular orbitals, as generated
by the VECTORS directive, are interchanged. When all the interchanging
lines have been presented, the directive should be terminated by a data
line containing the character string END in the first data field.
The following notes may prove helpful:

\begin{itemize}
\item The SWAP directive is normally used to switch from a configuration
known not to be the ground state into the ground state.
\item upon completion of the SWAP directive, revised molecular orbital
lists are held in memory, but not in the Dumpfile.
\end{itemize}

{\bf Example}
{
\footnotesize
\begin{verbatim}
          SWAP
          10 12
          11 13
          END
\end{verbatim}
}
Molecular orbitals 10 and 11 are interchanged with molecular orbitals
12 and 13 respectively.

In the case of UHF calculations,  the syntax of this directive is 
modified to reflect the presence of both $\alpha$ and $\beta$ orbitals.
Now the directive initiator is read to variables TEXT,SPIN
using format (2A).
\begin{itemize}
\item  TEXT should be set to character string SWAP.
\item  SPIN should be set to one of the character strings ALPHA or
BETA, and will cause swapping of either $\alpha$ or $\beta$ spin
vectors respectively. The SPIN parameter may be omitted
(thus the directive is the same as that above), and
will cause $\alpha$-spin vectors to be interchanged. 
\end{itemize}


\section[Controlling Geometry and Transition State Optimization]{Controlling Geometry and Transition \protect \\State Optimization}

There are a variety of methods available for controlling the
search for a stationary point on a potential energy surface. Each may
be requested through appropriate keyword specification
on the RUNTYPE directive, and in the following section we detail
subsequent data requirements of each method. 
In most cases  adequate control of the optimisation pathways 
is provided through a set of `built-in' parameters, and the
user need only consider overriding these defaults in
troublesome cases, through use of the  directives described below.
An introduction to the various
methods and their usage has already been presented in Part 2, and
should be used in conjunction with the notes below.

\subsection[Geometry Optimisation and RUNTYPE Specification]{Geometry Optimisation and RUNTYPE Specification}

Three methods are available to search for a minimum on
a potential Surface,
\begin{enumerate}
\item the recommended method, a quasi-Newton rank--2 update procedure,
is driven through the specification

{
\footnotesize
\begin{verbatim}
           RUNTYPE OPTIMIZE
\end{verbatim}
}
This method performs optimisation in internal coordinates, 
and thus requires initial ZMATRIX and VARIABLES specification of the 
molecular geometry, or ZMATRIX construction from a  set
of cartesian coordinates supplied under control of the
GEOMETRY directive.
\item the second internal coordinate--driven method is that 
based on the hill-walking algorithm due to 
Simons and Jorgensen  \cite{simons}. 
While primarily intended for transition state usage, it may also
be employed in geometry optimisation. A more detailed account of
the method and associated data is given below in section 10. We note
here that the procedure is driven through additional keyword
specification on the RUNTYPE directive, thus;

{
\footnotesize
\begin{verbatim}
           RUNTYPE OPTIMIZE JORGENSEN
\end{verbatim}
}
\item the third method, perhaps less robust and flexible than the
others, is a cartesian-driven update method. This is requested
through the following RUNTYPE specification,

{
\footnotesize
\begin{verbatim}
           RUNTYPE OPTXYZ
\end{verbatim}
}
\end{enumerate}
We shall use the RUNTYPE keywords -- OPTIMIZE, JORGENSEN and 
OPTXYZ -- to subsequently refer to the
the three methods. 

\subsection{Stopping an optimisation when molecule dissociates - CHECK, DISS or DIST}
Appending the keywords ''CHECK'', ''DISS'' or ''DIST'' in A format to
any of the following runtypes:

\begin{itemize}
\item OPTIMIZE
\item SADDLE
\item OPTX
\end{itemize}

Will cause a check to be carried out as to whether the number of bonds
within the molecule changes (defined as the distance between two
bonded atoms changing by more than 15\% over the course of the
optimisation). If the number of bonds changes, then the optimisation
will abort. This can often be useful to halt a geometry optimisation
when a molecule appears to be dissociating.

\subsection[Keeping copies of orbitals at intermediate points]{Keeping copies of orbitals at intermediate point}
During an optimisation or saddle search the orbitals may be dumped at regular
intervals (COPying the Q vectors COPQ ) Thus a calculation may
more easily be restarted at one of the intermediate geometries. This directive consists of a single 
data line read to the variables TEXT, INT, MIN-SECTON, MAX-SECTION, using format (A,3I).
\begin{itemize}
\item TEXT is set to the character string COPQ.
\item INT is an integer specifying  the number op steps taken between orbitals are dumped again
(default 5)
\item MIN-SECTION specifies the first section (default 100)j.
\item MAX-SECTION specifies the last section permitted (default 200).
\end{itemize}
The default, just specifying COPQ, thus corresponds to presenting the data line

{
\footnotesize
\begin{verbatim}
           COPQ  5 100 200
\end{verbatim}
}


\subsection[OPTIMIZE Data; MINMAX, XTOL, STEPMAX and VALUE]{OPTIMIZE Data Specification}

Four directives are provided to control the OPTIMIZE search procedure, 
MINMAX, XTOL, STEPMAX and VALUE. The user should note that the
present implementation is based on maintaining a history of the
optimisation pathway, that will be worked through on each restart of
the optimisation. This appears on the output as a sequence of both
`old' and `new' calculations, with the history printed on each
restart. This printing may be suppressed through use of the
NOPRINT directive, with specification of the HISTORY keyword.

\subsubsection[OPTIMIZE Data - MINMAX]{OPTIMIZE Data - MINMAX}

This directive may be used to
control the number of energy evaluations and line-searches permitted
in optimising a given structure, and consists of a single 
data line read to the variables TEXT, IVAL, LINE using format (A,2I).
\begin{itemize}
\item TEXT is set to the character string MINMAX.
\item IVAL is an integer specifying  the maximum number of 
energy evaluations allowed in the optimisation.
\item LINE is an integer defining the maximum number of line searches
permitted in the optimisation.
\end{itemize}
The MINMAX directive may be omitted, when both IVAL and LINE will
be set to the maximum allowed value of 60. 
The following specification is thus 
equivalent to the default;

{
\footnotesize
\begin{verbatim}
          MINMAX 60 60
\end{verbatim}
}
{\bf Example}\\

In some cases the user may wish to perform just the
initial point on the optimisation pathway to gauge the quality
of the starting geometry though the magnitude of the gradient at
that point. This may be achieved though use of MINMAX, as shown
below:\\

{\bf Start-up Job}
{
\footnotesize
\begin{verbatim}
          TITLE\H2O DZ OPTIMIZATION STARTUP
          ZMAT ANGSTROM\O\H 1 ROH\H 1 ROH 2 THETA
          VARIABLES\ROH 0.956 HESSIAN 0.7\THETA 104.5 HESSIAN 0.2 \END
          BASIS DZ
          RUNTYPE OPTIMIZE\MINMAX 1 1
          ENTER
\end{verbatim}
}
Here we are using the MINMAX directive to terminate the
optimisation after the first point. This may then be restarted
as shown below, where the default MINMAX settings will apply.\\

{\bf Restart Job}
{
\footnotesize
\begin{verbatim}
     RESTART OPTIMIZE
     TITLE\H2O DZ OPTIMIZE
     ZMAT ANGSTROM\O\H 1 ROH\H 1 ROH 2 THETA
     VARIABLES\ROH 0.956 HESSIAN 0.7\THETA 104.5 HESSIAN 0.2 \END
     BASIS DZ
     RUNTYPE OPTIMIZ\ENTER
\end{verbatim}
}

\subsubsection[OPTIMIZE Data - XTOL]{OPTIMIZE Data - XTOL}

This directive may be used to define the convergence thresholds
for the optimisation, and consists of a single data line
read to the variables TEXT, TOL using format (A,F):
\begin{itemize}
\item TEXT should be set to the character string XTOL;
\item TOL should be set to the value to be used in
defining the four acceptance criteria for the convergence of the
optimisation algorithm. These criteria are:

{
\footnotesize
\begin{verbatim}
          maximum change in variables  <  TOL
          average change in variables  <  TOL * 2/3
          maximum gradient             <  TOL * 1/4
          average gradient             <  TOL * 1/6
\end{verbatim}
}
\end{itemize}
The XTOL directive may be omitted, when TOL will be set to 0.003.
The default thus corresponds to presenting the data line

{
\footnotesize
\begin{verbatim}
           XTOL 0.003
\end{verbatim}
}

\subsubsection[OPTIMIZE Data - STEPMAX]{OPTIMIZE Data - STEPMAX}

This directive may be used to define the 
the maximum allowed movement in any of the variables
in a single step of the geometry optimisation. Note that 
the internal units of the variables are
bohr for bond lengths and radians for angles. 
The directive consists of a single data line
read to the variables TEXT, STEP using format (A,F):
\begin{itemize}
\item TEXT should be set to the character string STEPMAX;
\item STEP should be set to the maximum permitted movement
in any variable.
\end{itemize}
The STEPMAX directive may be omitted, when STEP will be set to 0.2.
The default thus corresponds to presenting the data line

{
\footnotesize
\begin{verbatim}
           STEPMAX 0.2
\end{verbatim}
}
There is certainly at least one circumstance where changes to
the default setting will prove crucial in achieving controlled
convergence. If the starting geometry is known to be poor, or
if ZMATRIX specification is such that a specific bond is not
explicitly defined (as can happen for example with aromatic
compounds), then the first step taken on the optimisation
can cause the energy to go {\em up} and, at best,  several extra points
will be required to recover from this effect. This effect
is fairly common if in addition the starting hessian is also poorly
defined. When this happens, the user should consider starting
the optimisation again, presenting a STEPMAX directive of the 
form:

{
\footnotesize
\begin{verbatim}
           STEPMAX 0.1
\end{verbatim}
}
An additional side effect of excessive steps in the optimisation
is a possible change of state in the SCF calculation, particularly
in closed--shell wavefunctions. If this happens, the subsequent
optimisation will almost certainly prove meaningless. 
Presenting the LOCK directive in the closed--shell case may
act to minimise this occurrence.

\subsubsection[OPTIMIZE Data - VALUE]{OPTIMIZE Data - VALUE}

This directive may be used to 
control the accuracy of the search for a turning point during
a line search, and consists of a single data line read to
the variables TEXT, TURN using format (A,F);
\begin{itemize}
\item TEXT should be set to the character string VALUE;
\item TURN should be set to a value between 0.0 and 1.0 
that will control the accuracy of the line search procedure. Note
that the smaller TURN, the more accurate the line search.
\end{itemize}
The VALUE directive may be omitted, when TURN will be set to 0.6.
The default thus corresponds to presenting the data line

{
\footnotesize
\begin{verbatim}
            VALUE 0.6
\end{verbatim}
}

\subsection[Modifying the Optimisation Pathway]{Modifying the Optimisation Pathway}

In some cases the user may wish to modify the parameters
controlling geometry optimisation, though XTOL, VALUE and STEPMAX
specification, during the
course of the optimisation. Note that these parameters may only be
modified between line searches, and not between individual energy or
gradient evaluations. The most straightforward example would be
initialising the optimisation with stringent controls, then 
relaxing these controls as the optimisation proceeds in some
subsequent restart job. The startup job  may either be interrupted
under control of the MINMAX directive, or through time specification
on the TIME pre-directive (see Parts 12-16 of the manual).

Consider the example of Part~3 section 8.4 on the optimisation of HCN. The
following data file requests termination after two line searches
through the MINMAX specification, during which conservative settings
of the optimisation parameters will apply:

{
\footnotesize
\begin{verbatim}
          TITLE
          HCN  DUNNING DZ + BOND(S,P) 
          ZMAT ANGSTROM
          C
          BQ 1 RCN2
          X 2 1.0 1 90.0
          N 2 RCN2 3 90.0 1 180.0
          X 1 1.0 2 90.0 3 0.0
          H 1 RCH 5 90.0 4 180.0
          VARIABLES
          RCN2 0.580 
          RCH 1.056 
          END
          BASIS
          DZ H
          S BQ
          1.0 1.0
          P BQ
          1.0 0.7
          DZ C
          DZ N
          END
          RUNTYPE OPTIMIZE
          MINMAX 60 2
          XTOL 0.005
          STEPMAX 0.1   
          VALUE 0.3
          ENTER
\end{verbatim}
}
In the restart job shown below, the modified parameter settings will
apply from the third line search onwards.

{
\footnotesize
\begin{verbatim}
          RESTART OPTIMIZE
          TITLE
          HCN  DUNNING DZ + BOND(S,P) 
          ZMAT ANGSTROM
          C
          BQ 1 RCN2
          X 2 1.0 1 90.0
          N 2 RCN2 3 90.0 1 180.0
          X 1 1.0 2 90.0 3 0.0
          H 1 RCH 5 90.0 4 180.0
          VARIABLES
          RCN2 0.580 
          RCH 1.056 
          END
          BASIS
          DZ H
          S BQ
          1.0 1.0
          P BQ
          1.0 0.7
          DZ C
          DZ N
          END
          RUNTYPE OPTIMIZE
          XTOL 0.0005
          STEPMAX 0.2   
          VALUE 0.6
          ENTER
\end{verbatim}
}

A somewhat more complex situation may arise when the user wishes
to modify optimisation processing already performed in the
startup job, either because
\begin{itemize}
\item problems have been encountered that can only be remedied by
modifying the optimisation control parameters, or
\item the user wishes to terminate the optimisation in a controlled
fashion, since the optimisation is deemed to be complete.
\end{itemize}
The first case would not be cured merely by presenting revised
parameters in a restart job, since the program will initially
run through the pathway prior to applying the new parameters, by
which time the position may not be recoverable. In such cases the
user must identify from the output of the startup job 
a line search in the optimisation pathway prior to the problem, and
present this information to the program via a revised form of the
MINMAX directive in a restart job. The revised format 
consists of a single 
data line read to the variables TEXT, TEXT1, LINES using format (2A,I):
\begin{itemize}
\item TEXT is set to the character string MINMAX;
\item TEXT1 is set to the character string REVISE;
\item LINES is an integer defining the line search after which
any optimisation parameters presented in the input stream
are to apply.
\end{itemize}
The following specification  would cause the original optimisation
pathway to be followed for the first three line searches only;  beyond
that point the optimisation would be restarted based on any revised
control parameters.

{
\footnotesize
\begin{verbatim}
          MINMAX REVISE 4
\end{verbatim}
}
In the example below we consider the SCF geometry optimisation of
C$_{4}$F$_{4}$. In fact this optimisation proceeds smoothly,
converging to the default accuracy (XTOL = 0.003) on the
seventh line search using the data shown below.

{
\footnotesize
\begin{verbatim}
          TITLE  
          **** C4F4  3/21G ****
          ZMAT ANGS                   
          X
          C 1  R1
          C 1  R2  2  90.
          C 1  R1  3  90.   2 180.
          C 1  R2  4  90.   3 180.
          X 2 1.   1 90.    3 0.
          F 2 R3   6 90.    3 180.
          X 4 1.   1 90.    3 0.
          F 4 R3   8 90.    3 180.
          X 3 1.   1 90.    4 0.
          F 3 R3   10 90.   4 180.
          X 5 1.   1 90.    4 0.
          F 5 R3   12 90.   4 180.
          VARIABLES
          R1 1.2 
          R2  1.3 
          R3 1.313
          END
          RUNTYPE OPTIMIZE
          LEVEL 2.0 40 1.0 
          ENTER 
\end{verbatim}
}
Now let us assume we wish to tighten the convergence threshold,
using an XTOL setting of 0.001. Merely presenting a revised XTOL
specification in a restart job will not have the desired effect, for
the job will merely work through the optimisation pathway, reaching
convergence before the revised XTOL setting will come into effect.
The user must inform the optimisation that the previous actions
on the seventh line search are to be ignored, and repeated with
the revised XTOL setting, by presenting the following data file:

{
\footnotesize
\begin{verbatim}
          RESTART OPTIMIZE
          TITLE  
          **** C4F4  3/21G ****
          ZMAT ANGS                   
          X
          C 1  R1
          C 1  R2  2  90.
          C 1  R1  3  90.   2 180.
          C 1  R2  4  90.   3 180.
          X 2 1.   1 90.    3 0.
          F 2 R3   6 90.    3 180.
          X 4 1.   1 90.    3 0.
          F 4 R3   8 90.    3 180.
          X 3 1.   1 90.    4 0.
          F 3 R3   10 90.   4 180.
          X 5 1.   1 90.    4 0.
          F 5 R3   12 90.   4 180.
          VARIABLES
          R1 1.2 
          R2  1.3 
          R3 1.313
          END
          RUNTYPE OPTIMIZE
          LEVEL 1.0
          MINMAX REVISE 7
          XTOL 0.001
          ENTER  
\end{verbatim}
}

\subsection[OPTXYZ Data; MINMAX, XTOL and STEPMAX ]{OPTXYZ Data; MINMAX, XTOL and STEPMAX }

Three directives are provided to control the OPTXYZ search procedure, 
MINMAX, XTOL and  STEPMAX. Note that directive specifications are
similar, but not identical to, the descriptions above. Note also that
algorithm employed {\em only} guarantees convergence to a 
{\em stationary point}, not necessarily to a minimum.


\subsubsection[OPTXYZ Data - MINMAX]{OPTXYZ Data - MINMAX}

This directive may be used to
control the number of energy evaluations and searches permitted
in optimising a given structure, and consists of a single 
data line read to the variables TEXT, NPTS, NSERCH using format (A,2I):
\begin{itemize}
\item TEXT is set to the character string MINMAX;
\item NPTS is an integer specifying  the maximum number of 
energy evaluations allowed in the optimisation;
\item NSERCH is an integer defining the maximum number of searches
permitted in the BFGS update procedure.
\end{itemize}
The MINMAX directive may be omitted, when both NPTS and NSERCH will
be set to the maximum allowed value of 60. 
The following specification is thus 
equivalent to the default;

{
\footnotesize
\begin{verbatim}
          MINMAX 60 60
\end{verbatim}
}
{\bf Example}\\

In some cases the user may wish to perform just the
initial point on the optimisation pathway to gauge the quality
of the starting geometry though the magnitude of the gradient at
that point. This may be achieved though use of MINMAX, as shown
below:\\

{\bf Start-up Job}
{
\footnotesize
\begin{verbatim}
          TITLE\H2O DZ OPTIMIZATION STARTUP
          ZMAT ANGSTROM\O\H 1 ROH\H 1 ROH 2 THETA
          VARIABLES\ROH 0.956 HESSIAN 0.7\THETA 104.5 HESSIAN 0.2 \END
          BASIS DZ
          RUNTYPE OPTXYZ\MINMAX 1 1
          ENTER
\end{verbatim}
}
Here we are using the MINMAX directive to terminate the
optimisation after the first point. This may then be restarted
as shown below, where the default MINMAX settings will apply.\\

{\bf Restart Job}
{
\footnotesize
\begin{verbatim}
          RESTART OPTXYZ
          TITLE\H2O DZ OPTIMIZE
          ZMAT ANGSTROM\O\H 1 ROH\H 1 ROH 2 THETA 
          VARIABLES\ROH 0.956 HESSIAN 0.7\THETA 104.5 HESSIAN 0.2 \END
          BASIS DZ
          RUNTYPE OPTXYZ
          ENTER
\end{verbatim}
}

\subsubsection[OPTXYZ Data - XTOL]{OPTXYZ Data - XTOL}

This directive may be used to define the convergence threshold
for the optimisation, and consists of a single data line
read to the variables TEXT, TOL using format (A,F):
\begin{itemize}
\item TEXT should be set to the character string XTOL;
\item TOL should be set to the value to be used in
defining the  criteria for convergence of the
optimisation algorithm, the maximum component of the gradient.
\end{itemize}
The XTOL directive may be omitted, when TOL will be set to 0.001.
The default thus corresponds to presenting the data line

{
\footnotesize
\begin{verbatim}
           XTOL 0.001
\end{verbatim}
}

\subsubsection[OPTXYZ Data - STEPMAX]{OPTXYZ Data - STEPMAX}

This directive may be used to define the 
the maximum allowed movement in any of the cartesian coordinates
in a single step of the geometry optimisation (in units of bohr).
The directive consists of a single data line
read to the variables TEXT, STEP using format (A,F):
\begin{itemize}
\item TEXT should be set to the character string STEPMAX;
\item STEP should be set to the maximum permitted movement
in any coordinate.
\end{itemize}
The STEPMAX directive may be omitted, when STEP will be set to 0.2.
The default thus corresponds to presenting the data line

{
\footnotesize
\begin{verbatim}
           STEPMAX 0.2
\end{verbatim}
}


\subsection[JORGENSEN Data Specification]{JORGENSEN Data Specification}

An extended discussion of the data requirements for use
when invoking the Simons and Jorgensen algorithm is
given in section 10. For completeness we include here
the data file required in optimising the geometry of
\formaldehyde, noting that in most cases the
default settings will prove satisfactory.

{
\footnotesize
\begin{verbatim}
          TITLE
          H2CO - DZ BASIS - JORGENSEN OPTIMISATION
          ZMATRIX ANGSTROM
          C
          O 1 CO
          H 1 CH 2 HCO
          H 1 CH 2 HCO 3 180.0
          VARIABLES
          CO 1.203
          CH 1.099
          HCO 121.8
          END
          BASIS DZ
          RUNTYPE OPTIMIZE JORGENSEN
          ENTER
\end{verbatim}
}

\subsection[Transition State Location and RUNTYPE Specification]{Transition State Location and RUNTYPE Specification}

Three methods are available to search for a transition state on
a potential Surface, each driven through SADDLE specification on the
RUNTYPE directive and each relying on internal coordinate 
specification through the ZMATRIX directive.
\begin{enumerate}
\item the recommended method, a modification to the Cerjan and Miller
`trust-region' algorithm, is driven through the specification

{
\footnotesize
\begin{verbatim}
           RUNTYPE SADDLE
\end{verbatim}
}
This method performs optimisation in internal coordinates, 
and thus requires initial ZMATRIX and VARIABLES specification of the 
molecular geometry, or ZMATRIX construction from an initial set
of cartesian coordinates supplied under control of the
GEOMETRY directive. Note that the success of the method
is dependent on the quality of the initial Hessian, and
the user is reminded of the need to address this issue through
appropriate TYPE specifications on the VARIABLE definition lines
of the ZMATRIX (see Part~3 section 8).
\item the second internal coordinate--driven method is that 
based on the hill-walking algorithm due  to
Jorgensen and coworkers \cite{simons}. 
A more detailed account of
the method and associated data is given below in section 10. We note
here that the procedure is driven through additional keyword
specification on the RUNTYPE directive, thus;

{
\footnotesize
\begin{verbatim}
           RUNTYPE SADDLE JORGENSEN
\end{verbatim}
}
The method is again reliant on a quality initial hessian
for success.
\item the third method, perhaps less reliable and requiring
additional input data than the
others, is the synchronous--transit internal coordinate based 
method due to Bell and Crighton. This is again requested
through the  RUNTYPE SADDLE specification, together with
appropriate usage of the LSEARCH directive (see below).
\end{enumerate}
We consider the data input requirements for the trust-region and
synchronous--transit methods below, and those for the Jorgenson
algorithm in section 10.

\subsection[SADDLE Data Specification: Trust-Region]{SADDLE Data Specification: Trust-Region}

Four directives are provided to control the trust--region
search procedure, MINMAX, XTOL, STEPMAX and VALUE, with specifications
very similar to the corresponding directives described above
for OPTIMIZE usage.
The user should note that the
implementation is also based on maintaining a history of the
optimisation pathway, that will be worked through on each restart of
the optimisation. This appears on the output as a sequence of both
`old' and `new' calculations, with the history printed on each
restart. This printing may be suppressed through use of the
NOPRINT directive, with specification of the HISTORY keyword.

\subsubsection[SADDLE Data - MINMAX]{SADDLE Data - MINMAX}

This directive may be used to
control the number of energy evaluations and line-searches permitted
in the location of the transition state, and consists of a single 
data line read to the variables TEXT, IVAL, LINE using format (A,2I).
\begin{itemize}
\item TEXT is set to the character string MINMAX.
\item IVAL is an integer specifying  the maximum number of 
energy evaluations allowed in the optimisation.
\item LINE is an integer defining the maximum number of line searches
permitted in the optimisation.
\end{itemize}
The MINMAX directive may be omitted, when both IVAL and LINE will
be set to the maximum allowed value of 60. 
The following specification is thus 
equivalent to the default;

{
\footnotesize
\begin{verbatim}
          MINMAX 60 60
\end{verbatim}
}
{\bf Example}\\

In some cases the user may wish to perform just the
initial point on the optimisation pathway to gauge the quality
of the starting geometry though the magnitude of the gradient at
that point. This may be achieved though use of MINMAX, as shown
below:\\

{\bf Start-up Job}
{
\footnotesize
\begin{verbatim}
          TITLE\HCCH/CCH2 . RHF3-21G . START-UP JOB.
          ZMAT ANGS\C\ C 1 L1\ H 2 L2  1 A1
          X 2 1.0 1 90.0 3 180.0\ H 2 L3  4 A2   1 180.0
          VARIABLES
          L1 1.24054 TYPE 3
          L2 1.65694 TYPE 3
          L3 1.06318 TYPE 3
          A1 60.3568 TYPE 3
          A2 60.3568 TYPE 3
          END 
          RUNTYPE SADDLE
          XTOL 0.002\MINMAX 60 1
          ENTER
\end{verbatim}
}
Here we are using the MINMAX directive to terminate the
optimisation after the first line search. Note that using the
number of energy evaluations as the criterion may not
be productive, for  this will cause termination at the first
point in the evaluation of the 2nd-derivative matrix requested through
the TYPE 3 specifications.  This may then be restarted
as shown below, where the default MINMAX settings will apply.\\

{\bf Restart Job}
{
\footnotesize
\begin{verbatim}
          RESTART SADDLE
          TITLE\HCCH/CCH2 . RHF3-21G . START-UP JOB.
          ZMAT ANGS\C\ C 1 L1\ H 2 L2  1 A1
          X 2 1.0 1 90.0 3 180.0\ H 2 L3  4 A2   1 180.0
          VARIABLES
          L1 1.24054 TYPE 3
          L2 1.65694 TYPE 3
          L3 1.06318 TYPE 3
          A1 60.3568 TYPE 3
          A2 60.3568 TYPE 3
          END 
          RUNTYPE SADDLE
          XTOL 0.002
          ENTER
\end{verbatim}
}

\subsubsection[SADDLE Data - XTOL]{SADDLE Data - XTOL}

This directive may be used to define the convergence thresholds
for the optimisation, and consists of a single data line
read to the variables TEXT, TOL using format (A,F):
\begin{itemize}
\item TEXT should be set to the character string XTOL;
\item TOL should be set to the value to be used in
defining the four acceptance criteria for the convergence of the
optimisation algorithm. These criteria are:

{
\footnotesize
\begin{verbatim}
          maximum change in variables  <  TOL
          average change in variables  <  TOL * 2/3
          maximum gradient             <  TOL * 1/4
          average gradient             <  TOL * 1/6
\end{verbatim}
}
\end{itemize}
The XTOL directive may be omitted, when TOL will be set to 0.001.
The default thus corresponds to presenting the data line

{
\footnotesize
\begin{verbatim}
           XTOL 0.001
\end{verbatim}
}

\subsubsection[SADDLE Data - STEPMAX]{SADDLE Data - STEPMAX}

This directive may be used to define the 
the maximum allowed movement in any of the variables
in a single step of the transition state location. Note that 
the internal units of the variables are
bohr for bond lengths and radians for angles. 
The directive consists of a single data line
read to the variables TEXT, STEP using format (A,F);
\begin{itemize}
\item TEXT should be set to the character string STEPMAX;
\item STEP should be set to the maximum permitted movement
in any variable.
\end{itemize}
The STEPMAX directive may be omitted, when STEP will be set to 0.2.
The default thus corresponds to presenting the data line

{
\footnotesize
\begin{verbatim}
           STEPMAX 0.2
\end{verbatim}
}
There is certainly at least one circumstance where changes to
the default setting may prove crucial in achieving controlled
convergence. If the starting geometry is known to be poor, or
if ZMATRIX specification is such that a specific bond is not
explicitly defined (as can happen for example with aromatic
compounds), then the first step taken on the optimisation
may prove both excessive and counter-productive; at best several 
extra points
will be required to recover from this effect, at worst
a change of state may be induced in the SCF wavefunction. This effect
is fairly common if in addition the starting hessian is also poorly
defined. When this happens, the user should consider starting
the optimisation again, presenting a STEPMAX directive of the 
form:

{
\footnotesize
\begin{verbatim}
           STEPMAX 0.1
\end{verbatim}
}

\subsubsection[SADDLE Data - VALUE]{SADDLE Data - VALUE}

This directive may be used to 
control the accuracy of the search for a turning point during
a line search, and consists of a single data line read to
the variables TEXT, TURN using format (A,F);
\begin{itemize}
\item TEXT should be set to the character string VALUE;
\item TURN should be set to a value between 0.0 and 1.0 
that will control the accuracy of the line search procedure. Note
that the smaller TURN, the more accurate the line search.
\end{itemize}
The VALUE directive may be omitted, when TURN will be set to 0.3.
The default thus corresponds to presenting the data line

{
\footnotesize
\begin{verbatim}
            VALUE 0.3
\end{verbatim}
}


\subsection[Synchronous--Transit Data]{Synchronous--Transit Data}

The present implementation of the synchronous--transit method is driven
under specification of the LSEARCH directive. Successful results from
the method rely on the specification of not only a reasonable
guess for the initial geometry, but on presenting the equilibrium
geometries as data for the two minima involved on the potential
surface. These minima are specified on the variable definition 
lines of the
ZMATRIX, and require that the form of the ZMATRIX has been
constructed in such a way as to yield VARIABLES that transform
smoothly from one minima, though the transition state and onto
the second minima. This is shown below for the transition state
involved in the HCN to HNC isomerisation process.

{
\footnotesize
\begin{verbatim}
          TITLE
          HCN SADDLE POINT - SYNCHRONOUS TRANSIT
          ZMAT ANGS
          C
          X 1 1.0
          N 1 CN 2 90.0
          H 1 CH 2 90.0 3 HCN
          VARIABLES
          CN 1.1484 MINIMA 1.1371 1.1597
          CH 1.5960 MINIMA 1.0502 2.1429
          HCN 90.0   MINIMA 180.0  0.0
          END
          BASIS SV 4-31G
          RUNTYPE SADDLE
          LSEARCH 0 4
          ENTER
\end{verbatim}
}

\subsubsection[Synchronous--Transit Data - LSEARCH]{Synchronous--Transit Data - LSEARCH}

The LSEARCH directive may be used to request and characterise
the synchronous--transit
method, overriding  the default trust-region, and consists of a 
single data line read to the variables TEXT, LINE, IPOL
using format (A,2I)
\begin{itemize}
\item TEXT should be set to the character string LSEARCH
\item LINE is an integer used to identify the type of line search
to be performed. The two valid settings are LINE=0, requesting
subsequent line searches be based on energy evaluation alone, and
LINE=1, when both the energy and gradient of the energy will
be used at each point in each line search.
\item IPOL is an integer used to specify the form of polynomial
be employed for the principal direction
of negative curvature \cite{bell}, The two valid settings are
LINE=2  (quadratic polynomial) or LINE=4 (quartic polynomial).
\end{itemize}
The synchronous--transit method is invoked by presenting the
data line

{
\footnotesize
\begin{verbatim}
          LSEARCH 0 4
\end{verbatim}
}
which must be presented {\em after} the RUNTYPE directive.
Note that the default trust region method corresponds to the
specification

{
\footnotesize
\begin{verbatim}
          LSEARCH 0 5
\end{verbatim}
}

The LSEARCH directive may also be used to influence OPTIMISE and OPTXYZ runs. For
an OPTIMISE run, a LSEARCH 1 specification requests (as above) that function and
gradient evaluations are used during a line search.  For an OPTXYZ run, which 
usually tries to do a line search until the energy goes down, the LSEARCH 1 
specification indicates that a gradient evaluation is requested and each point is
used, even if the energy is higher. For OPTXYZ specifying the first parameter
as 0 (function evaluations) and specifying the second variable allows one to override
the standard maximum number of steps along a line (normally 3), before the hessian is
reset and the program tries again. To try for 7 points in OPTXYZ one specifies 
{
\footnotesize
\begin{verbatim}
          LSEARCH 0 7
\end{verbatim}
}

\subsubsection[Synchronous--Transit Data - TOLMAX]{Synchronous--Transit Data - TOLMAX}

The TOLMAX directive  may be used to control how
far a search for a minimum in the n-1 subspace \cite{bell} may proceed
before another search for a maximum is performed. 
Smaller values will
cause the program to search for a maximum more often. The
default is

{
\footnotesize
\begin{verbatim}
          TOLMAX 0.1
\end{verbatim}
}

\subsubsection[Synchronous--Transit Data - TOLSTEP]{Synchronous--Transit Data - TOLSTEP}

The TOLSTEP  directive
is used to maintain `good' conjugate directions to the principal
direction of negative curvature. If the step along this
direction was too large in the previous iteration then the program takes
a small step in order to estimate a better set of conjugate
directions. The default  is equivalent to

{
\footnotesize
\begin{verbatim}
          TOLSTEP 0.1
\end{verbatim}
}

\subsubsection[Synchronous--Transit Data - TANSTEP]{Synchronous--Transit Data - TANSTEP}

The TANSTEP directive may be used when the
TOLSTEP test requires the conjugate directions to be recalculated.
A step (TANSTEP*previous step length) is taken along the current
tangent to the polynomial and the function and gradients are
calculated at this point. The default is

{
\footnotesize
\begin{verbatim}
          TANSTEP 0.1
\end{verbatim}
}

\subsubsection[Synchronous--Transit Data - MINMAX]{Synchronous--Transit Data - MINMAX}

This directive may be used to
control the number of energy evaluations and line-searches permitted
in optimising a given structure. The default is

{
\footnotesize
\begin{verbatim}
          MINMAX 60 60
\end{verbatim}
}
The first integer refers to the maximum number of energy evaluations allowed,
and the second to the maximum number of line searches.


\subsubsection[Synchronous--Transit Data - XTOL]{Synchronous--Transit Data - XTOL}

Defines the four acceptance criteria for the convergence of the
synchronous--transit algorithm. The criteria are:

{
\footnotesize
\begin{verbatim}
          maximum change in variables  <  xtol
          average change in variables  <  xtol*2/3
          maximum gradient             <  xtol*1/4
          average gradient             <  xtol*1/6
\end{verbatim}
}
The default corresponds to presenting the data line
{
\footnotesize
\begin{verbatim}
          XTOL 0.001
\end{verbatim}
}

\subsubsection[Synchronous--Transit Data - ]{Synchronous--Transit Data - STEPMAX}

Defines the maximum allowed movement in any of the variables
in a single step. The internal units of the variables are
bohr for bond lengths and radians for angles. The default is
equivalent to

{
\footnotesize
\begin{verbatim}
           STEPMAX 0.2
\end{verbatim}
}

\subsection[Restoring the Force Constant Matrix - FCM]{Restoring the Force Constant Matrix - FCM}

An FCM directive may be given to specify the Force Constant Matrix (Hessian) to be restored
in an OPTIMISE or SADDLE calculation. Specifying FCM or a dumpfile name on a RUNTYPE OPTIMISE
or RUNTYPE SADDLE directive is equivalent to providing a separate FCM directive.
The directive consists of a single line containing
\begin{verbatim}
           FCM  DD  IBLOCK  TYPE
\end{verbatim}
where DD is the name of the dumfile containing the Hessian and IBLOCK is its starting block.
TYPE is the type of Hessian requested and can be any of : MP2, SCF or OPTIMISE, where the
former denote analytical hessians generated using MP2 or SCF respectively and OPTIMISE 
requests the Hessian produced during an OPTIMISE or SADDLE run. If the TYPE is omitted,
the dumpfile is searched for hessian sections in the order given above. The Dumpfile 
specification may be omitted, in which case the current dumpfile is searched.
For example the next sets of input lines are equivalent, assuming the default dumpfile :
{
\footnotesize
\begin{verbatim}
           RUNTYPE OPTIMISE
           FCM ED3 1
           ........
           RUNTYPE OPTIMISE FCM
           ........
           RUNTYPE OPTIMISE ED3 1
\end{verbatim}
}
The FCM directive may also read
{
\footnotesize
\begin{verbatim}
           FCM UNIT7
\end{verbatim}
}
in which case the Force Constant Matrix is read from the punchfile.
 


\section{Jorgensen and Simons Optimisation Algorithm}

An alternative  stationary point optimisation procedure, 
code-named JORGENSEN, is available within GAMESS--UK.
An efficient quasi 
Newton--Raphson algorithm for locating transition states, 
this  procedure is based on a modification to the Newton--Raphson 
step first proposed by Cerjan and Miller \cite{cerjan}, 
although the major 
part of the algorithm is founded on the later developments of Simons, 
Jorgensen and coworkers \cite{simons}. 
The algorithm is capable of locating transition states 
even if started in the `wrong' region of the energy surface, and, by 
invoking Hessian mode following, can locate transition states for 
alternative rearrangement and/or dissociation reactions from the same 
starting point. The algorithm may also be used to locate minima on a 
surface.

A description of the formalism and the ideas behind it, together with a 
description of the algorithm and some practical examples are given in 
reference \cite{baker}.

\subsection{Invocation of The Algorithm and Associated Parameters}

The Jorgensen and Simons optimisation algorithm is invoked by 
appending the keyword JORGENSEN 
on the appropriate RUNTYPE directive i.e.

{
\footnotesize
\begin{verbatim}
       RUNTYPE SADDLE  JORGENSEN  ......  Search for saddle point.

       RUNTYPE OPTIMIZE JORGENSEN ......  Search for minimum point. 
\end{verbatim}
}
Omission of  the keyword will cause the program to default to 
the optimisation algorithms described above.
The following discussion makes 
the assumption that a search is being made for such a stationary point.
If no other relevant directives are specified the default is to 
evaluate the Hessian according to the TYPE specifications in the 
VARIABLES definition lines of the ZMATRIX directive.
A Hessian may still, however, be restored from a  foreign dumpfile 
by specifying the LFN, starting block and 
section number of the foreign dumpfile
in the 4th and 5th fields on the RUNTYPE directive line, e.g.

{
\footnotesize
\begin{verbatim}
          RUNTYPE SADDLE JORGENSEN ED4 1  
\end{verbatim}
}
In addition to the keyword JORGENSEN, there are a number
of sub-directives that may be used to control 
the search procedure. These should be specified immediately
following the RUNTYPE directive, and are summarised below.
 
\subsection{POWELL}

The POWELL directive consists of a single data line comprising
the keyword POWELL in the first data field. 
The directive may be used to request
updating of the  Hessian using the Powell update procedure,
and should be specified in Transition state location.

\subsection{BFGS}

The BFGS directive consists of a single data line comprising
the keyword BFGS in the first data field. 
The directive may be used to request
updating of the  Hessian using the BFGS update procedure,
and should be specified in geometry optimisation.

\subsection{BFGX}

The BFGX directive may be used to request use of a modified
BFGS update procedure in geometry optimisation 
calculations, with safeguards to ensure retention of a 
positive definite hessian.

\subsection{RECALCULATE} 

This directive consists of a single data line
read to the variables TEXT, NSTEP using format (A,I).
\begin{itemize}
\item TEXT should be set to the character string RECALCULATE
\item NSTEP is an integer used to specify
the frequency of recalculation of the Hessian
using the method dictated by the "TYPE" keywords
of the Variables Definition line of the ZMATRIX directive.
\end{itemize}
The directive may be omitted, when the default of updating the
hessian only will hold throughout. An alternative specification
permits this recalculation to be suppressed through a data line
of the form

{
\footnotesize
\begin{verbatim}
         RECALCULATE OFF
\end{verbatim}
}

\subsection{CUTOFFS}

This directive may be used to request
relaxation of  certain constraints in optimisation, and should be
used with caution.

\subsection{MINE}

The MINE directive consists of a single data line
read to variables TEXT, VMIN using format (A,F):
\begin{itemize}
\item TEXT should be set to the character string MINE;
\item VMIN should be set to the
minimum allowed value for eigenvalues of the Hessian matrix
\end{itemize}

\subsection{MAXE}

The MAXE directive consists of a single data line
read to variables TEXT, VMAX using format (A,F):
\begin{itemize}
\item TEXT should be set to the character string MAXE;
\item VMAX should be set to the
maximum allowed value for eigenvalues of the Hessian matrix
\end{itemize}

\subsection{MAXJORGEN}

The MAXJORGEN directive consists of a single data line
read to variables TEXT, MAXJOR using format (A,I):
\begin{itemize}
\item TEXT should be set to the character string MAXJORGEN;
\item MAXJOR should be set to the maximum number of 
allowed cycles in the optimisation.
\end{itemize}
In the absence of this directive, MAXJOR will be set to 40.

\subsection{RFO}

The RFO directive consists of a single data line
read to variables TEXT, TXTRFO using format (2A);
\begin{itemize}
\item TEXT should be set to the character string RFO;
\item TXTRFO should be used to control the nature
of the steps taken in the search procedure, and may be
set to the character string ON or OFF.
The default specification, corresponding to TXTRFO=ON, results
in the RFO or P-RFO steps only (see \cite{baker}).
Specifying OFF will
result in the taking of Newton-Raphson steps where appropriate 
instead of RFO or P-RFO steps.
\end{itemize}

\subsection{OPTPRINT}

This directive may be used to control
diagnostic output during optimisation.
A data line of the form

{
\footnotesize
\begin{verbatim}
          OPTPRINT ON
\end{verbatim}
}
may be used to increase diagnostic output.

\subsection{Example 1}

The following data file may be used to perform the 
search for the FCN/FNC transition state, using an STO-3G basis set.
The initial Hessian is computed through TYPE 3 specification,
with the Powell update requested throughout Jorgensen optimisation.

{
\footnotesize
\begin{verbatim}
          TITLE                
          FCN/FNC TS SEARCH . STO3G BASIS.
          ZMAT ANGS                         
          C                    
          N 1 L1               
          F 1 L2 2 A1
          VARIABLES            
          L1 1.2   TYPE 3
          L2 1.3   TYPE 3
          A1 135.0 TYPE 3
          END                     
          BASIS  STO3G
          RUNTYPE SADDLE JORGENSEN 
          POWELL
          MAXJOR 55
          RECALC OFF
          RFO OFF
          CUTOFFS
          OPTPRINT ON
          XTOL 0.0018
          ENTER
\end{verbatim}
}

%%%%%%%%%%%%%%%%%%%%%%%%%%%%%%%%%%%%%%%%%%%%%%%%%%%%%%%%%%%%%%%%%%%%%%%%%%%%%
\section{Optimisation using DL-FIND}
%%%%%%%%%%%%%%%%%%%%%%%%%%%%%%%%%%%%%%%%%%%%%%%%%%%%%%%%%%%%%%%%%%%%%%%%%%%%%

DL-FIND is a self-contained module that is included into
GAMESS-UK. Additional information on DL-FIND, as well as the
developmental version of the code can be found on the DL-FIND website
at:

http://ccpforge.cse.rl.ac.uk/projects/dl-find/

DL-FIND offers several methods for geometry optimisation and transition state
search. The geometry can be optimised in a range of coordinate systems:
Cartesian coordinates, Mass-weighted Cartesian coordinates, delocalised
(redundant) internal coordinates, and hybrid delocalised internal
coordinates.

L-BFGS is the recommended method for minimum search. 

Transition state search may be done by P-RFO, an improved version of the dimer
method, or nudged-elastic band using a climbing image.

DL-FIND is invoked by the runtype optimize (or saddle) and the directive
{
\footnotesize
\begin{verbatim}
          DLFIND
             DL-FIND directives
          END
\end{verbatim}
}



%%%%%%%%%%%%%%%%%%%%%%%%%%%%%%%%%%%%%%%%%%%%%%%%%%%%%%%%%%%%%%%%%%%%%%%%%%%%%
\subsection{DL-FIND directives}

\subsubsection{COORDINATES}

The directive COORDINATES TEXT specifies the coordinate system in which the
optimisation should be performed. Possible choices for TEXT are:
\begin{description}
\item[CART] Cartesian coordinates (default)
\item[MASS] Mass-weighted Cartesian coordinates
\item[DLC] Delocalised internal coordinates. A redundant set of bonds, angles,
  torsions, and sometimes inversions is created. A non-redundant combination
  of them is found by diagonalising the spectroscopic G-matrix. The
  optimisation is performed in this non-redundant set.
\item[TC] Same as DLC, but the redundant set consists only of bonds -- all
  atoms in the system are connected.
\item[HDLC] Hybrid delocalised internal coordinates. The system is
  partitioned into fragments. Delocalised coordinates (as in DLC) are used
  within each fragment. The fragments are coupled via Cartesian
  coordinates. This version is recommended for large systems, as the
  coordinate transformation scales linearly with the number of atoms.
\end{description}

\subsubsection{DIMER}

The directive DIMER is used to start a transition state search using the dimer
method. Two images of the system (their distance is specified by the DELTA
directive) are calculated. They are optimised along the force in all
directions perpendicular to the dimer axis, and against the force along the
dimer axis. Thus the system converges to a first order saddle point.

After a rotation step, the gradient can be interpolated (default) or
recalculated. In this case, specify the directive as DIMER NOINTERPOLATION.

\subsubsection{DELTA}

DELTA specifies the dimer distance and the elongation for finite-difference
Hessian evaluation. The units are atomic units if Cartesian coordinates are
used, and undefined if delocalised internals are used. Default:

{
\footnotesize
\begin{verbatim}
          DELTA 0.01
\end{verbatim}
}

\subsubsection{NEB}

The directive NEB specifies a nudged elastic band calculations. The
improved-tangent NEB algorithm with a climbing image is implemented. The
L-BFGS optimiser is recommended with NEB, although other optimisers may as
well be tried. The syntax is:

{
\footnotesize
\begin{verbatim}
          NEB NIMAGE K
\end{verbatim}
}

Where NIMAGE is an integer specifying the number of images (default: 10), and
K is a real number specifying the NEB force constant (default: 0.01).

Two structures have to be provided for an NEB calculation. One endpoint will
be the geometry provided in the normal GAMESS input. The other structure will
be provided by the directive GEOMETRY.

\subsubsection{NEBMODE}

The directive NEBMODE specifies details of an NEB calculation.

{
\footnotesize
\begin{verbatim}
          NEBMODE MODE CART
\end{verbatim}
} 

Where MODE can be FREE, FROZEN, or PERP. It specifies if the endpoints of the
NEB path will be freely minimised, frozen, or only free to move perpendicular
to the path, respectively. FROZEN is the default. 

If CART is specified, only the initialisation of the path will be done in the
coordinate system specified by COORDINATES. The following minimisation will be
done in Cartesian coordinates.

\subsubsection{GEOMETRY}

The directive GEOMETRY specifies the endpoint of an initial NEB path. The
other endpoint is specified by the normal input geometry. The directive
GEOMETRY can also be used to specify an initial dimer direction.

{
\footnotesize
\begin{verbatim}
          GEOMETRY TEXT
\end{verbatim}
} 

TEXT is the name of a file in xyz format. It has to contain the same atoms as
the GAMESS input geometry.

\subsubsection{OPTIMISER}

The directive OPTIMISER TEXT specifies the optimisation algorithm. Possible
choices for TEXT are:
\begin{description}
\item[LBFGS] Limited-memory Broyden--Fletcher--Goldfarb--Shanno optimisation.
  Recommended for minimum search, NEB, and the dimer method. The time and
  memory requirements spent for determining the search direction and step
  length scale linearly with the system size. And additional integer parameter
  can be specified, which determines the number of steps kept in
  memory. Default is the number of degrees of freedom.
\item[PRFO] The partitioned rational function optimisation method. Recommended
  for transition state search (unless done by the dimer or NEB methods). It
  requires the calculation of the Hessian. The UPDATE directive (see below)
  can be used to specify how the Hessian should be updated.
\item[CG] Conjugate gradient method following Polak--Ribi\`ere \cite{pol69}.
\item[SD] Steepest descent.
\item[DYN] Damped molecular dynamics. Four additional real parameters can be
  specified, which determine (1) the time step in atomic units (default 1.0),
  (2) the start friction (default: 0.3), (3) the factor to reduce the friction
  each time the energy decreases (default: 0.95), and the friction to apply if
  the energy increases (default 0.3). The frictions are defined so that 0
  corresponds to free (undamped) dynamics, and 1 corresponds to steepest
  descent. This variable friction facilitates convergence to an energy minimum.
\end{description}

\subsubsection{UPDATE}

The directive UPDATE specifies the Hessian update details in case an explicit
Hessian is calculated (i.e. for the PRFO optimiser). Syntax:

{
\footnotesize
\begin{verbatim}
          UPDATE METHOD RECALC FD SOFT
\end{verbatim}
} 

Where RECALC is the (integer) number of updates before the Hessian is
recalculated (default: 100), FD is 1 or 2 to use a one-point or two-point
formula to calculate the finite-difference Hessian (default: 2). SOFT is a
positive real number (default: 0.003). Eigenmodes with an absolute eigenvalue
below SOFT are ignored by the P-RFO algorithm. This avoids steps into the
translation and rotation directions. METHOD may be one of
\begin{description}
\item[BOFILL] Bofill update of the Hessian (default)
\item[POWELL] Powell update of the Hessian
\item[NONE]   No update, recalculate the Hessian in each step
\end{description}

\subsubsection{XTOL}

In accordance to the other optimisers, XTOL specifies the convergence
criterion applied to the optimisation. Note that XTOL has to be specified
within the DLFIND block. Default:

{
\footnotesize
\begin{verbatim}
          XTOL 0.003
\end{verbatim}
} 

\subsubsection{MINMAX}

In accordance to the other optimisers, MINMAX specifies the maximum number of
energy evaluations. Note that MINMAX has to be specified
within the DLFIND block. Default:

{
\footnotesize
\begin{verbatim}
          MINMAX 60
\end{verbatim}
} 

\subsubsection{STEPMAX}

In accordance to the other optimisers, STEPMAX specifies the maximum step in
one coordinate component. Note that STEPMAX has to be specified
within the DLFIND block. Default:

{
\footnotesize
\begin{verbatim}
          STEPMAX 0.5
\end{verbatim}
} 

%%%%%%%%%%%%%%%%%%%%%%%%%%%%%%%%%%%%%%%%%%%%%%%%%%%%%%%%%%%%%%%%%%%%%%%%%%%%%
\subsection{Example 1}

The following data file performs a simple minimisation of H$_2$CO in redundant
internal coordinates using the L-BFGS algorithm.

{
\footnotesize
\begin{verbatim}
          TITLE
          H2CO - DZ - DL-FIND energy = -113.8307609
          ZMATRIX ANGSTROM
          C
          O 1 CO
          H 1 CH 2 HCO
          H 1 CH 2 HCO 3 180.0
          VARIABLES
          CO 1.203
          CH 1.099
          HCO 121.8
          END
          BASIS DZ
          RUNTYPE OPTIMIZE
          DLFIND
            COORDINATES DLC
          END
          ENTER
\end{verbatim}
}

%%%%%%%%%%%%%%%%%%%%%%%%%%%%%%%%%%%%%%%%%%%%%%%%%%%%%%%%%%%%%%%%%%%%%%%%%%%%%
\subsection{Example 2}

The following data files illustrate a NEB optimisation of part of the reaction
path of H$_2$CO dissociation.

{
\footnotesize
\begin{verbatim}
          TITLE
          H2CO - DZ - DL-FIND NEB energy = -113.6497325
          ZMATRIX ANGSTROM
          C
          O 1 CO
          H 1 L2 2 A1
          H 1 L3 2 A2 3 0.0
          VARIABLES
          CO 1.25
          L2 1.5
          L3 1.1
          A1 120.0
          A2 170.0
          END
          BASIS DZ
          RUNTYPE OPTIMIZE
          DLFIND
            NEB 6 0.01
            GEOMETRY neb_endpoint.xyz
          END
          ENTER
\end{verbatim}
}

An additional file with the name neb\_endpoint.xyz is required. It has the
contents:

{
\footnotesize
\begin{verbatim}
            4

C    0.3526549   0.4601460   0.0000000
O    0.1899991  -0.6367263   0.0000000
H   -1.3596765   1.1431771   0.0000000
H   -0.7056650   1.4269885   0.0000000
\end{verbatim}
}

%%%%%%%%%%%%%%%%%%%%%%%%%%%%%%%%%%%%%%%%%%%%%%%%%%%%%%%%%%%%%%%%%%%%%%%%%%%%%
%%%%%%%%%%%%%%%%%%%%%%%%%%%%%%%%%%%%%%%%%%%%%%%%%%%%%%%%%%%%%%%%%%%%%%%%%%%%%
\section[The ENTER Directive]{The ENTER Directive}

The ENTER directive should be the last directive specified, since
it initiates the calculation. The directive may also be used to
nominate the section(s) on the Dumpfile where eigenvectors generated
by the processing requested through SCFTYPE during the present run of
the program are to be written. Since more than one set of vectors may be
generated by several of the SCF options (two, for example, in open-shell
RHF, GVB, UHF, MCSCF and CASSCF calculations) it follows that multiple
section specification may feature on the ENTER directive.

The directive consists of a single data line read to variables TEXT,
ISECT1 and ISECT2 using format (A,2I).
\begin{itemize}
\item TEXT should be set to the character string ENTER;
\item ISECT1 may be used to specify the section number on the Dumpfile
where the first set of vectors associated with the requested SCFTYPE
are to be written;
\item ISECT2 may be used to specify the section number on the Dumpfile
where the second set of vectors associated with the requested SCFTYPE
are to be written.
\end{itemize}

If both ISECT1 and ISECT are omitted, it will be assumed that the
default sections of Table~\ref{table:2} are in effect. Note that if
explicit section specification is used, then ISECT2 should be omitted
in those cases involving a single set of eigenvectors. We illustrate
below typical ENTER data lines as a function of SCFTYPE, assuming the
defaults of Table~\ref{table:2} are to be overwritten;

\begin{itemize}
\item For a closed shell SCF calculation the data line
{
\footnotesize
\begin{verbatim}
          ENTER 10     
\end{verbatim}
}
will route the eigenvectors to section 10 of the Dumpfile.
\item For a UHF calculation the data line
{
\footnotesize
\begin{verbatim}
          ENTER 12 13
\end{verbatim}
}
will route the $\alpha$-spin eigenvectors to section 12 of the Dumpfile
$\beta$-spin vectors  to section 13.
\item For a GVB or open shell calculation the data line
{
\footnotesize
\begin{verbatim}
          ENTER 16 17
\end{verbatim}
}
will  route the un-canonicalised vectors to section 16, and the
canonicalised orbitals to section 17. Omitting the second section
in such calculations will result in the canonicalised set {\em only}
appearing on the Dumpfile, the un-canonicalised set being overwritten
on convergence of the SCF.
\item For a CASSCF calculation the data line
{
\footnotesize
\begin{verbatim}
          ENTER 18 19
\end{verbatim}
}
will route the CASSCF orbitals to section 18, and the canonicalised CASSCF
vectors to section 19. Note that the present set of CI coefficients will
be appended to this section.
\end{itemize}


\section[STOP]{STOP}

The STOP directive may be presented as an alternative to ENTER and, 
with the same syntax, may be used simply as a check of the 
data input. No modifications
are made to the Dumpfile, execution terminating once data processing
is complete.

\clearpage

\begin{thebibliography}{10}

\bibitem{pulay} 
P. Pulay,
  Chem. Phys. Lett. {\bf 73} (1980) 393, \doi{10.1016/0009-2614(80)80396-4};
P. Pulay,
  J. Comp. Chem. {\bf 3} (1982) 556--560, \doi{10.1002/jcc.540030413}.

\bibitem{bobrow}
F.W. Bobrowicz and W.A. Goddard, in `Modern Theoretical Chemistry',
Vol. 3, ed. H.F. Schaefer, Plenum, New York (1977) 79.

\bibitem{roos} 
B. Jonsson, B.O. Roos, P.R. Taylor and P.E.M. Siegbahn,
  J. Chem. Phys. {\bf 74} (1981) 4566, \doi{10.1063/1.441645};
B.O. Roos, P. Linse, P.E.M. Siegbahn and M.R.A. Blomberg,
  Chem. Phys. {\bf 66} (1982) 197, \doi{10.1016/0301-0104(82)88019-1};
P.J. Knowles, G.J. Sexton and N.C. Handy,
  Chem. Phys.  {\bf 72} (1982) 337, \doi{10.1016/0301-0104(82)85131-8}:
The original CASSCF module, as developed by Dr. P.J. Knowles
was incorporated into GAMESS in April 1983.

\bibitem{bell} 
S. Bell and J.S. Crighton, J. Chem. Phys. {\bf 80} (1984) 2464,
\doi{10.1063/1.446996}.

\bibitem{cerjan} 
C.J. Cerjan and W.H. Miller, J. Chem. Phys. {\bf 75} (1981) 2800,
\doi{10.1063/1.442352}.

\bibitem{simons}  
J. Simons, P. Jorgensen, H. Taylor and J. Ozment,
J. Phys. Chem. {\bf 87} (1983) 2745,
\doi{10.1021/j100238a013};
A. Banerjee, N. Adams, J. Simons and R. Shepard,
J. Phys. Chem. {\bf 89} (1985) 52,
\doi{10.1021/j100247a015}.

\bibitem{baker}  
J. Baker,
  J. Comp. Chem. {\bf 7} (1986) 385--395, \doi{10.1002/jcc.540070402}.

\bibitem{moncrieff} 
D. Moncrieff and V.R. Saunders, ATMOL-Introduction 
Notes, UMRCC, May, 1986; Cyber-205 Note Number 32, 
UMRCC, September, 1985; V.R. Saunders and M.F. Guest, ATMOL3 Part 9, 
RL-76-106 (1976);
M.F. Guest and V.R. Saunders,
  Mol. Phys. {\bf 28} (1974) 819, \doi{10.1080/00268977400102171}.

\bibitem{dupuis} 
M. Dupuis and H.F. King,
  Int. J. Quantum Chem. {\bf 11} (1977) 613, \doi{10.1002/qua.560110408};
M. Dupuis and H.F. King, J. Chem. Phys {\bf 68} (1978) 3998,
\doi{10.1063/1.436313};
H.F. King and M. Dupuis, J. Comp. Phys. {\bf 21} (1976) 144,
\doi{10.1016/0021-9991(76)90008-5};
M. Dupuis, J. Rys and H.F. King, J. Chem. Phys. {\bf 65} (1976) 111,
\doi{10.1063/1.432807}.

\bibitem{saunders} 
V.R. Saunders and I.H. Hillier,
  Int. J. Quant. Chem. {\bf 7} (1973) 699, \doi{10.1002/qua.560070407}.

\bibitem{warren}
R.W. Warren and B.I. Dunlap,
{\it Fractional occupation numbers and density functional energy gradients
     with the linear combination of Gaussian-type orbitals approach},
Chem. Phys. Lett. {\bf 262} (1996) 384--392,
\doi{10.1016/0009-2614(96)01107-4}.
   
\bibitem{stephens94}
P.J.~Stephens, F.J.~Devlin, C.F.~Chabalowski, and M.J.~Frisch,
{\it Ab initio calculation of vibrational absorption and circular
dichroism spectra using density functional force fields},
J. Phys. Chem. {\bf 98} (1994) 11623-11627,
\doi{10.1021/j100096a001}.

\bibitem{hertwig97}
R.H.~Hertwig, and W. Koch,
{\it On the parameterization of the local correlation functional:
What is Becke-3-LYP?},
  Chem. Phys. Lett. {\bf 268} (1997) 345-351,
  \doi{10.1016/S0009-2614(97)00207-8}.

\bibitem{zhao04}
Y.~Zhao, B.J.~Lynch, and D.G.~Truhlar,
{\it Development and assessment of a new hybrid density functional
model for thermochemical kinetics}
J. Phys. Chem. {\bf A108} (2004) 2715-2719,
\doi{10.1021/jp049908s}.

\bibitem{becke96}
A.D.~Becke,
{\it Density-functional thermochemistry. IV.
A new dynamical correlation functional and implications
for exact-exchange mixing},
J. Chem. Phys. {\bf 104} (1996) 1040-1046,
\doi{10.1063/1.470829}.

\bibitem{adamson98}
R.D.~Adamson, P.M.W.~Gill, and J.A.~Pople,
{\it Empirical density functionals},
Chem. Phys. Lett. {\bf 284} (1998) 6-11,
\doi{10.1016/S0009-2614(97)01282-7}.

\bibitem{hamprecht98}
F.A.~Hamprecht, A.J.~Cohen, D.J.~Tozer, and N.C.~Handy, 
{\it Development and assessment of new exchange-correlation functionals},
J. Chem. Phys. {\bf 109} (1998) 6264-6271,
\doi{10.1063/1.477267}.

\bibitem{boese00}
A.D.~Boese, N.L.~Doltsinis, N.C.~Handy, and M.~Sprik,
{\it New generalized gradient approximation functionals},
J. Chem. Phys. {\bf 112} (2000) 1670-1678,
\doi{10.1063/1.480732}.

\bibitem{boese01}
A.D.~Boese and N.C.~Handy,
{\it A new parametriztion of exchange-correlation generalized gradient
approximation functionals},
J. Chem. Phys. {\bf 114} (2001) 5497-5503,
\doi{10.1063/1.1347371}.

\bibitem{perdew96}
J.P.~Perdew, K.~Burke, and M.~Ernzerhof,
{\it Generalized gradient approximation made simple},
Phys. Rev. Lett. {\bf 77} (1996) 3865-3868,
\doi{10.1103/PhysRevLett.77.3865}.

\bibitem{zhang98}
Y.~Zhang, and W.~Yang,
{\it Comment on: Generalized gradient approximation made simple},
Phys. Rev. Lett. {\bf 80} (1998) 890-890,
\doi{10.1103/PhysRevLett.80.890}.

\bibitem{hammer99}
B.~Hammer, L.B.~Hansen, and J.K.~N{\/o}rskov,
{\it Improved adsorption energetics within density-functional theory using
     revised Perdew-Burke-Ernzerhof functionals},
Phys. Rev. B. {\bf 59} (1999) 7413-7421,
\doi{10.1103/PhysRevB.59.7413}.

\bibitem{perdew92}
J.P.~Perdew, J.A.~Chevary, S.H.~Vosko, K.A.~Jackson,
M.R.~Pederson, D.J.~Singh, and C.~Fiolhais,
{\it Atoms, molecules, solids and surfaces:
Applications of the generalized gradient approximation
for exchange and correlation},
Phys. Rev. {\bf B46} (1992) 6671--6687,
\doi{10.1103/PhysRevB.46.6671}.

\bibitem{perdew92a}
J.P.~Perdew, and Y.~Wang,
{\it Accurate and simple analytic representation of
the electron-gas correlation energy},
Phys. Rev. {\bf B45} (1992) 13244-13249,
\doi{10.1103/PhysRevB.45.13244}.

\bibitem{becke97}
A.D.~Becke,  J. Chem. Phys. {\bf 107} (1997) 8554,
\doi{10.1063/1.475007}.

\bibitem{wilson01}
P.J.~Wilson, T.J.~Bradley, and D.J.~Tozer,
{\it Hybrid exchange-correlation functional determined from thermochemical
     data and {\em ab initio} potentials},
J. Chem. Phys. {\bf 115} (2001) 9233-9242,
\doi{10.1063/1.1412605}.

\bibitem{keal05}
T.W.~Keal, and D.J.~Tozer,
{\it Semiempirical hybrid functional with improved performance in an extensive
     chemical assessment},
J. Chem. Phys. {\bf 123} (2005) 121103,
\doi{10.1063/1.2061227}.

\bibitem{grimme06}
S.~Grimme, 
{\it Semiempirical GGA-type density functional constructed with a long-range
     dispersion correction},
J. Comp. Chem. {\bf 27} (2006) 1787-1799,
\doi{10.1002/jcc.20495}.

\bibitem{mura96}
M.E. Mura, and P.J. Knowles,
{\it Improved radial grids for quadrature in molecular density-functional
calculations},
J. Chem. Phys. {\bf 104} (1996) 9848-9858,
\doi{10.1063/1.471749}.

\bibitem{strat96}
R.E. Stratmann, G.E. Scuseria, and M.J. Frisch,
{\it Achieving linear scaling in exchange-correlation density functional
quadratures},
Chem. Phys. Lett. {\bf 257} (1996) 213-223,
\doi{10.1016/0009-2614(96)00600-8}.

\bibitem{john95}
B.G. Johnson,
{\it Development, implementation and applications of efficient methodologies
for density functional calculations} pages 169 and further,
In: J. M. Seminario and P. Politzer (Ed.),
{\it Modern Density Functional Theory: A Tool for Chemistry},
Theoretical and Computational Chemistry, Vol. 2
(Elsevier Science B.V., 1995)

\bibitem{becke93}
A.D. Becke,
{\it Density-functional thermochemistry. III.
The role of exact exchange}, 
J. Chem. Phys. {\bf 98} (1993) 5648,
\doi{10.1063/1.464913}.

\bibitem{sg1}
P.M.W. Gill, B.G. Johnson, and J.A. Pople,
{\it A standard grid for density functional calculations},
Chem. Phys. Lett. {\bf 209} (1993) 506-512,
\doi{10.1016/0009-2614(93)80125-9}.

\bibitem{johnson93}
B.G. Johnson, P.M.W. Gill, J.A. Pople,
{\it The performance of a family of density functional methods},
J.~Chem.~Phys.~ {\bf 98} (1993) 5612-5626,
\doi{10.1063/1.464906}.

\bibitem{murray93}
C.W. Murray, N.C. Handy, and G.J. Laming,
{\it Quadrature schemes for integrals of density functional theory},
Mol. Phys. {\bf 78} (1993) 997,
\doi{10.1080/00268979300100651}.

\bibitem{becke88}
A.D. Becke,
{\it Density-functional exchange-energy approximation with correct asymptotic 
behaviour},
Phys. Rev. {\bf A38} (1988) 3098,
\doi{10.1103/PhysRevA.38.3098}.

\bibitem{becke88a}
A.D. Becke,
{\it A multicenter numerical integration scheme for polyatomic molecules},
J. Chem. Phys. {\bf 88} (1988) 2547,
\doi{10.1063/1.454033}.

\bibitem{lyp}
C. Lee, W. Yang, and R.G. Parr,
{\it Development of the Colle-Salvetti correlation-energy formula into a
functional of the density},
Phys. Rev. {\bf B37} (1988) 785-789,
\doi{10.1103/PhysRevB.37.785}.

\bibitem{perdew86}
John P. Perdew,
{\it Density-functional approximation for the correlation energy of the 
inhomogeneous electron gas},
Phys. Rev. {\bf B33} (1986) 8822-8824,
\doi{10.1103/PhysRevB.33.8822}.

\bibitem{perdew81}
John P. Perdew, and Alex Zunger,
{\it Self-interaction correction to density-functional approximations for 
many-electron systems},
Phys. Rev. {\bf B23} (1981) 5048-5079,
\doi{10.1103/PhysRevB.23.5048}.

\bibitem{vosko80}
S.J. Vosko, L. Wilk, and M. Nusair,
Can. J. Phys. {\bf 58} (1980) 1200,
\doi{10.1139/p80-159}.

\bibitem{filatov97c}
M. Filatov, and W. Thiel,
{\it A nonlocal correlation energy density functional from a
Coulomb hole model},
  Int. J. Quant. Chem. {\bf 62} (1997) 603-616.
  % \doi{10.1002/(SICI)1097-461X(1997)62:6<603::AID-QUA4>3.0.CO;2-#}.
  % Although the above is the official DOI it contains an '#' character
  % that causes problems in LaTeX as well as with the DOI lookup at
  % http://dx.doi.org/ which renders it useless. So much for this prestigious
  % journal.

\bibitem{filatov97x}
M. Filatov, and W. Thiel,
{\it A new gradient-corrected exchange-correlation density functional},
Mol. Phys. {\bf 91} (1997) 847-859,
\doi{10.1080/002689797170950}.

\bibitem{dunlap79}
B.I. Dunlap, J.W.D. Connolly, and J.R. Sabin,
{\it On some approximations in applications of $X\alpha$ theory},
J. Chem. Phys. {\bf 71} (1979) 3396-3402,
\doi{10.1063/1.438728}.

\bibitem{godbout}
N. Godbout, D. R. Salahub, J. Andzelm and E. Wimmer,
Can. J. Chem. {\bf 70}, (1992) 560,
\doi{10.1139/v92-079}.
 
\bibitem{ahlrichs}
K. Eichkorn, O. Treutler, H. Ohm, M. Haser and R. Ahlrichs,
Chem. Phys. Lett. {\bf 240} (1995) 283,
\doi{10.1016/0009-2614(95)00621-A};
K. Eichkorn, F. Weigend, O. Treutler and R. Ahlrichs,
  Theor. Chim. Acta {\bf 97} (1997) 119, \doi{10.1007/s002140050244}.

\bibitem{lebe99}
V. I. Lebedev and D. N. Laikov,
{\it A quadrature formula for the sphere of the 131st
algebraic order of accuracy},
Doklady Mathematics {\bf 59} (1999) 477-481.
% No doi available.

%\bibitem{lebe75}
%V.I. Lebedev,
%{\it Values of the nodes and weights of ninth to seventeenth order
%Gauss-Markov quadrature formulae invariant under the octahedron group
%with inversion},
%Computational Mathematics and Mathematical Physics {\bf 15} (1975) 44-51.

%\bibitem{lebe76}
%V.I. Lebedev
%{\it Quadratures on a sphere} (Eng.),
%Computational Mathematics and Mathematical Physics {\bf 16} (1976) 10-24.

%\bibitem{lebe77}
%V.I. Lebedev,
%{\it Spherical quadrature formulas exact to orders 25-29} (Eng.)
%Siberian Mathematical Journal {\bf 18} (1977) 99-107. \newline
%{\bf Erratum:} There is a typo on page 103 of this paper. For {\it n = 27}
%{\em m$_5$}should read 0.9896948074629, instead of the
%{\em 0.9869...} printed.

\bibitem{kohn65}
W. Kohn, and L. J. Sham,
{\it Self-Consistent Equations Including Exchange and Correlation Effects},
Phys. Rev. {\bf 140} (1965) A1133-A1138,
\doi{10.1103/PhysRev.140.A1133}.

\bibitem{hohen64}
P. Hohenberg, and W. Kohn,
{\it Inhomogeneous Electron Gas},
Phys. Rev. {\bf 136} (1964) B864,
\doi{10.1103/PhysRev.136.B864}.

\bibitem{slater64}
J. C. Slater, Atomic Radii in Crystals,
J. Chem. Phys. {\bf 41} (1964) 3199-3204,
\doi{10.1063/1.1725697}.
{\bf Note:} For some elements there are no Bragg-Slater radii 
available. In that case radii are taken from,
E. Clementi, D. L. Raimondi, and W. P. Reinhardt,
Atomic Screening Constants from SCF Functions.
II. Atoms with 37 to 86 Electrons,
J. Chem. Phys. {\bf 47} (1967) 1300-1307,
\doi{10.1063/1.1712084}.

\bibitem{grimme04}
S.~Grimme,
{\it Accurate Description of van der Waals Complexes by Density Functional
     Theory Including Empirical Corrections},
J. Comp. Chem. {\bf 25} (2004) 1463-1473,
\doi{10.1002/jcc.20078}.

\bibitem{antony06}
J.~Antony, S.~Grimme,
{\it Density functional theory including dispersion corrections for
     intermolecular interactions in a large benchmark set of biologically
     relevant molecules},
Phys. Chem. Chem. Phys. {\bf 8} (2006) 5287-5293,
\doi{10.1039/b612585a}.

%MASSCF items
\bibitem{mcscf_background}
"The Construction and Interpretation of MCSCF wavefunctions", 
M.W.~Schmidt and M.S.~Gordon, Ann. Rev. Phys. Chem. {\bf 49} (1998) 233-266,
\doi{10.1146/annurev.physchem.49.1.233};
"The Multiconfiguration SCF Method", B.O.Roos, in 
"Methods in Computational Molecular Physics", 
edited by G.H.F.Diercksen and S.Wilson, 
D.Reidel Publishing, Dordrecht, Netherlands, 1983, pp 161-187;
"The Multiconfiguration SCF Method" B.O.Roos, in 
"Lecture Notes in Quantum Chemistry", edited by B.O.Roos, 
Lecture Notes in Chemistry v58, Springer-Verlag, Berlin, 1994, pp 177-254;
"Optimization and Characterization of a MCSCF State", J.Olsen, 
D.L.~Yeager, P.~Jorgensen,
  Adv. Chem. Phys. {\bf 54} (1983) 1-176, \doi{10.1002/9780470142783.ch1};
"Matrix Formulated Direct MCSCF and Multiconfiguration Reference CI Methods", 
H.-J.~Werner,
  Adv. Chem. Phys. {\bf 69} (1987) 1-62, \doi{10.1002/9780470142943.ch1};
"The MCSCF Method", R.~Shepard,
  Adv. Chem. Phys. {\bf 69} (1987) 63-200, \doi{10.1002/9780470142943.ch2};
"The CASSCF Method and its Application in Electronic Structure Calculations", 
B.O. Roos, in "Advances in Chemical Physics", vol.69, edited by K.P.Lawley, 
Wiley Interscience, New York, 1987, pp 339-445, \doi{10.1002/9780470142943.ch7}.

\bibitem{fullnr}
"General second order MCSCF theory: A Density Matrix Directed Algorithm", 
B.H.Lengsfield, III, J. Chem. Phys. 73 (1980) 382-390,
\doi{10.1063/1.439885}; 
"The use of the Augmented Matrix in MCSCF Theory", D.R.Yarkony, 
Chem. Phys. Lett. {\bf 77},634-635(1981),
\doi{10.1016/0009-2614(81)85223-2};
M. Dupuis, P. Mougenot, J.D. Watts, in "Modern Techniques in Theoretical 
Chemistry", E.Clementi, editor, ESCOM, Leiden, 1989, chapter 7.  

\bibitem{ormas}
"Direct configuration interaction and multiconfigurational self-consistent 
field method for multiple active spaces with variable occupations. 
I. Method", J. Ivanic, J. Chem. Phys. {\bf 119}, 9364-9376(2003),
\doi{10.1063/1.1615954}.

 \bibitem{ormas2}
"Direct configuration interaction and multiconfigurational self-consistent 
field method for multiple active spaces with variable occupations. 
II. Application to oxoMn(Salen) and N2O4", 
J. Ivanic, J. Chem. Phys. {\bf 119}, 9377-9385(2003),
\doi{10.1063/1.1615955}.
%End MASSCF items

\end{thebibliography}

\end{document}

Default settings

If the DFT module is switched on without specifying any options then the 
following functional and quadrature settings will apply;
\begin{itemize}
\item the LDA exchange functional
\item the Vosko, Wilk, and Nusair (VWN) correlation functional 
    \cite{vosko80} 
\item the Standard Grid No. 1 (SG1) \cite{sg1}
    implying
\begin{itemize}
\item Lebedev angular grids \cite{lebe99}
\item Euler-MacLaurin radial grids \cite{murray93}, with 50 points
\item Different angular grid sizes in different radial regimes, 
         employing 6, 38, 86, 194, 86 point angular grids from the
         nucleus outward
\item Becke weighting scheme \cite{becke88a}
\end{itemize}
\item no screening 
\item no pruning
\end{enumerate}
